%!TEX encoding = UTF-8 Unicode
\documentclass[a4paper]{compendium}
\usepackage[swedish]{babel}
\addto\captionsswedish{%
  \renewcommand{\appendixname}{Appendix}%
}
%TODO: Glossary
%http://tex.stackexchange.com/questions/5821/creating-a-standalone-glossary/5837#5837

\setlength{\columnsep}{16mm}

\title{
{\bf\sffamily\Huge\selectfont  Introduktion till programmering med Scala och Java} 
\\ \vspace{1em}%\hspace*{1.5cm}\inputgraphics[width=0.6\textwidth]{../img/gurka} \\
{\sffamily  Grundkurs}\\\vspace{2cm}
\includegraphics[height=4cm]{../img/scala-logo.png}
\includegraphics[height=4cm]{../img/java-logo.png}
}

%\author{Redaktör: Björn Regnell}
\date{EDAA45, Lp1-2, HT 2016 \\ 
Datavetenskap, LTH \\ 
Lunds Universitet  \\~\\ Kompileringsdatum: \today \\
\url{http://cs.lth.se/pgk}
}

\usepackage{pgffor}  %% http://stackoverflow.com/questions/2561791/iteration-in-latex
                     %  allows:  \foreach \n in {1,...,4}{ do something with \n }

\usepackage{framed}  %  allows:   \begin{framed}\end{framed}
%\newenvironment{Slide}[2][]
%  {\begin{framed}\setlist{noitemsep}\section*{#2}}
%  {\end{framed}}

\newcommand{\SlideHeading}[1]{\section*{#1}}

\usepackage[most]{tcolorbox}
\newenvironment{Slide}[2][]
  {\vspace{0.5em}\begin{tcolorbox}[%width=1.05\textwidth,
  grow to right by=0.03\textwidth,grow to left by=0.03\textwidth,%breakable, 
                                   enhanced]\setlist{noitemsep}\SlideHeading{#2}}
  {\end{tcolorbox}\vspace{0.5em}}

\newcommand{\Subsection}[1]{} %ignore slide sections
\newcommand{\SlideOnly}[1]{} %ignore slide font size

\newif\ifkompendium  % to allow conditional text in slides only showing up in compendium
\kompendiumtrue      % in slides: \kompendiumfalse
                

%!TEX encoding = UTF-8 Unicode
\newcommand{\ExeWeekONE}{expressions}
\newcommand{\LabWeekONE}{kojo}

\newcommand{\ExeWeekTWO}{programs}
\newcommand{\LabWeekTWO}{--}

\newcommand{\ExeWeekTHREE}{functions}
\newcommand{\LabWeekTHREE}{irritext}

\newcommand{\ExeWeekFOUR}{objects}
\newcommand{\LabWeekFOUR}{blockmole}

\newcommand{\ExeWeekFIVE}{classes}
\newcommand{\LabWeekFIVE}{turtlegraphics}

\newcommand{\ExeWeekSIX}{sequences}
\newcommand{\LabWeekSIX}{shuffle}

\newcommand{\ExeWeekSEVEN}{sets-maps}
\newcommand{\LabWeekSEVEN}{words}

\newcommand{\ExeWeekEIGHT}{matrices}
\newcommand{\LabWeekEIGHT}{maze}

\newcommand{\ExeWeekNINE}{inheritance}
\newcommand{\LabWeekNINE}{turtlerace-team}

\newcommand{\ExeWeekTEN}{patterns}
\newcommand{\LabWeekTEN}{chords-team}

\newcommand{\ExeWeekELEVEN}{scala-java}
\newcommand{\LabWeekELEVEN}{lthopoly-team}

\newcommand{\ExeWeekTWELVE}{sorting}
\newcommand{\LabWeekTWELVE}{survey}

\newcommand{\ExeWeekTHIRTEEN}{--}
\newcommand{\LabWeekTHIRTEEN}{Projekt}

\newcommand{\ExeWeekFOURTEEN}{threads}
\newcommand{\LabWeekFOURTEEN}{--}


\begin{document}
\maketitle
%!TEX root = ../compendium.tex

\clearpage\null\thispagestyle{empty}
\vfill

{
\setlength{\parindent}{0pt}
\emph{Editor}: Björn Regnell, Faculty of Engineering LTH, Lund University. \\ 

\emph{Contributors}: 
Björn Regnell,
Per Holm,
Sandra Nilsson,
Patrik Andersson,
Gustav Cedersjö,
Maj Stenmark,
Anna Axelsson,
Roy Andersson,
Markus Borg,
Anton Klarén.
\\

\emph{Repo}: \url{https://github.com/lunduniversity/introprog} \\ \newline

This manuscript is on-going work. Contributions are welcome! \\ 
\emph{Contact}: \url{bjorn.regnell@cs.lth.se}
\\ \newline

\emph{LICENCE}: CC BY-NC-SA 4.0 \\
\url{http://creativecommons.org/licenses/by-nc-sa/4.0/}
\\ \newline
Copyright \copyright~Computer Science, LTH \& Björn Regnell. 2016. Lund. Sweden.\\
}

%!TEX encoding = UTF-8 Unicode
%!TEX root = ../compendium.tex

\ChapterUnnum{Framstegsprotokoll}\label{progress-protocoll}


\section*{Genomförda övningar}

\vspace{1em}\noindent
{Till varje laboration hör en övning med uppgifter som utgör förberedelse inför labben. Du behöver minst behärska grunduppgifterna för att klara labben inom rimlig tid. Om du känner att du behöver öva mer på grunderna, gör då även extrauppgifterna. Om du vill fördjupa dig, gör fördjupningsuppgifterna som är på mer avancerad nivå. Kryssa för nedan vilka övningar du har gjort, så blir det lättare för din handledare att anpassa dialogen till de kunskaper du förvärvat hittills.}

\newcommand{\TickBox}{\raisebox{-.50ex}{\Large$\square$}}
\newcommand{\ExeRow}[1]{\hyperref[section:exe:#1]{\texttt{#1}} & \TickBox  &  \TickBox &  \TickBox  \\ \addlinespace }

\begin{table}[h]
%\centering
\vspace{2em}
\begin{tabular}{lccc}
\toprule \addlinespace
{\sffamily Övning} &
{\sffamily Grund} &
{\sffamily Extra} &
{\sffamily Fördjupning}\\ \addlinespace \midrule \\[-0.7em]
\ExeRow{expressions}
\ExeRow{programs}
\ExeRow{functions}
\ExeRow{data}
\ExeRow{vectors}
\ExeRow{classes}
\ExeRow{traits}
\ExeRow{matching}
\ExeRow{matrices}
\ExeRow{sorting}
\ExeRow{scalajava}
\ExeRow{threads}
\bottomrule
\end{tabular}
\end{table}

\newpage

\section*{Godkända obligatoriska moment}

\vspace{1em}\noindent
För att bli godkänd på laborationsuppgifterna och projektuppgiften måste du lösa deluppgifterna och diskutera dina lösningar med en handledare. Denna diskussion är din möjlighet att få feedback på dina lösningar. Ta vara på den!
Se till att handledaren noterar nedan när du blivit godkänd på respektive obligatorisk moment. Spara detta blad tills du fått slutbetyg i kursen.


\vspace{2.5em}\noindent Namn: \dotfill\\

\vspace{1em}\noindent Namnteckning: \dotfill\\

\newcommand{\LabRow}[1]{\\[-1.1em] \hyperref[section:lab:#1]{\texttt{#1}} & \dotfill &  \dotfill  \\ \addlinespace }

\begin{table}[h]
%\centering
\vspace{1em}
\begin{tabular}{lcc}
\toprule \addlinespace
{\sffamily\bfseries\small Lab} & {\sffamily\small Datum gk} &	
{\sffamily\small Handledares signatur + namnförtydligande}\\ \addlinespace 
%\midrule 
\\[-0.5em]
%!TEX encoding = UTF-8 Unicode
%!TEX root = ../compendium2.tex
\LabRow{kojo}
\LabRow{irritext}
\LabRow{blockmole}
\LabRow{blockbattle}
\LabRow{shuffle}
\LabRow{words}
\LabRow{life}
\LabRow{snake}
\LabRow{tabular}
\LabRow{javatext}
%\toprule
\addlinespace 
%\midrule 
\addlinespace\addlinespace
{\sffamily\small {\bfseries Projektuppgift} (välj en)	} & \dotfill &  \dotfill  \\
\addlinespace\addlinespace %\midrule
{\Large$\square$}\texttt{~~~\hyperref[section:proj:bank]{bank}} &
\multicolumn{2}{c}{\textit{Om egendef., ge kort beskrivning här:}}  \\ \addlinespace
{\Large$\square$}\texttt{~~~\hyperref[section:proj:tabular]{tabular}} \\ \addlinespace
{\Large$\square$}\texttt{~~~\hyperref[section:proj:music]{music}} \\ \addlinespace
{\Large$\square$}\texttt{~~~\hyperref[section:proj:photo]{photo}}  \\ \addlinespace
{\Large$\square$}\texttt{~~~}\textit{egendefinerad}  \\
%\dotfill  \\
\addlinespace\addlinespace
%\midrule
\addlinespace
{\sffamily\small {\bfseries Muntligt prov}} &  & \\
\addlinespace\addlinespace{}
{\Large$\square$}\texttt{~~~} godkänd & \dotfill &  \dotfill \\
\addlinespace\addlinespace\bottomrule
\end{tabular}
\end{table}

%!TEX root = ../compendium.tex


\ChapterUnnum{Förord} 

Programmering är inte bara ett sätt att ta makten över systemen som styr vårt samhälle. Det är också ett kraftfullt verktyg för tanken. Att lära sig programmering och systemutveckling är första steget på en livslång resa av kontinuerligt lärande. Programmeringsspråk och utvecklingsverktyg kommer och går, men de grundläggande koncepten sekvens, alternativ, repetition och abstraktion som ligger bakom all mjukvara består. 

Detta kompendium utgör kursmaterial för studier i grundläggande programmering, med syfte att ge en solid bas för ingenjörsstudenter och andra som utvecklar system som innehåller mjukvara. 

Kompendiet är framtaget av, med och för studenter och lärare på universitetsnivå, och distribueras som öppen källkod. Det får användas fritt så länge erkännande ges och eventuella ändringar också publiceras som öppen källkod under samma licens som ursprungsmaterialet. På kursens hemsida \href{http://cs.lth.se/pgk}{cs.lth.se/pgk} och repo \href{http://github.com/lunduniversity/introprog}{github.com/lunduniversity/introprog} finns instruktioner om hur du kan bidra till kursmaterialet.

Läromaterialet fokuserar på lärande genom eget arbete och innehåller övningar och laborationer som är organiserade i moduler. Varje modul har ett tema och tillhörande föreläsningsanteckningar.

I kursen används språken Scala och Java för att illustrera grunderna i imperativ och objektorienterad programmering, tillsammans med elementär funktionsprogrammering. Mer avancerad objektorientering och funktionsprogrammering och  lämnas till fortsättningskurser. 



Den kanske viktigaste framgångsfaktorn vid studier i programmering är att bejaka din egen upptäckarglädje och experimentlusta. Det fantastiska med programmering är att dina egna intellektuella konstruktioner faktiskt \emph{gör} något som just \emph{du} har bestämt! Ta vara på det och prova dig fram genom att koda egna idéer -- det är kul när det funkar men minst lika lärorikt är felsökning, buggrättande och alla misslyckade försök som efter hårt arbete vänds till lyckade lösningar och bestående lärdomar. 

Välkommen i programmeringens fascinerande värld och hjärtligt lycka till med dina studier!



\mainmatter
\tableofcontents

\part{Om kursen}      
%!TEX root = ../compendium.tex

\ChapterUnnum{Kursens arkitektur}

%!TEX encoding = UTF-8 Unicode
%!TEX root = ../lect-week01.tex

%%%%%%%%%%%%%%%%%%%%%%%%%%%%%%%%%%%%%%
\Subsection{Om kursen}

%%%

\ifkompendium\else
\begin{Slide}{Nytt för i år}
\begin{itemize}
\item \Emph{Scala} införs som förstaspråk på Datateknikprogrammet.
\item Den \Emph{största förnyelsen} av den inledande programmeringskursen sedan vi införde Java 1997.
\item Allt kursmaterial är \Emph{öppen källkod}.
\item \Emph{Studentermedverkan} i kursutvecklingen.
\end{itemize}
\vspace{1em}\hskip1em\href{https://www.lth.se/nyheter-och-press/nyheter/visa-nyhet/article/scala-blir-foerstaspraak-paa-datateknikprogrammet/}{www.lth.se/nyheter-och-press/nyheter/visa-nyhet/article/\\\hskip1emscala-blir-foerstaspraak-paa-datateknikprogrammet/}
\end{Slide}
\fi

\begin{Slide}{Veckoöversikt}
\noindent\resizebox{0.9\columnwidth}{!}{
%!TEX encoding = UTF-8 Unicode
\begin{tabular}{l|l|l|l}
\textit{W} & \textit{Modul} & \textit{Övn} & \textit{Lab} \\ \hline \hline
W01 & Introduktion & expressions & kojo \\
W02 & Kodstrukturer & programs & -- \\
W03 & Funktioner, objekt & functions & blockmole \\
W04 & Datastrukturer & data & pirates \\
W05 & Sekvensalgoritmer & sequences & shuffle \\
W06 & Klasser & classes & turtlegraphics \\
W07 & Arv & traits & turtlerace-team \\
KS & KONTROLLSKRIVN. & -- & -- \\
W08 & Repetition, trösklar, luckor & reboot-init & reboot-check \\
W09 & Mönster, undantag & matching & chords-team \\
W10 & Matriser, typparametrar & matrices & maze \\
W11 & Sökning, sortering & sorting & survey \\
W12 & Scala och Java & scalajava & lthopoly-team \\
W13 & Extra: design, api, trådar, webb & threads & Projekt \\
W14 & Tentaträning & Extenta & -- \\
T & TENTAMEN & -- & -- \\
\end{tabular}

}
\end{Slide}

\ifkompendium
\noindent Kursen består av en \textbf{modul} per läsvecka med två \textbf{föreläsningar}, en \textbf{övning} och en \textbf{laboration} (undantaget W02, W13 \& W14 som saknar labb och/eller övning). 
Föreläsningarna ger en översikt av den teori som ingår i varje modul. Genom att göra övningarna bearbetar du teorin och förebereder dig inför laborationerna. När du klarat övningen och laborationen i en modul är du redo att gå vidare till nästa. Tabellen på nästa uppslag visar begrepp som ingår i varje modul. 

Kursen är uppdelad i två läsperioder. Efter första läsperioden gör du en diagnostisk \textbf{kontrollskrivning} som kontrollerar ditt kunskapsläge. Andra läsperioden avslutas med ett större \textbf{projekt} och en skriftlig \textbf{tentamen}.



\clearpage
\hyphenation{intro-duktion sekvens-algoritmer kod-strukturer data-strukturer}
{\fontsize{11}{13}\selectfont\renewcommand{\arraystretch}{1.75}
\begin{longtable}{@{}p{.05\textwidth} | >{\hspace{0.1em}\raggedright\bfseries\sffamily}p{.15\textwidth}  >{\raggedleft\arraybackslash\hspace{0.0em}\fontsize{10.5}{12}\selectfont}p{0.735\textwidth}}
W01 & Introduktion & sekvens, alternativ, repetition, abstraktion, programmeringsspråk, programmeringsparadigmer, editera-kompilera-exekvera, datorns delar, virtuell maskin, REPL, literal, värde, uttryck, identifierare, variabel, typ, tilldelning, namn, val, var, def, inbyggda grundtyper, Int, Long, Short, Double, Float, Byte, Char, String, println, typen Unit, enhetsvärdet (), stränginterpolatorn s, if, else, true, false, MinValue, MaxValue, aritmetik, slumptal, math.random, logiska uttryck, de Morgans lagar, while-sats, for-sats \\
W02 & Kodstrukturer & iterering, for-uttryck, map, foreach, Range, Array, Vector, algoritm vs implementation, pseudokod, algoritm: SWAP, algoritm: SUM, algoritm: MIN/MAX, algoritm: MININDEX, block, namnsynlighet, namnöverskuggning, lokala variabler, paket, import, filstruktur, jar, dokumentation, programlayout, JDK, main i Java vs Scala, java.lang.System.out.println \\
W03 & Funktioner & definera funktion, anropa funktion, parameter, returtyp, värdeandrop, namnanrop, default-argument, namngivna argument, applicera funktion på alla element i en samling, procedur, värdeanrop vs namnanrop, uppdelad parameterlista, skapa egen kontrollstruktur, funktionsvärde, funktionstyp, äkta funktion, stegad funktion, apply, lazy val, lokala funktioner, anonyma funktioner, lambda, aktiveringspost, anropsstacken, objektheapen, rekursion, cslib.window.SimpleWindow \\
W04 & Objekt & objekt, modul, paket, punktnotation, tillstånd, metod, medlem, funktioner är objekt, cslib.window.SimpleWindow \\
W05 & Klasser & objektorientering, klass, Point, Square, Complex, new, null, this, inkapsling, accessregler, private, private[this], kompanjonsobjekt, getters och setters, klassparameter, primär konstruktor, objektfabriksmetod, överlagring av metoder, referenslikhet vs strukturlikhet, eq vs == \\
W06 & Sekvensalgoritmer & sekvensalgoritm, algoritm: SEQ-COPY, in-place vs copy, algoritm: SEQ-REVERSE, algoritm: SEQ-REGISTER, sekvenser i Java vs Scala, for-sats i Java, java.util.Scanner, scala.collection.mutable.ArrayBuffer, StringBuilder, java.util.Random, slumptalsfrö \\
W07 & Datastrukturer & attribut (fält), medlem, metod, tupel, klass, Any, isInstanceOf, toString, case-klass, samling, scala.collection, föränderlighet vs oföränderlighet, List, Vector, Set, Map, typparameter, generisk samling som parameter, översikt samlingsmetoder, översikt strängmetoder, läsa/skriva textfiler, Source.fromFile, java.nio.file \\
KS & \multicolumn{2}{l}{KONTROLLSKRIVN.} \\
W08 & Matriser, typparametrar & matris, nästlad samling, nästlad for-sats, typparameter, generisk funktion, generisk klass, fri vs bunden typparameter, matriser i Java vs Scala, allokering av nästlade arrayer i Scala och Java \\
W09 & Arv & arv, polymorfism, trait, extends, asInstanceOf, with, inmixning, supertyp, subtyp, bastyp, override, klasshierarkin i Scala: Any AnyRef Object AnyVal Null Nothing, referenstyper vs värdetyper, klasshierarkin i scala.collection, Shape som bastyp till Rectangle och Circle, accessregler vid arv, protected, final, klass vs trait, abstract class, case-object, typer med uppräknade värden, gränssnitt, trait vs interface, programmeringsgränssnitt (api) \\
W10 & Mönster, undantag, likhet & mönstermatchning, match, Option, throw, try, catch, Try, unapply, sealed, flatten, flatMap, partiella funktioner, collect, speciella matchningar: wildcard pattern; variable binding; sequence wildcard; back-ticks, equals, hashcode, exempel: equals för klassen Complex, switch-sats i Java \\
W11 & Scala och Java & syntaxskillnader mellan Scala och Java, klasser i Scala vs Java, referensvariabler vs enkla värden i Java, referenstilldelning vs värdetilldelning i Java, alternativ konstruktor i Scala och Java, for-sats i Java, for-each-sats i Java, java.util.ArrayList, autoboxing i Java, primitiva typer i Java, wrapperklasser i Java, samlingar i Java vs Scala, scala.collection.JavaConverters, namnkonventioner för konstanter \\
W12 & Sökning, sortering, ordning & strängjämförelse, compareTo, implicit ordning, linjärsökning, binärsökning, algoritm: LINEAR-SEARCH, algoritm: BINARY-SEARCH, algoritmisk komplexitet, sortering till ny vektor, sortering på plats, insättningssortering, urvalssortering, algoritm: INSERTION-SORT, algoritm: SELECTION-SORT, Ordering[T], Ordered[T], Comparator[T], Comparable[T] \\
W13 & \multicolumn{2}{l}{Repetition, tentaträning, projekt} \\
W14 & Extra: jämlöpande exekvering & tråd, jämlöpande exekvering, icke-blockerande anrop, callback, java.lang.Thread, java.util.concurrent.atomic.AtomicInteger, scala.concurrent.Future, kort om html+css+javascript+scala.js och webbprogrammering \\
T & \multicolumn{2}{l}{TENTAMEN} \\
\end{longtable}
}
\fi

\begin{Slide}{Vad lär du dig?}
\begin{itemize}
\item Grundläggande principer för programmering:\\ Sekvens, Alternativ, Repetition, Abstraktion (SARA)\\$\implies$Inga förkunskaper i programmering krävs!
\item Konstruktion av algoritmer
\item Tänka i abstraktioner
\item Förståelse för flera olika angreppssätt: 
\begin{itemize}
\item \Emph{imperativ programmering}%: satser, föränderlighet
\item \Emph{objektorientering}%: inkapsling, återanvändning
\item \Emph{funktionsprogrammering}%: uttryck, oföränderlighet
\end{itemize}
\item Programspråken \Emph{Scala} och \Emph{Java}
\item Utvecklingsverktyg (editor, kompilator, utvecklingsmiljö)
\item Implementera, testa, felsöka
\end{itemize}
\end{Slide}

\begin{Slide}{Hur lär du dig?}
\begin{itemize}
\item Genom praktiskt \Alert{eget arbete}: \Emph{Lära genom att göra!}
\begin{itemize}
\item Övningar: applicera koncept på olika sätt
\item Laborationer: kombinera flera koncept till en helhet
\end{itemize}
\item Genom studier av kursens teori: \Emph{Skapa förståelse!}
\item Genom samarbete med dina kurskamrater: \Emph{Gå djupare!}
\end{itemize}
\end{Slide}


\begin{Slide}{Kurslitteratur}
\begin{minipage}{0.45\textwidth}\SlideFontSmall
\hskip1.33em\includegraphics[width=0.65\textwidth]{../img/compendium-front-page.png}
\begin{itemize}
\item \Emph{Kompendium} med teori, övningar \& laborationer
\item Trycks \& säljs av institutionen %på KFS \\ \url{http://www.kfsab.se/}
 efter beställning
\end{itemize}
\end{minipage}
\hskip1em\begin{minipage}{0.5\textwidth}\SlideFontSize{8}{10}
Bra, men ej nödvändig, \Emph{bredvidläsning}:\\ 
-- för \Emph{nybörjare}:
\vskip0.2mm
\includegraphics[width=0.33\textwidth]{../img/lewisbook.jpg}\hskip4mm
\includegraphics[width=0.33\textwidth]{../img/ankbok.jpg}

\noindent -- för de som \Emph{redan kodat} en del:
\vskip0.7mm
\includegraphics[width=0.45\textwidth]{../img/pinsbook.jpg}\hskip4mm
\includegraphics[width=0.47\textwidth]{../img/koffmanbook.jpg}
\end{minipage}
\end{Slide}

\ifkompendium
\noindent Kompendiet är den huvudsakliga kurslitteraturen och definierar kursinnehållet. Föreläsningar, övningar och laborationer i kompendiet är kursens primära kunskapskällor, tillsammans med de öppna resurser på nätet som kompendiet hänvisar till. Kompendiet är öppen källkod och du välkomnas varmt att bidra!

Om du gärna vill ha en eller flera mer traditionella läroböcker som bredvidläsning rekommenderas följande:
\begin{itemize}[noitemsep, leftmargin=*]
\item För de som aldrig kodat, och vill läsa om kodning från grunden:
\begin{itemize}[nolistsep]
\item ''Introduction to Programming and Problem-Solving Using Scala, Second Edition'', Mark C. Lewis, Lisa Lacher.  {\href{https://www.crcpress.com/Introduction-to-Programming-and-Problem-Solving-Using-Scala-Second-Edition/Lewis-Lacher/p/book/9781498730952}{www.crcpress.com/Introduction-to-Programming-and-Problem-Solving-Using-Scala-Second-Edition/Lewis-Lacher/p/book/9781498730952}}
\item ''Objektorienterad programmering och Java'', Per Holm, Tredje upplagan (2007). \href{https://www.studentlitteratur.se/#6735}{www.studentlitteratur.se/\#6735}
\end{itemize}
\item För de som redan kodat en hel del i ett objektorienterat språk:
\begin{itemize}[nolistsep, noitemsep]
\item ''Programming in Scala, Third Edition -- A comprehensive step-by-step guide'', Martin Odersky, Lex Spoon, and Bill Venners. \\ \href{http://www.artima.com/shop/programming_in_scala_3ed}{www.artima.com/shop/programming\_in\_scala\_3ed} 
\item ''Data Structures: Abstraction and Design Using Java, 3rd Edition'', Elliot B. Koffman, Paul A. T. Wolfgang. \\
\href{http://eu.wiley.com/WileyCDA/WileyTitle/productCd-1119186528.html}{http://eu.wiley.com/WileyCDA/WileyTitle/productCd-1119186528.html}
\end{itemize}
\end{itemize}
Dessa läroböcker följer inte direkt kursens upplägg vad gäller omfång och progression och du får själv göra den nyttiga hemläxan att koppla  deras innehåll till det vi går igenom i kursens olika moduler.

\else
\begin{Slide}{Beställning av kompendium och snabbreferens}
\begin{itemize}
\item \Emph{Kompendiet} finns i pdf för fri nedladdning, men det \Alert{rekommenderas starkt} att du köper den på papper.
\item Det är mycket lättare att ha övningar och labbar på papper bredvid skärmen, när du ska tänka, koda och plugga!
\item \Emph{Snabbreferensen} finns också i pdf men du behöver ha en tryckt version eftersom det är enda tillåtna hjälpmedlet på skriftliga kontrollskrivningen och tentamen.
\item Kompendium och bok trycks här i E-huset och säljs av institutionen till självkostnadspris.
\item Pris för kompendium + snabbreferens: ??? kr
\item Skriv upp dig på listan -- tryckning sker efter beställning.
\item Du betalar med jämna pengar på cs expedition, våning 2
\end{itemize}
\end{Slide}
\fi

\ifkompendium\else
\begin{Slide}{Personal}\SlideFontSmall
\begin{description}
\item [\bfseries Kursansvarig:] ~\\Björn Regnell, bjorn.regnell@cs.lth.se
\item [\bfseries Kurssekreterare:]  ~\\Lena Ohlsson \\Exp.tid 09.30 -- 11.30 samt 12.45 -- 13.30
\item [\bfseries Handledare:] ~\\
\Emph{Doktorander}: \\ 
Tekn. Lic. Maj Stenmark, Gustav Cedersjö\\
\Emph{Teknologer}: \\
Anders Buhl, 
Anna Palmqvist Sjövall, 
Anton Andersson,
Cecilia Lindskog, 
Emil Wihlander, 
Erik Bjäreholt, 
Erik Grampp, 
Filip Stjernström, 
Fredrik Danebjer, 
Henrik Olsson, 
Jakob Hök, 
Jonas Danebjer, 
Måns Magnusson, 
Oscar Sigurdsson, 
Oskar Berg, 
Oskar Widmark, 
Sebastian Hegardt, 
Stefan Jonsson, 
Tom Postema, 
Valthor Halldorsson
\end{description}
\end{Slide}
\fi

\begin{Slide}{Föreläsningsanteckningar}
\begin{itemize}
\item Föreläsningsanteckningar utvecklas under kursens gång
\item Några av bilderna finns i kompendiet
\item Alla bilder läggs ut här: \\
\href{https://github.com/lunduniversity/introprog/tree/master/slides}{github.com/lunduniversity/introprog/tree/master/slides} \\
och uppdateras kontinuerligt allt eftersom de utvecklas
\item Förslag på förbättringar välkomna!
\end{itemize}
\end{Slide}

\begin{Slide}{Kursmoment --- varför?}\SlideOnly{\footnotesize}
\begin{itemize}
\item \Emph{Föreläsningar}: skapa översikt, ge struktur, förklara teori, svara på frågor, motivera varför
\item \Emph{Övningar}: \Alert{förbereda} laborationerna, bearbeta teorins olika delar med avgränsade deluppgifter, \Emph{grundövningar} för alla, \Emph{extraövningar} om du vill/behöver öva mer, \Emph{fördjupningsövningar} om du vill gå djupare 
\item \Emph{Laborationer}: \Alert{obligatoriska}, sätta samman teorins delar i ett större program; lösningar redovisas för handledare; gk på alla för att få tenta, 
\item \Emph{Resurstider}: få hjälp med övningar och laborationsförberedelser av handledare, fråga vad du vill
\item \Emph{Samarbetsgrupper}: grupplärande genom samarbete, hjälpa varandra 
\item \Emph{Kontrollskrivning}: \Alert{obligatorisk}, diagnostisk, kamraträttad; kan ge samarbetsbonuspoäng till tentan
\item \Emph{Individuell projektuppgift}: \Alert{obligatorisk}, du visar att du kan skapa ett större program självständigt; redovisas för handledare
\item \Emph{Tentamen}: \Alert{obligatorisk}, skriftlig, enda hjälpmedel:   \href{https://github.com/lunduniversity/introprog/blob/master/quickref/quickref.pdf}{snabbreferensen}
\end{itemize}
\end{Slide}

\ifkompendium\else
\begin{Slide}{Detta är bara början... }
Exempel på efterföljande kurser som bygger vidare på denna:
\begin{itemize}
\item Årskurs 1
\begin{itemize}
\item Programmeringsteknik -- fördjupningskurs
\item Utvärdering av programvarusystem
\item Diskreta strukturer
\end{itemize}
\item Årskurs 2
\begin{itemize}
\item Objektorienterad modellering och design
\item Programvaruutveckling i grupp
\item Algoritmer, datastrukturer och komplexitet
\item Funktionsprogrammering
\end{itemize}
\end{itemize}
\end{Slide}


\begin{Slide}{Registrering}
\begin{itemize}
\item Fyll i listan som skickas runt.
\item Kryssa i kolumnen \Emph{Ska gå} om du ska gå kursen\footnote{\scriptsize D1:a som redan gått motsvarande högskolekurs? Uppsök studievägledningen}\footnote{\scriptsize D2:a eller äldre som vill bli omregistrerad? Prata med kursansvarig på rasten}
\item Kryssa i kolumnen \Emph{Kursombud} om du kan tänka dig att vara kursombud under kursens gång
\begin{itemize}
\item Alla LTH-kurser ska utvärderas under kursens gång och efter kursens slut.
\item Till det behövs kursombud -- ungefär 2 D-are och 2 W-are.
\item Ni kommer att bli kontaktade av studierådet. \\SRD ordf: Amelia Andersson
\end{itemize}
\end{itemize}
\end{Slide}

%%%
\begin{Slide}{Förkunskaper}
\begin{itemize}
\item Förkunskaper $\neq$ Förmåga
\item Varken kompetens eller personliga egenskaper är statiska 
\item ''Programmeringskompetens'' är inte \textit{en} enda enkel förmåga utan en komplex sammansättning av flera olika förmågor som utvecklas genom hela livet
\item Ett innovativt utvecklar\Alert{team} behöver många olika kompetenser för att vara framgångsrikt
\end{itemize}
\end{Slide}

%%%
\begin{Slide}{Förkunskapsenkät}
\begin{itemize}
\item Om du inte redan gjort det: fyll i denna enkät \Alert{snarast}:\\
\url{http://cs.lth.se/pgk/presurvey} \\
\item Dina svar behandlas internt och all statistik anonymiseras.
\item Enkäten ligger till grund för randomiserad gruppindelning i samarbetsgrupper, så att det blir en spridning av förkunskaper inom gruppen.
\item Gruppindelnig publiceras här: \\ \url{http://cs.lth.se/pgk/grupper/}
\end{itemize}
\end{Slide}

\begin{Slide}{Samarbetgrupper}\footnotesize
\begin{itemize}
\item Ni delas in i \Emph{samarbetsgrupper} om ca 5 personer baserat på förkunskapsenkäten, så att olika förkunskapsnivåer sammanförs
\item Några av laborationerna är mer omfattande \Emph{grupplabbar} och kommer att göras i samarbetsgrupperna \\ \vspace{1em}
\item Kontrollskrivningen i halvtid kan ge \Emph{samarbetsbonus} (max 5p) som adderas till ordinarie tentans poäng (max 100p) med medelvärdet av gruppmedlemmarnas individuella kontrollskrivningspoäng 
\scriptsize \parbox{7cm}{Bonus $b$ för varje person i en grupp med $n$ medlemmar med $p_i$ poäng vardera på kontrollskrivningen:} 
 \hspace{5mm} $\displaystyle b = \sum\limits_{i=1}^n \frac{p_i}{n}$
\end{itemize}
\end{Slide}

\fi

%%%
\begin{Slide}{Varför studera i samarbetsgrupper?}

Huvudsyfte: \Emph{Bra lärande!}

\begin{itemize}
\item Pedagogisk forskning stödjer tesen att lärandet blir mer djupinriktat om det sker i utbyte med andra
\item Ett studiesammanhang med höga ambitioner och respektfull gemenskap gör att vi \Emph{når mycket längre}
\item Varför ska du som redan kan mycket aktivt dela med dig av dina kunskaper?
\begin{itemize}
\item Förstå bättre själv genom att förklara för andra
\item Träna din pedagogiska förmåga
\item Förbered dig för ditt kommande yrkesliv som mjukvaruutvecklare 
\end{itemize}
\end{itemize}
\end{Slide}

%%%

\ifkompendium\else
\begin{Slide}{Samarbetskontrakt}
Gör ett skriftligt \href{https://github.com/bjornregnell/lth-eda016-2015/blob/master/assignments/collaboration-contract.tex}{\bf samarbetskontrakt} med dessa och ev. andra punkter som ni också tycker bör ingå:
\begin{enumerate}
\item Återkommande mötestider per vecka
\item Kom i tid till gruppmöten
\item Var väl förberedd genom självstudier inför gruppmöten
\item Hjälp varandra att förstå, men ta inte över och lös allt
\item Ha ett respektfullt bemötande även om ni har olika åsikter
\item Inkludera alla i gemenskapen
\end{enumerate}

Diskutera hur ni ska uppfylla dessa innan alla skriver på. \\ Ta med samarbetskontraktet och visa för handledare på labb 1.

\vskip1em

\Alert{Om arbetet i samarbetsgruppen inte fungerar ska ni mejla kursansvarig och boka mötestid!}
\end{Slide}

\begin{Slide}{Bestraffa inte frågor!}
\begin{itemize}
\item Det finns bättre och sämre frågor vad gäller hur mycket man kan lära sig av svaret, men \Emph{all undran är en chans} att i dialog utbyta erfarenheter och lärande
\item Den som frågar \Emph{vill veta} och berättar genom frågan något om nuvarande kunskapsläge
\item Den som svarar får chansen att \Emph{reflektera} över vad som kan vara svårt och olika vägar till djupare förståelse
\item I en hälsosam lärandemiljö är det \Emph{helt tryggt} att visa att man ännu inte förstår, att man gjort ''fel'', att man har mer att lära, etc. 
\item Det är viktigt att våga försöka även om det blir ''fel'':\\ \Emph{det är ju då man lär sig!}
\end{itemize}
\end{Slide}

%%%
\begin{Slide}{Plagiatregler}
Läs dessa regler noga och diskutera i samarbetsgrupperna:
\begin{itemize}
\footnotesize
\item \url{http://cs.lth.se/utbildning/samarbete-eller-fusk/}
\item \url{http://cs.lth.se/utbildning/foereskrifter-angaaende-obligatoriska-moment/}
\end{itemize}
Ni ska lära er genom \Emph{eget arbete} och genom  \Emph{bra samarbete}. Samarbete gör att man lär sig bättre, men man lär sig inte av att bara kopiera andras lösningar. Plagiering är förbjuden och kan medföra disciplinärende och avstängning.
\end{Slide}

\fi %%%%%%%%%%%%%%%%%%%%%%%%%%%%%%%%

%%%
\begin{Slide}{En typisk kursvecka}
\begin{enumerate}
\item Gå på \Emph{föreläsningar} på \Alert{måndag--tisdag}
\item Jobba med \Emph{individuellt} med teori, övningar, labbförberedelser på  \Alert{måndag--torsdag}
\item Kom till \Emph{resurstiderna} och få hjälp och tips av handledare och kurskamrater på \Alert{onsdag--torsdag}
\item Genomför den obligatoriska \Emph{laborationen} på \Alert{fredag}
\item Träffas i \Emph{samarbetsgruppen} och hjälp varandra att förstå mer och fördjupa lärandet, förslagsvis på återkommande tider varje vecka då alla i gruppen kan
\end{enumerate}
Se detaljerna och undantagen i schemat: \href{http://cs.lth.se/pgk/schema}{cs.lth.se/pgk/schema}
\end{Slide}

\ifkompendium\else  %%%%%%%%%%%%%%%%%%%%%%%%%
%%%
\begin{Slide}{Laborationer}\footnotesize
\begin{itemize}
\item \Alert{Programmering lär man sig bäst genom att programmera...}
\item Labbarna är \Emph{individuella} (utom 2) och \Emph{obligatoriska}
\item Gör övningarna och labbförberedelserna noga \textit{innan} själva labben -- detta är ofta helt nödvändigt för att du ska hinna klart. Dina labbförberedelserna kontrolleras av handledare under labben.
\item Är du sjuk? Anmäl det \Alert{före} labben till \url{bjorn.regnell@cs.lth.se}, \\ få hjälp på resurstid och redovisa på resurstid (eller labbtid, när handledaren har tid över)
\item Hinner du inte med hela? Se till att handledaren noterar din närvaro, och fortsätt på resurstid och ev. uppsamlingstider.
\item Läs noga anvisningarna i kompendiet
\item Laborationstiderna är gruppindelade enligt \href{http://cs.lth.se/eda016/schema/}{schemat}. Du ska gå till den tid och den sal som motsvarar din \href{http://cs.lth.se/eda016/grupper/}{grupp}.
\end{itemize}
\end{Slide}

%%%
\begin{Slide}{Resurstider}
\begin{itemize}
\item På resurstiderna får du hjälp med övningar och laborationsförberedelser
\item Kom till minst en resurstid per vecka (se \href{http://cs.lth.se/eda016/schema/}{schema})
\item Handledare gör ibland \Emph{genomgångar} för alla under resurstiderna. Tipsa om handledare om vad du finner svårt.
\item Resurstiderna är inte gruppindelade i schemat. Du får i mån av plats gå på flera resurstider per vecka. Om det blir fullt i ett rum prioriteras dessa grupper för att minimera schemakrockar: 
\end{itemize}
\begin{table}[]
\centering\scriptsize
\begin{tabular}{lllll}
Tid Lp1 & Sal & Grupper med prio \\
\hline
Ons 10-12 v1-7 & Hacke  &   09 \& 10 \\
Ons 13-15 v1-7 & Hacke  &   07 \& 08  \\
Ons 15-17 v1-7 & Panter  & 05 \& 06   \\
Ons 15-17 v1-7 & Val       &  03 \& 04   \\
Tor 13-15 v1-7 & Mars     & 01 \& 02  \\
Tor 15-17 v1-7 & Mars     & 11 \& 12 \\ 
\end{tabular}
\end{table}
\end{Slide}

\fi

%!TEX root = ../compendium.tex

\ChapterUnnum{Anvisningar}

\SectionUnnum{Samarbetsgrupper}
\subsection*{Samarbetskontrakt}
\SectionUnnum{Föreläsningar}
\SectionUnnum{Övningar}
\SectionUnnum{Laborationer}
\SectionUnnum{Resurstider}
\SectionUnnum{Kontrollskrivning}
\SectionUnnum{Tentamen}


\renewcommand{\SlideHeading}[1]{\section{#1}}

\part{Moduler}         
\foreach \n in {1,...,9}{%
  \input{modules/w0\n-chapter.tex} 
  \input{modules/w0\n-exercise.tex}
  \input{modules/w0\n-lab.tex}
}
\foreach \n in {10,...,12}{%
  \input{modules/w\n-chapter.tex} 
  \input{modules/w\n-exercise.tex}
  \input{modules/w\n-lab.tex}
}

%!TEX root = ../compendium.tex

%!TEX encoding = UTF-8 Unicode
\chapter{Design}\label{chapter:W13}
Koncept du ska lära dig denna vecka:
\begin{multicols}{2}\begin{itemize}[nosep,label={$\square$},leftmargin=*]
\item\end{itemize}\end{multicols}

    
%!TEX encoding = UTF-8 Unicode
%!TEX root = ../compendium.tex

\Assignment{bank}

\subsection{Obligatoriska uppgifter}

\Task En uppgift.

\Subtask En underuppgift.

\Subtask En underuppgift.

\subsection{Frivilliga extrauppgifter}

\Task En uppgift.

\Subtask En underuppgift.

\Subtask En underuppgift.


%!TEX encoding = UTF-8 Unicode
%!TEX root = ../compendium.tex

\Assignment{tictactoe}
I detta projekt ska du implementera din egen version av spelet tic-tac-toe (eller som vi på svenska kallar det, tre i rad)! Du kommer börja med att implementera en version där du kan spela mot en kursare och sen gå vidare till att implementera en datorspelare som lägger sin pjäs slumpmässigt och till slut en som inte kan förlora!

\subsection{Regler}
%Om du känner dig säker på hur reglerna i tic-tac-toe funkar kan du skippa detta. 
\begin{itemize}
	\item Spelplanen består av ett rutnät av storlek 3x3.
	\item Det finns två spelare: \texttt{x} och \texttt{o}.
	\item Spelarna placerar ut en pjäs var i växlande ordning där \texttt{x} börjar.
	\item Spelet tar slut om en spelare har fått antingen en rad, diagonal eller kolumn ifylld av sin spelpjäs eller om spelplanen är fylld.
\end{itemize}
\textit{Notera att pjäserna INTE får flyttas när de väl ligger på spelplanen.}

\subsection{Teori}
Representationen är vald till en endimensionell vektor av typen Int av storlek 9 där elementen 0 till och med 2  representerar den första raden, [3, 5] andra raden och [6, 8] den tredje. Anledningen till detta är att vi vill ha en representation så att spelaren kan svara vilket drag den vill göra med ett heltal.
Varje element i vektorn ska kunna representera en tom plats, en plats allokerad av \texttt{x} och en plats allokerad av \texttt{o}. Detta innebär att en vektor av typen Boolean inte räcker till. Istället väjs den (kanske lite minnesöverflödiga) typen Int. Hint: en bra representation är 0 för tom plats, 1 för \texttt{x} och -1 för \texttt{o}. 
 
\subsection{Obligatoriska uppgifter}

\Task Implementera ett fungerande spel genom att utöka kodskeletten i klasserna Player, HumanPlayer och Game.

\Subtask Implementera funktionen gameWon i klassen Player som testar huruvida spelaren \code{who} vunnit spelet.

\Subtask Implementera en HumanPlayer.

\Subtask Implementera första version av Game.

\Task Randomized player

\Subtask Implementera en spelare som väljer ett slumpmässigt giltigt drag.

\Subtask Ändra Game så att användaren tillåts stänga av ritfunktionen och tillåts spela många spel.

\Subtask Vad är sannolikheterna för att \texttt{x} vinner, \texttt{o} vinner och att det blir oavgjort om två randomized players spelar mot varandra?

Hamnar man i närheten av dessa resultat tror vi på er randomized player.
\begin{itemize}
	\item P(\texttt{x} vinner) = 0.586
	\item P(\texttt{o} vinner) = 0.288
	\item P(lika) = 0.126
\end{itemize}


\Subtask Varför är det större sannolikhet för \texttt{x} att vinna än \texttt{o}?

\Task Optimal Player

\Subtask Läs igenom eval-funktionen och Appendix om max-min-evaluering.

\Subtask Implementera Optimal Players move-funktion.

\Subtask testa att spela mot din Optimal player med en human player, kan du spela lika? Kan du vinna?

\Subtask Vad händer om du sätter en random player mot Optimal player? Blir det någonsin oavgjort, hur ofta?

\subsection{Frivilliga extrauppgifter}

\Task Hashning.

\Subtask En underuppgift.

\Subtask En underuppgift.
%!TEX encoding = UTF-8 Unicode
%!TEX root = ../compendium.tex

\Assignment{imageprocessing}

\subsection{Obligatoriska uppgifter}

\Task En uppgift.

\Subtask En underuppgift.

\Subtask En underuppgift.

\subsection{Frivilliga extrauppgifter}

\Task En uppgift.

\Subtask En underuppgift.

\Subtask En underuppgift.
%!TEX encoding = UTF-8 Unicode

%!TEX root = ../compendium.tex

%!TEX encoding = UTF-8 Unicode
\chapter{Muntlig examen}\label{chapter:W14}


%!TEX encoding = UTF-8 Unicode
%!TEX root = ../lect-week14.tex

%%%

\Subsection{Tentatips}
\begin{Slide}{Före tentan:}\SlideFontSmall
\begin{enumerate}
\item Repetera övningar och labbar i kompendiet. 
\item Läs igenom föreläsningsanteckningar.
\item Studera \Emph{snabbref} \Alert{mycket noga} så att du vet vad som är givet och var det står, så att du kan hitta det du behöver snabbt.
\item Skapa och \Emph{memorera} en personlig \Emph{checklista} med programmeringsfel du brukar göra, som även inkluderar småfel, så som glömda parenteser och semikolon, och annat som en kompilator/IDE normalt hittar.
\item Tänk igenom hur du ska disponera dina 5 timmar på tentan.
\item Gör den minst en extenta som om det vore \Alert{skarpt läge}: 
\begin{enumerate}\SlideFontTiny
\item Avsätt 5 ostörda timmar (stäng av telefon, dator etc).
\item Inga hjälpmedel. Bara snabbref.
\item Förbered dryck och tilltugg.
\end{enumerate}
\end{enumerate}
\end{Slide}

\begin{Slide}{På tentan:} \SlideFontTiny
\begin{enumerate}
\item Läs igenom \Alert{hela} tentan först. \\ \Emph{Varför?} Förstå helheten. Delarna hänger ihop.
\item Notera och begrunda specifika begrepp och definitioner. \\ \Emph{Varför?} Begreppen är avgörande för förståelsen av uppgiften.
\item Notera förenklingar, antaganden och specialfall. \\ \Emph{Varför?} Uppgiften blir mkt enklare om du inte behöver hantera dessa.
\item \Alert{Fråga} tentamensansvarig om du inte förstår uppgiften -- speciellt om det finns misstänkta felaktigheter eller förmodat oavsiktliga oklarheter. \\ \Emph{Varför?} Det är inte lätt att konstruera en ''perfekt'' tenta. \\ Du får fråga vad du vill, men det är inte säkert du får svar :)
\item Läs specifikationskommentarerna och metodsignaturerna i alla givna klass-specifikationer \Alert{mycket noga}. \\ \Emph{Varför?} Det är ett vanligt misstag att förbise de ledtrådar som ges där.
\item Återskapa din memorerade personliga checklista för vanliga fel som du brukar göra och avsätt tid till att gå igenom den på tentan. Varje fix plockar poäng!
\item Lämna in ett försök även om du vet att lösningen inte är fullständig. Det gäller att ''plocka poäng'' på så mycket som möjligt. En dålig lösning kan ändå ge poäng.

\item Om du har svårigheter kan det bli kamp mot klockan. Försök hålla huvudet kallt och prioritera utifrån var du kan plocka flest poäng. Ge inte upp! Ta en kort äta-dricka-paus för att få mer energi!

\end{enumerate}
\end{Slide}

\ifkompendium\else

\begin{Slide}{Planeringstips}\SlideFontTiny
Exempel på saker som du kan lägga in tid för i din julpluggkalender:
\begin{enumerate}
\item Välja ut övningar att repetera
\item Repetera övning X, Y, Z, ... Både läsa och skriva kod. Fundera på typ och värde.
\item Välja ut labbar att repetera
\item Repetera labb X, Y, Z, ... Lär dig ''trick'' och ''mönster''.
\item Träna på att skriva program med papper och penna
\item Göra checklista med vanliga fel
\item Läsa igenom extentor i Java
\item Välja ut minst en Java-extenta att göra som i skarpt läge i Scala
\item Gör Java-extentor X, Y, Z, ... implementera (delar) i Scala
\item Gör utvalda delar av extenta X, Y, Z, ... i Java
\end{enumerate}
\end{Slide}

\Subsection{Avgränsning}

\begin{Slide}{Tentans struktur}
\begin{itemize}
\item Del A 20\%:\\\Emph{Läsa uttryck} där du ska \Alert{ange typ och värde}
\begin{itemize}\SlideFontTiny
\item Du kommer att behöva skriva ner delsteg och variablers värden (minnet)
\item Testar förståelse av variabler, uttryck, samlingar, algoritmer, arv, etc.
\item Uppdaterad (mildare) regel om ''rättningströskel'': \\
Ur senaste compendium.pdf kapitel 0.8: \textit{Om du på del A erhåller färre poäng än vad som krävs för att nå upp till en bestämd ''rättningströskel'', kan din tentamen komma att underkännas utan att del B bedöms.}
\item Liknar kompendiets övningar; rimlig att lösa och dubbelkolla på 1h
\end{itemize}


\item Del B 80\%:\\\Emph{Skriva kod} som uppfyller \Alert{krav och designspecifikation}
\begin{itemize}\SlideFontTiny
\item Testar att du själv kan skapa kod med delar som samverkar
\item Testar förmåga att gå från indata-utdata till algoritm \\
 givet: ledtrådar, design, ev. skiss på lösning, ev. pseudokod etc.
\item Liknar kompendiets labbar; rimlig att lösa och dubbelkolla på 4h 
\end{itemize}

\end{itemize}
\end{Slide}



\begin{Slide}{Vad kommer på tentan? (1 av 3)}\SlideFontTiny
\hspace{-1em}\begin{minipage}{1.0\textwidth}
Allmänt: 
\begin{itemize}\SlideFontTiny
\item Begrepp som är ''fördjupning'' krävs ej på tentan (men ökar förståelse)
\item Ok om du väljer en enklare lösning med basala begrepp som fungerar bra, \\i stället för en kortare/elegantare/mer avancerad lösning
\item Dessa moduler ingår ej på tentan: ''Trådar, webb'', ''Design, api''
\end{itemize}

\vspace{1em}\begin{tabular}{l | l | l}
\textbf{Modul} & \textit{Ingår t.ex.}& \textit{Avgränsning} (ej krav; ok anv. om lämpl.)\\\hline
Introduktion & uttryck, aritmetik, slumptal, & kan ha nytta av deMorgan men ej krav\\
             & strängar, typer, Unit   & skriva egna \code|s"$x"| (men kunna läsa)  \\
             & skillnad mellan heltal \& flyttal & Float, Byte, Short\\
             & variabler, for, while, if & hex-literaler, backticks\\ 
\hline
Kodstrukturer & iterering, SWAP, SUM, MIN/MAX & import, paketnamn\\             
              & loopar, Range, sats vs uttryck & ok att välja vilken loop du tycker passar\\
              & namn, synlighet, skuggning & scaladoc, javadoc, jar \\        
\hline
Funtioner,    & definiera, anropa, parameter& skapa egen kontrollstruktur\\
objekt        & returtyp, namnarop, defaultarg & stegad funktion, rekursion\\        
              & punktnotation, objekt vs static & lazy val\\        
              & map/foreach med egen funktion & \\
              & anonyma funktioner (lambda)  & \\                              

\end{tabular}
\end{minipage}
\end{Slide}


\begin{Slide}{Vad kommer på tentan? (2 av 3)}\SlideFontTiny
\hspace{-2em}\begin{minipage}{1.0\textwidth}
\begin{tabular}{l | l | l}
\textbf{Modul} & \textit{Ingår t.ex.}& \textit{Avgränsning} (ej krav; ok anv. om lämpl.)\\\hline
Datastrukt. & attribut, medlem, metod, klass & isInstanceOf (anv. match istället) \\
            & tupler, Vector, Set, Map & List (oftast Vector istället)\\
            & Source.fromFile          & java.nio.file \\
\hline
Sekvensalg. &  skapa ny samling från befintlig &  \\      
            &  registrering, Scanner, ArrayBuffer & StringBuilder\\
            &  uppdatera Array, ArrayBuffer, Vector & \\
            &  slumptalsfrö, scala.util.Random  &  \\
\hline

Klasser     &  new, this, synlighet  & null \\
            &  inkapsling, accessregler, private  & private[this] \\
            &  klassparameter, fabriksmetod  & \\
            &  class vs case class    & \\
            &  referenslikhet vs innehållslikhet    & \\
            &  föränderlig vs oföränderlig klass & \\
\hline
Arv         &  bastyp, subtyp, trait, extends  & \\
            &  överskuggning,                  & inmixning, \\
            &  Any, AnyVal, AnyRef, Object     & Null, Nothing\\
            &  accessregler vid arv, protected & final\\
            &  abstract class, case object     & \\
            
\end{tabular}
\end{minipage}
\end{Slide}


\begin{Slide}{Vad kommer på tentan? (3 av 3)}\SlideFontTiny
\hspace{-2em}\begin{minipage}{1.0\textwidth}
\begin{tabular}{l | l | l}
\textbf{Modul} & \textit{Ingår t.ex.}& \textit{Avgränsning} (ej krav; ok anv. om lämpl.)\\\hline

Mönster     & match, Option, Try & try catch, unapply\\
            & flatten, sealed            & flatMap, partiella funktioner\\
            & enkel equals utan arv     & hashcode, fullständig equals   \\ 
            & wildcard-mönster  & variabelbindn. i mönster, sekvensmönster\\
\hline

Matriser,     & indexering i nästlade strukturer & \\
typparametrar & nästlad for-sats  & \\ 
              & matriser i Java med array  & \\
              & använda generiska strukturer & skapa generiska strukturer\\ 
\hline

Sök, sortera & linjärsökning, binärsökning & algoritmisk komplexitet\\
            & compareTo, strängjämförelse & Ordering, Ordered\\
            & insättningssortering & räcker kunna en valfri sortering \\
\hline


Scala/Java & översätta enkel Java/Scala & try catch i Java \\
           & implemenetera Java-klass     &  arv i Java med super vid konstr.\\
           & grundläggande syntaxskillnader & \\
           & ArrayList vs ArrayBuffer & java.util.\{List, Set\}\\
           & Autoboxing vid ArrayList<Integer> & \\
\multicolumn{3}{c}{OBS! Java-övningar finns även här och där i andra moduler}\\
\hline           
     
\end{tabular}
\end{minipage}
\end{Slide}


\Subsection{Tips vid val av lösningar}


\begin{Slide}{Tips om val av klass/trait}\SlideFontSmall
Ofta ger tentan en specifik design, men du kan ha stor nytta av egna abstraktioner, speciellt \Emph{lokala funktioner} för att göra enklare dellösningar!

\pause\vspace{1em}Om du skulle behöva samla både attribut och metoder utöver givan specifikationer:
Singelobjekt, case-klass, klass, trait eller abstrakt klass?
\begin{itemize}\SlideFontTiny
\item Använd \code{object} om du behöver samla metoder (och variabler) i en modul som bara finns i en upplaga. Du får lokal namnrymd och punktnotation på köpet.
\item Använd en \code{case class} om du har \Emph{oföränderlig data}. Du får då även innehållslikhet, möjlighet till mönstermatchning, etc. på köpet! 
\item Behöver du \Alert{föränderligt tillstånd} använd en vanlig \code{class}.\\ Det normala är att tillståndet (alla attribut) är \code{private} eller \code{protected} och att all uppdatering och avläsning av tillståndet sker indirekt genom metoder (getters/setters/...). 
\item Behöver du en abstrakt bastyp utan konstruktorparametrar använd en \code{trait}. \\(Du får inmixningsmöjlighet med \code{with} på köpet. Inmixning kommer ej på tenta.)
\item Behöver du en abstrakt bastyp med konstruktorparametrar använd en \code{abstract class}. (Går dock ej att använda vid inmixning med \code{with}.)
\end{itemize}
\end{Slide}


\begin{Slide}{Tips om hur man läser en specifikation}\SlideFontSmall
När du läser en specifikation av en klass, en trait, eller ett singelobjekt:
\begin{itemize}
\item Tänk igenom vilket ansvar olika delar av koden har
\item Vad håller klassen reda på? \\$\rightarrow$ Ledtrådar till attribut
\item Vad ska klassen göra/räkna ut? \\$\rightarrow$ Ledtrådar till metoder och deras algoritm
\item Vilka andra klasser har nytta av denna metod? \\$\rightarrow$ Ledtrådar till hur klasserna samverkar för att lösa uppgiften
\end{itemize}
Rita gärna en bild med ett specifikt exempel på vilken data som olika instanser håller reda på och fundera på hur data skapas/beräknas/förändras
\end{Slide}


\begin{Slide}{Tips om val av samling}\SlideFontSmall

Generellt: Det är ofta enklare med oföränderliga samlingar med oföränderliga element och skapa nya samlingar vid förändring. Men ibland blir det enklare om man har föränderliga samlingar.

\begin{itemize}
\item Behöver du hantera värden \code{x} av t.ex. typen String med \Emph{heltalsindex}?
\begin{itemize}\SlideFontTiny
\item Om du klarar dig utan förändring av innehållet:\\ \code{ val xs: Vector[String]}
\item Om du behöver ändra innehåll men \Alert{inte} antal element: \\ \code{ val xs: Array[String]} 
\item Om du behöver ändra innehåll \Alert{och} antal element: 
\\ \code{ var xs: Vector[String] } (se metoden \code{patch}) eller \\ \code{ val xs: ArrayBuffer[String]} (har metoden \code{insert})
\end{itemize}

\item Behöver du hantera värden \code{x} som ska vara unika?
\begin{itemize}\SlideFontTiny
\item Oföränderlig: \code{  val xs: Set[String] }
\item Förändringsbar: \code{val xs: scala.collection.mutable.Set[String]}
\end{itemize}

\item Behöver du leta upp värden \code{x:Int} utifrån en nyckel av t.ex. String?
\begin{itemize}\SlideFontTiny
\item Oföränderlig: \code{   val xs: Map[String, Int] }
\item Förändringsbar: \code{val xs: scala.collection.mutable.Map[String, Int]}
\end{itemize}


\end{itemize}
\end{Slide}

\begin{Slide}{Tillåtna uppdateringar i din QuickRef}
Du får med egen penna göra dessa fixar i din QuickRef:
\begin{itemize}
\item Grundtypernas implementation, sid 4: 
\begin{itemize}

\item omfång för Int ska ha exponent 31 (inte 15), 
\item omfång för Long ska ha exponent 63 (inte 15).
\end{itemize}

\item Saknade samlingsmetoder: 
\begin{itemize}
\item Under rubriken "Methods in trait Map[K, V]" saknas metoderna keySet och mapValues. 
\item Saknade metoderna för mutable.ArrayBuffer[T]: \\ \code{update} \code{insert} \code{remove} \code{append} \code{prepend}, etc. \\ lägg till beskrivning på lediga platsen på sista sidan \\ 
(se vidare commit \href{https://github.com/lunduniversity/introprog/commit/a5e29d000062}{a5e29d000062a} i kursrepot)
\end{itemize}
\end{itemize}
\end{Slide}


\begin{Slide}{ArrayBuffer}
Viktigast att känna till: update, insert, remove, append
{\SlideFontTiny

\vspace{2.5em}\begin{tabular}{@{}p{4.2cm}  p{6.5cm}}
\texttt{xs(i) = x \newline xs.update(i, x)} & Replace element at index i with x. \newline Return type Unit.\\   \cline{1-2}

\texttt{xs.insert(i, x)\newline xs.remove(i)} & Insert x at index \texttt{i}. Remove element at i. \newline Return type Unit.\\   \cline{1-2}

\texttt{xs.append(x)~~~xs~+=~x} & Insert x at end.  Return type Unit.\\   \cline{1-2}

\texttt{xs.prepend(x)~~x~+=:~xs} & Insert x in front.  Return type Unit.\\   \cline{1-2}

\texttt{xs -= x} & Remove first occurance of x (if exists). \newline Returns xs itself. \\\cline{1-2}

\texttt{xs ++= ys} & Appends all elements in ys to xs and returns xs itself. \\

\end{tabular}
}
\end{Slide}



\Subsection{Genomgång av extenta}
\begin{Slide}{Extenta 2016-08-24 TimePlanner}\SlideFontSmall
\url{http://cs.lth.se/pgk/examination/}

\vspace{1em}\Alert{TimePlanner}: 
\begin{itemize}
\item \href{http://fileadmin.cs.lth.se/cs//Education/grundkurs/extentor/160824.pdf}{tentamen 160824} 
\item \href{http://fileadmin.cs.lth.se/cs//Education/grundkurs/extentor/sol-160824.pdf}{lösningsförslag Java} 
\item \href{https://github.com/lunduniversity/introprog/tree/master/compendium/examples/exam/re-impl-java-exams/timeplanner-160824}{översättning av lösning till Scala}
\end{itemize}
\end{Slide}


\Subsection{Avslutning av kursen}

\begin{Slide}{Obligatoriska moment}\SlideFontSmall
\begin{itemize}
\item Kolla vilka oblikatoriska moment du har kvar här:
\url{http://fileadmin.cs.lth.se/pgk/SAM-EDAA45-snapshot.html}
\item Sök på din födelsemånad/dag, tex 0401 för första April.
\item OBS! Kan ännu saknas rapportering av det som hände i fredags.
\item Läs \Alert{alla} instruktioner \Alert{noga} och \Alert{anmäl dig} här: \\
\href{http://www.student.lth.se/studieinformation/anonyma-tentor/}{www.student.lth.se/studieinformation/anonyma-tentor}
\item Du ska vara godkänd på alla labbar+projekt för att få tenta pgk EDAA45!
\item Du ska vara godkänd på alla labbar+projekt för att få gå pfk \href{http://cs.lth.se/edaa01vt}{EDAA01}!
\item Använd återstående \Emph{resurstider} för \Alert{redovisning av labbar/projekt}.
\end{itemize}
\end{Slide}
%
%\begin{Slide}{CEQ -- Course Experience Questionnaire}\SlideFontSmall
%\begin{itemize}
%\item Görs på hela LTH på samma sätt. Alla får länkar via mejl.
%\item Snälla fyll i CEQ! Jag är \Alert{mycket tacksam} för all konstruktiv feedback! \\ Hög svarsfrekvens är viktigt för att kunna dra slutsatser om variationen i svaren och signifikansen i sammanställningen.
%\item Del 1: Generella påståenden, alla med 5-gradig skala: \\ tar helt avstånd ... instämmer helt
%\item Del 2: \Emph{Fritextfrågor}: \\
%''Vad  tycker  du  var  det  bästa  med  den här  kursen?'' \\
%''Vad  tycker  du  främst  behöver  förbättras?''
%\item Fördel med CEQ: Samma alla kurser alla år medger jämförelse över tid.
%\item Begränsning med CEQ: Saknar frågor kopplat till specifika kursmoment.
%\item Mer om CEQ här: \url{https://www.ceq.lth.se/}
%\end{itemize}
%\end{Slide}
%
%\begin{Slide}{Kursspecifik utvärdering om specifika kursmoment}\SlideFontSmall
%\begin{itemize}
%\item Jag vill gärna att alla gör den LTH-gemensamma, anonyma kursutvärderingsenkäten \href{https://www.ceq.lth.se/}{CEQ}. Dina fritext-kommentarer om vad som är det bästa med kursen och vad som främst behöver förbättra emottages mycket tacksamt i CEQ-utvärderingen!
%\item Jag kommer att komplettera CEQ med en \Emph{kursspecifik} utvärdering av specifika kursmoment i denna kurs och jag är därför \Alert{mycket tacksam} om alla fyller enkäten när länk kommer via mejl. 
%\item Jag behandlar dina svar konfidentiellt, men ber om din STiL-id så att jag kan återkomma om jag mot förmodan undrar något mer.
%\item Din input är mycket värdefull vid framtida kursutveckling!
%\end{itemize}
%\end{Slide}
%
%\begin{Slide}{Intresserad av att arbeta som handledare?}\SlideFontSmall
%\begin{itemize}
%\item 
%\end{itemize}
%\end{Slide}
%
%\begin{Slide}{Utblick}\SlideFontSmall
%Framtiden för \Emph{Scala}:
%\begin{itemize}
%\item Scala 2.12 bättre bytekod, Scala 2.13 bättre standardbibliotek
%\item dotty och tasty
%\item Scala.JS: dela kod+kompetens mellan backend och frontend
%\item Scala native: kör Scala kompilerat direkt ''på metallen''
%\end{itemize}
%Några framtida \Emph{kurser} som direkt bygger på pgk:
%\begin{itemize}
%\item Fördjupningskursen (Java)
%\item Utvärdering av programvarusystem (R)
%\item Diskreta strukturer (Clojure)
%\item Programvaruutveckling i grupp 
%\item Objekt-orienterad modellering och design
%\item Funktionsprogrammering 
%\end{itemize}
%
%\end{Slide}
%
%
%\begin{Slide}{Hoppas att kursen varit kul och lärorik!}
%\includegraphics[width=5cm]{../img/gurka.jpg}\includegraphics[width=5cm]{../img/ukulele.jpg}
%\end{Slide}
%
%\begin{Slide}{Ett stort TACK för...}
%\begin{itemize}
%\item ... att ni kämpat så hårt!
%\item ... att ni ställt massor med frågor!
%\item ... att det har varit så hög närvaro på föreläsningarna!
%\item ... att ni är så konstruktiva och verkligen vill lära er!
%\end{itemize}
%\vspace{2em} \pause
%
%\Alert{Ett stort LYCKA TILL på vägen till att bli en \\ kompetent och innovativ systemutvecklare!}
%\end{Slide}


\fi
     


\part{Appendix}         
\appendix
%!TEX root = ../compendium.tex

\chapter{Terminalfönster och kommandoskal}

\section{Vad är ett terminalfönster?}

I ett terminalfönster kan man skriva kommandon som till exempel kör program och hanterar filer på din dator. När man programmerar använder man ofta terminalkommando för att kompilera och exekvera sina program.   
 
\subsubsection{Terminal i Linux}

\subsubsection{PowerShell i Microsoft Windows}
Microsoft Windows är inte Unix-baserat, men i kommandotolken PowerShell finns alias definierat för en del vanliga unix-kommandon. Du startar Powershell t.ex. genom att genom att trycka på Windows-knappen och skriva \texttt{powershell}.

\subsubsection{Terminal i Apple OS X}
Apple OS X är ett Unix-baserat operativsystem. Många kommandon som fungerar under Linux fungerar också under Apple OS X.

\section{Några viktiga terminalkommando}
%!TEX encoding = UTF-8 Unicode
%!TEX root = ../compendium.tex

\chapter{Editera}\label{appendix:edit}
\section{Vad är en editor?}

En editor används för att redigera programkod. Det finns många olika editorer att välja på. Erfarna utvecklare lägger ofta mycket energi på att lära sig att använda favoriteditorns kortkommandon och specialfunktioner, eftersom detta påverkar stort hur snabbt kodredigeringen kan göras. 

En bra editor har \textbf{syntaxfärgning} för språket du använder, så att olika delar av koden visas i olika färger. Då går det lättare att läsa och hitta i koden. 

I en integrerad utvecklingsmiljö (se appendix \ref{appendix:ide}) finns en inbyggd editor som, förutom syntaxfärgning, har fler avancerade funktioner. 

\section{Välj editor}

I tabell \ref{editor:popular-editors} visas en lista med några populära editorer. Det är en stor fördel om en editor finns på flera plattformar så att du har nytta av dina inövade färdigheter när du behöver växla mellan olika operativsystem. 

Om du inte vet vilken du ska välja, börja med \textit{gedit}, som inte är så avancerad, men därför lätt att kommna igång med. När du sedan är redo att investera din lärtid i en mer avancerad editor rekommenderas \textit{Atom}, eftersom den är öppen, gratis och finns för Linux, Windows och macOS. 

Det är är också bra att lära sig åtminståne de mest basala kommandona i editorn \textit{vim} eftersom denna  editor kan köras direkt i terminalen, även vid fjärrinloggning, och finns förinstallerad i de flesta Linux-system.
%!TEX encoding = UTF-8 Unicode
%!TEX root = ../compendium.tex

\chapter{Kompilera och exekvera}\label{appendix:compile}

\section{Vad är en kompilator?}

\section{Java JDK}

\subsection{Installera Java JDK}

\section{Scala}

\subsection{Installera Scala-kompilatorn}

\subsection{Scala Read-Evaluate-Print-Loop (REPL)}\label{appendix:compile:REPL}

För många språk, t.ex. Scala och Python, finns det en interaktiv tolk som gör det möjligt att exekvera enstaka programrader och direkt se effekten. En sådan tolk kallas Read-Evaluate-Print-Loop eftersom den läser en rad i taget och översätter till maskinkod som körs direkt.    

\TODO Kortkommandon: Ctrl+K etc.

\TODO :paste


%!TEX root = ../compendium.tex

\chapter{Dokumentation}

\section{Vad gör ett dokumentationsverktyg?}

\section{scaladoc}

\section{javadoc}
%!TEX root = ../compendium.tex

\chapter{Integrerad utvecklingsmiljö}

\section{Vad är en IDE?}

\section{ScalaIDE och Eclipse}

\subsection{Installera ScalaIDE}

\section{Handledning ScalaIDE/Eclipse}
%!TEX root = ../compendium.tex

\chapter{Byggverktyg}

\section{Vad gör ett byggverktyg?}

\section{Byggverktyget sbt}

\subsection{Installera sbt}

\subsection{Använda sbt}
%!TEX encoding = UTF-8 Unicode
%!TEX root = ../compendium.tex


\chapter{Versionshantering och kodlagring}

\section{Vad är versionshantering?}

\textbf{Versionshantering}\footnote{\href{https://en.wikipedia.org/wiki/Version_control}{en.wikipedia.org/wiki/Version\_control}} \Eng{version control eller revision control} av mjukvara innebär att hålla koll olika versioner av koden i ett utvecklingsprojekt allteftersom koden ändras. Versionshantering är en deldisciplin inom \textbf{konfigurationshantering} \Eng{software configuration managament} som inbegriper allt i processen för att identifiera, besluta, genomföra och följa upp ändringar.

En viktig del av versionshantering är att \textit{lagra} olika versioner av koden allt eftersom den utvecklas, så att tidigare versioner kan \textit{återskapas} vid behov. Ett bra verktygsstöd och en väldefinierad arbetsprocess för versionshanteringen, som alla i utvecklingsprojektet följer, möjliggör att flera utvecklare kan \textit{arbeta parallellt} med att sammanfoga \Eng{merge} varandras tillägg och ändringar i den gemensamma kodbasen utan att det blir kaos och förvirring.

God versionshantering är helt avgörande för utvecklarnas produktivitet, speciellt för stora projekt med många utvecklare som jobbar parallellt mot en omfattande kodbas med många olika interna och externa komponenter. 
Men även ett litet projekt med en enda utvecklare kan ha god nytta av ett versionshanteringsverktyg och ett disciplinerat förfarande för att namge versioner, t.ex. för att kunna återskapa tidigare versioner av projektets olika kodfiler när en ändring visar sig mindre lyckad.   

Det finns flera olika modeller för hur kodlagringen sker:
\begin{itemize}
\item \textbf{lokal}; alla utvecklare jobbar i samma, lokala filsystem där alla olika versioner lagras.
\item \textbf{centraliserad}; ett repositorium (förk. repo), alltså en databas med koden, finns centralt på en server som alla jobbar mot med hjällp av en versionshanteringsklient.
\item \textbf{distribuerad}; alla utvecklare har sitt eget lokala repo och varje utvecklare initierar enskilt när ändringar ska delas mellan olika repo. 
\end{itemize}


\section{Versionshanteringsverktyget Git}

Det finns många olika versionshanteringsverktyg\footnote{\href{https://en.wikipedia.org/wiki/List_of_version_control_software}{https://en.wikipedia.org/wiki/List\_of\_version\_control\_software}}
 som använder olika modeller för kodlagring; lokal, centraliserad, distribuerad eller kombinationer därav. 
På senare tid har verktyget \textbf{Git}\footnote{\href{https://en.wikipedia.org/wiki/Git_(software)}{https://en.wikipedia.org/wiki/Git\_(software)}} fått en stark ställning, speciellt i öppenkällkodsvärlden. Git utvecklades ursprungligen av Linus Torvalds för att versionshantera Linuxkärnan, men har växt till ett omfattande öppenkällkodsprojekt med stor spridning och många användare och bidragsgivare. 

Git möjliggör \textbf{distribuerad} versionshantering där var och en kan jobba snabbt och smidigt i sitt lokala repo, utan att behöva vänta på att en server ska synkronisera ett centralt repo över nätverket. Ändringar delas mellan repo på begäran ev enskilda utvecklare. 

Varje ny version av koden lagras som en avgränsad mängd ändringar sedan förra versionen, en s.k. \textbf{commit}%
\footnote{På svenska kan t.ex. ''inlämning'' användas, men låneordet commit är redan etablerat.}%
, och hanteras internt av Git i en lokal databas i katalogen \code{.git} som ligger överst i din projektkatalog. Genom olika kommandon i terminalen, eller via en klient med ett grafiskt användargränssnitt, kan din kod överföras till och från den lokala koddatabasen, alternativt delas med andra repon via nätet. 

Det finns en välskriven bok kallad \textit{''Pro Git''} som förklarar Git på djupet och är tillgänglig fritt här: 
\url{https://git-scm.com/book/en/v2}.
Läs kapitel 1 och 2 så får du en bra grund att stå på. 

Dessa termer är bra att kunna utantill innan du kör igång med Git:
\newcommand{\TermItem}[3]{\item \textbf{#1} (\textit{substantiv}: #2, \textit{verb}: #3).}
\begin{itemize}
\item \textbf{repo} (\textit{substantiv}: ett repositorium, \textit{eng. a repository}) En koddatabas med ändringshistorik. 
\TermItem{commit}{inlämning}{lämna in} 
  En avgränsad mängd ändringar sedan förra versionen lämnas in i det lokala repot.
\TermItem{push}{en leverans}{att leverera, att trycka upp} En eller flera inlämningar trycks upp till ett annat repo.
\TermItem{pull}{en hämtning}{att hämta, att dra ner} En eller flera inlämningar dras ner från ett annat repo.
\TermItem{merge}{en ihopslagning}{att sammanfoga} En eller flera inlämningar slås samman till en ny inlämning. 
\item \textbf{merge conflict} (\textit{substantiv}: en sammanfogningskonflikt, \textit{eng. a merge conflict}) Ändringar i samma kodfil som inte enkelt kan sammanfogas på ett entydigt sätt.
\TermItem{pull request}{en hämtningsbegäran, förk. PR}{att begära en hämtning} Utvecklare A ber en annan utvecklare B att hämta en eller flera inlämningar från A:s repo och sammanfoga med B:s repo.

\end{itemize}

\subsection{Installera git}

Git finns förinstallerat på LTH:s Linuxdatorer. Du kan kolla om Git redan finns på din maskin genom att skriva \code{git help} i terminalen. 

Det finns bra instruktioner om hur du installerar Git på din egen maskin här: \url{https://git-scm.com/book/en/v2/Getting-Started-Installing-Git}

Om du vill ha en Git-klient med grafiskt användargränssnitt finns det många att välja på, se här:  \url{https://git-scm.com/downloads/guis} 

Om du inte vet vilken du ska välja, prova GitGraken som är gratis, fri, öppen och finns för alla plattformar och kan laddas ner här: \\ \url{https://www.gitkraken.com/}

\subsection{Använda git}

\subsubsection{Anpassa Git}

Innan du börjar använda git, konfigurera ditt användarnamn och din email med nedan terminalkommando, där du anger ditt användarnamn i stället för \code{fornamnefternamn} och din mejladress i stället för \code{mejladr@plats.se}:
\begin{REPLnonum}
$ git config --global user.name fornamnefternamn
$ git config --global user.email mejladr@plats.se
\end{REPLnonum}
Det är bra att välja \textit{ett} användarnamn, för \textit{alla} repo, även kodlagringsplatser på nätet; förslagsvis \code{fornamnefternamn} utan svenska tecken,  så att du blir lätt att känna igen, speciellt om du jobbar med öppen källkod där ditt namn kommer associerat med alla de kodbidrag du gör under ditt yrkesliv.

Läs mer om hur du gör andra inställningar här, t.ex. hur du anger vilken editor som git startar när du ska skriva commit-beskrivningar: \\ \url{https://git-scm.com/book/en/v2/Getting-Started-First-Time-Git-Setup}
  
  
\subsubsection{Några vanliga kommandon}


  
\section{Kodlagringsplastser på nätet}

\begin{itemize}

\TermItem{fork}{en förgrening av ett helt repo}{att förgrena ett repo, ''forka''} En kopia av ett annat repo som utvecklas separat. Gör det möjligt för dig att införa ändringar i en kodbas, även om du inte har rättigheter att leverera till (''pusha till'') originalet. 
\item \textbf{upstream} (\textit{preposition}: uppströms, \textit{substantiv}: uppströmsrepo) Ett uppströmsrepo utgör orginal till ett förgrenat repo (fork). 
\begin{itemize}
\item Här beskrivs hur du länkar en förgrening uppströms: \\ 
{\small\url{https://help.github.com/articles/configuring-a-remote-for-a-fork/}}

\item Här beskrivs hur du synkar en förgrening uppströms:\\
{\small\url{https://help.github.com/articles/syncing-a-fork/}}

\end{itemize}

\end{itemize}



\subsection{GitLab}


\subsection{GitHub}

\subsubsection{Installera klienten för GitHub}

\subsubsection{Använda GitHub}


\subsection{BitBucket}

\subsubsection{Installera SourceTree}

\subsubsection{Använda BitBucket och SourceTree}

%!TEX root = ../compendium.tex

\chapter{Virtuell maskin}\label{appendix:vbox}

\section{Vad är en virtuell maskin?}

Du kan köra alla kursens verktyg i en så kallad virtuell maskin (vm). Det är ett enkelt och säkert sätt att installera ett nytt operativsystem i en "sandlåda" som inte påverkar din dators ursprungliga operativsystem. 

\section{Installera kursens vm}
Det finns en virtuell maskin förberedd med alla verktyg som du behöver förinstallerade. Gör så här:
\begin{enumerate}
\item     Installera VirtualBox v5 här: \\ \url{https://www.virtualbox.org/wiki/Downloads}
\item     Ladda ner filen vbox.zip här: \\ \url{http://fileadmin.cs.lth.se/pgk/vbox.zip} \\ OBS! Då filen är på nästan 4GB kan nedladdningen ta mycket lång tid.
\item     Packa upp filen vbox.zip i biblioteket "VirtualBox VMs" som du fick i din hemkatalog när du installerade VirtualBox. Du får då 3 filer som heter något med "introprog-ubuntu-64bit".
\item     Kolla med hjälp av denna sida: \\ \url{https://md5file.com/calculator} \\ så att filen "introprog-ubuntu-64bit.vdi" har denna sha256-cheksumma: \\ --- ska-stå-checksumma-här-sen ---
\item     Öppna VirtualBox och lägg till maskinen introprog-ubuntu-64bit genom menyn ''add''.
\item     Starta maskinen.
\item     Öppna ett terminalfönster och skriv scala och du är igång och kan göra första övningen!
\end{enumerate}

\section{Vad innehåller kursens vm?}

Den virtuella maskinen kör Xubuntu 14.04 med fönstermiljön XFCE, vilket är samma miljö som E-husets linuxdatorer kör. 

I den virtuella maskinen finns detta förinstallerat:

\begin{itemize}
\item Java JDK 8
\item Scala 2.11.8
\item Kojo 2.4.08
\item Eclipse Mars.2 med ScalaIDE 4.3
\item gedit med syntaxfärgning för Scala och Java
\item git
\item sbt
\item Ammonite REPL
\end{itemize}
%!TEX encoding = UTF-8 Unicode
%!TEX root = ../compendium.tex

\chapter{Nyckelord}\label{appendix:keywords}

\section{Vad är ett nyckelord ord?}

Nyckelord är ord i ett programmeringsspråk som som har speciell betydelse och reserverade för endast ett användningsområde. Nyckelord kallas även \emph{reserverade ord}\footnote{Läs mer här: \href{https://en.wikipedia.org/wiki/Reserved\_word}{en.wikipedia.org/wiki/Reserved\_word}}. 
Man kan till exempel inte använda nyckelordet \code{def} som namn på en variabel. Nyckelord ges ofta en speciell färg av de kodeditorer som erbjuder \emph{syntaxstyrd färgning}. 

\section{Nyckelord i Scala} 

\begin{Code}
abstract    case        catch       class       def
do          else        extends     false       final
finally     for         forSome     if          implicit
import      lazy        macro       match       new
null        object      override    package     private
protected   return      sealed      super       this
throw       trait       try         true        type
val         var         while       with        yield
_    :    =    =>    <-    <:    <%     >:    #    @	
\end{Code}


\section{Nyckelord i Java}

Here is a list of keywords in the Java programming language. You cannot use any of the following as identifiers in your programs. The keywords const and goto are reserved, even though they are not currently used. true, false, and null might seem like keywords, but they are actually literals; you cannot use them as identifiers in your programs.

\begin{Code}[language=Java]
abstract 	continue 	for 	new 	switch
assert *** 	default 	goto * 	package 	synchronized
boolean 	do 	if 	private 	this
break 	double 	implements 	protected 	throw
byte 	else 	import 	public 	throws
case 	enum **** 	instanceof 	return 	transient
catch 	extends 	int 	short 	try
char 	final 	interface 	static 	void
class 	finally 	long 	strictfp ** 	volatile
const * 	float 	native 	super 	while
* 	  	not used
** 	  	added in 1.2
*** 	  	added in 1.4
**** 	  	added in 5.0
\end{Code}


%!TEX root = ../compendium.tex


\ChapterUnnum{Hur bidra till kursmaterialet?}


\chapter{Ordlista}

\chapter{Lösningar till övningarna}\label{chapter:solutions}
\foreach \n in {1,...,9}{%
  \input{modules/w0\n-solutions.tex}
}
\foreach \n in {10,...,14}{%
  \input{modules/w\n-solutions.tex}
}

\chapter{Snabbreferens}\label{chapter:quickref}

Detta appendix innehåller en snabbreferens för Scala och Java. Snabbreferensen är enda tillåtna hjälpmedel under kursens skriftliga tentamen. 

Lär dig vad som finns i snabbreferensen så att du snabbt hittar det du behöver och träna på hur du  effektivt kan dra nytta av den när du skriver program med papper och penna utan datorhjälpmedel.

\includepdf[pages={1-10}, scale=0.75, frame]{../quickref/quickref.pdf}


\end{document}
