%!TEX encoding = UTF-8 Unicode
\documentclass[a4paper]{compendium}

%\usepackage{xr} %to crossreference ???
\externaldocument{compendium2} %to crossreference to compendium2.tex

\usepackage[swedish]{babel}
\addto\captionsswedish{%
  \renewcommand{\appendixname}{Appendix}%
}
%TODO: Glossary
%http://tex.stackexchange.com/questions/5821/creating-a-standalone-glossary/5837#5837

\setlength{\columnsep}{16mm}

\newcommand{\LibVersion}{1.1.5} % latest version of introlib at https://github.com/lunduniversity/introprog-scalalib
\newcommand{\LibJar}{\texttt{introprog\_3-\LibVersion.jar}}
\newcommand{\JDKApiUrl}{\url{https://docs.oracle.com/en/java/javase/11/docs/api/}}
\newcommand{\CurrentYear}{2021}
\newcommand{\VMName}{vm2020} %TODO: update vm
\newcommand{\VMPassword}{pgkBytMig\CurrentYear}
\newcommand{\VirtualBoxVersion}{6.1} %https://www.virtualbox.org/wiki/Downloads
\newcommand{\UbuntuVersion}{20.04}
\newcommand{\ScalaVersion}{3.0.1} %https://www.scala-lang.org/
\newcommand{\SbtVersion}{1.5.3} %https://eed3si9n.com/category/tags/sbt
\newcommand{\JDKVersion}{11} %https://adoptopenjdk.net/
\newcommand{\KojoVersion}{2.9.10} %https://www.kogics.net/kojo-download
\newcommand{\VSCodeVersion}{1.41} %https://code.visualstudio.com/updates
\newcommand{\MetalsVersion}{v1.10.6} %https://marketplace.visualstudio.com/items?itemName=scalameta.metals
\newcommand{\WindowsVersion}{10}
\newcommand{\ScalaIDEVersion}{4.7.0} %%DEPRECATED




\title{
{\vspace{-3.0cm}\bf\sffamily\Huge\selectfont  Introduktion till programmering med Scala}
\\ \vspace{2em}%\hspace*{1.5cm}\inputgraphics[width=0.6\textwidth]{../img/gurka} \\
{\sffamily \textbf{Kompendium 1}\\ Första läsperioden: Modul 1 -- 7}\\\vspace{3cm}
%\includegraphics[height=8cm]{../img/logoLUeng.pdf}
%\includegraphics[height=5cm]{../img/scala-icon.png}
\includegraphics[height=8cm]{../img/blockworm}
%\includegraphics[height=4cm]{../img/java-logo.png}
%\includegraphics[height=12cm]{cover/gurka.jpg}
\vspace{2cm}
}

\author{Björn Regnell}
\date{\raggedbottom%
\vspace{1em}\begin{minipage}{1.0\textwidth}\centering
EDAA45, Lp1-2, HT \CurrentYear\\
Datavetenskap, LTH\\
Lunds universitet\\
~\\
Kompileringsdatum: \today \\
\url{https://lunduniversity.github.io/pgk}
\end{minipage}
}

\usepackage{multicol}

\usepackage{pgffor}  %% http://stackoverflow.com/questions/2561791/iteration-in-latex
                     %  allows:  \foreach \n in {1,...,4}{ do something with \n }

\usepackage{framed}  %  allows:   \begin{framed}\end{framed}
\FrameSep5pt
\OuterFrameSep0pt

% \newenvironment{Slide}[2][]{%
% \begin{oframed}\setlist{noitemsep}%
% {\vspace{-1.5\topsep}}%tighter frames
% \subsection{#2}%
% }%
% {\end{oframed}}

\newenvironment{Slide}[2][]{%
%\noindent\rule{\textwidth}{0.4pt}%
\setlist{noitemsep}%
%{\vspace{-1.5\topsep}}%tighter frames
\subsection{#2}%
}%
{~\newline\noindent\rule{\textwidth}{0.4pt}}

%\newenvironment{Slide}[2][]{\setlist{noitemsep}\subsection{#2}}{}


%\newcommand{\SlideHeading}[1]{\section*{#1}}

%\usepackage[most]{tcolorbox}
% \newenvironment{Slide}[2][]
%   {\vspace{0.5em}\begin{tcolorbox}[left=1.5em,%width=1.05\textwidth,
%   grow to right by=0.05\textwidth,grow to left by=0.05\textwidth,%
%   %breakable,
%   %frame hidden,
%   colframe=gray!20,
%   enhanced]\setlist{noitemsep}\SlideHeading{#2}}
%   {\end{tcolorbox}\vspace{0.5em}}

\newcommand{\Subsection}[1]{} %ignore slide sections
\newcommand{\SlideOnly}[1]{} %ignore slide font size

\usepackage[framemethod=tikz]{mdframed}

\newif\ifkompendium  % to allow conditional text in slides only showing up in compendium
\kompendiumtrue      % in slides: \kompendiumfalse

\newif\ifPreSolution  % to allow tasks and solutions in same file
\PreSolutiontrue      % in solutions: \PreSolutionfalse

\let\QUESTBEGIN\ifPreSolution  % to mark formatting and numbering of exercises
\let\SOLUTION\else  % to mark solutions in the same file as questions
\let\QUESTEND\fi    % to mark end of exercise


%!TEX encoding = UTF-8 Unicode
\newcommand{\ExeWeekONE}{expressions}
\newcommand{\LabWeekONE}{kojo}

\newcommand{\ExeWeekTWO}{programs}
\newcommand{\LabWeekTWO}{--}

\newcommand{\ExeWeekTHREE}{functions}
\newcommand{\LabWeekTHREE}{irritext}

\newcommand{\ExeWeekFOUR}{objects}
\newcommand{\LabWeekFOUR}{blockmole}

\newcommand{\ExeWeekFIVE}{classes}
\newcommand{\LabWeekFIVE}{turtlegraphics}

\newcommand{\ExeWeekSIX}{sequences}
\newcommand{\LabWeekSIX}{shuffle}

\newcommand{\ExeWeekSEVEN}{sets-maps}
\newcommand{\LabWeekSEVEN}{words}

\newcommand{\ExeWeekEIGHT}{matrices}
\newcommand{\LabWeekEIGHT}{maze}

\newcommand{\ExeWeekNINE}{inheritance}
\newcommand{\LabWeekNINE}{turtlerace-team}

\newcommand{\ExeWeekTEN}{patterns}
\newcommand{\LabWeekTEN}{chords-team}

\newcommand{\ExeWeekELEVEN}{scala-java}
\newcommand{\LabWeekELEVEN}{lthopoly-team}

\newcommand{\ExeWeekTWELVE}{sorting}
\newcommand{\LabWeekTWELVE}{survey}

\newcommand{\ExeWeekTHIRTEEN}{--}
\newcommand{\LabWeekTHIRTEEN}{Projekt}

\newcommand{\ExeWeekFOURTEEN}{threads}
\newcommand{\LabWeekFOURTEEN}{--}


\begin{document}

\pagenumbering{roman}

\frontmatter
\maketitle
%!TEX root = ../compendium.tex

\clearpage\null\thispagestyle{empty}
\vfill

{
\setlength{\parindent}{0pt}
\emph{Editor}: Björn Regnell, Faculty of Engineering LTH, Lund University. \\ 

\emph{Contributors}: 
Björn Regnell,
Per Holm,
Sandra Nilsson,
Patrik Andersson,
Gustav Cedersjö,
Maj Stenmark,
Anna Axelsson,
Roy Andersson,
Markus Borg,
Anton Klarén.
\\

\emph{Repo}: \url{https://github.com/lunduniversity/introprog} \\ \newline

This manuscript is on-going work. Contributions are welcome! \\ 
\emph{Contact}: \url{bjorn.regnell@cs.lth.se}
\\ \newline

\emph{LICENCE}: CC BY-NC-SA 4.0 \\
\url{http://creativecommons.org/licenses/by-nc-sa/4.0/}
\\ \newline
Copyright \copyright~Computer Science, LTH \& Björn Regnell. 2016. Lund. Sweden.\\
}

%!TEX encoding = UTF-8 Unicode
%!TEX root = ../compendium.tex

\ChapterUnnum{Framstegsprotokoll}\label{progress-protocoll}


\section*{Genomförda övningar}

\vspace{1em}\noindent
{Till varje laboration hör en övning med uppgifter som utgör förberedelse inför labben. Du behöver minst behärska grunduppgifterna för att klara labben inom rimlig tid. Om du känner att du behöver öva mer på grunderna, gör då även extrauppgifterna. Om du vill fördjupa dig, gör fördjupningsuppgifterna som är på mer avancerad nivå. Kryssa för nedan vilka övningar du har gjort, så blir det lättare för din handledare att anpassa dialogen till de kunskaper du förvärvat hittills.}

\newcommand{\TickBox}{\raisebox{-.50ex}{\Large$\square$}}
\newcommand{\ExeRow}[1]{\hyperref[section:exe:#1]{\texttt{#1}} & \TickBox  &  \TickBox &  \TickBox  \\ \addlinespace }

\begin{table}[h]
%\centering
\vspace{2em}
\begin{tabular}{lccc}
\toprule \addlinespace
{\sffamily Övning} &
{\sffamily Grund} &
{\sffamily Extra} &
{\sffamily Fördjupning}\\ \addlinespace \midrule \\[-0.7em]
\ExeRow{expressions}
\ExeRow{programs}
\ExeRow{functions}
\ExeRow{data}
\ExeRow{vectors}
\ExeRow{classes}
\ExeRow{traits}
\ExeRow{matching}
\ExeRow{matrices}
\ExeRow{sorting}
\ExeRow{scalajava}
\ExeRow{threads}
\bottomrule
\end{tabular}
\end{table}

\newpage

\section*{Godkända obligatoriska moment}

\vspace{1em}\noindent
För att bli godkänd på laborationsuppgifterna och projektuppgiften måste du lösa deluppgifterna och diskutera dina lösningar med en handledare. Denna diskussion är din möjlighet att få feedback på dina lösningar. Ta vara på den!
Se till att handledaren noterar nedan när du blivit godkänd på respektive obligatorisk moment. Spara detta blad tills du fått slutbetyg i kursen.


\vspace{2.5em}\noindent Namn: \dotfill\\

\vspace{1em}\noindent Namnteckning: \dotfill\\

\newcommand{\LabRow}[1]{\\[-1.1em] \hyperref[section:lab:#1]{\texttt{#1}} & \dotfill &  \dotfill  \\ \addlinespace }

\begin{table}[h]
%\centering
\vspace{1em}
\begin{tabular}{lcc}
\toprule \addlinespace
{\sffamily\bfseries\small Lab} & {\sffamily\small Datum gk} &	
{\sffamily\small Handledares signatur + namnförtydligande}\\ \addlinespace 
%\midrule 
\\[-0.5em]
%!TEX encoding = UTF-8 Unicode
%!TEX root = ../compendium2.tex
\LabRow{kojo}
\LabRow{irritext}
\LabRow{blockmole}
\LabRow{blockbattle}
\LabRow{shuffle}
\LabRow{words}
\LabRow{life}
\LabRow{snake}
\LabRow{tabular}
\LabRow{javatext}
%\toprule
\addlinespace 
%\midrule 
\addlinespace\addlinespace
{\sffamily\small {\bfseries Projektuppgift} (välj en)	} & \dotfill &  \dotfill  \\
\addlinespace\addlinespace %\midrule
{\Large$\square$}\texttt{~~~\hyperref[section:proj:bank]{bank}} &
\multicolumn{2}{c}{\textit{Om egendef., ge kort beskrivning här:}}  \\ \addlinespace
{\Large$\square$}\texttt{~~~\hyperref[section:proj:tabular]{tabular}} \\ \addlinespace
{\Large$\square$}\texttt{~~~\hyperref[section:proj:music]{music}} \\ \addlinespace
{\Large$\square$}\texttt{~~~\hyperref[section:proj:photo]{photo}}  \\ \addlinespace
{\Large$\square$}\texttt{~~~}\textit{egendefinerad}  \\
%\dotfill  \\
\addlinespace\addlinespace
%\midrule
\addlinespace
{\sffamily\small {\bfseries Muntligt prov}} &  & \\
\addlinespace\addlinespace{}
{\Large$\square$}\texttt{~~~} godkänd & \dotfill &  \dotfill \\
\addlinespace\addlinespace\bottomrule
\end{tabular}
\end{table}

%!TEX root = ../compendium.tex


\ChapterUnnum{Förord} 

Programmering är inte bara ett sätt att ta makten över systemen som styr vårt samhälle. Det är också ett kraftfullt verktyg för tanken. Att lära sig programmering och systemutveckling är första steget på en livslång resa av kontinuerligt lärande. Programmeringsspråk och utvecklingsverktyg kommer och går, men de grundläggande koncepten sekvens, alternativ, repetition och abstraktion som ligger bakom all mjukvara består. 

Detta kompendium utgör kursmaterial för studier i grundläggande programmering, med syfte att ge en solid bas för ingenjörsstudenter och andra som utvecklar system som innehåller mjukvara. 

Kompendiet är framtaget av, med och för studenter och lärare på universitetsnivå, och distribueras som öppen källkod. Det får användas fritt så länge erkännande ges och eventuella ändringar också publiceras som öppen källkod under samma licens som ursprungsmaterialet. På kursens hemsida \href{http://cs.lth.se/pgk}{cs.lth.se/pgk} och repo \href{http://github.com/lunduniversity/introprog}{github.com/lunduniversity/introprog} finns instruktioner om hur du kan bidra till kursmaterialet.

Läromaterialet fokuserar på lärande genom eget arbete och innehåller övningar och laborationer som är organiserade i moduler. Varje modul har ett tema och tillhörande föreläsningsanteckningar.

I kursen används språken Scala och Java för att illustrera grunderna i imperativ och objektorienterad programmering, tillsammans med elementär funktionsprogrammering. Mer avancerad objektorientering och funktionsprogrammering och  lämnas till fortsättningskurser. 



Den kanske viktigaste framgångsfaktorn vid studier i programmering är att bejaka din egen upptäckarglädje och experimentlusta. Det fantastiska med programmering är att dina egna intellektuella konstruktioner faktiskt \emph{gör} något som just \emph{du} har bestämt! Ta vara på det och prova dig fram genom att koda egna idéer -- det är kul när det funkar men minst lika lärorikt är felsökning, buggrättande och alla misslyckade försök som efter hårt arbete vänds till lyckade lösningar och bestående lärdomar. 

Välkommen i programmeringens fascinerande värld och hjärtligt lycka till med dina studier!




\setcounter{tocdepth}{2} % set headings level in table of contents

\tableofcontents
\mainmatter

\pagenumbering{arabic}


\part{Om kursen}
\setcounter{chapter}{-3}
%!TEX root = ../compendium.tex

\ChapterUnnum{Kursens arkitektur}

%!TEX encoding = UTF-8 Unicode
%!TEX root = ../lect-week01.tex

%%%%%%%%%%%%%%%%%%%%%%%%%%%%%%%%%%%%%%
\Subsection{Om kursen}

%%%

\ifkompendium\else
\begin{Slide}{Nytt för i år}
\begin{itemize}
\item \Emph{Scala} införs som förstaspråk på Datateknikprogrammet.
\item Den \Emph{största förnyelsen} av den inledande programmeringskursen sedan vi införde Java 1997.
\item Allt kursmaterial är \Emph{öppen källkod}.
\item \Emph{Studentermedverkan} i kursutvecklingen.
\end{itemize}
\vspace{1em}\hskip1em\href{https://www.lth.se/nyheter-och-press/nyheter/visa-nyhet/article/scala-blir-foerstaspraak-paa-datateknikprogrammet/}{www.lth.se/nyheter-och-press/nyheter/visa-nyhet/article/\\\hskip1emscala-blir-foerstaspraak-paa-datateknikprogrammet/}
\end{Slide}
\fi

\begin{Slide}{Veckoöversikt}
\noindent\resizebox{0.9\columnwidth}{!}{
%!TEX encoding = UTF-8 Unicode
\begin{tabular}{l|l|l|l}
\textit{W} & \textit{Modul} & \textit{Övn} & \textit{Lab} \\ \hline \hline
W01 & Introduktion & expressions & kojo \\
W02 & Kodstrukturer & programs & -- \\
W03 & Funktioner, objekt & functions & blockmole \\
W04 & Datastrukturer & data & pirates \\
W05 & Sekvensalgoritmer & sequences & shuffle \\
W06 & Klasser & classes & turtlegraphics \\
W07 & Arv & traits & turtlerace-team \\
KS & KONTROLLSKRIVN. & -- & -- \\
W08 & Repetition, trösklar, luckor & reboot-init & reboot-check \\
W09 & Mönster, undantag & matching & chords-team \\
W10 & Matriser, typparametrar & matrices & maze \\
W11 & Sökning, sortering & sorting & survey \\
W12 & Scala och Java & scalajava & lthopoly-team \\
W13 & Extra: design, api, trådar, webb & threads & Projekt \\
W14 & Tentaträning & Extenta & -- \\
T & TENTAMEN & -- & -- \\
\end{tabular}

}
\end{Slide}

\ifkompendium
\noindent Kursen består av en \textbf{modul} per läsvecka med två \textbf{föreläsningar}, en \textbf{övning} och en \textbf{laboration} (undantaget W02, W13 \& W14 som saknar labb och/eller övning). 
Föreläsningarna ger en översikt av den teori som ingår i varje modul. Genom att göra övningarna bearbetar du teorin och förebereder dig inför laborationerna. När du klarat övningen och laborationen i en modul är du redo att gå vidare till nästa. Tabellen på nästa uppslag visar begrepp som ingår i varje modul. 

Kursen är uppdelad i två läsperioder. Efter första läsperioden gör du en diagnostisk \textbf{kontrollskrivning} som kontrollerar ditt kunskapsläge. Andra läsperioden avslutas med ett större \textbf{projekt} och en skriftlig \textbf{tentamen}.



\clearpage
\hyphenation{intro-duktion sekvens-algoritmer kod-strukturer data-strukturer}
{\fontsize{11}{13}\selectfont\renewcommand{\arraystretch}{1.75}
\begin{longtable}{@{}p{.05\textwidth} | >{\hspace{0.1em}\raggedright\bfseries\sffamily}p{.15\textwidth}  >{\raggedleft\arraybackslash\hspace{0.0em}\fontsize{10.5}{12}\selectfont}p{0.735\textwidth}}
W01 & Introduktion & sekvens, alternativ, repetition, abstraktion, programmeringsspråk, programmeringsparadigmer, editera-kompilera-exekvera, datorns delar, virtuell maskin, REPL, literal, värde, uttryck, identifierare, variabel, typ, tilldelning, namn, val, var, def, inbyggda grundtyper, Int, Long, Short, Double, Float, Byte, Char, String, println, typen Unit, enhetsvärdet (), stränginterpolatorn s, if, else, true, false, MinValue, MaxValue, aritmetik, slumptal, math.random, logiska uttryck, de Morgans lagar, while-sats, for-sats \\
W02 & Kodstrukturer & iterering, for-uttryck, map, foreach, Range, Array, Vector, algoritm vs implementation, pseudokod, algoritm: SWAP, algoritm: SUM, algoritm: MIN/MAX, algoritm: MININDEX, block, namnsynlighet, namnöverskuggning, lokala variabler, paket, import, filstruktur, jar, dokumentation, programlayout, JDK, main i Java vs Scala, java.lang.System.out.println \\
W03 & Funktioner & definera funktion, anropa funktion, parameter, returtyp, värdeandrop, namnanrop, default-argument, namngivna argument, applicera funktion på alla element i en samling, procedur, värdeanrop vs namnanrop, uppdelad parameterlista, skapa egen kontrollstruktur, funktionsvärde, funktionstyp, äkta funktion, stegad funktion, apply, lazy val, lokala funktioner, anonyma funktioner, lambda, aktiveringspost, anropsstacken, objektheapen, rekursion, cslib.window.SimpleWindow \\
W04 & Objekt & objekt, modul, paket, punktnotation, tillstånd, metod, medlem, funktioner är objekt, cslib.window.SimpleWindow \\
W05 & Klasser & objektorientering, klass, Point, Square, Complex, new, null, this, inkapsling, accessregler, private, private[this], kompanjonsobjekt, getters och setters, klassparameter, primär konstruktor, objektfabriksmetod, överlagring av metoder, referenslikhet vs strukturlikhet, eq vs == \\
W06 & Sekvensalgoritmer & sekvensalgoritm, algoritm: SEQ-COPY, in-place vs copy, algoritm: SEQ-REVERSE, algoritm: SEQ-REGISTER, sekvenser i Java vs Scala, for-sats i Java, java.util.Scanner, scala.collection.mutable.ArrayBuffer, StringBuilder, java.util.Random, slumptalsfrö \\
W07 & Datastrukturer & attribut (fält), medlem, metod, tupel, klass, Any, isInstanceOf, toString, case-klass, samling, scala.collection, föränderlighet vs oföränderlighet, List, Vector, Set, Map, typparameter, generisk samling som parameter, översikt samlingsmetoder, översikt strängmetoder, läsa/skriva textfiler, Source.fromFile, java.nio.file \\
KS & \multicolumn{2}{l}{KONTROLLSKRIVN.} \\
W08 & Matriser, typparametrar & matris, nästlad samling, nästlad for-sats, typparameter, generisk funktion, generisk klass, fri vs bunden typparameter, matriser i Java vs Scala, allokering av nästlade arrayer i Scala och Java \\
W09 & Arv & arv, polymorfism, trait, extends, asInstanceOf, with, inmixning, supertyp, subtyp, bastyp, override, klasshierarkin i Scala: Any AnyRef Object AnyVal Null Nothing, referenstyper vs värdetyper, klasshierarkin i scala.collection, Shape som bastyp till Rectangle och Circle, accessregler vid arv, protected, final, klass vs trait, abstract class, case-object, typer med uppräknade värden, gränssnitt, trait vs interface, programmeringsgränssnitt (api) \\
W10 & Mönster, undantag, likhet & mönstermatchning, match, Option, throw, try, catch, Try, unapply, sealed, flatten, flatMap, partiella funktioner, collect, speciella matchningar: wildcard pattern; variable binding; sequence wildcard; back-ticks, equals, hashcode, exempel: equals för klassen Complex, switch-sats i Java \\
W11 & Scala och Java & syntaxskillnader mellan Scala och Java, klasser i Scala vs Java, referensvariabler vs enkla värden i Java, referenstilldelning vs värdetilldelning i Java, alternativ konstruktor i Scala och Java, for-sats i Java, for-each-sats i Java, java.util.ArrayList, autoboxing i Java, primitiva typer i Java, wrapperklasser i Java, samlingar i Java vs Scala, scala.collection.JavaConverters, namnkonventioner för konstanter \\
W12 & Sökning, sortering, ordning & strängjämförelse, compareTo, implicit ordning, linjärsökning, binärsökning, algoritm: LINEAR-SEARCH, algoritm: BINARY-SEARCH, algoritmisk komplexitet, sortering till ny vektor, sortering på plats, insättningssortering, urvalssortering, algoritm: INSERTION-SORT, algoritm: SELECTION-SORT, Ordering[T], Ordered[T], Comparator[T], Comparable[T] \\
W13 & \multicolumn{2}{l}{Repetition, tentaträning, projekt} \\
W14 & Extra: jämlöpande exekvering & tråd, jämlöpande exekvering, icke-blockerande anrop, callback, java.lang.Thread, java.util.concurrent.atomic.AtomicInteger, scala.concurrent.Future, kort om html+css+javascript+scala.js och webbprogrammering \\
T & \multicolumn{2}{l}{TENTAMEN} \\
\end{longtable}
}
\fi

\begin{Slide}{Vad lär du dig?}
\begin{itemize}
\item Grundläggande principer för programmering:\\ Sekvens, Alternativ, Repetition, Abstraktion (SARA)\\$\implies$Inga förkunskaper i programmering krävs!
\item Konstruktion av algoritmer
\item Tänka i abstraktioner
\item Förståelse för flera olika angreppssätt: 
\begin{itemize}
\item \Emph{imperativ programmering}%: satser, föränderlighet
\item \Emph{objektorientering}%: inkapsling, återanvändning
\item \Emph{funktionsprogrammering}%: uttryck, oföränderlighet
\end{itemize}
\item Programspråken \Emph{Scala} och \Emph{Java}
\item Utvecklingsverktyg (editor, kompilator, utvecklingsmiljö)
\item Implementera, testa, felsöka
\end{itemize}
\end{Slide}

\begin{Slide}{Hur lär du dig?}
\begin{itemize}
\item Genom praktiskt \Alert{eget arbete}: \Emph{Lära genom att göra!}
\begin{itemize}
\item Övningar: applicera koncept på olika sätt
\item Laborationer: kombinera flera koncept till en helhet
\end{itemize}
\item Genom studier av kursens teori: \Emph{Skapa förståelse!}
\item Genom samarbete med dina kurskamrater: \Emph{Gå djupare!}
\end{itemize}
\end{Slide}


\begin{Slide}{Kurslitteratur}
\begin{minipage}{0.45\textwidth}\SlideFontSmall
\hskip1.33em\includegraphics[width=0.65\textwidth]{../img/compendium-front-page.png}
\begin{itemize}
\item \Emph{Kompendium} med teori, övningar \& laborationer
\item Trycks \& säljs av institutionen %på KFS \\ \url{http://www.kfsab.se/}
 efter beställning
\end{itemize}
\end{minipage}
\hskip1em\begin{minipage}{0.5\textwidth}\SlideFontSize{8}{10}
Bra, men ej nödvändig, \Emph{bredvidläsning}:\\ 
-- för \Emph{nybörjare}:
\vskip0.2mm
\includegraphics[width=0.33\textwidth]{../img/lewisbook.jpg}\hskip4mm
\includegraphics[width=0.33\textwidth]{../img/ankbok.jpg}

\noindent -- för de som \Emph{redan kodat} en del:
\vskip0.7mm
\includegraphics[width=0.45\textwidth]{../img/pinsbook.jpg}\hskip4mm
\includegraphics[width=0.47\textwidth]{../img/koffmanbook.jpg}
\end{minipage}
\end{Slide}

\ifkompendium
\noindent Kompendiet är den huvudsakliga kurslitteraturen och definierar kursinnehållet. Föreläsningar, övningar och laborationer i kompendiet är kursens primära kunskapskällor, tillsammans med de öppna resurser på nätet som kompendiet hänvisar till. Kompendiet är öppen källkod och du välkomnas varmt att bidra!

Om du gärna vill ha en eller flera mer traditionella läroböcker som bredvidläsning rekommenderas följande:
\begin{itemize}[noitemsep, leftmargin=*]
\item För de som aldrig kodat, och vill läsa om kodning från grunden:
\begin{itemize}[nolistsep]
\item ''Introduction to Programming and Problem-Solving Using Scala, Second Edition'', Mark C. Lewis, Lisa Lacher.  {\href{https://www.crcpress.com/Introduction-to-Programming-and-Problem-Solving-Using-Scala-Second-Edition/Lewis-Lacher/p/book/9781498730952}{www.crcpress.com/Introduction-to-Programming-and-Problem-Solving-Using-Scala-Second-Edition/Lewis-Lacher/p/book/9781498730952}}
\item ''Objektorienterad programmering och Java'', Per Holm, Tredje upplagan (2007). \href{https://www.studentlitteratur.se/#6735}{www.studentlitteratur.se/\#6735}
\end{itemize}
\item För de som redan kodat en hel del i ett objektorienterat språk:
\begin{itemize}[nolistsep, noitemsep]
\item ''Programming in Scala, Third Edition -- A comprehensive step-by-step guide'', Martin Odersky, Lex Spoon, and Bill Venners. \\ \href{http://www.artima.com/shop/programming_in_scala_3ed}{www.artima.com/shop/programming\_in\_scala\_3ed} 
\item ''Data Structures: Abstraction and Design Using Java, 3rd Edition'', Elliot B. Koffman, Paul A. T. Wolfgang. \\
\href{http://eu.wiley.com/WileyCDA/WileyTitle/productCd-1119186528.html}{http://eu.wiley.com/WileyCDA/WileyTitle/productCd-1119186528.html}
\end{itemize}
\end{itemize}
Dessa läroböcker följer inte direkt kursens upplägg vad gäller omfång och progression och du får själv göra den nyttiga hemläxan att koppla  deras innehåll till det vi går igenom i kursens olika moduler.

\else
\begin{Slide}{Beställning av kompendium och snabbreferens}
\begin{itemize}
\item \Emph{Kompendiet} finns i pdf för fri nedladdning, men det \Alert{rekommenderas starkt} att du köper den på papper.
\item Det är mycket lättare att ha övningar och labbar på papper bredvid skärmen, när du ska tänka, koda och plugga!
\item \Emph{Snabbreferensen} finns också i pdf men du behöver ha en tryckt version eftersom det är enda tillåtna hjälpmedlet på skriftliga kontrollskrivningen och tentamen.
\item Kompendium och bok trycks här i E-huset och säljs av institutionen till självkostnadspris.
\item Pris för kompendium + snabbreferens: ??? kr
\item Skriv upp dig på listan -- tryckning sker efter beställning.
\item Du betalar med jämna pengar på cs expedition, våning 2
\end{itemize}
\end{Slide}
\fi

\ifkompendium\else
\begin{Slide}{Personal}\SlideFontSmall
\begin{description}
\item [\bfseries Kursansvarig:] ~\\Björn Regnell, bjorn.regnell@cs.lth.se
\item [\bfseries Kurssekreterare:]  ~\\Lena Ohlsson \\Exp.tid 09.30 -- 11.30 samt 12.45 -- 13.30
\item [\bfseries Handledare:] ~\\
\Emph{Doktorander}: \\ 
Tekn. Lic. Maj Stenmark, Gustav Cedersjö\\
\Emph{Teknologer}: \\
Anders Buhl, 
Anna Palmqvist Sjövall, 
Anton Andersson,
Cecilia Lindskog, 
Emil Wihlander, 
Erik Bjäreholt, 
Erik Grampp, 
Filip Stjernström, 
Fredrik Danebjer, 
Henrik Olsson, 
Jakob Hök, 
Jonas Danebjer, 
Måns Magnusson, 
Oscar Sigurdsson, 
Oskar Berg, 
Oskar Widmark, 
Sebastian Hegardt, 
Stefan Jonsson, 
Tom Postema, 
Valthor Halldorsson
\end{description}
\end{Slide}
\fi

\begin{Slide}{Föreläsningsanteckningar}
\begin{itemize}
\item Föreläsningsanteckningar utvecklas under kursens gång
\item Några av bilderna finns i kompendiet
\item Alla bilder läggs ut här: \\
\href{https://github.com/lunduniversity/introprog/tree/master/slides}{github.com/lunduniversity/introprog/tree/master/slides} \\
och uppdateras kontinuerligt allt eftersom de utvecklas
\item Förslag på förbättringar välkomna!
\end{itemize}
\end{Slide}

\begin{Slide}{Kursmoment --- varför?}\SlideOnly{\footnotesize}
\begin{itemize}
\item \Emph{Föreläsningar}: skapa översikt, ge struktur, förklara teori, svara på frågor, motivera varför
\item \Emph{Övningar}: \Alert{förbereda} laborationerna, bearbeta teorins olika delar med avgränsade deluppgifter, \Emph{grundövningar} för alla, \Emph{extraövningar} om du vill/behöver öva mer, \Emph{fördjupningsövningar} om du vill gå djupare 
\item \Emph{Laborationer}: \Alert{obligatoriska}, sätta samman teorins delar i ett större program; lösningar redovisas för handledare; gk på alla för att få tenta, 
\item \Emph{Resurstider}: få hjälp med övningar och laborationsförberedelser av handledare, fråga vad du vill
\item \Emph{Samarbetsgrupper}: grupplärande genom samarbete, hjälpa varandra 
\item \Emph{Kontrollskrivning}: \Alert{obligatorisk}, diagnostisk, kamraträttad; kan ge samarbetsbonuspoäng till tentan
\item \Emph{Individuell projektuppgift}: \Alert{obligatorisk}, du visar att du kan skapa ett större program självständigt; redovisas för handledare
\item \Emph{Tentamen}: \Alert{obligatorisk}, skriftlig, enda hjälpmedel:   \href{https://github.com/lunduniversity/introprog/blob/master/quickref/quickref.pdf}{snabbreferensen}
\end{itemize}
\end{Slide}

\ifkompendium\else
\begin{Slide}{Detta är bara början... }
Exempel på efterföljande kurser som bygger vidare på denna:
\begin{itemize}
\item Årskurs 1
\begin{itemize}
\item Programmeringsteknik -- fördjupningskurs
\item Utvärdering av programvarusystem
\item Diskreta strukturer
\end{itemize}
\item Årskurs 2
\begin{itemize}
\item Objektorienterad modellering och design
\item Programvaruutveckling i grupp
\item Algoritmer, datastrukturer och komplexitet
\item Funktionsprogrammering
\end{itemize}
\end{itemize}
\end{Slide}


\begin{Slide}{Registrering}
\begin{itemize}
\item Fyll i listan som skickas runt.
\item Kryssa i kolumnen \Emph{Ska gå} om du ska gå kursen\footnote{\scriptsize D1:a som redan gått motsvarande högskolekurs? Uppsök studievägledningen}\footnote{\scriptsize D2:a eller äldre som vill bli omregistrerad? Prata med kursansvarig på rasten}
\item Kryssa i kolumnen \Emph{Kursombud} om du kan tänka dig att vara kursombud under kursens gång
\begin{itemize}
\item Alla LTH-kurser ska utvärderas under kursens gång och efter kursens slut.
\item Till det behövs kursombud -- ungefär 2 D-are och 2 W-are.
\item Ni kommer att bli kontaktade av studierådet. \\SRD ordf: Amelia Andersson
\end{itemize}
\end{itemize}
\end{Slide}

%%%
\begin{Slide}{Förkunskaper}
\begin{itemize}
\item Förkunskaper $\neq$ Förmåga
\item Varken kompetens eller personliga egenskaper är statiska 
\item ''Programmeringskompetens'' är inte \textit{en} enda enkel förmåga utan en komplex sammansättning av flera olika förmågor som utvecklas genom hela livet
\item Ett innovativt utvecklar\Alert{team} behöver många olika kompetenser för att vara framgångsrikt
\end{itemize}
\end{Slide}

%%%
\begin{Slide}{Förkunskapsenkät}
\begin{itemize}
\item Om du inte redan gjort det: fyll i denna enkät \Alert{snarast}:\\
\url{http://cs.lth.se/pgk/presurvey} \\
\item Dina svar behandlas internt och all statistik anonymiseras.
\item Enkäten ligger till grund för randomiserad gruppindelning i samarbetsgrupper, så att det blir en spridning av förkunskaper inom gruppen.
\item Gruppindelnig publiceras här: \\ \url{http://cs.lth.se/pgk/grupper/}
\end{itemize}
\end{Slide}

\begin{Slide}{Samarbetgrupper}\footnotesize
\begin{itemize}
\item Ni delas in i \Emph{samarbetsgrupper} om ca 5 personer baserat på förkunskapsenkäten, så att olika förkunskapsnivåer sammanförs
\item Några av laborationerna är mer omfattande \Emph{grupplabbar} och kommer att göras i samarbetsgrupperna \\ \vspace{1em}
\item Kontrollskrivningen i halvtid kan ge \Emph{samarbetsbonus} (max 5p) som adderas till ordinarie tentans poäng (max 100p) med medelvärdet av gruppmedlemmarnas individuella kontrollskrivningspoäng 
\scriptsize \parbox{7cm}{Bonus $b$ för varje person i en grupp med $n$ medlemmar med $p_i$ poäng vardera på kontrollskrivningen:} 
 \hspace{5mm} $\displaystyle b = \sum\limits_{i=1}^n \frac{p_i}{n}$
\end{itemize}
\end{Slide}

\fi

%%%
\begin{Slide}{Varför studera i samarbetsgrupper?}

Huvudsyfte: \Emph{Bra lärande!}

\begin{itemize}
\item Pedagogisk forskning stödjer tesen att lärandet blir mer djupinriktat om det sker i utbyte med andra
\item Ett studiesammanhang med höga ambitioner och respektfull gemenskap gör att vi \Emph{når mycket längre}
\item Varför ska du som redan kan mycket aktivt dela med dig av dina kunskaper?
\begin{itemize}
\item Förstå bättre själv genom att förklara för andra
\item Träna din pedagogiska förmåga
\item Förbered dig för ditt kommande yrkesliv som mjukvaruutvecklare 
\end{itemize}
\end{itemize}
\end{Slide}

%%%

\ifkompendium\else
\begin{Slide}{Samarbetskontrakt}
Gör ett skriftligt \href{https://github.com/bjornregnell/lth-eda016-2015/blob/master/assignments/collaboration-contract.tex}{\bf samarbetskontrakt} med dessa och ev. andra punkter som ni också tycker bör ingå:
\begin{enumerate}
\item Återkommande mötestider per vecka
\item Kom i tid till gruppmöten
\item Var väl förberedd genom självstudier inför gruppmöten
\item Hjälp varandra att förstå, men ta inte över och lös allt
\item Ha ett respektfullt bemötande även om ni har olika åsikter
\item Inkludera alla i gemenskapen
\end{enumerate}

Diskutera hur ni ska uppfylla dessa innan alla skriver på. \\ Ta med samarbetskontraktet och visa för handledare på labb 1.

\vskip1em

\Alert{Om arbetet i samarbetsgruppen inte fungerar ska ni mejla kursansvarig och boka mötestid!}
\end{Slide}

\begin{Slide}{Bestraffa inte frågor!}
\begin{itemize}
\item Det finns bättre och sämre frågor vad gäller hur mycket man kan lära sig av svaret, men \Emph{all undran är en chans} att i dialog utbyta erfarenheter och lärande
\item Den som frågar \Emph{vill veta} och berättar genom frågan något om nuvarande kunskapsläge
\item Den som svarar får chansen att \Emph{reflektera} över vad som kan vara svårt och olika vägar till djupare förståelse
\item I en hälsosam lärandemiljö är det \Emph{helt tryggt} att visa att man ännu inte förstår, att man gjort ''fel'', att man har mer att lära, etc. 
\item Det är viktigt att våga försöka även om det blir ''fel'':\\ \Emph{det är ju då man lär sig!}
\end{itemize}
\end{Slide}

%%%
\begin{Slide}{Plagiatregler}
Läs dessa regler noga och diskutera i samarbetsgrupperna:
\begin{itemize}
\footnotesize
\item \url{http://cs.lth.se/utbildning/samarbete-eller-fusk/}
\item \url{http://cs.lth.se/utbildning/foereskrifter-angaaende-obligatoriska-moment/}
\end{itemize}
Ni ska lära er genom \Emph{eget arbete} och genom  \Emph{bra samarbete}. Samarbete gör att man lär sig bättre, men man lär sig inte av att bara kopiera andras lösningar. Plagiering är förbjuden och kan medföra disciplinärende och avstängning.
\end{Slide}

\fi %%%%%%%%%%%%%%%%%%%%%%%%%%%%%%%%

%%%
\begin{Slide}{En typisk kursvecka}
\begin{enumerate}
\item Gå på \Emph{föreläsningar} på \Alert{måndag--tisdag}
\item Jobba med \Emph{individuellt} med teori, övningar, labbförberedelser på  \Alert{måndag--torsdag}
\item Kom till \Emph{resurstiderna} och få hjälp och tips av handledare och kurskamrater på \Alert{onsdag--torsdag}
\item Genomför den obligatoriska \Emph{laborationen} på \Alert{fredag}
\item Träffas i \Emph{samarbetsgruppen} och hjälp varandra att förstå mer och fördjupa lärandet, förslagsvis på återkommande tider varje vecka då alla i gruppen kan
\end{enumerate}
Se detaljerna och undantagen i schemat: \href{http://cs.lth.se/pgk/schema}{cs.lth.se/pgk/schema}
\end{Slide}

\ifkompendium\else  %%%%%%%%%%%%%%%%%%%%%%%%%
%%%
\begin{Slide}{Laborationer}\footnotesize
\begin{itemize}
\item \Alert{Programmering lär man sig bäst genom att programmera...}
\item Labbarna är \Emph{individuella} (utom 2) och \Emph{obligatoriska}
\item Gör övningarna och labbförberedelserna noga \textit{innan} själva labben -- detta är ofta helt nödvändigt för att du ska hinna klart. Dina labbförberedelserna kontrolleras av handledare under labben.
\item Är du sjuk? Anmäl det \Alert{före} labben till \url{bjorn.regnell@cs.lth.se}, \\ få hjälp på resurstid och redovisa på resurstid (eller labbtid, när handledaren har tid över)
\item Hinner du inte med hela? Se till att handledaren noterar din närvaro, och fortsätt på resurstid och ev. uppsamlingstider.
\item Läs noga anvisningarna i kompendiet
\item Laborationstiderna är gruppindelade enligt \href{http://cs.lth.se/eda016/schema/}{schemat}. Du ska gå till den tid och den sal som motsvarar din \href{http://cs.lth.se/eda016/grupper/}{grupp}.
\end{itemize}
\end{Slide}

%%%
\begin{Slide}{Resurstider}
\begin{itemize}
\item På resurstiderna får du hjälp med övningar och laborationsförberedelser
\item Kom till minst en resurstid per vecka (se \href{http://cs.lth.se/eda016/schema/}{schema})
\item Handledare gör ibland \Emph{genomgångar} för alla under resurstiderna. Tipsa om handledare om vad du finner svårt.
\item Resurstiderna är inte gruppindelade i schemat. Du får i mån av plats gå på flera resurstider per vecka. Om det blir fullt i ett rum prioriteras dessa grupper för att minimera schemakrockar: 
\end{itemize}
\begin{table}[]
\centering\scriptsize
\begin{tabular}{lllll}
Tid Lp1 & Sal & Grupper med prio \\
\hline
Ons 10-12 v1-7 & Hacke  &   09 \& 10 \\
Ons 13-15 v1-7 & Hacke  &   07 \& 08  \\
Ons 15-17 v1-7 & Panter  & 05 \& 06   \\
Ons 15-17 v1-7 & Val       &  03 \& 04   \\
Tor 13-15 v1-7 & Mars     & 01 \& 02  \\
Tor 15-17 v1-7 & Mars     & 11 \& 12 \\ 
\end{tabular}
\end{table}
\end{Slide}

\fi

%!TEX root = ../compendium.tex

\ChapterUnnum{Anvisningar}

\SectionUnnum{Samarbetsgrupper}
\subsection*{Samarbetskontrakt}
\SectionUnnum{Föreläsningar}
\SectionUnnum{Övningar}
\SectionUnnum{Laborationer}
\SectionUnnum{Resurstider}
\SectionUnnum{Kontrollskrivning}
\SectionUnnum{Tentamen}

%!TEX root = ../compendium.tex


\ChapterUnnum{Hur bidra till kursmaterialet?}


%\renewcommand{\SlideHeading}[1]{\subsection{#1}}  %numbering sections in compendium slides

\part{Moduler}

  %remove space before subsection
  %https://stackoverflow.com/questions/3191640/heading-subsection-at-start-of-framed-environment-in-latex-without-leading-padd


%!TEX encoding = UTF-8 Unicode
%!TEX root = ../compendium1.tex

\renewcommand{\vecka}{1}

%!TEX encoding = UTF-8 Unicode
\chapter{Introduktion}\label{chapter:W01}
Begrepp som ingår i denna veckas studier:
\begin{multicols}{2}\begin{itemize}[noitemsep,label={$\square$},leftmargin=*]
\item sekvens
\item alternativ
\item repetition
\item abstraktion
\item editera
\item kompilera
\item exekvera
\item datorns delar
\item virtuell maskin
\item litteral
\item värde
\item uttryck
\item identifierare
\item variabel
\item typ
\item tilldelning
\item namn
\item val
\item var
\item def
\item definera och anropa funktion
\item funktionshuvud
\item funktionskropp
\item procedur
\item inbyggda grundtyper
\item Int
\item Long
\item Short
\item Double
\item Float
\item Byte
\item Char
\item String
\item println
\item typen Unit
\item enhetsvärdet ()
\item stränginterpolatorn s
\item if
\item else
\item true
\item false
\item MinValue
\item MaxValue
\item aritmetik
\item slumptal
\item math.random
\item logiska uttryck
\item de Morgans lagar
\item while-sats
\item for-sats\end{itemize}\end{multicols}

\clearpage\section{Teori}
%!TEX encoding = UTF-8 Unicode
%!TEX root = ../lect-w01.tex

%%%%%%%%%%%%%%%%%%%%%%%%%%%%%%%%%%%%%%

\Subsection{Att lära denna läsvecka \texttt{w01}}

\ifkompendium\else  %%%%%%%%%%%%%%%%%%%%%%%%%%%%%%%%%%%%%%%%%%%%%%%%%
\begin{SlideExtra}{Att lära denna läsvecka \texttt{w01}}
%!TEX encoding = UTF-8 Unicode

    Modul \Emph{Introduktion}: Övn \Alert{\texttt{expressions}} $\rightarrow$ Labb \Alert{\texttt{kojo, linux}}
    \begin{multicols}{3}\SlideFontTiny
    $\square$ sekvens \\
$\square$ alternativ \\
$\square$ repetition \\
$\square$ abstraktion \\
$\square$ editera \\
$\square$ kompilera \\
$\square$ exekvera \\
$\square$ datorns delar \\
$\square$ virtuell maskin \\
$\square$ litteral \\
$\square$ värde \\
$\square$ uttryck \\
$\square$ identifierare \\
$\square$ variabel \\
$\square$ typ \\
$\square$ tilldelning \\
$\square$ namn \\
$\square$ val \\
$\square$ var \\
$\square$ def \\
$\square$ definiera och anropa funktion \\
$\square$ funktionshuvud \\
$\square$ funktionskropp \\
$\square$ procedur \\
$\square$ inbyggda grundtyper \\
$\square$ println \\
$\square$ typen Unit \\
$\square$ enhetsvärdet () \\
$\square$ stränginterpolatorn s \\
$\square$ aritmetik \\
$\square$ slumptal \\
$\square$ logiska uttryck \\
$\square$ de Morgans lagar \\
$\square$ if \\
$\square$ true \\
$\square$ false \\
$\square$ while \\
$\square$ for \\
$\square$ DoD: Operativsystem \\
    \end{multicols}
    
\end{SlideExtra}
\fi

\Subsection{Om programmering}

\ifkompendium\else  %%%%%%%%%%%%%%%%%%%%%%%%%%%%%%%%%%%%%%%%%%%%%%%%%

\begin{SlideExtra}{Att skapa koden som styr världen}
\begin{multicols}{2}\footnotesize
I stort sett \Alert{alla} delar av samhället är beroende av programkod:
\begin{itemize}\scriptsize
\item kommunikation
\item transport
\item byggsektorn
\item statsförvaltning
\item finanssektorn
\item media \& underhållning
\item sjukvård
\item övervakning
\item integritet
\item upphovsrätt
\item miljö \& energi
\item sociala relationer
\item utbildning
\item ...
\end{itemize}
\columnbreak %---------
Hur blir ditt framtida yrkesliv som systemutvecklare?
\begin{itemize}
\item  Det är sedan lång tid en \Alert{skriande brist} på utvecklare och det blir bara värre... \\
\href{https://cio.idg.se/2.1782/1.710012/kompetenslarm-jobb-om-fem-ar?queryText=kompetensbrist}{CIO 2018-11-09} %\\
%\href{http://computersweden.idg.se/2.2683/1.663879/oppen-kallkod-brist-kompetens}{CS 2016-08-23} 

\item Störst brist är det på \Emph{kvinnliga} utvecklare: \\
\href{https://www.svt.se/nyheter/inrikes/stor-brist-pa-programmerare-kvinnor-lockas-till-yrket}{SVT 2016-12-03}

\item Global kompetensmarknad \\
  \href{https://cio.idg.se/2.1782/1.648294/hitta-it-kompetens/sida/2/global-rekrytering-aktivt-hr-arbete}{CIO 2016-02-01}\\
  \href{http://computersweden.idg.se/2.2683/1.630901/det-finns-programmerare-och-sa-finns-det-programmerare}{CS 2015-06-14} \\
  \href{http://computersweden.idg.se/2.2683/1.662186/25-miljoner-utvecklare?queryText=miljoner\%20utvecklare}{CS 2016-07-14 }
\end{itemize}
\end{multicols}

\end{SlideExtra}

\begin{SlideExtra}{Utveckling av mjukvara i praktiken}
\begin{itemize}
\item \Emph{Inte bara kodning:} kravbeslut, releaseplanering, design, test, versionshantering, kontinuerlig integration, driftsättning, återkoppling från dagens användare, ekonomi \& investering, gissa om morgondagens användare, ...
\item \Emph{Teamwork:} Inte ensamma hjältar utan autonoma team i decentraliserade organisationer med innovationsuppdrag
\item \Emph{Snabbhet:} Att koda innebär att hela tiden uppfinna nya ''byggstenar'' som ökar organisationens förmåga att snabbt skapa värde med hjälp av mjukvara. \Alert{Öppen källkod}. Skapa kraftfulla API:er.
\item \Emph{Livslångt lärande:} Lär nytt och dela med dig hela tiden. Exempel på pedagogisk utmaning: hjälp andra förstå och använda ditt API $\implies$ \Alert{Samarbetskultur}
\end{itemize}
\end{SlideExtra}


\fi %%%%%%%%%%%%%%%%%%%%%%%%%%%%%%%%%%%%%%%%%%%%%%%%%%%%


% \ifkompendium\else
% \SlideImg{Programming unplugged: Två frivilliga?}{../img/unplugged}
% \SlideImg{Editera och exekvera ett program}{../img/kojo}
% \fi

%%%

\ifkompendium\else
\SlideImg{Vad är en dator?}{../img/eniac}
\fi

\begin{Slide}{Hur fungerar en dator?}
\begin{tikzpicture}[node distance=2.0cm]
\node (input)  [startstop]               {Indata-enhet};
\node (cpu)    [process, below of=input] {CPU};
\node (output) [startstop,below of=cpu]  {Utdata-enhet};

\node (mem) [right of=cpu, xshift=0.4\textwidth, draw = black, thick] {
\begin{minipage}{0.5\textwidth}\centering
\textbf{Minne} med minnesceller
\vspace{1em}

\begin{tabular}{|l | l|}
address & innehåll \\ \hline
0   & 42 \\ \hline
1   & 13 \\ \hline
2   & 18 \\ \hline
3   & 21 \\ \hline
4   & 55 \\ \hline
5   & 64 \\ \hline
6   & 48 \\ \hline
... & ...
\end{tabular}
\end{minipage}
};

\draw [arrow] (input) -- (cpu);
\draw [arrow] (cpu) -- (output);
\draw [arrow] (cpu) -- (mem);
\draw [arrow] (mem) -- (cpu);

\end{tikzpicture}

{\hfill
\begin{minipage}{0.65\textwidth}\vspace{1em}
Minnet innehåller endast \Alert{heltal} som \newline representerar \Emph{data} \Alert{och} \Emph{instruktioner}.
\end{minipage}
}
\end{Slide}

\begin{Slide}{Vad är programmering?}
\begin{itemize}
\item Programmering innebär att ge instruktioner till en maskin.
\item Ett \Emph{programmeringsspråk} används av människor för att skriva \Emph{källkod} som kan översättas av en \Emph{kompilator} till \Emph{maskinspråk} som i sin tur \Emph{exekveras} av en dator.
\end{itemize}


\begin{minipage}{.8\textwidth}
\begin{itemize}
\item Ada Lovelace publicerade det första programmet redan på 1800-talet ämnat för en kugghjulsdator.
\end{itemize}
\end{minipage}%
\begin{minipage}{.2\textwidth}
\centering\includegraphics[width=0.6\columnwidth]{../img/ada}
\end{minipage}%
\begin{itemize}
\item \href{https://sv.wikipedia.org/wiki/Programmering}{sv.wikipedia.org/wiki/Programmering}
\item \href{https://en.wikipedia.org/wiki/Computer\_programming}{en.wikipedia.org/wiki/Computer\_programming}
\item Ha picknick i \href{http://kartor.lund.se/wiki/lundanamn/index.php/Ada_Lovelace-parken}{Ada Lovelace-parken} på Brunnshög!
\end{itemize}
\end{Slide}


\begin{Slide}{Vad är en kompilator?}
\begin{minipage}{.35\textwidth}
\includegraphics[width=0.6\textwidth]{../img/grace}
\end{minipage}%
\begin{minipage}{.67\textwidth}
%https://www.sharelatex.com/blog/2013/08/29/tikz-series-pt3.html
\begin{tikzpicture}[node distance=1.4cm,scale=0.8]
  \node (input) [startstop] {Källkod};
  \node(inptext) [right of=input, text width=2cm, scale=0.8,xshift=1.5cm]{För\\människor};
  \node (compile) [process, below of=input] {Kompilator};
  \node (output) [startstop, below of=compile] {Maskinkod};
  \node(outtext) [right of=output, text width=2cm, scale=0.8,xshift=1.5cm]{För\\maskiner};
  \draw [arrow] (input) -- (compile);
  \draw [arrow] (compile) -- (output);
  \end{tikzpicture}%

\vspace{1em}\noindent Grace Hopper uppfann kompilatorn 1952. \\ \href{https://en.wikipedia.org/wiki/Grace\_Hopper}{\SlideFontTiny en.wikipedia.org/wiki/Grace\_Hopper}
\end{minipage}%
\end{Slide}


\begin{Slide}{Virtuell maskin (VM) == abstrakt hårdvara}
\begin{multicols}{2}
\begin{itemize}
\item En VM är en ''dator'' implementerad i mjukvara som kan tolka en generell ''maskinkod'' som \Emph{översätts under körning} till den \Alert{verkliga} maskinens specifika maskinkod.

\item Med en VM blir källkoden \Emph{plattformsoberoende} och fungerar på många olika maskiner.

\item Exempel JVM: \\ \Emph{Java Virtual Machine}


\end{itemize}

\columnbreak %---------

%https://www.sharelatex.com/blog/2013/08/29/tikz-series-pt3.html
\begin{tikzpicture}[node distance=1.4cm]
\node (input) [startstop] {Källkod};
\node (compile) [process, below of=input] {Kompilator};
\node (output) [startstop, below of=compile] {Generell ''maskinkod''};
\node (interp) [process, below of=output] {VM interpreterar};
\node (output2) [startstop, below of=interp] {Specifik maskinkod};
\draw [arrow] (input) -- (compile);
\draw [arrow] (compile) -- (output);
\draw [arrow] (output) -- (interp);
\draw [arrow] (interp) -- (output2);
\end{tikzpicture}
\end{multicols}
\end{Slide}

\begin{Slide}{Vad består ett program av?}
\begin{itemize}
\item Text som följer entydiga språkregler (grammatik):
\begin{itemize}
\item \Emph{Syntax}: textens konkreta utseende
\item \Emph{Semantik}: textens betydelse (vad maskinen gör/beräknar)
\end{itemize}
\item \Emph{Nyckelord}: ord med speciell betydelse, t.ex. \code{if}, \code{while}
\item \Emph{Deklarationer}: definitioner av nya ord: \code{def gurka = 42}
\item \Emph{Satser} är instruktioner som \Alert{gör} något: \code{print("hej")}
\item \Emph{Uttryck} är instruktioner som beräknar ett \Alert{resultat}: \code{1 + 1}
\item \Emph{Data} är information som behandlas: t.ex. heltalet \code{42}
\item Instruktioner ordnas i kodstrukturer: \Alert{SARA}
\begin{itemize}
  \item \Emph{Sekvens}: ordningen spelar roll för vad som händer
  \item \Emph{Alternativ}: olika resultat beroende på uttrycks värde
  \item \Emph{Repetition}: instruktioner upprepas många gånger
  \item \Emph{Abstraktion}: nya byggblock skapas för att återanvändas
\end{itemize}
\end{itemize}
\end{Slide}

\begin{Slide}{Exempel på programmeringsspråk}
Det finns massor med olika språk och det kommer ständigt nya.
\vspace{1em}
\begin{multicols}{2}
Exempel:
\begin{itemize}
\item Java
\item C
\item C++
\item C\#
\item Python
\item JavaScript
\item Scala
\end{itemize}

\columnbreak %---------

Några topplistor:
\begin{itemize}
\item \href{https://redmonk.com/sogrady/2022/03/28/language-rankings-1-22/}{Redmonk}   
%\item \href{https://redmonk.com/sogrady/2020/07/27/language-rankings-6-20/}{Redmonk language rankings}   
%\item \href{https://redmonk.com/sogrady/2019/07/18/language-rankings-6-19/}{Redmonk language rankings}   
%\item \href{https://redmonk.com/sogrady/2018/08/10/language-rankings-6-18/}{Redmonk language rankings}   
\item \href{http://pypl.github.io/PYPL.html}{PYPL}
\item \href{http://www.tiobe.com/index.php/content/paperinfo/tpci/index.html}{TIOBE}
\end{itemize}
% \vspace{1em}
% \includegraphics[width=0.8\columnwidth]{../img/pypl}\\\SlideFontSmall{[PYPL (2016)]}
\end{multicols}

\end{Slide}

\ifkompendium\else
\begin{SlideExtra}{Redmonk Language Rankings: Github, Stackoverflow}
\includegraphics[width=0.80\columnwidth]{../img/redmonk-Q122}
\end{SlideExtra}
% \begin{SlideExtra}{Några språks utveckling över tid enl. PYPL}
% \TODO\includegraphics[width=0.95\columnwidth]{../img/pypl-06-20}
% \end{SlideExtra}

\fi

\begin{Slide}{Olika programmeringsparadigm}
\begin{itemize}
\item Det finns många olika \href{https://en.wikipedia.org/wiki/Programming_paradigm}{programmeringsparadigm} (sätt att programmera på), till exempel:
\begin{itemize}\SlideFontSmall
\item \Emph{imperativ programmering:} programmet är uppbyggt av satser som påverkar systemets tillstånd
\item \Emph{objektorienterad programmering:} en sorts imperativ programmering där programmet består av objekt som kapslar in data och erbjuder operationer som bearbetar dessa data
\item \Emph{funktionsprogrammering:} programmet är uppbyggt av samverkande funktioner som undviker förändringar av data
\item \Emph{deklarativ programmering, logikprogrammering:} programmet är uppbyggt av logiska uttryck som beskriver olika fakta eller villkor och exekveringen utgörs av en bevisprocedur som söker efter värden som uppfyller fakta och villkor
\end{itemize}
\end{itemize}
Denna kurs behandlar de tre första.
\end{Slide}


\begin{Slide}{Hello world}

Kör rad för rad i Scala REPL (Read-Evaluate-Print-Loop):
\begin{REPLnonum}
scala> println("Hello World!")
Hello World!
\end{REPLnonum}

\pause
\noindent Scala-applikation med \code{@main} framför valfri funktion anger var programmet ska starta:
\begin{Code}
@main def hi = println("Hello world!")
\end{Code}
\noindent Spara texten ovan i filen \code{hello.scala}.\\
Kompilera:
\begin{REPLnonum}
scalac hello.scala
\end{REPLnonum}
\noindent Kör:
\begin{REPLnonum}
scala hi
\end{REPLnonum}
\end{Slide}

\begin{Slide}{Utvecklingscykeln}
editera; kompilera; hitta fel och förbättringar; editera; kompilera; hitta fel och förbättringar; editera; kompilera; hitta fel och förbättringar; editera; kompilera; hitta fel och förbättringar; editera; kompilera; hitta fel och förbättringar; editera; kompilera; hitta fel och förbättringar; ...

\begin{Code}
upprepa(1000){
  editera
  kompilera
  testa
}
\end{Code}
\end{Slide}

\begin{Slide}{Utvecklingsverktyg}
\begin{itemize}\SlideFontSmall
\item Din verktygskunskap är mycket viktig för din produktivitet.
\item Lär dig kortkommandon för vanliga handgrepp.
\item Verktyg vi använder i kursen:
\begin{itemize}\SlideFontTiny
\item Scala \Emph{REPL}: från övn 1
\item \Emph{Texteditor} för kod, t.ex \Emph{VS code}: från övn 2
\item Kompilera en fil i taget med \Emph{\code{scalac}}: från övn 2
\item Samkompilera många filer med \Emph{\texttt{scala-cli}}
\item Integrerad utvecklingsmiljö (IDE)
\begin{itemize}\SlideFontTiny
\item \Emph{VS code} med extension ''Scala Metals'', lab 1 med Kojo
\item \Emph{IntelliJ IDEA}: valfri, lämpligt från läsperiod 2
\end{itemize}\SlideFontTiny
\item \Emph{jar} för att packa ihop och distribuera klassfiler
\item \Emph{scaladoc} för dokumentation av kodbibliotek
\end{itemize}
\item Andra verktyg som är bra att lära sig:
\begin{itemize}\SlideFontTiny
\item Git för versionshantering
\item GitHub för kodlagring -- men \Alert{inte} av lösningar till labbar!
\item \code{sbt} -- Scala Build Tool för professionella byggen av öppen källkod (lp2)
\end{itemize}
\end{itemize}
\end{Slide}





\Subsection{De enklaste beståndsdelarna: litteraler, uttryck, variabler}


\begin{Slide}{Litteraler}
\begin{itemize}
\item En literal representerar ett fixt \Emph{värde} i koden och används för att skapa \Alert{data} som programmet ska bearbeta.
\item Exempel: \\
\begin{tabular}{l l}
\code|42| & heltalslitteral\\
\code|42.0| & decimaltalslitteral\\
\code|'!'| & teckenlitteral, omgärdas med 'enkelfnuttar' \\
\code|"hej"| & stränglitteral, omgärdas med ''dubbelfnuttar'' \\
\code|true| & litteral för sanningsvärdet ''sant''\\
\end{tabular}
\item Literaler har en \Emph{typ} som avgör vad man kan göra med dem.
\end{itemize}
\end{Slide}

\begin{Slide}{Exempel på inbyggda datatyper i Scala}\SlideFontSmall
\begin{itemize}
\item Alla värden, uttryck och variabler har en \href{https://sv.wikipedia.org/wiki/Datatyp}{\Emph{datatyp}}, t.ex.:
\begin{itemize}\footnotesize
\item \code{Int} för heltal
\item \code{Long} för \textit{extra} stora heltal (tar mer minne)
\item \code{Double} för decimaltal, så kallade flyttal med flytande decimalpunkt
\item \code{String} för strängar
\end{itemize}

\item Kompilatorn håller reda på att uttryck kombineras på ett \Emph{typsäkert} sätt. Annars blir det \Alert{kompileringsfel}.

\item Scala och Java är s.k. \href{https://sv.wikipedia.org/wiki/Typsystem}{\Emph{statiskt typade}} språk, vilket innebär att kontroll av typinformation sker vid kompilering \Eng{compile time}\footnote{Andra språk, t.ex. Python och Javascript är \Emph{dynamiskt typade} och där skjuts typkontrollen upp till körningsdags \Eng{run time} \\ Vilka är för- och nackdelarna med statisk vs. dynamisk typning?}.

\item Scala-kompilatorn gör \href{https://en.wikipedia.org/wiki/Type_inference}{\Emph{typhärledning}}: man \Alert{slipper skriva typerna} om kompilatorn kan lista ut dem med hjälp av typerna hos deluttrycken.

\end{itemize}
\end{Slide}


\begin{Slide}{Grundtyper i Scala}\SlideFontSmall
Dessa \Emph{grundtyper} \Eng{basic types} finns inbyggda i Scala:

\begin{table}[H]
\renewcommand{\arraystretch}{1.4}
\begin{tabular}{p{0.24\textwidth}|p{0.21\textwidth}|l}
\textit{Svenskt namn} & \textit{Engelskt namn} & \Emph{Grundtyper} \\ \hline
heltalstyp & integral type & \texttt{Byte}, \texttt{Short}, \texttt{Int}, \texttt{Long}, \texttt{Char} \\
flyttalstyp  &  floating point \newline number types & \texttt{Float}, \texttt{Double} \\
numeriska typer & numeric types & heltalstyper och flyttalstyper \\
strängtyp \newline (teckensekvens) & string type & \texttt{String}  \\
sanningsvärdestyp  \newline (boolesk typ)& truth value type & \texttt{Boolean} \\
\end{tabular}
\end{table}

\end{Slide}

\begin{Slide}{Grundtypernas omfång}\SlideFontSmall
\begin{table}[H]
\renewcommand{\arraystretch}{1.4}
\begin{tabular}{l|l|l}
\Alert{Grundtyp}& Antal bitar &  Omfång: minsta \& största värde\\\hline
\texttt{Byte}   &  8  & $-2^7$ ... $2^7-1$    \\
\texttt{Short}  &  16 & $-2^{15}$ ... $2^{15}-1$ \\
\texttt{Char}   &  16 & $0$ ... $2^{16}-1$ \\
\texttt{Int}    &  32 & $-2^{31}$ ... $2^{31}-1$ \\
\texttt{Long}   &  64 & $-2^{63}$ ... $2^{63}-1$ \\
\texttt{Float}  &  32 & ± $3.4028235 \cdot 10^{38}$ \\
\texttt{Double} &  64 & ± $1.7976931348623157 \cdot 10^{308}$ \\
\end{tabular}
\end{table}

Grundtypen \texttt{String} lagras som en \emph{sekvens} av 16-bitars tecken av typen \texttt{Char} och kan vara av godtycklig längd (tills minnet tar slut).

\end{Slide}


\begin{Slide}{Uttryck}
\begin{itemize}
\item Ett \Emph{uttryck} består av en eller flera delar som efter \Emph{evaluering} ger ett \Alert{resultat}.
\item Delar i ett uttryck kan t.ex. vara: \\ literaler (42), operatorer (+), funktioner (sin), ...
\item Exempel:
\begin{itemize}
\item Ett enkelt uttryck: \\ \code{42.0}
\item Sammansatta uttryck: \\
\code{40 + 2} \\
\code{(20 + 1) * 2} \\
\code{sin(0.5 * Pi)} \\
\code{"hej" + " på " + "dej"}
\end{itemize}

\item När programmet tolkas sker \Emph{evaluering} av uttrycket, vilket ger ett resultat i form av ett \Emph{värde} som har en \Emph{typ}.
\end{itemize}
\end{Slide}


\begin{Slide}{Variabler}\SlideFontSmall
\begin{itemize}
\item En \Emph{variabel} kan tilldelas värdet av ett enkelt eller sammansatt uttryck.
\item En variabel har ett \Emph{variabelnamn}, vars utformning följer språkets regler för s.k. \Emph{identifierare}.
\item En ny variabel införs i en \Emph{variabeldeklaration} och då den kan ges ett värde, \Emph{initialiseras}. Namnet användas som \Emph{referens} till värdet.
\item Exempel på variabeldeklarationer i Scala, notera \Emph{nyckelordet} \code{val}:
\begin{Code}
val a = 0.5 * Pi
val length = 42 * sin(a)
val exclamationMarks = "!!!"
val greetingSwedish = "Hej på dej" + exclamationMarks
\end{Code}

\item Vid exekveringen av programmet lagras variablernas värden i minnet och deras respektive värde hämtas ur minnet när de \Emph{refereras}.

\item Variabler som deklareras med \code{val} kan endast tilldelas ett värde \Alert{en enda gång}, vid den initialisering som sker vid deklarationen.
\end{itemize}

\end{Slide}


\begin{Slide}{Regler för identifierare}
\begin{itemize}
\item \Emph{Enkel} identifierare: t.ex. \code{gurka2tomat}
\begin{itemize}
\item Börja med bokstav
\item ...följt av bokstäver eller siffror
\item Kan även innehålla understreck
\end{itemize}

\item \Emph{Operator}-identifierare, t.ex. \code{+:}
\begin{itemize}
\item Börjar med ett \Emph{operatortecken}, t.ex. \code{+ - * / : ? ~ #}
\item Kan följas av fler operatortecken
\end{itemize}


\item En identifierare får \Alert{inte} vara ett \Emph{reserverat ord}, se \href{http://cs.lth.se/pgk/quickref}{snabbreferensen} för alla reserverade ord i Scala.

\item \Emph{Bokstavlig} identifierare: \code{`kan innehålla allt`}
\begin{itemize}
\item Börjar och slutar med \Emph{backticks}  \code{` `}
\item Kan innehålla vad som helst (utom backticks)
\item Kan användas för att undvika krockar med reserverade ord: \texttt{\code{`}val\code{`}}
\end{itemize}

\end{itemize}
\end{Slide}


\begin{Slide}{Att bygga strängar: konkatenering och interpolering}
\begin{itemize}
\item Man kan \Emph{konkatenera} strängar med operatorn + \\ \code{"hej" + " på " + "dej"}
\item Efter en sträng kan man konkatenera vilka uttryck som helst; uttryck inom parentes evalueras först och värdet görs sen om till en sträng före konkateneringen:
\begin{Code}
val x = 42
val msg = "Dubbla värdet av " + x + " är " + (x * 2) + "."
\end{Code}
\item Man kan i Scala få hjälp av kompilatorn att övervaka bygget av strängar med \Emph{stränginterpolatorn} \Alert{s}:
\begin{Code}
val msg = s"Dubbla värdet av $x är ${x * 2}."
\end{Code}

\end{itemize}
\end{Slide}

\begin{Slide}{Heltalsaritmetik}\SlideFontSmall
\begin{itemize}
\item De fyra räknesätten skrivs som i matematiken (vanlig \href{https://en.wikipedia.org/wiki/Order_of_operations}{precedens}):
\begin{REPL}
scala> 3 + 5 * 2 - 1
res0: Int = 12
\end{REPL}
\item \Emph{Parenteser} styr \Alert{evalueringsordningen}:
\begin{REPL}
scala> (3 + 5) * (2 - 1)
res1: Int = 8
\end{REPL}
\item \Emph{Heltalsdivision} sker med \Alert{decimaler avkortade}:
\begin{REPL}
scala> 41 / 2
res2: Int = 20
\end{REPL}
\item \href{https://en.wikipedia.org/wiki/Modulo_operation}{\Emph{Moduloräkning}} med restoperatorn \code{%}
\begin{REPL}
scala> 41 % 2
res3: Int = 1
\end{REPL}
\end{itemize}
\end{Slide}


\begin{Slide}{Flyttalsaritmetik}\SlideFontSmall
\begin{itemize}
\item Decimaltal representeras med s.k. \href{https://sv.wikipedia.org/wiki/Flyttal}{\Emph{flyttal}} av typen \code{Double}:
\begin{REPL}
scala> math.Pi
res4: Double = 3.141592653589793
\end{REPL}

\item Stora tal så som $\pi*10^{12}$ skrivs:
\begin{REPL}
scala> math.Pi * 1E12
res5: Double = 3.141592653589793E12
\end{REPL}
\item Det finns \Alert{inte} oändligt antal decimaler vilket ger problem med \Alert{avvrundingsfel}:
\begin{REPL}
scala> 0.0000000000001
res6: Double = 1.0E-13

scala> 1E10 + 0.0000000000001
res7: Double = 1.0E10
\end{REPL}
\end{itemize}
\end{Slide}

% \ifkompendium\else
% \begin{SlideExtra}{På rasten}
% En per grupp kommer fram hit och tar en grupp-skylt
% \begin{itemize}
%   \item Samlas i era samarbetsgrupper i foajen
%   \item D1.01a längst västerut (mot havet), D1.12b längst österut
%   \item Lär allas namn
%   \item Bestäm tid för första möte
%   \item Vid behov: \\ Bestäm vem som mejlar till de i gruppen som inte är här idag 
% \end{itemize}  
% \end{SlideExtra}
% \fi

\Subsection{Funktioner}

\begin{Slide}{Definiera namn på uttryck}
\begin{itemize}
\item Med nyckelordet \code{def} kan man låta ett \Emph{namn} betyda samma sak som ett \Emph{uttryck}.
\item Exempel:
\begin{Code}
def gurklängd = 42 + x
\end{Code}
\item Uttrycket till höger evalueras \Alert{varje} gång \Emph{anrop} sker,\\
d.v.s. varje gång namnet används på annat ställe i koden.
\begin{Code}
gurklängd
\end{Code}

\end{itemize}
\end{Slide}

\begin{Slide}{Funktion, argument, parameter}\SlideFontSmall
\begin{itemize}
\item En \Emph{funktion} räknar ut \Alert{resultat} baserat på indata som kallas \Emph{argument}.

\item Argument ges namn genom deklaration av \Emph{parametrar}.

\item Exempel på deklaration av en funktion med en parameter:
\begin{Code}
def dubblera(x: Int) = 2 * x
\end{Code}

\item Parametrarnas typ \Alert{måste} beskrivas efter \Emph{kolon}.
\item Kompilatorn kan härleda \Emph{returtypen}, men den kan också med fördel, för tydlighetens skull, anges \Alert{explicit}:
\begin{Code}
def dubblera(x: Int): Int = 2 * x
\end{Code}

\item Observera att namnet \code{x} blir ett ''nytt fräscht'' \Emph{lokalt namn} som \Alert{bara finns och syns  ''inuti'' funktionen} och har inget med ev. andra \code{x} utanför funktionen att göra.

\item Beräkningen sker först vid \Alert{anrop} av funktionen:
\begin{REPL}
scala> dubblera(42)
res1: Int = 84
\end{REPL}

\end{itemize}
\end{Slide}






\begin{Slide}{Färdiga matte-funktioner i paketet \texttt{scala.math}}\SlideFontSmall
\begin{itemize}
\item I paketet \texttt{\Emph{scala.math}} finns många användbara funktioner: t.ex.\\
\code{math.random()} ger slumptal mellan \code{0.0} och \code{0.99999999999999999}
\begin{REPLnonum}
scala> val x = math.random()
x: Double = 0.27749191749889635

scala> val length = 42.0 * math.sin(math.Pi / 3.0)
length: Double = 36.373066958946424
\end{REPLnonum}

\item Studera dokumentationen här: \\{\SlideFontTiny
\url{http://www.scala-lang.org/api/current/scala/math/}}

\item Paketet \code{scala.math} delegerar ofta till Java-klassen \texttt{\Emph{java.lang.Math}} som är dokumenterad här: \\{\SlideFontTiny
\url{https://docs.oracle.com/javase/8/docs/api/java/lang/Math.html}}

\end{itemize}
\end{Slide}



\Subsection{Logik}

\begin{Slide}{Logiska uttryck}\SlideFontSmall
\begin{minipage}{.8\textwidth}
\begin{itemize}
\item Datorn kan ''räkna'' med sanning och falskhet: \\
s.k. \href{https://en.wikipedia.org/wiki/Boolean_algebra}{booelsk algebra} efter \href{https://en.wikipedia.org/wiki/George_Boole}{George Boole}

\item Enkla logiska uttryck: (finns bara två stycken)
\begin{itemize}
\item[] \code{true}
\item[] \code{false}
\end{itemize}
\end{itemize}
\end{minipage}%
\begin{minipage}{.2\textwidth}
\centering\includegraphics[width=0.9\columnwidth]{../img/boole}
\end{minipage}%
\begin{itemize}


\item Sammansatta logiska uttryck med logiska operatorer:\\
\code{&&} och, \texttt{||} eller, \texttt{!} icke, \code{==} likhet, \code{!=} olikhet,
relationer: \code{> < >= <=}

\item Exempel:
\begin{itemize}
\item[] \code{true && true}
\item[] \code{false || true}
\item[] \code{!false}
\item[] \code{42 == 43}
\item[] \code{42 != 43}
\item[] \code{(42 >= 43) || (1 + 1 == 2)}
\end{itemize}

\end{itemize}
\end{Slide}

\begin{Slide}{De Morgans lagar}

\href{https://en.wikipedia.org/wiki/Augustus_De_Morgan}{\Emph{De Morgans lagar}} beskriver vad som händer om man \Alert{negerar} ett logiskt uttryck. Kan användas för att göra \Emph{förenklingar}.

%$p$ och $q$ är logiska uttryck, $\neg$ står för ''icke'', $\wedge$ för ''och'', $\vee$ för ''eller'':
%\begin{eqnarray*}
%\neg (p \wedge q) & \Longleftrightarrow & (\neg p) \vee (\neg q)\\
%\neg (p \vee q) & \Longleftrightarrow & (\neg p) \wedge (\neg q)\\
%\end{eqnarray*}

\begin{itemize}
\item I alla deluttryck sammanbundna med \code{&&} eller \code{||}, \\ ändra alla \code{&&} till \code{||} och omvänt.
\item Negera alla ingående deluttryck. En relation negeras genom att man byter \texttt{==} mot \texttt{!=}, \texttt{<} mot \texttt{>=}, etc.
\end{itemize}

Exempel på förenkling där de Morgans lagar används upprepat:

\begin{Code}[escapechar=X,backgroundcolor=,frame=none,basicstyle=\ttfamily\fontsize{10}{12}\selectfont]
! (a < b || (a == 1 && b == 1))             X$\iff$X
! (a < b) && ! (a == 1 && b == 1)           X$\iff$X
! (a < b) && (! (a == 1) || ! (b == 1))     X$\iff$X
a >= b && (a != 1 || b != 1)
\end{Code}
\end{Slide}

\begin{Slide}{Alternativ med if-uttryck}
\begin{itemize}
\item Ett if-uttryck börjar med nyckelordet \code{if}, följt av ett logiskt uttryck (villkor) inom parentes och två grenar.
\begin{Code}
def slumpgrönsak = if math.random() < 0.8 then "gurka" else "tomat"
\end{Code}
\item Uttrycket efter \code{then} blir resultatet om villkoret är \code{true}
\item Uttrycket efter \code{else} blir resultatet om villkoret är \code{false}
\begin{REPLnonum}
scala> slumpgrönsak
res13: String = gurka

scala> slumpgrönsak
res14: String = gurka

scala> slumpgrönsak
res15: String = tomat

\end{REPLnonum}
\end{itemize}
\end{Slide}

\Subsection{Satser}

\begin{Slide}{Uttryck eller sats?}
Skillnad mellan uttryck och sats:
\begin{itemize}
\item Ett uttryck ger ett \Emph{resultat}. Exempel: \code{1+1}
\item En sats har en \Alert{effekt}. \\Exempel: utskrift, spara på fil, tilldela variabel nytt värde.
\end{itemize}
Skriv ett \Emph{uttryck} när du är intresserad av \Emph{värdet} som beräknas.\\
Skriv en \Alert{sats} när du vill att något ska \Alert{göras}.\\~\\
{\SlideFontSmall Både satser och uttryck kan i sin tur innehålla satser och uttryck i godtyckligt komplexa \textbf{nästlade strukturer} (mer om det senare).}
\end{Slide}

%%%%%%%%%%%%%%%%%%%%%%%%
\begin{Slide}{Variabeldeklaration och tilldelningssats}\SlideFontTiny

\begin{itemize}
\item En \Emph{variabeldeklaration} medför att \Alert{plats i datorns minne} reserveras så att värden av den typ som variabeln kan referera till får plats där.

\item Vid deklaration ska variabeln \Emph{initialiseras} med ett startvärde.

\item En \code{val}-deklaration ger en variabel som efter initialisering inte kan ändras.


\begin{multicols}{2}
Dessa deklarationer...
\begin{lstlisting}
var x = 42
val y = x + 1
\end{lstlisting}
... ger detta innehåll någonstans i minnet:

%http://tex.stackexchange.com/questions/18521/tikz-matrix-as-a-replacement-for-tabular
\begin{tikzpicture}[]
\matrix [matrix of nodes, row sep=0, column 2/.style={nodes={rectangle,draw,minimum width=3em}}]
{
x   & 42 \\
y   & 43 \\
};
\end{tikzpicture}
%\end{column}

%\end{columns}

\end{multicols}


\item Med en \Emph{tilldelningssats} ges en tidigare \code{var}-deklarerad variabel ett nytt värde:
\begin{lstlisting}
x = 13
\end{lstlisting}

\item Det gamla värdet försvinner för alltid och det nya värdet lagras istället: \\
\begin{tikzpicture}[]
\matrix [matrix of nodes, row sep=0, column 2/.style={nodes={rectangle,draw,minimum width=3em}}]
{
x   & 13 \\
y   & 43 \\
};
\end{tikzpicture}

Observera att \code{y} här inte påverkas av att x ändrade värde.
\end{itemize}
\end{Slide}

\begin{Slide}{Tilldelningssatser är \emph{inte} matematisk likhet}\SlideFontSmall

\begin{itemize}

\item Likhetstecknet används alltså för att \Emph{tilldela} variabler nya värden och det är \Alert{inte} samma sak som matematisk likhet. Vad händer här?
\begin{lstlisting}
x = x + 1
\end{lstlisting}

\item Denna syntax är ett arv från de gamla språken C, Fortran mfl.

\item I \href{https://en.wikipedia.org/wiki/Assignment_(computer_science)}{andra språk} används  t.ex.  \\\vspace{1em}
\texttt{x := x + 1}  \hspace{2em} eller  \hspace{2em} \texttt{x <- x + 1} \\\vspace{0.5em}

\item Denna syntax visar kanske bättre att tilldelning är en \Emph{stegvis process}:

\begin{enumerate}\SlideFontTiny
\item Först beräknas \Emph{uttrycket till höger} om tilldelningstecknet.
\item Sedan \Emph{ersätts värdet} som variabelnamnet refererar till av det beräknade uttrycket. Det gamla värdet \Alert{försvinner för alltid}.
\end{enumerate}

\end{itemize}

\end{Slide}


\begin{Slide}{Förkortade tilldelningssatser}
\begin{itemize}
\item Det är vanligt att man vill tilldela en variabel ett nytt värde som beror av det gamla, så som i \\\code{x = x + 1}

\item Därför finns \Emph{förkortade tilldelningssatser} som gör så att man sparar några tecken och det blir tydligare (?) vad som sker (när man vant sig vid detta skrivsätt):
\begin{Code}
x += 1
\end{Code}

\item Uttrycket ovan expanderas av kompilatorn till \code{x = x + 1}
\end{itemize}


\end{Slide}


\begin{Slide}{Exempel på förkortade tilldelningssatser}
\begin{REPLnonum}
scala> var x = 42
val x: Int = 42

scala> x *= 2

scala> x
val res0: Int = 84


scala> x /= 3

scala> x
val res1: Int = 28
\end{REPLnonum}
\end{Slide}


\ifkompendium\else
\begin{Slide}{Övning: Tilldelningar i sekvens}\footnotesize

%\begin{columns}
%\begin{column}{0.32\textwidth}
\begin{minipage}{0.32\textwidth}

Rita hur minnet ser ut efter varje rad nedan:
\vskip1em
\begin{lstlisting}[ numbers=left,]
var u = 42
var x = 10
var y = 2 * x + 1
x = 20
var z = y + x + y - x
x += 1; y *= 2
\end{lstlisting}
%\end{column}
\end{minipage}\hspace{2em}%
%\begin{column}{0.6\textwidth}
\begin{minipage}{0.6\textwidth}


\scriptsize En variabel som ännu inte \Emph{initierats} har ett \Alert{odefinierat} värde, anges nedan med frågetecken.
\begin{table}[]
\centering\scriptsize
%http://tex.stackexchange.com/questions/83930/what-are-the-different-kinds-of-boxes-in-latex
\newcommand{\mybox}[1]{\raisebox{-0.5mm}{\framebox(21,14){#1}}\vspace{0.5mm}}
\begin{tabular}{@{}ccccccc}
 & rad 1 & rad 2 & rad 3 & rad 4  & rad 5 & rad 6\\
u& \mybox{42 } &  \mybox{}   &   \mybox{}   & \mybox{} & \mybox{} & \mybox{} \\
x& \mybox{? } &  \mybox{}   &   \mybox{}   & \mybox{} & \mybox{}  & \mybox{} \\
y& \mybox{? } &  \mybox{}   &   \mybox{}   & \mybox{} & \mybox{}  & \mybox{} \\
z& \mybox{? } &  \mybox{}   &   \mybox{}   & \mybox{} & \mybox{}  & \mybox{} \\
\end{tabular}
\end{table}

%\end{column}
%\end{columns}
\end{minipage}%
\end{Slide}
\fi



\begin{Slide}{Variabler som ändrar värden kan vara knepiga}
\begin{itemize}
\item Kod som innehåller variabler som \Alert{förändras} över tid är ofta svårare att läsa och begripa.

\item Många buggar beror på att variabler av misstag förändras på felaktiga och oanade sätt.

\item Föränderliga värden blir speciellt svåra i kod som körs jämlöpande (parallellt).

\item I kod som körs i skarpt läge med många användare (s.k. produktionskod) är därför \code{val} att föredra, medan \code{var} endast används om det \Alert{verkligen} behövs.
\item Alltså: räkna hellre ut nya värden än förändra befintliga.
\end{itemize}
\end{Slide}
%%%%%%%%%%


\Subsection{Kontrollstrukturer}

\begin{Slide}{Kontrollstrukturer: alternativ och repetition}\SlideFontSmall
Används för att kontrollera (förändra) sekvensen och skapa \Emph{alternativa} vägar genom koden. Vägen  bestäms vid körtid.
\begin{itemize}
\item if-sats:
\begin{Code}
if math.random() < 0.8 then println("gurka") else println("tomat")
\end{Code}
\end{itemize}

Olika sorters \Emph{loopar} för att repetera satser. Antalet repetitioner ges vid körtid.
\begin{itemize}
\item \code{while}-sats: bra när man \Alert{inte vet hur många gånger} det kan bli.
\begin{Code}
while math.random() < 0.8 do println("gurka")
\end{Code}

\item \code{for}-sats: bra när man \Alert{vill ange antalet repetitioner}:
\begin{Code}
for i <- 1 to 10 do println(s"gurka nr $i")
\end{Code}

\end{itemize}
\end{Slide}

\begin{Slide}{Scala-2-syntax för kontrollstrukturer fungerar i Scala 3}\SlideFontSmall
I Scala 2 användes en gammal syntax för kontrollstrukturer som liknar mer C/C++/Java. Den är tillåten i Scala 3, men nya mer lättlästa syntaxen är att föredra.
\begin{itemize}
\item Scala-2-syntax för alternativ: parenteser men inget \code{then}
\begin{Code}
if (math.random() < 0.8) println("gurka") else println("tomat")
\end{Code}
\end{itemize}

Scala-2-syntax för repetition:
\begin{itemize}
\item \code{while}-sats: parenteser men inget \code{do}
\begin{Code}
while (math.random() < 0.8) println("gurka")
\end{Code}

\item \code{for}-sats: parenteser men inget \code{do}
\begin{Code}
for (i <- 1 to 10) println(s"gurka nr $i")
\end{Code}

\item Kojo Desktop funkar ännu bara med Scala 2 och gamla syntaxen, men Kojo kan även köras med Scala 3 (se hur i kompendiet).

\end{itemize}

\end{Slide}

\begin{Slide}{Repetera många satser}
Om du vill göra flera saker i sekvens inne i en repetition så kan du skriva flera satser inom \Emph{klammer-parenteser}:
\begin{Code}
while math.random() < 0.8 do {
  println("gurka")
  println("tomat")
}
println("Repetitionen är klar!")
\end{Code}
\pause
Du kan efter vissa nyckelord (t.ex. \code{do}, \code{then}, \code{else}) välja bort klammer-parenteser \Eng{optional braces}. 
\begin{Code}
while math.random() < 0.8 do
  println("gurka")
  println("tomat")

println("Repetitionen är klar!")
\end{Code}
Då är det \Emph{indenteringen} som avgör vilka satser som ingår. \\Detta fungerar i Scala 3 (men inte i Scala 2).
\end{Slide}


\begin{Slide}{Procedurer}\SlideFontSmall
\begin{itemize}
\item En \Emph{procedur} är en funktion som \Alert{gör} något intressant, men som \Alert{inte} lämnar något intressant returvärde.
\item Exempel på procedur i standardbiblioteket: \code{println("hej")}
\item Du \Emph{deklarerar egna procedurer} genom att ange \texttt{\Alert{Unit}} som returvärdestyp. Då returneras värdet \texttt{\Alert{()}} som betyder ''inget''.
\end{itemize}
\begin{REPLnonum}
scala> def hej(x: String): Unit = println(s"Hej på dej $x!")
hej: (x: String)Unit

scala> hej("Herr Gurka")
Hej på dej Herr Gurka!

scala> val x = hej("Fru Tomat")
Hej på dej Fru Tomat!
x: Unit = ()
\end{REPLnonum}
\begin{itemize}
\item Det som \Alert{görs} kallas (sido)\Emph{effekt}. Ovan är utskriften själva effekten.
\item Även funktioner kan ha sidoeffekter. De kallas då \Alert{oäkta} funktioner.
\end{itemize}
\end{Slide}

\begin{Slide}{Problemlösning: nedbrytning i abstraktioner som sen kombineras}\SlideFontSmall
\begin{itemize}
\item En av de allra viktigaste principerna inom programmering är \Emph{funktionell nedbrytning} där  \Emph{underprogram} i form av funktioner och procedurer skapas för att bli byggstenar som kombineras till mer avancerade funktioner och procedurer.

\item Genom de namn som definieras skapas \Emph{återanvändbara abstraktioner} som kapslar in det funktionen gör till ett ''byggblock''.

\item Bra ''byggblock'' gör det lättare att lösa svåra programmeringsproblem.

\item Abstraktioner som beräknar eller gör \Emph{en enda, väldefinierad sak} är enklare att använda, jämfört med de som gör många, helt olika saker.

\item Abstraktioner med \Emph{välgenomtänkta namn} är enklare att använda, jämfört med kryptiska eller missvisande namn.
\end{itemize}

\end{Slide}

\Subsection{Veckans uppgifter}

\begin{Slide}{Övning \texttt{expressions} och labb \texttt{kojo}}
\label{w01:kojo-slide}
\begin{itemize}
  \item På övningen kör du Scala REPL för att träna på SARA.
  \item[] \Alert{Läs i Appendix} och på kursens hemsida under ''Verktyg'' om hur du installerar och får igång Scala REPL.
  \item På laborationen använder du barnvänliga \Emph{Kojo} för träna på SARA, med fokus på abstraktion.
  \item Det finns tre olika sätt att använda Kojo:
  \begin{enumerate}
  \item Grafikbiblioteket \textbf{\texttt{kojolib}} i ett fristående Scala program med hjälp av en professionell kodeditor och kompilering och exekvering i terminalen. \Emph{Fungerar fint med nya Scala 3}.
  \item Skrivbordsappen \textbf{Kojo Desktop} med inbyggd barnvänlig editor (endast Scala 2).
  \item Webbappen \textbf{\url{http://kojo.lu.se/}} som körs direkt i din webbläsare (endast Scala 2, begränsade funktioner).
  \end{enumerate}
\end{itemize}  
Alternativ 1 rekommenderas, men om du försenas av tekniskt strul, så kom igång med 2 el. 3 så länge tills du fått hjälp.
\end{Slide}

\ifkompendium\else

\begin{SlideExtra}{Om veckans övning: \code{expressions}}\SlideFontSmall

\begin{itemize}
%!TEX encoding = UTF-8 Unicode

\item Förstå vad som händer när satser exekveras och uttryck evalueras.
\item Förstå sekvens, alternativ och repetition.
\item Känna till litteralerna för enkla värden, deras typer och omfång.
\item Kunna deklarera och använda variabler och tilldelning, samt kunna rita bilder av minnessituationen då variablers värden förändras.
\item Förstå skillnaden mellan olika numeriska typer, kunna omvandla mellan dessa och vara medveten om noggrannhetsproblem som kan uppstå.
\item Förstå booleska uttryck och värdena \code{true} och \code{false}, samt kunna förenkla booleska uttryck.
\item Förstå skillnaden mellan heltalsdivision och flyttalsdivision, samt användning av rest vid heltalsdivision.
\item Förstå precedensregler och användning av parenteser i uttryck.
\item Kunna använda \code{if}-satser och \code{if}-uttryck.
\item Kunna använda \code{for}-satser och \code{while}-satser.
\item Kunna använda \code{math.random()} för att generera slumptal i olika intervaller.
\item Kunna beskriva skillnader och likheter mellan en procedur och en funktion.

\end{itemize}

\end{SlideExtra}

\begin{SlideExtra}{Om veckans labb: \code{kojo}}
\begin{itemize}
%!TEX encoding = UTF-8 Unicode
%!TEX root = ../compendium2.tex

\item Kunna tillämpa och kombinera principerna sekvens, alternativ, repetition, och abstraktion i skapandet av egna program om minst 20 rader kod.
\item Kunna förklara vad ett program gör i termer av sekvens, alternativ, repetition, och abstraktion.
\item Kunna formatera egna program så att de blir lätta att läsa och förstå.
\item Kunna förklara vad en variabel är och kunna deklarera oföränderliga och förändringsbara variabler, samt göra tilldelningar.
\item Kunna genomföra upprepade varv i cykeln \emph{editera-exekvera-felsöka/förbättra} för att stegvis bygga upp allt mer utvecklade program.

\end{itemize}
\end{SlideExtra}

\fi

%%%%%%%%%%%%%%%%%%%%%%%%%%%%%%%%%%%%%% nedan redan meddelat i introveckan
%\ifkompendium\else
%\Subsection{Meddelande från \href{http://lth.se/code}{Code@LTH}}
%\fi









%%!TEX encoding = UTF-8 Unicode
\chapter{Introduktion}\label{chapter:W01}
Begrepp som ingår i denna veckas studier:
\begin{multicols}{2}\begin{itemize}[noitemsep,label={$\square$},leftmargin=*]
\item sekvens
\item alternativ
\item repetition
\item abstraktion
\item editera
\item kompilera
\item exekvera
\item datorns delar
\item virtuell maskin
\item litteral
\item värde
\item uttryck
\item identifierare
\item variabel
\item typ
\item tilldelning
\item namn
\item val
\item var
\item def
\item definera och anropa funktion
\item funktionshuvud
\item funktionskropp
\item procedur
\item inbyggda grundtyper
\item Int
\item Long
\item Short
\item Double
\item Float
\item Byte
\item Char
\item String
\item println
\item typen Unit
\item enhetsvärdet ()
\item stränginterpolatorn s
\item if
\item else
\item true
\item false
\item MinValue
\item MaxValue
\item aritmetik
\item slumptal
\item math.random
\item logiska uttryck
\item de Morgans lagar
\item while-sats
\item for-sats\end{itemize}\end{multicols}

%!TEX encoding = UTF-8 Unicode
%!TEX root = ../exercises.tex

\ifPreSolution
\Exercise{\ExeWeekONE}\label{exe:W01}

\begin{Goals}
%!TEX encoding = UTF-8 Unicode

\item Förstå vad som händer när satser exekveras och uttryck evalueras.
\item Förstå sekvens, alternativ och repetition.
\item Känna till litteralerna för enkla värden, deras typer och omfång.
\item Kunna deklarera och använda variabler och tilldelning, samt kunna rita bilder av minnessituationen då variablers värden förändras.
\item Förstå skillnaden mellan olika numeriska typer, kunna omvandla mellan dessa och vara medveten om noggrannhetsproblem som kan uppstå.
\item Förstå booleska uttryck och värdena \code{true} och \code{false}, samt kunna förenkla booleska uttryck.
\item Förstå skillnaden mellan heltalsdivision och flyttalsdivision, samt användning av rest vid heltalsdivision.
\item Förstå precedensregler och användning av parenteser i uttryck.
\item Kunna använda \code{if}-satser och \code{if}-uttryck.
\item Kunna använda \code{for}-satser och \code{while}-satser.
\item Kunna använda \code{math.random()} för att generera slumptal i olika intervaller.
\item Kunna beskriva skillnader och likheter mellan en procedur och en funktion.

\end{Goals}

\begin{Preparations}
\item \StudyTheory{01}
\item Du behöver en dator med Scala och Kojo installerad, se appendix~\ref{appendix:compile} och  \ref{appendix:kojo}.
\end{Preparations}

\else

\ExerciseSolution{\ExeWeekONE}

\fi  %%% END \ifPreSolution


\BasicTasks
%%%% TODO Strukturera övningen annorlunda: atomer, sammansatta uttryck, funktiner, kojo ??}


\def\what{\emph{Para ihop begrepp med beskrivning.}}

\QUESTBEGIN

\Task \what

\vspace{1em}\noindent Koppla varje begrepp med den (förenklade) beskrivning som passar bäst: 

\begin{ConceptConnections}
  litteral & 1 & & A & att översätta kod till exekverbar form \\ 
  sträng & 2 & & B & anger ett specifikt datavärde \\ 
  sats & 3 & & C & decimaltal med begränsad noggrannhet \\ 
  uttryck & 4 & & D & bra då antalet repetitioner ej är bestämt i förväg \\ 
  funktion & 5 & & E & vid anrop sker (sido)effekt; returvärdet är tomt \\ 
  procedur & 6 & & F & sker innan exekveringen startat \\ 
  exekveringsfel & 7 & & G & bra då antalet repetitioner är bestämt i förväg \\ 
  kompileringsfel & 8 & & H & för att ändra en variabels värde \\ 
  abstrahera & 9 & & I & sker medan programmet kör \\ 
  kompilera & 10 & & J & beskriver vad data kan användas till \\ 
  typ & 11 & & K & antingen sann eller falsk \\ 
  for-sats & 12 & & L & vid anrop beräknas ett returvärde \\ 
  while-sats & 13 & & M & en kodrad som gör något; kan särskiljas med semikolon \\ 
  tilldelning & 14 & & N & kombinerar värden och funktioner till ett nytt värde \\ 
  flyttal & 15 & & O & en sekvens av tecken \\ 
  boolesk & 16 & & P & att införa nya begrepp som förenklar kodningen \\ 
\end{ConceptConnections}

\SOLUTION

\TaskSolved \what

\begin{ConceptConnections}
  litteral & 1 & ~~\Large$\leadsto$~~ &  C & anger ett specifikt datavärde \\ 
  sträng & 2 & ~~\Large$\leadsto$~~ &  G & en sekvens av tecken \\ 
  sats & 3 & ~~\Large$\leadsto$~~ &  K & en kodrad som gör något; kan särskiljas med semikolon \\ 
  uttryck & 4 & ~~\Large$\leadsto$~~ &  N & kombinerar värden och funktioner till ett nytt värde \\ 
  funktion & 5 & ~~\Large$\leadsto$~~ &  L & vid anrop beräknas ett returvärde \\ 
  procedur & 6 & ~~\Large$\leadsto$~~ &  H & vid anrop sker (sido)effekt; returvärdet är tomt \\ 
  exekveringsfel & 7 & ~~\Large$\leadsto$~~ &  P & kan inträffa medan programmet kör \\ 
  kompileringsfel & 8 & ~~\Large$\leadsto$~~ &  A & kan inträffa innan exekveringen startat \\ 
  abstrahera & 9 & ~~\Large$\leadsto$~~ &  F & att införa nya begrepp som förenklar kodningen \\ 
  kompilera & 10 & ~~\Large$\leadsto$~~ &  B & att översätta kod till exekverbar form \\ 
  typ & 11 & ~~\Large$\leadsto$~~ &  D & beskriver vad data kan användas till \\ 
  for-sats & 12 & ~~\Large$\leadsto$~~ &  O & bra då antalet repetitioner är bestämt i förväg \\ 
  while-sats & 13 & ~~\Large$\leadsto$~~ &  E & bra då antalet repetitioner ej är bestämt i förväg \\ 
  tilldelning & 14 & ~~\Large$\leadsto$~~ &  I & för att ändra en variabels värde \\ 
  flyttal & 15 & ~~\Large$\leadsto$~~ &  M & decimaltal med begränsad noggrannhet \\ 
  boolesk & 16 & ~~\Large$\leadsto$~~ &  J & antingen sann eller falsk \\ 
\end{ConceptConnections}

\QUESTEND






\def\what{\emph{Utskrift i Scala REPL.}}

\QUESTBEGIN

\Task \what 

\vspace{1em}\noindent Starta Scala REPL \Eng{Read-Evaluate-Print-Loop}.

\begin{REPLnonum}
$ scala
Welcome to Scala version 2.11.8 (Java HotSpot(TM) 64-Bit Server VM, Java 1.8).
Type in expressions to have them evaluated.
Type :help for more information.

scala> 
\end{REPLnonum}

\Subtask Skriv efter prompten \code{scala>} en sats som skriver ut en valfri (bruklig/knasig) hälsningsfras, genom anrop av proceduren \code{println} med något strängargument. Tryck på \textit{Enter} så att satsen kompileras och exekveras. 

\Subtask Skriv samma sats igen (eller tryck pil-upp) men ''glöm bort'' att skriva högerparentesen efter argumentet innan du trycker på \textit{Enter}. Vad händer?

\begin{framed}
\noindent\emph{Tips inför fortsättningen:} Det finns många användbara kortkommandon och andra trix för att jobba snabbt i REPL. Be gärna någon som kan dessa trix att visa dig hur man kan jobba snabbare. Läs appendix \ref{appendix:compile:REPL} och prova sedan att kopiera och klistra in text. Använd piltangenterna för att bläddra i historiken och Ctrl+A för att komma till början av raden, Ctrl+K för att radera resten av raden, etc.
\end{framed}



\SOLUTION 
\TaskSolved \what

\SubtaskSolved Till exempel:
\begin{REPLnonum}
scala> println("hejsan svejsan")
\end{REPLnonum}

\SubtaskSolved Om högerparentes fattas får man fortsätta skriva på nästa rad. Detta indikeras med vertikalstreck i början av varje ny rad:
\begin{REPLnonum}
scala> println("hejsan svejsan"
     | + "!" 
     | )
hejsan svejsan!
\end{REPLnonum}

\QUESTEND



\def\what{\emph{Konkatenering av strängar.}}

\QUESTBEGIN

\Task \what

\Subtask Skriv ett uttryck som konkatenerar två strängar, t.ex. \code{"gurk"} och \code{"burk"}, med hjälp av operatorn \code{+} och studera resultatet. Vad har uttrycket för värde och typ? Vilken siffra står efter ordet \code{res} i variabeln som lagrar resultatet?

\Subtask Använd resultatet från konkateneringen, t.ex. \code{res0} (byt ev. ut \code{0}:an mot siffran efter \code{res} i utskriften från förra evalueringen), och skriv ett uttryck med hjälp av operatorn \code{*} som upprepar resultatet från förra deluppgiften 42 gånger. 


\SOLUTION

\TaskSolved \what

\SubtaskSolved 
\begin{REPLnonum}
scala> "gurk" + "burk"
res1: String = gurkburk
\end{REPLnonum}
värde: \code{"gurkburk"}, typ:  \code{String}

\SubtaskSolved
\begin{REPLnonum}
scala> res1 * 42
res2: String = gurkatomatgurkatomatgurkatomatgurkatomatgurkatomatgurkatomatgurkatomatgurkatomatgurkatomatgurkatomatgurkatomatgurkatomatgurkatomatgurkatomatgurkatomatgurkatomatgurkatomatgurkatomatgurkatomatgurkatomatgurkatomatgurkatomatgurkatomatgurkatomatgurkatomatgurkatomatgurkatomatgurkatomatgurkatomatgurkatomatgurkatomatgurkatomatgurkatomatgurkatomatgurkatomatgurkatomatgurkatomatgurkatomatgurkatomatgurkatomatgurkatomatgurkatomat
\end{REPLnonum}

\QUESTEND




\def\what{\emph{När upptäcks felet?}}

\QUESTBEGIN

\Task \what 

\Subtask Vad har uttrycket \code{ "hej" * 3 } för typ och värde? Testa i REPL.

\Subtask Byt ut 3:an ovan mot ett så pass stort heltal så att minnet blir fullt. Hur börjar felmeddelandet? Är detta ett körtidsfel eller ett kompileringsfel?

\Subtask Välj ett värde på argumentet efter operatorn \code{*} så att ett typfel genereras. Hur börjar felmeddelandet? Är detta ett körtidsfel eller ett kompileringsfel?

\begin{framed}
\noindent\emph{Tips inför fortsättningen:} Gör gärna fel när du kodar så lär du dig mer! Träna på att tolka olika felmeddelanden och fråga någon om hjälp om du inte förstår. Kompilatorns utskrifter kan vara till stor hjälp, men är ibland kryptiska. Om du kör fast och inte kommer vidare själv så be om hjälp, \emph{men be om tips snarare än färdiga lösningar} så att du behåller initiativet själv och tar kontroll över nästa steg i ditt lärande.
\end{framed}


\SOLUTION

\TaskSolved \what

\SubtaskSolved Typ: \code{String}, värde: \code{"hejhejhej"}

\SubtaskSolved Körtiddsfel:
\begin{REPLnonum}
scala> "hej" * Int.MaxValue
java.lang.OutOfMemoryError: Java heap space
\end{REPLnonum}

\SubtaskSolved Kompileringsfel: (indikeras av texten \code{<console> ... error:})
\begin{REPLnonum}
scala> "hej" * true
<console>:12: error: type mismatch;
 found   : Boolean(true)
 required: Int
       "hej" * true
\end{REPLnonum}


\QUESTEND




\def\what{\emph{Litteraler och typer.}}

\QUESTBEGIN

\Task \what

\Subtask Ta hjälp av REPL-kommadot \verb+:type+ (kan förkortas \code{:t}) vid behov för att para ihop nedan litteraler med rätt typ. 

\begin{ConceptConnections}[0.35\textwidth]
  \code|1    | & 1 & & A & \code|Char   | \\ 
  \code|1L   | & 2 & & B & \code|Double | \\ 
  \code|1.0  | & 3 & & C & \code|Boolean| \\ 
  \code|1D   | & 4 & & D & \code|Int    | \\ 
  \code|1F   | & 5 & & E & \code|Boolean| \\ 
  \code|'1'  | & 6 & & F & \code|Double | \\ 
  \code|"1"| & 7 & & G & \code|Long   | \\ 
  \code|true | & 8 & & H & \code|Float  | \\ 
  \code|false| & 9 & & I & \code|String | \\ 
  \code|()   | & 10 & & J & \code|Unit   | \\ 
%\Connect{\code|1      |}  {\code|Int    |}
%\Connect{\code|1L     |}  {\code|Long   |}
%\Connect{\code|1.0    |}  {\code|Double |}
%\Connect{\code|1D     |}  {\code|Double |}
%\Connect{\code|1F     |}  {\code|Float  |}
%\Connect{\code|'1'    |}  {\code|Char   |}
%\Connect{\code|\"1\"  |}  {\code|String |}
%\Connect{\code|true   |}  {\code|Boolean|} 
%\Connect{\code|false  |}  {\code|Boolean|} 
%\Connect{\code|()     |}  {\code|Unit   |} 
\end{ConceptConnections}

\Subtask Vad händer om du adderar 1 till det största möjliga värdet av typen \code{Int}? 
\\\emph{Tips:} se snabbreferensen \footnote{\url{http://cs.lth.se/pgk/quickref/}} under rubriken ''The Scala type system'' avsnitt ''Methods on numbers''.

\Subtask Vad är skillnaden mellan typerna \code{Long} och \code{Int}?

\Subtask Vad är skillnaden mellan typerna \code{Double} och \code{Float}?


\SOLUTION

\TaskSolved \what

\SubtaskSolved 

\begin{ConceptConnections}
  \code|1    | & 1 & ~~\Large$\leadsto$~~ &  D & \code|Int    | \\ 
  \code|1L   | & 2 & ~~\Large$\leadsto$~~ &  J & \code|Long   | \\ 
  \code|1.0  | & 3 & ~~\Large$\leadsto$~~ &  G & \code|Double | \\ 
  \code|1D   | & 4 & ~~\Large$\leadsto$~~ &  C & \code|Double | \\ 
  \code|1F   | & 5 & ~~\Large$\leadsto$~~ &  B & \code|Float  | \\ 
  \code|'1'  | & 6 & ~~\Large$\leadsto$~~ &  H & \code|Char   | \\ 
  \code|"1"| & 7 & ~~\Large$\leadsto$~~ &  A & \code|String | \\ 
  \code|true | & 8 & ~~\Large$\leadsto$~~ &  E & \code|Boolean| \\ 
  \code|false| & 9 & ~~\Large$\leadsto$~~ &  F & \code|Boolean| \\ 
  \code|()   | & 10 & ~~\Large$\leadsto$~~ &  I & \code|Unit   | \\ 
%\ConnectSolved{\code|1      |}  {\code|Int    |}
%\ConnectSolved{\code|1L     |}  {\code|Long   |}
%\ConnectSolved{\code|1.0    |}  {\code|Double |}
%\ConnectSolved{\code|1D     |}  {\code|Double |}
%\ConnectSolved{\code|1F     |}  {\code|Float  |}
%\ConnectSolved{\code|'1'    |}  {\code|Char   |}
%\ConnectSolved{\code|\"1\"  |}  {\code|String |}
%\ConnectSolved{\code|true   |}  {\code|Boolean|} 
%\ConnectSolved{\code|false  |}  {\code|Boolean|} 
\end{ConceptConnections}

\SubtaskSolved Värdet går över gränsen för vad som får plats i ett 32 bitars heltal och ''börjar om'' på det minsta möjliga heltalet \code{Int.MinValue}
\begin{REPL}
scala> Int.MaxValue + 1
res3: Int = -2147483648

scala> Int.MinValue
res4: Int = -2147483648
\end{REPL}

\SubtaskSolved Båda är heltal men \code{Long} kan representera större tal än \code{Int}.

\SubtaskSolved Båda är flyttal men \code{Double} har dubbel precision och kan representera större tal med fler decimaler.



\QUESTEND





\def\what{\emph{Matematiska funktioner. Scaladoc.}}

\QUESTBEGIN

\Task \what

\Subtask Antag att du har ett schackbräde med 64 rutor. Tänk dig att du börjar med ett enda riskorn på första rutan och sedan lägger dubbelt så många riskorn i en ny hög för varje efterföljande ruta: 1, 2, 4, 8, ...  etc. Hur många riskorn\footnote{\url{https://en.wikipedia.org/wiki/Wheat_and_chessboard_problem}} blir det då i den sextiofjärde rishögen?

\emph{Tips:} Du ska beräkna $2^{64} - 1$. Om du skriver \code{math.} i REPL och trycker TAB får du se inbyggda matematiska funktioner i Scalas standardbibliotek:
\begin{REPL}
scala> math.    // Tryck TAB direkt efter punkten och betrakta listan
\end{REPL}
Använd funktionen \code{math.pow} och lämpliga argument. Om du skriver \code{math.pow} och trycker TAB \emph{två gånger} får du se funktionshuvudet med parameterlistan. 

Om du surfar till \url{http://www.scala-lang.org/api/current/} och skriver \code{math} i sökrutan och sedan, efter att du klickat på \textbf{\textsf{\small scala.math}}, skriver \textbf{\textsf{\small pow}} i rutan längre ner, så filtreras sidan och du hittar dokumentationen av \code{ def pow } som du kan klicka på och läsa mer om.   

\Subtask Definiera funktionen \code{omkrets} nedan i REPL. Går det bra att utelämna returtyp-annoteringen? Varför? Finns det anledning att ha den kvar?
\begin{Code}
def omkrets(radie: Double): Double = 2 * math.Pi * radie
\end{Code}

\Subtask Jordens (genomsnittliga) diameter (vid ekvatorn) är ca $12 750$ $km$. Anropa funktionen \code{omkrets} ovan för att beräkna hur många kilometer per dag man ungefär måste färdas om man vill åka jorden runt på 80 dagar. 

\SOLUTION

\TaskSolved \what

\SubtaskSolved Ja, returtyp-annoteringen \code{: Double} kan utelämnas. 

\begin{itemize}
\item Varför kan returtyp utelämnas?\\Eftersom kompilatorns typhärledning kan härleda returtypen. 
\item Varför kan man vilja utelämna den?\\Det blir kortare att skriva utan. 
\item Anledningar att ange returtyp: 
\begin{itemize}
\item  Med explicit returtyp får du hjälp av kompilatorn att redan under kompileringen kontrollera att uttrycket till höger om likhetstecknet har den typ som förväntas. 

\item Genom att du anger returtypen explicit får de som enbart läser metodhuvudet (och inte implementationen)
 tydligt se vad som returneras.
\end{itemize}
\end{itemize}	


\SubtaskSolved Beräkning av $2^{64} - 1$ med \code{math.pow} enligt nedan ger ungefär $1.8 \cdot 10^{19}$
\begin{REPL}
scala> math.pow(2, 64) - 1
res0: Double = 1.8446744073709552E19
\end{REPL}


\SubtaskSolved Ca $500$ $km$.
\begin{REPL}
scala> omkrets(12750 / 2) / 80
res0: Double = 500.6913291658733
\end{REPL}

\QUESTEND




\def\what{\emph{Förändringsbara variabler och tilldelning.}}

\QUESTBEGIN

\Task \what~Rita en \emph{ny} bild av datorns minne efter \emph{varje} exekverad rad 1--6 nedan. Varje bild ska visa alla variabler som finns i minnet och deras variabelnamn, typ och värde.

\begin{REPL}[numbers=left, numberstyle=\color{black}\ttfamily\scriptsize\selectfont]
scala> var a = 13
scala> var b = a + 1
scala> var c = (a + b) * 2.0
scala> b = 0
scala> a = 0
scala> c = c + 1
\end{REPL}
Efter första raden ser minnessituationen ut så här:

\MEM{a}{Int}{13}

\SOLUTION

\TaskSolved \what

\begin{tabular}{l l l}
\MEM{{\it Efter rad1:~~~~} a}{Int}{13}\\
\MEM{{\it Efter rad2:~~~~} a}{Int}{13} & \MEM{b}{Int}{14}\\
\MEM{{\it Efter rad3:~~~~} a}{Int}{13} & \MEM{b}{Int}{14} & \MEM{c}{Double}{54.0}\\
\MEM{{\it Efter rad4:~~~~} a}{Int}{13} & \MEM{b}{Int}{0} & \MEM{c}{Double}{54.0}\\
\MEM{{\it Efter rad5:~~~~} a}{Int}{0} & \MEM{b}{Int}{0} & \MEM{c}{Double}{54.0}\\
\MEM{{\it Efter rad6:~~~~} a}{Int}{0} & \MEM{b}{Int}{0} & \MEM{c}{Double}{55.0}\\
\end{tabular}

\QUESTEND


\def\what{\emph{Slumptal med \code{math.random}.}}

\QUESTBEGIN

\Task\label{exercise:expressions:roll} \what

\Subtask Vad ger funktionen \code{math.random} för resultatvärde? Vilken typ? Vad är största och minsta möjliga värde?
\\\emph{Tips:} Se scaladoc här: \Scaladoc och prova i REPL.

\Subtask Deklarera den parameterlösa funktionen \code{def roll: Int = ???} som ska representera ett tärningskast och ge ett slumpmässigt heltal mellan 1 och 6. Testa funktionen genom att anropa den många gånger. \\\emph{Tips:} Använd \code{math.random} och multiplicera och addera med lämpliga heltal. Omge beräkningen med parenteser och avsluta med \code{.toInt} för att avkorta decimaler och omvandla typen från \code{Double} till \code{Int}.

\SOLUTION

\TaskSolved \what

\SubtaskSolved Ur dokumentationen:
\begin{Code}
/** Returns a Double value with a positive sign, 
 *  greater than or equal to 0.0 and less than 1.0.
 */
def random(): Double
\end{Code}


\SubtaskSolved 
\begin{REPL}
scala> def roll: Int = (math.random * 6 + 1).toInt

scala> roll
res0: Int = 4

scala> roll
res1: Int = 1
\end{REPL}

\QUESTEND




\def\what{\emph{Repetition med \code{for}, \code{foreach} och \code{while}.}}

\QUESTBEGIN

\Task \what

\Subtask Så här kan en \code{for}-sats ser ut: 
\begin{Code}
for (i <- 1 to 10) print(i + ", ")
\end{Code}
Använd en \code{for}-sats för att skriva ut resultatet av 100 tärningskast med funktionen \code{roll} från uppgift \ref{exercise:expressions:roll}. 

\Subtask Så här kan en \code{foreach}-sats ser ut: 
\begin{Code}
(1 to 10).foreach { i => print(i + ",") }
\end{Code}
Använd en \code{foreach}-sats för att skriva ut resultatet av 100 tärningskast med funktionen \code{roll} från uppgift \ref{exercise:expressions:roll}. 

\Subtask Så här kan en \code{while}-sats ser ut: 
\begin{Code}
var i = 1
while (i <= 10) { print(i + ","); i = i + 1 }
\end{Code}
Använd en \code{while}-sats för att skriva ut resultatet av 100 tärningskast med funktionen \code{roll} från uppgift \ref{exercise:expressions:roll}. Vad händer om du glömmer \code{i = i + 1} ?


\SOLUTION

\TaskSolved \what

\SubtaskSolved \TODO

\QUESTEND


\def\what{\emph{Alternativ med \code{if}-sats och \code{if}-uttryck.}}

\QUESTBEGIN

\Task \what

\Subtask Så här kan en \code{if}-sats se ut (notera dubbla likhetstecken):
\begin{Code}
if (roll == 3) println("TRE") else println("INTE TRE") 
\end{Code}
Testa ovan i REPL. Skriv sedan en \code{for}-sats som kastar 100 tärningar och skriver ut strängen \code{"GRATTIS!"} om det blir en sexa, annars en ledsen smiley: \code{":("} 

\Subtask Så här kan ett \code{if}-uttryck se ut:
\begin{Code}
if (roll < 6) 0 else 1 
\end{Code}
Testa ovan i REPL. Skriv sedan en \code{while}-sats som kastar 100 tärningar och räknar antalet sexor. 

\SOLUTION

\TaskSolved \what

\SubtaskSolved \TODO

\QUESTEND



\def\what{\emph{Sekvens, sats och procedur.}}

\QUESTBEGIN

\Task \what

\Subtask Vad gör dessa satser? 
\begin{REPLnonum}
scala> def p = { print("san"); print("!"); println("hej")}
scala> p;p;p;p
\end{REPLnonum}

\Subtask
Använd pil-upp för att få tillbaka raden du skrev med definitionen av proceduren \code{p}. Byt plats på strängarna i utskriftsanropen i proceduren \code{p} så att utskriften blir: 
\begin{REPLnonum}
hejsan!
hejsan!
hejsan!
hejsan!
\end{REPLnonum}

\Subtask Hur tolkar kompilatorn klammerparenteser och semikolon?

\SOLUTION

\TaskSolved \what

\SubtaskSolved 
Satserna skapar denna utskrift:
\begin{REPLnonum}
san!hej
san!hej
san!hej
san!hej
\end{REPLnonum}

\SubtaskSolved 
\begin{REPLnonum}
scala> def p = { print("hej"); print("san"); println("!")}
scala> p;p;p;p
\end{REPLnonum}

\SubtaskSolved 
\begin{itemize}
\item Klammerparenteser används för att gruppera flera satser. Klammerparenteser behövs om man vill definiera en funktion som består av mer än en sats.  

\item Semikolon särskiljer flera satser. Semikolon behövs om man vill skriva många satser på samma rad.


\end{itemize}

\QUESTEND




\def\what{\emph{Heltalsdivision.}}

\QUESTBEGIN

\Task \what~Vilket värde och vilken typ hör till vilket uttryck?  Är du osäker på svaret, testa i REPL.

\begin{ConceptConnections}[0.3\textwidth]
  \code| 4 / 42      | & 1 & & A & \code|true : Boolean  | \\ 
  \code| 42.0 / 2    | & 2 & & B & \code|    2: Int      | \\ 
  \code| 42 / 4      | & 3 & & C & \code| 10.5: Double   | \\ 
  \code| 42 % 4      | & 4 & & D & \code|   10: Int      | \\ 
  \code| 4 % 42      | & 5 & & E & \code|    0: Int      | \\ 
  \code| 40 % 4 == 0 | & 6 & & F & \code|false: Boolean  | \\ 
  \code| 42 % 4 == 0 | & 7 & & G & \code|    4: Int      | \\ 
\end{ConceptConnections}

\SOLUTION

\TaskSolved \what

\begin{ConceptConnections}[0.3\textwidth]
  \code| 4 / 42      | & 1 & ~~\Large$\leadsto$~~ &  A & \code|    0: Int      | \\ 
  \code| 42.0 / 2    | & 2 & ~~\Large$\leadsto$~~ &  G & \code| 10.5: Double   | \\ 
  \code| 42 / 4      | & 3 & ~~\Large$\leadsto$~~ &  E & \code|   10: Int      | \\ 
  \code| 42 % 4      | & 4 & ~~\Large$\leadsto$~~ &  C & \code|    2: Int      | \\ 
  \code| 4 % 42      | & 5 & ~~\Large$\leadsto$~~ &  F & \code|    4: Int      | \\ 
  \code| 40 % 4 == 0 | & 6 & ~~\Large$\leadsto$~~ &  D & \code|true : Boolean  | \\ 
  \code| 42 % 4 == 0 | & 7 & ~~\Large$\leadsto$~~ &  B & \code|false: Boolean  | \\ 
\end{ConceptConnections}

\QUESTEND





\def\what{\emph{Booleska värden.}}

\QUESTBEGIN

\Task \what~Vilket värde har dessa uttryck?  % Uppgift 13

\Subtask \code{true && true}

\Subtask \code{false && true}

\Subtask \code{true || true}

\Subtask \code{false || true}

\Subtask \code{false || false}

\Subtask \code{true == true}

\Subtask \code{true != false}

\Subtask \code{true > false}

\Subtask \code{true && (1 / 0 > 1)}

\Subtask \code{false && (1 / 0 > 1)}

\SOLUTION

\TaskSolved \what

\SubtaskSolved \code{true}

\SubtaskSolved \code{false}

\SubtaskSolved \code{false}

\SubtaskSolved \code{true}

\SubtaskSolved \code{true}

\SubtaskSolved \code{false}

\SubtaskSolved \code{true}

\SubtaskSolved \code{true}

\SubtaskSolved Undantag kastas: \code{java.lang.ArithmeticException: / by zero}

\SubtaskSolved \code{false}

\QUESTEND





\def\what{\emph{Booleska variabler.}}

\QUESTBEGIN

\Task \what~Vad skrivs ut på rad 2 och 4 nedan?

\begin{REPL}
scala> var monster = false
scala> if (monster) println("akta dig!!!")
scala> monster = true
scala> if (monster) println("akta dig!!!")
\end{REPL}

\SOLUTION

\TaskSolved \what

\begin{itemize}
\item[Rad 2:] Ingenting skrivs ut.
\item[Rad 4:] \code{akta dig!!!}
\end{itemize}


\QUESTEND






\def\what{\emph{Turtle graphics med Kojo.}}

\QUESTBEGIN

\Task \what~På veckans laboration ska du använda Kojo för att verifiera att du kan använda sekvens, alternativ, repetition och abstraktion. Med Kojo kan du rita färgglada figurer med hjälp av ett lättanvänt Scala-bibliotek för \emph{turtle graphics}\footnote{\url{https://en.wikipedia.org/wiki/Turtle_graphics}}. 

Starta Kojo (se appendix \ref{appendix:kojo}). Om du inte redan har svenska menyer: välj svenska i språkmenyn och starta om Kojo.  Skriv in nedan program och tryck på den \emph{gröna} play-knappen. Notera kopplingen mellan satssekvensen och vad som händer i ritfönstret.

\begin{Code}
sudda

fram; höger
fram; vänster
färg(grön)
fram
\end{Code}
\noindent


\Subtask Vad händer om du \emph{inte} börjar programmet med \code{sudda} och kör samma program upprepade gånger? Varför är det bra att börja programmet med \code{sudda}?

\Subtask Skriv kod som ritar en kvadrat enligt bilden nedan.
\vspace{1em}\\\includegraphics[width=0.47\textwidth]{../img/kojo/kvadrat}

\noindent Prova gärna olika sätt att skriva din kod \emph{utan} att resultatet ändras: skriv satser i sekvens på flera rader eller satser i sekvens på samma rad med semikolon emellan; använd blanktecken och blanka rader i koden. Hur vill du gruppera dina satser så att de är lätta för en människa att läsa?
%Prova att ändra på \emph{ordningen} mellan satserna och studera hur resultatet påverkas. Använd den \emph{gula} play-knappen  (programspårning) för att studera exekveringen i detalj. Vad händer du klickar på satser i ditt program och på rutor i programspårningen?


\Subtask Rita en trappa enligt bilden nedan.

\includegraphics[width=0.3\textwidth]{../img/kojo/stairs}

\Subtask Rita valfri bild på valfri bakgrund med hjälp av några av procedurerna i tabellen nedan. Du kan till exempel rita en rosa triangel med lila konturer mot svart bakgrund. % \ref{lab:kojo:kojo-procedures}. 
Försök att underlätta läsbarheten av din kod med hjälp av lämpliga radbrytningar och gruppering av satser. 


\begin{table}[H]
\begin{longtable}{l l}\small
\code|fram(100)| & Paddan går framåt 100 steg (25 om argument saknas).\\
\code|färg(rosa)| & Sätter pennans färg till rosa. \\
\code|fyll(lila)| & Sätter ifyllnadsfärgen till lila. \\
\code|fyll(genomskinlig)| & Gör så att paddan \emph{inte} fyller i något när den ritar. \\
\code|bredd(20)| & Gör så att pennan får bredden 20. \\
\code|bakgrund(svart)| & Bakgrundsfärgen blir svart. \\
\code|bakgrund2(grön,gul)| & Bakgrund med övergång från grönt till gult. \\
\code|pennaNer|  & Sätter ner paddans penna så att den ritar när den går. \\
\code|pennaUpp|  & Sänker paddans penna så att den \emph{inte} ritar när den går. \\
\code|höger(45)|   & Paddan vrider sig 45 grader åt höger. \\
\code|vänster(45)| & Paddan vrider sig 45 grader åt vänster. \\
\code|hoppa|       & Paddan hoppar 25 steg utan att rita. \\
\code|hoppa(100)|  & Paddan hoppar 100 steg utan att rita. \\
\code|hoppaTill(100, 200)| & Paddan hoppar till läget (100, 200) utan att rita. \\
\code|gåTill(100, 200)|    & Paddan vrider sig och går till läget (100, 200). \\
\code|öster|   & Paddan vrider sig så att nosen pekar åt höger. \\
\code|väster|  & Paddan vrider sig så att nosen pekar åt vänster. \\
\code|norr|    & Paddan vrider sig så att nosen pekar uppåt. \\
\code|söder|   & Paddan vrider sig så att nosen pekar neråt. \\
\code|mot(100,200)|   & Paddan vrider sig så att nosen pekar mot läget (100, 200) \\
\code|sättVinkel(90)| & Paddan vrider nosen till vinkeln 90 grader. \\
\end{longtable}
%\label{lab:kojo:kojo-procedures}
%\caption{Några användbara procedurer i Kojo.}
\end{table}

\begin{framed}
\noindent\emph{Tips inför fortsättningen:} Ha gärna både REPL och Kojo igång samtidigt. Då kan du undersöka hur olika kodkonstruktioner fungerar i REPL, medan du stegvis skapar allt större program i editorn i Kojo. Detta sätt att jobba har du nytta av under resten av kursen, både om du använder en texteditor och kompilerar i terminalen, och om du använder en professionell integrerad utvecklingsmiljö. Oavsett vilka andra verktyg du kör är det användbart att ha REPL igång i ett eget fönster som hjälp i den kreativa processen, medan du jagar buggar och medan du lär dig nya koncept. Så fort du undrar hur något fungerar i Scala: fram med REPL och testa!
\end{framed}


\SOLUTION

\TaskSolved \what
 
\SubtaskSolved Genom att börja din Kojo-program med \code{sudda} så startar du exekveringen i samma utgångsläge: en tom rityta \Eng{canvas} där paddan pekar uppåt, pennan är nere och pennans färg är röd.  Då blir det lättare att resonera om vad programmet gör från början till slut, jämfört med om exekveringen beror på resultatet av tidigare exekveringar.


\SubtaskSolved
\begin{Code}
sudda

fram; vänster
fram; vänster
fram; vänster
fram; vänster
\end{Code}


\SubtaskSolved
\begin{Code}
sudda

fram; vänster
fram; höger

fram; vänster
fram; höger

fram; vänster
fram; höger

fram; vänster
\end{Code}


\QUESTEND









\clearpage

\ExtraTasks %%%%%%%%%%%%%%%%%% EXTRAUPPGIFTER



\def\what{\emph{Typ och värde.}}

\QUESTBEGIN

\Task \what~Vilket värde och vilken typ hör till vilket uttryck?  Är du osäker på svaret, testa i REPL.

\begin{ConceptConnections}[0.3\textwidth]
  \code|1.0 + 18          | & 1 & & A & \code|" ": String   | \\ 
  \code|(41 + 1).toDouble | & 2 & & B & \code|19.0: Double    | \\ 
  \code|1.042e42 + 1      | & 3 & & C & \code|57: Int         | \\ 
  \code|12E6.toLong       | & 4 & & D & \code|42.0: Double    | \\ 
  \code|32.toChar.toString| & 5 & & E & \code|48: Int         | \\ 
  \code|'A'.toInt         | & 6 & & F & \code|0: Int          | \\ 
  \code|0.toInt           | & 7 & & G & \code|1.042E42: Double| \\ 
  \code|'0'.toInt         | & 8 & & H & \code|'*': Char       | \\ 
  \code|'9'.toInt         | & 9 & & I & \code|12000000: Long  | \\ 
  \code|'A' + '0'         | & 10 & & J & \code|65: Int         | \\ 
  \code|('A' + '0').toChar| & 11 & & K & \code|'q': Char       | \\ 
  \code|"*!%#".charAt(0)| & 12 & & L & \code|113: Int        | \\ 
\end{ConceptConnections}

\SOLUTION

\TaskSolved \what

\begin{ConceptConnections}
  \code|1.0 + 18          | & 1 & ~~\Large$\leadsto$~~ &  B & \code|19.0: Double    | \\ 
  \code|(41 + 1).toDouble | & 2 & ~~\Large$\leadsto$~~ &  D & \code|42.0: Double    | \\ 
  \code|1.042e42 + 1      | & 3 & ~~\Large$\leadsto$~~ &  G & \code|1.042E42: Double| \\ 
  \code|12E6.toLong       | & 4 & ~~\Large$\leadsto$~~ &  I & \code|12000000: Long  | \\ 
  \code|32.toChar.toString| & 5 & ~~\Large$\leadsto$~~ &  A & \code|" ": String   | \\ 
  \code|'A'.toInt         | & 6 & ~~\Large$\leadsto$~~ &  J & \code|65: Int         | \\ 
  \code|0.toInt           | & 7 & ~~\Large$\leadsto$~~ &  F & \code|0: Int          | \\ 
  \code|'0'.toInt         | & 8 & ~~\Large$\leadsto$~~ &  E & \code|48: Int         | \\ 
  \code|'9'.toInt         | & 9 & ~~\Large$\leadsto$~~ &  C & \code|57: Int         | \\ 
  \code|'A' + '0'         | & 10 & ~~\Large$\leadsto$~~ &  L & \code|113: Int        | \\ 
  \code|('A' + '0').toChar| & 11 & ~~\Large$\leadsto$~~ &  K & \code|'q': Char       | \\ 
  \code|"*!%#".charAt(0)| & 12 & ~~\Large$\leadsto$~~ &  H & \code|'*': Char       | \\ 
\end{ConceptConnections}

%\Subtask \code{1.0 + 18}
%
%\Subtask \code{(41 + 1).toDouble}
%
%\Subtask \code{1.042e42 + 1}
%
%\Subtask \code{12E6.toLong}
%
%\Subtask \code{"gurk" + 'a'}
%
%\Subtask \code{32.toChar.toString}
%
%\Subtask \code{'A'.toInt}
%
%\Subtask \linebreak[0] \code{'0'.toInt}
%
%\Subtask \code{'0'.toInt}
%
%\Subtask \code{'9'.toInt}
%
%\Subtask \code{'A' + '0'}
%
%\Subtask \code{('A' + '0').toChar}
%
%\Subtask \code{"*!%#".charAt(0)}
%%%%%%%%%%%%%%%%%%%%%%%%%%%%%%%%%%%%%%%%%%%%%%%%
%\SubtaskSolved \code{Double, 19}
%
%\SubtaskSolved \code{Double, 42}
%
%\SubtaskSolved \code{Double, 1.042E42}
%
%\SubtaskSolved \code{Long, 12000000}
%
%\SubtaskSolved \code{String, gurka}
%
%\SubtaskSolved \code{String, " "}
%
%\SubtaskSolved \code{Int, 65}
%
%\SubtaskSolved \code{Int, 48}
%
%\SubtaskSolved \code{Int,49}
%
%\SubtaskSolved \code{Int,57}
%
%\SubtaskSolved \code{Int, 113}
%
%\SubtaskSolved \code{Char, 'q'}
%
%\SubtaskSolved \code{Char, '*'}


\QUESTEND




\def\what{\emph{Satser och uttryck.}}

\QUESTBEGIN

\Task \what

\Subtask Vad är det för skillnad på en sats och ett uttryck?

\Subtask Ge exempel på satser som inte är uttryck?

\Subtask Förklara vad som händer för varje evaluerad rad:
\begin{REPL}
scala> def värdeSaknas = ()
scala> värdeSaknas
scala> värdeSaknas.toString
scala> println(värdeSaknas)
scala> println(println("hej"))
\end{REPL}

\Subtask Vilken typ har literalen \code{()}?

\Subtask Vilken returtyp har \code{println}?

\SOLUTION

\TaskSolved \what

\SubtaskSolved  Ett utryck kan evalueras och resulterar då i ett användbart värde. En sats \emph{gör} något (t.ex. skriver ut något), men resulterat inte i något användbart värde.

\SubtaskSolved \code{println()}

\SubtaskSolved 

 Värdesaknas innehåller Unit

 Skriver ut \code{Unit}

 Skriver ut \code{"()"}

 Skriver ut \code{"()"}

 Skriver först ut hej med det innersta anropet och sen \code{()} med det yttre anropet

\SubtaskSolved  \code{Unit}

\SubtaskSolved  \code{Unit}

\QUESTEND



\def\what{\emph{Procedur med parameter.} \TODO}

\QUESTBEGIN

\Task \what~En procedur är en funktion som orsakar en effekt, till exempel en utskrift eller en variabeltilldelning, men som inte returnerar något intressant resultatvärde. \footnote{I Scala är procedurer funktioner som returnerar det \emph{tomma värdet}, vilket skrivs \code{()} och är av typen \code{Unit}. I Java och flera andra språk finns inget tomt värde och man har en specialsyntax för procedurer som använder nyckelordet \code{void}. }

\Subtask Deklarera en förändringsbar variabel \code{highscore} som initieras till 0.

\Subtask Deklarera en procedur \code{updateHighscore} som tar en parameter \code{points} och tilldelar \code{highscore} \TODO ...


\SOLUTION

\TaskSolved \what

\SubtaskSolved 

\QUESTEND





\def\what{\emph{\code{if}\textit{-sats}.}}

\QUESTBEGIN

\Task \what~För varje rad nedan, beskriv vad som skrivs ut.  % Uppgift 18
\begin{REPL}
scala> if (!true) println("sant") else println("falskt")
scala> if (!false) println("sant") else println("falskt")
scala> def singlaSlant = if (math.random > 0.5) "krona" else "klave"
scala> for (i <- 1 to 5) print(s"$i:$singlaSlant ")
\end{REPL}

\SOLUTION

\TaskSolved \what

\begin{enumerate}
\item Utskrift: \code{falskt}
\item Utskrift: \code{sant}
\item Inget skrivs ut, funktionen deklareras men körs ej.
\item Utskrift: code{1:krona 2:klave 3:krona 4:krona 5:klave }
\end{enumerate}

\QUESTEND





\def\what{\emph{\code{if}\textit{-uttryck}.}}

\QUESTBEGIN

\Task  Deklarera följande variabler med nedan initialvärden:  

\begin{REPLnonum}
scala> var grönsak = "gurka"
scala> var frukt = "banan"
\end{REPLnonum}

Ange för varje rad nedan vad uttrycket har för värde och typ:
\begin{REPLnonum}
scala> if (grönsak == "tomat") "gott" else "inte gott" 
scala> if (frukt == "banan") "gott" else "inte gott" 
scala> if (true) grönsak else 42 
scala> if (false) grönsak else 42 
\end{REPLnonum}

\SOLUTION


\TaskSolved \what~Notera typen \code{Any} på de sista två uttrycken.

\begin{REPLnonum}
scala> if (grönsak == "tomat") "gott" else "inte gott"
res0: String = inte gott

scala> if (frukt == "banan") "gott" else "inte gott"
res1: String = gott

scala> if (true) grönsak else 42
res2: Any = gurka

scala> if (false) grönsak else 42
res3: Any = 42
\end{REPLnonum}


\QUESTEND






\def\what{\emph{QUESTTEMPLATE}}

\QUESTBEGIN

\Task \what

\Subtask

\SOLUTION

\TaskSolved \what

\SubtaskSolved 

\QUESTEND




\clearpage

\AdvancedTasks   %%%%%%%%%%%%%%%%%%% FÖRDJUPNINGSUPPGIFTER




\def\what{\emph{Stränginterpolatorn \code{s}.}}

\QUESTBEGIN

\Task \what~Med ett \code{s} framför en strängliteral får man hjälp av kompilatorn att, på ett typsäkert sätt, infoga variabelvärden i en sträng. 
Variablernas namn ska föregås med ett dollartecken, t.ex. \code{s"Hej $namn"}.  
Om man vill evaluera ett uttryck placeras detta inom klammer direkt efter dollartecknet, t.ex.
\code/s"Dubbla längden: ${namn.size * 2}"/  

\Subtask Vad skrivs ut nedan?
\begin{REPL}
scala> val f = "Kim"
scala> val e = "Finkodare"
scala> println(s"Namnet '$f $e' har ${f.size + e.size} bokstäver.")
\end{REPL}

\Subtask Skapa följande utskrifter med hjälp av stränginterpolatorn \code{s} och variablerna \code{f} och \code{e} i föregående deluppgift.
\begin{REPL}
Kim har 3 bokstäver.
Finkodare har 9 bokstäver.
\end{REPL}

\SOLUTION

\TaskSolved \what

\SubtaskSolved 
\begin{REPLnonum}
Namnet 'Kim Finkodare' har 12 bokstäver.
\end{REPLnonum}

\SubtaskSolved 
\begin{REPLnonum}
println(s"$f har  ${f.size} bokstäver.")
println(s"$e har  ${e.size} bokstäver.")
\end{REPLnonum}

\QUESTEND






\def\what{\emph{Flyttalsaritmetik}}

\QUESTBEGIN

\Task \what

\Subtask Vilket är det minsta positiva värdet av typen \code{Double}?

\Subtask Vad är värdet av detta uttryck? Varför blir det så?
\begin{REPL}
scala> Double.MaxValue + Double.MinPositiveValue == Double.MaxValue
\end{REPL}

\SOLUTION

\TaskSolved \what

\SubtaskSolved 

\begin{REPL}
scala> Double.MinPositiveValue
res0: Double = 4.9E-324
\end{REPL}

\SubtaskSolved 

\begin{REPL}
scala> Double.MaxValue + Double.MinPositiveValue == Double.MaxValue
res2: Boolean = true
\end{REPL}

\QUESTEND




\def\what{\emph{Stora tal.}}

\QUESTBEGIN

\Task \what~Om vi vill beräkna $2^{64} -1$ som ett exakt heltal\footnote{\url{https://en.wikipedia.org/wiki/Wheat_and_chessboard_problem}} blir det större än \code{Int.MaxValue}, så vi kan tyvärr inte använda snabba \code{Int}. Till vår räddning: \code{BigInt} 

\Subtask Läs om \code{BigInt} och \code{BigDecimal} på \Scaladoc \\ Notera vad de kan användas till. 

\Subtask Du skapar ett \code{BigInt}-heltal med \code{BigInt(2)} och kan anropa funktionen \code{pow} på en \code{BigInt} med punktnotation. Beräkna $2^{64} -1$ som ett exakt heltal.

\Subtask Vilka nackdelar finns med \code{BigInt} och \code{BigDecimal}?

\SOLUTION

\TaskSolved \what

\SubtaskSolved \code{BigInt} kan användas i stället för \code{Int} vid mycket stora heltal. \code{BigDecimal} kan användas i stället för \code{Double} vid mycket stora decimaltal.

\SubtaskSolved 
\begin{REPL}
scala> BigInt(2).pow(64)
res0: scala.math.BigInt = 18446744073709551616
\end{REPL}

\SubtaskSolved Beräkningar går mycket långsammare och de är lite krångligare att använda.

\QUESTEND





\def\what{\emph{Precedensregler}}

\QUESTBEGIN

\Task \what~Evalueringsordningen kan styras med parenteser. Vilket värde och vilken typ har följande uttryck? 

\Subtask \code{23 + 2 * 2 + (23 + 2) * 2}

\Subtask \code{(-(2 - 42)) / (1 + 1 + 1)}

\Subtask \code{(-(2 - 42)) / (-1)/(1 + 1 + 1)}

\SOLUTION

\TaskSolved \what

\SubtaskSolved \code{77:  Int}

\SubtaskSolved \code{13: Int}

\SubtaskSolved \code{-13: Int}

\QUESTEND






\def\what{\emph{QUESTTEMPLATE}}

\QUESTBEGIN

\Task \what

\Subtask

\SOLUTION

\TaskSolved \what

\SubtaskSolved 

\QUESTEND




\subsection{TODO}

\TODO{SAKERNA NEDAN SKA FLYTTAS/UPPDATERAS/TAS BORT???} 
%%%%%%%%%%%%%%%%%%%%%%%%%%%%%%%%%%%%%%%%%%%%%%%%
%%%%%%%%%%%%%%%%%%%%%%%%%%%%%%%%%%%%%%%%%%%%%%%%
%%%%%%%%%%%%%%%%%%%%%%%%%%%%%%%%%%%%%%%%%%%%%%%%
%%%%%%%%%%%%%%%%%%%%%%%%%%%%%%%%%%%%%%%%%%%%%%%%
%%%%%%%%%%%%%%%%%%%%%%%%%%%%%%%%%%%%%%%%%%%%%%%%
%%%%%%%%%%%%%%%%%%%%%%%%%%%%%%%%%%%%%%%%%%%%%%%%
%%%%%%%%%%%%%%%%%%%%%%%%%%%%%%%%%%%%%%%%%%%%%%%%
%%%%%%%%%%%%%%%%%%%%%%%%%%%%%%%%%%%%%%%%%%%%%%%%
%%%%%%%%%%%%%%%%%%%%%%%%%%%%%%%%%%%%%%%%%%%%%%%%


\ifPreSolution  %%% TODO remove \fi at end of file and break sultions into pieces





\Task Klassen \code{java.lang.Math} och paketobjektet \code{scala.math}. % Uppgift 11
Genom att trycka på tab tagenten kan man se vad som finns i olika paket.

\begin{REPL}
scala> java.    //tryck TAB efter punkten
applet   awt   beans   io   lang   math   net   nio   rmi   security   sql

scala>
\end{REPL}

\Subtask Undersök genom att trycka på Tab-tangenten, vilka funktioner som finns i \code{Math} och \code{math}. Vad heter konstanten $\pi$ i java.lang.Math respektive scala.math?

\begin{REPL}
scala> java.lang.Math.    //tryck TAB efter punkten
scala> scala.math.        //tryck TAB efter punkten
\end{REPL}

\Subtask Undersök dokumentationen för klassen \code{java.lang.Math} här: \\ \url{https://docs.oracle.com/javase/8/docs/api/java/lang/Math.html} \\
Vad gör \code{java.lang.Math.hypot}?

\Subtask Undersök dokumentationen för paketobjektet \code{scala.math} här: \\
\url{http://www.scala-lang.org/api/current/#scala.math.package} \\
Ge exempel på någon funktion i \code{java.lang.Math} som inte finns i \code{scala.math}.

%\TaskSection{Noggrannhet och undantag i aritmetiska uttryck}

\Task Vad händer här? Notera undantag \Eng{exceptions} och noggrannhetsproblem. % Uppgift 12

\Subtask \code{Int.MaxValue} + 1

\Subtask \code{1 / 0}

\Subtask \code{1E8 + 1E-8}

\Subtask \code{1E9 + 1E-9}

\Subtask \code{math.pow(math.hypot(3,6), 2)}

\Subtask \code{1.0 / 0}

\Subtask \code{(1.0 / 0).toInt}

\Subtask \code{math.sqrt(-1)}

\Subtask \code{math.sqrt(Double.NaN)}

\Subtask \code{throw new Exception("PANG!!!")}





\Task \textit{Deklarationer: \code{var}, \code{val}, \code{def}}. Evaluera varje rad nedan i tur och ordning i Scala REPL.  % Uppgift 15
\begin{REPL}[numbers=left, numberstyle=\color{black}\ttfamily\scriptsize\selectfont]
scala> var x = 30
scala> x + 1
scala> x
scala> x = x + 1
scala> x
scala> x == x + 1
scala> val y = 20
scala> y = y + 1
scala> var z = {println("gurka"); 10}
scala> def w = {println("gurka"); 10}
scala> z
scala> z
scala> z = z + 1
scala> w
scala> w
scala> w = w + 1
\end{REPL}

\Subtask För varje rad ovan: förklara för vad som händer.

\Subtask Vilka rader ger kompileringsfel och i så fall vilket och varför?

\Subtask\Pen Vad är det för skillnad på \code{var}, \code{val} och \code{def}?

\Subtask\Pen Tilldela variabeln \code{val even } värdet av ett uttryck som med modulo-operatorn \code
och olikhetsoperatorn \code{!=} testar om ett tal \code{n} är udda.


\Task\Pen \emph{Tilldelningsoperatorer.} Man kan förkorta en tilldelningssats som förändrar en variabel, t.ex. \code{x = x + 1}, genom att använda så kallade tilldelningsoperatorer och skriva \code{x += 1} som betyder samma sak. Rita en ny bild av datorns minne efter varje evaluerad rad nedan. Bilderna ska visa variablers namn, typ och värde.  % Uppgift 16

\begin{REPL}
scala> var a = 40
scala> var b = a + 40
scala> a += 10
scala> b -= 10
scala> a *= 2
scala> b /= 2
\end{REPL}



\Task \emph{Stränginterpolatorn \code{s}.} Man behöver ofta skapa strängar som innehåller variabelvärden. Med ett \code{s} framför en strängliteral får man hjälp av kompilatorn att, på ett typsäkert sätt, infoga variabelvärden i en sträng. Variablernas namn ska föregås med ett dollartecken, t.ex. \code{s"Hej $namn"}.  Om man vill evaluera ett uttryck placeras detta inom klammer direkt efter dollartecknet, t.ex.
\code/s"Dubbla längden: ${namn.size * 2}"/  % Uppgift 17

\begin{REPL}
scala> val f = "Kim"
scala> val e = "Finkodare"
scala> val tot = f.size + e.size
scala> println(s"Namnet '$f $e' har $tot bokstäver.")
scala> println(s"Efternamnet '$e' har ${e.size} bokstäver.")
\end{REPL}

\Subtask Vad skrivs ut ovan?

\Subtask Skapa följande utskrifter med hjälp av stränginterpolatorn \code{s} och lämpliga variabler.
\begin{REPL}
Namnet 'Kim' har 3 bokstäver.
Namnet 'Finkodare' har 9 bokstäver.
\end{REPL}



\Task \code{if}\textit{-sats}.För varje rad nedan; förklara vad som händer.  % Uppgift 18
\begin{REPL}
scala> if (true) println("sant") else println("falskt")
scala> if (false) println("sant") else println("falskt")
scala> if (!true) println("sant") else println("falskt")
scala> if (!false) println("sant") else println("falskt")
scala> def singlaSlant =
scala> 	 if (math.random > 0.5) print(" krona") else print(" klave")
scala> singlaSlant; singlaSlant; singlaSlant
\end{REPL}


\Task \code{if}\textit{-uttryck}. Deklarera följande variabler med nedan initialvärden:  % Uppgift 19

\begin{REPLnonum}
scala> var grönsak = "gurka"
scala> var frukt = "banan"
\end{REPLnonum}

Vad har följande uttryck för värden och typ?

\Subtask \code{if (grönsak == "tomat") "gott" else "inte gott" }

\Subtask \code{if (frukt == "banan") "gott" else "inte gott" }

\Subtask \code{if (frukt.size == grönsak.size) "lika stora" else "olika stora" }

\Subtask \code{if (true) grönsak else frukt }

\Subtask \code{if (false) grönsak else frukt }


\Task \code{for}\textit{-sats}.  Med bakåtpilen \texttt{<-} kan man i en \code{for}-sats ange vilka värden som ska gås igenom i sekvens. Vid varje runda i loopen får en lokal variabel ett nytt värde i sekvensen. % Uppgift 20

\Subtask Vad ger nedan \code{for}-satser för utskrift?

\begin{REPL}
scala> for (i <- 1 to 10) print(i + ", ")
scala> for (i <- 1 until 10) print(i + ", ")
scala> for (i <- 1 to 5) print((i * 2) + ", ")
scala> for (i <- 1 to 92 by 10) print(i + ", ")
scala> for (i <- 10 to 1 by -1) print(i + ", ")
\end{REPL}

\Subtask Skriv en \code{for}-sats som ger följande utskrift:
\begin{REPLnonum}
A1, A4, A7, A10, A13, A16, A19, A22, A25, A28, A31, A34, A37, A40, A43,
\end{REPLnonum}

\Task Repetition med metoden \code{foreach}. Efter framåtpilen \texttt{=>} (se nedan) anges vad som ska hända för varje element som gås igenom sekventiellt. Vid varje runda i loopen får en lokal variabel ett nytt värde i sekvensen.   % Uppgift 21

\Subtask Vad ger nedan satser för utskrifter?

\begin{REPL}
scala> (9 to 19).foreach{i => print(i + ", ")}
scala> (1 until 20).foreach{i => print(i + ", ")}
scala> (0 to 33 by 3).foreach{i => print(i + ", ")}
\end{REPL}

\Subtask Använd \code{foreach} och skriv ut följande:
\begin{REPLnonum}
B33, B30, B27, B24, B21, B18, B15, B12, B9, B6, B3, B0,
\end{REPLnonum}

\Task \code{while}\textit{-sats}. En sats eller ett block med satser upprepas så länge ett villkor är sant.  % Uppgift 22

\Subtask Vad ger nedan satser för utskrifter?
\begin{REPL}
scala> var i = 0
scala> while (i < 10) { println(i); i = i + 1 }
scala> var j = 0; while (j <= 10) { println(j); j = j + 2 }; println(j)
\end{REPL}

\Subtask Skriv en \code{while}-sats som ger följande utskrift. Använd en variabel \code{k} som initialiseras till 1.
\begin{REPLnonum}
A1, A4, A7, A10, A13, A16, A19, A22, A25, A28, A31, A34, A37, A40, A43,
\end{REPLnonum}

\Subtask\Pen Vilken av \code{for}, \code{while} och \code{foreach} är kortast att skriva om man vill repetera mer än en sats 100 gånger? Vilken tycker du är lättast att läsa?

\Task \textit{Slumptal}. Undersök vad dokumentationen säger om funktionen \code{scala.math.random}:\\  % Uppgift 23
\url{http://www.scala-lang.org/api/current/#scala.math.package}

\Subtask\Pen Vilken typ har värdet som returneras av funktionen \code{random}?

\Subtask\Pen Vilket är det minsta respektive största värde som kan returneras?

\Subtask\Pen Är \code{random} en \textit{äkta} funktion \Eng{pure function} i matematisk mening?

\Subtask Anropa funktionen \code{math.random} upprepade gånger och notera vad som händer. Använd pil-upp-tangenten.
\begin{REPLnonum}
scala> math.random
\end{REPLnonum}


\Subtask Vad händer? Använd \textit{pil-upp} och kör nedan \code{for}-sats flera gånger. Förklara vad som sker.

\begin{REPLnonum}
scala> for (i <- 1 to 20) println((math.random * 3 + 1).toInt)
\end{REPLnonum}

\Subtask Skriv en \code{for}-sats som skriver ut 100 slumpmässiga heltal från 0 till och med 9 på var sin rad.

\begin{REPLnonum}
scala> for (i <- 1 to 100) println(???)
\end{REPLnonum}

\Subtask Skriv en \code{for}-sats som skriver ut 100 slumpmässiga heltal från 1 till och med 6 på samma rad.

\begin{REPLnonum}
scala> for (i <- 1 to 100) print(???)
\end{REPLnonum}


\Subtask Använd \textit{pil-upp} och kör nedan \code{while}-sats flera gånger. Förklara vad som sker.

\begin{REPLnonum}
scala> while (math.random > 0.2) println("gurka")
\end{REPLnonum}

\Subtask Ändra i \code{while}-satsen ovan så att sannolikheten ökar att riktigt många strängar ska skrivas ut.

\Subtask Förklara vad som händer nedan.
\begin{REPL}
scala> var slumptal = math.random
scala> while (slumptal > 0.2) { println(slumptal); slumptal = math.random }
\end{REPL}

\Task\Pen \textit{Logik och De Morgans Lagar}. Förenkla följande uttryck. Antag att \code{poäng} och \code{highscore} är heltalsvariabler medan \code{klar} är av typen \code{Boolean}.
  % Uppgift 24

\Subtask \code{poäng > 100 && poäng > 1000}

\Subtask \code{poäng > 100 || poäng > 1000}

\Subtask \code{!(poäng > highscore)}

\Subtask \code{!(poäng > 0 && poäng < highscore) }

\Subtask \code{!(poäng < 0 || poäng > highscore) }

\Subtask \code{klar == true}

\Subtask \code{klar == false}


\clearpage

\ExtraTasks

\Task \textit{Slumptal}.

\Subtask Ersätt \code{???} nedan med literaler så att \code{tärning} returnerar ett slumpmässigt heltal mellan 1 och 6.
\begin{REPLnonum}
scala> def tärning = (math.random * ??? + ???).toInt
\end{REPLnonum}

\Subtask Ersätt \code{???} med literaler så att \code{rnd} blir ett decimaltal med max en decimal mellan 0.0 och 1.0.
\begin{REPLnonum}
scala> def rnd = math.round(math.random * ???) / ???
\end{REPLnonum}

\Subtask Vad blir det för skillnad om \code{math.round} ersätts med \code{math.floor} ovan? (Se dokumentationen av \code{java.lang.Math.round} och \code{java.lang.Math.floor}.)

\Task Undersök vad som finns i paketet \code{scala.math} genom att studera dess dokumentation: \href{http://www.scala-lang.org/api/current/#scala.math.package}{www.scala-lang.org/api/current/\#scala.math.package} och gör några matematiska beräkningar i REPL som använder olika funktioner i \code{math}-paketet.

\Task\Pen Antag att du byter plats mellan satsen efter villkoret och satsen efter \code{else} i \code{if}-satsen nedan. Hur kan du ändra i villkoret så att det ändå skrivs ut samma sak som före bytet?
\begin{Code}
if (x == 42) println("the meaning of it all") else println(":(")
\end{Code}

\Task\Pen Rita en ny bild av datorns minne efter varje evaluerad rad nedan. Bilderna ska visa variablers namn, typ och värde.
\begin{REPL}
scala> var x = 42
scala> var y = x + 1
scala> x += -1
scala> y -= 1
\end{REPL}

\Task Skapa med hjälp av \code{while} några olika oändliga loopar som skriver ut olika saker vid varje loop-runda.

\Task Hitta på några egna övningar för att träna mer på De Morgans lagar.



\clearpage

\AdvancedTasks

\Task Läs om moduloräkning här \href{https://en.wikipedia.org/wiki/Modulo\_operation}{en.wikipedia.org/wiki/Modulo\_operation} och undersök hur det blir med olika tecken (positivt resp. negativt) på divisor och dividend.



\Task Läs om identifierare i Scala och speciellt \emph{literal identifiers} här: \url{http://www.artima.com/pins1ed/functional-objects.html#6.10}.

\Subtask Förklara vad som händer nedan:
\begin{REPLnonum}
scala> val `konstig val` = 42
scala> println(`konstig val`)
\end{REPLnonum}

\Subtask Scala och Java har olika uppsättningar med reserverade ord. På vilket sätt kan ''backticks'' vara använbart med anledning av detta?


\Task Sök upp dokumentationen för \code{java.lang.Integer}.

\Subtask Undersök i REPL hur metoderna \code{toBinaryString} och \code{toHexString} fungerar.

\Subtask Vad betyder literalen \code{0x2a}?

\Task Typannoteringar skapas genom att i ett uttryck placera ett kolon följt av en typ, vid behov  omslutet av en parentes. Skapa ett större uttryck med typannoteringar och försök få kompilatorn att kontrollera typen på intressanta ställen. Märk att typannoteringar också ibland kan användas för att konvertera mellan numeriska typer.


\Task Förklara vad som händer nedan:
\begin{REPL}
scala> var i = 42
scala> i += 1
scala> i *= 2
scala> i /= 3
\end{REPL}


\Task Läs om BigInt och BigDecimal här: \href{http://alvinalexander.com/scala/how-to-use-large-integer-decimal-numbers-in-scala-bigint-bigdecimal}{alvinalexander.com/scala/how-to-use-large-integer-decimal-numbers-in-scala-bigint-bigdecimal} och prova att skapa riktigt stora tal med hjälp av metoden \code{pow} på BigInt och tal med riktigt många decimaler med BigDecimal dess metod \code{pow}.

\Task Sök upp dokumentationtionen för \code{java.lang.Math.multiplyExact} och läs om vad den metoden gör.

\Subtask Vad händer här?
\begin{REPLnonum}
scala> Math.multiplyExact(2, 42)
scala> Math.multiplyExact(Int.MaxValue, Int.MaxValue)
\end{REPLnonum}

\Subtask\Pen Varför kan man vilja använda \code{java.lang.Math.multiplyExact} i stället för ''vanlig'' multiplikation?



\Subtask\Pen Sök med Ctrl+F i webbläsaren och efter förekomster av texten \textit{''overflow''} i javadoc för klassen \code{java.lang.Math} i JDK 8. Vad är ''overflow''? Vilka metoder finns i \code{java.lang.Math} som hjälper dig att upptäcka om det blir overflow?

\Task Använda Scala REPL för att undersöka konstanterna nedan. Vilket av dessa värden är negativt? Vad kan man ha för praktisk nytta av dessa värden i ett program som gör flyttalsberäkningar?

\Subtask \code{java.lang.Double.MIN_VALUE}

\Subtask \code{scala.Double.MinValue}

\Subtask \code{scala.Double.MinPositiveValue}

\Task För typerna \code{Byte}, \code{Short}, \code{Char}, \code{Int}, \code{Long}, \code{Float}, \code{Double}: Undersök hur många bitar som behövs för att representera varje typs omfång? \\*
\textit{Tips:} Några användbara uttryck: \\*
 \code{Integer.toBinaryString(Int.MaxValue + 1).size} \\*
 \code{Integer.toBinaryString((math.pow(2,16) - 1).toInt).size} \\*
 \code{1 + math.log(Long.MaxValue)/math.log(2)}
Se även språkspecifikationen för Scala, kapitlet om heltalsliteraler: \\
\url{http://www.scala-lang.org/files/archive/spec/2.11/01-lexical-syntax.html#integer-literals}

\Subtask Undersök källkoden för paketobjektet \code{scala.math} här: \\
\url{https://github.com/scala/scala/blob/v2.11.7/src/library/scala/math/package.scala} \\
Hur många olika överlagrade varianter av funktionen \code{abs} finns det och för vilka parametertyper är den definierad?

\Task Läs mer om stränginterpolatorer här:\\ \href{http://docs.scala-lang.org/overviews/core/string-interpolation.html}{docs.scala-lang.org/overviews/core/string-interpolation.html} \\ Hur kan du använda \code{f}-interpolatorn för att göra följande utskrift i REPL? Byt ut \code{???} mot lämpliga tecken.
\begin{REPLnonum}
scala> val g: Double = 1 / 3.0
scala> val s: String = f"Gurkan är ??? meter lång"
scala> println(s)
Gurkan är 0.333 meter lång
\end{REPLnonum}

\fi %%% TODO fix solutions




%!TEX encoding = UTF-8 Unicode
%!TEX root = ../labs.tex

\Lab{\LabWeekONE}
%\externaldocument{compendium}
\begin{Goals}
%!TEX encoding = UTF-8 Unicode
%!TEX root = ../compendium2.tex

\item Kunna tillämpa och kombinera principerna sekvens, alternativ, repetition, och abstraktion i skapandet av egna program om minst 20 rader kod.
\item Kunna förklara vad ett program gör i termer av sekvens, alternativ, repetition, och abstraktion.
\item Kunna formatera egna program så att de blir lätta att läsa och förstå.
\item Kunna förklara vad en variabel är och kunna deklarera oföränderliga och förändringsbara variabler, samt göra tilldelningar.
\item Kunna genomföra upprepade varv i cykeln \emph{editera-exekvera-felsöka/förbättra} för att stegvis bygga upp allt mer utvecklade program.

\end{Goals}

\begin{Preparations}
\item Repetera veckans föreläsningsmaterial.
\item \DoExercise{\ExeWeekONE}{01}%Gör övning {\tt \ExeWeekONE} i kapitel \ref{exe:W01}.
\item Läs om Kojo i appendix \ref{appendix:kojo}. Kojo Desktop är förinstallerat på LTH:s datorer; om du vill installera Kojo Desktop på din egen dator, följ instruktionerna i \ref{appendix:ide:kojo:install}.
\item Läs igenom hela laborationen nedan. Fundera på möjliga lösningar till de uppgifter som är markerade med en penna i marginalen.
\item Hämta given kod via \href{https://github.com/lunduniversity/introprog/tree/master/workspace/}{kursen github-plats} eller via hemsidan under \href{https://cs.lth.se/pgk/download/}{Download}.
% \item Ladda hem och studera översiktligt detta dokument (25 sidor, det räcker att du bläddrar igenom dokumentet och får en uppfattning om hur Kojo kan användas): \\ ''Introduction to Kojo'' \url{http://www.kogics.net/kojo-ebooks#intro}
\end{Preparations}

\subsection{Obligatoriska uppgifter}

Om det förekommer en penna i marginalen ska du anteckna något inför redovisningen.


%%%%%%%%%%%%%%NEDAN ÄR FLYTTAT TILL ÖVNING 1 FÖR ATT GÖRA TYDLIGARE KOPPLING MELLAN LABBAR OCH ÖVN
%\Task \textit{Sekvens}.
%
%\Subtask Starta Kojo. Om du inte redan har svenska menyer: välj svenska i språkmenyn och starta om Kojo.  Skriv in nedan program och tryck på den \emph{gröna} play-knappen.
%
%\begin{Code}
%sudda
%
%fram; höger
%fram; vänster
%färg(grön)
%fram
%\end{Code}
%\noindent
%%Genom att börja din Kojo-program med \code{sudda} så startar du exekveringen i samma utgångsläge: en tom canvas där paddan pekar uppåt, pennan är nere och pennans färg är röd.
%%Då blir det lättare att resonera om vad programmet gör från början till slut, jämfört med om exekveringen beror på resultatet av tidigare exekveringar.
%
%\Subtask\Pen Vad händer om du \emph{inte} börjar programmet med \code{sudda} och kör samma program upprepade gånger? Varför är det bra att börja programmet med \code{sudda}?
%
%\Subtask Rita en kvadrat enligt bilden nedan.
%\vspace{1em}\\\includegraphics[width=0.45\textwidth]{../img/kojo/kvadrat}
%
%\Subtask Prova olika sätt att skriva din kod \emph{utan} att resultatet ändras: skriv satser i sekvens på flera rader eller satser i sekvens på samma rad med semikolon emellan; använd blanktecken och blanka rader i koden. Hur vill du gruppera dina satser så att de är lätta för en människa att läsa?
%
%\Subtask Prova att ändra på \emph{ordningen} mellan satserna och studera hur resultatet påverkas. Använd den \emph{gula} play-knappen  (programspårning) för att studera exekveringen i detalj. Klicka på satser i ditt program och på rutor i programspårningen och se vad som händer.
%
%
%\Subtask Rita en trappa enligt bilden nedan.
%
%\includegraphics[width=0.2\textwidth]{../img/kojo/stairs}
%
%\Subtask Rita valfri bild på valfri bakgrund med hjälp av några av procedurerna i tabellen nedan. Du kan till exempel rita en rosa triangel med lila konturer mot svart bakgrund. % \ref{lab:kojo:kojo-procedures}.
%Försök att underlätta läsbarheten av din kod med hjälp av lämpliga radbrytningar och gruppering av satser. Undersök hur ordningen av satserna i din kod påverkar resultatet.
%
%
%
%\begin{table}[H]
%\begin{tabular}{l l}\small
%\code|fram(100)| & Paddan går framåt 100 steg (25 om argument saknas).\\
%\code|färg(rosa)| & Sätter pennans färg till rosa. \\
%\code|fyll(lila)| & Sätter ifyllnadsfärgen till lila. \\
%\code|fyll(genomskinlig)| & Gör så att paddan \emph{inte} fyller i något när den ritar. \\
%\code|bredd(20)| & Gör så att pennan får bredden 20. \\
%\code|bakgrund(svart)| & Bakgrundsfärgen blir svart. \\
%\code|bakgrund2(grön,gul)| & Bakgrund med övergång från grönt till gult. \\
%\code|pennaNer|  & Sätter ner paddans penna så att den ritar när den går. \\
%\code|pennaUpp|  & Sänker paddans penna så att den \emph{inte} ritar när den går. \\
%\code|höger(45)|   & Paddan vrider sig 45 grader åt höger. \\
%\code|vänster(45)| & Paddan vrider sig 45 grader åt vänster. \\
%\code|hoppa|       & Paddan hoppar 25 steg utan att rita. \\
%\code|hoppa(100)|  & Paddan hoppar 100 steg utan att rita. \\
%\code|hoppaTill(100, 200)| & Paddan hoppar till läget (100, 200) utan att rita. \\
%\code|gåTill(100, 200)|    & Paddan vrider sig och går till läget (100, 200). \\
%\code|öster|   & Paddan vrider sig så att nosen pekar åt höger. \\
%\code|väster|  & Paddan vrider sig så att nosen pekar åt vänster. \\
%\code|norr|    & Paddan vrider sig så att nosen pekar uppåt. \\
%\code|söder|   & Paddan vrider sig så att nosen pekar neråt. \\
%\code|mot(100,200)|   & Paddan vrider sig så att nosen pekar mot läget (100, 200) \\
%\code|sättVinkel(90)| & Paddan vrider nosen till vinkeln 90 grader. \\
%\end{tabular}
%%\label{lab:kojo:kojo-procedures}
%%\caption{Några användbara procedurer i Kojo.}
%\end{table}


%%% NEDAN ÄR BORTTAGEN FÖR ATT MINSKA MÄNGDEN ARBETE

%\Subtask \emph{Rita och mät}.
%\begin{itemize}[noitemsep]
%\item Börja ditt program med dessa satser:\\ \code{sudda; axesOn; gridOn; sakta(0); osynlig}
%\item Rita sedan en kvadrat som har 444 längdenheter i omkrets.
%\item Ta fram linjalen med höger-klick i ritfönstret och mät så exakt du kan hur lång diagonalen i kvadraten är. Skriv ner resultatet. \\ \emph{Tips:} Du kan zooma med mushjulet om du håller nere Ctrl-knappen. Du kan flytta linjalen om du klick-drar på linjalens skalstreck. Du kan vrida linjalen om du klickar på skalstrecken och håller nere Shift-tangenten.
%\item Kontrollera med hjälp av \code{math.hypot} och \code{println} vad det exakta svaret är. Skriv ner svaret med 3 decimalers noggrannhet. Du kan t.e.x. använda REPL i ett terminalfönster bredvid, eller öppna ett nytt extra Kojo-fönster i Arkiv-menyn, eller lägga in utskrifterna sist i ditt befintliga program. Utskrifter med \code{println} i Kojo sker i utdatafönstret.
%\end{itemize}
%
%\Subtask Rita en liksidig triangel med sidan 300 längdenheter genom att ge lämpliga argument till \code{fram} och \code{höger}. Vinklar anges i grader.
%
%\Subtask\Checkpoint Visa dina resultat för en handledare och diskutera hur uppgifterna ovan illustrerar principen om sekvens.

\vspace{1em}

% \Task Läs om hur du gör grafikprogram med Kojo i Appendix \ref{appendix:kojo} och övning {\tt \ExeWeekONE} i kapitel \ref{exe:W01}.


\Task \textit{Sekvens och repetition}. Rita en kvadrat med hjälp av \code+upprepa(n){ ??? }+ där du ersätter \code{n} med antalet repetitioner och \code{???} med de satser som ska repeteras.

%\Subtask Om du kör Kojo Desktop: Prova att köra ditt program med den \emph{gula} play-knappen för programspårning. Studera exekveringssekvensen. Klicka på anropen i programspårningsfönstret och studera markeringarna i ritfönstret.





\Task \textit{Variabel och repetition}.

\Subtask Funktionen \code{System.currentTimeMillis} ingår i Javas standardbibliotek och ger ett heltal av typen \code{Long} med det nuvarande antalet millisekunder sedan midnatt den första januari 1970.  Med Kojo-proceduren \code{sakta(0)} blir det ingen fördröjning när paddan ritar och utritningen sker så snabbt som möjligt. Prova nedan program och förklara vad som händer.
\begin{Code}
sakta(0)
val n = 800 * 4
val t1 = System.currentTimeMillis
upprepa(n){ upprepa(4){ fram; höger } }
val t2 = System.currentTimeMillis
println(s"$n kvadratvarv tog ${t2 - t1} millisekunder")
\end{Code}
\noindent Om du kör Kojo Desktop är det bra att börja programmet med \code{sudda}. (Varför?)

\Subtask\Pen Anteckna ungefär hur många kvadratvarv per sekund som paddan kan rita när den är som snabbast. Kör flera gånger eftersom den virtuella maskinen behöver ''värmas upp'' för att maskinkoden ska optimeras. Vissa körningar kan gå långsammare om skräpsamlaren behöver lägga tid på att frigöra minne.

\Subtask\Pen Vad har variablerna i koden ovan för namn? Vad har variablerna för värden?

\Subtask Rita en kvadrat igen, men nu med hjälp av en \code{while}-sats och en loopvariabel. %Studera exekveringen med programspårning (den gula play-knappen).

\begin{Code}
sakta(100)
var i = 0
while (???) { fram; höger; i = ??? }
\end{Code}

\Subtask\Pen Vad är det för skillnad på variabler som deklareras med \code{val} respektive \code{var}?

\Subtask Rita en kvadrat igen, men nu med hjälp av en \code{for}-sats. Skriv ut värdet på den lokala variabeln \code{i} i varje loop-runda.

\begin{Code}
for (i <- 1 to ???) { ??? }
\end{Code}

\Subtask\Pen Går det att tilldela variabeln \code{i} ett nytt värde i loopen?

\Subtask\Pen Går det att referera till namnet \code{i} utanför loopen?


\Subtask Rita en kvadrat igen, men nu med hjälp av \code{foreach}. Skriv ut loopvariabelns värde i varje runda.

\begin{Code}
(1 to ???).foreach{ i => ??? }
\end{Code}

%\Subtask\Pen För var och en av de fyra repetitionskonstruktionerna du sett ovan, \code{upprepa}, \code{while}, \code{for} och \code{foreach}: skriv kod med penna på papper som skriver ut de första 100 jämna heltalen med blanktecken emellan: \code{2 4 6 8 10 12 ...} etc.\\ Vilken typ av loop tycker du är enklast att använda i detta fall?


\Task \textit{Abstraktion}.

\Subtask Använd en repetition för att abstrahera nedan sekvens, så att programmet blir kortare:
\begin{Code}
fram; höger; hoppa; fram; vänster; hoppa; fram; höger;
hoppa; fram; vänster; hoppa; fram; höger; hoppa; fram;
vänster; hoppa; fram; höger; hoppa; fram; vänster; hoppa;
fram; höger; hoppa; fram; vänster; hoppa
\end{Code}

%\Subtask\Pen Sök på nätet efter ''DRY principle programming'' och beskriv med egna ord vad DRY betyder och varför det är en viktig princip.

\Subtask Definiera en egen procedur som heter \code{kvadrat} med hjälp av nyckelordet \code{def} som vid anrop ritar en kvadrat med hjälp av en \code{for}-loop.

\begin{Code}
def kvadrat = for (???) {???}
\end{Code}


\Subtask Anropa din abstraktion efter att den deklarerats och efter att du exekverat:\\\code{sakta(100)}


\Subtask Anropa din abstraktion inuti en \code{for}-loop så att paddan ritar en stapel som är 10 kvadrater hög enligt bilden nedan.

\begin{figure}
  \begin{multicols}{2}

  \includegraphics[scale=0.6]{../img/kojo/square-column}

  \columnbreak

  \begin{Code}
  def kvadrat = for (???) {???}
  for (???) {???}
  \end{Code}

  \end{multicols}
  \caption{En kvadratstapel.\label{fig:kojo-lab:column}}
\end{figure}

\Subtask %Kör ditt program med den \emph{gula} play-knappen. 
Studera hur anrop av proceduren \code{kvadrat} påverkar exekveringssekvensen av dina satser genom att göra lämpliga utskrifter så att du kan se när olika delar av koden exekveras. Vid vilka punkter i programmet sker ett ''hopp'' i sekvensen i stället för att efterföljande sats exekveras?  Använd lämpligt argument till \code{sakta} för att du ska hinna studera exekveringen.


\Subtask Rita samma bild med 10 staplade kvadrater (se bild \ref{fig:kojo-lab:column} på sidan \pageref{fig:kojo-lab:column}), men nu \emph{utan} att använda abstraktionen \code{kvadrat} -- använd i stället en nästlad repetition (alltså en upprepning inuti en upprepning). Vilket av de två sätten (med och utan abstraktionen \code{kvadrat}) är lättast att läsa? %\emph{Tips:} Varje gång du trycker på någon av play-knapparna, sparas ditt program. Du kan se dina sparade program om du klickar på \emph{Historik}-fliken. Du kan också stega bakåt och framåt i historiken med de blå pilarna bredvid play-knapparna.

\Subtask Generalisera din abstraktion \code{kvadrat} genom att ge den en parameter \code{sida: Double} som anger kvadratens storlek. Rita flera kvadrater i likhet med bild \ref{fig:kojo-lab:resize} på sidan \pageref{fig:kojo-lab:resize}).

\begin{figure}[H]
\includegraphics{../img/kojo/square-param}
  \caption{Olika stora kvadrater.\label{fig:kojo-lab:resize}}

\end{figure}



%\Subtask\Pen%\Checkpoint
%Se över ditt program i föregående uppgift och säkerställ att det är lättläst och följer en struktur som börjar med alla definitioner i logisk ordning och därefter fortsätter med huvudprogrammet.
%%Diskutera ditt program med en handledare.



%\Subtask\Pen Spara ditt program i en fil men lämpligt namn och ha programmet redo när det är din tur att redovisa vad du gjort under laborationen.
%Anteckna några åtgärder du vidtagit för att göra programmet mer lättläst.







\Task \emph{Alternativ.} \label{kojo:alt}

\Subtask Kör programmet nedan. Förklara vad som händer. %Använd den gula play-knappen för att studera exekveringen.

\begin{Code}
sakta(5000)

def move(key: Int): Unit = {
  println("key: " + key)
  if (key == 87) fram(10)
  else if (key == 83) fram(-10)
}

move(87); move('W'); move('W')
move(83); move('S'); move('S'); move('S')
\end{Code}

\Subtask \label{subtask:keypress}  Kör programmet nedan. Notera \code{activateCanvas()} för att du ska slippa klicka i ritfönstret innan du kan styra paddan. Anropet \code{onKeyPress(move)} gör så att \code{move} kommer att anropas då en tangent trycks ned. Lägg till kod i \code{move} som gör att tangenten A ger en vridning moturs med 5 grader medan tangenten D ger en vridning medurs 5 grader. Med \code{onKeyPress} bestämmer man vilken procedur som ska köras vid tangenttryck.

\begin{Code}
sakta(0); activateCanvas()

def move(key: Int): Unit = {
  println("key: " + key)
  if (key == 'W') fram(10)
  else if (key == 'S') fram(-10)
}

onKeyPress(move)
\end{Code}



%\Subtask Spara ditt program i en fil men lämpligt namn och ha programmet redo när det är din tur att redovisa vad du gjort under laborationen.


\subsection{Kontrollfrågor}\Checkpoint

\noindent Repetera teorin för denna vecka och var beredd på att kunna svara på dessa frågor när det blir din tur att redovisa vad du gjort under laborationen:

\begin{enumerate}
\item Vad innebär sekventiell exekvering av satser?
\item Vad är skillnaden mellan en sats och ett uttryck?
\item Vad är skillnaden mellan en procedur och en funktion?
\item Spelar ordningen mellan argument någon roll vid anrop av en funktion med flera parametrar?
\item Vad är en variabel? Ge exempel på deklaration, initialisering och tilldelning av variabler, samt användning av variabler i uttryck.
\item Vad är ett logiskt uttryck? Ge exempel på användning av logiska uttryck.
\item Vad är abstraktion? Ge exempel på användning av abstraktion.
\item Vad är nyttan med abstraktion?
\item Hur deklareras och initialiseras en variabel vars värde är förändringsbart?
\item Hur deklareras och initialiseras en variabel vars värde är oföränderligt?
\item Är det ett körtidsfel eller kompileringsfel att tilldela en oföränderlig variabel ett nytt värde?
\item Ange vilken av \code{for} och \code{while} som är lämpligast i dessa fall:
\begin{itemize}[noitemsep, nolistsep]
\item[A.] Summera de hundra första heltalen.
\item[B.] Räkna antal tecken i en sträng innan första blanktecken.
\item[C.] Dra 100 slumptal mellan 1 och 6 och summera de tal som är mindre än 3.
\item[D.] Summera de första heltalen från 1 och uppåt tills summan är minst 100.
\end{itemize}
\end{enumerate}


\subsection{Frivilliga extrauppgifter}

\noindent Gör i mån intresse och träningsbehov nedan uppgifter i valfri ordning.

\Task \emph{Abstraktion och generalisering}.

\Subtask Skapa en abstraktion \code{def stapel = ???} som använder din abstraktion \code{kvadrat}.

\Subtask Du ska nu \emph{generalisera} din procedur så att den inte bara kan rita exakt 10 kvadrater i en stapel. Ge proceduren \code{stapel} en parameter \code{n} som styr hur många kvadrater som ritas.
\begin{Code}
def kvadrat = ???
def stapel(n: Int) = ???

sakta(100)
stapel(42)
\end{Code}



\Subtask Rita nedan bild med hjälp av abstraktionen \code{stapel}. Det är totalt 100 kvadrater och varje kvadrat har sidan 25. \emph{Tips:} Med ett negativt argument till proceduren \code{hoppa} kan du få sköldpaddan att hoppa baklänges utan att rita, t.ex. \code{hoppa(-10*25)}

\includegraphics[width=0.3\textwidth]{../img/kojo/square-grid}

\Subtask Generalisera dina abstraktioner \code{kvadrat} och \code{stapel} så att man kan påverka storleken på kvadraterna som ritas ut.

\Subtask Skapa en abstraktion \code{rutnät} med lämpliga parametrar som gör att man kan rita rutnät med olika stora kvadrater och olika många kvadrater i både x- och y-led.

\Subtask Generalisera dina abstraktioner \code{kvadrat} och \code{stapel} så att man kan påverka fyllfärgen och pennfärgen för kvadraterna som ritas ut. 

Färgerna i Kojo är av typen \code{java.awt.Color}. Typen är tillgänglig under namnet \code{Color} eftersom namnet är importerat med \code{export java.awt.Color} i filen \code{kojo.scala} (mer om nyckelorden \code{export} och \code{import} läsvecka 4).


\Task \emph{Växling med booleska värden.}

\Subtask Bygg vidare på programmet i uppgift \ref{kojo:alt} och lägg till nedan kod i början av programmet. Lägg även till kod som gör så att om man trycker på tangenten G så sätts rutnätet omväxlande på och av. Observera att det är exakt \emph{en} procedur som anropas vid \code{onKeyPress}.

\begin{Code}
var isGridOn = false

def toggleGrid =
  if (isGridOn) {
    gridOff
    isGridOn = false
  } else {
    gridOn
    isGridOn = true
  }
\end{Code}

\Subtask Gör så att när man trycker på tangenten X så sätter man omväxlande på och av koordinataxlarna. Använd en variabel \code{isAxesOn} och definiera en abstraktion \code{toggleAxes} som anropar \code{axesOn} och \code{axesOff} på liknande sätt som i föregående uppgift.


\Task \emph{Repetition.}~Skriv en procedur \code{randomWalk} med detta huvud: \\
\code{def randomWalk(n: Int, maxStep: Int, maxAngle: Int): Unit}\\ som gör så att paddan tar \code{n} steg av slumpmässig längd mellan \code{0} och \code{maxStep}, samt efter varje steg vrider sig åt vänster en slumpmässig vinkel mellan \code{0} och \code{maxAngle}. Anropa din procedur med olika argument och undersök hur dess värden påverkar bildens utseende. \emph{Tips:} Uttrycket \code{math.random() * 100} ger ett tal från 0 till (nästan) 100. Du kan styra hur långsamt paddan ritar genom anrop av \code{sakta(???)} (prova dig fram till något  lämpligt heltalsargument i stället för \code{???}).
\vspace{2em}\\\includegraphics[width=\textwidth]{../img/kojo/random-walk.png}


\Task \emph{Variabler, namngivning och formatering.}

\Subtask Klistra in nedan konstigt formatterade program \emph{exakt} som det står med blanktecken, indragningar och radbrytningar. Kör programmet och förklara vad som händer.

\begin{figure}[H]
\begin{Code}
// Ett konstigt formaterat program med en del konstiga namn.

def gurka(x: Double,
y: Double, namn: String,
typ: String,
värde:String) = {
val tomat = 15
val h = 30
hoppaTill(x,y)
norr
skriv(namn+": "+typ)
hoppaTill(x+tomat*(namn.size+typ.size),y)
skriv(värde); söder; fram(h); vänster
fram(tomat * värde.size); vänster
fram(h); vänster
fram(tomat * värde.size); vänster }
sudda; färg(svart); val s = 130
val h = 40
var x = 42; gurka(10, s-h*0, "x","Int", x.toString)
var y = x; gurka(10, s-h*1, "y","Int", y.toString)
x = x + 1; gurka(10, s-h*2, "x","Int", x.toString)
gurka(10, s-h*3, "y","Int", y.toString); osynlig
\end{Code}
\end{figure}

\Subtask\Pen Skriv ner namnet på alla variabler som förekommer i programmet.

\Subtask\Pen Vilka av dessa variabler är lokala?

\Subtask\Pen Vilka av dessa variabler kan förändras efter initialisering?

\Subtask\Pen Föreslå tre förändringar av programmet ovan (till exempel namnbyten) som gör att det blir lättare att läsa och förstå.

\Subtask Gör sök-ersätt av \code{gurka} till ett bättre namn. \emph{Tips:} undersök kontextmenyn i editorn i Kojo genom att högerklicka. Använd kortkommandot för Sök/Ersätt.

\Subtask Gör automatisk formatering av koden med hjälp av lämpligt kortkommando. Notera skillnaderna. Vilka autoformateringar gör programmet lättare att läsa? Vilka manuella formateringar tycker du bör göras för att öka läsbarheten? Ge funktionen \code{gurka} ett bättre namn.  Diskutera läsbarheten med en handledare.



\Task \label{task:measuretime} \emph{Tidmätning.} Hur snabb är din dator?

\Subtask \label{task:timer} Skriv in koden nedan i Kojos editor och kör upprepade gånger med den gröna play-knappen. Tar det lika lång tid varje gång? Varför?

\begin{Code}
object timer {
  def now: Long = System.currentTimeMillis
  var saved: Long = now
  def elapsedMillis: Long = now - saved
  def elapsedSeconds: Double = elapsedMillis / 1000.0
  def reset: Unit = { saved = now }
}

// HUVUDPROGRAM:
timer.reset
var i = 0L
while (i < 1e8.toLong) { i += 1 }
val t = timer.elapsedSeconds
println("Räknade till " + i + " på " + t + " sekunder.")
\end{Code}


\Subtask Ändra i loopen i uppgift \ref{task:timer}) så att den räknar till 4.4 miljarder. Hur lång tid tar det för din dator att räkna så långt?\footnote{Det går att göra ungefär en heltalsaddition per klockcykel per kärna. Den första elektroniska datorn \href{https://sv.wikipedia.org/wiki/ENIAC}{Eniac} hade en klockfrekvens motsvarande 5 kHz. Den dator på vilken denna övningsuppgift skapades hade en i7-4790K turboklockad upp till 4.4 GHz.
%\href{http://www.extremetech.com/computing/185512-overclocking-intels-core-i7-4790k-can-devils-canyon-fix-haswells-low-clock-speeds/2}{www.extremetech.com/computing/185512-overclocking-intels-core-i7-4790k-can-devils-canyon-fix-haswells-low-clock-speeds/2}
}

\Subtask  Om du kör på en Linux-maskin: Kör nedan Linux-kommando upprepade gånger i ett terminalfönster. Med hur många MHz kör din dators klocka för tillfället? Hur förhåller sig klockfrekvensen till antalet rundor i while-loopen i föregående uppgift? (Det kan hända att din dator kan variera centralprocessorns klockfrekvens. Prova både medan du kör tidmätningen i Kojo och då din dator ''vilar''. Vad är det för poäng med att en processor kan variera sin klockfrekvens?)
\begin{REPLnonum}
> lscpu | grep MHz
\end{REPLnonum}


\Subtask Ändra i koden i uppgift \ref{task:timer}) så att \code{while}-loopen bara kör 5 gånger. %Kör programmet med den \emph{gula} play-knappen. Scrolla i programspårningen och förklara vad som händer. Klicka på \code{CALL}-rutorna och se vilken rad som markeras i ditt program.

\Subtask Lägg till koden nedan i ditt program och försök ta reda på ungefär hur långt din dator hinner räkna till på en sekund för \code{Long}- respektive \code{Int}-variabler. Använd den gröna play-knappen.
\begin{CodeSmall}
def timeLong(n: Long): Double = {
  timer.reset
  var i = 0L
  while (i < n) { i += 1 }
  timer.elapsedSeconds
}

def timeInt(n: Int): Double = {
  timer.reset
  var i = 0
  while (i < n) { i += 1 }
  timer.elapsedSeconds
}

def show(msg: String, sec: Double): Unit = {
  print(msg + ": ")
  println(sec + " seconds")
}

def report(n: Long): Unit = {
  show("Long " + n, timeLong(n))
  if (n <= Int.MaxValue) show("Int  " + n, timeInt(n.toInt))
}

// HUVUDPROGRAM, mätningar:

report(Int.MaxValue)
for (i <- 1 to 10) report(4.26e9.toLong)
\end{CodeSmall}

\Subtask Hur mycket snabbare går det att räkna med \code{Int}-variabler jämfört med \code{Long}-variabler? Diskutera gärna svaret med en handledare.

\Task Lek med färg i Kojo. Sök på internet efter dokumentationen för klassen \code{java.awt.Color} och studera vilka heltalsparametrar den sista konstruktorn i listan med konstruktorer tar för att skapa sRGB-färger. Om du högerklickar i editorn i Kojo och väljer ''Välj färg...'' får du fram färgväljaren och med den kan du välja fördefinierade färger eller blanda egna färger. När du har valt färg får du se vilka parametrar till \code{java.awt.Color} som skapar färgen. Testa detta i REPL:

\begin{REPL}
scala> val c = new java.awt.Color(124,10,78,100)
c: java.awt.Color = java.awt.Color[r=124,g=10,b=78]

scala> c.  // tryck på TAB
asInstanceOf    getColorComponents      getRGBComponents
brighter        getColorSpace           getRed
createContext   getComponents           getTransparency
darker          getGreen                isInstanceOf
getAlpha        getRGB                  toString
getBlue         getRGBColorComponents

scala> c.getAlpha
res3: Int = 100
\end{REPL}
Skriv ett program som ritar många figurer med olika färger, till exempel cirklar som nedan. Om du använder alfakanalen blir färgerna genomskinliga.

\includegraphics[width=0.82\textwidth]{../img/kojo/random-color-circles.png}


\Task Ladda ner ''Uppdrag med Kojo'' från \href{http://lth.se/programmera/uppdrag}{lth.se/programmera/uppdrag}  och gör några uppgifter som du tycker verkar intressanta.

%\Subtask ''Programming Fundamentals with Kojo'' som kan laddas ner här:\\
%\href{http://wiki.kogics.net/kojo-codeactive-books}{wiki.kogics.net/kojo-codeactive-books}

\Task Om du vill jobba med att hjälpa skolbarn att lära sig programmera med Kojo, kontakta \url{http://www.vattenhallen.lth.se} och anmäl ditt intresse att vara handledare.


%!TEX encoding = UTF-8 Unicode
%!TEX root = ../compendium1.tex

\renewcommand{\vecka}{2}

\chapter{Kodstrukturer}
\begin{itemize}[nosep]
\item while-sats
\item for-sats
\item algoritm: min/max
\item MIN_VALUE
\item MAX_VALUE
\item paket
\item import
\item filstruktur
\item jar
\item dokumentation
\item programlayout
\item JDK
\item konstanter vs föränderlighet
\item objektorientering
\item klasser
\item objekt
\item referensvariabler
\item referenstilldelning
\item anropa metoder
\item SimpleWindow\end{itemize}
\clearpage\section{Teori}
%!TEX encoding = UTF-8 Unicode
%!TEX root = ../lect-w02.tex


\ifkompendium

\noindent Ett program innehåller satser och uttryck. En \Emph{kontrollstruktur}, t.ex. \code{while}, styr i vilken \Alert{ordning} satser och uttryck exekveras. Data kan placeras i en \Emph{datastruktur}, t.ex. en \code{Vector}, så att man senare kan komma åt data igen.    
\fi


\ifkompendium\else
\begin{SlideExtra}{Från förra veckan: \texttt{val} \texttt{var} \texttt{def}}
  Vad är det för skillnad på (och likhet mellan) \code{val}, \code{var} och \code{def}? \pause
  \begin{itemize}
    \item Med \code{val} deklareras en variabel som tilldelas ett värde vid initialisering och som sedan \Emph{aldrig ändras}.
    \item Med \code{var} deklareras en variabel som tilldelas ett värde vid initialisering och som sedan \Alert{kan uppdateras} hur många gånger som helst med hjälp av tilldelningssatser.
    \item Med \code{def} deklareras en funktion som körs vid \Emph{varje anrop}
  \end{itemize}
  \pause\vspace{1em} En \Alert{konstighet i REPL}: 
  \begin{itemize}
  \item Man kan i REPL \Emph{deklarera} variabler och funktioner med \Emph{samma namn} \Alert{flera gånger} på samma nivå. 
  \item Detta ger i vanliga, fristående program \Emph{kompileringsfel}. 
  \end{itemize}
\end{SlideExtra}

\begin{SlideExtra}{Från förra veckan: \texttt{if} \texttt{then} \texttt{else}}\SlideFontSmall
\Emph{Spelets regler:} Singla slant. Om du får krona har du vunnit. \\[0.5em]
Förenkla dessa if-uttryck:
\begin{Code}
def singlaSlant: Boolean = if math.random() < 0.5 then true else false

def harVunnit(slantÄrKrona: Boolean): Boolean = 
  if slantÄrKrona == true then true else false
\end{Code}
\pause förenkling av ''kaka-på-kaka'': \textbf{if uttryck then true else false}
\begin{Code}
def singlaSlant: Boolean = math.random() < 0.5
\end{Code}
\pause förenkling av ''kaka-på-kaka'':  \textbf{uttryck == true}
\begin{Code}
def harVunnit(slantÄrKrona: Boolean): Boolean = slantÄrKrona
\end{Code}
\pause Hur ska vi ändra \code{harVunnit} om vi ändrar reglerna till att Klave ger vinst? \\
\pause Repetera även \Alert{de Morgans lagar} från vecka 1 så du har koll på dem.
\end{SlideExtra}

\fi 

\Subsection{Datastrukturer}

\begin{Slide}{Vad är en datastruktur?}\SlideFontSmall
\begin{itemize}
\item En \href{https://sv.wikipedia.org/wiki/Datastruktur}{datastruktur} är en struktur för organisering av data som...
\begin{itemize}\SlideFontTiny
\item kan innehålla \Alert{många} element,
\item kan \Emph{refereras} till som en \Alert{helhet}, och
\item ger möjlighet att \Emph{komma åt} \Alert{enskilda element}.
\end{itemize}

\item En \Emph{samling} \Eng{collection} är en datastruktur som kan innehålla många element av \Alert{samma typ}.

\item Exempel på olika samlingar där elementen är organiserade på olika vis: \\
\vspace{0.5em}
\begin{tabular}{l c}
\Emph{Sekvens} & \includegraphics[width=5cm]{../img/list.pdf} \\
\Emph{Träd}  & \includegraphics[width=2.2cm]{../img/tree.pdf} \\
\Emph{Graf}  & \includegraphics[width=2.2cm]{../img/graph.pdf} \\
\end{tabular}
\end{itemize}
{
\SlideFontTiny \vspace{1em }\hskip2em
Mer om sekvenser \& träd i \href{https://cs.lth.se/edaa01vt}{EDAA01 pfk}.
Mer om träd, grafer i \href{https://cs.lth.se/edaa40}{Diskreta strukturer.}
}

\end{Slide}

\begin{Slide}{Några samlingar i \texttt{scala.collection}}\SlideFontSmall
\SlideOnly{\setlength{\leftmargini}{0pt}}
\begin{itemize}
\item En \Emph{samling} \Eng{collection} är en datastruktur som kan innehålla många element av \Alert{samma typ}.
\item En \Emph{sekvens} \Eng{sequence} är en samling där alla element är ordnade.

\item Exempel på \Emph{färdiga samlingar} i Scalas standardbibliotek där elementen är organiserade  internt på \Alert{olika} vis så att samlingen får olika egenskaper som passar \Alert{olika användningsområden}:
\begin{itemize}\SlideFontTiny
\item \texttt{scala.collection.immutable.\Emph{Vector}}, sekvens med snabb access \Alert{överallt}.
\item \texttt{scala.collection.immutable.\Emph{List}}, sekvens med snabb access \Alert{i början}.
\item \texttt{scala.collection.immutable.\Emph{Set}}, \texttt{scala.collection.\Alert{mutable}.\Emph{Set}}, mängd med unika element; ej i sekvens men snabb innehållstest.
\item \texttt{scala.collection.immutable.\Emph{Map}}, \texttt{scala.collection.\Alert{mutable}.\Emph{Map}}, mängd med par av nyckel \& tillhörande värde, snabb access via nyckel.
\item \texttt{scala.collection.\Alert{mutable}.\Emph{ArrayBuffer}}, förändringsbar sekvens kan ändra storlek.
\item \texttt{scala.\Emph{Array}}, förändringsbar sekvens som \Alert{inte} kan ändra storlek. Alla element är lagrade efter varandra i minnet: snabbast access av alla samlingar, men har speciella begränsningar.
\end{itemize}
\end{itemize}
\end{Slide}


\begin{Slide}{Olika strukturer för att hantera data}
\begin{itemize}\SlideFontSmall
\item \Emph{Tupel} \Eng{tuple}
\begin{itemize}\SlideFontTiny
\item samla flera datavärden t.ex. \code{(1, "hej", true)} i element \Emph{\code{_1}}, \Emph{\code{_2}}, \Emph{\code{_3}} 
\item elementen kan vara av \Alert{olika} typ
\end{itemize}
\item \Emph{Enumeration} (även kallad \emph{uppräkning}) \Eng{enumeration}
\begin{itemize}\SlideFontTiny
\item Namnge uppräknade värden t.ex. \code+enum Color { case Red, Black }+
\item Värdena har ordningsnummer och är alla av \Alert{samma} typ (här \code{Color})
\end{itemize}
\item \Emph{Klass} \Eng{class}
\begin{itemize}\SlideFontTiny
\item samlar data i \Emph{attribut} med (väl valda!) namn
\item attributen kan vara av \Alert{olika} typ
\item definierar även \Emph{metoder} som använder attributen \\ (kallas även \Emph{operationer} på data)
\end{itemize}

\item \Emph{Färdig samling}
  \begin{itemize}
  \item speciella klasser som samlar data i element av \Alert{samma} typ
  \item exempel: \code{scala.collection.immutable.}\Emph{\code{Vector}}
  \item har ofta \emph{många} färdiga \Emph{bra-att-ha-metoder}, \\ se snabbreferensen \url{https://cs.lth.se/pgk/quickref}
  \end{itemize}

\item \Emph{Egenimplementerade samlingar}
  \begin{itemize}
  \item $\rightarrow$ fördjupningskurs
  \end{itemize}

\end{itemize}
\end{Slide}





\begin{Slide}{Vad är en vektor?}\SlideFontSmall
En \Emph{vektor}\footnote{Vektor kallas ibland på svenska även \href{https://sv.wikipedia.org/wiki/F\%C3\%A4lt_\%28datastruktur\%29}{fält}, men det skapar stor förvirring eftersom det engelska ordet \emph{field} ofta används för \emph{attribut} (förklaras senare).}
\Eng{vector} är en \Emph{sekvens} som är \Alert{snabb} att \Emph{indexera} i.
Åtkomst av element i en sekvens som t.ex. heter \code{xs} sker i Scala med \code{xs.apply(platsnummer)}:

\begin{REPL}
scala> val heltal = Vector(42, 13, -1, 0, 1)
val heltal: scala.collection.immutable.Vector[Int] = Vector(42, 13, -1, 0, 1)

scala> heltal.apply(0)   // platsnummer räknas från noll
val res0: Int = 42

scala> heltal(1)         // man kan i Scala skippa .apply före (
val res1: Int = 13

scala> heltal(5)         // ger körtidsfel då sjätte platsen inte finns
java.lang.IndexOutOfBoundsException: 5
  at scala.collection.immutable.Vector.checkRangeConvert(Vector.scala:132)
\end{REPL}
Utelämnar du \code{.apply} så skapar kompilatorn automatiskt ett anrop av \code{apply}.
\end{Slide}

\begin{Slide}{En konceptuell bild av en vektor}

\begin{REPLnonum}
scala> val heltal = Vector(42, 13, -1, 0, 1)

scala> heltal(0)
val res0: Int = 42
\end{REPLnonum}

\begin{tikzpicture}[font=\ttfamily]
\matrix [matrix of nodes, row sep=0, column 2/.style={nodes={rectangle,draw,minimum width=3em}}] (var) at (0cm, 2.8cm)
{
heltal   &  \makebox(16,12){ }\\
};
\matrix [matrix of nodes, draw=black,row sep=0, column 2/.style={nodes={rectangle,draw,minimum width=4em}}] (vec) at (4cm, 1cm)
{
\textit{plats} &  \\
0   &  \makebox(16,12){42}\\
1   &  \makebox(16,12){13}\\
2   &  \makebox(16,12){-1}\\
3   &  \makebox(16,12){0}\\
4   &  \makebox(16,12){1}\\
};
\filldraw[black] (0.7cm,2.8cm) circle (3pt) node[] (ref) {};
 \draw [arrow] (ref) -- (vec);
\end{tikzpicture}

%\vspace{1em} Elementen ligger på rad någonstans i minnet.
\end{Slide}



\begin{Slide}{En samling strängar}

\begin{itemize}
\item En vektor kan lagra \Emph{många} värden av samma typ.
\item Elementen kan vara till exempel heltal eller strängar.
\item Eller faktiskt vad som helst. (En s.k. \emph{generisk} samling.)
\end{itemize}

\begin{REPL}
scala> val grönsaker = Vector("gurka","tomat","paprika","selleri")
grönsaker: scala.collection.immutable.Vector[String] =
  Vector(gurka, tomat, paprika, selleri)

scala> val g = grönsaker(1)
val g: String = tomat

scala> val xs = Vector(42, "gurka", true, 42.0)
val xs: Vector[Matchable] = Vector(42, gurka, true, 42.0)
\end{REPL}
\SlideFontSmall Notera typen \code{Matchable} som betyder ''\Emph{nästan vilken typ som helst}''\\
%\footnote{
%(\code{Matchable} liknar \code{Object} i Java/C\# men är \Alert{mer generell}; 
(Mer om \code{Matchable} senare.)
%}
\end{Slide}



\Subsection{Kontrollstrukturer}


\begin{Slide}{Vad är en kontrollstruktur?}
\begin{itemize}
\item En \Emph{kontrollstruktur} påverkar i vilken ordning (sekvens) satser exekveras och uttryck evalueras.
\begin{itemize}
\item[] Exempel på \Emph{inbyggda} kontrollstrukturer:
\\\vspace{0.5em}\code{for}-\code{do}-sats \\ \code{while}-\code{do}-sats \\ \code{for}-\code{yield}-uttryck
\end{itemize}

\item[]

\item I Scala kan man definiera \Alert{egna} kontrollstrukturer.
\begin{itemize}
\item[] Exempel: \code{upprepa} som du använt i Kojo
\\\vspace{0.5em}\code|upprepa(4){fram; höger}|
\end{itemize}
\end{itemize}
\end{Slide}

\ifkompendium\else
\begin{SlideExtra}{Mitt första program: en oändlig loop på ABC80}
\begin{minipage}{0.8\textwidth}
\begin{verbatim}
10 print "hej"
20 goto 10
\end{verbatim}
\includegraphics[width=0.8\textwidth]{../img/abc80.jpg}
\end{minipage}%
\begin{minipage}{0.2\textwidth}
\pause
\begin{verbatim}
hej
hej
hej
hej
hej
hej
hej
hej
hej
hej
hej
hej
<Ctrl+C>
\end{verbatim}
\end{minipage}
\end{SlideExtra}
\fi

\begin{Slide}{Loopa genom elementen i en vektor}
En \code{for}-\code{do}-\Emph{sats} som skriver ut alla element i en vektor:
\begin{REPL}
scala> val grönsaker = Vector("gurka","tomat","paprika","selleri")

scala> for g <- grönsaker do println(g)
gurka
tomat
paprika
selleri

\end{REPL}
\code{for ... do ...} gör så att följande händer:
\begin{itemize}
  \item Plocka ut \Emph{varje element} ur samlingen. 
  \item \Emph{Namnet} före pilen (här \code{g}) \Alert{refererar} till ett \Emph{nytt} värde för varje runda i loopen.
  \item Detta namn motsvarar en \Emph{lokal} \code{val}-variabel.
\end{itemize}
\end{Slide}


\begin{Slide}{Bygg ny samling från befintlig med for-yield-uttryck}
Ett \code{for}-\code{yield}-\Emph{uttryck} som \Emph{skapar en \Alert{ny} samling}.

\begin{Code}[basicstyle=\ttfamily\fontsize{12}{14}\selectfont]
for g <- grönsaker yield s"god $g"
\end{Code}

\begin{REPL}
scala> val grönsaker = Vector("gurka","tomat","paprika","selleri")

scala> val åsikter = for g <- grönsaker yield s"god $g"
val åsikter: Vector[String] =
  Vector(god gurka, god tomat, god paprika, god selleri)
\end{REPL}

\end{Slide}


\begin{Slide}{Samlingen \code{Range} håller reda på intervall}
\begin{itemize}
\item Med en \code{Range(start, slut)} kan du skapa ett \Emph{intervall}: \\ från och med \code{start} till (men inte med) \code{slut}
\end{itemize}

\begin{REPLnonum}
scala> Range(0, 42)
val res0: Range =
  Range(0, 1, 2, 3, 4, 5, 6, 7, 8, 9, 10, 11, 12, 13, 14,
    15, 16, 17, 18, 19, 20, 21, 22, 23, 24, 25, 26, 27, 28,
    29, 30, 31, 32, 33, 34, 35, 36, 37, 38, 39, 40, 41)
\end{REPLnonum}

\begin{itemize}
\item Men alla värden däremellan skapas inte förrän de behövs:
\end{itemize}

\begin{REPL}
scala> val jättestortIntervall = Range(0, Int.MaxValue)
val jättestortIntervall: Range.Exclusive = Range 0 until 2147483647

scala> jättestortIntervall.end
val res1: Int = 2147483647

scala> jättestortIntervall.toVector
java.lang.OutOfMemoryError: Java heap space
\end{REPL}

\end{Slide}

\begin{Slide}{Loopa med Range}
\code{Range} används i for-loopar för att hålla reda på antalet rundor.
\begin{REPLsmall}
scala> for i <- Range(0, 6) do print(s" gurka $i")
 gurka 0 gurka 1 gurka 2 gurka 3 gurka 4 gurka 5
\end{REPLsmall}
Du kan skapa en \code{Range} med \code{until} efter ett heltal:
\begin{REPLsmall}
scala> 1 until 7
val res1: Range =
  Range(1, 2, 3, 4, 5, 6)

scala> for i <- 1 until 7 do print(s" tomat $i")
 tomat 1 tomat 2 tomat 3 tomat 4 tomat 5 tomat 6

\end{REPLsmall}
Med metoden \code{indices} på kan du få en \code{Range} med alla index:
\begin{REPLsmall}
scala> val xs = Vector("gurka1","gurka2","tomat1")
val xs: Vector[String] = Vector(gurka1, gurka2, tomat1)

scala> xs.indices
val res0: Range = Range 0 until 3
\end{REPLsmall}
\end{Slide}

\begin{Slide}{Loopa med Range skapad med \texttt{to}}

Med \code{to} efter ett heltal får du en \code{Range} till och \Emph{med} sista:
\begin{REPLnonum}
scala> 1 to 6
res2: Range.Inclusive =
  Range(1, 2, 3, 4, 5, 6)

scala> for i <- 1 to 6 do print(" gurka " + i)
 gurka 1 gurka 2 gurka 3 gurka 4 gurka 5 gurka 6

\end{REPLnonum}


\end{Slide}



\begin{Slide}{Vad är en \code{Array}?}


\begin{itemize}
\item En \href{https://en.wikipedia.org/wiki/Array_data_structure}{\code{Array}} liknar en \code{Vector} men har en särställning i JVM:
\begin{itemize}
\item Lagras som en sekvens i minnet på efterföljande adresser.
\item \Emph{Fördel}: snabbaste samlingen för element-access i JVM.
\item Men det finns en hel del \Alert{nackdelar} som vi ska se senare.
\end{itemize}

\end{itemize}

\begin{REPLnonum}
scala> val heltal = Array(42, 13, -1, 0 , 1)
\end{REPLnonum}

\begin{tikzpicture}[font=\ttfamily,scale=0.75, every node/.style={scale=0.75}]
\matrix [matrix of nodes, row sep=0, column 2/.style={nodes={rectangle,draw,minimum width=3em}}] (var) at (0cm, 2.8cm)
{
heltal   &  \makebox(16,12){ }\\
};
\matrix [matrix of nodes, draw=black,row sep=0, column 2/.style={nodes={rectangle,draw,minimum width=4em}}] (vec) at (4cm, 1cm)
{
\textit{plats} &  \\
0   &  \makebox(16,12){42}\\
1   &  \makebox(16,12){13}\\
2   &  \makebox(16,12){-1}\\
3   &  \makebox(16,12){0}\\
4   &  \makebox(16,12){1}\\
};
\filldraw[black] (0.7cm,2.8cm) circle (3pt) node[] (ref) {};
 \draw [arrow] (ref) -- (vec);
\end{tikzpicture}
\end{Slide}

\begin{Slide}{Några likheter \& skillnader mellan \texttt{Vector} och \texttt{Array}}\SlideFontSmall
\begin{multicols}{2}
\begin{REPLnonum}
scala> val xs = Vector(1,2,3)
\end{REPLnonum}

\columnbreak

\begin{REPLnonum}
scala> val xs = Array(1,2,3)
\end{REPLnonum}
\end{multicols}


Några likheter mellan \texttt{Vector} och \texttt{Array}
\begin{itemize}
\item Båda är samlingar som kan innehålla många element.

\item Med båda kan man snabbt accessa vilket element som helst: \code{xs(2)}
\end{itemize}
Några viktiga skillnader:

\vspace{-0.5em}\begin{multicols}{2}
\Emph{Vector}
\begin{itemize}
\item Är \Emph{oföränderlig}: du kan lita på att elementreferenserna aldrig någonsin kommer att ändras.

\item Är \Emph{snabb på att skapa en delvis förändrad kopia}, t.ex. tillägg/borttagning/uppdatering mitt i sekvensen.

\end{itemize}


\columnbreak

\Alert{Array}
\begin{itemize}
\item Är \Alert{föränderlig}: \code{xs(2) = 42}

\item Är \Alert{snabb} om man bara vill läsa eller skriva på befintliga platser.

\item Kan \Alert{ej} ändra storlek: tillägg eller borttagning mitt i kräver \Alert{långsam} kopiering av resten.

\end{itemize}
\end{multicols}
\end{Slide}



\Subsection{Fristående applikation}

\begin{Slide}{Kompilering i terminalen}
  När du ska skriva kod i en editor, kompilera i terminalen och köra ditt program som en \Emph{fristående applikation}, så behövs: 
  \begin{itemize}
    \item En editor: \Emph{VS Code} med tillägget \Alert{Scala (Metals)}
    \item Körmiljön \Emph{OpenJDK} 
    \item Kommandoverktyg i terminalen: \Alert{\texttt{scala}} eller \texttt{scala-cli}
    \item Installera så här: \url{https://cs.lth.se/pgk/verktyg}
    \item Läs mer i Appendix C.
    \item Tips om du kör Windows: installera %\href{https://docs.microsoft.com/en-us/windows/terminal/get-started}
    {nya Windows Terminal}

  \end{itemize}
    Få hjälp i kanalerna \texttt{\#installationskrångel} och \texttt{\#frågor-och-svar} på vår Discord-server eller fråga handledare på resurstid.
  
\end{Slide}

\begin{Slide}{Scala Command Line Interface (CLI)}
\begin{itemize}
\item Utvecklingen av ett nytt kommandogränssnitt \Eng{Command Line Interface (CLI)} för Scala startades 2022 i ett öppenkällkodsprojekt som leds av Virtuslab. 
\item I augusti 2024 blev \Alert{\texttt{scala-cli}} det nya \Emph{\texttt{scala}}%\footnote{I skrivande stund så har skiftet ännu inte skett. När det sker kan du ersätta alla förekomster av \code{scala-cli} med det kortare \code{scala}.}
\item Du kan nu ersätta \code{scala-cli} med \code{scala}
\item Läs mer i Appendix C och F, samt här: \url{https://scala-cli.virtuslab.org/}
\item Se vad Scala CLI kan göra med underkommandot \texttt{help}
\begin{REPLnonum}
  scala help
\end{REPLnonum}
\end{itemize}
\end{Slide}

\begin{Slide}{Ett minimalt fristående program i Scala}\SlideFontSmall
Spara nedan Scala-kod i filen \code{hej.scala}:
\begin{Code}
@main def run = println("Hej Scala!")
\end{Code}

Kompilera och kör i terminalen:
\begin{REPL}
> scala run hej.scala 
Compiling project (Scala 3.5.0, JVM (21))
Compiled project (Scala 3.5.0, JVM (21))
Hej Scala! 
\end{REPL}

Innan körning kompileras dina kodfiler automatiskt vid behov. Du kan se maskinkoden i en underkatalog i till katalogen \texttt{.scala-build}: 
\begin{REPL}
> ls .scala-build/*/classes/main
'hej$package.class'  'hej$package$.class'  'hej$package.tasty'   run.class   run.tasty
\end{REPL}
\end{Slide}


\begin{Slide}{Loopa genom en samling med en \texttt{while}-sats}
\begin{REPLnonum}
scala> val xs = Vector("Hej","på","dej","!!!")
val xs: Vector[String] =
  Vector(Hej, på, dej, !!!)

scala> xs.size
val res0: Int = 4

scala> var i = 0
val i: Int = 0

scala> while i < xs.size do { println(xs(i)); i = i + 1 }
Hej
på
dej
!!!
\end{REPLnonum}
\end{Slide}


\begin{Slide}{Strängargument till i ett program med primitiv main}
Skriv och spara nedan kod i filen \texttt{helloargs1scala}
\begin{REPLnonum}
> code helloargs1.scala
\end{REPLnonum}
\begin{Code}
object HelloScalaArgs:
  def main(args: Array[String]): Unit = // en primitiv main-metod utan @main
    var i = 0
    while i < args.size do
      println(args(i))
      i = i + 1
\end{Code}
En primitiv \code{main}-metod har ej \code{@main} och måste vara i ett objekt. \\
Kompilera och kör med programargument efter \code{--}
\begin{REPL}
> scala run helloargs1.scala -- morot gurka tomat
morot
gurka
tomat
\end{REPL}
\end{Slide}

\begin{Slide}{Typsäkra argument till i ett program med @main}
  \SlideFontSmall
Skriv och spara nedan kod i filen \texttt{helloargs2.scala}
\begin{REPLnonum}
> code helloargs2.scala
\end{REPLnonum}
\begin{Code}
@main def hej(heltal: Int, resten: String*): Unit =  // notera * efter String
  for i <- 0 until heltal do println(resten(i))
\end{Code}
Med \code{@main} behövs inget objekt.\\
Kompilera och kör med programargument efter \code{--}
\begin{REPL}
> scala run helloargs2.scala -- 2 morot gurka tomat
morot
gurka
> scala run helloargs2.scala -- aj morot gurka tomat
Illegal command line: java.lang.NumberFormatException: For input string: "aj"
\end{REPL}
Med \code{@main} genereras automatiskt en primitiv main som kollar att argumenten har rätt typ.
\end{Slide}


\begin{Slide}{För kännedom: Scala-\textbf{skript}}
\begin{itemize}
  \SlideFontSmall
  \item 
Scala-kod kan köras som ett \Emph{skript}.
\item Ett skript finns i en enda fristående fil med ändelsen \code{.sc}
\item Skript behöver inget huvudprogram. 
\item Skript har automatiskt alla programargument i strängsekvensen \code{args}

\begin{Code}
// spara detta i filen 'myscript.sc'
println("Hej alla mina argument:")
for a <- args do println(s"Hej: $a") 
\end{Code}
\begin{REPLnonum}
> scala run myscript.sc -- ett två tre
Hej alla mina argument:
Hej: ett
Hej: två
Hej: tre
\end{REPLnonum}
\end{itemize}
\pause {\SlideFontTiny Ett Scala-skript kan ej anropa andra skript utan speciella åtgärder, se detaljer här: \url{https://scala-cli.virtuslab.org/docs/guides/scripting/scripts/}
}
\end{Slide}



\Subsection{Algoritmer: stegvisa lösningar}

\begin{Slide}{Vad är en algoritm?}
En \href{https://sv.wikipedia.org/wiki/Algoritm}{algoritm} är en sekvens av instruktioner som beskriver hur man löser ett problem.

\vspace{1em}\Emph{Exempel}:
\begin{itemize}
\item	 baka en kaka
\pause\item räkna ut din pensionsprognos
\pause\item köra bil
\pause\item kolla om highscore i ett spel
\end{itemize}
\ifkompendium\else
\begin{tikzpicture}[overlay]
\node[xshift=0.85\textwidth, scale=2.0] { \includegraphics[width=0.25\textwidth]{../img/highscore}};
\end{tikzpicture}
\fi
\end{Slide}


\ifkompendium\else
\begin{SlideExtra}{Algoritm-exempel: HIGHSCORE}
\Emph{Problem}: Kolla om high-score i ett spel \\ \vspace{1em}

\Emph{Varför?} \pause Så att de som spelar uppmuntras att spela mer :) \\ \vspace{1em}

\Emph{Algoritm:}\pause
\begin{enumerate}
\item $points$ $\leftarrow$ poängen efter senaste spelet
\item $highscore$ $\leftarrow$ bästa resultatet innan senaste spelet
\item \Key{om} $points$ är större än $highscore$ \Key{så}
\begin{enumerate}[ ~~]
\item  Skriv ''Försök igen!''
\end{enumerate}
\Key{annars}
\begin{enumerate}[ ~~]
\item  Skriv ''Grattis!''
\end{enumerate}
\end{enumerate}
\pause
\scriptsize \Alert{Hittar du buggen?}

\pause Utskriften blir fel; vänd villkor eller byt plats på grenarna i if-satsen
\end{SlideExtra}
\fi

\ifkompendium\else
\begin{SlideExtra}{HIGHSCORE implementerad i Scala}
\begin{Code}
import scala.io.StdIn.readLine

@main 
def run = 
  val points = readLine("Hur många poäng fick du?").toInt
  val highscore = readLine("Vad var highscore före senaste spelet?").toInt
  val msg = if points > highscore then "GRATTIS!" else "Försök igen!"
  println(msg)
\end{Code}
\SlideFontSmall %(Mer om \code{import} senare.)\\
\pause
Är det en bugg eller en feature att det står\\ \texttt{points > highscore} \\ och inte \\ \texttt{points >= highscore} \\ ?
\pause Man får ej GRATTIS om poäng == highscore vilket är tråkigt :)
\end{SlideExtra}



\fi

\begin{Slide}{Algoritmexempel: N-FAKULTET}
\begin{algorithm}[H]
 \SetKwInOut{Input}{Indata}\SetKwInOut{Output}{Utdata}

 \Input{heltalet $n$}
 \Output{produkten av de första $n$ positiva heltalen}
 ~\\
 $prod \leftarrow 1$ \\
 $i \leftarrow 2$  \\
 \While{$i \leq n$}{
  $prod \leftarrow prod * i$\\
  $i \leftarrow i + 1$
 }
 $prod$
\end{algorithm}
\pause\vspace{1em}
\begin{itemize}\SlideFontSmall
\item Vad händer om $n$ är noll?
\item Vad händer om $n$ är ett?
\item Vad händer om $n$ är två?
\item Vad händer om $n$ är tre?
\end{itemize}
\end{Slide}

\begin{Slide}{Algoritmexempel: MIN}
\begin{algorithm}[H]
 \SetKwInOut{Input}{Indata}\SetKwInOut{Output}{Utdata}

 \Input{Array $args$ med strängar som alla innehåller heltal}
 \Output{minsta heltalet }
 ~\\
 $min \leftarrow$ det största heltalet som kan uppkomma  \\
 $n \leftarrow $ antalet heltal \\
 $i \leftarrow 0$ \\
 \While{$i < n$}{
   $x \leftarrow args(i).toInt$ \\
   \If{( x < $min$)}{$min \leftarrow x$}
   $i \leftarrow i + 1$
 }
 $min$
\end{algorithm}
\pause{\hfill \SlideFontTiny \Emph{Testa med indata}: \code{args = Array("2", "42", "1", "2")}}
\end{Slide}


\Subsection{Funktioner skapar struktur}

\ifkompendium
\noindent En program delas ofta upp i många olika \Emph{funktioner}. En funktion kan ha parametrar och ge ett returvärde. Om du delar upp ditt program i många enkla funktioner med bra namn, så blir ditt program lättare att läsa och begripa. Om en vältestad och buggfri funktion användas på flera ställen, så kan risken för buggar minskas.
\fi 

\begin{Slide}{Mall för funktionsdefinitioner}
\code{def} funktionsnamn(parameterdeklarationer): returtyp = uttryck

\pause\vspace{0.3em}\SlideFontSmall
\Emph{Exempel}:

\begin{Code}[basicstyle=\ttfamily\fontsize{9}{11}\selectfont]
def öka(i: Int): Int = i + 1
\end{Code}
\pause Returtypen kan härledas av kompilatorn:
\begin{Code}[basicstyle=\ttfamily\fontsize{9}{11}\selectfont]
def öka(i: Int) = i + 1
\end{Code}
Men för att få hjälp av kompilatorn är det bra att ange returtyp!

\pause 

Om flera parametrar använd kommatecken. Om flera satser använd indentering (och eventuell valfria klammerparenteser).
\begin{Code}[basicstyle=\ttfamily\fontsize{8}{10}\selectfont]
def isHighscore(points: Int, high: Int): Boolean = {
  val highscore: Boolean = points > high
  if highscore then println(":)") else println(":(")
  highscore
}
\end{Code}
\pause Ovan funktion har \Alert{sidoeffekten} att skriva ut en smiley.
\end{Slide}

\begin{Slide}{Bättre många små abstraktioner som gör en sak var}

\begin{Code}[basicstyle=\ttfamily\fontsize{8}{11}\selectfont]
def isHighscore(points: Int, high: Int): Boolean = points > high

def printSmiley(isHappy: Boolean): Unit =
  if isHappy then println(":)") else print(":(")
\end{Code}

\pause\vspace{1em}
\begin{REPLnonum}
  printSmiley(isHighscore(113,99))
\end{REPLnonum}

\pause
\begin{itemize}
  \item Denna bättre \code{isHighscore} är nu en \Emph{äkta funktion} som alltid ger samma svar för samma inparametrar och \Alert{saknar sidoeffekter}; dessa funktioner är ofta lättare att förstå.
  \item Funktioner som ger ett booleskt värde kallas för \Emph{predikat}.
\end{itemize}

\end{Slide}



\begin{Slide}{Vad är ett block?}

\begin{itemize}
\item Ett block \Emph{kapslar in} flera satser/uttryck och ser ''utifrån'' ut som en enda sats/uttryck.

\item Ett block skapas med hjälp av klammerparenteser (''krullparenteser'')

\item [] {\fontsize{14}{18}\selectfont \code|{ uttryck1; uttryck2; ... uttryckN }|}\\~

\pause

\item I Scala (till skillnad från många andra språk) har ett block ett \Emph{värde} och är alltså ett \Emph{uttryck}.

\item Värdet ges av \Emph{sista uttrycket} i blocket.

\begin{REPLnonum}
scala> val x = { println(1 + 1); println(2 + 2); 3 + 3 }
2
4
x: Int = 6
\end{REPLnonum}


\end{itemize}

\end{Slide}

\begin{Slide}{Namn i block blir \textbf{lokala}}
Synlighetsregler:
\begin{enumerate}
\item Identifierare deklarerade inuti ett block blir \Emph{lokala}.

\item Lokala namn \Alert{överskuggar} namn i yttre block om samma.


\item Namn syns i nästlade underblock.

\end{enumerate}

\begin{REPL}
scala> def a = { val lokaltNamn = 42; println(lokaltNamn) }
scala> a 
42

scala> println(lokaltNamn)                                                                                                                  
1 |println(lokaltNamn)
  |        ^^^^^^^^^^
  |        Not found: lokaltNamn

scala> def b = { val x = 42; { val x = 76; println(x) }; println(x) }
scala> def c = { val x = 42; { val b = x + 1; println(b) } }
scala> b  // vad händer?
scala> c  // vad händer?
\end{REPL}

\end{Slide}


\begin{Slide}{Parameter och argument}

Skilj på parameter och argument!
\begin{itemize}
\item En \Alert{parameter} är det deklarerade namnet som används \Alert{lokalt} i en funktion för att referera till...

\item \Emph{argumentet} som är värdet som skickas med \Emph{vid anrop} och binds till det lokala parameternamnet.

\end{itemize}


\begin{REPLnonum}
scala> val ettArgument = 42

scala> def öka(minParameter: Int) = minParameter + 1

scala> öka(ettArgument)
\end{REPLnonum}


Speciell syntax: anrop med s.k. \Emph{namngivet argument}
\begin{REPLnonum}
scala> öka(minParameter = ettArgument)
\end{REPLnonum}
{\SlideFontSmall Namngivna argument kan ges i valfri ordning; då riskerar man inte fel ordning.}

\end{Slide}

\begin{Slide}{Procedurer}\SlideFontSmall
\begin{itemize}
\item En \Emph{procedur} är en funktion som \Alert{gör} något intressant, men som \Alert{inte} lämnar något intressant returvärde.
\item Exempel på befintlig procedur: \code{println("hej")}
\item Du \Emph{deklarerar egna procedurer} genom att ange \texttt{\Alert{Unit}} som returvärdestyp. Då ges värdet \texttt{\Alert{()}} som betyder ''inget''.
\end{itemize}
\begin{REPLsmall}$%dummydollar 
scala> def hej(x: String): Unit = println(s"Hej på dej $x!")

scala> hej("Herr Gurka")
Hej på dej Herr Gurka!

scala> val x = hej("Fru Tomat")
Hej på dej Fru Tomat!

scala> :type x 
Unit

scala> println(x)    // vad händer?
\end{REPLsmall}
\begin{itemize}
\item Det som \Alert{görs} kallas (sido)\Emph{effekt}. Ovan är utskriften själva effekten.
\item Funktioner kan också ha sidoeffekter. De kallas då \Alert{oäkta} funktioner.
\end{itemize}
\end{Slide}

\begin{Slide}{''Ingenting'' \emph{är} faktiskt någonting i Scala}
\begin{itemize}
\item I många språk (Java, C, C++, ...) är funktioner som saknar värden speciella.
 Java m.fl. har speciell syntax för procedurer med nyckelordet \jcode{void}, men \Alert{inte} Scala.

\item I Scala är procedurer inte specialfall; de är vanliga funktioner som returnerar ett värde som \Emph{representerar} ingenting, nämligen () som är av typen Unit.

\item På så sätt blir procedurer inget undantag utan följer vanlig syntax och semantik precis som för alla andra funktioner.

\item Detta är typiskt för Scala: generalisera koncepten och vi slipper besvärliga undantag! \\(Men vi måste förstå generaliseringen...)


\item [] {\SlideFontSmall
\url{https://en.wikipedia.org/wiki/Void_type}
\url{https://en.wikipedia.org/wiki/Unit_type}
}

\end{itemize}

\end{Slide}

\begin{Slide}{Problemlösning: nedbrytning i abstraktioner som sen kombineras}\SlideFontSmall
\begin{itemize}
\item En av de allra viktigaste principerna inom programmering är \Emph{funktionell nedbrytning} där  \Emph{underprogram} i form av funktioner och procedurer skapas för att bli byggstenar som kombineras till mer avancerade funktioner och procedurer.

\item Genom de namn som definieras skapas \Emph{återanvändbara abstraktioner} som kapslar in det funktionen gör.

\item Problemet blir med bra byggblock lättare att lösa.

\item Abstraktioner som beräknar eller gör \Emph{en enda, väldefinierad sak} är enklare att använda, jämfört med de som gör många, helt olika saker.

\item Abstraktioner med \Emph{välgenomtänkta namn} är enklare att använda, jämfört med kryptiska eller missvisande namn.
\end{itemize}

\end{Slide}



\begin{Slide}{Exempel på \textbf{funktionell nedbrytning}}

Kojo-labben gav exempel på \Emph{funktionell nedbrytning} där ett antal abstraktioner skapas och återanvänds.

\begin{Code}
// skapa abstraktioner som bygger på varandra

def kvadrat = upprepa(4){fram; höger}

def stapel = {
  upprepa(10){kvadrat; hoppa}
  hoppa(-10*25)
} 

def rutnät = upprepa(10){stapel; höger; fram; vänster}

// huvudprogram

sudda; sakta(200)
rutnät
\end{Code}
\end{Slide}


\begin{Slide}{Varför abstraktion?}
\begin{itemize}
\item Stora program behöver delas upp annars blir det mycket svårt att förstå och bygga vidare på programmet.
\item Vi behöver kunna välja namn på saker i koden \textit{lokalt}, utan att det krockar med samma namn i andra delar av koden.
\item Abstraktioner hjälper till att hantera och kapsla in komplexa delar så att de blir enklare att använda om och om igen.

\item Exempel på \Emph{abstraktionsmekanismer} i Scala:
\begin{itemize}

\item \href{https://sv.wikipedia.org/wiki/Klass_\%28programmering\%29}{Klasser} är ''byggblock'' med kod som används för att skapa \href{https://sv.wikipedia.org/wiki/Objektorienterad_programmering\#Objekt}{objekt}, innehållande delar som hör ihop. \\ Nyckelord: \code{class} och \code{object}

\item \href{https://en.wikipedia.org/wiki/Method_\%28computer_programming\%29}{Metoder} är funktioner som finns i klasser/objekt och används för att lösa specifika uppgifter.  Nyckelord: \code{def}

\item \href{https://en.wikipedia.org/wiki/Java_package}{Paket} används för att skapa namnrymder och organisera maskinkod i en hierarkisk katalogstruktur. \\Nyckelord: \Key{package}

\end{itemize}

\end{itemize}
\end{Slide}


\Subsection{Katalogstruktur för kodfiler med paket}



\begin{Slide}{Från källkod till maskinkod med JVM}
\begin{tikzpicture}[node distance=1.5cm]
\node (input) [startstop] {\texttt{hello.scala}};
\node(inptext) [right of=input, text width=5.5cm, scale=1.2,xshift=3.5cm]{Källkodsfil};
\node (compile) [process, below of=input] {\texttt{scalac}};
\node(comptext) [right of=compile, text width=7.2cm, scale=1.0,xshift=4.5cm]{Kompilatorn skapar \textit{abstrakt} maskinkod (s.k. bytekod)};
\node (output) [startstop, below of=compile] {\texttt{hello.class}};
\node(outtext) [right of=output, text width=5.5cm, scale=1.2,xshift=3.5cm]{\texttt{.class}-fil med bytekod};
\node (jvm) [process, below of=output] {JVM};
\node(jvmtext) [right of=jvm, text width=7.2cm, scale=1.0,xshift=4.5cm]{\textit{Java Virtual Machine}\\Översätter bytekod till \textit{konkret}\\ maskinkod som passar din specifika CPU \textbf{under körning} (s.k. interpretering)};
\draw [arrow] (input) -- (compile);
\draw [arrow] (compile) -- (output);
\draw [arrow] (output) -- (jvm);
\end{tikzpicture}
\end{Slide}




\begin{Slide}{Paket}\SlideFontSmall

\begin{Code}
package greeting

@main def run = println("Hello world!")
\end{Code}

\begin{itemize}
\item Paket \Eng{package} skapar namnrymder och i en hierarkisk struktur. 

\item Paket kan vara \Emph{nästlade}: ofta finns paket i paket i paket.

\item Paket är speciellt bra om man har mycket kod i många kodfiler. 

\item Kompilatorn placerar maskinkoden i kataloger enligt paketstrukturen.%
\footnote{\SlideFontTiny Katalogstrukturen för källkoden \emph{måste} i många andra språk, t.ex. Java, \emph{exakt motsvara paketstrukturen}, men detta är inte nödvändigt i Scala -- alla Scala-kodfiler kan ligga i samma katalog på toppnivå eller i underkatalog med valfritt namn, oavsett hur din kod använder \code{package}.} 
\item[] Är du nyfiken, kolla underkataloger i \code{.scala-build}:
\begin{REPLsmall}
ls -R .scala-build
\end{REPLsmall}

\end{itemize}

% \vspace{1em}
% \begin{tikzpicture}[node distance=1.5cm,scale=0.8, every node/.style={transform shape}]
% \node (input) [startstop] {\texttt{greeting/Hello.scala}};
% \node(inptext) [right of=input, text width=4cm, scale=1.2,xshift=4.5cm]{\lstinline{package greeting}\\\lstinline{object Hello  ... }};
% \node (compile) [process, below of=input] {\texttt{scalac  greeting/Hello.scala}};
% \node (output) [startstop, below of=compile] {\texttt{greeting/Hello.class}};
% \node(outtext) [right of=output, text width=4cm, scale=1.2,xshift=4.5cm]{Paketens maskinkod hamnar i katalog med samma namn som paketnamnet};
% \node (jvm) [process, below of=output] {\texttt{scala greeting.Hello}};
% \draw [arrow] (input) -- (compile);
% \draw [arrow] (compile) -- (output);
% \draw [arrow] (output) -- (jvm);
% \end{tikzpicture}

\end{Slide}

\begin{Slide}{Import}
Med hjälp av punktnotation kommer man åt innehåll i ett paket.\\
\begin{Code}
val age = scala.io.StdIn.readLine("Ange din ålder:")
\end{Code}

En \code{import}-sats...

\begin{Code}
import scala.io.StdIn.readLine
\end{Code}

...gör så att namnet syns \Emph{direkt}, och man slipper skriva hela vägen till namnet:
\begin{Code}
val age = readLine("Ange din ålder:")
\end{Code}

Man säger att det importerade namnet hamnar \Emph{\textit{in scope}}.
\end{Slide}


\begin{Slide}{Jar-filer}
\begin{itemize}\SlideFontTiny
\item 
\texttt{jar}-filer liknar \texttt{zip}-filer och används för att sammanföra många kompilerade kodfiler i \Emph{en komprimerad fil} för enkel distribution och körning.
\item Du använder jar-filer med optionen \code{--jar}
\begin{REPLsmall}
scala run . --jar introprog.jar
\end{REPLsmall}
\item Du kan skapa egna jar-filer med \texttt{scala package} där optionen \code{--library} gör så att endast den komilerade koden inkluderas. Utan optionen \code{--library} så görs jar-filen exekverbar. Med optionen \code{--assembly} tas allt med i jar-filen som behövs för att köra jar-filen helt fristående med ett dubbelklick eller \code{java -jar myapp.jar}
\begin{REPLsmall}
scala package . --library --output myapp.jar
scala run --jar myapp.jar
scala package . --assembly --output my-fat-jar-app.jar
java -jar my-fat-jar-app.jar
\end{REPLsmall}
Optionen \code{--assembly} kräver power-läge enl. instruktioner i varning.\\
Läs mer om jar-filer i Appendix F.
\end{itemize}

% \vspace{2em}
% \begin{tikzpicture}[node distance=1.5cm,scale=0.8, every node/.style={transform shape}]
% \node (input) [startstop] {\texttt{greeting/}};
% \node(inptext) [right of=input, text width=4cm, scale=1.2,xshift=4.5cm]{en katalog med filer};
% \node (jar) [process, below of=input]
% {\texttt{jar cvf minjarfil.jar greeting}};

% \node (output) [startstop, below of=compile] {\texttt{minjarfil.jar}};

% \node(outtext) [right of=output, text width=4cm, scale=1.2,xshift=4.5cm]{En jar-fil med alla filer inpackade};

% \node (jvm) [process, below of=output] {\texttt{scala -cp minjarfil.jar}};

% \node(outtextjvm) [right of=jvm, text width=4cm, scale=1.2,xshift=4.5cm]{Lägg jar-filen till \\ ''classpath''};
% \draw [arrow] (input) -- (jar);
% \draw [arrow] (jar) -- (output);
% \draw [arrow] (output) -- (jvm);
% \end{tikzpicture}
\end{Slide}

\ifkompendium\else

\subsection{Att göra denna vecka}
%%%
\begin{SlideExtra}{Att göra denna vecka}

\begin{enumerate}
%\item Laborationer är \Alert{obligatoriska}.\\ Ev. sjukdom måste anmälas \Alert{före} via mejl till kursansvarig!
\item Gör övning \texttt{programs}
\item OBS! Ingen lab denna vecka w02. Använd tiden att komma ikapp om du ligger efter!
\item Träffas i samarbetsgrupper och hjälp varandra att förstå.
\item Vi har nosat på flera koncept som vi kommer tillbaka till senare: du kommer förstå mer detaljer på djupet då.
\item Om ni inte redan gjort det: \\Visa \href{https://github.com/bjornregnell/lth-eda016-2015/tree/master/assignments}{samarbetskontrakt} för handledare på resurstid.
\item \Alert{Koda på resurstiderna} och få hjälp och tips!
\end{enumerate}
\end{SlideExtra}

\begin{SlideExtra}{Veckans övning: \texttt{\ExeWeekTWO}}\SlideFontTiny
\vspace{-0.5em}
\setlength{\leftmargini}{0pt}
\begin{itemize}
%!TEX encoding = UTF-8 Unicode
%!TEX root = ../exercises.tex

\item Kunna skapa, kompilera och köra en enkel applikation i terminalen.
\item Kunna skapa samlingarna Range, Array och Vector med heltal och strängar.
\item Kunna indexera i en indexerbar samling, t.ex. Array och Vector.
\item Kunna anropa operationerna size, mkString, sum, min, max på samlingar som innehåller heltal.
\item Känna till skillnader och likheter mellan samlingarna Range, Array och Vector.
\item Förstå skillnaden mellan en while-sats och ett for-uttryck.
\item Kunna skapa samlingar med heltalsvärden som resultat av enkla for-uttryck.
\item Förstå skillnaden mellan en algoritm i pseudo-kod och dess implementation.
\item Kunna implementera algoritmerna SUM, MIN, MAX med en indexerbar samling och en while-sats.

\end{itemize}
\end{SlideExtra}

\begin{SlideExtra}{Labb läsvecka 3:  \texttt{\LabWeekTHREE}}\SlideFontSmall
\begin{itemize}
\item Skapa ett \Emph{lagom} irriterande textspel som körs i terminalen:
\begin{itemize}\SlideFontTiny
  \item ska vara \Emph{lagom} irriterande om man \Emph{först läser koden}
  \item får gärna vara \Alert{orimligt} irriterande om man \Alert{inte läser koden}
  \item koden ska vara \Emph{lättläst} och uppdelad i \Emph{många små funktioner} med bra, förklarande funktionsnamn, parameternamn och variablenamn
\end{itemize}
\item Använd en editor, kompilera och kör i terminalen
\item Mål: skapa \Emph{eget} program med \Emph{många små funktioner} och träna på \Emph{alla begrepp} vi använt hittills. Ju fler begrepp du kan använda på olika sätt desto bättre. Fokusera på det \Alert{du} behöver träna mest på.
\item Spela varandras textspel inom din samarbetsgrupp
\item Utveckla ditt spel \Emph{stegvis} och spela varandras halvfärdiga spel i flera omgångar. Ge varandra tips om förbättringar för att spelet ska bli mer lagom irriterande på ett kul sätt och för att koden ska bli mer lättläst.
\item Skriv ner den återkoppling du fått av din grupp inför labbredovisningen.
\item Läs igenom labbuppgiften redan nu och börja fundera. Ta dig inte ''vatten över huvudet''. Ta små steg i början och ha hela tiden körbar kod.


\end{itemize}
\end{SlideExtra}
\fi

\ifkompendium\else
\begin{SlideExtra}{}
  \Huge \Emph{Alla kodar loss!} \\~\\ Vi ses nästa vecka!
\end{SlideExtra}
\fi 



%\chapter{Kodstrukturer}
\begin{itemize}[nosep]
\item while-sats
\item for-sats
\item algoritm: min/max
\item MIN_VALUE
\item MAX_VALUE
\item paket
\item import
\item filstruktur
\item jar
\item dokumentation
\item programlayout
\item JDK
\item konstanter vs föränderlighet
\item objektorientering
\item klasser
\item objekt
\item referensvariabler
\item referenstilldelning
\item anropa metoder
\item SimpleWindow\end{itemize}

%!TEX encoding = UTF-8 Unicode
%!TEX root = ../exercises.tex

\ifPreSolution

\Exercise{\ExeWeekTWO}\label{exe:W02}
\begin{Goals}
%!TEX encoding = UTF-8 Unicode
%!TEX root = ../exercises.tex

\item Kunna skapa, kompilera och köra en enkel applikation i terminalen.
\item Kunna skapa samlingarna Range, Array och Vector med heltal och strängar.
\item Kunna indexera i en indexerbar samling, t.ex. Array och Vector.
\item Kunna anropa operationerna size, mkString, sum, min, max på samlingar som innehåller heltal.
\item Känna till skillnader och likheter mellan samlingarna Range, Array och Vector.
\item Förstå skillnaden mellan en while-sats och ett for-uttryck.
\item Kunna skapa samlingar med heltalsvärden som resultat av enkla for-uttryck.
\item Förstå skillnaden mellan en algoritm i pseudo-kod och dess implementation.
\item Kunna implementera algoritmerna SUM, MIN, MAX med en indexerbar samling och en while-sats.

\end{Goals}

\begin{Preparations}
\item \StudyTheory{02}
\item Bekanta dig med grundläggande terminalkommandon, se appendix~\ref{appendix:terminal}.
\item Bekanta dig med den editor du vill använda, se appendix~\ref{appendix:compile}.
\end{Preparations}

\else

\ExerciseSolution{\ExeWeekTWO}

\fi


% terminalkommando
% scalac -> hello world; scala som script; javac
% paket, import, jar, main,


\BasicTasksNoLab %%%%%%%%%%%%%%%%




\WHAT{Para ihop begrepp med beskrivning.}

\QUESTBEGIN

\Task \what

\vspace{1em}\noindent Koppla varje begrepp med den (förenklade) beskrivning som passar bäst: 

\begin{ConceptConnections}
  kompilerad & 1 & & A & där exekveringen av kompilerad app startar \\ 
  skript & 2 & & B & en samling som representerar ett intervall av heltal \\ 
  objekt & 3 & & C & maskinkod sparad och kan köras igen utan kompilering \\ 
  main & 4 & & D & en oföränderlig, indexerbar sekvenssamling \\ 
  programargument & 5 & & E & applicerar en funktion på varje element i en samling \\ 
  datastruktur & 6 & & F & stegvis beskrivning av en lösning på ett problem \\ 
  samling & 7 & & G & maskinkod sparas ej utan skapas vid varje körning \\ 
  sekvenssamling & 8 & & H & samlar variabler och funktioner \\ 
  Array & 9 & & I & överförs via parametern args i main \\ 
  Vector & 10 & & J & en specifik realisering av en algoritm \\ 
  Range & 11 & & K & används i for-uttryck för att skapa ny samling \\ 
  yield & 12 & & L & en förändringsbar, indexerbar sekvenssamling \\ 
  map & 13 & & M & datastruktur med element av samma typ \\ 
  algoritm & 14 & & N & många olika element i en helhet; elementvis åtkomst \\ 
  implementation & 15 & & O & datastruktur med element i en viss ordning \\ 
\end{ConceptConnections}

\SOLUTION

\TaskSolved \what

\begin{ConceptConnections}
  kompilerad & 1 & ~~\Large$\leadsto$~~ &  C & maskinkod sparad och kan köras igen utan kompilering \\ 
  skript & 2 & ~~\Large$\leadsto$~~ &  G & maskinkod sparas ej utan skapas vid varje körning \\ 
  objekt & 3 & ~~\Large$\leadsto$~~ &  H & samlar variabler och funktioner \\ 
  main & 4 & ~~\Large$\leadsto$~~ &  A & där exekveringen av kompilerad app startar \\ 
  programargument & 5 & ~~\Large$\leadsto$~~ &  I & överförs via parametern args i main \\ 
  datastruktur & 6 & ~~\Large$\leadsto$~~ &  N & många olika element i en helhet; elementvis åtkomst \\ 
  samling & 7 & ~~\Large$\leadsto$~~ &  M & datastruktur med element av samma typ \\ 
  sekvenssamling & 8 & ~~\Large$\leadsto$~~ &  O & datastruktur med element i en viss ordning \\ 
  Array & 9 & ~~\Large$\leadsto$~~ &  L & en förändringsbar, indexerbar sekvenssamling \\ 
  Vector & 10 & ~~\Large$\leadsto$~~ &  D & en oföränderlig, indexerbar sekvenssamling \\ 
  Range & 11 & ~~\Large$\leadsto$~~ &  B & en samling som representerar ett intervall av heltal \\ 
  yield & 12 & ~~\Large$\leadsto$~~ &  K & används i for-uttryck för att skapa ny samling \\ 
  map & 13 & ~~\Large$\leadsto$~~ &  E & applicerar en funktion på varje element i en samling \\ 
  algoritm & 14 & ~~\Large$\leadsto$~~ &  F & stegvis beskrivning av en lösning på ett problem \\ 
  implementation & 15 & ~~\Large$\leadsto$~~ &  J & en specifik realisering av en algoritm \\ 
\end{ConceptConnections}

\QUESTEND






%%%%%%%%%%%%%%%%%%% SKA FIXAS:




\WHAT{Datastrukturen \code+Range+.}

\QUESTBEGIN

\Task  \what~Evaluera nedan uttryck i Scala REPL. Vad har respektive uttryck för värde och typ?

\Subtask \code{Range(1, 10)}

\Subtask \code{Range(1, 10).inclusive}

\Subtask \code{Range(0, 50, 5)}

\Subtask \code{Range(0, 50, 5).size}

\Subtask \code{Range(0, 50, 5).inclusive}

\Subtask \code{Range(0, 50, 5).inclusive.size}

\Subtask \code{0.until(10)}

\Subtask \code{0 until (10)}

\Subtask \code{0 until 10}

\Subtask \code{0.to(10)}

\Subtask \code{0 to 10}

\Subtask \code{0.until(50).by(5)}

\Subtask \code{0 to 50 by 5}

\Subtask \code{(0 to 50 by 5).size}

\Subtask \code{(1 to 1000).sum}


\SOLUTION


\TaskSolved \what
 

\SubtaskSolved  värde: \code{Range(1,2,3,4,5,6,7,8,9)}

typ: \code{scala.collection.immutable.Range}

\SubtaskSolved  värde: \code{Range(1,2,3,4,5,6,7,8,9,10)}

typ: \code{scala.collection.immutable.Range}

\SubtaskSolved  värde: \code{Range(0,5,10,15,20,25,30,35,40,45)}

 typ: \code{scala.collection.immutable.Range}

\SubtaskSolved  värde: \code{10}, typ: \code{Int}

\SubtaskSolved  värde: \code{Range(0,5,10,15,20,25,30,35,40,45,50)}

typ: \code{scala.collection.immutable.Range}

\SubtaskSolved  värde: \code{11}, typ: \code{Int}

\SubtaskSolved  värde: \code{Range(0,1,2,3,4,5,6,7,8,9)}

typ: \code{scala.collection.immutable.Range}

\SubtaskSolved  värde: \code{Range(0,1,2,3,4,5,6,7,8,9)}

typ: \code{scala.collection.immutable.Range}

\SubtaskSolved  värde: \code{Range(0,1,2,3,4,5,6,7,8,9)}

typ: \code{scala.collection.immutable.Range}

\SubtaskSolved  värde: \code{Range(0,1,2,3,4,5,6,7,8,9,10)}

typ: \code{scala.collection.immutable.Range.Inclusive}

\SubtaskSolved  värde: \code{Range(0,1,2,3,4,5,6,7,8,9,10)}

typ: \code{scala.collection.immutable.Range.Inclusive}

\SubtaskSolved  värde: \code{Range(0,5,10,15,20,25,30,35,40,45)}

typ: \code{scala.collection.immutable.Range}

\SubtaskSolved  värde: \code{Range(0,5,10,15,20,25,30,35,40,45,50)}

typ: \code{scala.collection.immutable.Range}

\SubtaskSolved  värde: \code{11}, typ: \code{Int}

\SubtaskSolved  värde: \code{500500}, typ: \code{Int}




\QUESTEND




%%<AUTOEXTRACTED by mergesolu>%%      %Uppgift 2




\WHAT{Datastrukturen \code+Array+.}

\QUESTBEGIN

\Task \label{task:array} \what~   Kör nedan kodrader i Scala REPL. Beskriv vad som händer.

\Subtask \code{val xs = Array("hej","på","dej", "!")}

\Subtask \code{xs(0)}

\Subtask \code{xs(3)}

\Subtask \code{xs(4)}

\Subtask \code{xs(1) + " " + xs(2)}

\Subtask \code{xs.mkString}

\Subtask \code{xs.mkString(" ")}

\Subtask \code{xs.mkString("(", ",", ")")}

\Subtask \code{xs.mkString("Array(", ", ", ")")}

\Subtask \code{xs(0) = 42}

\Subtask \code{xs(0) = "42"; println(xs(0))}

\Subtask \code{val ys = Array(42, 7, 3, 8)}

\Subtask \code{ys.sum}

\Subtask \code{ys.min}

\Subtask \code{ys.max}

\Subtask \code{val zs = Array.fill(10)(42)}

\Subtask \code{zs.sum}

\Subtask\Pen Datastrukturen \code{Range} håller reda på start- och slutvärde, samt stegstorleken för en uppräkning, men alla talen i uppräkningen genereras inte förrän så behövs. En \code{Int} tar 4 bytes i minnet. Ungefär hur mycket plats i minnet tar de objekt som variablerna \code{r} respektive \code{a} refererar till nedan?
\begin{REPL}
scala> val r = (1 to Int.MaxValue by 2)
scala> val a = r.toArray
\end{REPL}
\emph{Tips:} Använd uttrycket \code{ BigInt(Int.MaxValue) * 2 } i dina beräkningar.

\SOLUTION


\TaskSolved \what
 

\SubtaskSolved  Ett objekt av typen \code{Array[String]} skapas med värdet 

\code{Array(hej, på, dej, !)} och med namnet \code{xs}.

\SubtaskSolved  Returnerar en sträng med värdet \code{hej}.

\SubtaskSolved  Returnerar en sträng med värdet \code{!}.

\SubtaskSolved  Ett exception genereras. Skriver ut:

\code{java.lang.ArrayIndexOutOfBoundsException: 4}

\SubtaskSolved  Returnerar en sträng med värdet \code{på dej}.

\SubtaskSolved  Returnerar en sträng med värdet \code{hejpådej!}.

\SubtaskSolved  Returnerar en sträng med värdet \code{hej på dej !}.

\SubtaskSolved  Returnerar en sträng med värdet \code{(hej,på,dej,!)}.

\SubtaskSolved  Returnerar en sträng med värdet \code{Array(hej,på,dej,!)}.

\SubtaskSolved  Ett fel uppstår av typen \code{type mismatch}. Konsollen talar om för oss vad den fick, dvs värdet \code{42} av typen \code{Int}. Den talar även om för oss vad den ville ha, dvs något värde av typen \code{String}. Till sist skriver den ut vår kodrad och pekar ut felet.

\SubtaskSolved  Det första elementet i \code{xs} ändras till värdet \code{42}. Därefter skrivs det första värdet i \code{xs} ut.

\SubtaskSolved  Ett objekt av typen \code{Array[Int]} skapas med värdet \code{Array(42, 7, 3, 8)} och med namnet \code{ys}.

\SubtaskSolved  Returnerar summan av elementen i \code{ys}. Resultatet är \code{60}.

\SubtaskSolved  Returnerar det minsta värdet i \code{ys}. Resultatet är \code{3}.

\SubtaskSolved  Returnerar det största värdet i \code{ys}. Resultatet är \code{42}.

\SubtaskSolved  Ett nytt värde av typen \code{Array[Int]} skapas med \code{10} stycken element, alla med värdet \code{42}.

\SubtaskSolved  Returnerar summan av elementen i \code{zs}. Resultatet blir 420 (42 multiplicerat med 10).

\SubtaskSolved  \code{r} tar upp 12 bytes. \code{a} tar upp ca 4 miljarder bytes.



\QUESTEND




%%<AUTOEXTRACTED by mergesolu>%%      %Uppgift 3




\WHAT{Datastrukturen \code+Vector+.}

\QUESTBEGIN

\Task  \what~  Kör nedan kodrader i Scala REPL. Beskriv vad som händer.

\Subtask \code{val words = Vector("hej","på","dej", "!")}

\Subtask \code{words(0)}

\Subtask \code{words(3)}

\Subtask \code{words.mkString}

\Subtask \code{words.mkString(" ")}

\Subtask \code{words.mkString("(", ",", ")")}

\Subtask \code{words.mkString("Ord(", ", ", ")")}

\Subtask \code{words(0) = "42"}

\Subtask \code{val numbers = Vector(42, 7, 3, 8)}

\Subtask \code{numbers.sum}

\Subtask \code{numbers.min}

\Subtask \code{numbers.max}

\Subtask \code{val moreNumbers = Vector.fill(10000)(42)}

\Subtask \code{moreNumbers.sum}

\Subtask\Pen Jämför med uppgift \ref{task:array}. Vad kan man göra med en \code{Array} som man inte kan göra med en \code{Vector}?

\SOLUTION


\TaskSolved \what
 

\SubtaskSolved  Ett objekt av typen \code{scala.collection.immutable.Vector[String]} initieras med värdet \code{Vector(hej, på dej, !)}.

\SubtaskSolved  Returnerar det nollte elementet i \code{words}, dvs strängen \code{hej}.

\SubtaskSolved  Returnerar det tredje elementet i \code{words}, dvs strängen \code{!}.

\SubtaskSolved  Omvandlar vektorn till en Sträng.

\SubtaskSolved  Samma som ovan, fast den här gången används mellanrum för att seperera elementen.

\SubtaskSolved  Samma som ovan, fast den här gången sepereras elementen av kommatecken istället för mellanrum och dessutom börjar och slutar den resulterande strängen med parenteser.

\SubtaskSolved  Samma som ovan, fast med ordet \code{Ord} tillagt i början av den resulterande strängen.

\SubtaskSolved  Ett fel uppstår. Typen \code{Vector} är immutable. Dess element kan alltså inte bytas ut.

\SubtaskSolved  En ny \code{Vector[Int]} skapas med värdet \code{Vector(42, 7, 3, 8)}. 

\SubtaskSolved  Returnerar summan av vektorn \code{numbers}.

\SubtaskSolved  Returnerar vektorns minsta element.

\SubtaskSolved  Returnerar vektorns största element. 

\SubtaskSolved  En ny vektor skapas innehållandes tiotusen 42or.

\SubtaskSolved  Returnerar summan av vektorns element.

\SubtaskSolved  Byta ut element.



\QUESTEND




%%<AUTOEXTRACTED by mergesolu>%%      %Uppgift 4




\WHAT{\code+for+-uttryck}

\QUESTBEGIN

\Task  \what~ . Evaluera nedan uttryck i Scala REPL. Vad har respektive uttryck för värde och typ?

\Subtask \code{for (i <- Range(1,10)) yield i}

\Subtask \code{for (i <- 1 until 10) yield i}

\Subtask \code{for (i <- 1 until 10) yield i + 1}

\Subtask \code{for (i <- Range(1,10).inclusive) yield i}

\Subtask \code{for (i <- 1 to 10) yield i}

\Subtask \code{for (i <- 1 to 10) yield i + 1}

\Subtask \code{(for (i <- 1 to 10) yield i + 1).sum}

\Subtask \code{for (x <- 0.0 to 2 * math.Pi by math.Pi/4) yield math.sin(x)}


\SOLUTION


\TaskSolved \what
 

\SubtaskSolved  typ: \code{scala.collection.immutable.IndexedSeq[Int]}

värde: \code{Vector(1, 2, 3, 4, 5, 6, 7, 8, 9)}

\SubtaskSolved  typ: \code{scala.collection.immutable.IndexedSeq[Int]}

värde: \code{Vector(1, 2, 3, 4, 5, 6, 7, 8, 9)}

\SubtaskSolved  typ: \code{scala.collection.immutable.IndexedSeq[Int]}

värde: \code{Vector(2, 3, 4, 5, 6, 7, 8, 9, 10)}

\SubtaskSolved  typ: \code{scala.collection.immutable.IndexedSeq[Int]}

värde: \code{Vector(1, 2, 3, 4, 5, 6, 7, 8, 9, 10)}

\SubtaskSolved  typ: \code{scala.collection.immutable.IndexedSeq[Int]}

värde: \code{Vector(1, 2, 3, 4, 5, 6, 7, 8, 9, 10)}

\SubtaskSolved  typ: \code{scala.collection.immutable.IndexedSeq[Int]}

värde: \code{Vector(2, 3, 4, 5, 6, 7, 8, 9, 10, 11)}

\SubtaskSolved  typ: \code{Int}, värde: \code{Vector(65)}

\SubtaskSolved  typ: \code{scala.collection.immutable.IndexedSeq[Int]}

värde: \code{Vector(0.0, 0.707, 1.0, 0.707, 0.0, -0.707, -1.0, -0.707)}



\QUESTEND




%%<AUTOEXTRACTED by mergesolu>%%      %Uppgift 5




\WHAT{Metoden \code+map+ på en samling.}

\QUESTBEGIN

\Task  \what~  Evaluera nedan uttryck i Scala REPL. Vad har respektive uttryck för värde och typ?

\Subtask \code{Range(0,10).map(i => i + 1)}

\Subtask \code{(0 until 10).map(i => i + 1)}

\Subtask \code{(1 to 10).map(i => i * 2)}

\Subtask \code{(1 to 10).map(_ * 2)}

\Subtask \code{Vector.fill(10000)(42).map(_ + 43)}

\SOLUTION


\TaskSolved \what
 

\SubtaskSolved  typ: \code{scala.collection.immutable.IndexedSeq[Int]}

värde: \code{Vector(1, 2, 3, 4, 5, 6, 7, 8, 9, 10)}

\SubtaskSolved  typ: \code{scala.collection.immutable.IndexedSeq[Int]}

värde: \code{Vector(1, 2, 3, 4, 5, 6, 7, 8, 9, 10)}

\SubtaskSolved  typ: \code{scala.collection.immutable.IndexedSeq[Int]}

värde: \code{Vector(2, 4, 6, 8, 10, 12, 14, 16, 18, 20)}

\SubtaskSolved  typ: \code{scala.collection.immutable.IndexedSeq[Int]}

värde: \code{Vector(2, 4, 6, 8, 10, 12, 14, 16, 18, 20)}

\SubtaskSolved  typ: \code{scala.collection.immutable.Vector[Int]}

värde: En vector av tiotusen 85or (85 = 42 + 43).



\QUESTEND




%%<AUTOEXTRACTED by mergesolu>%%      %Uppgift 6




\WHAT{Metoden \code+foreach+ på en samling.}

\QUESTBEGIN

\Task  \what~  Kör nedan satser i Scala REPL. Vad händer?

\Subtask \code{Range(0,10).foreach(i => println(i))}

\Subtask \code{(0 until 10).foreach(i => println(i))}

\Subtask \code|(1 to 10).foreach{i => print("hej"); println(i * 2)}|

\Subtask \code{(1 to 10).foreach(println)}

\Subtask \code{Vector.fill(10000)(math.random).foreach(r => }\\
           \code{      if (r > 0.99) print("pling!"))}


\SOLUTION


\TaskSolved \what
 

\SubtaskSolved  En \code{Range} skapas och dess element skrivs ut ett och ett.

\SubtaskSolved  Samma sak händer.

\SubtaskSolved  De tio första jämna talen (noll ej inräknat) skrivs ut med ett "hej" framför.

\SubtaskSolved  Talen 1 till 10 skrivs ut.

\SubtaskSolved  Tiotusen slumptal mellan 0 och 1 genereras. Varje gång ett tal är större än 0.99 kommer det ett pling.



\QUESTEND




%%<AUTOEXTRACTED by mergesolu>%%      %Uppgift 7




\WHAT{Algoritm: SWAP.}

\QUESTBEGIN

\Task  \what~ 

\Subtask Skriv med \emph{pseudo-kod} algoritmen SWAP. Beskriv på vanlig svenska, steg för steg, hur en variabel $temp$ används för mellanlagring vid värdebytet:

\emph{Indata:} två heltalsvariabler $x$ och $y$

\emph{???}

\emph{Utdata:} variablerna $x$ och $y$ vars värden har bytt plats.

\Subtask Implementerar algoritmen SWAP. Ersätt \code{???} nedan med satser separerade av semikolon:

\begin{REPL}
scala> var (x, y) = (42, 43)
scala> ???
scala> println("x är " + x + ", y är " + y)
x är 43, y är 42
\end{REPL}



\SOLUTION


\TaskSolved \what
 

\SubtaskSolved  Pseudokoden kan se ut såhär:

Skapa heltalsvariabel temp. 
Flytta värdet från x till temp. 
Flytta värdet från y till x. 
Flytta värdet från temp till y.

\SubtaskSolved 
\begin{REPLnonum}
scala> var (x, y) = (42, 43)
x: Int = 42
y: Int = 43
scala> var temp = x; x = y; y = temp;
temp: Int = 42
x: Int = 43
y: Int = 42
scala> println("x är " + x + ", y är " + y)
x är 43, y är 42
\end{REPLnonum}



\QUESTEND




%%<AUTOEXTRACTED by mergesolu>%%      %Uppgift 8




\WHAT{Skript.}

\QUESTBEGIN

\Task  \what~  Skapa en fil med namn \texttt{hello-script.scala} med hjälp av en editor som innehåller denna enda rad:
\begin{Code}
println("hej skript")
\end{Code}
Spara filen och kör kommandot \code{scala hello-script.scala} i terminalen:
\begin{REPLnonum}
> scala hello-script.scala
\end{REPLnonum}

\Subtask Vad händer?

\Subtask Ändra i filen så att högerparentesen saknas. Spara och kör skriptfilen igen. Vad händer?

\Subtask Lägg till en sats sist i skriptet som skriver ut summan av de ett tusen stycken heltalen från och med 2 till och med 1001, så som visas nedan.
\begin{REPL}
> scala hello-script.scala
hej skript
501500
\end{REPL}

\Subtask Ändra i hello-script.scala genom att införa \code{val n = args(0).toInt} och använd \code{n} som övre gräns för summeringen av de n första heltalen.
\begin{REPL}
> scala hello-script.scala 5001
hej skript
12507501
\end{REPL}

\Subtask Vad blir det för felmeddelande om du glömmer ge programmet ett argument?


\SOLUTION


\TaskSolved \what
 

\SubtaskSolved  Skriver ut "hej skript".

\SubtaskSolved  Ett felmeddelande skrivs ut.

\SubtaskSolved  Lägg till raden:
\code{println((2 to 1001).sum)} 
eller motsvarande.

\SubtaskSolved  Filen ska se ut ungefär såhär: \\
\begin{Code} 
val n = args(0).toInt 
println("hej skript") 
println((1 to n).sum)
\end{Code}

\SubtaskSolved  \code{java.lang.ArrayIndexOutOfBoundsException: 0}



\QUESTEND




%%<AUTOEXTRACTED by mergesolu>%%      %Uppgift 9




\WHAT{Applikation med \code+main+-metod.}

\QUESTBEGIN

\Task  \what~  Skapa med hjälp av en editor en fil med namn \texttt{hello-app.scala}.
\begin{REPLnonum}
> gedit hello-app.scala
\end{REPLnonum}
Skriv dessa rader i filen:


\scalainputlisting{examples/hello-app.scala}

\Subtask Kompilera med \code{scalac hello-app.scala} och kör koden med \code{scala Hello}.
\begin{REPLnonum}
> scalac hello-app.scala
> ls
> scala Hello
\end{REPLnonum}
Vad heter filerna som kompilatorn skapar?

\Subtask Ändra i din kod så att kompilatorn ger följande felmeddelande: \\
\texttt{Missing closing brace}

\Subtask\Pen Varför behövs \code{main}-metoden?

\Subtask\Pen Vilket alternativ går snabbast att köra igång, ett skript eller en kompilerad applikation? Varför? Vilket alternativ kör snabbast när väl exekveringen är igång?


\SOLUTION


\TaskSolved \what
 

\SubtaskSolved  Hello.class och Hello\$.class

\SubtaskSolved  Ta bort en av krullparenteserna i slutet.

\SubtaskSolved  I ett skript behöver man inte skriva någon main-metod. Kompilatorn lägger till en automatiskt precis när koden ska köras. I en applikation behöver man däremot det. För att göra en applikation definierar vi ett objekt som vi i det här fallet kallar för \code{Hello}. Från början gör inte objekt någonting. De bara finns. För att objekt ska kunna göra något behövs det metoder. I vanliga fall utförs inte metoder förrän en annan metod "ropar" på metoden. main-metoden ropas dock automatiskt när en applikation startas. Annars hade ju ingenting hänt, eftersom alla metoderna väntar på att någon annan metod ska börja. \\
\SubtaskSolved  Första gången man ska köra en applikation måste den först kompileras innan den exekveras. Skript kompileras automatiskt samtidigt som de exekveras, vilket totalt sett görs på kortare tid. Därför tar det längre tid att starta en applikation första gången än att starta ett skript första gånge. När en applikation väl har kompileras och kan exekveras, går det dock mycket fortare. Fördelen med applikationer är att de kan exekveras flera gånger utan att kompileras om.



\QUESTEND




%%<AUTOEXTRACTED by mergesolu>%%      %Uppgift 10




\WHAT{Java-applikation.}

\QUESTBEGIN

\Task \label{task:java} \what~   Skapa med hjälp av en editor en fil med namn \texttt{Hi.java}.
\begin{REPLnonum}
> gedit Hi.java
\end{REPLnonum}
Skriv dessa rader i filen:

\javainputlisting{examples/Hi.java}

\noindent Kompilera med \code{javac Hi.java} och kör koden med \code{java Hi}.
\begin{REPLnonum}
> javac Hi.java
> ls
> java Hi
\end{REPLnonum}

\Subtask\Pen Vad heter filen som kompilatorn skapat?

\Subtask\Pen Jämför signaturen för Java-programmets main-metod med signaturen för Scala-programmets main-metod. De betyder samma sak men syntaxen är olika. Beskriv skillnader och likheter i syntaxen.

\Subtask\Pen Vad blir det för felmeddelande om källkodsfilen och klassnamnet inte överensstämmer i ett Java-program?


\SOLUTION


\TaskSolved \what
 

\SubtaskSolved  Hi.class

\SubtaskSolved  I javas syntax börjar man med orden \code{public static}. I scala uteblir dessa. I scala är alla metoder automatiskt publika om inget annat används. Därför behövs aldrig ordet \code{public} i scala. I scala finns det tekniskt sett inga statiska metoder. Men i praktiken fungerar vanliga metoder i ett scala-objekt på ungefär samma sätt som statiska metoder i en java-klass. I scala används ordet \code{def} varje gång en funktion ska definieras. I java slipper man det. I java skriver man returtypen (\code{void}) innan parametrarna. I scala kommer istället metodens returtyp (\code{Unit}) i slutet. Javas \code{void} motsvarar scalas \code{Unit}. I scalas syntax kommer parameterns namn (\code{args}) före parameterns typ (\code{Array[String]}), separerat med ett kolon. I java kommer typen (\code{String[]}) först och sen kommer namnet (\code{args}). \code{String[]} i java betyder ungefär samma sak som \code{Array[String]} i scala.

\SubtaskSolved  -



\QUESTEND




%%<AUTOEXTRACTED by mergesolu>%%      %Uppgift 11




\WHAT{Algoritm: SUMBUG}

\QUESTBEGIN

\Task  \what~ . Nedan återfinns pseudo-koden för SUMBUG.

\begin{algorithm}[H]
 \SetKwInOut{Input}{Indata}\SetKwInOut{Output}{Resultat}

 \Input{heltalet $n$}
 \Output{utskrift av summan av de första $n$ heltalen }
 $sum \leftarrow 0$ \\
 $i \leftarrow 1$  \\
 \While{$i \leq n$}{
  $sum \leftarrow sum + 1$
 }
 skriv ut $sum$
\end{algorithm}

\Subtask\Pen Kör algoritmen steg för steg med penna och papper, där du skriver upp hur värdena för respektive variabel ändras. Det finns två buggar i algoritmen. Vilka? Rätta buggarna och test igen genom att ''köra'' algoritmen med penna på papper och kontrollera så att algoritmen fungerar för $n=0$, $n=1$, och $n=5$. Vad händer om $n=-1$?

\Subtask Skapa med hjälp av en editor filen \code{sumn.scala}. Implementera algoritmen SUM enligt den rättade pseudokoden och placera implementationen i en main-metod i ett objekt med namnet \code{sumn}. Du kan skapa indata \code{n} till algoritmen med denna deklaration i början av din main-metod: \\ \code{val n = args(0).toInt} \\ Vad ger applikationen för utskrift om du kör den med argumentet 8888?

\begin{REPLnonum}
> scalac sumn.scala
> scala sumn 8888
\end{REPLnonum}

\Subtask Kontrollera att din implementation räknar rätt genom att jämföra svaret med detta uttrycks värde, evaluerat i Scala REPL:
\begin{REPLnonum}
scala> (1 to 8888).sum
\end{REPLnonum}

\Subtask Implementera algoritmen SUM enligt pseudokoden ovan, men nu i Java. Skapa filen \code{SumN.java} och använd koden från uppgift \ref{task:java} som mall för att deklarera den publika klassen \code{SumN} med en main-metod. Några tips om Java-syntax och standarfunktioner i Java:

\begin{itemize}[noitemsep, nolistsep]
\item Alla satser i Java måste avslutas med semikolon.
\item Heltalsvariabler deklareras med nyckelordet \lstinline[language=Java]{int} (litet i).
\item Typnamnet ska stå \emph{före} namnet på variabeln. Exempel: \\ \lstinline[language=Java]{int sum = 0;}
\item Indexering i en array görs i Java med hakparenteser: \code{args[0]}
\item I stället för Scala-uttrycket \code{args(0).toInt}, använd Java-uttrycket: \\ \code{Integer.parseInt(args[0])}
\item \code{while}-satser i Scala och Java har samma syntax.
\item Utskrift i Java görs med \code{System.out.println}
\end{itemize}


\SOLUTION


\TaskSolved \what
 

\SubtaskSolved  Bugg: Eftersom \code{i} inte ökar, fastnar programmet i en oändlig loop. Fix: Lägg till en sats i slutet av while-blocket som ökar värdet på i med 1.
Bugg: Eftersom man bara ökar summan med 1 varje gång, kommer resultatet att bli summan av n stycken 1or, inte de n första heltalen. Fix: Ändra så att summan ökar med \code{i} varje gång, istället för 1.
För -1, blir resultatet 0. Förklaring: i börjar på 1 och är alltså aldrig mindre än n som ju är -1. while-blocket genomförs alltså noll gånger, och efter att \code{sum} får sitt ursprungsvärde förändras den aldrig.
\SubtaskSolved  39502716
\SubtaskSolved  -
\SubtaskSolved  Såhär kan implementationen se ut:
\begin{Code}
public class SumN {
  public static void main(String[] args) {
    int n = Integer.parseInt(args[0]);
    int sum = 0;
    int i = 1;
    while(i <= n){
      sum = sum + i;
      i = i + 1;
      }
    }
    System.out.println(sum);
}
\end{Code}



\QUESTEND




%%<AUTOEXTRACTED by mergesolu>%%      %Uppgift 12




\WHAT{Algoritm: MAXBUG}

\QUESTBEGIN

\Task  \what~ . Nedan återfinns pseudo-koden för MAXBUG.

\begin{algorithm}[H]
 \SetKwInOut{Input}{Indata}\SetKwInOut{Output}{Resultat}

 \Input{Array $args$ med strängar som alla innehåller heltal}
 \Output{utskrift av största heltalet }
 $max \leftarrow$ det minsta heltalet som kan uppkomma  \\
 $n \leftarrow $ antalet heltal \\
 $i \leftarrow 0$ \\
 \While{$i < n$}{
   $x \leftarrow args(i).toInt$ \\
   \If{( x > $max$)}{$max \leftarrow x$}
  % $i \leftarrow i + 1$
 }
 skriv ut $max$
\end{algorithm}

\Subtask\Pen Kör med penna och papper. Det finns en bugg i algoritmen ovan. Vilken? Rätta buggen.

\Subtask Implementera algoritmen MAX (utan bugg) som en Scala-applikation. Tips:
\begin{itemize}[noitemsep, nolistsep]
\item Det minsta \code{Int}-värdet som någonsin kan uppkomma: \code{Int.MinValue}
\item Antalet element i $args$ ges av: \code{args.size}
\end{itemize}

\begin{REPL}
> gedit maxn.scala
> scalac maxn.scala
> scala maxn 7 42 1 -5 9
42
\end{REPL}

\Subtask\Pen \label{subtask:arg0} Skriv om algoritmen så att variabeln $max$ initialiseras med det första talet i sekvensen.

\Subtask Implementera den nya algoritmvarianten från uppgift \ref{subtask:arg0} och prova programmet. Vad händer om $args$ är tom?

\SOLUTION


\TaskSolved \what
 

\SubtaskSolved  Bugg: i ökar aldrig. Programmet fastnar i en oändlig loop. Fix: Lägg till en sats som ökar i med 1, i slutet av while-blocket.

\SubtaskSolved  Så här kan implementationen se ut:
\begin{Code}
object Max {
  def main(args: Array[String]): Unit = {
    var max = Int.MinValue
    val n = args.size
    var i = 0
    while(i < n) {
      val x = args(i).toInt
      if(x > max) {
        max = x
      }
      i = i + 1
    }
    println(max)
  }
}
\end{Code}
\SubtaskSolved  Raden där max initieras ändras till \code{var max = args(0).toInt} 

\SubtaskSolved  \code{java.lang.ArrayIndexOutOfBoundsException: 0}



\QUESTEND




%%<AUTOEXTRACTED by mergesolu>%%      %Uppgift 13




\WHAT{Block, namnsynlighet, namnöverskuggning}

\QUESTBEGIN

\Task  \what~ . Kör nedan kod i Scala REPL eller i Kojo. Vad händer nedan? Varför?

\Subtask \code|val a = {1 + 1; 2 + 2; 3 + 3; 4 + 4}; println(a)|

\Subtask \code|val b = {1; 2; 3; {val b = 4; b + b; b + 1}}; println(b)|

\Subtask \code|{val a = 42; println(a)}|

\Subtask \code|{val a = 42}; println(a)|

\Subtask \code|{val a = 42; {val a = 43; println(a)}; println(a)}|

\Subtask \code|{var a = 42; {a = a + 1}; var a = 43}|

\Subtask \code|{var a = 42; {a = a + b; var b = 43}; println(a)}|

\Subtask \code|{var a = 42; {var b = 43; a = a + b}; println(a)}|

\Subtask \code|{var a = 42; {a = a + b; def b = 43}; println(a)}|

\Subtask \code|{object a{var b=42;object a{var a=43}};println(a.b+a.a.a)}|

\Subtask

\begin{Code}
{
  object a {
    var b = 42
    object a {
      var a = 43
    }
  }
  println(a.b + a.a.a)
}
\end{Code}

\Subtask Vad är fördelen med att namn deklarerade inne i ett block är lokala i stället för globala?


\SOLUTION


\TaskSolved \what


\SubtaskSolved  Skriver ut talet 8. \code{a} får värdet \code{4 + 4} eftersom detta är den sista satsen i blocket. Man får också tre stycken varningar. Detta beror på att det förekommer tre satser i blocket som inte gör någon skillnad.

\SubtaskSolved  Skriver ut talet 5. De tre första satserna i det yttre blocket ignoreras. \code{b} får värdet som returneras av det yttre blocket. Det yttre blocket returnerar värdet som returneras i den sista satsen i blocket, som i sin tur är ett block. I det inre blocket skapas en ny \code{val} som också får namnet \code{b}. Notera att detta alltså inte är samma värde, även om det har samma namn. Den andra satsen räknar summan av \code{b} med sig själv. Eftersom vi nu befinner oss i det block där det andra \code{b}et precis har definieras så är det detta \code{b} som används och summan blir alltså åtta. Detta är dock helt irrelevant eftersom resultatet inte sparas någonstans. I den sista satsen blir resultatet 5 (eftersom \code{b} är fyra och vi adderar ett). Detta resultatet returneras från det innre blocket och vidare ur det yttre blocket.

\SubtaskSolved  Skriver ut talet 42. Blockets satser exekveras i ordning. 

\SubtaskSolved  Skriver inte ut 42. I blocket skapas ett \code{val} med namnet \code{a} och värdet \code{42}. Detta värde finns inte utanför blocket och kommer därför inte att skrivas ut. Om du däremot definierat \code{a} som något annat tidigare så kommer istället det värdet att skrivas ut.

\SubtaskSolved  Skriver först ut \code{43} och sedan \code{42}. Förklaring:

\code{a} initieras med värdet \code{42}. Ett nytt värde som också har namnet \code{a} initieras med värdet \code{43}. Eftersom detta sker innanför ett nytt block, befinner vi oss i ett annat "namespace" och det gör alltså inget att vi använder samma namn. \code{a} skrivs ut. Eftersom vi befinner oss i det inre blocket är det \code{43} som skrivs ut, inte \code{42}. Scala kollar först efter värden som heter \code{a} i det inre "namespacet". Det är först i andra hand som den skulle upptäcka att det finns ett \code{a} i det yttre blocket. Till sist körs den sista satsen i det yttre blocket. Då skrivs \code{a} ut. Eftersom vi nu befinner oss i det yttre blocket, vet inte ens scala om att det andra \code{a}:et existerar. Resultatet av den här utskriften blir alltså \code{42}.

\SubtaskSolved  Ett fel uppstår. Variabeln \code{a} initieras två gånger i samma namespace. Förklaring till felet:

I det yttre blockets första sats initieras variablen \code{a} med värdet \code{42}. I det yttre blockets tredje sats försöker vi definiera en ny variabel med samma namn. I och med att vi befinner oss i samma namespace, krockar namnen.

Förklaring till vad som händer i sats två:

I det inre blocket har vi inte definierat någon variabel \code{a}. Till en början hittar alltså inte scala något sådant. Då letar scala vidare i det namespace som finns utanför det inre blocket och hittar variabeln som vi definierade i det yttre blockets första sats. Denna variabel får sitt värde förändrat.

\SubtaskSolved  Fel. Framåtreferens. Förklaring:

Det är inte tillåtet att referera till variabler som initieras senare i koden.

\SubtaskSolved  Skriver ut \code{85}. Förklaring:

I och med att vi den här gången initierade variabeln \code{b} och gav den ett värde innan vi använder oss av den, slipper vi problemet ovan.

\SubtaskSolved  Skriver ut \code{85}. Förklaring:

Det är tillåtet att referera till funktioner som definieras senare i koden.

\SubtaskSolved  Skriver ut \code{85}. Förklaring:

\code{a.b} refererar till variabeln \code{b} som ingår i objektet \code{a}.
\code{a.a.a} refererar till variabeln \code{a}, som ingår i ett objekt som heter \code{a} som i sin tur befinner sig i ett annat objekt som också heter \code{a}.

\SubtaskSolved  Skriver ut \code{85}. Förklaring:

Koden är identisk med förra deluppgiften förutom att ny rad används istället för semikolon.

\SubtaskSolved  I stora projekt med mycket kod, kan det vara svårt att hitta unika namn till alla sina variabler. Då är det en fördel om man kan hålla sina variabler i begränsade namespaces, så att de bara är tillgängliga precis när de behöver användas. 



\QUESTEND




%%<AUTOEXTRACTED by mergesolu>%%      %Uppgift 14??? NUMMER I KOMMENTAR STÄMMER EJ MED GENERERAT NUMMER




\WHAT{Paket, \code{import} och klassfilstrukturer.}

\QUESTBEGIN

\Task \label{task:package} \what~   Med Java-8-plattformen kommer 4240 färdiga klasser, som är organiserade i 217 olika paket.\footnote{Se Stackoverflow: \href{http://stackoverflow.com/questions/3112882/how-many-classes-are-there-in-java-standard-edition}{how-many-classes-are-there-in-java-standard-edition}}

\Subtask Vilka paket finns i paketet javax som börjar på s?

\begin{REPLnonum}
scala> javax.s   //tryck på TAB-tangenten
\end{REPLnonum}

\Subtask Kör raderna nedan i REPL. Beskriv vad som händer för varje rad.
\begin{REPL}[numbers=left, numberstyle=\color{black}\ttfamily\scriptsize\selectfont]
scala> import javax.swing.JOptionPane
scala> def msg(s: String) = JOptionPane.showMessageDialog(null, s)
scala> msg("Hej på dej!")
scala> def input(msg: String) = JOptionPane.showInputDialog(null, msg)
scala> input("Vad heter du?")
scala> import JOptionPane.{showOptionDialog => optDlg}
scala> def inputOption(msg: String, opt: Array[Object]) =
         optDlg(null, msg, "Option", 0, 0, null, opt, opt(0))
scala> inputOption("Vad väljer du?", Array("Sten", "Sax", "Påse"))
\end{REPL}

\Subtask\Pen Vad hade du behövt ändra på efterföljande rader om import-satsen på rad 1 ovan ej hade gjorts?

\Subtask Skapa med en editor filen paket.scala och kompilera. Rita en bild av hur katalogstrukturen ser ut.

\begin{Code}
package gurka.tomat.banan

package p1 {
  package p11 {
    object hello {
      def hello = println("Hej paket p1.p11!")
    }
  }
  package p12 {
    object hello {
      def hello = println("Hej paket p1.p12!")
    }
  }
}

package p2 {
  package p21 {
    object hello {
      def hello = println("Hej paket p2.p21!")
    }
  }
}

object Main {
  def main(args: Array[String]): Unit = {
    import p1._
    p11.hello.hello
    p12.hello.hello
    import p2.{p21 => apelsin}
    apelsin.hello.hello
  }
}
\end{Code}

\begin{REPL}
> gedit paket.scala
> scalac paket.scala
> scala gurka.tomat.banan.Main
> ls -R
\end{REPL}

\SOLUTION


\TaskSolved \what
 

\SubtaskSolved  \code{script   security   smartcardio   sound   sql   swing}

\SubtaskSolved  Radernas funktion i ordning:

1. Importerar JOptionPane från javax.swing

2. Definierar en metod som tar en sträng och öppnar en dialogruta med strängen.

3. Testar funktionen med argumentet "Hej på dej!". En dialogruta öppnas med texten "Hej på dej!".

4. Definierar en metod som tar emot en sträng som argument och öppnar en input-dialogruta med strängen.

5. Testar funktionen med argumentet "Vad heter du?". En dialogruta öppnas med texten "Vad heter du?". I ett fält kan man fylla i sitt namn. Funktionen returnerar namnet.

6. Importerar showOptionDialog från JOptionPane under namnet optDlg.

7. Definierar en metod som tar emot en sträng och en Array som argument och öppnar en flervalsdialog. Strängen ska innehålla frågan som flervalsdialogen visar upp. Arrayn ska innehålla alternativen som användaren ska välja mellan.

8.Testar funktionen med argumenten \code{"Vad väljer du?"} och \\ \code{Array("Sten, "Sax", "Påse")}. En dialogruta kommer upp och man får möjlighet att välja sten sax eller påse. Funktionen returnerar valet som man gör.

\SubtaskSolved  På alla ställen där \code{JOptionPane} förekommer, hade man istället fått skriva \code{javax.swing.JOptionPane}.

\SubtaskSolved  -



\QUESTEND




%%<AUTOEXTRACTED by mergesolu>%%      %Uppgift 15




\WHAT{Skapa \code{jar}-filer och använda classpath}

\QUESTBEGIN

\Task  \what~ 

\Subtask Skriv kommandot \code{jar} i terminalen och undersök vad som finns för optioner. Se speciellt ''Example 1.'' i hjälputskriften. Vilket kommando ska du använda för att packa ihop flera filer i en enda jar-fil?

\Subtask Som en fortsättning på uppgift \ref{task:package}, packa ihop biblioteket \code{gurka} i en jar-fil med nedan kommando, samt kör igång REPL med jar-filen på classpath.

\begin{REPL}
> jar cvf mittpaket.jar gurka
> scala -cp mittpaket.jar
scala> gurka.tomat.banan.Main.main(Array())
\end{REPL}


\SOLUTION


\TaskSolved \what
 

\SubtaskSolved  jar cvf [namn på skapad fil] [namn på input-filer]

\SubtaskSolved  -



\QUESTEND




%%<AUTOEXTRACTED by mergesolu>%%      %Uppgift 16




\WHAT{Skapa dokumentation med \code{scaladoc}-kommandot}

\QUESTBEGIN

\Task  \what~ 

\Subtask Som en fortsättning på uppgift \ref{task:package}, kör nedan kommando i terminalen:

\begin{REPL}
> scaladoc paket.scala
> ls
> firefox index.html   # eller öppna index.html i valfri webbläsare
\end{REPL}

Vad händer?

\Subtask Lägg till några fler metoder i något av objekten i filen \code{paket.scala} och lägg även till några dokumentationskommentarer. Kompilera om och kör. Generera om dokumentationen.

\begin{verbatim}
//... ändra i filen paket.scala

/** min paketdokumentationskommentar p2 */
package p2 {
  /** min paketdokumentationskommentar p21 */
  package p21 {
    /** ett hälsningsobjekt */
    object hello {
      /** en hälsningsmetod i p2.p21 */
      def hello = println("Hej paket p2.p21!")

      /** en metod som skriver ut tiden */
      def date = println(new java.util.Date)
    }
  }
}

\end{verbatim}

\begin{REPL}
> gedit paket.scala
> scalac paket.scala
> jar cvf mittpaket.jar gurka
> scala -cp mittpaket.jar
scala> gurka.tomat.banan.p2.p21.hello.date
scala> :q
> scaladoc paket.scala
> firefox index.html
\end{REPL}

\newpage

\ExtraTasks %%%%%%%%%%%%%%%%%%%

\SOLUTION


\TaskSolved \what
 

\SubtaskSolved  -

\SubtaskSolved  -
\QUESTEND






\WHAT{NEEDS A TOPIC DESCRIPTION}

\QUESTBEGIN

\Task \label{task:minindex} \what~  Implementera algoritmen MININDEX som söker index för minsta heltalet i en sekvens. Pseudokod för algoritmen MININDEX:

\begin{algorithm}[H]
 \SetKwInOut{Input}{Indata}\SetKwInOut{Output}{Utdata}

 \Input{Sekvens $xs$ med $n$ st heltal.}
 \Output{Index för det minsta talet eller $-1$ om $xs$ är tom.  }
 $minPos \leftarrow 0 $\\
 $i \leftarrow 1$ \\
 \While{$i < n$}{
   \If{xs(i) < $xs(minPos)$}{$minPos \leftarrow i$}
   $i \leftarrow i + 1$
 }
 \eIf{$n > 0$}{\Return{$minPos$}}{\Return{$-1$}}
\end{algorithm}

\Subtask Prova algoritmen med penna och papper på sekvensen $(1, 2, -1, 4)$ och rita minnessituationen efter varje runda i loopen. Vad blir skillnaden i exekveringsförloppet om loopvariablen $i$  initialiserats till $0$ i stället för $1$?

\Subtask Implementera algoritmen MININDEX i Scala i en funktion med denna signatur:
\begin{Code}
def indexOfMin(xs: Array[Int]): Int = ???
\end{Code}
Testa för olika fall: tom sekvens; sekvens med endast ett tal; lång sekvens med det minsta talet först, någonstans mitt i, samt sist.

\begin{Code}
// kod till facit
def indexOfMin(xs: Array[Int]): Int = {
  var minPos = 0
  var i = 1
  while (i < xs.size) {
    if (xs(i) < xs(minPos)) minPos = i
    i += 1
  }
  if (xs.size > 0) minPos else -1
}


\end{Code}

\newpage

\AdvancedTasks %%%%%%%%%%%%%%%%%


\SOLUTION


\QUESTEND






\WHAT{NEEDS A TOPIC DESCRIPTION}

\QUESTBEGIN

\Task  \what~ Läs om krullparenteser och vanliga parenteser på stack overflow: \\ \href{http://stackoverflow.com/questions/4386127/what-is-the-formal-difference-in-scala-between-braces-and-parentheses-and-when}{stackoverflow.com/questions/4386127/what-is-the-formal-difference-in-scala-between-braces-and-parentheses-and-when} och prova själv i REPL hur du kan blanda dessa olika slags parenteser på olika vis.

\SOLUTION


\QUESTEND






\WHAT{Tips:}

\QUESTBEGIN

\Task  \what~ Gör jämförande studier av Scalas api-dokumentation för \code{ArrayBuffer}, \code{Array} och \code{Vector}. Ge exempel på metoder som finns på objekt av typen \code{Array} och \code{ArrayBuffer} men inte på objekt av typen \code{Vector}.  Kolla efter metoder som returnerar \code{Unit}. Prova några muterande metoder på \code{Array} och \code{ArrayBuffer} i REPL.

\SOLUTION


\QUESTEND






\WHAT{Tips:}

\QUESTBEGIN

\Task  \what~ Bygg vidare på koden nedan och gör ett Sten-Sax-Påse-spel\footnote{\href{https://sv.wikipedia.org/wiki/Sten,\_sax,\_p\%C3\%A5se}{sv.wikipedia.org/wiki/Sten,\_sax,\_p\%C3\%A5se}} som även meddelar vem som vinner. Koden fungerar att köra som den är, men funktionen \code{winnerMsg} är ej klar.  Du kan använda modulo-räkning med \code{%}-operatorn för att avgöra vem som vinner.

\begin{Code}[basicstyle=\ttfamily\footnotesize\selectfont]]
object Rock {
  import javax.swing.JOptionPane
  import JOptionPane.{showOptionDialog => optDlg}

  def inputOption(msg: String, opt: Vector[String]) =
    optDlg(null, msg, "Option", 0, 0, null, opt.toArray[Object], opt(0))

  def msg(s: String) = JOptionPane.showMessageDialog(null, s)

  val opt =  Vector("Sten", "Sax", "Påse")

  def userChoice = inputOption("Vad väljer du?", opt)

  def computerChoice = (math.random * 3).toInt

  def winnerMsg(user: Int, computer: Int) = "??? vann!"

  def main(args: Array[String]): Unit = {
    var keepPlaying = true
    while (keepPlaying) {
      val u = userChoice
      val c = computerChoice
      msg("Du valde " + opt(u) + "\n" +
          "Datorn valde " + opt(c) + "\n" +
          winnerMsg(u, c))
      if (u != c) keepPlaying = false
    }
  }
}
\end{Code}\SOLUTION


\QUESTEND


\input{modules/w02-programs-lab.tex}

%!TEX encoding = UTF-8 Unicode
%!TEX root = ../compendium1.tex

\renewcommand{\vecka}{2}

\chapter{Funktioner, Objekt}\label{chapter:W03}
\begin{multicols}{2}\begin{itemize}[nosep,label={$\square$}]
\item definera funktion
\item anropa funktion
\item parameter
\item returtyp
\item värdeandrop
\item namnanrop
\item default-argument
\item namngivna argument
\item applicera funktion på alla element i en samling
\item procedur
\item typen Unit och värdet ()
\item värdeanrop vs namnanrop
\item uppdelad parameterlista
\item skapa egen kontrollstruktur
\item objekt
\item modul
\item punktnotation
\item tillstånd
\item funktionsvärde
\item funktionstyp
\item äkta funktion
\item stegad funktion
\item apply
\item lazy val
\item aktiveringspost
\item rekursion
\item basfall
\item anropsstacken
\item objektheapen\end{itemize}\end{multicols}

\clearpage\section{Teori}
%!TEX encoding = UTF-8 Unicode
%!TEX root = ../lect-w03.tex


% \ifkompendium\else
% \Subsection{Scala 2.12.10}

% \begin{SlideExtra}{Scala 2.12.10}
%   Den 10:e September släpptes Scala 2.12.10 med viktigaste förbättringen att filtrering i Scaladoc nu fungerar igen. 
%   \\~\\
%   \url{https://www.scala-lang.org/news/2.12.10}
%   \\~\\ Du kan gärna installera 2.12.10 om du vill, men inte nödvändigt.
%   \\~\\ När du kollar api-dokumentationen för Scalas standardbibliotek så använd denna länk:\\ 
%   \url{https://www.scala-lang.org/api/2.12.10}

% \end{SlideExtra}
% \fi

\Subsection{Abstraktion}

\begin{Slide}{Vad är abstraktion?}
\begin{itemize}
  \item \Emph{Abstraktion} innebär att skapa en förenklad \Emph{modell} ur konkreta detaljer 
  \item Vi ''hittar på'' nya \Emph{begrepp} som ger oss återanvändbara ''byggblock'' för våra tankar och vår kommunikation
  \item Vi får ett abstrakt \Emph{namn} som kan användas i stället för en massa \Alert{konkreta detaljer}
  \item Skilj på abstraktionens \Emph{namn} (begrepp, koncept), dess \Emph{användning} (anrop) och dess detaljerade \Emph{beskrivning} (definition, implementation)
  \item \Emph{Funktioner} (som du redan känner från matematiken) är en av våra \Alert{viktigaste} abstraktionsmekanismer
\end{itemize}
\url{https://sv.wikipedia.org/wiki/Abstraktion}
\url{https://en.wikipedia.org/wiki/Abstraction}
\end{Slide}

\begin{Slide}{Exempel på abstraktionsmekanismer inom datavetenskapen}
Vi kommer att behandla flera olika, alltmer \Emph{kraftfulla} abstraktionsmekanismer i denna kurs:
\begin{itemize}
  \item Funktioner
  \item Objekt
  \item Klasser
  \item Arv
  \item Generiska strukturer
  \item Kontextuella abstraktioner
\end{itemize}
Dessa abstraktionsmekanismer blir \Emph{extra kraftfulla} om de \Alert{kombineras}! 
\end{Slide}

\Subsection{Vad är en funktion?}


\ifkompendium\else
\begin{SlideExtra}{Om veckans tema: Funktioner}
\begin{itemize}
  \item Funktioner är en av de viktigaste abstraktionsmekanismerna inom datavetenskapen
  \item Du kan redan massor om funktioner, bl.a. från matematiken.
  \item Denna vecka ska vi fördjupa förståelsen:
  \begin{itemize}\SlideFontSize{6}{8}
    \item överlagring
    \item enhetlig access
    \item defaultargument
    \item namngivna argument
    \item lokala funktioner
    \item funktioner som äkta värden
    \item anonyma funktioner
    \item klammerparentes vid ensam paramenter
    \item multipla parameterlistor
    \item egendefinierade kontrollstrukturer
    \item fördröjd evaluering (''call-by-name'')
    \item stegade funktioner (''Curry-funktioner'')
    \item fångad variablelrymd (''closure'')
    \end{itemize}
\end{itemize}  
\end{SlideExtra}

\begin{SlideExtra}{Om veckans tema: Funktioner}
  \includegraphics[width=1.0\textwidth]{../img/coffee-grinder}
\end{SlideExtra}
\fi


\begin{Slide}{Funktion: deklaration och anrop}
\SlideOnly{\setlength{\leftmargini}{0pt}}

\code{def} funktionsnamn(parameterdeklarationer): returtyp = uttryck
\vspace{0.5em}

% \begin{Code}
% def namn(param1: Typ1, param2: Typ2): Returtyp = uttryck
% \end{Code}

\begin{itemize}\SlideFontSmall
  \item En funktion har ett \Emph{huvud} och efter \code{=} kommer dess \Emph{kropp}.
  \item En \Alert{namngiven} funktion \Emph{deklareras} med nyckelordet \code{def}
  \item En funktion kan ha \Emph{parametrar} som deklareras i huvudet. 
  \item \Alert{Kroppen} ska vara ett \Emph{uttryck} (ev. ett block med flera uttryck).
  \item \Emph{Parametrar} binds till \Emph{argument} vid \Alert{anrop}.
  \item Uttrycket i funktionens kropp \Emph{evalueras} vid \Alert{varje anrop}. 
  \item Värdet av uttrycket blir funktionens \Emph{returvärde}. 
\end{itemize}

\pause
Exempel:
\begin{Code}
def öka(a: Int, b: Int): Int = a + b
\end{Code}
\pause
\begin{REPLnonum}
scala> öka(42, 1)
val res0: Int = 43
\end{REPLnonum}

\end{Slide}


\begin{Slide}{Deklarera funktioner, överlagring}
\begin{itemize}
\item Överlagrade funktioner i samma namnrymd:
\begin{REPL}
scala> object matte:
         def öka(a: Int): Int = a + 1
         def öka(a: Int, b: Int): Int = a + b

scala> matte.öka(1)
val res0: Int = 2

scala> matte.öka(1, 2)
val res1: Int = 3

\end{REPL}
\item Båda funktionerna ovan kan finnas samtidigt! Trots att de har \Emph{samma namn} är de \Alert{olika funktioner}; kompilatorn kan skilja dem åt med hjälp av de \Alert{olika parameterlistorna}.

\item Detta kallas \Emph{överlagring} \Eng{overloading} av funktioner.
\item Överlagring ger \Alert{flexibilitet i användningen}; vi slipper hitta på nytt namn så som \code{öka2} vid 2 parametrar.
\end{itemize}
\end{Slide}



\begin{Slide}{Funktioner med defaultargument}\SlideFontSmall

\begin{itemize}
\item Vi kan ofta åstadkomma samma flexibilitet som vid överlagring, men med \Alert{en enda} funktion, om vi i stället använder \Emph{defaultargument}:
\begin{REPLnonum}
scala> def inc(a: Int, b: Int = 1) = a + b

scala> inc(42, 2)
val res0: Int = 44

scala> inc(42, 1)
val res1: Int = 43

scala> inc(42)
val res2: Int = 43

\end{REPLnonum}
\item Om ett argument utelämnas och parametern deklarerats med defaultargument så appliceras detta. Kompilatorn fyller alltså i argumentet åt oss, om det är \Alert{entydigt} vilken parameter som avses.
\end{itemize}
\end{Slide}


\begin{Slide}{Funktioner med namngivna argument}
\begin{itemize}\SlideFontTiny
\item Genom att använda \Emph{namngivna argument} behöver man inte hålla reda på ordningen på parametrarna, bara man känner till parameternamnen.
\item Namngivna argument går fint att \Alert{kombinera} med defaultargument.
\begin{REPLnonum}[basicstyle=\SlideFontSize{7}{9}\ttfamily\color{white}]
scala> def namn(
         förnamn: String,
         efternamn: String,
         förnamnFörst: Boolean = true,
         ledtext: String = "Namn:"
       ): String = 
         if förnamnFörst 
         then s"$ledtext $förnamn $efternamn"
         else s"$ledtext $efternamn, $förnamn"

scala> namn(ledtext = "Name:", efternamn = "Coder", förnamn = "Kim")
val res0: String = Name: Kim Coder
\end{REPLnonum}
\end{itemize}
\end{Slide}


\begin{Slide}{Enhetlig access}\SlideFontSmall
\begin{itemize}
\item Om en funktion \Emph{deklareras} \Alert{med} tom parameterlista \code{()} så \emph{ska} den \Emph{anropas} \Alert{med} tom parameterlista. \hfill (Undantag: Java-metoder)
\begin{REPLsmall}
scala> def tomParameterlista() = 42

scala> tomParameterlista()
val res1: Int = 42

scala> tomParameterlista                                                                                                                    
1 |tomParameterlista
  |^^^^^^^^^^^^^^^^^
  |method tomParameterlista must be called with () argument
\end{REPLsmall}

\item En parameterlös funktion deklarerad \Alert{utan} \code{()} ska anropas \Alert{utan} \code{()}. 
\begin{REPLsmall}
scala> def ingenParameterlista = 42
scala> ingenParameterlista()
1 |ingenParameterlista()
  |^^^^^^^^^^^^^^^^^^^
  |method ingenParameterlista does not take parameters
\end{REPLsmall}
\pause
\item Deklaration utan \code{()} möjliggör \Emph{enhetlig access}: implementationen kan ändras från \code{val} till \code{def} eller tvärtom, \Emph{utan} att \Alert{användandet} påverkas.
%\item Om parameterlista saknas får man alltså \Alert{inte} använda \code{()} vid anrop:

\end{itemize}
\end{Slide}

\Subsection{Hur ser det ut i minnet när funktioner anropas?}

\begin{Slide}{Anropsstacken och objektheapen}\SlideFontSmall
Minnet som innehåller ett programs data är uppdelat i två delar:
\begin{itemize}
\item \Emph{Anropsstacken}: 
\begin{itemize}\SlideFontSmall
\item På anropsstacken läggs en \Emph{aktiveringspost} \Eng{stack frame\footnote{\href{https://en.wikipedia.org/wiki/Call_stack}{en.wikipedia.org/wiki/Call\_stack}}, activation record} för varje funktionsanrop med plats för \Alert{parametrar} och \Alert{lokala variabler}.
\item Aktiveringsposten \Alert{raderas} när \Emph{returvärdet} har levererats.
\item Stacken \Alert{växer} vid \Emph{nästlade funktionsanrop}, då en funktion i sin tur anropar en annan funktion.
\end{itemize}
\item \Emph{Objektheapen}: I objektheapen\footnote{\href{https://en.wikipedia.org/wiki/Memory_management}{en.wikipedia.org/wiki/Memory\_management}}$^{,}$\footnote{Ej att förväxlas med datastrukturen heap  \href{https://sv.wikipedia.org/wiki/Heap}{sv.wikipedia.org/wiki/Heap}} sparas alla objekt (data) som allokeras under körning. Heapen städas då och då av \Emph{skräpsamlaren} \Eng{garbage collector}, och minne som inte används längre frigörs. \\\vspace{0.5em}
%\href{http://stackoverflow.com/questions/1565388/increase-heap-size-in-java}{stackoverflow.com/questions/1565388/increase-heap-size-in-java}
% \href{https://stackoverflow.com/questions/1441373/increase-jvm-heap-size-for-scala}{stackoverflow.com/questions/1441373/increase-jvm-heap-size-for-scala}
% \begin{REPLnonum}
% scala -J-Xmx16g -J-XX:-UseGCOverheadLimit Main
% \end{REPLnonum}
\end{itemize}
\end{Slide}

% \begin{Slide}{Aktiveringspost}\SlideFontSmall
% Nästlade anrop ger växande anropsstack.
% \begin{REPL}
% scala> def f(x: Int, y: Int): Int = { val z = x + y; println(z); z}
% scala> def g(a: Int, b: Int): Int = { val c = a + b; println(c); f(c, 2 * c) }
% scala> def h(i: Int): Int = { val n = 5; g(i, i * n) }
% scala> h(2)
% \end{REPL}
%
% \pause
% \Alert{Stacken}
%
% \begin{tabular}{|r | l | l |} \hline
%
% variabel & värde & Aktiveringspost för anrop av... \\ \hline \hline
% \pause
%  i & 2 & h \\
%  n & 5 & \\ \hline
%  \pause
%  a & 2 & g \\
%  b & 10 &  \\
%  c & 12  &  \\  \hline
%  \pause
%  x & 12  & f \\
%  y & 24 &  \\
%  z & 36 & \\ \hline
% \end{tabular}
% \end{Slide}

\begin{Slide}{Anropsstacken och aktiveringsposter}\SlideFontSmall
Nästlade anrop ger växande anropsstack. Vid varje anrop allokeras en s.k. \Emph{aktiveringspost} \Eng{activation record} med plats i minnet för parametrar, lokala variabler och ev. returvärde. När funktionen är klar så raderas aktiveringsposten och stacken krymper.
\begin{REPLsmall}
scala> def h(x: Int, y: Int): Unit = { val z = x + y; println(z) }
       def g(a: Int, b: Int): Unit = { val x = 1; h(x + 1, a + b) }
       def f(): Unit = { val n = 5; g(n, 2 * n) }

scala> f()

\end{REPLsmall}

\pause
\Alert{Stacken} med 3 aktiveringsposter då f anropar g som anropar h:

\begin{tabular}{|r | l | l |} \hline

variabel & värde & Anrop av... \\ \hline \hline
\pause
 n & 5 & f \\ \hline
 \pause
 a & 5 & g \\
 b & 10 &  \\
 x & 1  &  \\  \hline
 \pause
 x & 2  & h \\
 y & 15 &  \\
 z & 17 & \\ \hline
\end{tabular}
\end{Slide}

% \begin{Slide}{Anropsstacken i Kojo Desktop}
% Tryck på orange playknapp i Kojo och se anropsstacken.\vspace{0.5em}

% \includegraphics[width=1.0\textwidth]{../img/kojo-trace.png}  
% \end{Slide}

% \begin{Slide}{Anropsstacken i VS Code}
% Lägg till en brytpunkt på rad 4 nedan och klicka på debug över \code{@main} i VS Code och se anropsstacken.\vspace{0.5em}

% \includegraphics[width=0.85\textwidth]{../img/vscode-trace.png}  
% \end{Slide}


\begin{Slide}{Vad är en stack trace?}\SlideFontSmall
När du letar buggar vid körtidsfel har du nytta av att \Alert{noga studera} \Emph{utskriften av anropsstacken} \Eng{stack trace}:
% \begin{CodeSmall}
% // Program i filen BMI.scala
% object BMI: 
%   def main(args: Array[String]): Unit = 
%     println(bmi(args(0).toInt, args(1).toInt))

%   def bmi(heightCm: Int, weightKg: Int) = 
%     safeDiv(weightKg, heightCm * heightCm) 

%   def safeDiv(numerator: Int, denominator: Int): (Int, String) = 
%     if (denominator == 0) (numerator / denominator, "")  // ser du buggen?
%     else (0, "division by zero") 
% \end{CodeSmall}
  
\begin{Code}[numbers=left]
// Program i filen bmi.scala

@main 
def bmi(heightCm: Int, weightKg: Int) = 
  safeDiv(weightKg, heightCm * heightCm) 

def safeDiv(numerator: Int, denominator: Int): (Int, String) = 
  if denominator == 0 then (numerator / denominator, "")  // ser du buggen?
  else (0, "division by zero")

\end{Code}
\begin{REPL}
> scala run bmi.scala -- 0 42
Exception in thread "main" java.lang.ArithmeticException: / by zero
        // HÄR KOMMER STACK TRACE pga körtidsfel - se nästa bild
\end{REPL}
\end{Slide}

\begin{Slide}{Hur läsa en stack trace?}
\begin{REPL}
Exception in thread "main" java.lang.ArithmeticException: / by zero
        at bmi$package$.safeDiv(bmi.scala:8)
        at bmi$package$.bmi(bmi.scala:5)
        at bmi.main(bmi.scala:3)

\end{REPL}
\begin{itemize}\SlideFontSmall
  \item En \Emph{stack trace} skrivs ut efter en krasch p.g.a. körtidsfel.
  \item Körtidsfel känns igen med ordet \Alert{Exception}.
  \item Först kommer en beskrivning av felet som orsakat kraschen, här: \\\code{java.lang.ArithmeticException: / by zero} 
  \item Därefter visas anropsstacken.
  \item För varje funktionsanrop anges: \Emph{\texttt{klass.metod(kodfil:radnummer)}}
  \item Main-funktioner läggs i ett singelobjekt i ett speciellt paket
  \item Singelobjekt i Scala kodas som en Java-klass med dollar-tecken efter namnet, eftersom det inte finns singelobjekt i JVM.
  %\item Efter anropet av din \code{main}-procedur ligger JVM-interna anrop (\code{java.base} etc.) som du inte behöver bry dig om. %% syns inte längre i Scala 3 stack trace
\end{itemize}
\end{Slide}
  

\begin{Slide}{Lokala funktioner}\SlideFontSmall
Med lokala funktioner kan delproblem lösas med nästlade abstraktioner.

\begin{CodeSmall}
def gissaTalet(max: Int, min: Int = 1): Unit = 
  def gissat = io.StdIn.readLine(s"Gissa talet mellan $min och $max: ").toInt

  val hemlis = (math.random() * (max - min) + min).toInt

  def skrivLedtrådOmEjRätt(gissning: Int): Unit =
    if gissning > hemlis then println(s"$gissning är för stort :(")
    else if gissning < hemlis then println(s"$gissning är för litet :(")

  def inteRätt(gissning: Int): Boolean = 
    skrivLedtrådOmEjRätt(gissning)
    gissning != hemlis
  

  def loop: Int = { var i = 1; while inteRätt(gissat) do i += 1; i }

  println(s"Du hittade talet $hemlis på $loop gissningar :)")
\end{CodeSmall}

Lokala, nästlade funktionsdeklarationer är tyvärr inte tillåtna i många andra språk, t.ex. Java.\footnote{\href{http://stackoverflow.com/questions/5388584/does-java-support-inner-local-sub-methods}{\SlideFontSize{8}{9}stackoverflow.com/questions/5388584/does-java-support-inner-local-sub-methods}}

\end{Slide}

\Subsection{En funktion är ett värde}

\begin{Slide}{Funktioner är äkta värden i Scala}\SlideFontSmall
\begin{itemize}
\item En funktion är ett \Alert{äkta värde}.
\item Vi kan till exempel tilldela en variabel ett \Emph{funktionsvärde}.
\pause
\item Med hjälp \Alert{enbart} \Emph{funktionsnamnet} får vi funktionen som har ett \Alert{värde} (inga argument har applicerats än):
\begin{REPLnonum}
scala> def add(a: Int, b: Int) = a + b

scala> val f = add  
val f: (Int, Int) => Int = Lambda7210/0x0000000841e4e040@1ce2db23

scala> f(21, 21)
val res0: Int = 42
\end{REPLnonum}
\item Ett funktionsvärde har en \Alert{typ} precis som alla värden: \\
\code{f: (Int, Int) => Int}
\pause
\item Ett funktionsvärde har till skillnad från en funktionsdeklaration inget namn (variabeln \code{f} har ett namn, men inte själva funktionen). Den kallas därför en \Emph{anonym} funktion eller \Alert{lambda} (mer om detta snart).
\end{itemize}
\end{Slide}

\begin{Slide}{Funktionsvärden kan vara argument}\SlideFontSmall
Funktioner kan ha funktioner som parametrar:
\begin{REPL}
scala> def tvåGånger(x: Int, f: Int => Int) = f(f(x))

scala> def öka(x: Int) = x + 1

scala> def minska(x: Int) = x - 1

scala> tvåGånger(42, öka)
val res1: Int = 44

scala> tvåGånger(42, minska)
val res1: Int = 40
\end{REPL}
En funktion som har funktionsvärden som indata (eller utdata) kallas en\\ \Emph{högre ordningens funktion}  \Eng{higher-order function}.

\end{Slide}



\begin{Slide}{Applicera funktioner på element i samlingar med \texttt{map}}\SlideFontSmall
\begin{Code}
def öka(x: Int) = x + 1

def minska(x: Int) = x - 1

val xs = Vector(1, 2, 3)
\end{Code}
\pause
Metoden \Emph{\texttt{map}} fungerar på alla Scala-samlingar och tar \Emph{en funktion som argument} och applicerar denna funktion på alla element och \Alert{skapar en ny samling} med resultaten:
\begin{REPL}
scala> xs.map(öka)
val res0: ???   // vad blir resultatet?

scala> xs.map(minska)
val res1: ???   // vad blir resultatet?
\end{REPL}
\end{Slide}


\begin{Slide}{Applicera funktioner på element i samlingar med \texttt{map}}\SlideFontSmall
\begin{Code}
def öka(x: Int) = x + 1

def minska(x: Int) = x - 1

val xs = Vector(1, 2, 3)
\end{Code}
Metoden \Emph{\texttt{map}} fungerar på alla Scala-samlingar och tar \Emph{en funktion som argument} och applicerar denna funktion på alla element och \Alert{skapar en ny samling} med resultaten:
\begin{REPL}
scala> xs.map(öka)
val res0: scala.collection.immutable.Vector[Int] = Vector(2, 3, 4)

scala> xs.map(minska)
val res1: scala.collection.immutable.Vector[Int] = Vector(0, 1, 2)
\end{REPL}
Metoden \Emph{\texttt{map}} är en smidig och ofta använd \Alert{högre ordningens funktion}.
\end{Slide}

\Subsection{Äkta funktioner}

\begin{Slide}{Äkta funktioner}
\begin{itemize}\SlideFontSmall
\item En \Emph{äkta} \Eng{pure} funktion är en funktion som ger ett resultat som \Alert{enbart} beror av dess argument. Alltså som funktioner i matematiken.
\item En äkta (matematisk) funktion är \Emph{referentiellt transparent} \Eng{referentially transparent}. Det innebär att \Alert{varje anrop} \Emph{kan bytas ut} mot \Alert{värdet av} \Emph{funktionskroppen} där parametrarna ersatts med motsvarande argument före evaluering.
\item En äkta funktion har \Alert{inga sidoeffekter}, t.ex. utskrift, skriva/läsa filer,  eller uppdateringar av variabler \Alert{synliga utanför} funktionen.
\item Exempel:
\begin{Code}
def add(x: Int, y: Int): Int = x + y              // äkta funktion
def rnd(n: Int): Int = (math.random() * n).toInt  // oäkta funktion
\end{Code} 
\begin{itemize}\SlideFontTiny

\item Uttrycket \code{add(41, 1)} kan ersättas med 41 +1 som i sin tur kan ersättas med 42 utan att det påverkar resultatet. Resultatet av \code{add(41, 1)} blir \Emph{samma varje gång} funktionen appliceras med dessa argument
\item Uttrycket \code{rnd(42)} kan \Alert{inte} bytas ut mot ett specifikt värde. \\Alltså: \emph{ej referentiellt transparent}.
\end{itemize}  
\end{itemize}  
\end{Slide}

\begin{Slide}{Exempel på oäkta funktioner: slumptal}

  \begin{itemize}
    \item Funktioner vars värden på något sätt beror av slumpen är \Alert{inte} äkta funktioner.
    \item Även om samma argument ges vid upprepad applicering, så kan ju resultatet bli olika.
    \item Studera dokumentationen för \code{scala.util.Random} här:\\ \href{https://www.scala-lang.org/api/current/scala/util/Random.html}{\SlideFontSmall https://www.scala-lang.org/api/current/scala/util/Random.html}
    \item Du har nytta av funktionen \code{Random.nextInt} och slumptalsfrö \Eng{random seed} i veckans uppgifter.
  \end{itemize}

\end{Slide}

\begin{Slide}{Slumptalsfrö: få samma slumptal varje gång}\SlideFontTiny
\begin{itemize}
\item Om man använder slumptal kan det vara svårt att leta buggar, eftersom det blir \Alert{olika varje gång} man kör programmet och buggen kanske bara uppstår ibland.

\item Med klassen \code{scala.util.Random} kan man skapa \Emph{pseudo}-slumptalssekvenser.
\pause
\item Om man ger ett s.k. \Emph{frö} \Eng{seed}, av heltalstyp, som argument till konstruktorn när man skapar en instans av klassen \code{scala.util.Random}, får man samma ''slumpmässiga'' sekvens \Alert{varje gång} man kör programmet.

\begin{Code}
  val seed = 42
  val rnd = util.Random(seed) // skapa ny slumpgenerator med frö 42
  val r = rnd.nextInt(6)      // något av heltalen 0, 1, 2, 3, 4, 5
\end{Code}
\pause
\item Om man \Alert{inte} ger ett \Emph{frö} så sätts fröet till ''\emph{a value very likely to be distinct from any other invocation of this constructor}''. Då vet vi inte vilket fröet blir och det blir olika varje gång man kör programmet.
\begin{Code}
  val rnd = util.Random() // OLIKA frö vid varje programkörning
  val r = rnd.nextInt(6) 
\end{Code}
\end{itemize}
\end{Slide}

%\begin{Slide}{Syresättning av hjärnan vid sövande föreläsning}
%Prova nedan kod som finns här:\\
%%\href{https://github.com/lunduniversity/introprog/blob/master/compendium/examples/workspace/w05-seqalg/src/NanananananananaNanananananananaBatman.scala}{\SlideFontTiny github.com/lunduniversity/introprog/.../NanananananananaNanananananananaBatman.scala} \\
%
%
%
%\vspace{0.65em}\scalainputlisting[numbers=left,numberstyle=,basicstyle=\fontsize{6.5}{8}\ttfamily\selectfont]{../compendium/examples/workspace/w05-seqalg/src/FixSleepyBrain.scala}
%
%\pause
%Medan du lyssnar till: \href{https://www.youtube.com/watch?v=zUwEIt9ez7M}{\SlideFontSmall www.youtube.com/watch?v=zUwEIt9ez7M}\\
%Eller: \href{https://www.youtube.com/watch?v=rvXxlXg_V-k}{\SlideFontSmall www.youtube.com/watch?v=rvXxlXg\_V-k}
%\end{Slide}

\Subsection{Anonyma funktioner}


\begin{Slide}{Anonyma funktioner}
\begin{itemize}\SlideFontSmall
\item  Man behöver inte ge funktioner namn. De kan i stället skapas med hjälp av \Emph{funktionslitteraler}.\footnote{Även kallat ''lambda-värde'' eller bara ''lambda'' efter den s.k. lambdakalkylen. \href{https://en.wikipedia.org/wiki/Anonymous_function}{en.wikipedia.org/wiki/Anonymous\_function}}

\item En funktionslitteral har ...
\begin{enumerate}
\item en parameterlista (utan funktionsnamn, utan returtyp),
\item sedan den reserverade teckenkombinationen \code{=>}
\item och sedan ett uttryck (eller ett block).
\end{enumerate}
\pause
\item Exempel:
\begin{Code}[basicstyle=\ttfamily\SlideFontSize{9}{11}]
(x: Int, y: Int) => x + y
\end{Code}
Vilken typ har denna funktionslitteral? \pause \hfill\code{(Int, Int) => Int}
\pause
\item Om kompilatorn kan gissa typerna från sammanhanget så behöver typerna inte anges i själva  funktionslitteralen:
\begin{Code}[basicstyle=\ttfamily\SlideFontSize{9}{11}]
val f: (Int, Int) => Int = (x, y) => x + y
\end{Code}
\end{itemize}
\end{Slide}


\begin{Slide}{Applicera anonyma funktioner på element i samlingar}\SlideFontSmall
Anonym funktion skapad med funktionslitteral direkt i anropet:
\begin{REPL}
scala> val xs = Vector(1, 2, 3)

scala> xs.map((x: Int) => x + 1)
res0: scala.collection.immutable.Vector[Int] = Vector(2, 3, 4)
\end{REPL}
\pause
Eftersom kompilatorn här kan härleda typen \code{Int} så behövs den inte:
\begin{REPL}
scala> xs.map(x => x + 1)
res1: scala.collection.immutable.Vector[Int] = Vector(2, 3, 4)
\end{REPL}
\pause
Om man bara använder parametern en enda gång i funktionen så kan man byta ut parameternamnet mot ett understreck.
\begin{REPL}
scala> xs.map(_ + 1)
res2: scala.collection.immutable.Vector[Int] = Vector(2, 3, 4)
\end{REPL}
\end{Slide}



\begin{Slide}{Platshållarsyntax för anonyma funktioner}\SlideFontSmall
Understreck i funktionslitteraler kallas \Emph{platshållare} \Eng{placeholder} och medger ett förkortat skrivsätt \Alert{om} den parameter som understrecket representerar används \Alert{endast en gång}.
\begin{Code}[basicstyle=\ttfamily\fontsize{10}{12}\selectfont]
_ + 1
\end{Code}
Ovan expanderas av kompilatorn till följande funktionslitteral \\(där namnet på parametern är godtyckligt):
\begin{Code}[basicstyle=\ttfamily\fontsize{10}{12}\selectfont]
x => x + 1
\end{Code}
\pause
Det kan förekomma flera understreck; det första avser första parametern, det andra avser andra parametern etc.
\begin{Code}[basicstyle=\ttfamily\fontsize{10}{12}\selectfont]
_ + _
\end{Code}
\pause
... expanderas till:
\begin{Code}[basicstyle=\ttfamily\fontsize{10}{12}\selectfont]
(x, y) => x + y
\end{Code}
\end{Slide}


\begin{Slide}{Exempel på platshållarsyntax med \texttt{reduceLeft}}\SlideFontSmall
Metoden \code{reduceLeft} applicerar en funktion på de två första elementen i en sekvens och tar sedan resultatet som första argument och nästa element som andra argument och upprepar detta genom hela samlingen.
\begin{REPL}
scala> def summa(x: Int, y: Int) = x + y

scala> val xs = Vector(1, 2, 3, 4, 5)

scala> xs.reduceLeft(summa)
res20: Int = 15

scala> xs.reduceLeft((x, y) => x + y)
res21: Int = 15

scala> xs.reduceLeft(_ + _)
res22: Int = 15

scala> xs.reduceLeft(_ * _)
res23: Int = 120
\end{REPL}
\end{Slide}


\begin{Slide}{Predikat, med och utan namn}
\begin{itemize}\SlideFontTiny
\item En funktion som har \code{Boolean} som returtyp kallas för ett \Emph{predikat}. 
\item Exempel:
\begin{Code}
def isTooLong(name: String): Boolean = name.length > 10

def isTall(heightInMeters: Double, limit: Double = 1.78): Boolean = 
  heightInMeters > limit
\end{Code}
\item Predikat ges ofta ett namn som börjar på \code{is} eller \code{has} så att man lätt kan se att det är ett predikat när man läser kod som anropar funktionen.
\item Många av samlingsmetoderna i Scalas standardbibliotek tar predikat som funktionsargument. Exempel med predikat som anonym funktion: 
\begin{REPLnonum}
scala> val parts = Vector(3, 1, 0, 5).partition(_ > 1)
val parts: (Vector[Int], Vector[Int]) = 
  (Vector(3, 5),Vector(1, 0))
\end{REPLnonum} 
\item Studera snabbreferensen och försök hitta samlingsmetoder som tar predikat som funktionsargument. \url{https://fileadmin.cs.lth.se/pgk/quickref.pdf} \\I anropsexempel med predikat-argument används bokstaven \code{p}.
\end{itemize}  
\end{Slide}

\begin{Slide}{Funktionsvärde vid tom parameterlista: använd ''thunk''}\SlideFontSmall
\begin{itemize}\SlideFontSmall

\item Om du vill ha funktionen som ett värde så skriv bara namnet och inte parameterlistan (samma exempel som tidigare):
\begin{REPLsmall}
scala> def add(a: Int, b: Int) = a + b

scala> val f = add     // inget anrop sker 
val f: (Int, Int) => Int = Lambda7210/0x0000000841e4e040@1ce2db23
\end{REPLsmall}

\item Vid \Alert{tom parameterlista} behövs anonym funktion som \Emph{fördröjer anrop}: 
\begin{REPLsmall}
scala> def a() = 42
def a(): Int

scala> val b = a
1 |val b = a
  |        ^
  |        method a must be called with () argument

scala> val b = () => a()   // anonym funktion, fördröjd evaluering
val b: () => Int = Lambda7214/0x0000000841e50440@565d794
\end{REPLsmall}
\item Notera typen: \code{() => Int} ~~Ett sådant funktionsvärde kallas \Alert{thunk}\\ \url{https://en.wikipedia.org/wiki/Thunk} 
\end{itemize}
\end{Slide}



\Subsection{Skapa din egen kontrollstruktur}  

\begin{Slide}{Hur fungerar egentligen \code{upprepa} i Kojo?}
\begin{Code}[basicstyle=\ttfamily\SlideFontSize{14}{16}]
upprepa(10) {
  println("hej")
}
\end{Code}

\pause
Vi ska nu se hur vi, genom att kombinera ett antal koncept, kan skapa egna kontrollstrukturer likt upprepa ovan:
\begin{itemize}
\item klammerparentes vid ensam paramenter
\item multipla parameterlistor
\item namnanrop (fördröjd evaluering)
\end{itemize}
\end{Slide}



\begin{Slide}{Multipla parameterlistor}
Vi har tidigare sett att man kan ha mer än en parameter:
\begin{REPLnonum}
scala> def add(a: Int, b: Int) = a + b

scala> add(21, 21)
res0: Int = 42
\end{REPLnonum}
Man kan även ha \Alert{mer än en} \Emph{parameterlista}:
\begin{REPLnonum}
scala> def add(a: Int)(b: Int) = a + b

scala> add(21)(21)
res1: Int = 42
\end{REPLnonum}
\Eng{multiple parameter lists}

\href{http://docs.scala-lang.org/style/declarations.html#multiple-parameter-lists}{\SlideFontTiny docs.scala-lang.org/style/declarations.html\#multiple-parameter-lists}
\end{Slide}



\begin{Slide}{Värdeanrop och namnanrop}\SlideFontSmall
Det vi sett hittills är \Emph{värdeanrop}: argumentet evalueras \Alert{först} innan dess \Alert{värde} \emph{sedan} appliceras:
\begin{REPL}
scala> def byValue(n: Int): Unit = for i <- 1 to n do print(" " + n)

scala> byValue(21 + 21)
 42 42 42 42 42 42 42 42 42 42 42 42 42 42 42 42 42 42 42 42 42 42 42 42 42 42 42 42 42 42 42 42 42 42 42 42 42 42 42 42 42 42

scala> byValue({print(" hej"); 21 + 21})
 hej 42 42 42 42 42 42 42 42 42 42 42 42 42 42 42 42 42 42 42 42 42 42 42 42 42 42 42 42 42 42 42 42 42 42 42 42 42 42 42 42 42 42
\end{REPL}
\pause
Men man kan med \code{=>} före parametertypen åstadkomma \Emph{namnanrop}: argumentet \Alert{''klistras in''} i stället för \Alert{namnet} och evalueras \Alert{varje gång} (kallas även \Emph{fördröjd evaluering}):
\begin{REPL}
scala> def byName(n: => Int): Unit = for i <- 1 to n do print(" " + n)

scala> byName({print(" hej"); 21 + 21})
 hej hej 42 hej 42 hej 42 hej 42 hej 42 hej 42 hej 42 hej 42 hej 42 hej 42 hej 42 hej 42 hej 42 hej 42 hej 42 hej 42 hej 42 hej 42 hej 42 hej 42 hej 42 hej 42 hej 42 hej 42 hej 42 hej 42 hej 42 hej 42 hej 42 hej 42 hej 42 hej 42 hej 42 hej 42 hej 42 hej 42 hej 42 hej 42 hej 42 hej 42 hej 42 hej 42
\end{REPL}
\Alert{Kluring}: Varför skrivs ''hej'' ut en extra gång i början? \pause ledtråd: \texttt{1 to \Alert{n}}
%evalueringen av n i 1 to n ger ett extra hej
\end{Slide}

\begin{Slide}{Klammerparenteser vid ensam parameter}
Så här har vi sett nyss att man man göra:
\begin{REPL}
scala> def twice(action: => Unit): Unit = { action; action }

scala> twice( { print("hej"); print("san ") } )
hejsan hejsan
\end{REPL}

Det ser rätt klyddigt ut med \code+({+  och \code+})+ eller vad tycker du? \pause Men...
För alla funktioner \code{f} gäller att: \\ det är helt ok att byta ut vanliga parenteser: \hfill\code{f(uttryck)} \\ mot krullparenteser: \hfill\code|f{uttryck}| \\ \Alert{om} parameterlistan har \Alert{exakt en} parameter.

\vspace{0.5em}Man kan alltså skippa yttre parentesparet för \Alert{bättre läsbarhet}:
\begin{REPLnonum}
scala> twice { print("hej"); print("san ") }
\end{REPLnonum}
\end{Slide}



\begin{Slide}{Skapa din egen kontrollstruktur}\SlideFontSmall
\begin{itemize}
\item Genom att \Alert{kombinera} \Emph{multipla parameterlistor} med \Emph{namnanrop} med \Emph{klammerparentes vid ensam parameter} kan vi skapa vår egen kontrollstruktur: \code{upprepa} \pause
\begin{Code}
upprepa(42){
  if math.random() < 0.5 then print(" gurka")
  else print(" tomat")
}
\end{Code}
Hur då?
\pause
 Till exempel så här:
\begin{Code}
def upprepa(n: Int)(block: => Unit) = for i <- 0 until n do block
\end{Code}

\pause

\begin{REPLnonum}
gurka gurka gurka tomat tomat gurka gurka gurka gurka tomat tomat tomat tomat tomat
\end{REPLnonum}
\end{itemize}
\end{Slide}


\begin{Slide}{Kolon vid ensam parameter}\SlideFontSmall
Du kan i Scala 3 i stället för klammerparentes vid ensam parameter använda kolon för att få färre ''krullisar'' \Eng{fewer braces}.
\begin{Code}
  upprepa(42):
    if math.random() < 0.5 
    then print(" gurka")
    else print(" tomat")
\end{Code}
Denna förenklade syntax föregicks av långa diskussioner innan den till slut accepterades.\footnote{
Den nyfikne kan läsa förslaget före omröstning här: \\ \url{https://docs.scala-lang.org/sips/fewer-braces.html}}
\end{Slide}

\begin{Slide}{Stegade funktioner, ''Curry-funktioner''}
Om en funktion har multipla parameterlistor kan man skapa \Emph{stegade funktioner}, även kallat \Emph{partiellt applicerade} funktioner \Eng{partially applied functions} eller \Emph{''Curry''-funktioner}.
\begin{REPLnonum}
scala> def add(x: Int)(y: Int) = x + y

scala> val öka = add(1)
val öka: Int => Int = Lambda7339/0x0000000841eb7040@19c8add7

scala> Vector(1,2,3).map(öka)
val res0: Vector[Int] = Vector(2, 3, 4)

scala> Vector(1,2,3).map(add(2))
val res1: Vector[Int] = Vector(3, 4, 5)
\end{REPLnonum}
\end{Slide}


\begin{Slide}{Funktion med fångad variabelrymd: \textit{closure}}
\begin{Code}
def f(x: Int): Int => Int = 
  val a = 42 + x
  def g(y: Int): Int = y + a
  g
\end{Code}
Funktionen \code{g} \Alert{fångar} den lokala variabeln \code{a} i ett \Emph{funktionsobjekt}.
\pause
\begin{REPLnonum}
scala> val funkis = f(1)
val funkis: Int => Int = Lambda7356/0x0000000841ed2840@1bda26bc

scala> funkis(2)
val res0: Int = 45
\end{REPLnonum}
\pause
Ett funktionsobjekt med ''fångade'' variabler kallas \Alert{closure}. \\
(Mer om funktioner som objekt senare.)
\end{Slide}

\ifkompendium\else
\begin{SlideExtra}{Översikt av begrepp vi gått igenom hittills}
\begin{enumerate}
\item överlagring
\item utelämna tom parameterlista (enhetlig access)
\item defaultargument
\item namngivna argument
\item lokala funktioner
\item funktioner som äkta värden
\item anonyma funktioner
\item klammerparentes vid ensam paramenter
\item multipla parameterlistor
\item namnanrop (fördröjd evaluering)
\item egendefinierade kontrollstrukturer
\item stegade funktioner (''Curry-funktioner'')
\item fångad variablelrymd i funktionsobjekt (''closure'')
\end{enumerate}
\end{SlideExtra}
\fi



\Subsection{Kort om rekursion}

\begin{Slide}{Rekursiva funktioner}
\begin{itemize}
\item Funktioner som \Alert{anropar sig själv} kallas \Emph{rekursiva}.


\begin{REPLnonum}
scala> def fakultet(n: Int): Int =
         if n < 2 then 1 else n * fakultet(n - 1)

scala> fakultet(5)
val res0: Int = 120
\end{REPLnonum}

\item För varje nytt anrop läggs en ny aktiveringspost på stacken.

\item I aktiveringsposten sparas varje returvärde som gör att \code{5 * (4 * (3 * (2 * 1)))} kan beräknas.

\item Rekrusionen avbryts när man når \Emph{basfallet}, här \code{n < 2}

\item En rekursiv funktion \Alert{måste} ha en returtyp.

\end{itemize}

\end{Slide}

\begin{Slide}{Loopa med rekursion}
\begin{Code}
def gissaTalet(max: Int, min: Int = 1): Unit =
  def gissat = 
    io.StdIn.readLine(s"Gissa talet mellan [$min, $max]: ").toInt

  val hemlis = (math.random() * (max - min) + min).toInt

  def skrivLedtrådOmEjRätt(gissning: Int): Unit =
    if gissning > hemlis then println(s"$gissning är för stort :(")
    else if (gissning < hemlis) println(s"$gissning är för litet :(")

  def ärRätt(gissning: Int): Boolean = 
    skrivLedtrådOmEjRätt(gissning)
    gissning == hemlis

  def loop(n: Int = 1): Int = if ärRätt(gissat) then n else loop(n + 1)

  println(s"Du hittade talet $hemlis på ${loop()} gissningar :)")
\end{Code}
\end{Slide}


\begin{Slide}{Rekursiva datastrukturer}
\begin{itemize}
\item Datastrukturena Lista och Träd är exempel på datastrukturer som passar bra ihop med rekursion.
\item Båda dessa datastrukturer kan beskrivas rekursivt:
\begin{itemize}
\item En lista består av ett huvud och en lista, som i sin tur består av ett huvud och en lista, som i sin tur...
\item Ett träd består av grenar till träd som i sin tur består av grenar till träd som i sin tur, ...
\end{itemize}
\item Dessa datastrukturer bearbetas med fördel med rekursiva algoritmer.
\item I denna kursen ingår rekursion endast ''för kännedom'': \\ du ska veta vad det är och kunna skapa en enkel rekursiv funktion, t.ex. fakultets-beräkning. Du kommer jobba mer med rekursion och rekursiva datastrukturer i fortsättningskursen.
\end{itemize}
\end{Slide}

\Subsection{Automatisk omkompilering}

\begin{Slide}{Kompilera om det som ändrats vid varje sparning}
\begin{itemize}
  \item Den kreativa programmeringsprocessen innehåller många korta cykler av koda, ändra, testa.
  \item Det blir \Alert{många omkompileringar} och då vill man gärna slippa skriva samma kommando om och om igen. %En lösning är att skapa ett skript, t.ex. i språket \Emph{bash}, som kör kompileringen.
  \item Vid \Emph{varje liten ändring} vill man \Alert{kompilera om} det som ändrats och se om det fortfarande kompilerar utan fel. 
  \item Då kan du använda:\\\code{scala compile . --watch}\\Ändringar bevakas och kompileras om direkt.
  %\item  Du kan också använda ett s.k. \Emph{byggverktyg},\\t.ex. Scala Build Tool (\code{sbt}), se Appendix F.

\end{itemize}
\end{Slide}

% \begin{Slide}{Bash-skript för kompilering}\SlideFontSmall
% \begin{itemize}
%   \item Det gamla skriptspråket \Emph{bash} funkar i Linux och MacOS.
%   \item Bash är smidigt för enkla program som använder terminalkommando, men syntaxen är knepig och det finns många fallgropar.
%   \item I ett bash-skript kan du t.ex. kompilera och köra ett program. Exempel i filen \code{build.sh} nedan:
% \begin{Code}
% scalac mitt-program.scala && scala MinMain
% \end{Code}
% Med pil-upp kan du enkelt kompilera om efter varje ändring:
% \begin{REPLnonum}
% > sh build.sh
% \end{REPLnonum}
%   \item Det går att få \Emph{bash} och ubuntu-terminalen att funka i Windows 10 med WSL (Windows Linux Subsystem) där du kan välja Ubuntu 18.04 LTS: \\
%   {\SlideFontTiny\url{https://docs.microsoft.com/en-us/windows/wsl/install-win10}}
% \end{itemize}
% {\noindent   Det finns dock stora begränsningar med WSL och om du vill ha Linux ''på riktigt'' rekommenderas att du installera Ubuntu med dual-boot: \SlideFontTiny\url{https://linoxide.com/distros/install-ubuntu-18-04-dual-boot-windows-10/}}
% \end{Slide}

% \begin{Slide}{Scala Build Tool: \texttt{sbt}}
% \begin{itemize}
%   \item Ett \Emph{byggverktyg}, exempelvis \code{sbt}, kan användas för att kompilera, testköra, ladda ner, paketera och distribuera kodbibliotek och applikationer.
%   \item Du behöver en fil som heter \code{build.sbt} som innehåller:\\\code{scalaVersion := "3.2.2"}
%   \item Det är enkelt att använda \code{sbt} för att kompilera om ditt program varje gång du gör \code{Ctrl+S}:
% \begin{REPLnonum}
% > sbt
% sbt> ~compile   // med ~ görs omkompilering vid ändring
% \end{REPLnonum}
% Tecknet \code{~} kallas \emph{tilde} och skrivs med högra Alt-tangenten nere och två tryck på tangenten bredvid Enter.

%   \item Läs mer om \code{sbt} i Appendix F och här: \\\url{https://www.scala-sbt.org}

% \end{itemize}

% \end{Slide}


\ifkompendium\else
\Subsection{Veckans uppgifter}

\begin{SlideExtra}{Mål med övning \ExeWeekTHREE}
\begin{itemize}\SlideFontSmall
  %!TEX encoding = UTF-8 Unicode
%!TEX root = ../exercises.tex

\item Kunna skapa och använda funktioner med en eller flera parametrar, default-argument, namngivna argument, och uppdelad parameterlista.
\item Kunna använda funktioner som äkta värden.
\item Kunna skapa och använda anonyma funktioner (ä.k. lambda-funktioner).
\item Kunna applicera en funktion på element i en samling.
\item Förstå skillnader och likheter mellan en funktion och en procedur.
\item Förstå vad ett block och en lokal variabel är.
\item Kunna skapa och använda lokala funktioner och förklara nyttan med dessa.
\item Förstå skillnader och likheter mellan värdeanrop och namnanrop.
\item Kunna skapa en enkel kontrollstruktur med fördröjd evaluering av ett block.
\item Förstå skillnaden mellan äkta funktioner och funktioner med sidoeffekter.
%\item Kunna skapa och använda variabler med fördröjd initialisering och förstå när de är användbara.
\item Kunna förklara hur nästlade funktionsanrop sker med   aktiveringsposter.
\item Känna till rekursion och kunna förklara hur rekursiva funktioner fungerar.
\item Känna till att det går att partiellt applicera argument på funktioner med uppdelad parameterlista för att skapa s.k. stegade funktioner (ä.k. curry-funktioner).

%\item Känna till svansrekursion och att svansrekursiva funktioner kan optimeras till loopar.

\end{itemize}
\end{SlideExtra}

\begin{SlideExtra}{Mål med laboration \LabWeekTHREE}
\begin{itemize}
  %!TEX encoding = UTF-8 Unicode
%!TEX root = ../compendium2.tex

%\item Kunna kompilera Scalaprogram med \texttt{scalac}.
%\item Kunna köra Scalaprogram med \texttt{scala}.
%\item Kunna definiera och anropa funktioner.
%\item Kunna använda och förstå default-argument.
%\item Kunna ange argument med parameternamn.
\item Kunna skapa ett större program med din egen kod efter dina egna idéer.
\item Kunna använda en editor och terminalen för att iterativt editera, kompilera, och testa din kod.
\item Kunna använda variabler i kombination med alternativ och repetetition i flera nivåer.
\item Kunna stegvis förbättra din kod för att underlätta förändring och öka läsbarhet.
\item Kunna skapa och använda abstraktioner för att generalisera och möjliggöra återanvändning av kod.

\end{itemize}
Ni ska spela \Emph{varandras} textspel i din \Alert{samarbetsgrupp}.\\
Läs labbinstruktioner:\url{http://fileadmin.cs.lth.se/pgk/compendium.pdf/}
\end{SlideExtra}


\begin{SlideExtra}{Tips till ditt textspel.}
\begin{CodeSmall}
"Yes".toLowerCase.startsWith("y")    // true
"hejsan".contains("ejsa")            // true
"42".toInt                           // 42 
"?".toInt                            // ger krasch (undantaget NumberFormatException)
"?".toIntOption.getOrElse(42)        // 42 (toIntOption kan inte krascha)
// ett annat sätt att förhindra krasch med try ... catch:
val x = try { "?".toInt } catch { case e: Exception => 42 }

val i = 42
s"Livets mening är $i!" // dollar $ före namn vid stränginterpolering med s""
s"Livets mening är inte ${i-1}!"  // klamrar ${} vid evaluering av uttryck

"""|en sträng som spänner över
   |flera rader där marginalen fram till vertikalstreck
   |är bortplockad med stripMargin (kan kombineras med s-interpolatorn)
""".stripMargin

math.random() < 0.8                  // true i 80% av fallen
scala.util.Random.nextInt(42)      // ger slumptal mellan 0 och 41
scala.io.StdIn.readLine("prompt>") // ger sträng som användaren skriver

Thread.sleep(1000)    // sova i 1 sekunder, en lagom irriterande fördröjning
\end{CodeSmall}
Kolla \Emph{snabbreferensen} vad mer du kan göra med strängar!
\end{SlideExtra}

\begin{SlideExtra}{Exempel på en början till ett textspel}
  Här finns en exempel på en enkel \emph{början} på ett textspel som du stegvis kan ändra och bygga ut till något du själv vill göra:
  \url{https://github.com/lunduniversity/introprog/tree/master/workspace/w03_irritext}

\begin{itemize}
  \item Vilka begrepp och principer ger koden träning i?
\end{itemize}

\end{SlideExtra}

\begin{SlideExtra}{Jobba så här}
\begin{itemize}\SlideFontTiny
  \item Skriv koden i editorn vs code som du startar med \code{code .}
  \item Dra igång 3 olika terminalfönster:
  \begin{enumerate}\SlideFontTiny
  \item \code{scala compile . --watch}~\\så att din kod kompileras om automatiskt vid varje sparning med Ctrl+S.
  \item \code{scala repl .}~\\så att du kan göra mindre undersökningar rad för rad, medan du tänker. Ctrl+D avslutar REPL, så du kan börja om efter kodändring.
  \item \code{scala run .}~\\för att se om programmet funkar som det ska. Om programmet väntar på input kan du behöva avbryta för att köra om efter ändring. 
  \end{enumerate}
  \item Börja enkelt och bygg vidare steg för steg.
  \item Bygg om koden allteftersom den växer genom att införa nya abstraktioner med väl valda namn (s.k. ''refaktorisering'') .
  \item Fixa \Alert{alla} kompileringsfel och \Alert{alla} körtidsfel \Emph{innan} du går vidare.
  \item Fokusera på kodens \Alert{läsbarhet}: Om en funktion blir stor så försök dela upp den i flera funktioner med bra namn. 
  \item En åtgärd som förbättrar läsbarheten utan att ändra hur koden fungerar ur användarens synvinkel kallas \Emph{refaktorisering} \Eng{refactoring}. \\\url{https://sv.wikipedia.org/wiki/Omstrukturering_av_kod}\\\url{https://en.wikipedia.org/wiki/Code_refactoring} 
\end{itemize}

\end{SlideExtra}

\fi


%\chapter{Funktioner, Objekt}\label{chapter:W03}
\begin{multicols}{2}\begin{itemize}[nosep,label={$\square$}]
\item definera funktion
\item anropa funktion
\item parameter
\item returtyp
\item värdeandrop
\item namnanrop
\item default-argument
\item namngivna argument
\item applicera funktion på alla element i en samling
\item procedur
\item typen Unit och värdet ()
\item värdeanrop vs namnanrop
\item uppdelad parameterlista
\item skapa egen kontrollstruktur
\item objekt
\item modul
\item punktnotation
\item tillstånd
\item funktionsvärde
\item funktionstyp
\item äkta funktion
\item stegad funktion
\item apply
\item lazy val
\item aktiveringspost
\item rekursion
\item basfall
\item anropsstacken
\item objektheapen\end{itemize}\end{multicols}

%!TEX encoding = UTF-8 Unicode
%!TEX root = ../compendium1.tex

\ifPreSolution

\Exercise{\ExeWeekTHREE}\label{exe:W03}
\begin{Goals}
%!TEX encoding = UTF-8 Unicode
%!TEX root = ../exercises.tex

\item Kunna skapa och använda funktioner med en eller flera parametrar, default-argument, namngivna argument, och uppdelad parameterlista.
\item Kunna använda funktioner som äkta värden.
\item Kunna skapa och använda anonyma funktioner (ä.k. lambda-funktioner).
\item Kunna applicera en funktion på element i en samling.
\item Förstå skillnader och likheter mellan en funktion och en procedur.
\item Förstå vad ett block och en lokal variabel är.
\item Kunna skapa och använda lokala funktioner och förklara nyttan med dessa.
\item Förstå skillnader och likheter mellan värdeanrop och namnanrop.
\item Kunna skapa en enkel kontrollstruktur med fördröjd evaluering av ett block.
\item Förstå skillnaden mellan äkta funktioner och funktioner med sidoeffekter.
%\item Kunna skapa och använda variabler med fördröjd initialisering och förstå när de är användbara.
\item Kunna förklara hur nästlade funktionsanrop sker med   aktiveringsposter.
\item Känna till rekursion och kunna förklara hur rekursiva funktioner fungerar.
\item Känna till att det går att partiellt applicera argument på funktioner med uppdelad parameterlista för att skapa s.k. stegade funktioner (ä.k. curry-funktioner).

%\item Känna till svansrekursion och att svansrekursiva funktioner kan optimeras till loopar.

\end{Goals}

\begin{Preparations}
\item \StudyTheory{03}
\end{Preparations}

\BasicTasks %%%%%%%%%%%%%%%%

\else

\ExerciseSolution{\ExeWeekTHREE}

\fi





\WHAT{Para ihop begrepp med beskrivning.}

\QUESTBEGIN

\Task \what~Koppla varje begrepp med den (förenklade) beskrivning som passar bäst:

\begin{ConceptConnections}
  funktionshuvud & 1 & & A & har parameterlista och eventuellt en returtyp \\ 
  funktionskropp & 2 & & B & beskriver namn och typ på parametrar \\ 
  parameterlista & 3 & & C & argumentet evalueras innan anrop \\ 
  block & 4 & & D & en funktion som anropar sig själv \\ 
  namngivna argument & 5 & & E & gör att argument kan utelämnas \\ 
  defaultargument & 6 & & F & koden som exekveras vid funktionsanrop \\ 
  värdeanrop & 7 & & G & gör att en funktion kan flera resultatvärden \\ 
  namnanrop & 8 & & H & gör att argument kan ges i valfri ordning \\ 
  tupel & 9 & & I & fördröjd evaluering av argument \\ 
  tupelreturtyp & 10 & & J & kan ha lokala namn; sista raden ger värdet \\ 
  äkta funktion & 11 & & K & funktion utan namn; kallas även lambda \\ 
  predikat & 12 & & L & ger alltid samma resultat om samma argument \\ 
  slumptalsfrö & 13 & & M & lista med bestämt antal (heterogena) värden \\ 
  anonym funktion & 14 & & N & ger återupprepningsbar sekvens av pseudoslumptal \\ 
  rekursiv funktion & 15 & & O & en funktion som ger ett booleskt värde \\ 
\end{ConceptConnections}

\SOLUTION

\TaskSolved \what

\begin{ConceptConnections}
  funktionshuvud & 1 & ~~\Large$\leadsto$~~ &  K & har parameterlista och eventuellt returtyp \\ 
  funktionskropp & 2 & ~~\Large$\leadsto$~~ &  M & koden som exekveras vid funktionsanrop \\ 
  parameterlista & 3 & ~~\Large$\leadsto$~~ &  I & beskriver namn och typ på parametrar \\ 
  parameter & 4 & ~~\Large$\leadsto$~~ &  N & namn i funktionshuvud; binds till argument \\ 
  argument & 5 & ~~\Large$\leadsto$~~ &  E & uttryck som är invärde vid funktionsanrop \\ 
  block & 6 & ~~\Large$\leadsto$~~ &  G & kan ha lokala namn; sista raden ger värdet \\ 
  namngivna argument & 7 & ~~\Large$\leadsto$~~ &  H & gör att argument kan ges i valfri ordning \\ 
  default-argument & 8 & ~~\Large$\leadsto$~~ &  L & gör att argument kan utelämnas \\ 
  värdeanrop & 9 & ~~\Large$\leadsto$~~ &  B & argumentet evalueras innan anrop \\ 
  namnanrop & 10 & ~~\Large$\leadsto$~~ &  A & fördröjd evaluering av argument \\ 
  tupel & 11 & ~~\Large$\leadsto$~~ &  J & lista med bestämt antal (heterogena) värden \\ 
  tupelreturtyp & 12 & ~~\Large$\leadsto$~~ &  D & gör att en funktion kan flera resultatvärden \\ 
  anonym funktion & 13 & ~~\Large$\leadsto$~~ &  F & funktion utan namn; kallas även lambda \\ 
  rekursiv funktion & 14 & ~~\Large$\leadsto$~~ &  C & en funktion som anropar sig själv \\ 
\end{ConceptConnections}

\QUESTEND





\WHAT{Definiera och anropa funktioner.}

\QUESTBEGIN

\Task \label{task:funcall} \what~
En funktion med en parameter definieras med följande syntax i Scala:
\vspace{0.5em} \\
\texttt{\code{def} \textit{namn}(\textit{parameter}: \textit{Typ} = \textit{defaultArgument}): \textit{Returtyp} = \textit{returvärde}}

% En funktion med två parametrar definieras med följande syntax i Scala: \vspace{0.5em} \\  \texttt{\code{def} \textit{namn}(\textit{parameter1}: \textit{Typ1}, \textit{parameter2}: \textit{Typ2}): \textit{Returtyp} = \textit{returvärde}}

\Subtask Definiera funktionen \code{öka} som har en heltalsparameter \code{x} och vars returvärde är argumentet plus 1. Defaultargument ska vara 1. Ange returtypen explicit.

\Subtask Vad har uttrycket \code{öka(öka(öka(öka())))} för värde?

\Subtask Definiera funktionen \code{minska} som har en heltalsparameter \code{x} och vars returvärde är argumentet minus 1. Defaultargument ska vara 1. Ange returtypen explicit.

\Subtask Vad är värdet av uttrycket \code{öka(minska(öka(öka(minska(minska())))))}

\Subtask Vad är det för skillnad mellan parameter och argument?

\SOLUTION

\TaskSolved \what

\SubtaskSolved
\begin{Code}
def öka(x: Int = 1): Int = x + 1
\end{Code}

\SubtaskSolved  \code{5}

\SubtaskSolved
\begin{Code}
def minska(x: Int = 1): Int = x - 1
\end{Code}

\SubtaskSolved  \code{1}

\SubtaskSolved
\begin{itemize}
  \item \emph{Kort, förenklad förklaring:} Parametern i funktionshuvudet är ett lokalt namn på indata som kan användas i funktionskroppen, medan argumentet är själva värdet på parametern som skickas med vid anrop.
  \item \emph{Längre, mer exakt förklaring:} En \textbf{parameter} är en deklaration av en oföränderlig variabel i ett funktionshuvud vars namn finns tillgängligt lokalt i funktionskroppen. Vid anrop \emph{binds} parameternamnet till ett specifikt argument. Ett \textbf{argument} är ett uttryck som  appliceras på en funktion vid anrop. Normalt evalueras argumentet innan anropet sker, men om parametertypen föregås av \code{=>} fördröjs evalueringen av argumentet och sker i stället \emph{varje gång} parameternamnet förekommer i funktionskroppen.
\end{itemize}

\QUESTEND



\WHAT{Implementera funktion på olika sätt.}

\QUESTBEGIN

\Task \label{task:funcsumfirst} \what~
Skapa en funktion som kan summera de första \code{n} positiva heltalen.

\Subtask Skriv först funktionshuvudet med \code{???} som funktionskropp. Ge funktionen ett bra namn. Ange returtyp. Kontrollera att din funktion kompilerar utan kompileringsfel innan du går vidare.

\Subtask Implementera funktionen med hjälp av ett intervall och metoden \code{sum}. Testa så att funktionen fungerar. Vad händer om du ger ett negativt argument?

\Subtask Implementera funktionen med hjälp av \code{while}-\code{do}. Vad händer om du ger ett negativt argument?

\SOLUTION

\TaskSolved \what

\SubtaskSolved
\begin{Code}
def sumFirst(n: Int): Int = ???
\end{Code}

\SubtaskSolved
\begin{Code}
def sumFirst(n: Int): Int = (1 to n).sum
\end{Code}
\begin{REPL}
scala> sumFirst(-1)
val res0: Int = 0
\end{REPL}

\SubtaskSolved
\begin{Code}
def sumFirst(n: Int): Int = 
  var result = 0
  var i = 1
  while i <= n do 
    result += i
    i += 1
  end while
  result
end sumFirst
\end{Code}
\begin{REPL}
scala> sumFirst(-1)
val res1: Int = 0
\end{REPL}

\QUESTEND




\WHAT{Textspelet AliensOnEarth.}

\QUESTBEGIN

\Task  \what~Ladda ner spelet nedan \footnote{
\url{https://raw.githubusercontent.com/lunduniversity/introprog/master/compendium/examples/AliensOnEarth.scala}} och studera koden.

\scalainputlisting[basicstyle=\ttfamily\fontsize{10}{12}\selectfont,numbers=left]{examples/AliensOnEarth.scala}

% def randomDistribution(weights: Vector[Int]): Int = {
%   require(weights.size > 0)
%   require(weights.forall(_ >= 0))
%
%   val probabilities = for (w <- weights) yield w / weights.sum.toDouble
%   val rnd = math.random()
%   var i = 0
%   var sum = probabilities(i)
%   while (i < probabilities.size - 1 && rnd > sum) {
%     i += 1
%     sum += probabilities(i)
%   }
%   i
% }

\Subtask Medan du läser koden, försök lista ut vilket som är bästa strategin för att få så mycket poäng som möjligt. Kompilera och kör spelet i terminalen med ditt favoritnamn som argument. Vilket av de tre objekten på planeten jorden har störst sannolikhet att vara bästa alternativet?

\Subtask Para ihop kodsnuttarna nedan med bästa beskrivningen.\footnote{Gör så gott du kan även om allt inte är solklart. Vissa saker kommer vi att gå igenom i detalj först under senare kursmoduler.}

\begin{ConceptConnections}
  \code|options.indices| & 1 & & A & fångar undantag för att förhindra krasch \\ 
  \code|"1X2".toLowercase| & 2 & & B & gör om en sträng till små bokstäver \\ 
  \code|Random.nextInt(n)| & 3 & & C & slumptal i intervallet \code|0 until n| \\ 
  \code|try { } catch { }| & 4 & & D & sträng som kan sträcka sig över flera kodrader \\ 
  \code|""" ... """| & 5 & & E & heltalssekvens med alla index i en sekvens \\ 
  \code|s.stripMargin| & 6 & & F & tar bort marginal till och med vertikalstreck \\ 
  \code|e.printStackTrace| & 7 & & G & skriver ut information om ett undantag \\ 
\end{ConceptConnections}

\noindent\emph{Tips:} Med hjälp av REPL kan du ta reda på hur olika delar fungerar, t.ex.:

\begin{REPL}
scala> val xs = Vector("p", "w", "a")
scala> xs.indices
scala> xs.indices.foreach(i => println(i))
scala> xs.indexOf("w")
scala> xs.indexOf("gurka")
scala> Vector("hej", "hejsan", "hej").indexOf("hej")
scala> try 1 / 0 catch case e: Exception => println(e)
\end{REPL}
%Kolla även dokumentationen för \code{nextInt}, \code{readLine}, m.fl genom att söka här: \\ \url{http://www.scala-lang.org/api/current/index.html}


%\begin{framed}
\noindent\emph{Tips inför fortsättningen:}

\begin{itemize}[nolistsep]
  \item När jag hittade på \code{AliensOnEarth} började jag med ett mycket litet program med en enkel \code{main}-funktion som bara skrev ut något kul. Sedan byggde jag vidare på programmet steg för steg och kompilerade och testade efter varje liten ändring.

  \item När jag kodar har jag REPL igång i ett eget terminalfönster och min kodeditor i ett annat fönster. I ett tredje fönster har jag en terminal med kompilering i \textit{watch mode}, se appendix \ref{appendix:build-scala-cli-watch-mode}. Fråga en handledare om hur du kan arbeta effektivt med stegvisa experimentering i REPL för att bygga upp ett allt större program i små steg.

  \item Detta arbetssätt tar ett tag att komma in i, men är ett bra sätt att uppfinna allt större och bättre program. Ett stort program byggs lättast i små steg och felsökning blir mycket lättare om man bara gör små tillägg åt gången.

  \item Du får också det mycket lättare att förstå ditt program om du delar upp koden i många korta funktioner med bra namn. Du kan sedan lättare hitta på mer avancerade funktioner genom att återanvända befintliga.

  \item Under veckans laboration ska du utveckla ditt eget textspel. Då har du nytta av att återanvända funktionerna för indata och slumpdragning från exempelprogrammet \code{AliensOnEarth}.
\end{itemize}

%\end{framed}


\SOLUTION

\TaskSolved \what~

\SubtaskSolved \code{"penguin"} är bästa alternativ med sannolikheten $\frac{1}{2} + \frac{1}{2}\cdot\frac{1}{3} = \frac{2}{3}$

\SubtaskSolved

\begin{ConceptConnections}
    \code|options.indices| & 1 & ~~\Large$\leadsto$~~ &  F & heltalssekvens med alla index i en sekvens \\ 
  \code|"1X2".toLowercase| & 2 & ~~\Large$\leadsto$~~ &  C & gör om en sträng till små bokstäver \\ 
  \code|Random.nextInt(n)| & 3 & ~~\Large$\leadsto$~~ &  D & slumptal i intervallet \code|0 until n| \\ 
  \code|try { } catch { }| & 4 & ~~\Large$\leadsto$~~ &  B & fångar undantag för att förhindra krasch \\ 
  \code|""" ... """| & 5 & ~~\Large$\leadsto$~~ &  G & sträng som kan sträcka sig över flera kodrader \\ 
  \code|s.stripMargin| & 6 & ~~\Large$\leadsto$~~ &  A & tar bort marginal till och med vertikalstreck \\ 
  \code|e.printStackTrace| & 7 & ~~\Large$\leadsto$~~ &  E & skriver ut information om ett undantag \\ 
\end{ConceptConnections}

\QUESTEND



\WHAT{Äkta funktioner.}

\QUESTBEGIN

\Task  \what~  En äkta funktion%
\footnote{Äkta funktioner uppfyller per definition  \textit{referentiell transparens} \Eng{referential transparency} som du kan läsa mer om här:  \href{https://simple.wikipedia.org/wiki/Referential_transparency}{simple.wikipedia.org/wiki/Referential\_transparency}}
\Eng{pure function} ger alltid samma resultat med samma argument (så som vi är vana vid inom matematiken) och har inga externt observerbara sidoeffekter (till exempel utskrifter).

Vilka funktioner nedan är äkta funktioner?
\begin{Code}
var x = 0
val y = x

def inc(i: Int) = i + 1

def nöff(i: Int) = 
  x = x + i
  "nöff " * x
end nöff

def addX(i: Int) = x + i

def addY(i: Int) = y + i

def isPalindrome(s: String) = s == s.reverse

def rnd(min: Int, max: Int) = math.random() * max + min
\end{Code}


\noindent\emph{Tips:} Skriv av och testa funktionerna i REPL en och en, så att du förstår exakt vad som händer.

\SOLUTION

\TaskSolved \what

\begin{itemize}
  \item Funktionerna  \code{inc}, \code{addY} och \code{isPalindrome} är äkta. Notera att \code{y}-variablen initialiseras till \code{0} och kan sedan inte ändras eftersom den är deklarerad med nyckelordet \code{val}.
\end{itemize}

\QUESTEND


\WHAT{Applicera funktion på varje element i en samling. Funktion som argument.}

\QUESTBEGIN

\Task  \what~

\noindent Deklarera funktionen \code{öka} och variabeln \code{xs} enligt nedan i REPL:
\begin{REPL}
scala> def öka(x: Int) = x + 1
scala> val xs = Vector(3, 4, 5)
\end{REPL}
\noindent Para ihop nedan uttryck till vänster med det uttryck till höger som har samma värde. Om du undrar något, testa uttrycken och olika varianter av dem i REPL.

\begin{ConceptConnections}
  \code|for (i <- 1 to 3) yield öka(i)| & 1 & & A & \code|Vector(5, 6, 7)| \\ 
  \code|Vector(2, 3, 4).map(i => öka(i))| & 2 & & B & \code|Vector(4, 5, 6)| \\ 
  \code|xs.map(öka)| & 3 & & C & \code|Vector(2, 3, 4)| \\ 
  \code|xs.map(öka).map(öka)| & 4 & & D & \code|()| \\ 
  \code|xs.foreach(öka)| & 5 & & E & \code|xs| \\ 
\end{ConceptConnections}

\SOLUTION

\TaskSolved \what

\begin{ConceptConnections}
    \code|for (i <- 1 to 3) yield öka(i)| & 1 & ~~\Large$\leadsto$~~ &  D & \code|Vector(2, 3, 4)| \\ 
  \code|Vector(2, 3, 4).map(i => öka(i))| & 2 & ~~\Large$\leadsto$~~ &  C & \code|xs| \\ 
  \code|xs.map(öka)| & 3 & ~~\Large$\leadsto$~~ &  E & \code|Vector(4, 5, 6)| \\ 
  \code|xs.map(öka).map(öka)| & 4 & ~~\Large$\leadsto$~~ &  A & \code|Vector(5, 6, 7)| \\ 
  \code|xs.foreach(öka)| & 5 & ~~\Large$\leadsto$~~ &  B & \code|()| \\ 
\end{ConceptConnections}

\QUESTEND




\WHAT{Anonyma funktioner.}

\QUESTBEGIN

\Task  \what~  Vi har flera gånger sett syntaxen \code{i => i + 1}, till exempel i en loop \code{(1 to 10).map(i => i + 1)} där funktionen \code{i => i + 1} appliceras på alla heltal från 1 till och med 10 och resultatet blir en ny sekvenssamling.

Syntaxen \code{(i: Int) => i + 1} är en litteral för att skapa ett \emph{funktionsvärde} (kallas även \emph{anonym funktion} eller \emph{lambda-uttryck}). Syntaxen liknar den för funktionsdeklarationer, men nyckelordet \code{def} saknas i funktionshuvudet och i stället för likhetstecken används \code{=>} för att avskilja parameterlistan från funktionskroppen.
Om kompilatorn kan härleda typen ur sammanhanget kan kortformen \code{i => i + 1} användas.

Det finns ett \emph{ännu} kortare sätt att skriva en anonym funktion \emph{om} typen kan härledas \emph{och} den bara använder sin parameter \emph{en enda gång}; då går funktionslitteraler att skriva med s.k. \emph{platshållarsyntax} som använder understreck, till exempel \code{ _ + 1} och som automatiskt expanderas av kompilatorn till \code{ngtnamn => ngtnamn + 1} (namnet på parametern spelar ingen roll; kompilatorn väljer något eget, internt namn).

Para ihop uttryck till vänster med uttryck till höger som har samma värde:

\begin{ConceptConnections}
\input{generated/quiz-w03-lambda-taskrows-generated.tex}
\end{ConceptConnections}

\noindent
Funktionslitteraler kallas \textit{anonyma funktioner}, eftersom de inte har något namn, till skillnad från t.ex. \code{def öka(i: Int): Int = i + 1}, som ju heter \code{öka}. Ett annat vanligt namn är \textit{lambda-uttryck} efter det datalogiska matematikverktyget \href{https://sv.wikipedia.org/wiki/Lambdakalkyl}{lambdakalkyl}.

\SOLUTION

\TaskSolved \what

\begin{ConceptConnections}
    \code|(0 to 2).map(i => i + 1)           | & 1 & ~~\Large$\leadsto$~~ &  B & \code|(2 to 4).map(i => i - 1)| \\ 
  \code|(1 to 3).map(_ + 1)                | & 2 & ~~\Large$\leadsto$~~ &  D & \code|Vector(2, 3, 4)         | \\ 
  \code|(2 to 4).map(math.pow(2, _))       | & 3 & ~~\Large$\leadsto$~~ &  A & \code|Vector(4.0, 8.0, 16.0)  | \\ 
  \code|(3 to 5).map(math.pow(_, 2))       | & 4 & ~~\Large$\leadsto$~~ &  C & \code|Vector(9.0, 16.0, 25.0) | \\ 
  \code|(4 to 6).map(_.toDouble).map(_ / 2)| & 5 & ~~\Large$\leadsto$~~ &  E & \code|Vector(2.0, 2.5, 3.0)   | \\ 
\end{ConceptConnections}

\QUESTEND




\WHAT{Skapa din egen kontrollstruktur med hjälp av namnanrop.}\label{func:upprepa}

\QUESTBEGIN

\Task  \what~Namnanrop skrivs med en raket efter kolon före parametertypen och innebär att argumentet evalueras på plats varje gång.

\Subtask Använd namnanrop i kombination med en uppdelad parameterlista och skapa din egen kontrollstruktur enligt nedan.\footnote{Det är så loopen \code{upprepa} i Kojo är definierad.}
\begin{Code}
def upprepa(n: Int)(block: => Unit): Unit =
  var i = 0
  while i < n do 
    ???
  end while
\end{Code}

\Subtask
Testa din kontrollstruktur i REPL. Låt upprepa 100 gånger att ett slumptal mellan 1 och 6 dras och sedan skrivs ut. Prova även att använda färre klammerparenteser med hjälp av kolon.

\Subtask
Varför behövs namnanrop här?

\SOLUTION

\TaskSolved \what

\SubtaskSolved
\begin{Code}
def upprepa(n: Int)(block: => Unit): Unit =
  var i = 0
  while i < n do
    block
    i += 1
  end while
\end{Code}

\SubtaskSolved
\begin{Code}
upprepa(100):
  val tärningskast = (math.random() * 6 + 1).toInt
  print(s"\$tärningskast ")
\end{Code}

\SubtaskSolved Om parametern \code{block} inte vore deklarerad med namnanrop så hade argumentet evaluerats en gång innan anropet och sedan hade det blivit samma resultat vid varje iteration. Med namnanrop kan block innehålla kod som t.ex. uppdaterar en variabel som vi vill ska ske vid varje iteration. Namn-anrop liknar att koden för argumentet ''klistras in'' på varje plats i funktionskroppen där parameternamnet förekommer. 

\QUESTEND



\WHAT{Lär dig läsa en stack trace.}

\QUESTBEGIN

\Task  \what~  Skriv ett program i filen \texttt{fel.scala} som orsakar ett \emph{körtidsfel} och kör igång det i terminalen med \code{scala-cli run fel.scala}. Studera den stack trace som skrivs ut. Vad innehåller en \code{stack trace}? Diskutera med handledare hur du kan ha nytta av en stack trace när du felsöker.

\SOLUTION

\TaskSolved \what En stack trace innehåller följande information:
\begin{enumerate}
  \item ett felmeddelande
  \item namn på alla funktioner som anropats vid tiden för körtidsfelet, enligt alla aktiveringsposter som ligger på anropsstacken 
  \item aktuell namnrymnd för varje funktionen, alltså paket/singelobjekt
  \item namnet på kodfilen för varje funktion
  \item radnummer i varje funktion 
  \item den funktion som kommer först är den funktion där felet inträffade
  \item eventuellt kan felet inträffa i standardbibliotekets funktioner och då är din egen funktion tidigare i anropskedjan
\end{enumerate}

Exempel på en stack trace:
\begin{REPLnonum}
> cat fel.scala 
@main def run = 
  println("Hej Scala!" + Vector().head)
> scala-cli run fel.scala
Compiling project (Scala 3.3.0, JVM)
Compiled project (Scala 3.3.0, JVM)
Exception in thread "main" java.util.NoSuchElementException: empty.head
	at scala.collection.immutable.Vector.head(Vector.scala:279)
	at fel$package$.run(fel.scala:2)
	at run.main(fel.scala:1)
>
\end{REPLnonum}

\QUESTEND


\ExtraTasks %%%%%%%%%%%%%%%%%%%%%%%%%%%%%%%%%%%%%%%%%%%%%%%%%%%%%%%%%%



\WHAT{Funktion med flera parametrar.}

\QUESTBEGIN

\Task  \what~  

\Subtask Definiera i REPL två funktioner \code{sum} och \code{diff} med två heltalsparametrar som returnerar summan respektive differensen av argumenten:
\begin{Code}
def sum(x: Int, y: Int): Int = ???

def diff(x: Int, y: Int): Int = ???
\end{Code}
Vad har nedan uttryck för värden? Förklara vad som händer.

\Subtask \code{diff(0, 100)}

\Subtask \code{diff(100, sum(42, 43))}

\Subtask \code{sum(sum(42, 43), diff(100, sum(0, 0)))}

\Subtask \code{sum(diff(Byte.MaxValue, Byte.MinValue), 1)}

\SOLUTION

\TaskSolved \what

\SubtaskSolved
\begin{Code}
  def sum(x: Int, y: Int): Int = x + y
  
  def diff(x: Int, y: Int): Int = x - y
\end{Code}
  

\SubtaskSolved  Det blir \code{-100} efter som \code{0 - 100 == -100} 

\SubtaskSolved  Det blir \code{15} eftersom det nästlade anropet motsvarar \\\code{diff(100, 42 + 43) == (100 - 85)}

\SubtaskSolved  Det blir \code{185} eftersom det nästlade anropet motsvarar \\\code{sum(42 + 43, 100 - 0) == (85 + 100)}

\SubtaskSolved  Det blir \code{256} eftersom \code{Byte.MaxValue == 127} och \  code{Byte.MinValue == -128} och \code{sum(127 + 128, 1) == 256}

\QUESTEND



\WHAT{Medelvärde.}

\QUESTBEGIN

\Task  \what~ Skriv och testa en funktion \code{avg} som räknar ut medelvärdet mellan två heltal och returnerar en \code{Double}.

\SOLUTION

\TaskSolved \what

\begin{Code}
def avg(x: Int, y: Int): Double = (x + y) / 2.0
\end{Code}

\QUESTEND




\WHAT{Funktionsanrop med namngivna argument.}

\QUESTBEGIN

\Task  \what~
\begin{REPL}
scala> def skrivNamn(efternamn: String, förnamn: String) =
         println(s"Namn: $efternamn, $förnamn")
scala> skrivNamn(förnamn = "Stina", efternamn = "Triangelsson")
scala> skrivNamn(efternamn = "Oval", "Viktor")

\end{REPL}

\Subtask Vad skrivs ut efter rad 3 resp. rad 4 ovan?

\Subtask Nämn tre fördelar med namngivna argument.

\SOLUTION

\TaskSolved \what~

\SubtaskSolved
\begin{REPL}
Namn: Triangelsson, Stina
Namn: Oval, Viktor
\end{REPL}

\SubtaskSolved
\begin{itemize}
  \item Anroparen kan själv välja ordning.
  \item Koden blir lättare att begripa om parameternamnen är självbeskrivande.
  \item Hjälper till att förhindra buggar som beror på förväxlade parametrar.
\end{itemize}

\QUESTEND



\WHAT{Funktion som äkta värde.}

\QUESTBEGIN

\Task  \what~  Funktioner är \emph{äkta värden} i Scala%\footnote{I likhet med t.ex. Javascript, men till skillnad från t.ex. Java.}
. Det betyder att variabler kan ha funktioner som värden och funktionsvärden kan vara argument till funktioner som har funktionsparametrar. Funktioner som tar funktioner som argument kallas \emph{högre ordningens funktioner}.

En funktion som har en heltalsparameter och ett heltalsresultat är av funktionstypen \code{Int => Int} (uttalas \emph{int-till-int}) och värdet av funktionen utgör ett objekt som har en metod som heter \code{apply} med motsvarande funktionstyp.

\Subtask \label{subtask:funcval} Deklarera nedan funktioner och variabler i REPL. Para sedan ihop nedan uttryck till vänster med det uttryck till höger som skapar samma utskrift. Om du undrar något, testa uttrycken och olika varianter av dem i REPL.

\begin{REPL}
scala> def hälsa(): Unit = println("Hej!")
scala> def fleraAnrop(antal: Int, f: () => Unit): Unit =
         for _ <- 1 to antal do f()
scala> val f1 = () => hälsa()
scala> var f2 = (s: String) => println(s)
scala> val f3 = () => f2("Thunk")
\end{REPL}

\begin{ConceptConnections}
  \code| fleraAnrop(1, hälsa) | & 1 & & A & \code| f2("Hej!\nHej!")| \\ 
  \code| fleraAnrop(3, hälsa) | & 2 & & B & \code| fleraAnrop(3, f1)  | \\ 
  \code| fleraAnrop(2, f1)    | & 3 & & C & \code| f3()               | \\ 
  \code| fleraAnrop(1, f3)    | & 4 & & D & \code| f2("Hej!")       | \\ 
\end{ConceptConnections}


\Subtask Vilka typer har variablerna \code{f1}, \code{f2} och \code{f3}?

\Subtask Funkar detta? Varför? \code{f2 = f1}

\Subtask Funkar detta? Varför? \code{val f4 = fleraAnrop}

\Subtask Funkar detta? Varför? \code{val f4 = hälsa}

\Subtask Funkar detta? Varför? \code{val f4: () => Unit = hälsa}

\SOLUTION

\TaskSolved \what

\SubtaskSolved

\begin{ConceptConnections}
    \code| fleraAnrop(1, hälsa) | & 1 & ~~\Large$\leadsto$~~ &  D & \code| f2("Hej!")       | \\ 
  \code| fleraAnrop(3, hälsa) | & 2 & ~~\Large$\leadsto$~~ &  B & \code| fleraAnrop(3, f1)  | \\ 
  \code| fleraAnrop(2, f1)    | & 3 & ~~\Large$\leadsto$~~ &  A & \code| f2("Hej!\nHej!")| \\ 
  \code| fleraAnrop(1, f3)    | & 4 & ~~\Large$\leadsto$~~ &  C & \code| f3()               | \\ 
\end{ConceptConnections}

\SubtaskSolved \code{f1} och \code{f3} är av typen \code{() => Unit} och \code{f2} av typen \code{String => Unit}.

\SubtaskSolved  Nej. \code{f1} och \code{f2} är av två olika funktionstyper.

\SubtaskSolved  Ja, det går fint.

\SubtaskSolved  Nej. När funktionen inte har någon parameter behöver kompilatorn mer information för att vara säker på att det är ett funktionsvärde du vill ha.

\SubtaskSolved Ja! Nu med typinformationen på plats är kompilatorn säker på vad du vill göra.

\QUESTEND



\WHAT{Bortkastade resultatvärden och returtypen \code{Unit}.}

\QUESTBEGIN

\Task  \what~ Undersök nedan kod i REPL och förklara vad som händer.

\Subtask
\begin{REPL}
scala> def tom = println("")
scala> println(tom)
\end{REPL}

\Subtask
\begin{REPL}
scala> def bortkastad: Unit = 1 + 1
scala> println(bortkastad)
\end{REPL}

\Subtask
\begin{REPL}
scala> def bortkastad2 = { val x = 1 + 1 }
scala> println(bortkastad2)
\end{REPL}

\Subtask Varför är det bra att explicit ange \code{Unit} som returtyp för procedurer?

\SOLUTION

\TaskSolved \what

\SubtaskSolved Procedurer returnerar tomma värdet och \code{println} är en procedur. När tomma värdet skrivs ut visas \code{()}.

\SubtaskSolved Procedurer returnerar tomma värdet. Om du anger returtyp \code{Unit} explicit, har du bättre chans att kompilatorn kan ge varning då uträkningar kommer att kastas bort. En varning avbryter inte exekveringen, utan är ett sätt för kompilatorn att ge dig tips om saker som kan behöva fixas till i din kod.

\SubtaskSolved I Scala är variabeldeklaration, precis som en tilldelningssats, och inte ett uttryck och saknar värde.

\SubtaskSolved  Koden blir lättare att läsa och kompilatorn får bättre möjlighet att hjälpa till med varningar om resultatvärden riskerar att bli bortkastade.

\QUESTEND


\WHAT{Namnanrop.}

\QUESTBEGIN

\Task  \what~

Deklarera denna procedur i REPL:
\begin{Code}
def görDettaTvåGånger(b: => Unit): Unit = { b; b }
\end{Code}

Anropa \code{görDettaTvåGånger} med ett block som parameter. Blocket ska innehålla en utskriftssats. Förklara vad som händer.

\SOLUTION

\TaskSolved \what

Blocket är ett uttryck som har värdet \code{(): Unit}. Evalueringen av blocket sker där namnet \code{b} förekommer i procedurkroppen, vilket är två gånger.
\begin{REPL}
scala> görDettaTvåGånger { println("goddag") }
goddag
goddag
\end{REPL}

\QUESTEND




\clearpage

\AdvancedTasks %%%%%%%%%%%%%%%%%%%%%%%%%%%%%%%%%%%%%%%%%%%%%%%%%%%%%%%%%%%




\WHAT{Föränderlighet av parametrar.}

\QUESTBEGIN

\Task \what~Vad tror du om detta: Är en parameter förändringsbar i funktionskroppen ...

\Subtask ... i Scala?  (Ja/Nej)

\Subtask ... i Java?  (Ja/Nej)

\Subtask ... i Python?  (Ja/Nej)


\SOLUTION

\TaskSolved \what~

\Subtask Nej, i Scala är parametern oföränderlig och det blir kompileringsfel om man försöker tilldela den ett nytt värde i funktionskroppen.

\Subtask \Subtask Ja det går utmärkt i både Java och Python att ändra värdet på parametern i funktionskroppen med tilldelning, men koden riskerar att bli förvirrande.\\
\url{https://stackoverflow.com/questions/2970984}

\QUESTEND



\WHAT{Värdeanrop och namnanrop.}

\QUESTBEGIN

\Task  \what~Normalt sker i Scala (och i Java) s.k. \emph{värdeanrop} vid anrop av funktioner, vilket innebär att argumentuttrycket evalueras \emph{före} bindningen till parameternamnet sker.

Man kan också i Scala (men inte i Java) med syntaxen \code{=>} framför parametertypen deklarera att \emph{namnanrop} ska ske, vilket innebär att evalueringen av argumentuttrycket \emph{fördröjs} och sker \emph{varje gång} namnet används i metodkroppen.

Deklarera nedan funktioner i REPL.

\begin{Code}
def snark: Int = { print("snark "); Thread.sleep(1000); 42 }
def callByValue(x: Int):   Int = x + x
def callByName(x: => Int): Int = x + x
lazy val zzz = snark
\end{Code}

\noindent Förklara vad som händer när nedan uttryck evalueras.

\Subtask \code{snark + snark}

\Subtask \code{callByValue(snark)}

\Subtask \code{callByName(snark)}

\Subtask \code{callByName(zzz)}

\SOLUTION

\TaskSolved \what

\SubtaskSolved Vid varje anrop av \code{snark} sker en utskrift och en fördröjnig innan $42$ returneras. \\\code{42 + 42 == 84} vilket blir värdet av uttrycket.
\begin{REPL}
scala> snark + snark
snark snark val res1: Int = 84
\end{REPL}

\SubtaskSolved Uttrycket \code{snark} evalueras direkt vid anropet och parametern \code{x} binds till värdet $42$ och i funktionskroppen beräknas $42+42$. Utskriften sker bara en gång.
\begin{REPL}
callByValue(snark)
snark val res2: Int = 84
\end{REPL}

\SubtaskSolved Evalueringen av uttrycket \code{snark} fördröjs tills varje förekomst av parametern \code{x} i funktionskroppen. Utskriften sker två gånger.
\begin{REPL}
callByName(snark)
snark snark val res3: Int = 84
\end{REPL}

\SubtaskSolved Evalueringen av uttrycket \code{zzz} fördröjs tills varje förekomst av parametern \code{x} i funktionskroppen. Utskriften sker en gång eftersom \code{val}-variabler tilldelas sitt värde en gång för alla vid den fördröjda initialiseringen.
\begin{REPL}
callByName(zzz)
snark val res4: Int = 84
\end{REPL}

\QUESTEND



\WHAT{Skapa egen kontrollstruktur för iteration med loop-variabel.}

\QUESTBEGIN

\Task  \what~

\Subtask Fördelen med \code{upprepa} i uppgift \ref{func:upprepa} är att den är koncis och lättanvänd. Men den är inte lika lätt att använda om man behöver tillgång till en loopvariabel. Implementera därför nedan kontrollstruktur.

\begin{Code}
def repeat(n: Int)(p: Int => Unit): Unit = 
  var i = 0
  while i < n do
    ??? 
\end{Code}

\Subtask Använd \code{repeat} för att 100 gånger skriva ut loopvariabeln och ett slumpdecimaltal mellan 0 och 1.


\SOLUTION

\TaskSolved \what

\SubtaskSolved
\begin{Code}
def repeat(n: Int)(p: Int => Unit): Unit = 
  var i = 0
  while i < n do
    p(i)
    i += 1
  end while
end repeat
\end{Code}

\SubtaskSolved

\begin{Code}
repeat(100){ i =>
  print("i ")
  println(math.random())
}
\end{Code}
Du kan använda färre klammerparenteser med hjälp av kolon:
\begin{Code}
repeat(100): i =>
  print("i ")
  println(math.random())
\end{Code}

\QUESTEND






\WHAT{Uppdelad parameterlista och stegade funktioner.}

\QUESTBEGIN

\Task \what~Man kan dela upp parametrarna till en funktion i flera parameterlistor. Funktionen \code{add1} nedan har en parameterlista med två parametrar medan \code{add2} har två parameterlistor med en parameter vardera:
\begin{Code}
  def add1(a: Int, b: Int) = a + b
  def add2(a: Int)(b: Int) = a + b
\end{Code}

\Subtask  När man anropar funktionen \code{add2} ska argumenten skrivas inom två olika parentespar. Hur kan du använda \code{add2} för att räkna ut \code{1 + 1}?

\Subtask En fördel med uppdelade parameterlistor är att man kan skapa s.k. \emph{stegade funktioner}\footnote{Kallas även Curry-funktioner efter matematikern och logikern Haskell Brooks Curry.} där argumenten är partiellt applicerade. Prova det stegade funktionsvärdet \code{singLa} nedan. Vad skrivs ut på efter raderna 3 och 5?

\begin{REPL}
scala> def repeat(s: String)(n: Int): String = s * n
scala> val song = repeat("doremi ")(3)
scala> println(song)
scala> val singLa = repeat("la")
scala> println(singLa(7))
\end{REPL}

\SOLUTION

\TaskSolved \what

\SubtaskSolved
\begin{REPL}
scala> def add2(a: Int)(b: Int) = a + b
def add2(a: Int)(b: Int): Int

scala> add2(1)(1)
val res0: Int = 2
\end{REPL}

\SubtaskSolved
\begin{itemize}

\item Rad 3:
\begin{REPLnonum}
doremi doremi doremi 
\end{REPLnonum}

\item Rad 5:
\begin{REPLnonum}
lalalalalalala
\end{REPLnonum}

\end{itemize}


\QUESTEND




\WHAT{Rekursion.}

\QUESTBEGIN

\Task\Uberkurs  \what~  En rekursiv funktion anropar sig själv.

\Subtask Förklara vad som händer nedan.

\begin{REPL}
scala> def countdown(x: Int): Unit = 
         if x > 0 then {println(x); countdown(x - 1)}
scala> countdown(10)
scala> countdown(-1)
scala> def finalCountdown(x: Byte): Unit =
         {println(x); Thread.sleep(100); finalCountdown((x-1).toByte); 1 / x}
scala> finalCountdown(Byte.MaxValue)
\end{REPL}

\Subtask Vad händer om du gör satsen som riskerar division med noll \emph{före} det rekursiva anropet i funktionen \code{finalCountdown} ovan?

\Subtask Förklara vad som händer nedan. Varför tar sista raden längre tid än näst sista raden?
\begin{REPL}
scala> def signum(a: Int): Int = if a >= 0 then 1 else -1
scala> def add(x: Int, y: Int): Int =
         if y == 0 then x else add(x + 1, y - signum(y))
scala> add(100, 100)
scala> add(Int.MaxValue, 0)
scala> add(0, Int.MaxValue)
\end{REPL}

\SOLUTION

\TaskSolved \what

\SubtaskSolved
\code{countdown} skriver ut x och gör ett rekursivt anrop med \code{x - 1} som argument, men bara om basvillkoret \code{x > 0} är uppfyllt. Resultatet blir en ändlig  repetition.
\code{finalCountdown} anropar sig själv rekursivt men saknar ett basvillkor som kan avbryta rekursionen, vilket genererar en oändlig repetition. Vid -128 blir det \emph{overflow} eftersom bitarna inte räcker till för större negativa tal och räkningen börjar om på 127. (Om minskar fördröjningen till \code{Thread.sleep(1)} blir det ganska snabbt \emph{stack overflow})

\SubtaskSolved
Eftersom vi hade \code{1/x} \emph{efter} det rekursiva anropet i föregående deluppgift, så kom vi aldrig till denna (potentiellt ödesdigra) beräkning, utan lade bara aktiveringsposter på hög på stacken vid varje anrop. Om vi placerar \code{1/x} \emph{före} det rekursiva anropet, så når vi detta uttryck direkt och det kastas ett undantag p.g.a. division med noll.

\SubtaskSolved
Den sista raden leder till många fler rekursiva anrop, så som basvillkoret och det rekursiva anropet är konstruerade. Lägg gärna in en \code{println}-sats före det rekursiva anropet och undersök i detalj vad som sker.

\QUESTEND



\WHAT{Undersök svansrekursion genom att kasta undantag.}

\QUESTBEGIN

\Task\Uberkurs  \what~  Förklara vad som händer. Kan du hitta bevis för att kompilatorn kan optimera rekursionen till en vanlig loop?

\begin{REPL}
scala> def explode = throw Exception("BANG!!!")
scala> explode
scala> def countdown(n: Int): Unit =
         if n == 0 then explode else countdown(n-1)
scala> countdown(10)
scala> countdown(10000)
scala> def countdown2(n: Int): Unit =
         if n == 0 then explode else {countdown2(n-1); print("no tailrec")}
scala> countdown2(10)
scala> countdown2(10000)
\end{REPL}

\SOLUTION

\TaskSolved \what~\code{countdown} är svansrekursiv eftersom det rekursiva anropet står \emph{sist} och kan då optimeras till en \code{while}-loop av kompilatorn. Det går fint att köra ända till det exploderar, även med 10000 anrop, och i felmeddelandet finns det endast ett anrop till \code{countdown}.

\code{countdown2} är inte svansrekursiv eftersom den har ett uttryck \code{efter} det rekursiva anropet. I felutskriften syns alla rekursiva anrop till \code{countdown2} innan basvillkoret inträffade. Vid \code{countdown2(10000)} uppfylls inte basvillkoret innan det blir \code{StackOverflowError}.

\QUESTEND



\WHAT{\code{@tailrec}-annotering.}

\QUESTBEGIN

\Task\Uberkurs  \what~  Du kan be kompilatorn att ge felmeddelande om den inte kan optimera koden till en motsvarande while-loop. Detta kan användas i de fall man vill vara helt säker på att kompilatorn kan optimera koden och det inte kan finnas risk för en överfull stack \Eng{stack overflow} på grund av för djup anropsnästling.

Prova nedan rader i REPL och förklara vad som händer.
\begin{REPL}
scala> def countNoTailrec(n: Long): Unit =
         if n <= 0L then println("Klar! " + n) else {countNoTailrec(n-1L); ()}
scala> countNoTailrec(1000L)
scala> countNoTailrec(100000L)
scala> import scala.annotation.tailrec
scala> @tailrec def countNoTailrec(n: Long): Unit =
         if n <= 0L then println("Klar! " + n) else {countNoTailrec(n-1L); ()}
scala> @tailrec def countTailrec(n: Long): Unit =
         if n <= 0L then println("Klar! " + n) else countTailrec(n-1L)
scala> countTailrec(1000L)
scala> countTailrec(100000L)
scala> countTailrec(Int.MaxValue.toLong * 2L)
\end{REPL}

\SOLUTION

\TaskSolved \what~Första gången \code{countNoTailrec(100000L)} anropas blir det \code{StackOverflowError}. Med annoteringen \code{@tailrec} får vi ett kompileringsfel eftersom kompilatorn inte kan optimera en icke svansrekursiv funktion. Om funktionen skrivs om kan kompilatorn optimera funktionen så att rekursionen byts ut mot en \code{while}-loop och vi kan köra så länge vi orkar utan att stacken flödar över. Och himla snabbt går det!!

\QUESTEND

%!TEX encoding = UTF-8 Unicode
%!TEX root = ../compendium2.tex

\Lab{\LabWeekTHREE}
\begin{Goals}
%!TEX encoding = UTF-8 Unicode
%!TEX root = ../compendium2.tex

%\item Kunna kompilera Scalaprogram med \texttt{scalac}.
%\item Kunna köra Scalaprogram med \texttt{scala}.
%\item Kunna definiera och anropa funktioner.
%\item Kunna använda och förstå default-argument.
%\item Kunna ange argument med parameternamn.
\item Kunna skapa ett större program med din egen kod efter dina egna idéer.
\item Kunna använda en editor och terminalen för att iterativt editera, kompilera, och testa din kod.
\item Kunna använda variabler i kombination med alternativ och repetetition i flera nivåer.
\item Kunna stegvis förbättra din kod för att underlätta förändring och öka läsbarhet.
\item Kunna skapa och använda abstraktioner för att generalisera och möjliggöra återanvändning av kod.

\end{Goals}

\begin{Preparations}
\item Gör övning \texttt{\ExeWeekTHREE} och repetera övning \texttt{\ExeWeekTWO} innan du påbörjar laborationen.
\item Läs appendix~\ref{appendix:terminal} och~\ref{appendix:compile}.
\item Hämta given kod via \href{https://github.com/lunduniversity/introprog/tree/master/workspace/}{kursen github-plats}.
\item Utveckla en första, spelbar version av ditt textspel, som du kan jobba vidare på under laborationen.
\item Hitta någon som spelar en tidig version av ditt spel och läser din kod och ger återkoppling på kodens läsbarhet. Skriv ner den återkoppling du får.
\item Spela någon annans textspel och ge återkoppling på kodens läsbarhet.
\end{Preparations}


\subsection{Krav}

\begin{itemize}
\item Du ska skapa ett lagom irriterande textspel med hjälp av en editor, till exempel VS \texttt{code} (se appendix~\ref{appendix:compile:edit}). Spelet ska köras i terminalen.

\item Under redovisningen av laborationen ska du redogöra för vilka programmeringskoncept du tränat på under utvecklingen av ditt textspel. Du ska också för handledaren beskriva hur du har förbättrat din kod genom den återkoppling du fått från någon som spelat ditt spel och läst koden.

\item Ditt textspel ska vara \emph{lagom} irriterande om den som spelar har läst koden, medan spelet gärna får vara orimligt irriterande för den som \emph{inte} läst koden. Det ska gå att klara spelet (du väljer själv vad det innebär) och därmed avsluta programmet inom rimlig tid med kännedom om koden.

\item Försök göra din kod \textit{lätt att läsa och förstå}, även om själva spelet stundtals kan vara mer eller mindre obegripligt, knasigt, eller besvärligt, för den spelare som inte har tillgång till koden... Observera att din kod inte behöver vara ''perfekt'' från början. Börja fritt och förbättra efterhand.

\item Allteftersom ditt program blir längre ska du omforma och dela upp din kod i många, korta abstraktioner med väl valda namn för att öka läsbarheten.

\item Din kod ska använda de viktiga begrepp som kursen hittills har behandlat, med speciellt fokus på det som just du behöver träna mest på.

%\item Slumptal ska ingå i ditt spel och styra valfria delar av exekveringen. Det ska även gå att ge ett valfritt slumptalsfrö som argument vid exekveringen av ditt program. Om fröargument ges ska exekveringen bli återupprepningsbar för en given indatasekvens, annars ska utfallet kunna bli olika vid upprepade körningar med samma indata.
\end{itemize}

\subsection{Tips för att komma igång}

\begin{itemize}
\item Skapa en katalog som innehåller en scala-kodfil med valfritt namn.
\item Skriv en enkel \code{@main}-metod i den nyskapade kodfilen som endast skriver ut strängen \code{"Hello World!"}.
\item Kompilera och kör, rätta eventuella fel tills programmet fungerar korrekt.
\item När programmet fungerar, börja utöka \code{@main}-metoden i din kodfil och implementera mer funktionalitet, ta en titt under inspiration nedan.
\item Börja enkelt och försök formulera vad ditt program ska göra med \emph{psuedokod} som kommentarer innan du skriver koden.
\item Kompilera och kör vid varje tillägg och håll varje tillägg så litet som möjligt, så slipper du reda ut en massa svåra följdfel vid kompilering och eventuella körtidsfel blir mer begripliga.  
\item Fortsätt utöka tills kraven för labben har uppnåtts.
\end{itemize}

\subsection{Inspiration}

Här följer en lista med olika förslag på funktioner som du kan välja bland, kombinera och variera på olika vis. Du kan också låta helt andra funktioner ingå i ditt spel. Det viktigaste är att du kombinerar kodglädje med lärorika utmaningar :)

\begin{itemize}
\item Be användaren logga in. Ge knasiga felmeddelande om användaren inte kan lösenordet.
\item Låt användaren hamna i en irriterande oändlig loop av meningslösa frågor om den gör ''fel''.
\item Beskriv en läskig fantasiplats där användaren befinner sig, till exempel en grotta | en källare | ett rymdskepp | Kemicentrum.
\item Låt användaren välja mellan fåniga vapen, till exempel golvmopp | örontops | foliehatt | förgiftad kexchoklad.
\item Låt användaren välja mellan olika vägar | dörrar | tunnlar | sektionscaféer. Låt valet styra vilka monster som påträffas. Låt användaren bekämpa monstret med olika vapen.
\item Inför någon slags poäng som redovisas under spelets gång och i slutet.
\item Inför olika sorters poäng för hälsa, stridskraft, uppnådd skicklighetsnivå, etc.
\item Fråga användaren om mer eller mindre relevanta detaljer: namn | skonummer | favorithusdjur. Ge knasiga kommentarer där dessa detaljer ingår som delsträngar.
\item Spela sten | sax | påse med användaren.
\item Spela ''gissa talet'' och ge ledtrådar om talet är för litet eller för stort.
\item Mät hur lång tid det tar för användaren att klara ditt spel och ge poäng därefter.
\item Kolla reaktionstiden hos användaren genom att mäta tiden det tar att trycka Enter efter att man fått vänta en slumpmässig tid på att strängen \code{"NU!"} skrivs ut. Om man trycker Enter innan startutskriften ges blir den uppmätta tiden 0 och på så sätt kan ditt program detektera att användaren har tryckt för tidigt. Mät reaktionstiden upprepade gånger och ge poäng efter medelvärdet.
\item Låt användaren på tid så snabbt som möjligt skriva olika ord baklänges.
\item Be användaren skriva en palindrom. Ge poäng efter längd.
\item Träna användaren i multiplikationstabellen på tid.
\item Låt användaren svara på flervalsfrågor om din favoritfilm.
\item Gör det möjligt att ge ett extra argument med en ''fuskkod'' som ger användaren speciella förmågor eller på annat sätt underlättar för användaren under spelets gång.
\end{itemize}

%\subsection{Tips}

%\begin{itemize}
%\item Du kommer åt första argumentet till ditt program genom att indexera i en array som heter \code{args} på plats noll så här: \code{args(0)}.
%\item Du kan kontrollera om det finns minst ett argument med hjälp av det booelska uttrycket \code{args(0).length > 0}.
%\item Metoden \code{toInt} kan göra om en sträng till ett heltal. Du kan vid felaktiga heltal ge ett defaultvärde med \code{scala.util.Try(args(0).toInt).getOrElse(42)}.
%\item Du läser från \textit{standard in} med \code{scala.io.StdIn.readline(prompt)} där \code{prompt} är en sträng som skrivs till \textit{standard out} innan inläsning sker.
%\item Sök upp och studera dokumentationen för klassen \code{scala.util.Random}.
%\item Du kan vänta i t.ex. 3 sekunder med hjälp av Thread.sleep(3000).
%\end{itemize}


%!TEX encoding = UTF-8 Unicode
%!TEX root = ../compendium1.tex

\chapter{Datastrukturer}\label{chapter:W04}
\begin{itemize}[nosep]
\item tupler
\item case-klasser
\item case-object
\item enum i java ???
\item Array
\item Map
\item List
\item Vector
\item föränderlighet
\item iterering
\item vektorer i Java vs Scala
\item Complex
\item Rational
\item läsa/skriva textfiler
\item Source.fromFile
\item java.nio.file
\end{itemize}

\clearpage

%!TEX encoding = UTF-8 Unicode
%!TEX root = ../lect-w04.tex

\ifkompendium\else
\begin{SlideExtra}{Denna vecka: Objekt}
\begin{itemize}\SlideFontSmall
\item Övning \texttt{objects} innehåller bland annat:
\begin{itemize}\SlideFontTiny
\item hur man kan kapsla in funktioner och variabler i singelobjekt
\item hur man kan skapa namnrymder, hantera namnöverskuggning, och använda punktnotation
%\item skapa objekt med hjälp av tupler
%\item lat initialisering
%\item jar-fil, paket, namnbyte vid import
\item använda färdiga klasser, t.ex. \code{java.awt.Color}
\item händelsehantering i ett grafiskt fönster
\end{itemize}

\item På laboration \texttt{blockmole} lär du dig bland annat:
\begin{itemize}\SlideFontTiny
  \item att dela upp din kod i flera singelobjekt
  \item att använda färdig klass: \code{introprog.PixelWindow}
  \item att skapa ett större program i form av ett grafiskt spel
  \item använda tidigare begrepp: uttryck, program och funktioner
\end{itemize}

\item Senaste versionen av kursbiblioteket \code{introprog} är \Emph{\LibVersion} 
\begin{itemize}\SlideFontTiny
\item Jar-fil kan laddas ned här: \url{https://fileadmin.cs.lth.se/introprog.jar}
\item Eller låt \texttt{scala-cli} ladda ner den med
\begin{Code}
//> using scala "3.1.3"
//> using lib "se.lth.cs::introprog:1.3.1"  
\end{Code} 
\item Se \Emph{dokumentation} här: \url{http://cs.lth.se/pgk/api}
\end{itemize}
\end{itemize}
\end{SlideExtra}
\fi


\Subsection{Vad är ett objekt?}

\ifkompendium\else
\begin{SlideExtra}{Objekt är ungefär som äggkartonger}
  \begin{tabular}{l r}
    \includegraphics[width=0.5\textwidth]{../img/egg-box}
    &
    \includegraphics[width=0.5\textwidth]{../img/egg-box-closed}
  \end{tabular}
  Men de kan innehålla mer än bara ägg...
\end{SlideExtra}
\fi

\begin{Slide}{Vad rymmer sköldpaddan i Kojo i sitt tillstånd?}
\centering
\includegraphics[width=0.7\textwidth]{../img/kojo}

\pause position, riktning, färg, bredd, penna uppe/nere, fyll-färg
\end{Slide}



\begin{Slide}{Vad är ett objekt?}
\begin{itemize}
\item Ett objekt är en abstraktion som...
\begin{itemize}
  \item kan innehålla \Emph{data} som objektet ''håller reda på'' och
  \item kan erbjuda \Emph{operationer} som \emph{gör} något eller ger ett \emph{värde}
\end{itemize}

\pause

\item Exempel: Sköldpaddan i Kojo \includegraphics[width=0.08\textwidth]{../img/turtle.png}
\begin{itemize}
  \item Vilken \Emph{data} sparas av sköldpaddan?
  \pause
  \item[] position, rikting, pennfärg, ...

  \item Vilka \Emph{operationer} kan man be sköldpaddan att utföra?

  \item[] fram, höger, vänster, ...
  \pause


\end{itemize}

\item Terminologi:
\begin{itemize}
  \item objektets data sparas i variabler som kallas \Alert{attribut}
  \item alla variabelvärden utgör tillsammans objektets \Alert{tillstånd}
  \item operationerna är funktioner i objektet och kallas \Alert{metoder}
  \item attribut, metoder (och annat i objektet) kallas \Alert{medlemmar}
\end{itemize}
\end{itemize}
\end{Slide}



\begin{Slide}{Deklarera, allokera, referera}
Olika saker man kan göra med objekt:
\begin{itemize}
  \item \Emph{deklarera}: att skriva kod som beskriver objekt; \\
  finns flera sätt: singelobjekt, klass, tupel, ...
  \item \Emph{allokera}: att skapa plats i minnet för objektet vid körtid
  \item \Emph{referera}: att använda objektet via ett namn;\\
  man kommer åt innehållet i ett objekt med \Alert{punktnotation}: \\
  \code{ref.medlem}
  \pause
  \item (\Emph{avallokera}): att frigöra minne för objekt som inte längre används;
  detta \Alert{sker automatiskt} i Scala, men i många andra språk,
  t.ex. C++, får man själv hålla reda på avallokering,
  vilket är knepigt och det blir lätt svåra buggar.
\end{itemize}
\end{Slide}


\begin{Slide}{Olika sätt att allokera objekt}\SlideFontSmall
\begin{enumerate}

\item Använda en \Emph{färdig funktion} som skapar ett objekt åt oss, t.ex. \code{apply}:
\begin{Code}
Vector(1,2,3)  // skapa Vector-objekt med apply-metod
Vector.apply(1,2,3)  // explicit apply
\end{Code}
{\SlideFontTiny En funktion som skapar objekt kallas \Alert{fabriksmetod} \Eng{factory method}.\vspace{0.5em}}

\item Göra \code{new} på en klass (mer om klasser senare):
\begin{Code}
new introprog.PixelWindow() // skapa ett fönsterobjekt
\end{Code}
{
\SlideFontSmall Med \code{new} kan man skapa \Alert{många upplagor} av samma typ av objekt.\\
I Scala 3 kan \code{new} ofta utelämnas: \code{introprog.PixelWindow()}
}

\item Deklarera ett \Emph{singelobjekt} med nyckelordet \code{object}
\begin{itemize}\SlideFontSmall
  \item Ett singelobjekt finns i exakt \Alert{en} upplaga.
  \item Allokeras \Alert{automatiskt} första gången man refererar objektet; \\
  man behöver inte, och kan inte, skriva \code{new}.
  \pause
  \item Medlemmar i ett Scala-singelobjekt liknar \jcode{static}-medlemmar i en Java/C++/C\#-klass.
\end{itemize}
\item Använda en \Emph{tupel}, exempel: \code{val p = (200, 300)}
\end{enumerate}
\end{Slide}

\Subsection{Singelobjekt}

\begin{Slide}{Vad är ett singelobjekt?}
\begin{itemize}\SlideFontSmall
\item Ett singelobjekt \Eng{singelton} deklareras med nyckelordet \code{object} och används för att samla \Emph{medlemmar} \Eng{members} som \Alert{hör ihop}.
\item Ett singelobjekt kallas också \Emph{modul} \Eng{module}.
\item Medlemmarna kan t.ex. vara \Emph{variabler} (\code{val}, \code{var}) och \Emph{metoder} (\code{def}). 
\item En \Alert{metod} är en \Emph{funktion} som finns i ett objekt. Metoder kallas även \Emph{operationer}.
\item Exempel: singelobjekt/modul som hanterar highscore:
\begin{Code}
object Highscore {
  var highscore = 0
  def isHighscore(points: Int): Boolean = points > highscore
}
\end{Code}
\item Krullparenteser är valfria i Scala 3:\\~~~du kan använda kolon och indentering i stället.
\pause
\item Tanken är ofta att abstraktioner ska vara användbar i annan kod, för att underlätta när man bygger applikationer, och kallas då ett \Emph{API} (Application Programming Interface). Exempel: ett highscore-API.
\end{itemize}
\end{Slide}


\begin{Slide}{Allokering: minne reserveras med plats för data}
\begin{Code}
object Highscore:
  var highscore = 0
  def isHighscore(points: Int): Boolean = points > highscore

\end{Code}
\pause
\begin{tikzpicture}[font=\large\sffamily]
\matrix [matrix of nodes, row sep=0, column 2/.style={nodes={rectangle,draw,minimum width=0.8cm}}] (mat)
{
\texttt{Highscore}   &  \makebox(10,10){ }\\
%\texttt{g2}   &  \makebox(16,12){ }\\
};
\node[cloud, cloud puffs=13.0, cloud ignores aspect, minimum width=2cm, minimum height=3.8cm,
 align=center, draw] (x) at (5.8cm, -1.5cm) {
 \begin{tabular}{r l}
 \texttt{highscore} & \fbox{~~~~~0~~} \\

 \end{tabular}
 };
\filldraw[black] (1.2cm,0.0cm) circle (3pt) node[] (ref) {};
 \draw [arrow, line width=0.7mm] (ref) -- (x);
% \node[cloud, cloud puffs=15.7, cloud ignores aspect, %minimum width=5cm, minimum height=2cm,
% align=center, draw] (g2) at (5cm, -2cm) {Gurka-\\objekt};
% \filldraw[black] (0.4cm,-0.4cm) circle (3pt) node[] (g2ref) {};
% \draw [arrow] (g2ref) -- (g2);
\end{tikzpicture}
\end{Slide}


\begin{Slide}{Punktnotation, tillståndsförändring med tilldelning}
\begin{REPLnonum}
scala> Highscore.isHighscore(5)
res0: Boolean = true

scala> Highscore.highscore = 42
\end{REPLnonum}
\pause
\begin{tikzpicture}[font=\large\sffamily]
\matrix [matrix of nodes, row sep=0, column 2/.style={nodes={rectangle,draw,minimum width=0.8cm}}] (mat)
{
\texttt{Highscore}   &  \makebox(10,10){ }\\
%\texttt{g2}   &  \makebox(16,12){ }\\
};
\node[cloud, cloud puffs=13.0, cloud ignores aspect, minimum width=2cm, minimum height=3.8cm,
 align=center, draw] (x) at (5.8cm, -1.5cm) {
 \begin{tabular}{r l}
 \texttt{highscore} & \fbox{~~~~42~~} \\

 \end{tabular}
 };
\filldraw[black] (1.2cm,0.0cm) circle (3pt) node[] (ref) {};
 \draw [arrow, line width=0.7mm] (ref) -- (x);
% \node[cloud, cloud puffs=15.7, cloud ignores aspect, %minimum width=5cm, minimum height=2cm,
% align=center, draw] (g2) at (5cm, -2cm) {Gurka-\\objekt};
% \filldraw[black] (0.4cm,-0.4cm) circle (3pt) node[] (g2ref) {};
% \draw [arrow] (g2ref) -- (g2);
\end{tikzpicture}
\end{Slide}


\begin{Slide}{Punktnotation och operatornotation}
Punktnotation där metodanropet har \Alert{ett} enda argument:\\~\\
\code{objekt.metod(argument)}
\\~\\kan även skrivas med infix \Emph{operatornotation}:\\~\\
\code{objekt metod argument}
\\~\\Exempel:\\
\code{1 + 2}\\\pause\vspace{0.5em}
\code{Highscore isHighscore 1000}
\pause 
{
\SlideFontSmall
\\\vspace{0.5em}Operatornotation med metoder vars namn börjar med bokstäver kommer i framtiden kräva deklaration med \code{infix} före \code{def}, detta för att uppmuntra konsekvent användning.
}
\end{Slide}


\begin{Slide}{Namnrymd och skuggning}
\begin{itemize}\SlideFontSmall
  \item En \Emph{namnrymd}\footnote{\url{https://sv.wikipedia.org/wiki/Namnrymd}}  \Eng{namespace} är en omgivning (kontext) i vilken alla namn är unika. Genom att skapa flera olika namnrymder
  kan man undvika ''\Alert{krockar}'' mellan lika namn med olika betydelser (homonymer). \\
  Exempel: mejladresser \code{kim@företag1.se}  ~$\neq$~  \code{kim@företag2.se}
  \item Medlemmarna i ett singelobjekt finns i en egen namnrymd,
  där alla namn måste vara unika på samma nivå. De ''krockar'' inte med namn ''utanför'' objektet. Dock kan det förekomma \Alert{skuggning} \Eng{shadowing}:
\end{itemize}
\begin{Code}
object Game {

  val highscore = 42   // ett annat värde än Game.Highscore.highscore

  object Highscore:
    var highscore = 0  // ett annat värde än Game.highscore
    def isHighscore(points: Int): Boolean = points > highscore
}
\end{Code}

\end{Slide}



\begin{Slide}{Inkapsling: att dölja interna delar}\SlideFontSmall
Med nyckelordet \code{private} döljs interna delar för omvärlden.
Privata medlemmar kan bara refereras \emph{inifrån} objektet.
Denna princip kallas \Emph{inkapsling} \Eng{encapsulation}.
\begin{CodeSmall}
object Highscore:
  private var myHighscore = 0        // namnet myHighscore syns ej utåt
  def highscore: Int = myHighscore   // en s.k. getter ger ett attributvärde
  def isHighscore(points: Int): Boolean = points > myHighscore
  def update(points: Int): Unit = if isHighscore(points) then myHighscore = points
\end{CodeSmall}
\pause
Varför har man nytta av detta?
\begin{itemize}
  \item Förhindra att man av misstag ändrar objekts tillstånd på fel sätt.
  \item Förhindra användning av kod som i framtiden kan komma att ändras.
  \item Erbjuder en enklare ''utsida'' genom dölja komplexitet ''på insidan''.
  \item Inte ''skräpa ner'' namnrymden med ''onödiga'' namn.
\end{itemize}
Nackdelar:
\begin{itemize}
  \item Begränsar användningen, har ej tillgång till alla delar.
  \item Svårare att experimentera med ett API medan man försöker förstå det.
\end{itemize}
\end{Slide}



\begin{Slide}{Idiom: Privata variabler med understreck vid ''krock''}\SlideFontSmall
\Emph{Idiom}: (d.v.s. ett typiskt, allmänt accepterat sätt att skriva kod)
\begin{itemize}
  \item Om namnet på en privat variabel krockar med namnet på en getter
  brukar man börja det privata namnet med ett understreck:
\end{itemize}

\begin{CodeSmall}
object Highscore:
  private var _highscore = 0
  def highscore: Int = _highscore
  def isHighscore(points: Int): Boolean = points > _highscore
  def update(points: Int): Unit = if isHighscore(points) then _highscore = points
\end{CodeSmall}

\pause

{\SlideFontTiny Namnkrock mellan metoder och variabler uppkommer inte i Java m.fl. språk, där dessa finns i \emph{olika} namnrymder.
Men i Scala har man valt att principen om \Emph{enhetlig access} ska gälla och alla medlemmar (både metoder och variabler) finns därmed i en gemensam namnrymd.}

\end{Slide}

\begin{Slide}{Principen om enhetlig access}\SlideFontSmall
  \begin{itemize}
    \item I Scala så ser access av attribut och anrop av metoder, som är deklarerade utan parameterlista, likadana ut. 
\begin{Code}
object A1 { val a = 42 }  
object A2 { def a = (41 + math.random()).round.toInt }
\end{Code}
\begin{REPLnonum}
scala> A1.a
scala> A2.a  
\end{REPLnonum}
    \item Många andra språk har olika syntax för access av attribut och anrop av metoder (t.ex. Java m.fl., där alla metodanrop måste ha parenteser).
    \item Fördel: Det går lätt att ändra i implementationen och växla mellan att använda attribut och använda metoder utan att den kod som använder din implementation behöver ändras.
    \item Nackdel: Det kan bli namnkrockar mellan metoder och attribut eftersom de finns i samma namnrymd.
  \end{itemize}
  
\end{Slide}



\begin{Slide}{Exempel: singelobjektet med förändringsbart tillstånd} \SlideFontSmall
\begin{Code}[basicstyle=\ttfamily\fontsize{9}{11}\selectfont]
object mittBankkonto:
  val kontonr: Long        = 1234567L
  var saldo: Int           = 1000
  def ärSkuldsatt: Boolean = saldo < 0
\end{Code}
\begin{REPLnonum}
scala> mittBankkonto.saldo -= 25000

scala> mittBankkonto.ärSkuldsatt
res0: Boolean = true
\end{REPLnonum}

(Vi ska i nästa vecka se hur man med s.k. klasser kan skapa många upplagor av samma  typ av objekt, så att vi kan ha flera olika bankkonto.)
\end{Slide}



\begin{Slide}{Exempel: tillstånd, attribut}
Ett objekts \Emph{tillstånd} är den samlade uppsättningen av värden av alla de attribut som finns i objektet.
\begin{Code}[basicstyle=\ttfamily\fontsize{9}{11}\selectfont]
object mittBankkonto
  val kontonr: Long        = 1234567L
  var saldo: Int           = 1000
  def ärSkuldsatt: Boolean = saldo < 0
\end{Code}
\begin{tikzpicture}[font=\large\sffamily]
\matrix [matrix of nodes, row sep=0, column 2/.style={nodes={rectangle,draw,minimum width=0.8cm}}] (mat)
{
\texttt{mittBankkonto}   &  \makebox(10,10){ }\\
%\texttt{g2}   &  \makebox(16,12){ }\\
};
\node[cloud, cloud puffs=13.0, cloud ignores aspect, minimum width=2cm, minimum height=3.8cm,
 align=center, draw] (x) at (5.8cm, -1.5cm) {
 \begin{tabular}{r l}
 \texttt{kontonr} & \fbox{1234567L} \\
 \texttt{saldo} & \fbox{1000}\\
 \end{tabular}
 };
\filldraw[black] (1.7cm,0.0cm) circle (3pt) node[] (ref) {};
 \draw [arrow, line width=0.7mm] (ref) -- (x);
% \node[cloud, cloud puffs=15.7, cloud ignores aspect, %minimum width=5cm, minimum height=2cm,
% align=center, draw] (g2) at (5cm, -2cm) {Gurka-\\objekt};
% \filldraw[black] (0.4cm,-0.4cm) circle (3pt) node[] (g2ref) {};
% \draw [arrow] (g2ref) -- (g2);
\end{tikzpicture}
\end{Slide}


\begin{Slide}{Tillståndsändring}

När en variabel tilldelas ett nytt värde sker en \Emph{tillståndsändring}. Ett \Emph{förändringsbart objekt} \Eng{mutable object} har ett \Emph{förändringsbart tillstånd} \Eng{mutable state}.

\begin{REPLnonum}
scala> mittBankkonto.saldo -= 25000

scala> mittBankkonto.saldo
res1: Int = -24000
\end{REPLnonum}
\begin{tikzpicture}[font=\large\sffamily]
\matrix [matrix of nodes, row sep=0, column 2/.style={nodes={rectangle,draw,minimum width=0.8cm}}] (mat)
{
\texttt{mittBankkonto}   &  \makebox(10,10){ }\\
%\texttt{g2}   &  \makebox(16,12){ }\\
};
\node[cloud, cloud puffs=13.0, cloud ignores aspect, minimum width=2cm, minimum height=3.8cm,
 align=center, draw] (x) at (5.8cm, -1.5cm) {
 \begin{tabular}{r l}
 \texttt{kontonr} & \fbox{1234567L} \\
 \texttt{saldo} & \fbox{-24000}\\
 \end{tabular}
 };
\filldraw[black] (1.7cm,0.0cm) circle (3pt) node[] (ref) {};
 \draw [arrow, line width=0.7mm] (ref) -- (x);
% \node[cloud, cloud puffs=15.7, cloud ignores aspect, %minimum width=5cm, minimum height=2cm,
% align=center, draw] (g2) at (5cm, -2cm) {Gurka-\\objekt};
% \filldraw[black] (0.4cm,-0.4cm) circle (3pt) node[] (g2ref) {};
% \draw [arrow] (g2ref) -- (g2);
\end{tikzpicture}
\end{Slide}

\Subsection{Paket}

\begin{Slide}{Modul}

  \begin{itemize}
    \item En modul samlar kod som utgör en sammanhållen, avgränsad \Emph{uppsättning abstraktioner} som kan användas av annan kod  för att lösa ett specifikt (del)problem.
    \item I Scala finns två sätt att skapa moduler:\footnote{\href{https://en.wikipedia.org/wiki/Modular_programming}{en.wikipedia.org/wiki/Modular\_programming}}
    \begin{itemize}
      \item \Emph{singelobjekt} med nyckelordet \code{object} och
      \item \Emph{paket} med nyckelordet \code{package}
      \pause
      \item Liknar varandra; t.ex. kan man använda punktnotation och göra \code{import} på medlemmar i både singelobjekt och paket.
      \pause
      \item Skillnader:
      \begin{itemize} 
        \item paket medför att \Alert{underkataloger} för maskinkoden skapas vid kompilering
        \item objekt kan ärva medlemmar från klasser och traits (mer om det senare)
        %\item paket får inte i Scala 2 innehålla variabel- och funktions- deklarationer på topp-nivå; dessa  måste i Scala 2 ligga inuti singelobjekt eller klasser
      \end{itemize}
      %\item I Scala 2 kan toppnivådekl. placeras i ett \code{package object}
    \end{itemize}
  
  \end{itemize}
\end{Slide}

\begin{Slide}{Deklarera paket}
Med nyckelordet \code{package} först i en kodfil ges alla deklarationer en gemensam namnrymd.\\
\vspace{1em}
Denna kod ligger i filen \texttt{f1.scala}:
\begin{Code}
package mittpaket

object A: 
  def hälsa: Unit = println(B.hälsning)
\end{Code}

Denna kod ligger i filen \texttt{f2.scala}:
\begin{Code}
package mittpaket

object B:
  def hälsning: String = "hejsan"
\end{Code}
Singelobjekten \code{A} och \code{B} finns båda i namnrymden \code{mittpaket}.
\end{Slide}

\begin{Slide}{Kompilera paket}\SlideFontSmall
Paketdeklarationer medför att kompilatorn placerar bytekodfiler i en katalog med samma namn som paketet:
\begin{REPL}
> scalac f1.scala f2.scala     // samkompilering av två filer
> ls
f1.scala  f2.scala  mittpaket
> ls mittpaket
A.class  'A$.class'   A.tasty   
B.class  'B$.class'   B.tasty 
\end{REPL}
\pause
Idiom, syntax och semantik:
\begin{itemize}
  \item Paketnamn brukar bestå av enbart små bokstäver.
  \item Om paketnamn innehåller punkt(er), skapas nästlade underpaket, exempel:  \code{p1.p2.p3} kompilerar kod till katalogen \code{p1/p2/p3}
  \item Du kan ha flera paket och även nästlade paket i \Alert{samma} kodfil, genom att använda klammerparentes (eller kolon+indentering):\\
  \code|package p1 { object A; package p2 { object B }}|
\end{itemize}
\end{Slide}

\begin{Slide}{Paket i REPL}
Paket funkar inte i REPL:
\begin{REPLnonum}
scala> package mittpaket { def hej = println("Hej") }
-- [E103] Syntax Error: -------------------------------
1 |package mittpaket { def hej = println("Hej") }
  |^^^^^^^
  |this kind of statement is not allowed here

\end{REPLnonum}
\end{Slide}


\Subsection{Tupler}


\begin{Slide}{Vad är en tupel?}\SlideFontSmall

\begin{itemize}
\item En $n$-tupel är ett objekt som samlar $n$ st objekt i en enkel datastruktur med koncis syntax;
du behöver bara parenteser och kommatecken för att skapa tupel-objekt: ~~\code{(1,'a',"hej")}
\item Elementen kan alltså vara av \Alert{olika} typ.

\item
\code{(1,'a',"hej")} är en \Emph{3-tupel} av typen: \code{(Int, Char, String)}

\pause

\item Du kan komma åt de enskilda elementen med \Emph{\code{_1}}, \Emph{\code{_2}}, ...  \Emph{\code{_}$n$}
\item Du kan även använda \Emph{\code{apply(0)}}, \Emph{\code{apply(1)}}, ...  \Emph{\code{apply(n-1)}}

\begin{REPL}
scala> val t = ("hej", 42, math.Pi)
t: (String, Int, Double) = (hej,42,3.141592653589793)

scala> t._1    // direkt access
res0: String = hej

scala> t(1)    // notera användningen av apply
res1: Int = 42
\end{REPL}

\pause

\item Tupler är praktiska när man inte vill ta det lite större arbetet att skapa en egen klass.
(Men med klasser kan man göra mycket mer än med tupler.)

% \item I Scala 2 kan du skapa tupler upp till en storlek av 22 element.
% \\ Behöver du fler element, använd i stället en samling, t.ex. \code{Vector}.
\end{itemize}

\end{Slide}


\begin{Slide}{Tupler som parametrar och returvärde.}\SlideFontSmall

\begin{itemize}

\item Tupler är smidiga som \Emph{parametrar} om man vill kombinera värden som hör ihop, till exempel
 x- och y-värdena i en punkt: \code{(3, 4)}
\pause
\item Tupler är smidiga när man på ett enkelt och typsäkert sätt
vill låta en funktion \Emph{returnera mer än ett värde}.

\begin{REPLsmall}
scala> def längd(p: (Double, Double)): Double = math.hypot(p._1, p._2)

scala> def vinkel(p: (Double, Double)): Double = math.atan2(p._1, p._2)

scala> def polär(p: (Double, Double)): (Double, Double) = (längd(p), vinkel(p))

scala> polär((3,4))
res2: (Double, Double) = (5.0,0.6435011087932844)

\end{REPLsmall}
\vspace{0.5em}
\item Om typerna passar kan man skippa dubbla parenteser vid \Emph{ensamt tupel-argument}:
\begin{REPL}
scala> polär(3,4)
res3: (Double, Double) = (5.0,0.6435011087932844)
\end{REPL}
\item[] {\SlideFontTiny\href{https://sv.wikipedia.org/wiki/Pol\%C3\%A4ra_koordinater}{https://sv.wikipedia.org/wiki/Polära\_koordinater}}


\end{itemize}
\end{Slide}



\begin{Slide}{Ett smidigt sätt att skapa 2-tupler med metoden \texttt{->}}
Det finns en metod vid namn \code{->} som kan användas på objekt av \Alert{godtycklig} typ för att \Emph{skapa par}:

\vspace{0.8em}
\begin{REPL}
scala> ("Ålder", 42)
res0: (String, Int) = (Ålder,42)

scala> "Ålder".->(42)
res1: (String, Int) = (Ålder,42)

scala> "Ålder" -> 42
res2: (String, Int) = (Ålder,42)

scala> Vector("Ålder" -> 42, "Längd" -> 178, "Vikt" -> 65)
res3: scala.collection.immutable.Vector[(String, Int)] =
        Vector((Ålder,42), (Längd,178), (Vikt,65))


\end{REPL}
\end{Slide}


\begin{Slide}{Typalias för att abstrahera typnamn}\SlideFontSmall
Med hjälp av nyckelordet \code{type} kan man deklarera ett \Emph{typalias} för att ge ett \Alert{alternativt} namn till en viss typ. Exempel:
\begin{REPL}
scala> type Pt = (Int, Int)            // typalias
scala> type Pts = Vector[Pt]           // nästlat typalias

scala> def distToOrigo(pt: Pt): Double = math.hypot(pt._1, pt._2)

scala> val xs: Pts = Vector((1,1), (2,2), (3,4))
val xs: Pts = Vector((1,1), (2,2), (3,4))

scala> xs.head
val res0: Pt = (1,1)

scala> xs.map(distToOrigo)                                                                  
val res1: Vector[Double] = Vector(1.4142135623730951, 2.8284271247461903, 5.0)
\end{REPL}

Typalias kan vara bra när:
\begin{itemize}
\item man har en lång och krånglig typ och vill använda ett kortare namn,

\item man vill kunna lätt byta implementation senare\\(t.ex. om man vill använda en egen klass i stället för en tupel).
\end{itemize}
\end{Slide}


\Subsection{Fördröjd initialisering}

\begin{Slide}{Lata variabler och fördröjd initialisering}
Med nyckelordet \code{lazy} före \code{val} sker ''\Emph{lat}'' evaluering av initialiseringsuttrycket. Motsatsen (det normala i Scala) kallas \Alert{strikt} evaluering.
\begin{REPL}
scala> val strikt = Vector.fill(1000000)(math.random())
strikt: scala.collection.immutable.Vector[Double] =
 Vector(0.7583305221813246, 0.9016192590993339, 0.770022134260162, 0.15667718184929746, ...

scala> lazy val lat = Vector.fill(1000000)(math.random())
lat: scala.collection.immutable.Vector[Double] = <lazy>

scala> lat
res0: scala.collection.immutable.Vector[Double] =
  Vector(0.5391685014341797, 0.14759775960530275, 0.722606095900537, 0.9025572787055386, ...
\end{REPL}

En \code {lazy val} initialiseras \Alert{inte} vid deklarationen utan när den \Alert{refereras första gången}. Uttrycket som anges i deklarationen evalueras med s.k. \Emph{fördröjd evaluering} (även ''lat'' evaluering).
\end{Slide}




\begin{Slide}{Singelobjekt är lata}

\begin{itemize}
  \item Singelobjekt allokeras \Alert{inte} direkt vid deklaration; allokeringen sker först då objektet refereras första gången.

\pause

  \item Exempel:

\end{itemize}

\begin{Code}[basicstyle=\ttfamily\fontsize{8}{11}\selectfont]
object mittLataObjekt:
  println("jag är lat")
  val storArray = { println("skapar stor Array"); Array.fill(10000)(42) }
  lazy val ännuStörreArray = Array.fill(Int.MaxValue)(42)
\end{Code}

När sker utskrifterna?

När allokeras variablerna?

\end{Slide}




\begin{Slide}{Vad är skillnaden mellan \texttt{val}, \texttt{var}, \texttt{def} och \texttt{lazy val}?}
\begin{Code}[basicstyle=\ttfamily\fontsize{8}{11}\selectfont]
object exempel:
  println("hej exempel")
  val förAlltidSammaReferens  = {println("hej val"); math.random()}
  var kanÄndrasMedTilldelning = {println("hej var"); math.random()}
  def evaluerasVidVarjeAnrop  = {println("hej def"); math.random()}
  lazy val fördröjdInit = {println("hej lazy val"); math.random()}
\end{Code}
\vspace{1em}\pause
Lat evaluering är en viktig princip inom funktionsprogrammering som möjliggör effektiva, oföränderliga datastrukturer där element allokeras först när de behövs. \\
\href{https://en.wikipedia.org/wiki/Lazy_evaluation}{en.wikipedia.org/wiki/Lazy\_evaluation}
\end{Slide}




\Subsection{Funktioner är objekt}

\begin{Slide}{Programmeringsparadigm}
\href{https://en.wikipedia.org/wiki/Programming_paradigm}{en.wikipedia.org/wiki/Programming\_paradigm}:
\begin{itemize}
\item \Emph{Imperativ programmering}: programmet är uppbyggt av sekvenser av olika satser som läser och \Alert{ändrar} tillstånd
\item \Emph{Objektorienterad programmering}: en sorts imperativ programmering där programmet består av objekt som kapslar in tillstånd och erbjuder operationer som läser och \Alert{ändrar} tillstånd.
\item \Emph{Funktionsprogrammering}: programmet är uppbyggt av samverkande (äkta) funktioner som \Alert{undviker} föränderlig data och tillståndsändringar. Oföränderliga datastrukturer skapar effektiva program i kombination med lat evaluering och rekursion.
\end{itemize}
\end{Slide}


\begin{Slide}{Funktioner är äkta objekt i Scala}
Scala visar hur man kan \Alert{förena} \Eng{unify} \\ \Emph{objektorientering} och \Emph{funktionsprogrammering}: \\\vspace{0.5em}

\textbf{En funktion är ett objekt som har en \code{apply}-metod.}
\pause
\begin{REPLnonum}
scala> object öka:
         def apply(x: Int) = x + 1

scala> öka.apply(1)
res0: Int = 2

scala> öka(1)   // metoden apply behöver ej skrivas explicit
res1: Int = 2
\end{REPLnonum}
\end{Slide}



\begin{Slide}{Fördjupning: Äkta funktionsobjekt är av funktionstyp}
Egentligen, mer precist:\\
\textbf{En funktion är ett objekt \Alert{av funktionstyp} som har en \code{apply}-metod.}
\pause
\begin{REPLnonum}
scala> object öka extends (Int => Int):
         def apply(x: Int) = x + 1
 
scala> öka(1)
res2: Int = 2

scala> Vector(1,2,3).map(öka)
res3: scala.collection.immutable.Vector[Int] = Vector(2, 3, 4)

scala> öka.   // tryck TAB
andThen   apply   compose   toString  ...
\end{REPLnonum}
Mer om \code{extends} senare i kursen... %extends (Int => Int skrivs om till Function1[Int, Int]
\end{Slide}



\Subsection{Använda färdiga klasser}

\begin{Slide}{Vad är en klass?}
Singelobjekt finns bara i exakt EN upplaga:
\begin{Code}
object mittBankkonto:
  val kontonr: Long        = 1234567L
  var saldo: Int           = 1000
  def ärSkuldsatt: Boolean = saldo < 0
\end{Code}
Om vi vill ha flera bankkonton behöver vi en \Alert{klass} \Eng{class}.
\end{Slide}

\begin{Slide}{Vad är en klass?}
En klass kan användas för att skapa många objekt av samma typ. Varje upplaga har sitt eget tillstånd och kallas en \Emph{instans} av klassen (mer om detta nästa vecka).
\begin{Code}
class Bankkonto(val kontonr: Long, var saldo: Int): // klassbeskrivning
  def ärSkuldsatt: Boolean = saldo < 0
\end{Code}
\pause
\begin{REPL}
scala> val bk1 = new Bankkonto(1234567L, 1000 )   // instansiera en klass
bk1: Bankkonto = Bankkonto@5d7399f9

scala> val bk2 = new Bankkonto(6789012L, -200 )
bk2: Bankkonto = Bankkonto@286855ea

scala> bk1.saldo
res0: Int = 1000

scala> bk2.ärSkuldsatt
res1: Boolean = true
\end{REPL}
\end{Slide}

\begin{Slide}{Använda klassen \code{Color}}\SlideFontSmall
\begin{itemize}
\item I JDK (Java Development Kit) finns hundratals paket (moduler) och tusentals färdiga klasser.
\footnote{\SlideFontTiny\url{https://stackoverflow.com/questions/3112882/}}

\item En av dessa klasser heter \code{Color} och ligger i paketet \code{java.awt} och används för att representera RGB-färger med ett tal som beskriver andelen Rött, Grönt och Blått.
\end{itemize}
\begin{REPL}
scala> val röd = java.awt.Color(255, 0, 0)    //  en maximalt röd färg

scala> import java.awt.Color  // namnet Color tillgängligt i aktuell namnrymd

scala> Color.    // tryck TAB och se alla publika medlemmar
\end{REPL}
\pause
\begin{itemize}
\item Använd klassen \code{java.awt.Color} på veckans övning.
\item Hur ska jag veta hur jag kan använda en färdig klass?
\pause
\begin{enumerate}\SlideFontTiny
  \item Läs koden, visar ''insidan'' med all sin komplexitet; kan vara knepigt...
  \item Läs \Emph{dokumentationen}, visar ''utsidan'' som är enklare (?) än ''insidan''
  \item \Alert{Experimentera} med hjälp av REPL och/eller en IDE
\end{enumerate}
\end{itemize}

\end{Slide}


\Subsection{Extensionsmetoder}

\begin{Slide}{Lägg till metoder i efterhand med \texttt{extension}}\SlideFontSmall
\begin{itemize}\SlideFontSmall
\item Ofta vill man kunna lägga till metoder på godtyckliga typer i efterhand, speciellt när det gäller typer som finns i kod som någon annan skrivit.
\item Detta går att göra i Scala med nyckelordet \code{extension}:\\{\SlideFontTiny\code{extension (s: String) def skrikBaklänges = s.reverse.toUpperCase}}
\item En \Emph{extensionsmetod} kan anropas med \Alert{punktnotation} som om den vore en medlem av typen.
\item Det går också att anropa en extensionsmetod som en fristående funktion utan punktnotation.
\end{itemize}  
\begin{REPL}
scala> extension (s: String) def skrikBaklänges = s.reverse.toUpperCase
def skrikBaklänges(s: String): String

scala> "hejsan".skrikBaklänges
val res1: String = NASJEH

scala> skrikBaklänges("goddag")
val res2: String = GADDOG
\end{REPL}
\end{Slide}

\begin{Slide}{Kollektiva extensionsmetoder}
\begin{itemize}
  \item 
Det går bra att sammanföra flera funktioner under en och samma \code{extension} så här:
\begin{Code}
extension (s: String)
  def baklänges = s.reverse
  def skrik = s.toUpperCase
\end{Code}
\item Detta kallas \Emph{kollektiva extensionsmetoder}. 
\item Notera att det \emph{inte} ska vara något kolon efter \code{extension}-deklarationens första rad.
  
\end{itemize}
\end{Slide}

\Subsection{Mer om import}

\begin{Slide}{Import av alla namn i en viss modul}\SlideFontSmall
\begin{itemize}
\item Man kan importera \Alert{alla} namn i en viss modul (singelobjekt eller paket). Detta kallas på engelska för \emph{wildcard import}.

\begin{itemize}\SlideFontTiny
  \item Syntax:  \code|import p1.p2.*|
\end{itemize}

\item Exempel:
\begin{REPL}
scala> import java.awt.*  // importera ALLA namn i paketet awt
\end{REPL}
\item \Emph{Fördelar}:
\begin{enumerate}\SlideFontTiny
  \item Slipper skriva import på varje enskilt namn.
  \item De abstraktioner som är tänkta att användas tillsammans blir alla synliga i aktuell namnrymd \Eng{in scope}.
\end{enumerate}
\item \Alert{Nackdelar}:
\begin{enumerate}\SlideFontTiny
  \item Kan ge namnkrockar och svåra buggar vid namnskuggning.
  \item Man ''skräpar ner'' sin namnrymd med namn som kanske inte är tänkta att användas, men som vid misstag, t.ex. felstavning, ändå ger effekt.
  \item Man kan inte genom att studera import-deklarationerna se exakt vilka namn som används, vilket kan göra det svårare att förstå vad koden gör.
\end{enumerate}
\end{itemize}
\end{Slide}


\begin{Slide}{Namnbyte vid import}
\begin{itemize}
\item Man kan undvika namnkrockar med \Emph{namnbyte vid import}.
\item Syntax:  \code|import p1.p2.befintligtNamn as nyttNamn|
\item Exempel:
\begin{REPL}
scala> import java.awt.Color as JColor //importera och byt namn

scala> val grön = JColor(0, 255, 0)  //skapa instans med nya namnet
grön: java.awt.Color = java.awt.Color[r=0,g=255,b=0]
\end{REPL}
%\item Detta går inte att göra i Java, men t.ex. i C\# och Kotlin.
\end{itemize}
\end{Slide}

\begin{Slide}{Exkludera (gömma) namn vid import}
\begin{itemize}
\item Man kan undvika namnkrockar vid import genom att exkludera vissa namn \Eng{import hiding}.
\item Syntax:  \code|import p1.p2.exkluderaMig as _|
\item Exempel:
\begin{REPL}
scala> import java.awt.{Event as _, *}  // importera allt UTOM Event
\end{REPL}
\item Kan kombineras med namnbyte och allimport:
\begin{REPL}
scala> import java.awt.{Event as _, Color as JColor, *}
\end{REPL}
%\item Detta går inte att göra i Java.
\end{itemize}
\end{Slide}

\begin{Slide}{Lokal import-deklaration}
\begin{itemize}
\item Man kan begränsa ''nedskräpningen'' av namnrymden genom att göra import-deklarationer så lokalt som möjligt, till exempel i ett objekt eller i en funktionskropp.
\item Exempel:
\begin{Code}
object A:
  def x = 
    import java.awt.Color.RED
    /* ... namnet RED syns bara lokalt i denna funktion */
\end{Code}
%\item Detta går inte i Java, där import ska stå i början av filen.
\end{itemize}
\end{Slide}

\begin{Slide}{Export}
\begin{itemize}\SlideFontSmall
  \item \code{import} ger direkt synlighet \Emph{lokalt} inuti en namnrymd
  \item Med \code{export} kan du göra \emph{motsatsen} till import: \\
göra medlemmar direkt synliga \Alert{utanför} en namnrymd.
\begin{CodeSmall}
object A:
  import java.awt.Color.* // gör färger synliga direkt inuti detta objekt
  def test = RED          // färgen RED synlig direkt i lokala namnrymden

object B:
  export java.awt.Color.* // RED blir medlem som syns utåt via B.RED
  export math.{sin, cos}  // sin och cos blir metoder i B
\end{CodeSmall}

\begin{REPLsmall}
scala> A.RED
-- [E008] Not Found Error: ---------------------------------------------
1 |A.RED
  |^^^^^
  |value RED is not a member of object A

scala> B.RED
val res0: java.awt.Color = java.awt.Color[r=255,g=0,b=0]

scala> (B.cos(0), B.sin(0))
val res1: (Double, Double) = (1.0,0.0)
\end{REPLsmall}
\end{itemize}

\end{Slide}

\Subsection{Dokumentation}

\begin{Slide}{Skapa dokumentation}
\code{scaladoc} är ett program som tar Scala-kod som indata och skapar en webbsajt med dokumentation.

\vspace{2em}
\begin{tikzpicture}[node distance=1.5cm,scale=0.8, every node/.style={transform shape}]
\node (input) [startstop] {\texttt{src/*.scala}};

\node(inptext) [right of=input, text width=4cm, scale=1.2,xshift=4.5cm]{en katalog \texttt{src} som innehåller \texttt{.scala}-filer};

\node (scaladoc) [process, below of=input]
{\texttt{scaladoc}};

\node (output) [startstop, below of=scaladoc] {\texttt{index.html} ~~~med mera...};

\node(outtext) [right of=output, text width=4cm, scale=1.2,xshift=4.5cm]{En webbsajt};


\draw [arrow] (input) -- (scaladoc);
\draw [arrow] (scaladoc) -- (output);
\end{tikzpicture}

\vspace*{1em}Läs mer i Appendix E: Dokumentation.
%\vspace{2em} För Java-kod finns motsvarande program som heter \code{javadoc}.
\end{Slide}

\begin{Slide}{Använda dokumentation för färdiga klasser.}
\begin{itemize}
  \item Dokumentation för standardbiblioteket i Scala finns här:  \\ \url{https://www.scala-lang.org/api/}
  \item Övning: Leta upp dokumentationen för metoden \code{reduceLeft} i klassen \code{Vector}.
  \item[]
  \item Dokumentation för standardbiblioteket i Java finns här:  \\ 
  \url{https://docs.oracle.com/en/java/javase/11/docs/api/index.html}
  %\url{https://docs.oracle.com/javase/8/docs/api/}
  \item Övning: Leta upp dokumentationen för \code{java.awt.Color}

\end{itemize}
\end{Slide}


\Subsection{Använda färdiga kodbibliotek}

\begin{Slide}{Vad är en jar-fil?}
\begin{itemize}
  \item Jar-filer används för att distribuera färdigkompilerad kod så att andra kan använda den enkelt
  \item Förkortningen \Emph{jar} kommer från ''Java Archive''
  \item En \Emph{jar}-fil följer ett standardiserat filformat och används för att \Alert{paketera flera filer} i en och samma fil, exempelvis:
  \begin{itemize}
    \item \texttt{.class}-filer med bytekod
    \item resursfiler för en applikation t.ex. bilder \code{.png}, \code{.jpg}, etc
    \item information om vilken klass som innehåller \code{main}-funktionen
    \item etc.
  \end{itemize}
  \item En \code{.jar}-fil komprimeras på samma sätt som en \code{.zip}-fil.
  \item Fördjupning för den intresserade:\\
  {\SlideFontTiny\url{https://en.wikipedia.org/wiki/JAR_(file_format)}}
\end{itemize}
\end{Slide}

\begin{Slide}{Öppen källkod på Maven Central}
\begin{itemize}
  \item På \Emph{Maven Central} som hanteras av företaget Sonatype finns tusentals öppet tillgängliga kodbibliotek publicerade som jarfiler.
  \item Du kan söka bland alla Scala-bibliotek här: \\\url{https://index.scala-lang.org/}
  \item Du kan söka bland alla bibliotek här: \\\url{https://search.maven.org/}
\end{itemize}
\end{Slide}


\begin{Slide}{Vad är \emph{classpath}?}\SlideFontSmall
\begin{itemize}
  \item Hur hittar kompilatorn färdiga moduler?
\pause
\item Kompilatorerna \code{scalac} och \code{javac} och programmen \code{scala} och \code{java} som kör igång JVM använder \Alert{en lista med filsökvägar} kallad \Emph{classpath} när de söker efter kompilerad kod.
\pause
\item Aktuell katalog samt Scalas standardbibliotek läggs automatiskt på classpath.
\item Med hjälp av optionen \code{-cp} kan du ange innehållet på classpath genom att ge en lista med sökvägar separerade med kolon\footnote{I Windows skiljs sökvägar med semikolon istället för kolon.}.
\item Det går bra att lägga till sökväg till jar-filer i listan.
\item Exempel: (punkt används för att ange aktuell katalog)
\begin{REPLnonum}
scala -cp "introprog.jar:." Main
scala-cli run . --jar introprog.jar
\end{REPLnonum}
\end{itemize}
\end{Slide}


\begin{Slide}{Färdiga grafikmetoder i klassen \texttt{PixelWindow}}\SlideFontSmall

\begin{itemize}
\item På labben ska du använda en \texttt{.jar}-fil med kodbiblioteket \code{introprog}.
%\item En \Emph{klass} är en ''mall'' för att göra \Emph{objekt}.
\item Där finns klassen \code{PixelWindow} som kan skapa ritfönster.
%\item När man skapar ett objekt från en klass använder man nyckelordet \code{new}.
\item Du kan starta REPL så här om du har laddat ner jar-filen manuellt från \url{https://fileadmin.cs.lth.se/introprog.jar}  %\pause
\begin{REPLnonum}
> scala -cp introprog.jar 
\end{REPLnonum}
(Men det är enklare att låta Scala CLI ladda ner den åt dig.)
\item Testa \code{PixelWindow} i REPL med:
\begin{REPLnonum}
scala> val w = introprog.PixelWindow(300, 200, "hejsan")
\end{REPLnonum}
\item Studera dokumentationen för \code{introprog.PixelWindow} här: \\\url{http://cs.lth.se/pgk/api/}
\end{itemize}
\end{Slide}


\begin{Slide}{Använda färdiga kodbiblitek med Scala CLI i REPL:}\SlideFontSmall
\begin{itemize}
\item \code{scala-cli} kan inkludera en jar-fil på classpath med optionen \code{--jar}
\begin{REPLsmall}
> curl -sLO https://fileadmin.cs.lth.se/introprog.jar
> scala-cli repl . --jar introprog.jar
Welcome to Scala 3.1.3 (17.0.3, Java OpenJDK 64-Bit Server VM).
Type in expressions for evaluation. Or try :help.
                                                                                                                               
scala> introprog.Dialog.show("hello introprog")
\end{REPLsmall}
\item Du kan istället låta \code{scala-cli} \Emph{automatiskt} ladda ner ett färdigt kodbibliotek som är publicerat på Maven Central och lägga det på classpath med optionen \code{--dep} som är en förkortning av \emph{dependency}. Notera antalet kolon i adressen till kodbiblioteket:
\begin{REPLsmall}
> scala-cli repl . --dep se.lth.cs::introprog:1.3.1
Welcome to Scala 3.1.3 (17.0.3, Java OpenJDK 64-Bit Server VM).
Type in expressions for evaluation. Or try :help.
                                                                                                                               
scala> introprog.Dialog.show("hello introprog")
\end{REPLsmall}
\end{itemize}

\end{Slide}


\begin{Slide}{Köra program + kodbiblitek med Scala CLI}\SlideFontTiny
\begin{itemize}
\item \code{scala-cli} kan inkludera kodbibliotek från Maven Central om du skriver en ''magisk'' kommentar i början av din \code{.scala-}filen:
\begin{Code}
//> using scala "3.1.3"
//> using lib "se.lth.cs::introprog:1.3.1"

@main def run = introprog.Dialog.show("hello introprog")
\end{Code}
Notera \texttt{>} efter \texttt{//}

\item När du kör ditt program såhär så kommer Scala CLI att ladda ner kodbiblioteket om det inte redan är gjort:
\begin{REPLsmall}
> scala-cli run .
\end{REPLsmall}
\item Läs mer här:\\\url{https://index.scala-lang.org/lunduniversity/introprog-scalalib} och i Appendix C, stycket om Scala CLI. Mer om \code{//> using} här:
\item[] \url{https://scala-cli.virtuslab.org/docs/reference/directives}
\end{itemize}

\end{Slide}

\begin{Slide}{Scala Build Tool: \texttt{sbt}}\SlideFontSmall
\begin{itemize}
\item Du kan låta byggverktyget \code{sbt} automatiskt sköta omkompilering och körning vid varje Ctrl+S med kommandot \code{~run}
och enbart omkompilering med \code{~compile}
\begin{REPL}
> sbt
[info] Set current project to hellosbt (in build file:/home/bjornr/tmp/hellosbt/)
[info] sbt server started at 127.0.0.1:5320
sbt:hellosbt> ~run
\end{REPL}
Avsluta med Enter. Om flera \code{main} gör: \code{ ~runMain DettaMainObj}
\item Kod antas finnas direkt i \Emph{aktuell katalog} eller i \code{src/main/scala}
\item Lägg \code{.jar}-filer i en katalog \code{lib} så hamnar de automatiskt på classpath
\item Skapa en fil med namnet \code{build.sbt} för att säkerställa rätt version:
\begin{Code}
scalaVersion := "3.1.3"
\end{Code}
I filen \code{project/build.properties} kan du ange versionen på \code{sbt}:
\begin{Code}
sbt.version=1.7.1
\end{Code}
Se Appendix F i kompendiet och \url{http://www.scala-sbt.org/}
\end{itemize}
\end{Slide}



\begin{Slide}{Använda \texttt{introprog} tillsammans med \texttt{sbt}}
Lägg raden med \code{libraryDependencies} i filen \code{build.sbt} efter \code{scalaVersion}:
\begin{Code}
scalaVersion := "3.1.3"
libraryDependencies += "se.lth.cs" %% "introprog" % "1.3.1"
\end{Code}
Första gången du kör \code{compile}, \code{run} eller \code{console} i \code{sbt} kommer en jar-fil med paketet \code{introprog} automatiskt att laddas ner från Maven Central.
\end{Slide}


\ifkompendium\else


\Subsection{Veckans övning och laboration}

\begin{SlideExtra}{Övning \texttt{objects}}\SlideFontTiny
\setlength{\leftmargini}{0pt}
\begin{itemize}
%!TEX encoding = UTF-8 Unicode
%!TEX root = ../compendium2.tex

\item Kunna skapa och använda objekt som moduler.
\item Känna till att funktioner är objekt med en \code{apply}-metod.
\item Förstå begreppen synlighet, \code{private}, \code{import}, namnrymd och skuggning.
\item \TODO{FLER MÅL OM OBJEKT HÄR}

%\item Känna till svansrekursion och att svansrekursiva funktioner kan optimeras till loopar.

\end{itemize}
\end{SlideExtra}

\begin{SlideExtra}{Lab \texttt{blockmole}}\SlideFontTiny
%\setlength{\leftmargini}{0pt}
\begin{itemize}
%!TEX encoding = UTF-8 Unicode
%!TEX root = ../labs.tex

\item Kunna förklara hur singelobjekt kan användas som moduler.
\item Kunna förklara hur åtkomst av medlemmar i singelobjekt sker.
\item Kunna skapa kod som reagerar på och förändrar objekts tillstånd.
\item Kunna förklara nyttan med att samla namngivna konstanter i egen modul.
\item Kunna förklara hur import påverkar synlighet av namn.
\item Kunna ge exempel på en situation där man har nytta av namnbyte vid import.
\item Kunna redogöra för skillnaden mellan paket och singelobjekt.
\item Kunna skapa och använda tupler.

\end{itemize}

\end{SlideExtra}
\fi


%\chapter{Datastrukturer}\label{chapter:W04}
\begin{itemize}[nosep]
\item tupler
\item case-klasser
\item case-object
\item enum i java ???
\item Array
\item Map
\item List
\item Vector
\item föränderlighet
\item iterering
\item vektorer i Java vs Scala
\item Complex
\item Rational
\item läsa/skriva textfiler
\item Source.fromFile
\item java.nio.file
\end{itemize}
%!TEX encoding = UTF-8 Unicode
%!TEX root = ../compendium2.tex

\Exercise{\ExeWeekFOUR}\label{exe:W04}
\begin{Goals}
%!TEX encoding = UTF-8 Unicode
%!TEX root = ../compendium2.tex

\item Kunna skapa och använda objekt som moduler.
\item Känna till att funktioner är objekt med en \code{apply}-metod.
\item Förstå begreppen synlighet, \code{private}, \code{import}, namnrymd och skuggning.
\item \TODO{FLER MÅL OM OBJEKT HÄR}

%\item Känna till svansrekursion och att svansrekursiva funktioner kan optimeras till loopar.

\end{Goals}

\begin{Preparations}
\item \StudyTheory{04}
\end{Preparations}

\BasicTasks %%%%%%%%%%%%%%%%

\TODO{ÖVNINGAR OM OBJEKT}

%!TEX encoding = UTF-8 Unicode
%!TEX root = ../compendium2.tex

\Lab{\LabWeekFOUR}
\begin{Goals}
%!TEX encoding = UTF-8 Unicode
%!TEX root = ../labs.tex

\item Kunna förklara hur singelobjekt kan användas som moduler.
\item Kunna förklara hur åtkomst av medlemmar i singelobjekt sker.
\item Kunna skapa kod som reagerar på och förändrar objekts tillstånd.
\item Kunna förklara nyttan med att samla namngivna konstanter i egen modul.
\item Kunna förklara hur import påverkar synlighet av namn.
\item Kunna ge exempel på en situation där man har nytta av namnbyte vid import.
\item Kunna redogöra för skillnaden mellan paket och singelobjekt.
\item Kunna skapa och använda tupler.

\end{Goals}

\begin{Preparations}
\item \DoExercise{\ExeWeekTHREE}{03}
\item \DoExercise{\ExeWeekFOUR}{04}
\end{Preparations}



\subsection{Obligatoriska uppgifter}


\begin{quote}
\textbf{Blockmullvad} (\textit{Talpa laterculus}) är ett fantasidjur i familjen mullvadsdjur.
Den är känd för sitt karaktäristiska kvadratiska utseende.
Den lever mest ensam i sina underjordiska gångar som till skillnad från mullvadens (\emph{Talpa europaea}) har helt raka väggar.
\end{quote}

\begin{figure}
\end{figure}

\Task
Du ska skriva ett Scala-program med en vanlig texteditor och kompilera ditt program med kommandot \texttt{scalac} och sedan köra programmet med kommandot \texttt{scala}.

\Subtask
Öppna en texteditor, till exempel gedit eller Atom (se appendix~\ref{appendix:edit} för hjälp).
Skapa en ny fil med namnet \texttt{Mole.scala} och spara den i en ny katalog i din hemkatalog, till exempel \texttt{\textasciitilde/pgk/mole/Mole.scala}, där \texttt{\textasciitilde} är din hemkatalog.

\Subtask
Öppna ett terminalfönster (se appendix~\ref{appendix:terminal} för hjälp).
Navigera till din nya katalog med \texttt{cd}-kommandot \Eng{change directory} och kontrollera med \texttt{ls}-kommandot \Eng{list} att din nya fil finns där.
\begin{REPLnonum}
> cd ~/pgk/mole
> ls
\end{REPLnonum}
Om allt går bra ska \texttt{ls}-kommandot skriva ut \texttt{Mole.scala}.

\Subtask
Gå tillbaka till din texteditor och skriv in ett objekt med namnet \code{Mole} i din fil.
Lägg till en \code{main}-funktion i objektet som skriver ut texten \emph{Keep on digging!} med hjälp av funktionen \code{println}.
Behöver du hjälp kan du gå tillbaka till övningarna i kapitel~\ref{exe:W03}.

\Subtask
Kör kommandot \texttt{scalac Mole.scala} i terminalfönstret för att kompilera ditt program.
Om kompilatorn rapporterar några fel rättar du till det i din texteditor kompilerar igen.
Kontrollera sedan med \texttt{ls}-kommandot att några filer som slutar på \texttt{class} har skapats.

\Subtask
Kör kommandot \texttt{scala Mole} för att köra ditt program.
Om att går bra ska texten du angivit skrivas ut i terminalfönstret.


\Task
Nu har du skrivit ett Scala-program som skriver ut en uppmaning till en mullvad att fortsätta gräva.
Det programmet är inte så användbart, eftersom mullvadar inte kan inte läsa.
Nästa steg är att skriva ett grafiskt program, snarare än ett textbaserat.

Funktionen \code{println} som anropas i \code{main}-funktionen ingår i Scalas standardbibliotek.
Ett programbibliotek innehåller kod eller kompilerade programsnuttar som kan användas av andra program, och för de flesta programspråk ingår ett standardbibliotek som alla program kan nyttja.
Till grafiken i denna uppgift ska du använda ett bibliotek som kallas \emph{cslib} och som kommer att användas även i senare labbar.

\Subtask

Ladda ner \texttt{cslib.jar} via länken \url{http://cs.lth.se/pgk/cslib} och lägg jar-filen i samma katalog som ditt Scala-program.
En jar-fil används för att paketera färdigkompilerade program, kod, dokumentation, resursfiler, etc, och är komprimerad på samma sätt som en zip-fil.

\Subtask
Byt ut \code{main}-funktionens kropp mot följande block:
\begin{Code}
{
	val w = new cslib.window.SimpleWindow(300, 500, "Digging")
	w.moveTo(10, 10)
	w.lineTo(10, 20)
	w.lineTo(20, 20)
	w.lineTo(20, 10)
	w.lineTo(10, 10)
}
\end{Code}
Den första raden skapar ett nytt \code{SimpleWindow} som ritar upp ett fönster som är 300 bildpunkter brett och 500 bildpunkter högt med titeln \emph{Digging}.
\code{SimpleWindow} har en \emph{penna} som kan flyttas runt och rita linjer.
Anropet \code{w.moveTo(10, 10)} flyttar pennan för fönstret \code{w} till position $(10,10)$ utan att rita något, och anropet \code{w.lineTo(10, 20)} ritar en linje därifrån till position $(10, 20)$.

\Subtask
Nu ska du kompilera ditt program, men eftersom \code{SimpleWindow} inte finns i Scalas standardbibliotek utan i \texttt{cslib.jar} behöver du visa kompilatorn var den ska leta.
Det gör du genom att ange en \emph{classpath}, dvs. en sökväg till \texttt{class}-filer, när du kompilerar.
Använd flaggan \texttt{-cp cslib.jar} för att ange \texttt{cslib.jar} som classpath och kompilera ditt Scala-program igen:
\begin{REPLnonum}
> scalac -cp cslib.jar Mole.scala
\end{REPLnonum}

\Subtask
Nu ska du köra ditt program, och då behöver du också ange var \texttt{class}-filerna ligger.
Du ska ange den katalog där \texttt{class}-filerna för \code{Mole} ligger, som du just kompilerat, men du ska också ange \texttt{cslib.jar}, och det gör du med en kolon-separerad lista\footnote{Kolon används i Linux och macOS, medan Windows använder semikolon.}, till exempel \code{"sökväg1:sökväg2:sökväg3"}.
Katalogen du står i, där dina \texttt{class}-filer ligger, kan anges med en punkt (\texttt{.}).
Kör programmet med följande kommando (om Windows använd semikolon):
\begin{REPLnonum}
> scala -cp ".:cslib.jar" Mole
\end{REPLnonum}
Du ska nu få upp ett fönster med en liten kvadrat utritad i övre vänstra hörnet.


\Task
Hela ditt program är för tillfället samlat i en och samma funktion, vilket fungerar bra för väldigt små program.
Nu ska vi strukturera programmet så det blir lättare att återanvända samma kodsnuttar.

\Subtask
Lägg till ett objekt med namnet \code{Graphics} i \texttt{Mole.scala} och flytta dit deklarationen av fönstret \code{w}.
Skapa en ny funktion med namnet \code{square} i det nya objektet och flytta dit koden som ritar kvadraten.
Anropa \code{square} i din \code{main}-funktion.
Filen \texttt{Mole.scala} ska se ut såhär (förutom \code{???}):
\begin{Code}
object Graphics {
	val w = new cslib.window.SimpleWindow(300, 500, "Digging")
	def square(): Unit = ???
}
object Mole {
	def main(args: Array[String]): Unit = {
		Graphics.square()
	}
}
\end{Code}
Observera att du inte kan anropa \code{square} direkt i funktionen \code{main}, utan måste ange att det är \code{square}-funktionen inuti \code{Graphics} du vill anropa.

\Subtask
Kompilera \texttt{Mole.scala} med \texttt{scalac}.
Glöm inte att ange korrekt classpath.
(\emph{Tips:} Du kan trycka uppåtpil för att komma till tidigare kommandon i terminalen.)
Kontrollera med \texttt{ls} att det nu också finns \texttt{class}-filer för \code{Graphics}-objektet.

\Subtask
Kör programmet \code{Mole} med \texttt{scala}.
Glöm inte att ange korrekt classpath.
Om allt fungerar ska programmet göra samma sak som innan.

\Task
Nu har du gjort ett grafiskt program, men ännu syns ingen mullvad.
Det är dags att ta reda på hur koordinatsystemet fungerar i denna grafiska miljö, så vi kan få mullvaden att hitta rätt.

\Subtask
Ändra i \code{Graphics.square} så att kvadraten ritas upp i \emph{övre högra} hörnet istället.
Prova dig fram för att ta reda på hur koordinatsystemet fungerar genom att ändra i koden, kompilera och köra programmet tills du får rätt på det.

\Subtask\Checkpoint
Visa kvadraten för din labbhandledare och förklara vad de två parametrarna gör genom att peka ut ungefär var positionerna $(0,0)$, $(300, 0)$, $(0, 300)$ och $(300, 300)$ ligger.

\Subtask
Ta bort anropet till funktionen \code{square} när du har visat den för din labbhandledare.

\Task
Nu ska du skapa ett nytt koordinatsystem för \code{Graphics} som har \emph{stora} bildpunkter.
Vi kallar \code{Graphics} stora bildpunkter för \emph{block} för att lättare skilja dem från \code{SimpleWindow}s bildpunkter.
Om blockstorleken är $b$, så ligger koordinaten $(x, y)$ i \code{Graphics} på koordinaten $(bx, by)$ i \code{SimpleWindow}.

\Subtask
Lägg till följande deklarationer överst i objektet \code{Graphics}.
\begin{Code}
val width = 30
val height = 50
val blockSize = 10
\end{Code}
Ändra bredden på ditt \code{SimpleWindow} till \code{width * blockSize} och ändra höjden till \code{height * blockSize}.

\Subtask
Skapa en ny funktion i \code{Graphics} med namnet \code{block} och två parametrar \code{x} och \code{y} av typen \code{Int} och returtypen \code{Unit}.
Metodens \emph{kropp} ska se ut såhär:
\begin{Code}
{
    val left = x * blockSize
    val right = left + blockSize - 1
    val top = y * blockSize
    val bottom = top + blockSize - 1

    for (row <- top to bottom) {
      w.moveTo(left, row)
      w.lineTo(right, row)
    }
}
\end{Code}

\Subtask\Pen
Metoden \code{block} ritar ett antal linjer.
Hur många linjer ritas ut?
I vilken ordning ritas linjerna?

\Subtask
Anropa funktionen \code{Graphics.block} några gånger i \code{Mole.main} så att några block ritas upp i fönstret när programmet körs.
Kompilera och kör ditt program.


\Task
Det finns många sätt att beskriva färger.
I naturligt språk har vi olika namn på färgerna, till exempel \emph{vitt}, \emph{rosa} och \emph{magenta}.
I datorn är det vanligt att beskriva färgerna som en blandning av \emph{rött}, \emph{grönt} och \emph{blått} i det så kallade RGB-systemet.
\code{SimpleWindow} använder typen \code{java.awt.Color} för att beskriva färger och \code{java.awt.Color} bygger på RGB.
Det finns några fördefinierade färger i \code{java.awt.Color}, till exempel \code{java.awt.Color.black} för svart och \code{java.awt.Color.green} för grönt.
Andra färger kan skapas genom att ange mängden rött, grönt och blått.

\Subtask
Skapa ett nytt objekt i \texttt{Mole.scala} med namnet \code{Colors} och lägg in följande definitioner:
\begin{Code}
val mole = new java.awt.Color(51, 51, 0)
val soil = new java.awt.Color(153, 102, 51)
val tunnel = new java.awt.Color(204, 153, 102)
\end{Code}
% val sky = new java.awt.Color(51, 51, 204)
% val grass = new java.awt.Color(51, 204, 51)
Den tre parametrarna till \code{new java.awt.Color(r, g, b)} anger hur mycket \emph{rött}, \emph{grönt} respektive \emph{blått} som färgen ska innehålla, och mängderna ska vara i intervallet 0--255.
Färgen $(153, 102, 51)$ innebär ganska mycket rött, lite mindre grönt och ännu mindre blått och det upplevs som brunt.
Objektet \code{Colors} är en färgpallett, men vi har inte ritat något med färg ännu.
Kompilera och kör ditt program ändå, för att se så programmet fungerar likadant som sist.

\Subtask
Lägg till en parameter till \code{Graphics.block} sist i parameterlistan med namnet \code{color} och typen \code{java.awt.Color}.
Låt \emph{default-argumentet} för den nya parametern vara \code{java.awt.Color.black}.
(Kommer du inte ihåg hur man gör default-argument kan du titta på övningarna i kapitel~\ref{exe:W03}.)
För att ändra färgen på blocket kan du byta linjefärg innan du ritar.
Lägg till följande rad i början på \code{Graphics.block}:
\begin{Code}
w.setLineColor(color)
\end{Code}
Kompilera och kör ditt program igen för att se om det fortfarande fungerar.

\Subtask\Pen
Funktionen \code{Graphics.block} har tre parametrar, men den anropas bara med två parametrar i \code{Mole.main}.
Varför är det tillåtet?
Vilket värde har den tredje parametern om ingen anges?

\Subtask
Ändra i \code{Mole.main} och lägg till en av definitionerna från objektet \code{Colors} som tredje parameter till \code{Graphics.block}.
Kompilera och kör ditt program och upplev världen i färg.

\Task
I programmet används många långa namn med punkter, som till exempel \code{java.awt.Color} och \code{Graphics.block}.
Dessa punkt-separerade namn kallas \emph{kvalificerade} namn.
För att slippa skriva dessa långa namn hela tiden kan man \emph{importera} en definition och sen använda bara den sista delen av namnet.

\Subtask
Importera namnet \code{java.awt.Color} i objektet \code{Colors}. Ändra sen alla \code{new java.awt.Color(...)} i objektet till \code{new Color(...)}.
(Har du glömt hur man importerar ett namn kan du gå tillbaka till övningarna i kapitel~\ref{exe:W02}.)

\Subtask\Pen
I vilka av objekten \code{Mole}, \code{Colors} och \code{Graphics} kan du använda det korta respektive det kvalificerade namnet av \code{java.awt.Color}?

\Subtask
Importera namnet \code{java.awt.Color} så att det korta namnet \code{Color} kan användas i objekten \code{Colors} och \code{Graphics} men inte i \code{Mole}.
Byt sedan ut de långa namnen mot de korta i \code{Graphics}.

\Task
Nu ska du skriva en funktion för att rita en rektangel. Rektangeln ska ritas med hjälp av funktionen \code{block}.
Sen ska du rita upp mullvadens underjordiska värld med hjälp av denna funktion.

\Subtask
Lägg till en funktion i objektet \code{Graphics} med namnet \code{rectangle} som tar fem parametrar \code{x}, \code{y}, \code{width} och \code{height} av typen \code{Int} och \code{color} av typen \code{Color}.
Parametrarna \code{x} och \code{y} anger \code{Graphics}-koordinaten för rektangelns övre vänstra hörn och \code{width} och \code{height} anger bredden respektive höjden.
Använd följande \code{for}-satser för att rita ut rektangeln.
\begin{Code}
for (yy <- y until (y + height)) {
	for (xx <- x until (x + width)) {
		block(xx, yy, color)
	}
}
\end{Code}

\Subtask\Pen
I vilken ordning ritas blocken ut?

% \Subtask\Pen (Fråga något om skuggning gällande \code{width} och \code{height}.)

\Subtask
Skriv en funktion i objektet \code{Mole} med namnet \code{drawWorld} som ritar ut mullvadens värld, det vill säga en massa jord där den kan gräva sina tunnlar.
\code{Mole.drawWorld} ska inte ha några parametrar och returtypen ska vara \code{Unit} och den ska anropa \code{Graphics.rectangle} för att rita en rektangel med färgen \code{Colors.soil} som precis täcker fönstret.
Eftersom funktionen har många parametrar som lätt kan blandas ihop ska du använda namngivna argument vid anropet.
(Om du har glömt hur man använder namngivna argument kan du titta på övningarna i kapitel~\ref{exe:W03}.)

\Subtask
Anropa \code{Mole.drawWorld} i \code{Mole.main} och testa så att det fungerar genom att kompilera och köra.

\Task
I \code{SimpleWindow} finns funktioner för att känna av tangenttryckningar och musklick.
Du ska använda de funktionerna för att styra en liten blockmullvad.

\Subtask
Importera \code{cslib.window.SimpleWindow} i objektet \code{Graphics} och lägg till följande funktion:
\begin{Code}
def waitForKey(): Char = {
	do {
		w.waitForEvent()
	} while (w.getEventType() != SimpleWindow.KEY_EVENT)
	w.getKey()
}
\end{Code}
Det finns olika sorters händelser som ett \code{SimpleWindow} kan reagera på, till exempel tangenttryckningar och musklick.
Funktionen som du precis lagt in väntar på en händelse i ditt \code{SimpleWindow} (\code{w.waitForEvent}) ända tills det kommer en tangenttryckning (\code{KEY_EVENT}).
När det kommit en tangenttryckning anropas \code{w.getKey} för att ta reda på vilken bokstav eller vilket tecken det blev, och det resultatet blir också resultatet av \code{waitForKey}, eftersom det ligger sist i det yttre \texttt{\{\}}-blocket.

\Subtask
Lägg till en funktion i objektet \code{Mole} med namnet \code{dig}, utan parametrar och med returtypen \code{Unit}.
Funktionens kropp ska se ut såhär (fast utan \code{???}):
\begin{Code}
{
  var x = Graphics.width / 2
  var y = Graphics.height / 2
  while (true) {
    Graphics.block(x, y, Colors.mole)
    val key = Graphics.waitForKey()
    if (key == 'w') ???
    else if (key == 'a') ???
    else if (key == 's') ???
    else if (key == 'd') ???
  }
}
\end{Code}
Fyll i alla \code{???} så att \code{'w'} styr mullvaden ett steg uppåt, \code{'a'} ett steg åt vänster, \code{'s'} ett steg nedåt och \code{'d'} ett steg åt höger.

\Subtask
Ändra \code{Mole.main} så att den bara innehåller två anrop: ett till \code{drawWorld} och ett till \code{dig}.
Kompilera och kör ditt program för att se om programmet reagerar på w, a, s och d.

\Subtask
Om programmet fungerar kommer det bli många mullvadar som tillsammans bildar en lång mask, och det är ju lite underligt.
Lägg till ett anrop i \code{Mole.dig} som ritar ut en bit tunnel på position $(x, y)$ efter anropet till \code{Graphics.waitForKey} men innan \code{if}-satserna.
Kompilera och kör ditt program för att gräva tunnlar med din blockmullvad.

\subsection{Frivilliga extrauppgifter}

\Task
Mullvaden kan för tillfället gräva sig utanför fönstret.
Lägg till några \code{if}-satser i början av \code{while}-satsen som upptäcker om \code{x} eller \code{y} ligger utanför fönstrets kant och flyttar i så fall tillbaka mullvaden precis innanför kanten.

\Task
Mullvadar är inte så intresserade av livet ovanför jord, men det kan vara trevligt att se hur långt ner mullvaden grävt sig.
Lägg till en himmelsfärg och en gräsfärg i objektet \code{Colors} och rita ut himmel och gräs i \code{Mole.drawWorld}.
Justera också det du gjorde i föregående uppgift, så mullvaden håller sig under jord.
(\emph{Tips:} Den andra parametern till \code{Color} reglerar mängden grönt och den tredje parametern reglerar mängden blått.)

\Task
Ändra så att mullvaden kan springa uppe på gräset också, men se till så att ingen tunnel ritas ut där.


%!TEX encoding = UTF-8 Unicode

%!TEX root = ../compendium1.tex

\chapter{Vektoralgoritmer}\label{chapter:W05}
\begin{itemize}[nosep]
\item vektoralgoritmer
\item min/max
\item strängar
\item registrering
\item java System.out.println
\item Scanner
\end{itemize}
\clearpage\section{Teori}
%!TEX encoding = UTF-8 Unicode
%!TEX root = ../lect-w05.tex

%%%

%TODO:
%  \begin{itemize}
%  \item Bygg upp \code{case class Complex(re: Double, im: Double)} steg för steg inspirerat av Pins3ed kap 6 i likhet med hur de gör med Rational
%  \item Illustrera följande begrepp: this (behövs i max(that)), method overloading behövs för att plussa med både Complex och Double
%  \item Till fördjupningsövning: dekorera Double med metoderna im och re samt (Double, Double) med metoden ir (för irrational) med implicit klass
%  \item Till extrauppgift: implementera klassen Polar(r, fi) med polära koordinater \url{https://sv.wikipedia.org/wiki/Pol%C3%A4ra_koordinater}
%  \end{itemize}

\Subsection{Vad är en klass?}

\begin{Slide}{Vad är en klass?}
\begin{itemize}
\item En klass är en mall för att skapa objekt.
\item Objekt skapas med \code{new Klassnamn} och kallas för  \Emph{instanser} av klassen \code{Klassnamn}.
\item En klass innehåller medlemmar \Eng{members}:
  \begin{itemize}
  \item \Emph{attribut}, kallas även fält \Eng{field}: \code{val}, \code{lazy val}, \code{var}
  \item \Emph{metoder}, kallas även operationer: \code{def}
  \end{itemize}
\item Varje instans har sin uppsättning värden på attributen
vilka tillsammans utgör instansens \Emph{tillstånd}.
\end{itemize}

\end{Slide}



\begin{Slide}{En klass liknar en stämpel}\SlideFontSmall
\begin{multicols}{2}

\emph{Metafor:} En klass liknar en \Emph{stämpel}

\begin{itemize}
\item En stämpel kan \Alert{tillverkas} -- motsvarar \Emph{deklaration} av klassen.
 \item Det händer inget förrän man \Alert{stämplar} -- motsvarar \code{new}.
\item Då skapas \Alert{avbildningar} av stämpeln -- motsvarar \Emph{allokering av ett objekt} som är en \Emph{instans} av klassen.
\item Allokering kallas också \Emph{konstruktion} och funktionen/koden som gör själva allokeringen kallas \Emph{konstruktor}.
\end{itemize}

\columnbreak

\includegraphics[width=0.4\textwidth]{../img/stamp}

\end{multicols}
\end{Slide}



\begin{Slide}[t]{Klass och instans}
\vspace{-0.65em}
\begin{REPLnonum}
scala> class C { var attr = 42 }

scala> val objRef1 = new C
\end{REPLnonum}
\vspace{3.7em}
\begin{tikzpicture}[font=\SlideFontSmall\sffamily]
\matrix [matrix of nodes, row sep=0, column 2/.style={nodes={rectangle,draw,minimum width=0.8cm}}] (mat)
{
\texttt{objRef1}   &  \makebox(10,10){ }\\
};

\node[cloud, cloud puffs=15.0, cloud ignores aspect, minimum width=2cm, minimum height=2cm,
 align=center, draw] (instance1) at (3.8cm, 0.0cm) {
 \begin{tabular}{r l}
 \texttt{attr} & \fbox{42} \\
 \end{tabular}
 };


\filldraw[black] ($ (mat-1-2) + (0.0cm,0.0cm) $) circle (3pt) node[] (ref1)  {};
\draw [arrow, line width=0.7mm] (ref1) -- (instance1);
\end{tikzpicture}
\end{Slide}



\begin{Slide}[t]{Klass och instans}
\vspace{-0.5em}
\begin{REPLnonum}
scala> class C { var attr = 42 }

scala> val objRef1 = new C

scala> val objRef2 = new C
\end{REPLnonum}
\vspace{2em}
\begin{tikzpicture}[font=\SlideFontSmall\sffamily]
\matrix [matrix of nodes, row sep=0, column 2/.style={nodes={rectangle,draw,minimum width=0.8cm}}] (mat)
{
\texttt{objRef1}   &  \makebox(10,10){ }\\
\texttt{objRef2}   &  \makebox(10,10){ }\\
};

\node[cloud, cloud puffs=15.0, cloud ignores aspect, minimum width=2cm, minimum height=2cm,
 align=center, draw] (instance1) at (3.8cm, 0.35cm) {
 \begin{tabular}{r l}
 \texttt{attr} & \fbox{42} \\
 \end{tabular}
 };

\node[cloud, cloud puffs=15.0, cloud ignores aspect, minimum width=2cm, minimum height=2cm,
 align=center, draw] (instance2) at (5.8cm, -1.5cm) {
 \begin{tabular}{r l}
 \texttt{attr} & \fbox{42} \\
 \end{tabular}
 };

\filldraw[black] ($ (mat-1-2) + (0.0cm,0.0cm) $) circle (3pt) node[] (ref1)  {};
\draw [arrow, line width=0.7mm] (ref1) -- (instance1);

\filldraw[black] ($ (mat-2-2) + (0.0cm,0.0cm) $) circle (3pt) node[] (ref2)  {};
\draw [arrow, line width=0.7mm] (ref2) -- (instance2);
\end{tikzpicture}
\end{Slide}



\begin{Slide}[t]{Klass och instans}
\vspace{-0.5em}
\begin{REPLnonum}
scala> class C { var attr = 42 }

scala> val objRef1 = new C

scala> val objRef2 = new C

scala> objRef2.attr = 43
\end{REPLnonum}
\begin{tikzpicture}[font=\SlideFontSmall\sffamily]
\matrix [matrix of nodes, row sep=0, column 2/.style={nodes={rectangle,draw,minimum width=0.8cm}}] (mat)
{
\texttt{objRef1}   &  \makebox(10,10){ }\\
\texttt{objRef2}   &  \makebox(10,10){ }\\
};

\node[cloud, cloud puffs=15.0, cloud ignores aspect, minimum width=2cm, minimum height=2cm,
 align=center, draw] (instance1) at (3.8cm, 0.35cm) {
 \begin{tabular}{r l}
 \texttt{attr} & \fbox{42} \\
 \end{tabular}
 };

\node[cloud, cloud puffs=15.0, cloud ignores aspect, minimum width=2cm, minimum height=2cm,
 align=center, draw] (instance2) at (5.8cm, -1.5cm) {
 \begin{tabular}{r l}
 \texttt{attr} & \fbox{43} \\
 \end{tabular}
 };


\filldraw[black] ($ (mat-1-2) + (0.0cm,0.0cm) $) circle (3pt) node[] (ref1)  {};
\draw [arrow, line width=0.7mm] (ref1) -- (instance1);

\filldraw[black] ($ (mat-2-2) + (0.0cm,0.0cm) $) circle (3pt) node[] (ref2)  {};
\draw [arrow, line width=0.7mm] (ref2) -- (instance2);
\end{tikzpicture}
\end{Slide}



\begin{Slide}{Klassdeklarationer och instansiering}\SlideFontSmall
\setlength{\leftmargini}{0pt}
\begin{itemize}
\item Syntax för deklaration av klass: \\ \vspace{0.5em}{\SlideFontSize{13}{16}\code|class Klassnamn(parametrar){ medlemmar }|}\vspace{0.5em}
\item Exempel: \Emph{deklaration}
\begin{Code}
class Klassnamn(val attribut1: Int, attribut2: String){
  val attribut3: Double = 42.0              //publikt oföränderligt attribut
  private var attribut4: Boolean = false    //privat medlem syns inte utåt
  def metod(parameter: Int) = parameter + 1 //funktion i klass kallas metod
  lazy val attr5 = Vector.fill(100000)(42.0)     //fördröjd initialisering
}
\end{Code}

\item Parametrar initialiseras med de argument som ges vid \code{new}.
\item Exempel: \Emph{instansiering} med argument för initialisering av klassparametrar
\begin{Code}
val instansReferens = new Klassnamn(42, "hej")
\end{Code}

\item Attribut blir \Emph{publika} (alltså synliga utåt) om inte modifieraren \code{private} anges.
\item Parametrar som inte föregås av modifierare (t.ex. private val, val, var) blir \Emph{attribut} som är \code{ private[this] val } och bara är synliga i \Alert{denna} instans.

\end{itemize}
\end{Slide}



\begin{Slide}{Övning: en klass som representerar en person}
\begin{enumerate}
  \item Deklarera en klass \code{Person} med dessa publika attribut:
  \begin{itemize}
    \item oföränderligt förnamn
    \item oföränderligt efternamn
    \item förändringsbar ålder med defaultargument \code{0}
  \end{itemize}
  \item lägg till en metod i klasskroppen med explicit returtyp som ger en 2-tupel med förnamn och efternamn
  \item skriv en deklaration som deklarerar en variabel \code{p} som initialiseras med värdet av ett uttryck som instansierar klassen \code{Person} med ditt namn och din födelseålder
  \item skriv en sats som skriver ut ditt förnamn genom att referera attribut med punktnotation
  \item skriv en tilldelningssats som ändrar tillståndet för den instans som referensen \code{p} refererar till så att åldersattributets värde blir din nuvarande ålder
\end{enumerate}
\end{Slide}



\begin{Slide}{Lösning: klassen Person}
\begin{Code}[basicstyle=\SlideFontSize{6.9}{9}\ttfamily]
class Person(val givenName: String, val familyName: String, var age: Int = 0){
  def name: (String, String) = (givenName, familyName)
}
\end{Code}
\begin{REPLnonum}[basicstyle=\SlideFontSize{7}{9}\ttfamily\color{white}]
val p = new Person("Björn", "Regnell")
println(p.name._1)
p.age = 50
\end{REPLnonum}
\pause
Hur kan vi ändra lösning så att vi får andra varianter som fortfarande uppfyller kraven i övningsuppgiften?
\end{Slide}


\begin{Slide}{Instansprivata klassparametrar}\SlideFontSmall
\setlength{\leftmargini}{0pt}

\begin{itemize}
\item Parametrar som inte föregås av modifierare (t.ex. private val, val, var) blir \Emph{attribut} som är \code{ private[this] val } och bara är synliga i \Alert{denna} instans.
\item Exempel på konsekvensen av \code{private[this]}:
\begin{REPL}
scala> class C(a: Int){ def add(other: C): Int = a + other.a }
error: value a is not a member of C

// betyder samma sak som:

scala> class C(private[this] a: Int){ def add(other: C): Int = a + other.a }
error: value a is not a member of C
\end{REPL}
\item Men detta fungerar fint:
\begin{REPL}
scala> class D(val a: Int){ def add(other: D): Int = a + other.a }

scala> (new D(42)).add(new D(43))
res0: Int = 85
\end{REPL}
...eftersom modifieraren \code{val} framför klassparameter ger publik synlighet.
\item Vad händer om du skriver \code{private val} framför klassparametern \code{a} ovan?
\end{itemize}
\end{Slide}





\begin{Slide}{Exempel: Klassen Complex i Scala}\SlideFontSmall
\scalainputlisting[basicstyle=\ttfamily\SlideFontSize{7}{9}]{../compendium/examples/complex1.scala}
Klassparametrarna är parametrar till den s.k. \Emph{primärkonstruktorn}.
\begin{REPL}
scala> val c1 = new Complex(3, 4)  // konstruktion av instans av Complex
c1: Complex = 3.0 + 4.0i

scala> val polarForm = (c1.r, c1.fi)
polarForm: (Double, Double) = (5.0,0.6435011087932844)

scala> val c2 = new Complex(1, 2)
c2: Complex = 1.0 + 2.0i

scala> c1 + c2
res0: Complex = 4.0 + 6.0i
\end{REPL}
\end{Slide}



\begin{Slide}{Exempel: Principen om enhetlig access}\SlideFontSmall
\scalainputlisting[basicstyle=\ttfamily\SlideFontSize{7}{9}]{../compendium/examples/complex2.scala}
\pause
\begin{itemize}
\item Efter som attributen \code{re} och \code{im} är oföränderliga, kan vi lika gärna ändra i klass-implementationen och göra om metoderna \code{r} och \code{fi} till \code{val}-variabler utan att klientkoden påverkas.

\item Då anropas \code{math.hypot} och \code{math.atan2} bara en gång vid initialisering (och inte varje gång som med \code{def}).

\item Vi skulle även kunna använda \code{lazy val} och då bara räkna ut \code{r} och \code{fi} om och när de verkligen refereras av klientkoden, annars inte.

\item Eftersom klientkoden inte ser skillnad på metoder och variabler, kallas detta \Emph{principen om enhetlig access}. (Många andra språk har \Alert{inte} denna möjlighet, tex Java där metoder \emph{måste} ha parenteser.)
\end{itemize}
\end{Slide}






\Subsection{Olika sätt att skapa instanser}

\begin{Slide}{Instansiering med direkt användning av \texttt{new}}

Instansiering genom \Emph{direkt användning} av \code{new}\\
{\SlideFontSmall (här första varianten av Complex med \code{r} och \code{fi} som metoder)}
\begin{REPLnonum}
scala> val c1 = new Complex(3, 4)
\end{REPLnonum}
\begin{tikzpicture}[font=\SlideFontSmall\sffamily]
\matrix [matrix of nodes, row sep=0, column 2/.style={nodes={rectangle,draw,minimum width=0.8cm}}] (mat)
{
\texttt{c1}   &  \makebox(10,10){ }\\
};

\node[cloud, cloud puffs=15.0, cloud ignores aspect, minimum width=2cm, minimum height=3.8cm,
 align=center, draw] (instance1) at (5.8cm, -1.5cm) {
 \begin{tabular}{r l l}
 \texttt{re:} & \texttt{Double} & \fbox{3.0} \\
 \texttt{im:} & \texttt{Double} & \fbox{4.0}\\
 \texttt{imSymbol:} & \texttt{Char} & \fbox{i}\\
 \end{tabular}
 };

\filldraw[black] ($ (mat-1-2) + (0.0cm,0.0cm) $) circle (3pt) node[] (ref1)  {};

\draw [arrow, line width=0.7mm] (ref1) -- (instance1);
\end{tikzpicture}
\pause
Ofta vill man göra \Alert{indirekt} instansiering så att vi senare har friheten att ändra hur instansiering sker.
\end{Slide}



\begin{Slide}{Indirekt instansiering med fabriksmetoder}\SlideFontSmall
En \Emph{fabriksmetod} är en metod som används för att instansiera objekt.
\begin{Code}[basicstyle=\SlideFontSize{8}{12}\ttfamily\selectfont]
object MyFactory {
  def createComplex(re: Double, im: Double) = new Complex(re, im)
  def createReal(re: Double)                = new Complex(re, 0)
  def createImaginary(im: Double)           = new Complex(0, im)
}
\end{Code}
\pause
Instansiera \Alert{inte direkt}, utan \Emph{indirekt} genom användning av \Emph{fabriksmetoder}:
\begin{REPL}
scala> import MyFactory._

scala> createComplex(3, 4)
res0: Complex = 3.0 + 4.0i

scala> createReal(42)
res1: Complex = 42.0 + 0.0i

scala> createImaginary(-1)
res2: Complex = 0.0 + -1.0i
\end{REPL}
\end{Slide}



\begin{Slide}{Hur förhindra direkt instansiering?}
Om vi vill \Emph{förhindra direkt instansiering} kan vi göra primärkonstruktorn \Alert{privat}:
\scalainputlisting[basicstyle=\ttfamily\SlideFontSize{7}{9}]{../compendium/examples/complex3.scala}
MEN... då går det ju \Alert{inte} längre att instansiera något alls!  \code{   :(}
\begin{REPLnonum}
scala> new Complex(3,4)
error:
 constructor Complex in class Complex cannot be accessed
\end{REPLnonum}
\end{Slide}



\begin{Slide}{Kompanjonsobjekt kan förhindra direkt instansiering}\SlideFontSmall
\setlength{\leftmargini}{0pt}
\begin{itemize}
\item Ett \Emph{kompanjonsobjekt} är ett objekt som ligger i samma kodfil som en klass och har \Alert{samma namn} som klassen.

\item Medlemmar i ett kompanjonsobjekt \Alert{får accessa privata} medlemmar i kompanjonsklassen (och vice versa) och kompanjonsobjektet får därför accessa privat konstruktor och kan göra \code{new}.
\scalainputlisting[basicstyle=\ttfamily\SlideFontSize{7}{9}]{../compendium/examples/complex4.scala}

\item Fabriksmetoder i kompanjonsobjektet ovan och privat kontstruktor gör att vi \Alert{enbart} tillåter \Emph{indirekt instansiering}.
\end{itemize}
\end{Slide}

\begin{Slide}{Användning av kompanjonsobjekt med fabriksmetoder för indirekt instansiering}
Nu kan vi \Alert{bara} instansiera \Emph{indirekt}!  \code{   :)}
\begin{REPLnonum}
scala> Complex.real(42.0)
res0: Complex = 42.0 + 0.0i

scala> Complex.imag(-1)
res1: Complex = 0.0 + -1.0i

scala> Complex.apply(3,4)
res2: Complex = 3.0 + 4.0i

scala> Complex(3,4)
res3: Complex = 3.0 + 4.0i

scala> new Complex(3, 4)
error:
     constructor Complex in class Complex cannot be accessed
\end{REPLnonum}
\end{Slide}


\begin{Slide}{Alternativa direktinstansieringar med default-argument}\SlideFontSmall
Med \Emph{default-argument} kan vi erbjuda \Emph{alternativa} sätt att direktinstansiera.
\scalainputlisting[basicstyle=\ttfamily\SlideFontSize{7}{9}]{../compendium/examples/complex5.scala}
\begin{REPL}
scala> new Complex()
res0: Complex = 0.0 + 0.0i

scala> new Complex(re = 42)  //anrop med namngivet argument
res1: Complex = 42.0 + 0.0i

scala> new Complex(im = -1)
res2: Complex = 0.0 + -1.0i

scala> new Complex(1)
res3: Complex = 1.0 + 0.0i
\end{REPL}
\end{Slide}

\begin{Slide}{Alternativa sätt att instansiera med fabriksmetod}
Vi kan också erbjuda \Emph{alternativa} sätt att instansiera \Emph{indirekt} med fabriksmetoden \code{apply} i ett kompanjonsobjekt genom default-argument:
\scalainputlisting[basicstyle=\ttfamily\SlideFontSize{7}{9}]{../compendium/examples/complex6.scala}

\end{Slide}

\begin{Slide}{Medlemmar som bara behövs i en enda upplaga}
Attributet \code{imSymbol} passar bättre att ha i \Emph{kompanjonsobjektet}, eftersom det räcker att ha \Alert{en enda upplaga}, som kan vara gemensam för alla objekt:
\scalainputlisting[basicstyle=\ttfamily\SlideFontSize{7}{9}]{../compendium/examples/complex7.scala}

\end{Slide}



\begin{Slide}{Medlemmar i singelobjekt är statiskt allokerade}\SlideFontTiny

Minnesplatsen för \Emph{attribut i singelobjekt} allokeras automatiskt en gång för alla, och kallas därför \Emph{statiskt} allokerad. Singelobjektets namn \code{Complex} utgör en statisk referens till den enda instansen och är av typen \texttt{Complex.type}.

\begin{tikzpicture}[font=\SlideFontSmall\sffamily]
\matrix [matrix of nodes, row sep=0, column 2/.style={nodes={rectangle,draw,minimum width=0.8cm}}] (mat)
{
\texttt{Complex}   &  \makebox(10,10){ }\\
};

\node[cloud, cloud puffs=15.0, cloud ignores aspect, minimum width=1cm, minimum height=2cm,
 align=center, draw] (instance1) at (5.8cm, 0.0cm) {
 \begin{tabular}{r l l}
 \texttt{imSymbol:} & \texttt{Char} & \fbox{i}\\
 \end{tabular}
 };

\filldraw[black] ($ (mat-1-2) + (0.0cm,0.0cm) $) circle (3pt) node[] (ref1)  {};

\draw [arrow, line width=0.7mm] (ref1) -- (instance1);
\end{tikzpicture}

Nu bereder vi inte plats för \code{imSymbol} i varenda \Emph{dynamiskt} allokerade instans:
\begin{REPLnonum}
scala> val c1 = Complex(3, 4)
\end{REPLnonum}

\begin{tikzpicture}[font=\SlideFontSmall\sffamily]
\matrix [matrix of nodes, row sep=0, column 2/.style={nodes={rectangle,draw,minimum width=0.8cm}}] (mat)
{
\texttt{c1}   &  \makebox(10,10){ }\\
};

\node[cloud, cloud puffs=15.0, cloud ignores aspect, minimum width=2cm, minimum height=2cm,
 align=center, draw] (instance1) at (5.8cm, -0.0cm) {
 \begin{tabular}{r l l}
 \texttt{re:} & \texttt{Double} & \fbox{3.0} \\
 \texttt{im:} & \texttt{Double} & \fbox{4.0}\\
 \end{tabular}
 };

\filldraw[black] ($ (mat-1-2) + (0.0cm,0.0cm) $) circle (3pt) node[] (ref1)  {};

\draw [arrow, line width=0.7mm] (ref1) -- (instance1);
\end{tikzpicture}


\end{Slide}




\begin{Slide}{Attribut i kompanjonsobjekt användas för sådant som är gemensamt för alla instanser}

Om vi ändrar på statiska \code{imSymbol} så ändras \code{toString} för \Alert{alla} dynamiskt allokerade instanser.
\begin{REPLnonum}
scala> val c1 = Complex(3, 4)
c1: Complex = 3.0 + 4.0i

scala> Complex.imSymbol = 'j'
Complex.imSymbol: Char = j

scala> val c2 = Complex(5, 6)
c2: Complex = 5.0 + 6.0j

scala> c1
res0: Complex = 3.0 + 4.0j
\end{REPLnonum}
\end{Slide}


\begin{Slide}{Övning: en läskig mutant}\SlideFontSmall
\begin{enumerate}
\item Skapa en klass med namnet \code{Mutant} som har ett förändringsbart attribut som klassparameter med namnet \code{i} av typen \code{Int} med default-argumentet \code{5}.
\vspace{0.5em}

\item \begin{minipage}{0.5\textwidth}
Skriv kod som deklarerar två variabler med namnet \code{fem1} och \code{fem2} som refererar till samma \code{Mutant}-instans.
\end{minipage}
\hfill\begin{minipage}{0.32\textwidth}
\hfill\includegraphics[width=3.4cm]{../img/mutant.png}

En \code{Mutant}-instans där \code{i} kanske är fem.
\vspace{1em}
\end{minipage}

\item Skriv kod som ändra tillstånd via den ena mutantreferensen.

\item Syns ändringen via den andra mutantreferensen?
\end{enumerate}
\end{Slide}


\Subsection{Case-klasser}


\begin{Slide}{Case-klasser}
Case-klasser är ett smidigt sätt att skapa \Emph{oföränderliga} datastrukturer. Med nyckelordet \code{case} framför \code{class} får du mycket ''godis på köpet'':

\begin{itemize}
\item Klassparametrar blir automatiskt publika oföränderliga attribut\footnote{alltså \Alert{inte} \texttt{private[this] val } som i vanliga klasser.} och du slipper alltså skriva \code{val}
\item Du får en automatisk \Emph{toString} med klassens namn och värdet av alla \code{val}-attribut som ges av klassparametrarna och du slipper alltså skriva en egen toString
\item Du slipper skriva \code{new} eftersom du får ett automatiskt kompanjonsobjekt med en fabriksmetod \code{apply} för indirekt instansiering där alla klassparametrarnas \code{val}-attribut initialiseras.
\pause
\item ... och mer därtill men mer om det senare...
\end{itemize}
\end{Slide}




\begin{Slide}{Exempel: oföränderliga case-klassen \code{Point}}

\begin{Code}[basicstyle=\SlideFontSize{10}{12}\ttfamily]
case class Point(x: Double, y: Double)
\end{Code}

\begin{REPLnonum}
scala> val p1 = Point(3, 4)
p1: Point = Point(3.0,4.0)

scala> val p2 = p1
p2: Point = Point(3.0,4.0)

scala> p1.x = 42
error: reassignment to val
\end{REPLnonum}

\end{Slide}






\Subsection{Konstruktor}



\begin{Slide}{Vad är en konstruktor?}
\begin{itemize}
\item En \Emph{konstruktor} är den kod som exekveras när klasser instansieras med \code{new}.

\item Konstruktorn skapar nya objekt i minnet under körning när den anropas.

\item I Scala \Alert{genererar} kompilatorn en \Emph{primärkonstruktor} med maskinkod som initialiserar alla attribut baserat på klassparametrar och \code{val}- och \code{var}-deklarationer.

\pause

\item I Java får man \Alert{explicit} skriva alla konstruktorer med speciell syntax och göra alla initialiseringar själv. Man kan ha många olika alternativa konstruktorer i Java.

\item I Scala \Emph{kan} man också skriva egna alternativa s.k. \Emph{hjälpkonstrukturer}, men det är \Alert{ovanligt}, eftersom man har möjligheten med fabriksmetoder i kompanjonsobjekt och default-argument (saknas i Java).
\end{itemize}
\end{Slide}


\begin{Slide}{Hjälpkonstruktorer i Scala}%\SlideFontSmall
Fördjupning för kännedom:
\begin{itemize}
\item I Scala kan man skapa ett alternativ till primärkonstruktorn, en så kallad \Emph{hjälpkonstruktor} \Eng{auxilliary constructor} genom att deklarera en metod med det speciella namnet \code{this}.


\item Hjälpkonstruktorer \Alert{måste} börja med att anropa en \Alert{annan} konstruktor som står \Alert{före} i koden, till exempel primärkonstruktorn.
\end{itemize}

\begin{Code}
class Point(val x: Int, val y: Int, val z: Int){
  def this(x: Int, y: Int) = this(x, y, 0)   //anrop av primärkonstruktorn
  def this(x: Int) = this(x, 0)              //anrop av hjälpkonstruktor
}
\end{Code}

{\SlideFontSmall Genom att känna till hur hjälpkonstruktorer fungerar i Scala, blir det lättare att begripa konstruktorer i Java.}

\end{Slide}

\begin{Slide}{Användning av hjälpkonstruktor}
\begin{REPL}
scala> val p1 = new Point(1)
p1: Point = Point(1,0)

scala> val p2 = new Point(1, 2)
p2: Point = Point(1,2)

scala> val p3 = new Point(1, 2, 3)
p3: Point = Point(1,2,3)
\end{REPL}
\pause
Men man gör \Alert{mycket oftare} så här i Scala:
\begin{Code}[basicstyle=\ttfamily\SlideFontSize{8.5}{12}]
case class Point(val x: Int, val y: Int = 0, val z: Int = 0)
\end{Code}
(Eller en vanlig klass med fabriksmetod i kompanjonsobjekt.)
\end{Slide}



% \begin{Slide}{Vad gör skräpsamlaren?}\SlideFontSmall
% \begin{itemize}
% \item Scala och Java är båda programmeringsspråk som förutsätter en körmiljö med \Alert{automatisk}  \Emph{skräpsamling} \Eng{garbage collection}.
%
% \item \Emph{Skräpsamlaren} \Eng{the garbage collector} är ett program som automatiskt körs i bakgrunden då och då och \Emph{städar minnet} genom att frigöra den plats som upptas av \Alert{objekt som inte längre används}.
%
% \item JVM:en bestämmer själv när skräpsamlaren ska jobba och programmeraren har ingen kontroll över detta.
%
% \item Den stora \Emph{fördelen} med automatisk skärpsamling är att man slipper bry sig om det svåra och felbenägna arbetet att \Alert{avallokera} minne.
%
% \item \Alert{Nackdelen} är att man inte kan styra exakt hur och när skräpsamlingen ska ske och man kan därmed inte bestämma när processorn ska belastas med minneshanteringen. Detta är normalt inget problem, utom i vissa tidskritiska realtidssystem med hårda minnesbegränsningar och svarstidskrav.
%
% \item I språk utan automatisk skräpsamling, t.ex. C++, måste man ta hand om destruktion av objekt och skriva egna s.k. \Emph{destruktorer}.
% \end{itemize}
% \end{Slide}


\Subsection{Referens saknas: \texttt{null}}

\begin{Slide}{Referens saknas: \texttt{null}}
\begin{itemize}
\item I Java och många andra språk använder man ofta literalen \code{null} för att representera att ett \Alert{värde saknas}.

\item En referens som är \code{null} refererar inte till någon instans.

\item Om du försöker referera till instansmedlemmar med punktnotation genom en referens som är \code{null} kastas ett \Alert{undantag} \code{NullPointerException}.

\item Oförsiktig användning av \code{null} är en vanlig källa till \Alert{buggar}, som kan vara svåra att hitta och fixa.

\end{itemize}
\end{Slide}


\begin{Slide}{Exempel: \texttt{null}}
\begin{REPL}
scala> class Gurka(val vikt: Int)
defined class Gurka

scala> var g: Gurka = null        // ingen instans allokerad än
g: Gurka = null

scala> g.vikt
java.lang.NullPointerException

scala> g = new Gurka(42)          // instansen allokeras
g: Gurka = Gurka@1ec7d8b3

scala> g.vikt
res0: Int = 42

scala> g = null         // instansen kommer att destrueras av skräpsamlaren
\end{REPL}

\begin{itemize} \SlideFontSmall
\item Scala har \code{null} av kompabilitetsskäl, men det är brukligt att \Alert{endast} använda \code{null} om man anropar Java-kod.

\item Scala erbjuder smidiga \code{Option}, \code{Some} och \code{None} för säker hantering av saknade värden; mer om detta i vecka 10.



\end{itemize}
\end{Slide}






\Subsection{Klasser i Java}

\begin{Slide}{Typisk utformning av Java-klass}
Typisk ''anatomi'' av en Java-klass:
\begin{Code}[language=Java]
class Klassnamn {
    attribut, normalt privata
    konstruktorer, normalt publika
    metoder: publika getters, och vid förändringsbara objekt även setters
    metoder: privata abstraktioner för internt bruk
    metoder: publika abstraktioner tänkta att användas av klientkoden
}
\end{Code}
\href{http://www.oracle.com/technetwork/java/codeconventions-141855.html#1852}{\SlideFontSize{9}{8}www.oracle.com/technetwork/java/codeconventions-141855.html\#1852}
\end{Slide}




\begin{Slide}{Java-exempel: Klassen JComplex}\SlideFontSmall
\javainputlisting[basicstyle=\SlideFontSize{5}{6}\ttfamily\selectfont]{../compendium/examples/JComplex.java}
\end{Slide}




\begin{Slide}{Exempel: Använda JComplex i Scala-kod}
\begin{REPL}
$ javac JComplex.java
$ scala
Welcome to Scala 2.12.3 (Java HotSpot(TM) 64-Bit Server VM, Java 1.8.0_66).
Type in expressions for evaluation. Or try :help.

scala> val jc1 = new JComplex(3, 4)
jc1: JComplex = 3.0 + 4.0i

scala> val polarForm = (jc1.getR, jc1.getFi)
polarForm: (Double, Double) = (5.0,0.6435011087932844)

scala> val jc2 = new JComplex(1, 2)
jc2: JComplex = 1.0 + 2.0i

scala> jc1 add jc2
res0: JComplex = 4.0 + 6.0i
\end{REPL}
\end{Slide}




\begin{Slide}{Exempel: Använda JComplex i Java-kod}\SlideFontSmall
\javainputlisting[basicstyle=\SlideFontSize{8}{10}\ttfamily\selectfont]{../compendium/examples/JComplexTest.java}
\begin{itemize}
\item I Java måste man skriva \Alert{tomma parentes-par} efter metodnamnet vid \Alert{anrop av parameterlösa metoder}.

\item \Alert{Tupler finns inte} i Java, så det går inte på ett enkelt sätt att skapa par av värden som i Scala; ovan görs polär form till en sträng för utskrift.

\item \Alert{Operatornotation för metoder finns inte} i Java, så man måste i Java använda punktnotation och skriva: \code{jc1.add(jc2)}
\end{itemize}
\end{Slide}










\begin{Slide}{Statiska medlemmar i Java}
\begin{itemize}
\item Man kan \Alert{inte} deklarera explicita singelobjekt i Java och det finns inget nyckelord \code{object}.

\item I stället kan man deklarera \Emph{statiska medlemmar} i en klass med Java-nyckelordet \jcode{static}.

\item Exempel på hur vi kan göra detta inuti klassen \code{JComplex}:

\begin{Code}[language=Java,basicstyle=\SlideFontSize{10}{12}\ttfamily\selectfont]
    public static char imSymbol = 'i';
\end{Code}

\item Effekten blir den samma som ett singelobjekt i Scala:
\begin{itemize}
\item Alla statiska medlemmar i en Java-klass allokeras automatiskt och hamnar i en egen singulär ''klassinstans'' som existerar oberoende av de dynamiska instanserna.
\item De statiska medlemmarna accessas med punktnotation genom klassnamnet:
\begin{Code}[language=Java,basicstyle=\SlideFontSize{11}{13}\ttfamily\selectfont]
    JComplex.imSymbol = 'j';
\end{Code}

\end{itemize}


\end{itemize}
\end{Slide}



\Subsection{Referensen \texttt{this}}
\begin{Slide}{Referensen \texttt{this}}\SlideFontSmall
\begin{itemize}
\item Nyckelordet \code{this} ger en referens till den aktuella instansen.
\begin{REPLnonum}
scala> class Gurka(var vikt: Int){def jagSjälv = this}
defined class Gurka

scala> val g = new Gurka(42)
g: Gurka = Gurka@5ae9a829

scala> g.jagSjälv
res0: Gurka = Gurka@5ae9a829

scala> g.jagSjälv.vikt
res1: Int = 42

scala> g.jagSjälv.jagSjälv.vikt
res2: Int = 42
\end{REPLnonum}
\item Referensen \code{this} används ofta för att komma runt ''namnkrockar'' där variabler med samma namn gör så att den ena variabeln inte syns.
\end{itemize}
\end{Slide}



\Subsection{Getters och setters}

\begin{Slide}{Getters och setters i Java}\SlideFontSmall
\begin{itemize}
\item I Java finns inget motsvarande nyckelord \code{val} som garanterar oföränderliga attributreferenser.\footnote{Det finns visserligen \jcode{final} men det är annorlunda som vi ska se senare.}

\item Därför gör man i Java nästan alltid attribut \Alert{privata} för att förhindra att de ändras på ett okontrollerat sätt.

\item Java följer inte principen om enhetlig access: åtkomst av metoder och variabler sker med olika syntax.

\item Därför är det normala i Java att införa metoder som kallas \Emph{getters} och \Emph{setters}, som används för att \Alert{indirekt} läsa och uppdatera \Emph{attribut}.

\item Dessa metoder känns igen genom Java-konventionen att de heter något som börjar med \Emph{get} respektive \Emph{set}.

\item Med indirekt access av attribut kan man i Java åstadkomma \Emph{flexibilitet}, så att klassimplementationen kan ändras utan att ändra i klientkoden:
\begin{itemize}\SlideFontSmall
\item man kan t.ex. i efterhand ändra representation av de privata attributen eftersom all access sker genom getters och setters.
\end{itemize}

\item Om klassen \Alert{inte} erbjuder en \Alert{setter} för privata attribut kan man åstadkomma \Emph{oföränderliga} datastrukturer där attributreferenserna inte förändras efter allokering.
\end{itemize}
\end{Slide}



\begin{Slide}{Java-exempel: Klassen JPerson}\SlideFontSmall
\Emph{Indirekt} access av \Alert{privata} attribut:
\vspace{-1em}\begin{multicols}{2}
\javainputlisting[basicstyle=\SlideFontSize{7}{8}\ttfamily\selectfont]{../compendium/examples/JPerson.java}

\columnbreak

\begin{REPLnonum}[basicstyle=\SlideFontSize{7}{9}\ttfamily\color{white}]
$ javac JPerson.java
$ scala
Welcome to Scala 2.11.8 (Java HotSpot(TM) 64-Bit Server VM, Java 1.8.0_66).
Type in expressions for evaluation. Or try :help.

scala> val p = new JPerson("Björn")
p: JPerson = JPerson@7e774085

scala> p.getAge
res0: Int = 0

scala> p.setAge(42)

scala> p.getAge
res1: Int = 42

scala> p.age
error:
value age is not a member of JPerson
\end{REPLnonum}
\end{multicols}
\end{Slide}


\begin{Slide}{Motsvarande JPerson men i Scala}
Så här brukar man åstadkomma ungefär motsvarande i Scala: \\~
\begin{Code}[basicstyle=\SlideFontSize{13}{15}\ttfamily\selectfont]
class Person(val name: String) {
  var age = 0
}
\end{Code}
~\\
Notera att alla attribut här är \Emph{publika}.
\end{Slide}


\begin{Slide}{Förhindra felaktiga attributvärden med setters}\SlideFontSmall
Med hjälp av \Emph{setters} kan vi förhindra \Alert{felaktig} uppdatering av attributvärden, till exempel \Alert{negativ ålder} i klassen \code{JPerson} i Java:
\begin{Code}[language=Java]
    public void setAge(int age){
        if (age >= 0) {
            this.age = age;
        else {
            this.age = 0;
        }
    }
\end{Code}
Hur kan vi åstadkomma \Emph{motsvarande i Scala}? \\
\pause
Antag att vi började med nedan variant, men \Alert{ångrar} oss och sedan vill införa funktionalitet som förhindrat negativ ålder \Emph{utan att ändra i klientkod}:
\begin{Code}
class Person(val name: String) {
  var age = 0
}
\end{Code}
Om vi inför en ny metod \code{setAge} och gör attributet \code{age} privat så funkar det \Alert{inte} längre att skriva  \code{ p.age = 42 } och vi ''kvaddar'' klientkoden! \code{  :(}
\end{Slide}



\begin{Slide}{Getters och setters i Scala}\SlideFontSmall
\setlength{\leftmargini}{0pt}
\begin{itemize}
\item Principen om \Emph{enhetlig access} tillsammans med \Alert{specialsyntax} för \Emph{setters} kommer till vår räddning!

\item
En \Emph{setter} i Scala är en \textbf{procedur som har ett namn som slutar med} \texttt{\_=}
\pause
\item I Scala kan man utan att kvadda klientkod införa getter+setter så här:
\end{itemize}
\begin{Code}
class Person(val name: String) { // ändrad implementation men samma access
  private var myPrivateAge = 0
  def age = myPrivateAge         // getter
  def age_=(a: Int): Unit =      // setter
    if (a >= 0) myPrivateAge = a else myPrivateAge = 0
}
\end{Code}
\pause\vspace{-0.5em}
\begin{REPL}
scala> val p = new Person("Björn")
p: Person = Person@28ac3dc3

scala> p.age = 42      // najs syntax om getter parad med setter enl ovan
p.age: Int = 42

scala> p.age = -1      // nu förhindras negativ ålder
p.age: Int = 0
\end{REPL}
\end{Slide}

%!TEX encoding = UTF-8 Unicode
%!TEX root = ../lect-w05.tex

%%%


\Subsection{Likhet}
\begin{Slide}{Referenslikhet eller strukturlikhet?}\SlideFontSmall
Det finns två \Alert{principiellt olika} sorters \Emph{likhet}:
\begin{itemize}
\item \Emph{Referenslikhet} \Eng{reference equality} där två referenser anses lika om de refererar till \Emph{samma instans} i minnet.
\item \Emph{Strukturlikhet} \Eng{structural equality} där två referenser anses lika om de refererar till instanser med \Emph{samma innehåll}.

\pause

\item I Scala finns flera metoder som testar likhet:
\begin{itemize}\SlideFontSmall
\item metoden \code{eq} testar referenslikhet och \code{r1.eq(r2)} ger \code{true} om \code{r1} och \code{r2} refererar till \Emph{samma} instans.

\item metoden \code{ne} testar referens\textbf{o}likhet och \code{r1.ne(r2)} ger \code{true} om \code{r1} och \code{r2} refererar till \Alert{olika} instanser.

\item metoden \code{==} som anropar metoden \code{equals} som default testar referenslikhet men som \Alert{kan överskuggas} om man \Emph{själv vill bestämma} om det ska vara referenslikhet eller strukturlikhet.
\end{itemize}

\pause

\item Scalas \Emph{standardbibliotek} och \Emph{grundtyperna} \code{Int}, \code{String} etc. testar \Emph{strukturlikhet} genom metoden \code{==}
\pause
\item I Java är det annorlunda: symbolen \code{==} är ingen metod i Java utan specialsyntax som  testar referenslikhet mellan instanser, medan metoden \code{equals} kan överskuggas med valfri likhetstest.
\end{itemize}
\end{Slide}


\begin{Slide}{Exempel: referenslikhet och strukturlikhet}
I Scalas standardbibliotek har man överskuggat \code{equals} så att metoden \code{==} ger test av \Emph{strukturlikhet} mellan instanser:
\begin{REPL}
scala> val v1 = Vector(1,2,3)
v1: scala.collection.immutable.Vector[Int] = Vector(1, 2, 3)

scala> val v2 = Vector(1,2,3)
v2: scala.collection.immutable.Vector[Int] = Vector(1, 2, 3)

scala> v1 eq v2                //referenslikhetstest: olika instanser
res0: Boolean = false

scala> v1 ne v2
res1: Boolean = true

scala> v1 == v2                //strukturlikhetstest: samma innehåll
res2: Boolean = true

scala> v1 != v2
res3: Boolean = false
\end{REPL}
\end{Slide}


\begin{Slide}{Referenslikhet och egna klasser}
Om du inte gör något speciellt med dina egna klasser så ger metoden \code{==} test av \Emph{referenslikhet} mellan instanser:
\begin{REPLnonum}
scala> class Gurka(val vikt: Int)

scala> val g1 = new Gurka(42)
g1: Gurka = Gurka@2cc61b3b

scala> val g2 = new Gurka(42)
g2: Gurka = Gurka@163df259

scala> g1 == g2       // samma innehåll men olika instanser
res0: Boolean = false

scala> g1.vikt == g2.vikt
res1: Boolean = true
\end{REPLnonum}
\end{Slide}

\ifkompendium\else
\fi

%!TEX encoding = UTF-8 Unicode
%!TEX root = ../lect-w05.tex

%%%

\begin{Slide}{Case-klasser ger innehållslikhet}
Förutom annat ''godis på köpet'' får du med \code{case class} även detta:
\begin{itemize}
\item Metoden \code{==} ger \Emph{innehållslikhet} (och inte referenslikhet).
\end{itemize}
\end{Slide}



\begin{Slide}{Likhet och case-klasser}
Metoden \code{equals} är i case-klasser automatiskt överskuggad så att metoden \code{==} ger test av strukturlikhet.
\begin{REPL}
scala> case class Gurka(vikt: Int)

scala> val g1 = Gurka(42)
g1: Gurka = Gurka(42)

scala> val g2 = Gurka(42)
g2: Gurka = Gurka(42)

scala> g1 eq g2          // olika instanser
res0: Boolean = false

scala> g1 == g2          // samma innehåll!
res1: Boolean = true
\end{REPL}
\end{Slide}



\begin{Slide}{Sammanfattning case-klass-godis}
Kom-ihåg-lista med ''godis'' i \code{case}-klasser så här långt:
\begin{enumerate}
\item klassparametrar blir \code{val}-attribut
\item najs toString
\item automatisk fabriksmetod \code{apply} i kompanjonsobjekt
\item == ger innehållslikhet \Eng{structural equality}
\pause~\\...
\end{enumerate}

\vspace{1em}Men vi har inte sett allt godis än: \\Mönstermatchning (mer om det senare).
\end{Slide}

%!TEX encoding = UTF-8 Unicode
%!TEX root = ../lect-w05.tex

%%%


\Subsection{Implementation saknas: ???}

\begin{Slide}{Implementation saknas: ???}
\begin{itemize}
\item Ofta vill man bygga kod iterativt och steg för steg lägga till olika funktionalitet.

\item Standardfunktionen \code{???} ger vid anrop undantaget \Alert{\texttt{NotImplementedError}} och kan användas på platser i koden där man ännu inte är färdig.

\item \code{???} tillåter \Emph{kompilering av ofärdig kod}.

\pause

\item Undantag har bottentypen \code{Nothing} som är subtyp till \emph{alla} typer och kan därmed tilldelas referenser av godtycklig typ.

\begin{REPLnonum}
scala> lazy val sprängsSnart: Int = ???

scala> sprängsSnart + 42
scala.NotImplementedError: an implementation is missing
\end{REPLnonum}

\end{itemize}
\end{Slide}

\begin{Slide}{Exempel: ofärdig kod}
\begin{Code}[basicstyle=\SlideFontSize{9}{11}\ttfamily\selectfont]
case class Person(name: String, age: Int):
  def ärTonåring = age >= 13 && age <= 19
  def ärUng = !ärGammal
  def ärGammal: Boolean = ???   //implementation ännu ej klar
\end{Code}
\begin{REPLnonum}
scala> Person("Björn", 51).ärTonåring
res23: Boolean = false

scala> Person("Sandra", 39).ärUng
scala.NotImplementedError: an implementation is missing
\end{REPLnonum}
\end{Slide}


% \Subsection{Klass-specifikationer}
%
%
%
%
% \begin{Slide}{Specifikationer av klasser i Scala}\footnotesize
% \begin{itemize}
% \item Specifikationer av klasser innehåller information som \emph{den som ska implementera} klassen behöver veta.
% \item Specifikationer innehåller liknande information som dokumentationen av klassen (scaladoc), som beskriver vad \emph{användaren} av klassen behöver veta.
% \end{itemize}
% \begin{ScalaSpec}{Person}
% /** Encapsulate immutable data about a Person: name and age. */
% case class Person(name: String, age: Int = 0){
%   /** Tests whether this Person is more than 17 years old. */
%   def isAdult: Boolean = ???
% }
% \end{ScalaSpec}
% \begin{itemize}
% \item Specifikationer av Scala-klasser utgör i denna kurs ofullständig kod som kan kompileras utan fel.
% \item Saknade implementationer markeras med \code{???}
% \item \Emph{Dokumentationskommentarer} utgör \Alert{krav} på implementationen.
% \end{itemize}
%
% \end{Slide}
%
%
% \begin{Slide}{Specifikationer av klasser och objekt}
% \begin{ScalaSpec}{MutablePerson}
% /** Encapsulates mutable data about a person. */
% class MutablePerson(initName: String, initAge: Int){
%   /** The name of the person. */
%   def getName: String = ???
%
%   /** Update the name of the Person */
%   def setName(name: String): Unit = ???
%
%   /** The age of this person. */
%   def getAge: Int = ???
%
%   /** Update the age of this Person */
%   def setAge(age: Int): Unit = ???
%
%   /** Tests whether this Person is more than 17 years old. */
%   def isAdult: Boolean = ???
%
%   /** A string representation of this Person, e.g.: Person(Robin, 25) */
%   override def toString: String = ???
% }
% object MutablePerson {
%   /** Creates a new MutablePerson with default age. */
%   def apply(name: String): MutablePerson = ???
% }
% \end{ScalaSpec}
%
% \end{Slide}
%
% \ifkompendium
% Man brukar inte använda \code{get} och \code{set} i metodnamn i Scala. Mer senare om principen om enhetlig access \Eng{uniform access principle} och hur man gör ''setters'' som möjliggör tilldelningssyntax.
% \fi

% \ifkompendium
% \begin{Slide}{Specifikationer av Java-klasser på extentor}
% \begin{itemize}\small
% \item Specificerar signaturer för konstruktorer och metoder.
% \item Kommentarerna utgör krav på implementationen.
% \item Används flitigt på extentor i EDA016, EDA011, EDA017...
% \item Javaklass-specifikationerna \Alert{saknar} \Emph{implementationer} och behöver kompletteras med metodkroppar och klassrubriker innan de kan kompileras.
% \end{itemize}
% \begin{JavaSpec}{class Person}
% /** Skapar en person med namnet name och åldern age. */
% Person(String name, int age);

% /** Ger en sträng med denna persons namn. */
% String getName();

% /** Ändrar denna persons ålder. */
% void setAge(int age);

% /** Anger åldersgränsen för när man blir myndig. */
% static int adultLimit = 18;
% \end{JavaSpec}
% \end{Slide}
% \fi


%\chapter{Vektoralgoritmer}\label{chapter:W05}
\begin{itemize}[nosep]
\item vektoralgoritmer
\item min/max
\item strängar
\item registrering
\item java System.out.println
\item Scanner
\end{itemize}

%!TEX encoding = UTF-8 Unicode
%!TEX root = ../exercises.tex

\ifPreSolution

\Exercise{\ExeWeekFIVE}\label{exe:W05}

\begin{Goals}
%!TEX encoding = UTF-8 Unicode

%!TEX root = ../compendium2.tex

\item Kunna deklarera klasser med klassparametrar.
\item Kunna skapa objekt med \code{new} och konstruktorargument.
\item Förstå innebörden av referensvariabler och värdet \code{null}.
\item Förstå innebörden av begreppen instans och referenslikhet.
\item Kunna använda nyckelordet \code{private} för att styra synlighet i primärkonstruktor.
\item Förstå i vilka sammanhang man kan ha nytta av en privat konstruktor.
\item Kunna implementera en klass utifrån en specifikation.
\item Förstå skillnaden mellan referenslikhet och strukturlikhet.
\item Känna till hur case-klasser hanterar likhet.
\item Förstå nyttan med att möjliggöra framtida förändring av attributrepresentation.
\item Känna till begreppen getters och setters.
\item Känna till accessregler för kompanjonsobjekt.
\item Känna till skillnaden mellan \code{==} och \code{eq}, samt \code{!=} versus \code{ne}.

\end{Goals}

\begin{Preparations}
\item \StudyTheory{05}
\end{Preparations}

\else


\ExerciseSolution{\ExeWeekFIVE}


\fi


\BasicTasks %%%%%%%%%%%


\WHAT{Para ihop begrepp med beskrivning.}

\QUESTBEGIN

\Task \what

\vspace{1em}\noindent Koppla varje begrepp med den (förenklade) beskrivning som passar bäst:

\begin{ConceptConnections}
  klass & 1 & & A & ett värde som ej refererar till någon instans \\ 
  instans & 2 & & B & nyckelord vid direkt instansiering av klass \\ 
  konstruktor & 3 & & C & ser privata medlemmar i klass med samma namn \\ 
  klassparameter & 4 & & D & binds till argument som ges vid konstruktion \\ 
  referenslikhet & 5 & & E & indirekt åtkomst av attributvärde \\ 
  innehållslikhet & 6 & & F & slipper skriva new; automatisk innehållslikhet \\ 
  case-klass & 7 & & G & instanser anses lika om de har samma tillstånd \\ 
  getter & 8 & & H & indirekt tilldelning av attributvärde \\ 
  setter & 9 & & I & hjälpfunktion för indirekt konstruktonsanrop \\ 
  kompanjonsobjekt & 10 & & J & upplaga av ett objekt med eget tillståndsminne \\ 
  fabriksmetod & 11 & & K & skapar instans, allokerar plats för tillståndsminne \\ 
  \code|null| & 12 & & L & en mall för att skapa flera instanser av samma typ \\ 
  \code|new| & 13 & & M & instanser anses olika även om tillstånden är lika \\ 
\end{ConceptConnections}

\SOLUTION

\TaskSolved \what

\begin{ConceptConnections}
  klass & 1 & ~~\Large$\leadsto$~~ &  I & en mall för att skapa flera instanser av samma typ \\ 
  instans & 2 & ~~\Large$\leadsto$~~ &  F & upplaga av ett objekt med eget tillståndsminne \\ 
  konstruktor & 3 & ~~\Large$\leadsto$~~ &  E & skapar instans, allokerar plats för tillståndsminne \\ 
  klassparameter & 4 & ~~\Large$\leadsto$~~ &  M & binds till argument som ges vid konstruktion \\ 
  referenslikhet & 5 & ~~\Large$\leadsto$~~ &  L & instanser anses olika även om tillstånden är lika \\ 
  innehållslikhet & 6 & ~~\Large$\leadsto$~~ &  J & instanser anses lika om de har samma tillstånd \\ 
  case-klass & 7 & ~~\Large$\leadsto$~~ &  D & slipper skriva new; automatisk innehållslikhet \\ 
  getter & 8 & ~~\Large$\leadsto$~~ &  A & indirekt åtkomst av attributvärde \\ 
  setter & 9 & ~~\Large$\leadsto$~~ &  B & indirekt tilldelning av attributvärde \\ 
  kompanjonsobjekt & 10 & ~~\Large$\leadsto$~~ &  H & ser privata medlemmar i klass med samma namn \\ 
  fabriksmetod & 11 & ~~\Large$\leadsto$~~ &  G & hjälpfunktion för indirekt konstruktonsanrop \\ 
  \code|null| & 12 & ~~\Large$\leadsto$~~ &  K & ett värde som ej refererar till någon instans \\ 
  \code|new| & 13 & ~~\Large$\leadsto$~~ &  C & nyckelord vid direkt instansiering av klass \\ 
\end{ConceptConnections}

\QUESTEND


\WHAT{Klass och instans.}

\QUESTBEGIN

\Task \what~Du har i övning \texttt{\ExeWeekFOUR}~sett hur singelobjekt i en egen namnrymd  kan samla funktioner (metoder) och ha tillstånd (attribut). Men singelobjekt finns bara i en upplaga.

Vill du kunna skapa många objekt av samma typ behöver du en \emph{klass}. En objektupplaga som skapats ur en klass kallas en \emph{instans} av klassen. Varje instans har sitt eget tillstånd.

Deklarera singelobjektet och klassen nedan i REPL.

\begin{Code}
object Singelpunkt { var x = 1; var y = 2 }
class  Punkt       { var x = 3; var y = 2 }
\end{Code}

\Subtask  Antag att uttrycken till vänster evalueras uppifrån och ned. Vilket resultat till höger hör ihop med respektive uttryck? Prova i REPL om du är osäker.\footnote{Strängen efter \code{@}-tecknet är en hexadecimal representation av det heltal som tillordnas varje objekt för att systemet ska kunna särskilja olika instanser. \url{https://stackoverflow.com/questions/4712139}}


\begin{ConceptConnections}
  \code|Singelpunkt.x               | & 1 & & A & \code|2| \\ 
  \code|Punkt.x                     | & 2 & & B & \verb|p2: Punkt = Punkt@51ab04bd| \\ 
  \code|val p  = new Singelpunkt    | & 3 & & C & \verb|p1: Punkt = Punkt@27a1a53c| \\ 
  \code|val p1 = new Punkt          | & 4 & & D & \verb|error: not found: type| \\ 
  \code|val p2 = new Punkt          | & 5 & & E & \code|java.lang.NullPointerException| \\ 
  \code|{ p1.x = 1; p2.x }          | & 6 & & F & \code|1| \\ 
  \code|(new Punkt).y               | & 7 & & G & \code|3| \\ 
  \code|{ val p: Punkt = null; p.x }| & 8 & & H & \code|error: not found: value| \\ 
\end{ConceptConnections}

\Subtask Vid tre tillfällen blir det fel. Varför? Är det kompileringsfel eller exekveringsfel?

\SOLUTION

\TaskSolved \what

\SubtaskSolved

\begin{ConceptConnections}
  \code|Singelpunkt.x               | & 1 & ~~\Large$\leadsto$~~ &  E & \code|1| \\ 
  \code|Punkt.x                     | & 2 & ~~\Large$\leadsto$~~ &  A & \code|error: not found: value| \\ 
  \code|val p  = new Singelpunkt    | & 3 & ~~\Large$\leadsto$~~ &  F & \verb|error: not found: type| \\ 
  \code|val p1 = new Punkt          | & 4 & ~~\Large$\leadsto$~~ &  D & \code|p1: Punkt = Punkt@27a1a53c| \\ 
  \code|val p2 = new Punkt          | & 5 & ~~\Large$\leadsto$~~ &  C & \code|p2: Punkt = Punkt@51ab04bd| \\ 
  \code|{ p1.x = 1; p2.x }          | & 6 & ~~\Large$\leadsto$~~ &  G & \code|3| \\ 
  \code|(new Punkt).y               | & 7 & ~~\Large$\leadsto$~~ &  B & \code|2| \\ 
  \code|{ val p: Punkt = null; p.x }| & 8 & ~~\Large$\leadsto$~~ &  H & \code|java.lang.NullPointerException| \\ 
\end{ConceptConnections}

\SubtaskSolved

\noindent\begin{tabular}{l l p{5cm}}

~\\ \emph{fel} & \emph{typ} & \emph{förklaring} \\\hline

\code|error: not found: value|
& kompileringsfel & det finns ingen instans med namnet \code|Punkt|\\

\verb|error: not found: type|
& kompileringsfel & det finns ingen klass som heter \code|Singelpunkt|\\

\code|NullPointerException|
& körtidsfel & det går inte att referera attribut i en instans som inte finns\\

\end{tabular}

\QUESTEND



\WHAT{Klassparametrar.}

\QUESTBEGIN

\Task \what~Klassen punkt i föregående uppgift är inte så smidig att använda eftersom man först \emph{efter} instansiering kan ge attributen \code{x} och \code{y} de koordinatvärden man önskar och detta måste ske med explicita tilldelningssatser.

Detta problem kan du lösa med \emph{klassparametrar} som låter dig initialisera attributen med konstruktionsargument och på så sätt ange ett initialtillstånd direkt i samband med instansiering.

Deklarera klassen nedan i REPL.

\begin{Code}
class Point(var x: Int, var y: Int)
\end{Code}


\Subtask  Antag att uttrycken till vänster evalueras uppifrån och ned. Vilket resultat till höger hör ihop med respektive uttryck? Prova i REPL om du är osäker.

\begin{ConceptConnections}
  \code|val p1 = Point(1, 2)        | & 1 & & A & \code|1| \\ 
  \code|val p2 = new Point          | & 2 & & B & \verb|p2: Point = Point@218cf600| \\ 
  \code|val p1 = new Point(1, 2)    | & 3 & & C & \code|error: not found: value| \\ 
  \code|val p2 = new Point(3, 4)    | & 4 & & D & \code|error: too many arguments| \\ 
  \code|p2.x - p1.x                 | & 5 & & E & \code|error: not enough arguments| \\ 
  \code|(new Point(0, 1)).y         | & 6 & & F & \code|2| \\ 
  \code|new Point(0, 1, 2)          | & 7 & & G & \verb|p1: Point = Point@30ef773e| \\ 
\end{ConceptConnections}

\Subtask Vid tre tillfällen blir det fel. Varför? Är det kompileringsfel eller exekveringsfel?

\SOLUTION

\TaskSolved \what

\SubtaskSolved

\begin{ConceptConnections}
  \code|val p1 = Point(1, 2)        | & 1 & ~~\Large$\leadsto$~~ &  C & \verb|p1: Point = Point@30ef773e| \\
  \code|val p2 = new Point          | & 2 & ~~\Large$\leadsto$~~ &  B & \verb|missing argument for parameter| \\
  \code|val p2 = new Point(3, 4)    | & 3 & ~~\Large$\leadsto$~~ &  D & \verb|p2: Point = Point@218cf600| \\
  \code|p2.x - p1.x                 | & 4 & ~~\Large$\leadsto$~~ &  F & \code|2| \\
  \code|(new Point(0, 1)).y         | & 5 & ~~\Large$\leadsto$~~ &  A & \code|1| \\
  \code|new Point(0, 1, 2)          | & 6 & ~~\Large$\leadsto$~~ &  E & \verb|too many arguments for constructor|

\end{ConceptConnections}

\SubtaskSolved

\noindent\begin{tabular}{l l p{5cm}}

  ~\\ \emph{fel} & \emph{typ} & \emph{förklaring} \\\hline

  \code|error: not found: value|
  & kompileringsfel & det finns ingen instans med namnet \code|Point|\\

  \verb|error: not enough arguments|
  & kompileringsfel  & du måste ge argument vid konstruktion av klassen \code|Point| \\

  \code|error: too many arguments|
  & kompileringsfel & antalet argument stämmer ej överens med antalet klassparametrar\\

\end{tabular}

\QUESTEND



\WHAT{Oföränderlig klass med defaultargument.}

\QUESTBEGIN

\Task \what~Det du tidigare lärt dig om parametrar och argument är tillämpligt även på klassparametrar, t.ex. defaultargument och namngivna argument. Man kan dessutom framför klassparametrar använda synlighetsmodifieraren \code{private} och nyckelorden \code{var} och \code{val}.

Om inget anges framför en klassparameter är det \code{private val} som gäller\footnote{För case-klasser, som vi ska se snart, är det i stället \code{val} som gäller (alltså inte \code{private}).}.

Deklarera nedan klass i REPL.

\begin{Code}
class Point3D(val x: Int = 0, val y: Int = 0, z: Int = 0)
\end{Code}

\Subtask Antag att uttrycken till vänster evalueras uppifrån och ned. Vilket resultat till höger hör ihop med respektive uttryck? Prova i REPL om du är osäker.

\begin{ConceptConnections}
  \code|val p1 = Point3D()          | & 1 & & A & \code|false| \\ 
  \code|val p2 = Point3D(y = 1)     | & 2 & & B & \code|Reassignment to val| \\ 
  \code|Point3D(z = 2).z            | & 3 & & C & \verb|p1: Point3D = Point3D@2eb37eee| \\ 
  \code|p2.y = 0                    | & 4 & & D & \code|true| \\ 
  \code|p2.y == 0                   | & 5 & & E & \code|value cannot be accessed| \\ 
  \code|p1.x == Point3D().x         | & 6 & & F & \verb|p2: Point3D = Point3D@65a9e8d7| \\ 
\end{ConceptConnections}

\Subtask Vad är problemet med ovan klass om man vill använda den för att representera punkter i 3 dimensioner?

\SOLUTION

\TaskSolved \what~

\SubtaskSolved

\begin{ConceptConnections}
  \code|val p1 = new Point3D        | & 1 & ~~\Large$\leadsto$~~ &  A & \verb|p1: Point3D = Point3D@2eb37eee| \\ 
  \code|val p2 = new Point3D(y = 1) | & 2 & ~~\Large$\leadsto$~~ &  B & \verb|p2: Point3D = Point3D@65a9e8d7| \\ 
  \code|(new Point3D(z = 2)).z      | & 3 & ~~\Large$\leadsto$~~ &  C & \code|error: not found: value| \\ 
  \code|p2.y = 0                    | & 4 & ~~\Large$\leadsto$~~ &  D & \code|error: reassignment to val| \\ 
  \code|p2.y == 0                   | & 5 & ~~\Large$\leadsto$~~ &  F & \code|false| \\ 
  \code|p1.x == (new Point3D).x     | & 6 & ~~\Large$\leadsto$~~ &  E & \code|true| \\ 
\end{ConceptConnections}

\SubtaskSolved Problemet är att så som klassen \code{Point3D} är deklarerad går det inte att avläsa \code{z}-koordinaten efter att en instans konstruerats. Det vore bättre om även \code{z}-attributet är \code{val}.

\QUESTEND



\WHAT{Case-klass.}

\QUESTBEGIN

\Task \what~\TODO

\begin{Code}
case class Pt(x: Int = 0, y: Int = 0) {
  def moved(dx: Int = 0, dy: Int = 0): Pt = Pt(x + dx, y + dy)
}

class MutablePt(private var p: (Int, Int) = (0, 0)) {
  def x: Int = p._1
  def y: Int = p._2
  def move(dx: Int = 0, dy: Int = 0) = { p = (x + dx, y+ dy); this }
}
\end{Code}

\Subtask Vilken returtyp kommer kompilatorn härleda för funktionen MutablePt.move?

\Subtask Implementera en fabriksmetod \code{apply} i ett kompanjonsobjekt till klassen \code{MutablePt} som gör att du inte behöver skriva \code{new} när du skapar instanser.

\SOLUTION

\TaskSolved \what~\TODO


\QUESTEND


\WHAT{Skapa en punktklass att använda på veckans laboration.}

\QUESTBEGIN

\Task \what~
Du ska som förberedelse till laborationen skapa den oföränderliga case-klassen \code{Point} som ska beskriva en koordinat i ett kartesiskt koordinatsystem\footnote{\url{https://sv.wikipedia.org/wiki/Kartesiskt_koordinatsystem}}. Skapa kod med hjälp av en editor, t.ex. \code{atom}, i filen  \code{Point.scala} enligt följande riktlinjer:
\begin{enumerate}[noitemsep]
\item \code{Point} ska ligga i paketet \code{graphics}.

\item \code{Point} ska ha följande två publika, oföränderliga klassparametrar:
\begin{itemize}[nolistsep, noitemsep]
\item \code{x: Double} för x-koordinaten.
\item \code{y: Double} för y-koordinaten.
\end{itemize}

\item \code{Point} ska ha följande publika medlemmar (två oföränderliga attribut och två metoder):
\begin{itemize}[nolistsep, noitemsep]
\item \code{val r: Double} ska ge motsvarande polära kordinatens%
\footnote{\url{https://sv.wikipedia.org/wiki/Pol\%C3\%A4ra\_koordinater}}
 avstånd till origo.
\item \code{val theta: Double} ska ge polära kordinatens vinkel i radianer.
\item \code{def negY: Point} ska ge en ny punkt med y-koordinaten negerad.
\item \code{def +(p: Point): Point} ska ge en ny punkt vars koordinat är summan av x- respektive y-kordinaterna för denna instans och punkten \code{p}.
\end{itemize}

\item \code{Point} ska ha ett kompanjonsobjekt med en metod som konstruerar en punkt från polära koortdinater. Metoden ska ha detta huvud: \\\code{def polar(r: Double, theta: Double): Point}

\end{enumerate}

\noindent Tips vid implementation och senare användning:
\begin{itemize}
\item Du har nytta av metoderna \code{math.hypot(x, y)} och \code{math.atan2(y, x)} vid omvandling till polära koordinater.

\item Du har nytta av metoderna \code{math.cos(x)} och \code{math.sin(y)} vid omvandling från polära koordinater.

\item Attributet \code{negY} kommer att underlätta för dig när du i metoden \code{draw} i klassen \code{Turtle} ska omvandla en punkt till fönsterkoordinater där y-axeln är omvänd jämfört med kartesiska koordinater.

\item Notera att klassens attribut är av typen \code{Double} och inte \code{Int}, trots att vi senare ska använda punkten för att beskriva en diskret pixelposition. Anledningen till detta är att det kan uppstå avrundningsfel vid numeriska beräkningar. Detta blir särskilt märkbart vid upprepad räkning med små värden, t.ex. när man ritar en approximerad cirkel med många linjesegment.
\end{itemize}

\SOLUTION

\TaskSolved \what~\TODO

\QUESTEND



\AdvancedTasks %%%%%%%%%%%%%%%%%%%%%%%%%%%%%%%%%%%%%%%%%%%%%%%%%%%%%%%%%%%%%%%%%

\WHAT{Ändra attributrepresentation. Privat konstruktor}

\QUESTBEGIN

\Task \what~Kim Kodkunnig skapade för länge sedan denna punktklass som används på många ställen i befintlig kod:

\begin{Code}
class Point private (val x: Int, val y: Int)
object Point {
  def apply(x: Int = 0, y: Int = 0): Point = new Point(x, y)
  def origo = apply()
}
\end{Code}

\Subtask Vad händer om du försöker instansiera Kim Kodkunnigs klass i din egen kod direkt med nyckelordet \code{new}?

\Subtask Varför använder Kim Kodkunnig ett kompanjonsobjekt med en fabriksmetod? Vilka accessregler gäller mellan ett kompanjonsobjekt och klassen med samma namn?

\Subtask Hjälp Kim Kodkunnig att ändra attributrepresentationen så att det oföränderliga tillståndet utgörs av en 2-tupel \code{val p: (Int, Int)} i stället. Befintlig kod ska inte behöva ändras och klassen \code{Point} ska bete sig från ''utsidan'' precis som innan.

\SOLUTION

\TaskSolved \what~

\SubtaskSolved Det blir kompileringsfel eftersom konstruktorn är privat.
\begin{REPL}
scala> :paste

class Point private (val x: Int, val y: Int)
object Point {
  def apply(x: Int = 0, y: Int = 0): Point = new Point(x, y)
  def origo = apply()
}

scala> new Point(0,0)
<console>:14: error: constructor Point in class Point cannot be accessed
\end{REPL}

\SubtaskSolved
\begin{itemize}
  \item Genom att ha en privat konstruktor och bara göra indirekt instansiering via fakriksmetod är det möjligt att ändra attributrepresentation i framtiden utan att befintlig kod behöver ändras.

  \item Med en \code{apply}-metod i kompansjonsobjektet kan man instansiera genom att skriva \code{Point(1, 2)} utan new.

  \item Accessreglerna för kompanjonsobjekt är sådana att kompanjoner ser varandras privata delar.
\end{itemize}

\SubtaskSolved

\begin{Code}
class Point private (private val p: (Int, Int)) {
  def x: Int = p._1
  def y: Int = p._2
}
object Point {
  def apply(x: Int = 0, y: Int = 0): Point = new Point(x, y)
  def origo = apply()
}
\end{Code}

\QUESTEND


\subsection{\TODO värdera nedan uppgifter}


\WHAT{Instansiering med \code{new} och värdet \code{null}.}

\QUESTBEGIN

\Task  \what~  Man skapar instanser av klasser med \code{new}. Då anropas konstruktorn och plats reserveras i datorns minne för objektet. Variabler av referenstyp som inte refererar till något objekt har värdet \code{null}.

\Subtask Vad händer nedan? Vilka rader ger felmeddelande och i så fall hur lyder felmeddelandet?

\begin{REPL}
scala> class Gurka(val vikt: Int)
scala> var g: Gurka = null
scala> g.vikt
scala> g = new Gurka(42)
scala> g.vikt
scala> g = null
scala> g.vikt
\end{REPL}

\Subtask\Pen Rita minnessituationen efter raderna 2, 4, 6.

\SOLUTION


\TaskSolved \what


\SubtaskSolved  Rad 3 och 7 ger båda felmeddelandet "java.lang.NullPointerException". Detta eftersom \code{g} i båda fallen inte innehåller en referens till en \code{Gurka} utan pekar på inget -- "null".

\SubtaskSolved  \includegraphics[scale=0.6]{../img/w06-solutions/1b}


\QUESTEND




%%<AUTOEXTRACTED by mergesolu>%%      % uppgift 1




\WHAT{Klasser och instanser.}

\QUESTBEGIN

\Task  \what~

\Subtask Vad händer nedan?
\begin{REPL}
scala> :pa
class Arm(val ärTillVänster: Boolean)
class Ben(val ärTillVänster: Boolean)
class Huvud(val harHår: Boolean)
class Rymdvarelse {
  var arm1 = new Arm(true)
  var arm2 = new Arm(false)
  var ben1 = new Ben(true)
  var ben2 = new Ben(false)
  var huvud1 = new Huvud(false)
  var huvud2 = new Huvud(true)
  def ärSkallig = !huvud1.harHår && !huvud2.harHår
}
scala> val alien = new Rymdvarelse
scala> alien.ärSkallig
scala> val predator = new Rymdvarelse
scala> predator.ärSkallig
scala> predator.huvud2 = alien.huvud1
scala> predator.ärSkallig
\end{REPL}

\Subtask\Pen Rita minnessituationen efter rad 18.

\Subtask\Pen Vad händer så småningom med det ursprungliga huvud2-objektet i predator efter tilldelningen på rad 18? Går det att referera till detta objekt på något sätt?


\SOLUTION


\TaskSolved \what


\SubtaskSolved  Vi skapar två rymdvarelser, \code{alien} och \code{predator}, med två ben, två armar samt två huvuden (där det ena är skalligt och det andra har hår) vardera. Efter det är varken \code{alien} eller \code{predator} skallig eftersom båda har ett huvud med hår. Sen låter man referensen till \code{predator}s huvud med hår referera till aliens huvud utan hår. Nu är predator helt skallig.

\SubtaskSolved   \includegraphics[scale=0.7]{../img/w06-solutions/2b}

\SubtaskSolved  Eftersom det inte längre finns någon referens som pekar på det objektet kommer Garbage Collector ta hand om det och kommer förr eller senare skrivas över av något annat som behöver sparas. Nej, det går inte att komma åt.

% uppgift 3

\QUESTEND




%%<AUTOEXTRACTED by mergesolu>%%      % uppgift 2




\WHAT{Synlighet i primärkonstruktorer.}

\QUESTBEGIN

\Task  \what~  Undersök nedan vad nyckelorden \code{val} och \code{private} får för konsekvenser. Förklara vad som händer. Vilka rader ger vilka felmeddelanden?

\begin{REPL}
scala> class Gurka1(vikt: Int)
scala> new Gurka1(42).vikt
scala> class Gurka2(val vikt: Int)
scala> new Gurka2(42).vikt
scala> class Gurka3(private val vikt: Int)
scala> new Gurka3(42).vikt
scala> class Gurka4(private val vikt: Int, kompis: Gurka4){
         def kompisVikt = kompis.vikt
       }
scala> val ingenGurka: Gurka4 = null
scala> new Gurka4(42, ingenGurka).kompisVikt
scala> new Gurka4(42, new Gurka4(84, null)).kompisVikt
scala> class Gurka5(private[this] val vikt: Int, kompis: Gurka5){
         def kompisVikt = kompis.vikt
       }
scala> class Gurka6 private (vikt: Int)
scala> new Gurka6(42)
scala> :pa
class Gurka7 private (var vikt: Int)
object Gurka7 {
  def apply(vikt: Int) = {
    require(vikt >= 0, s"negativ vikt: $vikt")
    new Gurka7(vikt)
  }
}
scala> new Gurka7(-42)
scala> Gurka7(-42)
scala> val g = Gurka7(42)
scala> g.vikt
scala> g.vikt = -1
scala> g.vikt
\end{REPL}


\SOLUTION


\TaskSolved \what
 Rad 2:
\begin{REPL}
	error: value vikt is not a member of Gurka1
\end{REPL}
Detta eftersom om man varken väljer att skriva \code{val} eller \code{var} skapar inte scala någon getter eller setter (metoder för att läsa/ändra en variabel) och därför ser det ut som att vikt inte finns för kompilatorn.

Rad 4: Denna rad skapar inte en error eftersom om man skriver \code{val} innan variabeln skapas en getter automatiskt och man kan därför komma åt \code{vikt}.

Rad 6:
\begin{REPL}
	error: value vikt in class Gurka3 cannot be accessed in Gurka3
\end{REPL}
I detta fallet skapas en \code{getter} men eftersom accessnivån sätts till \code{private} vet kompilatorn att man inte får komma åt variabeln utifrån.

Rad 11:
\begin{REPL}
	java.lang.NullPointerException
\end{REPL}
Detta eftersom \code{kompis} är \code{ingenGurka} som inte pekar på något objekt och när man då försöker komma åt ett attribut från den kommer det inte funka.

Rad 12: Kommer inte generera en error eftersom när man kallar \code{kompisVikt} (som är \code{public}) försöker den komma åt \code{Gurka4(84, null).vikt}. \code{vikt} är \code{private val} vilket innebär att det har en getter och eftersom huvudobjektet också är av typen \code{Gurka4} är accessnivån tillräckligt hög.

Rad 13:
\begin{REPL}
	error: value vikt is not a member of Gurka5
\end{REPL}
När man sätter ett attribut till \code{private[this]} tillåts inte ens objekt av samma typ att komma åt variabeln och därför får man en error som säger att den inte finns.

Rad 17:
\begin{REPL}
	error: constructor Gurka6 in class Gurka6 cannot be accessed in object
\end{REPL}
Eftersom man satt klassparametrarna till \code{private} kan man inte komma åt konstruktorn och därför får man en error.

Rad 26:
\begin{REPL}
	error: constructor Gurka7 in class Gurka7 cannot be accessed in object
\end{REPL}
Samma anledning som på rad 17.

Rad 27:
\begin{REPL}
	java.lang.IllegalArgumentException: requirement failed: negativ vikt: -42
\end{REPL}
Kompanjonsobjektet har en requirement på att \code{vikt >= 0} vilket innebär att om det inte stämmer kommer man få en error av typen \code{IllegalArgumentException}.

Rad 30: Anledningen till att man kan sätta vikten till något negativt är att checken om det är negativt endast görs när man skapar \code{Gurka7} vilket innebär att i efterhand kan man ändra den till vilket värde som helst (av typen \code{Int}).


\QUESTEND






\WHAT{Egendefinierad setter kombinerat med privat konstruktor.}

\QUESTBEGIN

\Task  \what~

\Subtask Förklara vad som händer nedan. Vilka rader ger vilka felmeddelanden?
\begin{REPL}
scala> :pa
class Gurka8 private (private var _vikt: Int) {
  def vikt = _vikt
  def vikt_=(v: Int): Unit = {
    require(v >= 0, s"negativ vikt: $v")
    _vikt = v
  }
}

object Gurka8 {
  def apply(vikt: Int) = {
    require(vikt >= 0, s"negativ vikt: $vikt")
    new Gurka8(vikt)
  }
}
scala> val g = Gurka8(-42)
scala> val g = Gurka8(42)
scala> g.vikt
scala> g.vikt = 0
scala> g.vikt = -1
scala> g.vikt += 42
scala> g.vikt -= 1000
\end{REPL}

\Subtask\Pen Vad är fördelen med möjligheten att skapa egendefinierade setters?

\SOLUTION


\TaskSolved \what


\SubtaskSolved  Rad 16:
\begin{REPL}
	java.lang.IllegalArgumentException: requirement failed: negativ vikt: -42
\end{REPL}
Kompanjonsobjektet har en requirement på att \code{vikt >= 0} vilket innebär att om det inte stämmer kommer man få en error.

Rad 20:
\begin{REPL}
	java.lang.IllegalArgumentException: requirement failed: negativ vikt: -1
\end{REPL}
Eftersom settern har implementerat ett krav på att vikten måste vara större eller lika med 0 får man en error när man försöker sätta den till -1.

Rad 22:
\begin{REPL}
	java.lang.IllegalArgumentException: requirement failed: negativ vikt: -958
\end{REPL}
Eftersom 42-1000 är mindre än noll får man en error.

\SubtaskSolved  Man kan sätta egna mer specifika krav på vad som får göras med värdena så man har större koll på att inget oväntat händer.

% uppgift 5

\QUESTEND




%%<AUTOEXTRACTED by mergesolu>%%      % uppgift 4




\WHAT{En oföränderlig kvadrat med alternativ fabriksmetod.}

\QUESTBEGIN

\Task \label{task:Square} \what~

\Subtask Implementera klassen \code{Square} enligt nedan specifikation. Gör  implementationen i en kodeditor, så som \code{gedit}, och klistra in klassen i Scala REPL efter kommandot \code{:pa} (förkortning av \code{:paste}). På så sätt blir \code{object Square} ett kompanjonsobjekt till \code{class Square}.

\begin{ScalaSpec}{Square}
/** A class representing a square object with position and side. */
class Square(val x: Int, val y: Int, val side: Int) {
  /** The area of this Square */
  val area: Int = ???

  /** Creates a new Square moved to position (x + dx, y + dy) */
  def move(dx: Int, dy: Int): Square = ???

  /** Tests if this Square has equal size as that Square */
  def isEqualSizeAs(that: Square): Boolean = ???

  /** Multiplies the side with factor and rounded to nearest integer */
  def scale(factor: Double): Square = ???

  /** A string representation of this Square */
  override def toString: String = ???
}

object Square {
  /** A square placed in origin with size 1 */
  val unit: Square = ???

  /** Constructs a new Square object at (x, y) with size side */
  def apply(x: Int, y: Int, side: Int): Square = ???

  /** Constructs a new Square object at (0, 0) with side 1 */
  def apply(): Square = ???
}
\end{ScalaSpec}

\Subtask Testa din kvadrat enligt nedan. Förklara vad som händer.

\begin{REPL}
scala> val (s1, s2) = (Square(), Square(1, 10, 1))
scala> val s3 = s1.move(1,-5)
scala> s1 isEqualSizeAs s3
scala> s2 isEqualSizeAs s1
scala> s1 isEqualSizeAs Square.unit
scala> s2.scale(math.Pi) isEqualSizeAs s2
scala> s2.scale(math.Pi) isEqualSizeAs s2.scale(math.Pi)
\end{REPL}

\SOLUTION


\TaskSolved \what
 \begin{CodeSmall}
	class Square(val x: Int, val y: Int, val side: Int) {
		val area: Int = side*side

		def move(dx: Int, dy: Int): Square = new Square(x + dx, y + dy, side)

		def isEqualSizeAs(that: Square): Boolean = this.side == that.side

		def scale(factor: Double): Square = new Square(x, y, (side*factor).toInt)

		override def toString: String = s"Square(x: $x, y: $y, side: $side)"
	}

	object Square {
		val unit: Square = new Square(0, 0, 1)

		def apply(x: Int, y: Int, side: Int): Square = new Square(x, y, side)

		def apply(): Square = new Square(0, 0, 1)
	}
\end{CodeSmall}

Eftersom \code{s1}, \code{s2}, \code{s3} och \code{Square.unit} alla har en sida med längden 1 så kommer rad 3-5 returnera \code{true}. Rad 6 kommer returnera \code{false} eftersom \code{s2.scale(math.Pi)} sida är $\pi$ och \code{s2} fortfarande har sidan 1. Rad 7 kommer däremot returnera \code{true} då båda har sidan $\pi$.


\QUESTEND






\WHAT{Referenslikhet versus strukturlikhet.}

\QUESTBEGIN

\Task  \what~  Metoden \code{==} på case-klasser ger \textbf{strukturlikhet} (även kallad innehållslikhet) så att \emph{innehållet} i klassens klassparametrar jämförs om de har lika värde, medan för vanliga klasser ger metoden \code{==} \textbf{referenslikhet} där olika objekt är olika även om de har samma innehåll (om man inte överskuggar metoden \code{equals} som anropas av \code{==} vilket vi ska titta närmare på i kapitel \ref{chapter:W08}).

\begin{REPL}
scala> class GurkaRef(val vikt: Int)
scala> case class GurkaStrukt(val vikt: Int)
scala> val a = new GurkaRef(42)
scala> val b = new GurkaRef(42)
scala> val c = new GurkaStrukt(42)
scala> val d = new GurkaStrukt(42)
scala> a == b
scala> c == d
\end{REPL}

\Subtask Förklara vad som händer ovan.

\Subtask Istället för \code{==}, prova metoden \code{eq} på objekten ovan. Metoden \code{eq} ger alltid referenslikhet (även om byter ut metoden \code{equals}).

\SOLUTION


\TaskSolved \what


\SubtaskSolved  Variablerna \code{a} och \code{b} är båda objekt av en vanlig klass vilket kommer innebära att de jämförs med referenslikhet och eftersom de inte är samma objekt retunerar \code{==} \code{false}. \code{c} och \code{d} är däremot objekt av en case klass så de jämförs med strukturlikhet och eftersom de har samma vikt returnerar \code{==} \code{true}.

\SubtaskSolved  Både \code{a eq b} och \code{c eq d} ska returnera \code{false} eftersom de alla är olika objekt och det är referenslikhetsom gäller.


\QUESTEND




%%<AUTOEXTRACTED by mergesolu>%%      % uppgift 6




\WHAT{Klassen \code{Point} med case-klass.}

\QUESTBEGIN

\Task \label{task:Point} \what~

\Subtask Implementera klassen \code{Point} som en oföränderlig case-klass med heltalsattributen \code{x} och \code{y}.

\Subtask Lägg till metoden \code{distanceTo(that: Point): Double} som räknar ut avståndet till en annan punkt med hjälp av \code{math.hypot}.

\Subtask Lägg till metoden \code{distanceTo(x: Int, y: Int): Double} som räknar ut avståndet till koordinaterna x och y med hjälpa av metoden i föregående deluppgift.

\Subtask Lägg till metoden \code{move(dx: Int, dy: Int): Point} som skapar en ny punkt på translaterad position enligt delta-koordinaterna \code{dx} och {dy}.

\Subtask Lägg till ett kompanjonsobjekt med medlemmen \code{val origin} som ger en punkt i origo.

\Subtask Undersök metoderna \code{==}, \code{!=}, \code{eq} och \code{ne} och förklara vad som händer nedan:
\begin{REPL}
scala> Point(1, 2) == Point(1, 3)
scala> Point(1, 2) != Point(1, 3)
scala> Point(1, 2) == Point(1, 2)
scala> Point(1, 2) != Point(1, 2)
scala> Point.origin.move(1, 1) == Point.origin.move(1, 1)
scala> Point.origin.move(1, 1).move(1, 1) != Point(2, 2)
scala> Point(0, 0) eq Point(0, 0)
scala> Point(0, 0) ne Point(0, 0)
scala> Point.origin eq Point.origin
scala> Point.origin ne Point.origin
scala> val p1 = Point(0, 0)
scala> val p2 = p1
scala> p1 eq p2
\end{REPL}

\Subtask Vad ger \code{Point.origin eq Point.origin} för resultat om \code{origin} istället  implementeras som \code{def origin: Point = Point(0, 0)}

\Subtask\Pen Vad är det för skillnad på strukturlikhet och referenslikhet?

\SOLUTION


\TaskSolved \what


\SubtaskSolved  se e) för komplett lösning

\SubtaskSolved  se e) för komplett lösning

\SubtaskSolved  se e) för komplett lösning

\SubtaskSolved  se e) för komplett lösning

\SubtaskSolved  \begin{CodeSmall}
case class Point(x: Int, y: Int) {

	def distanceTo(that: Point): Double = math.hypot(that.x - x, that.y -y)

	def distanceTo(x: Int, y: Int): Double = distanceTo(Point(x, y))

	def move(dxdy: (Int, Int)): Point = Point(dxdy._1 + x, dxdy._2 + y)
}

object Point {
	//val origin: Point = new Point(0, 0)
	def origin: Point = Point(0, 0)
}
\end{CodeSmall}

\SubtaskSolved  \code{==} och \code{!=} kollar strukturlikhet så om två objekt innehåller samma värden kommer \code{==} returnera \code{true} och \code{!=} \code{false} och vise versa. \code{eq} och \code{ne} kollar referenslikhet så om två variabler pekar på samma objekt kommer \code{eq} returnera \code{true} och \code{ne} \code{false} och vise versa.

\SubtaskSolved  \code{false}. Detta eftersom om origin implementeras som en metod som returnerar en ny \code{Point} varje gång den kallas kommer \code{Point.origin} inte peka på samma objekt varje gång metoden kallas (\code{eq} är referenslikhet).

\SubtaskSolved  Sturkturlikhet bryr sig endast om innehållet i objekten och jämför det. Det kvittar alltså om det är samma objekt eller två olika så länge de innehåller samma värden. Referenslikhet kollar endast på om det är samma objekt variablerna pekar på och struntar fullständigt i om de innehåller samma värden.

% uppgift 8

\QUESTEND




%%<AUTOEXTRACTED by mergesolu>%%      % uppgift 7




\WHAT{NEEDS A TOPIC DESCRIPTION}

\QUESTBEGIN

\Task \label{task:PointSquare} \what~ Ändra representationen av positionen i klassen \code{Square} från deluppgift \ref{task:Square} till att vara en \code{Point} från deluppgift \ref{task:Point}.


\SOLUTION


\TaskSolved \what
 \begin{CodeSmall}
class Square(val p: Point, val side: Int) {
	val area: Int = side*side

	def move(dx: Int, dy: Int): Square = new Square(p.move(dx, dy), side)

	def isEqualSizeAs(that: Square): Boolean = this.side == that.side

	def scale(factor: Double): Square = new Square(p, (side*factor).toInt)

	override def toString: String = s"Square(p: $p, side: $side)"
}

object Square {
	val unit: Square = new Square(new Point(0, 0), 1)

	def apply(x: Int, y: Int, side: Int): Square =
		new Square(new Point(x, y), side)

	def apply(): Square = new Square(new Point(0, 0), 1)
}
\end{CodeSmall}



\QUESTEND




%%<AUTOEXTRACTED by mergesolu>%%      % uppgift 9




\WHAT{Case-klassen \code{Point} med 2-tupel.}

\QUESTBEGIN

\Task \label{task:PointTuple} \what~   I ett utvecklingsprojekt vill man ändra representationen av positionen i den gamla klassen  \\ \code{case class Point(x: Int, y: Int)} så att positionen istället i den uppdaterade klassen representeras av en 2-tupel. Man kan då vid konstruktion utnyttja att n-tupler som parameter även kan skrivas som en parameterlista med n argument, varför både \code{Point(1,2)} och \code{Point((1,2))} fungerar fint. Samtidigt vill man att befintlig kod som fortfarande använder \code{x} och \code{y} ska fungera utan ändringar.  Implementera den nya \code{Point} enligt specifikationen nedan.
\begin{ScalaSpec}{Point}
/** A 2-dimensional immutable position p in an integer coordinate system */
case class Point(p:(Int, Int)) {
  /** The x-axis position of this Point */
  val x: Int = ???

  /** The y-axis position of this Point */
  val y: Int = ???

  /** The distance to another Point that */
  def distanceTo(that: Point): Double = ???

  /** The distance to another 2-tuple that representing (x, y). */
  def distanceTo(that: (Int, Int)): Double = ???

  /** A new Point that is moved (dx, dy) */
  def move(dxdy: (Int, Int)): Point = ???
}

object Point {
  /** A Point object at position (0, 0) */
  val origin: Point = ???
}
\end{ScalaSpec}

\SOLUTION


\TaskSolved \what
  \begin{CodeSmall}
case class Point(p:(Int,Int)) {
	val x: Int = p._1

	val y: Int = p._2

	def distanceTo(that: Point): Double = math.hypot(that.x - x, that.y -y)

	def distanceTo(that: (Int, Int)): Double = distanceTo(Point(that))
	def move(dx: Int, dy: Int): Point = Point(x + dx, y + dy)
}

object Point {
	val origin: Point = new Point(0, 0)
}
\end{CodeSmall}



\QUESTEND




%%<AUTOEXTRACTED by mergesolu>%%      % uppgift 10




\WHAT{NEEDS A TOPIC DESCRIPTION}

\QUESTBEGIN

\Task  \what~\Pen Vad behöver du ändra i klassen \code{Square} från uppgift \ref{task:PointSquare} för att den ska fungera med en \code{Point} med 2-tupel från uppgift \ref{task:PointTuple}?

\SOLUTION


\TaskSolved \what
 Inget! Eftersom både \code{Point(1,2)} och \code{Point((1,2))} är okej sätt att komma åt den nya klassen så kommer det se likadant utifrån och därför behöver man inte ändra något i \code{Square}.


\QUESTEND






\WHAT{Objekt med föränderligt tillstånd \Eng{mutable state}.}

\QUESTBEGIN

\Task  \what~  Du ska implementera en modell av en hoppande groda som uppfyller följande krav:
\begin{enumerate}[nolistsep, noitemsep]
\item Varje grodobjekt ska hålla reda på var den är.
\item Varje grodobjekt ska hålla reda på hur långt grodan hoppat totalt.
\item Varje grodobjekt ska kunna beräkna hur långt det är mellan grodans nuvarande position och utgångsläget.
\item Alla grodor börjar sitt hoppande i origo.
\item En groda kan hoppa enligt två metoder:
  \begin{itemize} [nolistsep, noitemsep]
  \item relativ förflyttning enligt parametrarna \code{dx} och \code{dy},
  \item slumpmässig förflyttning $[1, 10]$ i x-led och $[1, 10]$ i y-led.
  \end{itemize}
\end{enumerate}

\Subtask Implementera klassen \code{Frog} enligt nedan specifikation och ovan krav. \\  \emph{Tips:}
  \begin{itemize} [nolistsep, noitemsep]
  \item Om namnet man vill ge ett privat föränderligt attribut ''krockar'' med ett metodnamn, är det vanligt att man börjar attributets namn med understreck, t.ex. \code{private var _x } för att på så sätt undkomma namnkonflikten.
  \item Inför en metod i taget och klistra in den nya grodan i REPL efter varje utvidgning och testa.
  \end{itemize}

\begin{ScalaSpec}{Frog}
class Frog private (initX: Int = 0, initY: Int = 0) {
  def jump(dx: Int, dy: Int): Unit = ???
  def x: Int = ???
  def y: Int = ???
  def randomJump: Unit = ???
  def distanceToStart: Double = ???
  def distanceJumped: Double = ???
  def distanceTo(that: Frog): Double = ???
}
object Frog {
  def spawn(): Frog = ???
}
\end{ScalaSpec}

\Subtask Skriv ett testhuvudprogram som kontrollerar så att alla krav är uppfyllda och att alla metoder fungerar som de ska.

\Subtask\Pen Vad kallas en metod som enbart returnerar värdet av ett privat attribut?

\Subtask\Pen Hur kan man från en metods signatur få en ledtråd om att ett objekt har föränderligt tillstånd \Eng{mutable state}?

\Subtask Inför setters för attributen som håller reda på x- och y-postitionen. Förändringar av positionen i x- eller y-led ska räknas som ett hopp och alltså registreras i det attribut som håller reda på det ackumulerade hoppavståndet.

\Subtask Simulera ett massivt grodhoppande med krockdetektering genom att skapa 100 grodor som till att börja med är placerade på x-axeln med avståndet $8$ längdenheter mellan sig. Låt grodorna i en \code{while}-sats hoppa slumpmässigt tills någon groda befinner sig närmare än $0.5$ längdenheter som är definitionen på att de har krockat. Räkna hur många looprundor som behövs innan något grodpar krockar och skriv ut antalet. \\ \emph{Tips:} Börja med pseudokod på papper. Använd en grodvektor.

\clearpage

\ExtraTasks %%%%%%%%%%%%%%%%%%%

\SOLUTION


\TaskSolved \what


\SubtaskSolved  \begin{CodeSmall}
class Frog private (initX: Int = 0, initY: Int = 0) {
	private var _x: Int = initX
	private var _y: Int = initY
	private var _distanceJumped: Double = 0

	def jump(dx: Int, dy: Int): Unit = {
		_x += dx
		_y += dy
		_distanceJumped += Math.hypot(dx, dy)
	}

	def x: Int = _x
	def y: Int = _y

	def randomJump: Unit = {
		val r = scala.util.Random
		val xtmp = r.nextInt(10)+1
		val ytmp = r.nextInt(10)+1
		_x += xtmp
		_y += ytmp
		_distanceJumped += Math.hypot(xtmp, ytmp)
	}

	def distanceToStart: Double = Math.hypot(_x,_y)
	def distanceJumped: Double = _distanceJumped
	def distanceTo(that: Frog): Double = Math.hypot(_x - that.x, _y - that.y)
}

object Frog {
	def spawn(): Frog = new Frog()
}
\end{CodeSmall}

\SubtaskSolved  \begin{CodeSmall}
val f1 = Frog.spawn()
//test requirement 1 and 4
assert(f1.x == 0 && f1.y == 0, "Either x or y isn't 0")

f1.jump(4,3)
//test requirement 1 and 5
assert(f1.x == 4 && f1.y == 3, "Either x isn't 4 or y isn't 3")

f1.jump(4,3)
//test requirement 2
var text = "distanceJumped is " + f1.distanceJumped + ". Should be 10"
assert(f1.distanceJumped == 10, text)

f1.jump(-4,-3)
//test requirement 3
text = "distanceToStart is " + f1.distanceJumped + ". Should be 5"
assert(f1.distanceToStart == 5, text)

var f2 = Frog.spawn()
for (x <- 1 to 1000) {
	f2.randomJump
	//test requirement 5
	text = "Either x or y isn't in [1,10]. x:" + f2.x + ", y: " + f2.y
	assert(f2.x > 0 && f2.x <= 10 && f2.y > 0 && f2.y <= 10, text)
	f2 = Frog.spawn()
}

val f3 = Frog.spawn()
f3.jump(1,1)
val f4 = Frog.spawn()
f4.jump(4,5)
// Test distanceT()
text = "distanceTo is " + f3.distanceTo(f4) + ". Should be 5"
assert(f3.distanceTo(f4) == 5, text)
\end{CodeSmall}

\SubtaskSolved  Getter

\SubtaskSolved  Om metoden har parametrar och retur-typen \code{Unit}. Det betyder troligen att parametrarna ändrar något istället för att skapa något nytt.

\SubtaskSolved  \begin{CodeSmall}
class Frog private (initX: Int = 0, initY: Int = 0) {
	private var _x: Int = initX
	private var _y: Int = initY
	private var _distanceJumped: Double = 0

	def jump(dx: Int, dy: Int): Unit = {
		_x += dx
		_y += dy
		_distanceJumped += Math.hypot(dx, dy)
	}

	def x: Int = _x
	def y: Int = _y

	def x_= (newX: Int): Unit = {
		_distanceJumped += Math.abs(_x - newX)
		_x = newX
	}
	def y_= (newY: Int): Unit = {
		_distanceJumped += Math.abs(_y - newY)
		_y = newY
	}

	def randomJump: Unit = {
		val r = scala.util.Random
		val xtmp = r.nextInt(10)+1
		val ytmp = r.nextInt(10)+1
		_x += xtmp
		_y += ytmp
		_distanceJumped += Math.hypot(xtmp, ytmp)
	}

	def distanceToStart: Double = Math.hypot(_x,_y)
	def distanceJumped: Double = _distanceJumped
	def distanceTo(that: Frog): Double = Math.hypot(_x - that.x, _y - that.y)
}

object Frog {
	def spawn(): Frog = new Frog()
}
\end{CodeSmall}

\SubtaskSolved  \begin{CodeSmall}
var noCollision = true
var counter = 0
val numberOfFrogs = 100
val distanceBetweenFrogs = 8
val frogArray = Array.fill(numberOfFrogs){Frog.spawn()}
(0 until numberOfFrogs).foreach(i => frogArray(i).x(i*distanceBetweenFrogs))
while (noCollision) {
	frogArray.foreach(frog => frog.randomJump)
	for (frog <- frogArray) {
		for (frog2 <- frogArray) {
			if (frog != frog2 && frog.distanceTo(frog2) < 0.5) {
				noCollision = false
			}
		}
	}
	counter += 1
}
print(counter)
\end{CodeSmall}


\clearpage

\ExtraTasks %%%%%%%%%%%%


\QUESTEND




%%<AUTOEXTRACTED by mergesolu>%%      % uppgift 11




\WHAT{En kvadratklass med föränderligt tillstånd \Eng{mutable state}.}

\QUESTBEGIN

\Task  \what~  Webbshoppen UberSquare säljer flyttbara kvadrater. I affärsmodellen ingår att ta betalt per förflyttning. Du ska hjälpa UberSquare med att utveckla en enkel systemprototyp.

\Subtask Implementera \code{Square} enligt nedan specifikation, under uppfyllandet av följande krav:

\begin{enumerate}[nolistsep, noitemsep]
\item Till skillnad från uppgift \ref{task:Square} ska du nu göra en kvadrat med föränderligt tillstånd \Eng{mutable state}. I stället för att vid förflyttning returnera ett nytt kvadratobjekt, returneras \code{Unit} i samband med att privata attribut uppdateras.
\item Du ska införa funktionalitet som räknar antalet förflyttningar som gjorts för varje kvadrat som skapats och även räkna ut det totala antalet förflyttningar som någonsin gjorts.
\item Varje gång förflyttning sker adderas en kostnad till den ackumulerade kostnaden för respektive kvadrat. Kostnaden för varje förflyttning är avståndet till ursprungsläget multiplicerat med storleken på kvadraten.
\end{enumerate}

\begin{ScalaSpec}{Square}
/** A mutable and expensive Square. */
class Square private (val initX: Int, val initY: Int, val initSide: Int) {

  private var nMoves = 0;
  private var sumCost = 0.0;
  private var _x = initX;
  private var _y = initY;
  private var _side = initSide;

  private def addCost: Unit = {
   sumCost += ???
  }

  /** The current position on the x axis */
  def x: Int = ???

  /** The current position on the y axis */
  def y: Int = ???

  /** The size of this Square */
  def side = ???

  /** Scales the side of this square and rounds it to nearest integer */
  def scale(factor: Double): Unit = ???

  /** Moves this square to position (x + xd, y + dy) */
  def move(dx: Int, dy: Int): Unit = ???

  /** Moves this square to position (x, y) */
  def moveTo(x: Int, y: Int): Unit = ???

  /** The accumulated cost of this Square */
  def cost: Double = ???

  /** Reset the cost of this Square */
  def pay: Unit = ???

  /** A string representation of this Square */
  override def toString: String =
    s"Square[($x, $y), side: $side, #moves: $nMoves times, cost: $sumCost]"
}

object Square {
  private var created = Vector[Square]()

  /** Constructs a new Square object at (x, y) with size side */
  def apply(x: Int, y: Int, side: Int): Square = {
    require(side >= 0, s"side must be positive: $side")
    ???
  }

  /** Constructs a new Square object at (0, 0) with side 1 */
  def apply(): Square = apply(0, 0, 1)

  /** The total number of moves that have been made for all squares. */
  def totalNumberOfMoves: Int = ???

  /** The total cost of all squares. */
  def totalCost: Double = ???
}
\end{ScalaSpec}

\Subtask Testa din kvadratprototyp i REPL enligt nedan:
\begin{REPL}
scala> val xs = Vector.fill(10)(Square())
scala> xs.foreach(_.move(2,3))
scala> xs.foreach(_.scale(2.9))
scala> val (m, c) = (Square.totalNumberOfMoves, Square.totalCost)
m: Int = 10
c: Double = 36.055512754639885
\end{REPL}


\clearpage

\AdvancedTasks %%%%%%%%%%%%%%%%%


\SOLUTION


\TaskSolved \what


\vspace{1em} %tweak pagination

\begin{CodeSmall}
/** A mutable and expensive Square. */
class Square private (val initX: Int, val initY: Int, val initSide: Int) {

  private var nMoves = 0;
  private var sumCost = 0.0;
  private var _x = initX;
  private var _y = initY;
  private var _side = initSide;

  private def addCost: Unit = {
   sumCost += math.hypot(x - initX, y - initY) * side
  }

  /** The current position on the x axis */
  def x: Int = _x

  /** The current position on the y axis */
  def y: Int = _y

  /** The size of the side */
  def side = _side

  /** Scales the size of this square and rounds it to nearest integer */
  def scale(factor: Double): Unit = { _side = (_side * factor).round.toInt }

  /** Moves this square to position (x + xd, y + dy) */
  def move(dx: Int, dy: Int): Unit = {
    _x += dx; _y += dy;
    nMoves += 1
    addCost
  }

  /** Moves this square to position (x, y) */
  def moveTo(x: Int, y: Int): Unit = {
    _x = x; _y = y;
    nMoves += 1
    addCost
  }

  /** The accumulated cost of this Square */
  def cost: Double = sumCost

  /** Reset the cost of this Square */
  def pay: Unit = {sumCost = 0}

  /** A string representation of this Square */
  override def toString: String =
    s"Square[($x, $y), side: $side, #moves: $nMoves times, cost: $sumCost]"
}

object Square {
  private var created = Vector[Square]()

  /** Constructs a new Square object at (x, y) with size side */
  def apply(x: Int, y: Int, side: Int): Square = {
    require(side >= 0, s"side must be positive: $side")
    val sq = (new Square(x, y, side))
    created :+= sq
    sq
  }

  /** Constructs a new Square object at (0, 0) with side 1 */
  def apply(): Square = apply(0, 0, 1)

  /** The total number of moves that have been made for all squares. */
  def totalNumberOfMoves: Int = created.map(_.nMoves).sum

  /** The total cost of all squares. */
  def totalCost: Double = created.map(_.cost).sum
}
\end{CodeSmall}

\AdvancedTasks %%%%%%%%%

\TODO
\QUESTEND






\WHAT{Hjälpkonstruktor.}

\QUESTBEGIN

\Task \label{task:aux-constructor} \what~   I uppgift \ref{task:Square} erbjöds ett alternativt sätt att skapa \code{Square} med en extra fabriksmetod med namnet \code{apply} i kompanjonsobjektet. Ett annat sätt att göras detta på, som i Scala är mindre vanligt (men i Java är desto vanligare), är att definiera flera konstruktorer innuti klassen. I Scala kallas en sådan extra konstruktor för \textbf{hjälpkonstruktor} \Eng{auxiliary constructor}.

En hjälpkonstruktor skapar man i Scala genom att definiera en metod som har det speciella namnet \code{this}, alltså en deklaration \code{def this(...) = ...} Hjälponstruktorer måste börja med att anropa en annan konstruktor, antingen den primära konstruktorn eller en tidigare definierad  hjälpkonstruktor.

\Subtask Läs mer om hjälpkonstruktorer här: \\ \href{http://www.artima.com/pins1ed/functional-objects.html#6.7}{www.artima.com/pins1ed/functional-objects.html\#6.7}

\Subtask Hitta på en egen uppgift med hjälpkonstruktorer, baserat på någon av klasserna i tidigare övningar.


%\Task \TODO \\ \code{class Rational private (numerator: BigInt, denominator: BigInt)} \\
%Inspirerat av Rational i pins1ed med GCD\SOLUTION


\QUESTEND

%!TEX encoding = UTF-8 Unicode

%!TEX root = ../compendium2.tex

\Lab{\LabWeekFIVE}

\begin{Goals}
%!TEX encoding = UTF-8 Unicode

%!TEX root = ../compendium2.tex

%\item Kunna skapa en klass utifrån en textuell beskrivning. % av dess medlemmar.
%\item Kunna skapa en klass utifrån ofärdig kod och dokumentationskommentarer.
%\item Kunna införa privata attribut med lämpliga namn som representerar instansers förändringsbara tillstånd.
\item Kunna förklara skillnader och likheter mellan ett singelobjekt och objekt som är instanser av klasser.
\item Kunna förklara skillnaden mellan förändringsbara och oföränderliga objekt.
\item Kunna definiera och instansiera klasser och case-klasser, samt kunna beskriva när en case-klass är lämpligast och ge några exempel på vad en sådan erbjuder utöver en vanlig klass.
\item Kunna skapa och använda klasser vars instanser innehåller referenser till andra instanser (aggregering).
\item Förstå innebörden av instansreferensen \code{this}.

\end{Goals}

\begin{Preparations}
\item \DoExercise{\ExeWeekFIVE}{05}

\item Studera dokumentationen för klassen \jcode{cslib.window.SimpleWindow} här: \url{http://cs.lth.se/pgk/api/}


\end{Preparations}

\subsection{Bakgrund}

Under den här laborationen ska du skapa en samling av klasser som tillsammans kan användas för att rita i ett fönster. För att ge dig en grund att stå på får du tillgång till den färdigskrivna klassen SimpleWindow. SimpleWindow är implementerad i Java och kan skapa ett enkelt ritfönster på skärmen, med metoder för att rita linjer, etc. SimpleWindow håller koll på en ''penna'' som representerar \textit{aktuell ritposition}. Det finns metoder för att flytta pennan och att rita en rak linje från pennans aktuella ritposition till en ny pennposition.

Med hjälp av SimpleWindow ska du skapa en Turtle-klass, som ska fungera likt sköldpaddan i Kojo, vilken du använde i laborationen i avsnitt \ref{section:lab:kojo}. Delar av dokumentationen för SimpleWindow återspeglas i nedan specifikation. Den fullständiga dokumentationen återfinns här: \url{http://cs.lth.se/pgk/api/}

\vspace{1em}%hack to keep comment with method
\begin{JavaSpec}{class SimpleWindow}
  /** mouse click event type */
	public final static int MOUSE_EVENT = 1;

  /** key pressed event type */
	public final static int KEY_EVENT = 2;

  /** window closed event type */
	public final static int CLOSE_EVENT = 3;

  /** Creates a window and makes it visible. */
	public SimpleWindow(int width, int height, String title);

  /** Returns the width of the window. */
	public int getWidth();

	/** Returns the height of the window. */
	public int getHeight();

	/** Clears the window. */
	public void clear();

	/** Closes the window.*/
	public void close();

	/** Opens the window. */
	public void open();

	/** Moves the pen to a new position. */
	public void moveTo(int x, int y) ;

	/** Moves the pen to a new position while drawing a line. */
	public void lineTo(int x, int y);

	/** Writes a string at the current position.* /
	public void writeText(String txt);

	/** Draws a bitmap image at the current position.*/
	public void drawImage(Image image);

	/** Returns the pen's x coordinate. */
	public int getX();

	/** Returns the pen's y coordinate. */
	public int getY();

	/** Sets the line width.  */
	public void setLineWidth(int width);

	/** Sets the line color. */
	public void setLineColor(Color col);

	/**Returns the current line width. */
	public int getLineWidth();

	/** Returns the current line color. */
	public Color getLineColor();

	/**  Waits for a mouse click. */
	public void waitForMouseClick();

	/** Returns the mouse x coordinate at the last mouse click. */
	public int getMouseX();

	/** Returns the mouse y coordinate at the last mouse click. */
	public int getMouseY();

	/**Adds a sprite to the window. */
	public void addSprite(Sprite sprite);

	/** Wait for a specified time. */
	public static void delay(int ms);


\end{JavaSpec}




\clearpage

\subsection{Obligatoriska uppgifter}

\Task Skapa en klass Point för att beskriva en viss koordinat (x,y) i ett fönster. Klassen ska vara oföränderlig -- man ska alltså inte kunna ändra på en koordinat efter att den har skapats. Notera att klassens attribut är av typen \code{Double} och inte \code{Int}, trots att de beskriver en diskret pixelposition. Anledningen till detta är att det kan uppstå avrundningsfel vid upprepade förflyttningar. Detta blir särskilt märkbart vid många små förflyttningar, som t.ex. när en Turtle används för att rita en cirkel.

\ScalaSpecInputListing{Point}{../workspace/w06_turtlegraphics/src/main/scala/turtlegraphics/Point.scala}

\Subtask Implementera klassen Point.

\Task Skapa klassen Turtle:

\vspace{1em} % hack to fix pagination

\ScalaSpecInputListing{Turtle}{../workspace/w06_turtlegraphics/src/main/scala/turtlegraphics/Turtle.scala}

\Subtask Vilka attribut finns i klassen, och vilken synlighetsnivå har de? Vilken/vilka konstruktorer finns?

\Subtask Är klassen muterbar eller omuterbar? Motivera! Hade man kunnat göra tvärtom?

\Subtask Implementera klassen Turtle enligt specifikationen ovan. När klassen är färdigimplementerad ska den kunna användas för att rita figurer från labbens main-metod i filen Main.scala.

\ScalaSpecInputListing{Main}{../workspace/w06_turtlegraphics/src/main/scala/turtlegraphics/Main.scala}

\Subtask Kör main-metoden ovan för att bekräfta att din implementation fungerar. Bilden ska visa två lika stora rektanglar i samma höjd.

\Subtask Just nu behöver användaren av en Turtle specificera alla detaljer om en Turtles ursprungliga tillstånd som parametervärden för att skapa den. För att underlätta för användaren ska du nu skapa en alternativ konstruktor som kräver färre parametrar. Vilka konstruktorparametrar skulle kunna bytas ut mot rimliga default-värden?

\Subtask Använd din Turtle för att rita en cirkel. För att göra detta kan du t.ex. låta din Turtle gå ett kort steg och svänga någon grad tills den har gjort ett fullt varv.

\Subtask Skapa två stycken Turtles i samma fönsterobjekt som rör sig alternerande. Fungerar allt som tänkt?

\begin{itemize}
\item \textbf{Tips}: SimpleWindow har sitt origo i övre vänstra hörnet (och inte det nedre vänstra hörnet som är vanligt inom matematik).
\end{itemize}

\Task Skapa klassen Rectangle:

\vspace{1em}%hack to keep comment with method
\ScalaSpecInputListing{Rectangle}{../workspace/w06_turtlegraphics/src/main/scala/turtlegraphics/Rectangle.scala}

\Subtask Vilken synlighetsnivå bör konstruktorparametrarna ges? Motivera.

\Subtask I specifikationen står det att rektangeln roteras runt det övre vänstra hörnet, men finns det andra val av rotationsaxlar? Vilka fördelar/nackdelar finns för olika val?
Välj den implementationen du anser lämpligast.

\Subtask Implementera klassen Rectangle enligt specifikationen ovan.

\Subtask Använd din Rectangle för att skapa en animation som utnyttjar skalning, rotation och förflyttning.
Du kan skapa en animering genom att använda dig av SimpleWindow-objektens clear-metod för att rensa skärmen, samt SimpleWindow-klassens delay-metod.
Notera att delay-metoden inte kan anropas på objektet. Se nedanstående exempel.

\begin{Code}
val w = new SimpleWindow(500,500, "Animation")
while(true){
	w.clear()
	// Draw something here
	SimpleWindow.delay(50)
}
\end{Code}


\clearpage

\subsection{Frivilliga extrauppgifter}


\Task Skapa en klass RectangleSequence. I denna klass skall draw-metoden rita ut ett antal rektanglar där varje rektangel har förflyttat sig, roterats och skalats jämfört med föregående rektangel i sekvensen. Se bilder nedan.


\ScalaSpecInputListing{RectangleSequence}{../workspace/w06_turtlegraphics/src/main/scala/turtlegraphics/RectangleSequence.scala}

\Subtask Implementera RectangleSequence.

\Subtask I SimpleWindow kan man ange en färg via metoden setLineColor som ska användas vid utritning. Nyttja detta för att göra en färggladare visualisering av rektangelsekvensen.

\Subtask RectangleSequence är resultatet av flera lager utav abstraktioner. Vilka abstraktionslager ser du? Skulle man kunna abstrahera ytterligare?

\begin{figure}[H]
\centering
\includegraphics[width=0.7\textwidth, height = 0.3\pdfpageheight, keepaspectratio]{../img/w06-lab/RollingRectangle.png}
\caption {Resultatet av: \newline \texttt{RectangleSequence(\newline \mbox{~~Rectangle(Point(200, 200), 50, 30, 0), 5, 0, 70, -30, 0.67}\\) }}
\label{fig:classes:turtlegraphics:rollingrectangle}
\end{figure}

\begin{figure}[H]
\centering
\begin{Code}
val w = new SimpleWindow(500, 500, "Shapes")
val t = new Turtle(w, new Point(200, 200), 0, false)
val rect = Rectangle(Point(225, 235), 50, 30, 0)
val roll = RectangleSequence(rect, 100, 0, 2, 0, 0.98)
for(i <- 0 to 360 by 20) {
  roll.rotateLeft(i).draw(t)
}
\end{Code}
\caption {Kod som ritar bilden som visas i figur \ref{fig:classes:turtlegraphics:rectanglesequence} på sidan \pageref{fig:classes:turtlegraphics:rectanglesequence}.}
\label{fig:classes:turtlegraphics:rectanglesequence:code}
\end{figure}


\begin{figure}[H]
\centering
\includegraphics[width=0.7\textwidth, height = 0.3\pdfpageheight,keepaspectratio]{../img/w06-lab/RectangleSequence.png}
\caption {Resultatet av koden i figur \ref{fig:classes:turtlegraphics:rectanglesequence:code}.}
\label{fig:classes:turtlegraphics:rectanglesequence}

\end{figure}


\Task Studera dokumentationen för de SimpleWindow-metoder som erbjuder hantering av händelser \Eng{event} och använd dessa för lösa deluppgifterna nedan.

\Subtask Gör så att en Turtle kan styras med hjälp av tangentbordstryckningar A--S--D--W för vänster--ner--höger--upp och att den ritar ett spår allteftersom den förflyttas.

\Subtask Gör så att en andra Turtle kan styras med hjälp av tangentbordstryckningar J--K--L--I för vänster--ner--höger--upp och att den också ritar ett spår allteftersom den förflyttas.

\Subtask Gör så att, när de två sköldpaddorna ovan befinner sig tillräckligt nära varandra, det ritas ut en rektangel med hörn där de två sköldpaddorna finns. (Denna uppgift är lite svårare och kan behöva delas upp i delar.)


\Task Studera dokumentationen för de SimpleWindow-metoder som erbjuder hantering av flyttbara bilder \Eng{sprites}. Gör så att en fin Sprite ritas vid positionen för de styrbara sköldpaddorna i föregående uppgift.

\Task En riktig utmaning, för den som har lust: Implementera spelet ''Masken'' som beskrivs här: \url{https://sv.wikipedia.org/wiki/Snake}.


%!TEX encoding = UTF-8 Unicode

%!TEX root = ../compendium1.tex

%!TEX encoding = UTF-8 Unicode
\chapter{Klasser, Likhet}\label{chapter:W06}
Koncept du ska lära dig denna vecka:
\begin{multicols}{2}\begin{itemize}[nosep,label={$\square$},leftmargin=*]
\item objektorientering
\item klass
\item Point
\item Square
\item Complex
\item inkapsling
\item accessregler
\item private
\item private[this]
\item getters och setters
\item new
\item null
\item referensklasser vs värdeklasser
\item klassparameter
\item primär konstruktor
\item alternativ konstruktor
\item referenslikhet vs strukturlikhet
\item eq vs ==
\item compareTo
\item implementera equals\end{itemize}\end{multicols}

\clearpage\section{Teori}
%!TEX encoding = UTF-8 Unicode
%!TEX root = ../lect-w06.tex

\ifkompendium\else

\Subsection{Uppgifter denna vecka}

\begin{Slide}{Denna veckas övning: \texttt{sequences}}
\begin{itemize}\SlideFontTiny
\input{../compendium/modules/w06-sequences-exercise-goals.tex}
\end{itemize}
\end{Slide}

\begin{Slide}{Denna veckas laboration: \texttt{shuffle}}
\begin{itemize}\SlideFontSmall
%!TEX encoding = UTF-8 Unicode
%!TEX root = ../compendium2.tex

\item Kunna skapa och använda sekvenssamlingar.
\item Kunna använda sekvensalgoritmen SHUFFLE för blandning på plats av innehållet i en array.
\item Kunna registrera antalet förekomster av olika värden i en sekvens.

\end{itemize}
\end{Slide}
\fi

\begin{Slide}{Scala Build Tool: \texttt{sbt}}
\begin{itemize}
\item Läs appendix om sbt.
\item Med enkla medel sköter \code{sbt} omkompilering och körning vid varje Ctrl+S med kommandot \code{~run} 
\begin{REPLnonum}
$ sbt
> ~run
\end{REPLnonum}
\item Lägg din kod i biblioteket \code{src/main/scala}
\item Lägg till en enkel styrfil \code{build.sbt}
\begin{Code}
name := "hello"
scalaVersion := "2.12.13" 
\end{Code}
\end{itemize}
\end{Slide}

%!TEX encoding = UTF-8 Unicode
%!TEX root = ../lect-w06.tex

%%%

\Subsection{Matchning}

\ifkompendium
\noindent  I ett match-uttryck kan man matcha på ett visst värde eller på en viss typ och match-uttryck används gärna istället för nästlade if-uttryck, då de ofta är lättare att läsa och begripa. Med match-uttryck kan man också göra \Emph{mönstermatchning} mot case-klass-instanser, t.ex. för att på ett smidigt sätt undersöka om attribut har speciella värden. Match-uttryck i Scala är en mer kraftfull variant av \code{switch}-satser som finns i många andra språk.  
\else
\begin{SlideExtra}{Vad är matchning?}
\includegraphics[width=0.8\textwidth]{../img/plocklada.png}
\end{SlideExtra}
\fi

\begin{Slide}{Vad är matchning?}
Matchning gör man då man vill jämföra ett värde mot andra värden och hitta överensstämmelse \Eng{match}.

\pause

\vspace{1em}\noindent Detta kan man t.ex. göra med nästlade if-else-satser/uttryck:

\begin{Code}
val g = scala.io.StdIn.readLine("grönsak:")
val smak =
  if (g == "gurka") "gott!"
  else if (g == "tomat") "jättegott!"
  else if (g == "broccoli") "ganska gott..."
  else "inte gott :("

println(g + " är " + smak)
\end{Code}
\end{Slide}




\begin{Slide}{Javas switch-sats}\SlideFontSmall
De flesta C-liknande språk (men inte Scala) har en \jcode{switch}-sats som man kan använda istället för (vissa) nästlade if-else-satser:
\javainputlisting[basicstyle=\ttfamily\SlideFontSize{5.5}{6.8}\selectfont]{../compendium/examples/match/Switch.java}

\vspace{-0.5em}\code{switch} fungerar i Java 8 bara för primitiva typer och några till (t.ex. String).
\end{Slide}




\begin{Slide}{Javas switch-sats utan break}\SlideFontSmall
Saknad \jcode{break}-sats ''faller igenom'' till efterföljande gren:

\javainputlisting[basicstyle=\ttfamily\SlideFontSize{6}{7}\selectfont]{../compendium/examples/match/SwitchNoBreak.java}
En glömd \jcode{break} kan ge svårhittad bugg...
\end{Slide}

\begin{Slide}{Javas switch-sats med glömd break}\SlideFontSmall

\vspace{-0.5em}\javainputlisting[basicstyle=\ttfamily\SlideFontSize{5.5}{6.8}\selectfont]{../compendium/examples/match/SwitchForgotBreak.java}

\vspace{-0.7em}\pause
\begin{REPL}
> java SwitchForgotBreak
Skriv grönsak:
gurka
gott!
gott!
\end{REPL}

\end{Slide}


\begin{Slide}{Scalas \texttt{match}-uttryck}\SlideFontSmall
Scala har ingen \code{switch}-sats men erbjuder i stället ett \code{match}-\Emph{uttryck} som är kraftfullare och ger ett värde.

\begin{Code}
val g = scala.io.StdIn.readLine("grönsak:")
val smak = g match {
  case "gurka" => "gott!"
  case "tomat" => "jättegott!"
  case "broccoli" => "ganska gott..."
  case _ => "mindre gott..."
}
println(g + " är " + smak)
\end{Code}
Och den ''faller inte igenom'' som Javas \code{switch}-sats!
\begin{itemize}
\pause\item Varje \code{case}-gren testas var för sig i tur och ordning uppifrån och ned.
\pause\item Det som står mellan \code{case} och \code{=>} kallas ett \Emph{mönster} \Eng{pattern}
\pause\item Sista default-grenen ovan kallas \Emph{wildcard-mönster}: \code{case _ => }
\pause\item Ovan är exempel på matchning mot \Emph{konstant-mönster}, \\ i detta fallet tre stycken strängkonstantmönster.
\pause\item Det finns många andra sätt att skriva mönster.
\end{itemize}
\end{Slide}



\begin{Slide}{Matchning med gard}
Man kan stoppa in en s.k \Emph{gard} \Eng{guard} innan pilen \code{=>} för att villkora matchningen: (notera \code{if}, parenteser behövs ej)
\begin{Code}
val g = scala.io.StdIn.readLine("grönsak:")
def f = g match {
  case "gurka" if math.random > 0.5 => "gott ibland!"
  case "tomat" => "jättegott!"
  case "broccoli" => "ganska gott..."
  case _ => "mindre gott..."
}
\end{Code}
\code{case}-grenen med gard ger bara en lyckad matchning \\ om uttrycket efter \code{if} är sant; annars provas nästa gren, etc.
\end{Slide}

\begin{Slide}{Matchning med variabelmönster}\SlideFontSmall
Om det finns ett namn efter \code{case} som börjar med liten begynnelsebokstav, blir detta namn en variabel som automatiskt binds till uttrycket före \code{match}:

\begin{Code}
val g = scala.io.StdIn.readLine("grönsak:")
def f = g match {
  case "gurka" if math.random > 0.5 => "gott ibland!"
  case "tomat" => "jättegott!"
  case "broccoli" => "ganska gott..."
  case other => "smakar bakvänt: " + other.reverse
}
\end{Code}

Ett enkelt variabelmönster, så som \\ \code{case other => ...} \\ i exemplet ovan, matchar \Emph{allt}! \\\code{other} får alltså värdet av \code{g} om \code{g} \Alert{inte} är \code{"gurka"}, \code{"tomat"}, \code{"broccoli"}.

\end{Slide}


\begin{Slide}{Matchning med eller-mönster}\SlideFontSmall
Om man har samma utfall för olika grenar kan dessa slås ihop och mönstret separeras med vertikalstreck: \code{|}
\begin{Code}
val g = scala.io.StdIn.readLine("grönsak:")
def f = g match {
  case "gurka" => "gott"
  case "tomat" => "gott"
  case "lök"   => "gott"
  case _ => "inte gott"
}
\end{Code}

Mer koncist med eller-mönster:

\begin{Code}
val g = scala.io.StdIn.readLine("grönsak:")
def f = g match {
  case "gurka" | "tomat" | "lök" => "gott"
  case _ => "inte gott"
}
\end{Code}



\end{Slide}





\begin{Slide}{Matchning med typade mönster}\SlideFontSmall
Med en typannotering efter en variabel får man ett \Emph{typat mönster} \Eng{typed pattern}. Om matchningen lyckas blir värdet \Alert{omvandlat} till den specifika typen och binds till variabeln.
\begin{Code}
def f = if (math.random < 0.5) 42 + math.random else "gurka" + math.random
def g = f match {
  case x: Double => x.round.toInt
  case s: String => s.length
}
\end{Code}
Vad har funktionen \code{f} för returtyp? \\ \pause
Matchning mot specifika typer enl. ovan används i idiomatisk Scala hellre än \code{isInstanceOf} men man kan göra motsvarande ovan med if-uttryck:
\begin{Code}
def g2 = {
  val x = f
  if (x.isInstanceOf[Double]) x.asInstanceOf[Double].round.toInt
  else if (x.isInstanceOf[String]) x.asInstanceOf[String].length
}.asInstanceOf[Int]
\end{Code}
\end{Slide}


\begin{Slide}{Konstruktormönster med case-klasser}\SlideFontSmall
En basklass med gemensamma delar och två subtyper:
\begin{Code}
trait Grönsak {
  def vikt: Int
  def ärRutten: Boolean
}
case class Gurka(vikt: Int, ärRutten: Boolean) extends Grönsak
case class Tomat(vikt: Int, ärRutten: Boolean) extends Grönsak
\end{Code}
\pause
Tack vare case-klasserna kan man använda \Emph{konstruktormönster} \Eng{constructor pattern} för att kolla vad som finns \Alert{inuti} en instans:
\begin{Code}
def testa(g: Grönsak): String = g match {
  case Gurka(v, false) => "gott, väger " + v
  case Gurka(_, true)  => "inte gott"
  case Tomat(v, r)     => (if (r) "inte " else "") + "gott, väger " + v
  case _ => "okänd grönsak: " + g
}
\end{Code}

Konstruktormönster ''\Emph{plockar isär}'' det som matchas och binder variabler till de attribut som finns i case-klassens konstruktor.
\end{Slide}


\begin{Slide}{Plocka isär samlingar med djupa mönster}
Man kan plocka isär innehållet i en samling så här:
\begin{Code}
def visa(xs: Vector[Grönsak]): String = xs match {
  case Vector()               => "tom grönsaksvektor"
  case Vector(Gurka(v, true)) => "en rutten gurka som väger " + v
  case Vector(g)              => "exakt en grönsak: " + g
  case Vector(g1, g2)         => s"exakt två grönsaker: $g1, $g2"
  case g +: gs                => s"först en $g och sedan svansen: $gs"
}
\end{Code}
Övning: prova ovan i REPL. Vad händer om du byter ordning på andra och tredje mönstret ovan?
\end{Slide}

\begin{Slide}{Matchning på tupler}
Det går fint att plocka isär tupler med mönstermatchning:\footnote{\url{https://youtu.be/aboZctrHfK8}}
\begin{Code}
val pair = ("hej", 42)

pair match {
  case (a, b) if b == 42 => s"livets mening är funnen: $a"
  case (_, b)            => s"fattas mening: $b"
}
\end{Code}

\end{Slide}

\begin{Slide}{Mönstermatchning och case-objekt}
En bastyp och specifika singelobjekt av gemensam typ:
\begin{Code}
trait Färg
case object Spader  extends Färg
case object Hjärter extends Färg
case object Ruter   extends Färg
case object Klöver  extends Färg

def parallellFärg(f: Färg): Färg = f match {
  case Spader  => Klöver
  case Klöver  => Spader
  case Hjärter => Ruter
}
\end{Code}
Vilken case-gren har vi glömt? Kan kompilatorn hjälpa oss?
\pause
\begin{REPL}
scala> parallellFärg(Ruter)
scala.MatchError: Ruter (of class Ruter)
  at .parallellFärg(<console>:18)

\end{REPL}
\Alert{Undantag vid körtid} \code{:(}
\end{Slide}

\begin{Slide}{Mönstermatchning och förseglade typer}
Med nyckelordet \code{sealed} får vi en kompileringsvarning.
\begin{Code}
sealed trait Färg
case object Spader  extends Färg
case object Hjärter extends Färg
case object Ruter   extends Färg
case object Klöver  extends Färg

def parallellFärg(f: Färg): Färg = f match {
  case Spader  => Klöver
  case Klöver  => Spader
  case Hjärter => Ruter
}
\end{Code}
\begin{REPL}
<console>:23: warning: match may not be exhaustive.
It would fail on the following input: Ruter
       def parallellFärg(f: Färg): Färg = f match {
\end{REPL}
\Emph{Varning vid kompilering} \code{:)}
\end{Slide}

\begin{Slide}{Stora/små begynnelsebokstäver vid matchning}
\Alert{Fallgrop}: matcha \Alert{värde} som börjar med \Alert{liten} bokstav.
\begin{REPL}
scala> val livetsMening = 42

scala> def ärLivetsMeningBuggig(svar: Int) = svar match {
         case livetsMening => true    // lokalt namn som matchar allt!
         case _ => false
       }

scala> ärLivetsMeningBuggig(43)
res0: Boolean = true

scala> val LivetsMening = 42   // stor begynnelsebokstav

scala> def ärLivetsMening(svar: Int) = svar match {
         case LivetsMening => true    // funkar fint!
         case _ => false
       }

scala> ärLivetsMening(43)
res1: Boolean = false
\end{REPL}
\end{Slide}


\begin{Slide}{Stora/små begynnelsebokstäver vid matchning}
Ett sätt att komma runt problemet med liten begynnelsebokstav: \\
\Emph{backticks} to the rescue!
\begin{REPL}
scala> val livetsMening = 42

scala> def ärLivetsMeningBackTicks(svar: Int) = svar match {
         case `livetsMening` => true    // nu funkar det!
         case _ => false
       }

scala> ärLivetsMeningBackTicks(43)
res2: Boolean = false
\end{REPL}
\end{Slide}


\begin{Slide}{Mönster på andra ställen än i \texttt{match}}\SlideFontSmall
Mönster i \Emph{deklarationer}:
\vspace{-0.25em}\begin{REPL}
scala> case class Point(x: Int, y: Int)

scala> val p = Point(0, 1)

scala> val Point(x, y) = p          // konstruktormönster med case-klass
x: ???
y: ???

scala> val (x, y, z) = (0, 1, 2)    // konstruktormönster med tupel
x: ???
y: ???
z: ???

\end{REPL}
Mönster i \Emph{for-satser}:
\vspace{-0.25em}\begin{REPL}
scala> val xs = for ((x, y) <- Vector((1,2), (3,4))) yield x
xs: ???
\end{REPL}

\end{Slide}

\begin{Slide}{Mönster på andra ställen än i \texttt{match}}\SlideFontSmall
Mönster i \Emph{deklarationer}:
\vspace{-0.25em}\begin{REPL}
scala> case class Point(x: Int, y: Int)

scala> val p = Point(0, 1)

scala> val Point(x, y) = p          // konstruktormönster med case-klass
x: Int = 0
y: Int = 1

scala> val (x, y, z) = (0, 1, 2)    // konstruktormönster med tupel
x: Int = 0
y: Int = 1
z: Int = 2

\end{REPL}
Mönster i \Emph{for-satser}:
\vspace{-0.25em}\begin{REPL}
scala> val xs = for ((x, y) <- Vector((1,2), (3,4))) yield x
xs: scala.collection.immutable.Vector[Int] = Vector(1, 3)
\end{REPL}
\end{Slide}

\begin{Slide}{Fördjupning om mönster (ingår ej på tentan)}\SlideFontSmall
\begin{itemize}
\item binda variabler till mönsterdelar med \code{@} \\
\code{Vector(xs@Vector(a), 42)}

\item sekvensmönster med \code{_} och \code{_*} \\ \code{Vector(a, _, b)} och \code{Vector(a, _*)}

\item partiella funktioner: \code|val pf: Int => Double = { case z if z != 0 => 1/z }|

\item Läs mer om mönster här:  \href{http://www.artima.com/pins1ed/case-classes-and-pattern-matching.html}{\SlideFontTiny www.artima.com/pins1ed/case-classes-and-pattern-matching.html}

\item För djupare förståelse av hur \code{case} fungerar, läs speciellt om \Emph{partiella funktioner} här: \href{http://www.artima.com/pins1ed/case-classes-and-pattern-matching.html\#15.7}{\SlideFontTiny www.artima.com/pins1ed/case-classes-and-pattern-matching.html\#15.7}

\item Läs om extractors här: \href{http://www.artima.com/pins1ed/extractors.html}{\SlideFontTiny www.artima.com/pins1ed/extractors.html}

\end{itemize}
\end{Slide}


\begin{Slide}{Fördjupning: metoden \texttt{unapply}}\SlideFontSmall
När du deklarerar en case-klass kommer kompilatorn att \Alert{automatiskt generera en metod} med namnet \Emph{\texttt{unapply}} som kan plocka isär instansen.
\begin{REPL}
scala> case class Gurka(vikt: Int, ärRutten: Boolean)

scala> Gurka.unapply // tryck TAB två gånger för att se metodhuvudet
 case def unapply(x$0: Gurka): Option[(Int, Boolean)]

scala> val g = Gurka(100, false)

scala> Gurka.unapply(g)
res0: Option[(Int, Boolean)] = Some((100,false))
\end{REPL}
Vi ska snart se hur \code{Option} kan hantera värden som \Emph{eventuellt} \Alert{saknas}. \\
\pause

{\SlideFontTiny\vspace{1em}\emph{Fördjupning:} Ett anrop av metoden \code{unapply} genereras av kompilatorn vid matchning och det är det som gör att case-klasser kan användas i konstruktormönster. Man kan skapa en egna s.k. extraktorer \Eng{extractors} som funkar i matchningar (se övn. 22).}
\end{Slide}

%!TEX encoding = UTF-8 Unicode
%!TEX root = ../lect-w06.tex

%%%


\ifkompendium\else
%\Subsection{Kapsla in speciella värden: Option[kanske saknas] och Try[kanske misslyckas]}
\Subsection{Hantera speciella värden med inkapsling}

{
\setbeamertemplate{navigation symbols}{}
\setbeamercolor{background canvas}{bg=black}
\begin{frame}[plain]
    \color{white}{Inkapsling av speciella värden så att krasch kan undvikas}
    \makebox[\linewidth]{\includegraphics[width=\paperwidth]{../img/crystal.jpg}}
\end{frame}
}
\fi

\Subsection{Hantera saknade värden med \texttt{Option}}

\begin{Slide}{Hur hantera saknade värden?}\SlideFontSmall
Olika sätt att hantera saknade värden:
\begin{itemize}
\item Hitta på ett specialvärde: exempel -1 för saknat värde
\item \code{null} om värde saknas (vanligt i Java m.fl. språk, mkt ovanligt i Scala)
\item Använd en samling och låt tom samling representera saknat värde: \\
\code{val sums = Vector(Vector(42),Vector(32),Vector(),Vector(21))}

\item \code{Option[A]} gemensam bastyp för: \\
  \code{None} som representerar \Alert{saknat värde}, och \\ \code{Some[A]} som representerar att \Emph{värde finns}
\end{itemize}
\end{Slide}



\begin{Slide}{En gemensam bastyp för ett värde som kanske saknas}\SlideFontSmall\ifkompendium\footnotesize\fi
\vspace{-0.0em}\begin{center}
\newcommand{\TextBox}[1]{\raisebox{0pt}[1em][0.5em]{#1}}
\tikzstyle{umlclass}=[rectangle, draw=black,  thick, anchor=north, text width=3cm, rectangle split, rectangle split parts = 3]
\begin{tikzpicture}[inner sep=0.5em]
\node [umlclass, rectangle split parts = 2, xshift=0cm, text width=3.5cm] (BaseType)  {
            \textit{\textbf{\centerline{\TextBox{\code{Option[A]}}}}}
            \nodepart[]{second}
            \TextBox{\code{def get: A}}\newline
            \TextBox{\code{def isEmpty: Boolean}}

        };

\node [umlclass, rectangle split parts = 1]  at (-2.5cm,-3.0cm) (SubType1) {
            \textbf{\centerline{\TextBox{\code{Some[A]}}}}
            % \nodepart[]{second} \TextBox{\code{val x: A}}
        };

\node [umlclass, rectangle split parts = 1] at (2.5cm,-3.0cm) (SubType2)  {
            \textbf{\centerline{\TextBox{\code{None}}}}
        };
\draw[umlarrow] (SubType1.north) -- ++(0,0.5) -| (BaseType.south);
\draw[umlarrow] (SubType2.north) -- ++(0,0.5) -| (BaseType.south);
\end{tikzpicture}
\end{center}
\pause
\vspace{-0.5em}\begin{REPL}
scala> var x: Option[Int] = Some(42)

scala> x.isEmpty
val res0: Boolean = false

scala> x = None

scala> x.isEmpty
val res1: Boolean = true
\end{REPL}
\end{Slide}


\begin{Slide}{Option för hantering av ev. saknade värden}\SlideFontSmall
Alla vill inte berätta för Facebook vad de har för kön. \\ Förbättra Facebooks kod med ett litet Scala-program:
\begin{Code}
enum Gender:
  case Man, Woman

case class Person(name: String, gender: Option[Gender])
\end{Code}
\pause
\begin{REPL}
scala> val p1 = Person("Björn",  Some(Gender.Man))
scala> val p2 = Person("Sandra", Some(Gender.Woman))
scala> val p3 = Person("Kim",  None)
scala> val g2 = p2.gender
scala> def show(g: Option[Gender]): String = g match {
         case Some(x) => x.toString
         case None    => "unknown"
       }
scala> show(g2)
scala> show(p3.gender)
scala> val ps = Vector(p1,p2,p3)
scala> ps.map(_.gender).map(show)   // None ignoreras av map
\end{REPL}
\end{Slide}

\begin{Slide}{Några smidiga metoder på \code{Option}}\SlideFontSmall
Metoden \code{getOrElse} gör att man ofta kan undvika matchning.
\begin{Code}
var opt: Option[Int] = None

val x = opt.getOrElse(42)   // get the value, give default if missing
\end{Code}

Flera av de vanliga samlingsmetoderna funkar, t.ex. \code{foreach} och \code{map}.
\begin{Code}
opt.foreach(x => println(x)) // only done if value exists

opt.map(x => x + 1)          // only done if value exists

opt = Some(42)               // change opt to now have some value

opt.foreach(x => println(x)) // done as value now exists

opt.map(x => x + 1)          // done as value now exists

\end{Code}
\end{Slide}


\begin{Slide}{Några samlingsmetoder som ger en \code{Option}, övning}
\begin{REPL}
scala> val (xs, ys) = (Vector(1,2,3), Vector())

scala> xs.headOption
res0: ???

scala> ys.headOption
res1: ???

scala> xs.find(_ > 1)
res2: ???

scala> xs.find(_ > 5)
res3: ???

scala> val huvudstad = Map("Sverige" -> "Sthlm", "Skåne" -> "Malmö")

scala> huvudstad.get("Skåne")
res4: ???

scala> huvudstad.get("Danmark")
res5: ???
\end{REPL}
\end{Slide}

\begin{Slide}{Några samlingsmetoder som ger en \code{Option}, svar}
\begin{REPL}
scala> val (xs, ys) = (Vector(1,2,3), Vector())

scala> xs.headOption
res0: Option[Int] = Some(1)

scala> ys.headOption
res1: Option[Nothing] = None

scala> xs.find(_ > 1)
res2: Option[Int] = Some(2)

scala> xs.find(_ > 5)
res3: Option[Int] = None

scala> val huvudstad = Map("Sverige" -> "Sthlm", "Skåne" -> "Malmö")

scala> huvudstad.get("Skåne")
res4: Option[String] = Some(Malmö)

scala> huvudstad.get("Danmark")
res5: Option[String] = None
\end{REPL}
\end{Slide}

%!TEX encoding = UTF-8 Unicode
%!TEX root = ../lect-w06.tex

%%%



\Subsection{Undantag}

\ifkompendium\else
\begin{SlideExtra}{Undantag kan orsaka krasch...}
\includegraphics[width=1.0\textwidth]{../img/dynamite}  
\end{SlideExtra}

\begin{SlideExtra}{Undantag orsakar ingen krasch om inkapslad i en Try}
\hspace{0.3\textwidth}\includegraphics[width=0.6\textwidth,angle=-90,origin=c]{../img/bomb-shelter}  
\end{SlideExtra}
\fi

\begin{Slide}{Vad är ett undantag \Eng{exception}?}
Undantag representerar ett fel eller ett onormalt tillstånd som upptäcks under exekvering och som  behöver hanteras på särskilt sätt vid sidan av det normala exekveringsflödet.

\vspace{1em}\href{https://sv.wikipedia.org/wiki/Undantagshantering}{sv.wikipedia.org/wiki/Undantagshantering}


\vspace{1em} Exempel på undantag:

\pause

\begin{itemize} \SlideFontSmall
\item Indexering utanför vektorns indexgränser.

\item Läsning bortom filens slut.

\item Försök att öppna en fil som inte finns.

\item Minnet är slut.

\item Heltalsdivision med noll ger \code{java.lang.ArithmeticException}.

\item \code{"hej".toInt} ger \code{java.lang.NumberFormatException}

\end{itemize}

\end{Slide}


\begin{Slide}{Orsaka undantag indirekt med \texttt{require} och \texttt{assert}}

\begin{itemize}\SlideFontSmall
  \item Med funktionen \code{require(b)} skapas ett \\\code{IllegalArgumentException("requirement failed")} \\ om \code{b} är \code{false}
  \item \code{require} används om man vill begränsa vilka argument som är giltiga
  \item Med funktionen \code{assert(b)} skapas ett \code{AssertionError("assertion failed")} \\ om \code{b} är \code{false} 
  \item \code{assert} används om man vill förhindra ogiltiga tillstånd
\end{itemize}
{
  \ifkompendium\else
  \vfill\SlideFontTiny
  \fi
  Se implementationen av \code{require} här:\\
\url{https://github.com/scala/scala/blob/v2.13.6/src/library/scala/Predef.scala#L315}
}
\end{Slide}

\begin{Slide}{Kasta dina egna undantag med \texttt{throw}}\SlideFontSmall
Man kan själv generera ett undantag med \code{throw}, vilket kallas att \Emph{kasta} ett undantag som (om det inte \Emph{fångas}), gör att exekveringen \Alert{avbryts}.


\begin{REPL}
scala> def pang = throw Exception("PANG!")
pang: Nothing

scala> pang
java.lang.Exception: PANG!

\end{REPL}
\pause
Olika sätt att hantera undantag och förhindra att exekveringen avbryts:
\begin{itemize}
\item \code{try catch}-uttryck omvandlar undantag till ngt lämpligt värde.
%\item Java: Man kan använda en \code{try ... catch}-sats och \Alert{göra något} i händelse av undantag.

\item \texttt{scala.util.Try} \Emph{kapslar in} kod som kan ge undantag.  %(Finns ej i Java; att föredra i Scala.)
\end{itemize}
\end{Slide}


\Subsection{Hantera undantag med \texttt{Try}}

\begin{Slide}{En gemensam bastyp för något som kan misslyckas}\SlideFontSmall
\begin{Code}
import scala.util.{Try, Success, Failure}
\end{Code}
\ifkompendium\footnotesize\fi
\vspace{-0.5em}\begin{center}
\newcommand{\TextBox}[1]{\raisebox{0pt}[1em][0.5em]{#1}}
\tikzstyle{umlclass}=[rectangle, draw=black,  thick, anchor=north, text width=3.0cm, rectangle split, rectangle split parts = 3]
\begin{tikzpicture}[inner sep=0.5em]
\node [umlclass, rectangle split parts = 2, xshift=0cm, text width=3.8cm] (BaseType)  {
            \textit{\textbf{\centerline{\TextBox{\code{Try[T]}}}}}
            \nodepart[]{second}
            \TextBox{\code{def get: T}}\newline
            \TextBox{\code{def isFailure: Boolean}}\newline
            \TextBox{\code{def isSuccess: Boolean}}
        };

\node [umlclass, rectangle split parts = 2, text width=2.2cm]  at (-2.5cm,-3.7cm) (SubType1) {
            \textbf{\centerline{\TextBox{\code{Success[T]}}}}
            \nodepart[]{second} \TextBox{\code{val value: T}}
        };

\node [umlclass, rectangle split parts = 2, text width=4.2cm] at (2.5cm,-3.7cm) (SubType2)  {
            \textbf{\centerline{\TextBox{\code{Failure[T]}}}}
            \nodepart[]{second} \TextBox{\code{val exception: Throwable}}
        };
\draw[umlarrow] (SubType1.north) -- ++(0,0.5) -| (BaseType.south);
\draw[umlarrow] (SubType2.north) -- ++(0,0.5) -| (BaseType.south);
\end{tikzpicture}
\end{center}
\end{Slide}

\begin{Slide}{Hantera undantag med \texttt{Try}}
\vspace{-0.5em}\begin{REPLsmall}
scala> def pang = throw new Exception("PANG!")

scala> def kanskePang = if math.random() < 0.5 then 42 else pang

scala> import scala.util.{Try, Success, Failure}

scala> def försök = Try { kanskePang }

scala> val xs = Vector.fill(15){försök}

scala> val trettonde = xs(12) match
         case Success(value) => value
         case Failure(e) => println(e); -1

scala> (xs(12).isSuccess, xs(12).isFailure) 

scala> xs(12).getOrElse(0)

scala> xs(12).toOption

scala> försök.foreach(println)

scala> försök.map(_ + 1)

scala> for Success(x) <- xs yield x
\end{REPLsmall}
\end{Slide}

\Subsection{Hantera undantag med \texttt{try}-\texttt{catch}}


\begin{Slide}{\texttt{try}-\texttt{catch}-uttryck}\SlideFontSmall
Man kan fånga undantag med ett \code{try ... catch}-uttryck:
\begin{Code}
def carola = 
  try 
    if math.random() > 0.5 then throw Exception("stormvind")
    42
  catch 
    case e: Exception =>
      println("Fångad av en " + e.getMessage)
      -1

\end{Code}
\pause
\begin{REPL}
scala> Vector.fill(5)(carola)
Fångad av en stormvind
Fångad av en stormvind
Fångad av en stormvind
val res0: Vector[Int] = Vector(-1, 42, 42, -1, -1)
\end{REPL}
%Gör uppg. 9-11 i övn. \code{patterns} som visar hur man fångar undantag i Scala och Java. 
%Mer om undantag i fortsättningskursen.
\end{Slide}

\Subsection{För- och nackdelar med undantag}

\begin{Slide}{Unvik undantag om det går}
\SlideFontSmall
\Emph{Fördelar} med undantag: 
\begin{itemize}
\item Vid allvarliga fel då det inte är mycket att göra än att starta om, t.ex. \code{OutOfMemoryException}, är det bra att få veta vad som är fel.
\item Onormala fall som uppkommer sällan kan hanteras separat (t.ex. i huvudprogrammet) utan att koden för normalfallet blir tillkrånglad. 
\end{itemize}
\Alert{Nackdelar} med undantag: 
\begin{itemize}
\item Ett slags ''goto'' som gör exekveringsflödet svårt att följa.
\item Skapa stack-trace tar tid; undantag som sker ofta påverkar prestanda.
\end{itemize}
\pause Exempel: undantagslösa \code{toIntOption} är både säker och snabb!
\begin{REPLsmall}
scala> def time(op: => Unit): Long = {val t0 = System.nanoTime; op; System.nanoTime - t0}

scala> def min(op: => Unit, n: Int = 1000): Long = Seq.fill(n)(time(op)).drop(n / 20).min

scala> min(util.Try("hello".toInt))
val res0: Long = 3549

scala> min(try "hello".toInt catch (_: Throwable) => ())
val res1: Long = 3046

scala> min("hello".toIntOption)
val res2: Long = 157
\end{REPLsmall}
\end{Slide}

\begin{Slide}{Fördjupning: Kontrollerade undantag}
\begin{itemize}
\item Det finns möjligheter i Scala att låta kompilatorn kontrollera om undantag hanteras.
\item Läs mer här: \\ \url{https://docs.scala-lang.org/scala3/reference/experimental/canthrow.html}
\end{itemize}
  
\end{Slide}
%!TEX encoding = UTF-8 Unicode
%!TEX root = ../lect-w06.tex

%%%

\Subsection{Fördjupning: Implementera \texttt{equals}}

\ifkompendium
\noindent När du jämför värden med \code{==} anropas metoden \code{equals} som finns för alla typer. Du kan i dina egna klasser överskugga \code{equals} med en din egna definition av vad likhet ska innebära. Då är det lämpligt att använda matchning. Det är dock ett ganska omfattande arbete att implementera en korrekt likhetsjämförelse som fungerar under alla omständigheter. Ett recept för en fullständig implementation av \code{equals} ges i fördjupningen nedan. 
\fi

\begin{Slide}{Fördjupning: Implementera \texttt{equals} med \texttt{match}}
Det visar sig att \Emph{innehållslikhet} är \Alert{förvånansvärt komplicerat} att implementera, speciellt  i samband med arv.
\begin{itemize}\SlideFontSmall
\item Det enklare fallet: Gör fördjupningsuppgift \textit{''Metoden \code{equals}''} och implementera \code{equals} för innehållslikhet utan arv. \\ En bra träning på att använda \code{match}!

\item Svårare: Gör fördjupningsuppgifterna  \textit{''Överskugga \code{equals}''} och \textit{''Överskugga equals vid arv''} om du vill se hur en \Emph{komplett} \code{equals} ska se ut som fungerar \Alert{i alla lägen}.

\end{itemize}

\noindent Det krävs i denna kurs inte att du själv ska kunna implementera en generellt fungerande \code{equals}. Men du ska förstå skillnaden mellan referenslikhet och innehållslikhet. Mer om \code{equals} i fortsättningkursen, men en liten inblick i problemet nu...
\end{Slide}

\ifkompendium
\noindent Om en klass markeras \code{final} kan den ej ha några subklasser. Kompilatorn kontrollerar att detta gäller alla finala klasser och ger kompileringsfel om du försöker göra \code{extends} på en final klass. Om en klass garanterat inte har några subklasser kan implementationen av \code{equals} göra enklare.
\fi 

\begin{Slide}{Fördjupning: \texttt{equals} som fungerar för finala klasser}
Recept för implementation av \code{equals} som fungerar för typer som \Alert{inte} har några subtyper:
\begin{Code}
final class Gurka(val vikt: Int, val ärÄtbar: Boolean):
  override def equals(other: Any): Boolean = other match
    case that: Gurka => vikt == that.vikt && ärÄtbar == that.ärÄtbar
    case _ => false

  override def hashCode: Int = (vikt, ärÄtbar).## // ger bra hashcode
\end{Code}
\begin{itemize}\SlideFontSmall
\item
Du \Alert{måste} alltid överskugga \code{hashCode} också om du överskuggar \code{equals} annars funkar inte gurksamlingar (lång story ...)
\item
Notera typen \code{Any} -- detta följer hur man valde att göra i Java (tyvärr?).
\pause
\item
Ett \Alert{typsäkrare} innehållslikhetstest som \Emph{garanterat} bara jämför en gurka med en gurka och inget annat:
\begin{Code}
def ===(other: Gurka): Boolean =
  vikt == other.vikt && ärÄtbar == other.ärÄtbar
\end{Code}
\end{itemize}
\end{Slide}


\begin{Slide}{Fördjupning: Recept i 8 steg för arvssäker \code{equals}}\SlideFontTiny
%fungerar även för klasser som inte är \code{final}:
\SlideOnly{\setlength{\leftmargini}{0pt}}
\begin{enumerate}\SlideFontTiny
\item Inför denna metod: \code{ def canEqual(other: Any): Boolean}\\Observera att typen på parametern ska vara \code{Any}. Om subklass behövs \code{override}.

\item Metoden \code{canEqual} ska ge \code{true} om \code{other} är av samma typ som \code{this}, t.ex.: \\\code{override def canEqual(other: Any): Boolean = other.isInstanceOf[Gurka]}

\item Inför metoden \code{equals} och var noga med att parametern har typen \code{Any}: \\ \code{override def equals(other: Any): Boolean}

\item Implementera metoden \code{equals} med ett match-uttryck som börjar så här: \\
\code|other match { ... } |

\item Match-uttrycket ska ha två grenar. Den första grenen ska ha ett typat mönster för den klass som ska jämföras, t.ex.: \\ \code{  case that: Gurka =>}

\item Om du implementerar \code{equals} i den klass som inför \code{canEqual}, börja med: \\ \code{(that canEqual this) &&} \\
och skapa därefter en fortsättning som baseras på innehållet i klassen, t.ex.: \\ \code{this.vikt == that.vikt && this.längd == that.längd} \\
Om du överskuggar equals vill du nog börja med
 \code{super.equals(that) && }

\item Den andra grenen i matchningen ska vara:
\code{case _ => false}

\item Överskugga \code{hashCode}, t.ex. med tupel av attributvärden och metoden \code{##}: \\
\code{override def hashCode: Int  = (vikt, längd).## }

\end{enumerate}
\url{http://www.artima.com/pins1ed/object-equality.html}

\end{Slide}


\begin{Slide}{Fördjupning: Säkrare likhetstest i Scala 3}
\SlideFontSmall
\begin{itemize}
\item \Alert{Problem}: \code{equals} tar värden av vilken typ som helst.
\item Detta kallas \Alert{universell likhet}.
\item[]
\begin{REPLsmall}
scala> case class Hund(namn: String)
scala> case class Katt(namn: String)
scala> Hund("bob") == Katt("bob") // knasig jämförelse; kan aldrig bli sant
val res0: Boolean = false         // men kompilatorn låter dig göra likhetstestet
\end{REPLsmall}  
\pause
\item I Scala 3 kan du få typsäker likhetstest med~~\code{derives CanEqual}
\item Detta kalla \Emph{multiversell likhet}.
\item[]
\begin{REPLsmall}
scala> case class Hund(namn: String) derives CanEqual
scala> Hund("bob") == Katt("bob")   // tack kompilatorn för fel:
-- Error:
1 |Hund("bob") == Katt("bob")
  |^^^^^^^^^^^^^^^^^^^^^^^^^^
  |Values of types Hund and Katt cannot be compared with == or !=
\end{REPLsmall}  
\item Du \Emph{slipper} skriva \code{derives CanEqual} om du gör: \\ \code{import scala.language.strictEquality}
\item Läs mer här: \url{https://docs.scala-lang.org/scala3/reference/contextual/multiversal-equality.html}

\end{itemize}

\end{Slide}


%%!TEX encoding = UTF-8 Unicode
\chapter{Klasser, Likhet}\label{chapter:W06}
Koncept du ska lära dig denna vecka:
\begin{multicols}{2}\begin{itemize}[nosep,label={$\square$},leftmargin=*]
\item objektorientering
\item klass
\item Point
\item Square
\item Complex
\item inkapsling
\item accessregler
\item private
\item private[this]
\item getters och setters
\item new
\item null
\item referensklasser vs värdeklasser
\item klassparameter
\item primär konstruktor
\item alternativ konstruktor
\item referenslikhet vs strukturlikhet
\item eq vs ==
\item compareTo
\item implementera equals\end{itemize}\end{multicols}


%!TEX encoding = UTF-8 Unicode
%!TEX root = ../exercises.tex

\ifPreSolution



\Exercise{\ExeWeekSIX}\label{exe:W06}

\begin{Goals}
\item Kunna skapa och använda \code{match}-uttryck med konstanta värden, garder och mönstermatchning med case-klasser.
\item Kunna skapa och använda case-objekt för matchningar på uppräknade värden.
\item Kunna hantera saknade värden med hjälp av typen \code{Option} och mönstermatchning på \code{Some} och \code{None}.
\item Kunna fånga undantag med \code{scala.util.Try}.
\item Känna till \code{try}, \code{catch} och \code{throw}.
%\item Känna till \jcode{switch}-satser i Java.
\item Känna till nyckelordet \code{sealed} och förstå nyttan med förseglade typer.
%\item Känna till relationen mellan \code{hashCode} och \code{equals}.
%\item Kunna skapa partiella funktioner med case-uttryck.
%\item Känna till betydelsen av små och stora begynnelsebokstäver i case-grenar i en matchning, samt förstå hur namn binds till värden in en case-gren.
%\item Kunna använda \code{flatMap} tillsammans med \code{Option} och \code{Try}.
%\item Känna till skillnaderna mellan \code{try}-\code{catch} i Scala och java.
%\item Känna till att metoden \code{unapply} används vid mönstermatchning.
%\item Kunna implementera \code{equals} med hjälp av en \code{match}-sats, som fungerar för finala klasser utan arv.
%\item Känna till \code{null}.
\end{Goals}

\begin{Preparations}
\item \StudyTheory{06}
\end{Preparations}

\BasicTasks %%%%%%%%%%%%%%%%

\else



\ExerciseSolution{\ExeWeekSIX}

\BasicTasks %%%%%%%%%%%

\fi




\WHAT{Matcha på konstanta värden.}

\QUESTBEGIN

\Task \label{task:vegomatch} \what~   % I Scala finns ingen \jcode{switch}-sats. I stället har Scala ett \code{match}-uttryck som är mer kraftfullt. Dock saknar Scala nyckelordet \jcode{break} och Scalas \code{match}-uttryck kan inte ''falla igenom'' som skedde i uppgift \ref{task:switch}\ref{subtask:break}.

\Subtask \label{subtask:vegomatch} Skriv nedan program med en kodeditor och spara i filen \texttt{Match.scala}. Kompilera och kör och och ge som argument din favoritgrönsak. Vad händer? Förklara hur ett \code{match}-uttryck fungerar.

\scalainputlisting[numbers=left,basicstyle=\ttfamily\fontsize{11}{12}\selectfont]{examples/Match.scala}

\Subtask Vad blir det för felmeddelande om du tar bort case-grenen för defaultvärden och indata väljs så att inga case-grenar matchar? Är det ett exekveringsfel eller ett kompileringsfel?

% \Subtask Beskriv några skillnader i syntax och semantik mellan Javas flervalssats \jcode{switch} och Scalas flervalsuttryck \code{match}.



\SOLUTION


\TaskSolved \what


\SubtaskSolved  %Svaret blir identiskt mot föregående uppgiften i Java.\\
Scalas \code{match}-uttryck jämför stegvis värdet med varje \code{case} för att sedan returnera ett värde tillhörande motsvarande \code{case}.

\SubtaskSolved  \begin{REPL}
scala.MatchError 
\end{REPL}
Exekveringsfel, uppstår av en viss input under körningen.

% \SubtaskSolved  Scalas \code{match} ersätter kolonet (:) i \jcode{switch} med Scalas högerpil (=>).\\
% \code{match} returnerar ett värde till skillnad från \jcode{switch} som inte returnerar något.\\
% \code{match} kan inte $"$falla igenom$"$ så ett \jcode{break} efter varje \jcode{case} är inte nödvändigt.\\
% Till skillnad från \jcode{switch}-satsen kastar \code{match} ett \code{MatchError} om ingen matchning skulle ske.



\QUESTEND






\WHAT{Gard i case-grenar.}

\QUESTBEGIN

\Task  \what~  Med hjälp en gard \Eng{guard} i en case-gren kan man begränsa med ett villkor om grenen ska väljas.

Utgå från koden i uppgift \ref{task:vegomatch}\ref{subtask:vegomatch} och byt ut case-grenen för \code{'g'}-matchning till nedan variant med en gard med nyckelordet \code{if} (notera att det inte behövs parenteser runt villkoret):
\begin{Code}
    case 'g' if math.random() > 0.5 => "gurka är gott ibland..."
\end{Code}
Kompilera om och kör programmet upprepade gånger med olika indata tills alla grenar i \code{match}-uttrycket har exekverats. Förklara vad som händer.

\SOLUTION


\TaskSolved \what

Garden som införts vid \code{case 'g'} slumpar fram ett tal mellan 0 och 1 och om talet inte är större än $0.5$ så blir det ingen matchning med \code{case 'g'} och programmet testar vidare tills default-caset.\\
Gardens krav måste uppfyllas för att det ska matcha som vanligt.



\QUESTEND






\WHAT{Mönstermatcha på attributen i case-klasser.}

\QUESTBEGIN

%\Task \label{task:match-caseclass} \what~   Scalas \code{match}-uttryck är extra kraftfulla om de används tillsammans med \code{case}-klasser: då kan attribut extraheras automatiskt och bindas till lokala variabler direkt i case-grenen som nedan exempel visar (notera att \code{v} och \code{rutten} inte behöver deklareras explicit). Detta kallas för \textbf{mönstermatchning}.

\Task \label{task:match-caseclass} \what~   Scalas \code{match}-uttryck är extra kraftfulla om de används tillsammans med \code{case}-klasser: då kan attribut extraheras automatiskt och bindas till lokala variabler direkt i case-grenen som nedan exempel visar (notera att \code{v} och \code{rutten} inte behöver deklareras explicit). Detta kallas för \textbf{mönstermatchning}. 
Vad skrivs ut nedan? Varför? Prova att byta namn på \code{v} och \code{rutten}.
%\Subtask \label{subtask:autobinding-match} Vad skrivs ut nedan? Varför? Prova att byta namn på \code{v} och \code{rutten}.
\begin{REPL}
scala> case class Gurka(vikt: Int, ärRutten: Boolean)
scala> val g = Gurka(100, true)
scala> g match { case Gurka(v,rutten) => println("G" + v + rutten) }
\end{REPL}

%\TODO %Tab två gånger fungerar inte i scala3-repl, issue #536
%\Subtask Skriv sedan nedan i REPL och tryck TAB två gånger efter punkten. Vad har \code{unapply}-metoden för resultattyp?
%\begin{REPL}
%scala> Gurka.unapply   // Tryck TAB två gånger
%\end{REPL}
%\begin{Background}
%Case-klasser får av kompilatorn automatiskt ett kompanjonsobjekt \Eng{companion object}, i detta fallet \code{object Gurka}. Det objektet får av kompilatorn automatiskt en \code{unapply}-metod. Det är \code{unapply} som anropas ''under huven'' när case-klassernas attribut extraheras vid mönstermatchning, men detta sker alltså automatiskt och man behöver inte explicit nyttja \code{unapply} om man inte själv vill implementera s.k. extraherare \Eng{extractors}; om du är nyfiken på detta, se fördjupningsuppgift \ref{task:extractor}.
%\end{Background}

%\Subtask Anropa \code{unapply}-metoden enligt nedan. Vad blir resultatet?
%\begin{REPL}
%scala> Gurka.unapply(g)
%\end{REPL}
%Vi ska i senare uppgifter undersöka hur typerna \code{Option} och \code{Some} fungerar och hur man kan ha nytta av dessa i andra sammanhang.

% \Subtask Spara programmet nedan i filen \texttt{vegomatch.scala} och kompilera och kör med \code{scala vegomatch.Main 1000} i terminalen. Förklara hur predikatet \code{ärÄtvärd} fungerar.
% \scalainputlisting[numbers=left,basicstyle=\ttfamily\fontsize{11}{12}\selectfont]{examples/vegomatch.scala}
%

\SOLUTION


\TaskSolved \what \\
G100true. Vid byte av plats: Gtrue100.\\
\code{match} testar om kompanjonsobjektet \code{Gurka} är av typen \code{Gurka} med två parametervärden. De angivna parametrarna tilldelas namn, \code{vikt} får namnet \code{v} och \code{ärRutten} namnet \code{rutten} och skrivs sedan ut. Byts namnen dessa ges skrivs de ut i den omvända ordningen.

%\TODO % TAB+TAB fungerar inte i scala3-repl så svaret till uppgiften är felaktig
%\SubtaskSolved  \code{Option[(Int, Boolean)]}

%\SubtaskSolved	\code{Gurka(100, true)}

% \SubtaskSolved  \code{ärÄtvärd} testar om \code{Grönsak g} är av typen \code{Gurka(v, rutten)} eller \code{Tomat}. Dessa har sedan garder.\\ \code{Gurka} måste ha \code{vikt} över 100 och \code{ärRutten} vara \code{false} för att \code{case Gurka} ska returnera \code{true}.\\
% \code{Tomat} måste ha \code{vikt} över 50 och \code{ärRutten} vara \code{false} för att \code{case Tomat} ska returnera \code{true}.\\
% Matchas inte \code{Grönsak g} med någon av dessa returneras default-värdet \code{false}.



\QUESTEND







\WHAT{Matcha på case-objekt och nyttan med \code{sealed}.}

\QUESTBEGIN

\Task	\label{task:match-sealedtrait} \what~	Skriv nedan kodrader i en REPL en för en. Notera nyckelordet \code{sealed} som används för att försegla en typ. En \textbf{förseglad typ} måste ha alla sina subtyper i en och samma kodfil.
\begin{REPL}
scala> sealed trait Färg
scala> case object Spader extends Färg
\end{REPL}
\Subtask Hur lyder felmeddelandet och varför sker det? Är det ett kompileringsfel eller ett körtidsfel?

\Subtask  \label{subtask:match-sealedtrait-caseobject}
Skapa nu nedan kod i en editor och klistra in i REPL.
\begin{Code}
object Kortlek:
  sealed trait Färg
  object Färg:
      val values = Vector(Spader, Hjärter, Ruter, Klöver)
  case object Spader extends Färg
  case object Hjärter extends Färg
  case object Ruter extends Färg
  case object Klöver extends Färg
\end{Code}

\Subtask \label{subtask:match-sealedtrait-function}
Skapa en funktion \code{def parafärg(f: Färg): Färg} i en editor, som med hjälp av ett match-uttryck returnerar parallellfärgen till en färg. Parallellfärgen till \code{Hjärter} är \code{Ruter} och vice versa, medan parallellfärgen till \code{Klöver} är \code{Spader} och vice versa. Klistra in funktionen i REPL. Passa även på att skriva en \code{import}-sats för det yttre objektet \textbf{Kortlek}, så medlemmarna av objektet kan nås enkelt.
\begin{REPL}
scala> parafärg(Spader)
scala> val xs = Vector.fill(5)(Färg.values((math.random() * 4).toInt))
scala> xs.map(parafärg)
\end{REPL}

\Subtask \label{subtask:match-forgetcase}
Vi ska nu undersöka vad som händer om man glömmer en av case-grenarna i matchningen i \code{parafärg}. ''Glöm'' alltså avsiktligt en av case-grenarna och klistra in den nya \code{parafärg} med den ofullständiga matchningen. Hur lyder varningen? Kommer varningen vid körtid eller vid kompilering?

\Subtask Anropa \code{parafärg} med den ''glömda'' färgen. Hur lyder felmeddelandet? Är det ett kompileringsfel eller ett körtidsfel?

\Subtask Förklara vad nyckelordet \code{sealed} innebär och vilken nytta man kan ha av att \textbf{försegla} en supertyp.


\SOLUTION


\TaskSolved \what

\SubtaskSolved
\begin{REPL}
Cannot extend sealed trait Färg in a different source file
\end{REPL}
Felmeddelandet fås av att REPL:en behandlar varje inmatning individuellt och tillåter därför inte att subtypen \code{Spader} ärver från \Eng{extends} supertypen \code{Färg} eftersom denna var förseglad \Eng{sealed}. Mer om detta senare i kursen...

\SubtaskSolved
-

\SubtaskSolved
Förusatt att \code{import Kortlek._} har skrivits...
\begin{Code}
def parafärg(f: Färg): Färg = f match
  case Spader  => Klöver
  case Hjärter => Ruter
  case Ruter   => Hjärter
  case Klöver  => Spader
\end{Code}

\SubtaskSolved
\begin{REPL}
<console>:17: warning: match may not be exhaustive.
It would fail on the following input: Ruter
\end{REPL}
Varningen kommer redan vid kompilering.

\SubtaskSolved
\begin{REPL}
scala.MatchError: Ruter (of class Ruter)
  at .parafärg(<console>:17)
\end{REPL}
Detta är ett körtidsfel.

\SubtaskSolved  Om en klass är \code{sealed} innebär det att om ett element ska matchas och är en subtyp av denna klass så ger Scala varning redan vid kompilering om det finns en risk för ett \code{MatchError}, alltså om \code{match}-uttrycket inte är uttömmande och det finns fall som inte täcks av ett \code{case}.\\
En förseglad supertyp innebär att programmeraren redan vid kompileringstid får en varning om ett fall inte täcks och i sånt fall vilket av undertyperna, liksom annan hjälp av kompilatorn. Detta kräver dock att alla subtyperna delar samma fil som den förseglade klassen.



\QUESTEND


\WHAT{Mönstermatcha enumeration.}

\QUESTBEGIN
%\TODO %Se gärna över denna frågan samt facit.
\Task	\what~ Vi ska nu undersöka och jämföra skillnad mellan nyckelorden \code{enum} och \code{sealed trait}. Skriv nedan kod i en REPL.
\begin{Code}
enum Färg:
  case Spader, Hjärter, Ruter, Klöver
\end{Code}

\Subtask Skapa med hjälp av en editor igen en funktion \code{def parafärg(f: Färg): Färg}, nästintill likadan som den som vi skapade i deluppgift \ref{task:match-sealedtrait}\ref{subtask:match-sealedtrait-function}. Funktionen ska återigen utnyttja match-uttryck för att returnera paralellfärgen till argumentet som ges. Tänk på att denna gången är \code{Färg} inget \code{sealed trait}, utan istället en enumeration (\code{enum}). Klistra in funktionen i REPL.
\begin{REPL}
scala> parafärg(Färg.Ruter)
scala> val xs = Vector.fill(5)(Färg.values((math.random() * 4).toInt))
scala> xs.map(parafärg)
\end{REPL}


\Subtask
Fundera på skillnader och likheter mellan att utnyttja \code{sealed trait} ihop med \code{case}-objekt gentemot att använda sig av \code{enum} vid mönstermatchning.


\SOLUTION


\TaskSolved \what
\SubtaskSolved
\begin{Code}
def parafärg(f: Färg): Färg = f match
  case Färg.Spader  => Färg.Klöver
  case Färg.Hjärter => Färg.Ruter
  case Färg.Ruter   => Färg.Hjärter
  case Färg.Klöver  => Färg.Spader
\end{Code}
Likt uppgift \ref{task:match-sealedtrait}\ref{subtask:match-sealedtrait-function} så kan även här en \code{import}-sats skrivas för att nå medlemmarna i \code{Färg} utan punktnotation.
Det är dock inte alltid fördelaktigt att importera medlemmar till den globala namnrymden, då det kan förekomma namnkrockar. Anta ett exempel där vi jobbar på ett program med grafiskt användargränssnitt där vi har en färg \code{Red} definerad.
Anta också att vi nu till vårt program vill importera ytterligare en röd färg för kulörerna hjärter och ruter, denna också namngiven \code{Red}. I detta scenario hade det uppstått en namnkrock då \code{Red} redan är definerad så importeringen hade ej kunnat ske.

\SubtaskSolved
Vid mönstermatchning så fungerar \code{sealed trait} ihop med \code{case}-objekt i praktiken likadant som att använda sig av \code{enum}.
Vi såg att i deluppgift \ref{task:match-sealedtrait}\ref{subtask:match-forgetcase} så varnade REPL redan vid kompilering att denna matchning inte var uttömmande \Eng{exhaustive}. Detta gäller även vid användning av \code{enum}.

\QUESTEND



\WHAT{Betydelsen av små och stora begynnelsebokstäver vid matchning.}

\QUESTBEGIN

\Task  \what~  För att åstadkomma att namn kan bindas till variabler vid matchning utan att de behöver deklareras i förväg (som vi såg i uppgift \ref{task:match-caseclass}) så har identifierare med liten begynnelsebokstav fått speciell betydelse: den tolkas av kompilatorn som att du vill att en variabel  binds till ett värde vid matchningen. En identifierare med stor begynnelsebokstav tolkas däremot som ett konstant värde (t.ex. ett case-objekt eller ett case-klass-mönster).

\Subtask \emph{En case-gren som fångar allt}. En case-gren med en identifierare med liten begynnelsebokstav som saknar gard kommer att matcha allt. Prova nedan i REPL, men försök lista ut i förväg vad som kommer att hända. Vad händer?
\begin{REPL}
scala> val x = "urka"
scala> x match
         case str if str.startsWith("g") => println("kanske gurka")
         case vadsomhelst => println("ej gurka: " + vadsomhelst)
scala> val g = "gurka"
scala> g match
         case str if str.startsWith("g") => println("kanske gurka")
         case vadsomhelst => println("ej gurka: " + vadsomhelst)
\end{REPL}

\Subtask \emph{Fallgrop med små begynnelsbokstäver.} Innan du provar nedan i REPL, försök gissa vad som kommer att hända. Vad händer? Hur lyder varningarna och vad innebär de?
\begin{REPL}
scala> val any: Any = "varken tomat eller gurka"
scala> case object Gurka
scala> case object tomat
scala> any match
         case Gurka => println("gurka")
         case tomat => println("tomat")
         case _ => println("allt annat")
\end{REPL}

\Subtask \emph{Använd backticks för att tvinga fram match på konstant värde.} Det finns en utväg om man inte vill att kompilatorn ska skapa en ny lokal variabel: använd specialtecknet \emph{backtick}, som skrivs \`{} och kräver speciella tangentbordstryck.\footnote{Fråga någon om du inte hittar hur man gör backtick \`{} på ditt tangentbord.}  Gör om föregående uppgift men omgärda nu identifieraren \code{tomat} i tomat-case-grenen med backticks, så här: \code{  case `tomat` => ...}



\SOLUTION


\TaskSolved \what


\SubtaskSolved  Både \code{str} och \code{vadsomhelst} matchar med inputen, oavsett vad denna är på grund av att de har en liten begynnelsebokstav.\\
 \code{str} har dock en gard att strängen måste börja med $g$ vilket gör så endast \code{val g = "gurka"} matchar med denna. \code{val x = "urka"} plockas dock upp av \code{vadsomhelst} som är utan gard.

\SubtaskSolved
\begin{REPL}
<console>:16: warning: patterns after a variable pattern cannot match (SLS 8.1
.1)
\end{REPL}
och
\begin{REPL}
<console>:17: warning: unreachable code due to variable patter 'tomat' on line
16
\end{REPL}
Trots att en klass \code{tomat} existerar så tolkar Scalas \code{match} den som en \code{case}-gren som fångar allt på grund av en liten begynnelsebokstav. Detta gör så alla objekt som inte är av typen \code{Gurka} kommer ge utskriften \textit{tomat} och att sista caset inte kan nås.

\SubtaskSolved
\begin{Code}
case `tomat` => println("tomat")
\end{Code}



\QUESTEND





\WHAT{Matcha på innehåll i en Vector.}

\QUESTBEGIN

\Task \what ~ Kör nedan i REPL. Vad skrivs ut? Förklara vad som händer.
\begin{REPL}
scala> val xss = Vector(Vector("hej"),Vector("på", "dej"),Vector("4","x","2"))
scala> xss.map( _ match
  case Vector() => "tom"
  case Vector(a) => a.reverse
  case Vector(_, b) => b.reverse
  case Seq(a, "x", b) => a + b
  case _ => "ANNARS DETTA"
  ).foreach(println)
\end{REPL}


\SOLUTION

\TaskSolved \what

\begin{REPL}
jeh
jed
42
\end{REPL}
För varje element i \code{xss} görs en matching som resulterar i en sträng. Vad som händer i varje gren förklaras nedan.
\begin{enumerate}
  \item Första match-grenen aktiveras aldrig eftersom \code{xss} ej innehåller någon tom vektor.
  \item Andra grenen passar med \code{Vector("hej")} och variablen \code{a} binds till \code{"hej"}.
  \item Tredje grenen matchar \code{Vector("på", "dej")} där första värdet binds inte till någon variabel eftersom understreck finns på motsvarande plats, medan andra värdet binds till \code{b}.
  \item Fjärde grenen matchar en sekvens med tre värden där mittenvärdet är \code{"x"}. Den sista grenen aktiveras inte i detta exempel men hade matchat allt som inte fångas av tidigare grenar.
\end{enumerate}

\QUESTEND




\WHAT{Använda \code{Option} och matcha på värden som kanske saknas.}

\QUESTBEGIN

\Task  \what~  Man behöver ofta skriva kod för att hantera värden som eventuellt saknas, t.ex. saknade telefonnummer i en persondatabas. Denna situation är så pass vanlig att många språk har speciellt stöd för saknande värden.

I Java\footnote{Scala har också \code{null} men det behövs bara vid samverkan med Java-kod.} används värdet \code{null} för att indikera att en referens saknar värde. Man får då komma ihåg att testa om värdet saknas varje gång sådana värden ska behandlas, t.ex. med \code+if (ref != null) { ...} else { ... }+. Ett annat vanligt trick är att låta \code{-1} indikera saknade positiva heltal, till exempel saknade index, som får behandlas med \code+if (i != -1) { ...} else { ... }+.

I Scala finns en speciell typ \code{Option} som möjliggör smidig och typsäker hantering av saknade värden. Om ett kanske saknat värde packas in i en \code{Option} \Eng{wrapped in an Option}, finns det i en speciell slags samling som bara kan innehålla \emph{inget} eller \emph{något} värde, och alltså har antingen storleken \code{0} eller \code{1}.

\Subtask Förklara vad som händer nedan.
\begin{REPL}
scala> var kanske: Option[Int] = None
scala> kanske.size
scala> kanske = Some(42)
scala> kanske.size
scala> kanske.isEmpty
scala> kanske.isDefined
scala> def ökaOmFinns(opt: Option[Int]): Option[Int] = opt match
         case Some(i) => Some(i + 1)
         case None    => None
scala> val annanKanske = ökaOmFinns(kanske)
scala> def öka(i: Int) = i + 1
scala> val merKanske = kanske.map(öka)
\end{REPL}

\Subtask Mönstermatchingen ovan är minst lika knölig som en \code{if}-sats, men tack vare att en \code{Option} är en slags (liten) samling finns det smidigare sätt. Förklara vad som händer nedan.
\begin{REPL}
val meningen = Some(42)
val ejMeningen = Option.empty[Int]
meningen.map(_ + 1)
ejMeningen.map(_ + 1)
ejMeningen.map(_ + 1).orElse(Some("saknas")).foreach(println)
meningen.map(_ + 1).orElse(Some("saknas")).foreach(println)
\end{REPL}

\Subtask \emph{Samlingsmetoder som ger en \code{Option}.} Förklara för varje rad nedan vad som händer. En av raderna ger ett felmeddelande; vilken rad och vilket felmeddelande?
\begin{REPL}
val xs = (42 to 84 by 5).toVector
val e = Vector.empty[Int]
xs.headOption
xs.headOption.get
xs.headOption.getOrElse(0)
xs.headOption.orElse(Some(0))
e.headOption
e.headOption.get
e.headOption.getOrElse(0)
e.headOption.orElse(Some(0))
Vector(xs, e, e, e)
Vector(xs, e, e, e).map(_.lastOption)
Vector(xs, e, e, e).map(_.lastOption).flatten
xs.lift(0)
xs.lift(1000)
e.lift(1000).getOrElse(0)
xs.find(_ > 50)
xs.find(_ < 42)
e.find(_ > 42).foreach(_ => println("HITTAT!"))
\end{REPL}

\Subtask Vilka är fördelerna med \code{Option} jämfört med \code{null} eller \code{-1} om man i sin kod glömmer hantera saknade värden?

\SOLUTION


\TaskSolved \what


\SubtaskSolved  \begin{enumerate}
\item \code{var kanske} blir en \code{Option} som håller \code{Int} men är utan något värde, kallas då \code{None}.
\item Eftersom \code{var kanske} är utan värde är storleken av den 0.
\item \code{var kanske} tilldelas värdet 42 som förvaras i en \code{Some} som visar att värde finns.
\item Eftersom \code{var kanske} nu innehåller ett värde är storleken 1.
\item Eftersom \code{var kanske} innehåller ett värde är den inte tom.
\item Eftersom \code{var kanske} innehåller ett värde är den definierad.
\item \code{def ökaOmFinns} matchar en \code{Option[Int]} med dess olika fall.\\
Finns ett värde, alltså \code{opt: Option[Int]} är en \code{Some}, så returneras en \code{Some} med ursprungliga värdet plus 1.\\
Finns inget värde, alltså \code{opt: Option[Int]} är en \code{None}, så returneras en \code{None}.
\item -
\item -
\item -
\item \code{def ökaOmFinns} appliceras på \code{kanske} och returnerar en \code{Some} med värdet hos \code{kanske} plus 1, alltså 43.
\item \code{def öka} tar emot värdet av en \code{Int} och returnerar värdet av denna plus 1.
\item \code{map} applicerar \code{def öka} till det enda elementen i \code{kanske}, 42. Denna funktion returnerar en \code{Some} med värdet 43 som tilldelas \code{merKanske}.
\end{enumerate}

\SubtaskSolved  \begin{enumerate}
\item \code{val meningen} blir en \code{Some} med värdet 42.
\item \code{val ejMeningen} blir en \code{Option[Int]} utan något värde, en \code{None}.
\item \code{map(_ + 1)} appliceras på \code{meningen} och ökar det existerande värdet med 1 till 43.
\item \code{map(_ + 1)} appliceras på \code{ejMening} men eftersom inget värde existerar fortsätter denna vara \code{None}.
\item \code{map(_ + 1)} appliceras ännu en gång på \code{ejMening} men denna gång inkluderas metoden \code{orElse}. Om ett värde inte existerar hos en \code{Option}, alltså är av typen \code{None}, så utförs koden i \code{orElse}-metoden som i detta fall skriver ut \textit{saknas} för värdet som saknas.
\item Samma anrop från föregående rad utförs denna gång på \code{meningen} och eftersom ett värde finns utförs endast första biten som ökar detta värde med 1.
\end{enumerate}
Denna metod kan användas i stället för \code{match}-versionen i föregående exempel i och med dennas simplare form. En \code{Option} innehåller ju antingen ett värde eller inte så ett längre \code{match}-uttryck är inte nödvändigt.

\SubtaskSolved \begin{enumerate}
\item En vektor \code{xs} skapas med var femte tal från 42 till 82.
\item En tom \code{Int}-vektor \code{e} skapas.
\item \code{headOption} tar ut första värdet av vektorn \code{xs} och returnerar den sparad i en \code{Option}, \code{Some(42)}.
\item Första värdet i vektorn \code{xs} sparas i en \code{Option} och hämtas sedan av \code{get}-metoden, 42.
\item Som i föregående rad men denna gång används \code{getOrElse} som om den \code{Option} som returneras saknar ett värde, alltså är av typen \code{None}, returnerar 0 istället.\\
 Eftersom \code{xs} har minst ett värde så är den \code{Option} som returneras inte \code{None} och ger samma värde som i föregående, 42.
\item Som föregående rad fast istället för att returnera 0 om värde saknas så returneras en \code{Option[Int]} med 0 som värde.
\item \code{headOption} försöker ta ut första värdet av vektorn \code{e} men eftersom denna saknar värden returneras en \code{None}.
\item \begin{REPL}
java.util.NoSuchElementException: None.get
\end{REPL}
Liksom föregående rad returnerar \code{headOption} på den tomma vektorn \code{e} en \code{None}. När  \code{get}-metoden försöker hämta ett värde från en \code{None} som saknar värde ger detta upphov till ett körtidsfel.
\item Liksom i föregående returneras \code{None}  av \code{headOption} men eftersom \code{getOrElse}-metoden används på denna \code{None} returneras 0 istället.
\item Liksom föregående används \code{getOrElse}-metoden på den \code{None} som returneras. Denna gång returneras dock en \code{Option[Int]} som håller värdet 0.
\item En vektor innehållandes elementen \code{xs}-vektorn och 3 \code{e}-vektorer skapas.
\item \code{map} använder metoden \code{lastOption} på varje delvektor från vektorn på föregående rad. Detta sammanställer de sista elementen från varje delvektor i en ny vektor. Eftersom vektor \code{e} är tom returneras \code{None} som element från denna.
\item Samma sker som i föregående rad men \code{flatten}-metoden appliceras på slutgiltiga vektorn som rensar vektorn på \code{None} och lämnar endast faktiska värden.
\item \code{lift}-metoden hämtar det eventuella värdet på plats 0  i \code{xs} och returnerar den i en \code{Option} som blir \code{Some(42)}.
\item \code{lift}-metoden försöker hämta elementet på plats 1000 i \code{xs}, eftersom detta inte existerar returneras \code{None}.
\item  Samma sker som i föregående fast applicerat på vektorn \code{e}. Sedan appliceras \code{getOrElse(0)} som, eftersom \code{lift}-metoden returnerar \code{None}, i sin tur returnerar 0.
\item \code{find}-metoden anropas på \code{xs}-vektorn. Den letar upp första talet över 50 och returnerar detta värde i en \code{Option[Int]}, alltså \code{Some(52)}.
\item \code{find}-metoden anropas på \code{xs}-vektorn. Den letar upp första värdet under 42 men eftersom inget värde existerar under 42 i \code{xs} returneras \code{None} istället.
\item \code{find}-metoden anropas på \code{e}-vektorn och skriver ut \textit{HITTAT!} om ett element under 42 hittas. Eftersom \code{e}-vektorn är tom returneras \code{None} vilket \code{foreach} inte räknar som element och därav inte utförs på.
\end{enumerate}

\SubtaskSolved  Användning av -1 som returvärde vid fel eller avsaknad på värde kan ge upphov till körtidsfel som är svåra att upptäcka. \jcode{null} kan i sin tur orsaka kraschar om det skulle bli fel under körningen. \code{Option} har inte samma problem som dessa, används ett \code{getOrElse}-uttryck eller dylikt så kraschar inte heller programmet.\\
Dessutom behöver inte en funktion som returnerar en \code{Option} samma dokumentation av returvärdena. Istället för att skriva kommentarer till koden på vilka värden som kan returneras och vad dessa betyder så syns det direkt i koden.\\
Slutgiltligen är \code{Option} mer typsäkert än \code{null}. När du returnerar en \code{Option} så specificeras typen av det värde som den kommer innehålla, om den innehåller något, vilket underlättar att förstå och begränsar vad den kan returnera.



\QUESTEND






\WHAT{Kasta undantag.}

\QUESTBEGIN

\Task  \what~  Om man vill signalera att ett fel eller en onormal situtation uppstått så kan man \textbf{kasta} \Eng{throw} ett \textbf{undantag} \Eng{exception}. Då avbryts programmet direkt med ett felmeddelande, om man inte väljer att \textbf{fånga} \Eng{catch} undantaget.
\Subtask Vad händer nedan?
\begin{REPL}
scala> throw new Exception("PANG!")
scala> java.lang.   // Tryck TAB efter punkten
scala> throw new IllegalArgumentException("fel fel fel")
scala> val carola = 
         try 
           throw new Exception("stormvind!")
           42
         catch 
           case e: Throwable => 
             println("Fångad av en " + e)
             -1
\end{REPL}
\Subtask Nämn ett par undantag som finns i paketet \code{java.lang} som du kan gissa vad de innebär och i vilka situationer de kastas.

\Subtask Vilken typ har variabeln \code{carola} ovan? Vad hade typen blivit om catch-grenen hade returnerat en sträng i stället?

\SOLUTION


\TaskSolved \what


\SubtaskSolved  \begin{enumerate}
\item Ett \code{Exception} kastas med felmeddelandet \textit{PANG!}.
\item Flera olika typer av \code{Exception} visas.
\item En typ av \code{Exception}, \code{IllegalArgumentException}, kastas med felmeddelandet \textit{fel fel fel}.
\item Ett undantag med felmeddelandet \code{stormvind!} kastas och fångas av \code{catch}-uttrycket. Ett \code{match}-uttryck undersöker undantaget och skriver ut meddelandet, samt returnerar -1.
\end{enumerate}

\SubtaskSolved  Exempelvis: \\
\code{OutOfMemoryError}, om programmet får slut på minne.\\
\code{IndexOutOfBoundsException}, om en vektorposition som är större än vad som finns hos vektorn försöker nås.\\
\code{NullPointerException}, om en metod eller dylikt försöker användas hos ett objekt som inte finns och därav är en nullreferens.

\SubtaskSolved  om både try-grenen och catch-grenen har samma typ, här \code{Int}, så härleder kompilatorn samma typ för hela uttrycket. 
Skulle \code{catch}-grenen returnera ett värde av en helt annan typ istället, t.ex. \code{String}, så blir den mest precisa typen som kompilatorn kan härleda för hela uttrycket \code{Matchable}, som är en direkt subtyp till den mest generella typen \code{Any}.



\QUESTEND










\WHAT{Fånga undantantag med \code{scala.util.Try}.}

\QUESTBEGIN

\Task  \what~  I paketet \code{scala.util} finns typen \code{Try} med stort T som är som en slags samling som kan innehålla antingen ett ''lyckat'' eller ''misslyckat'' värde. Om beräkningen av värdet lyckades och inga undantag kastas blir värdet inkapslat i en \code{Success}, annars blir undantaget inkapslat i en \code{Failure}. Man kan extrahera värdet, respektive undantaget, med mönstermatchning, men det är oftast smidigare att använda samlingsmetoderna \code{map} och \code{foreach}, i likhet med hur \code{Option} används. Det finns även en smidig metod \code{recover} på objekt av typen \code{Try} där man kan skicka med kod som körs om det uppstår en undantagssituation.

\Subtask Förklara vad som händer nedan.
\begin{REPL}
scala> def pang = throw new Exception("PANG!")
scala> import scala.util.{Try, Success, Failure}
scala> Try{pang}
scala> Try{pang}.recover{case e: Throwable =>   "desarmerad bomb: " + e}
scala> Try{"tyst"}.recover{case e: Throwable => "desarmerad bomb: " + e}
scala> def kanskePang = if math.random() > 0.5 then "tyst" else pang
scala> def kanskeOk = Try{kanskePang}
scala> val xs = Vector.fill(100)(kanskeOk)
scala> xs(13) match
         case Success(x) => ":)"
         case Failure(e) => ":( " + e
scala> xs(13).isSuccess
scala> xs(13).isFailure
scala> xs.count(_.isFailure)
scala> xs.find(_.isFailure)
scala> val badOpt = xs.find(_.isFailure)
scala> val goodOpt = xs.find(_.isSuccess)
scala> badOpt
scala> badOpt.get
scala> badOpt.get.get
scala> badOpt.map(_.getOrElse("bomben desarmerad!")).get
scala> goodOpt.map(_.getOrElse("bomben desarmerad!")).get
scala> xs.map(_.getOrElse("bomben desarmerad!")).foreach(println)
scala> xs.map(_.toOption)
scala> xs.map(_.toOption).flatten
scala> xs.map(_.toOption).flatten.size
\end{REPL}


\Subtask Vad har funktionen \code{pang} för returtyp?

\Subtask Varför får funktionen \code{kanskePang} den härledda returtypen \code{String}?

\SOLUTION


\TaskSolved \what


\SubtaskSolved  \begin{enumerate}
\item \code{def pang} skapas som kastar ett \code{Exception} med felmeddelandet \textit{PANG!}.
\item Scalas verktyg \code{Try}, \code{Success} och \code{Failure} importeras.
\item \code{def pang} anropas i \code{Try} som fångar undantaget och kapslar in den i en \code{Failure}.
\item Metoden \code{recover} matchar undantaget i \code{Failure} från föregående rad med ett \code{case} och gör om föredetta \code{Failure} till \code{Success} vid matchning, liknande \code{catch}.
\item Strängen \textit{tyst} körs i föregående test men eftersom inget undantag kastas blir den inkapslad i en \code{Success} och \code{recover} behöver inte göra något. Den tar endast hand om undantag.
\item \code{def kanskePang} skapas som har lika stor chans att returnera strängen \textit{tyst} såsom anropa \code{def pang}.
\item \code{def kanskeOk} skapas som testar \code{def kanskePang} med \code{Try}.
\item En vektor \code{xs} fylls med resultaten, \code{Success} och \code{Failure}, från 100 körningar av \code{kanskeOk}.
\item Elementet på plats 13 i vektor \code{xs} matchas med något av 2 \code{case}. Om det är en \code{Success} skrivs \textit{:)} ut, om en \code{Failure} skrivs \textit{:(} plus felmeddelandet ut.
\item -
\item -
\item Metoden \code{isSuccess} testar om elementet på plats 13 i \code{xs} är en \code{Success} och returnerar \code{true} om så är fallet.
\item Metoden \code{isFailure} testar om elementet på plats 13 i \code{xs} är en \code{Failure} och returnerar \code{true} om så är fallet.
\item Metoden \code{count} räknar med hjälp av \code{isFailure} hur många av elementen i \code{xs} som är \code{Failure} och returnerar detta tal.
\item Metoden \code{find} letar upp med hjälp av \code{isFailure} ett element i \code{xs} som är \code{Failure} och returnerar denna i en \code{Option}.
\item \code{badOpt} tilldelas den första \code{Failure} som hittas i \code{xs}.
\item \code{goodOpt} tilldelas den första \code{Success} som hittas i \code{xs}.
\item Resultatet badOpt skrivs ut, \code{Option[scala.util.Try[String]] =}\\
\code{Some(Failure(java.lang.Exception: PANG!))}
\item Metoden \code{get} hämtar från \code{badOpt} den \code{Failure} som förvaras i en \code{Option}.
\item Metoden \code{get} anropas ännu en gång på resultatet från föregående rad, alltså en \code{Failure}, som hämtar undantaget från denna och som då i sin tur kastas.
\item Metoden \code{getOrElse} anropas på den \code{Failure} som finns i \code{badOpt}. Eftersom detta är en \code{Exception} utförs \code{orElse}-biten istället för att undantaget försöker hämtas. Då returneras strängen \textit{bomben desarmerad!}.
\item Metoden \code{getOrElse} anropas på den \code{Success} som finns i \code{goodOpt}. Eftersom detta är en \code{Success} med en normal sträng sparad i sig returneras denna sträng, \textit{tyst}.
\item Metoden från föregående används denna gång på alla element i \code{xs} där resultatet skrivs ut för varje.
\item Metoden \code{toOption} appliceras på alla \code{Success} och \code{Failure} i \code{xs}. De med ett exception, alltså \code{Failure}, blir en \code{None} medan de med värden i \code{Success} ger en \code{Some} med strängen \textit{tyst} i sig.
\item Metoden \code{flatten} appliceras på vektorn fylld med \code{Option} från föregående rad för att ta bort alla \code{None}-element.
\item Metoden \code{size} används på slutgiltiga listan från föregående rad för att räkna ut hur många \code{Some} som resultatet innehåller. Den har alltså beräknat antalet element i \code{xs} som var av typen \code{Success} med hjälp av \code{Option}-typen.
\end{enumerate}

\SubtaskSolved  \code{pang} har returtypen \code{Nothing}, en specialtyp inom Scala som inte är kopplad till \code{Any}, och som inte går att returnera.

\SubtaskSolved  Typen \code{Nothing} är en subtyp av varenda typ i Scalas hierarki. Detta innebär att den även är en subtyp av \code{String} vilket implicerar att \code{String} inkluderar både strängar och \code{Nothing} och därav blir returtypen.


\QUESTEND




% \WHAT{Laborationsförberedelse.}

% \QUESTBEGIN

% \Task  \what~ \label{task:labprep-patterns-tabular} På veckans laboration ska du hantera data som finns i tabeller med celler som kan bestå av decimaltal eller strängar. Studera den givna koden som du ska utgå ifrån; uppgifterna nedan berör \code{Cell.scala} och \code{Table.scala} här:
% \url{https://github.com/lunduniversity/introprog/tree/master/workspace-old/w13_tabular/src/main/scala/tabular}

% Bastypen \code{Cell} i koden nedan har två subtyper \code{Str} och \code{Num}.

% \begin{CodeSmall}
% sealed trait Cell { def value: String }
% case class Str(value: String) extends Cell
% case class Num(num: BigDecimal) extends Cell { def value = num.toString }
% \end{CodeSmall}
% \code{BigDecimal} används för att representera decimaltal med bättre precision än vanliga flyttal av typen \code{Double}.

% \Subtask Studera dokumentationen för \code{BigDecimal}: \url{https://www.scala-lang.org/api}\\
% Vad gör fabriksmetoden \code{def apply(x: String): BigDecimal} (se kompanjonsobj.).


% \Subtask Vad är fördelen med att \code{Cell} är förseglad?

% \Subtask Kör igång REPL med koden för \code{Cell}-hierarkin tillgänglig på classpath, t.ex. med \code{sbt console}. Vad ger koden nedan för resultat? Ange värde och typ för varje rad.

% \begin{REPL}
% scala> val xs = Seq[Cell](Str("!"), Num(BigDecimal("100000000.000000001")))
% scala> val ys = xs.map(_ match { case Num(n) => Some(n) case _ => None })
% scala> val b = ys.flatten.headOption.getOrElse(BigDecimal(0))
% \end{REPL}

% \Subtask Lägg till ett kompanjonsobjekt enligt nedan. Gör klart den saknade implementationen. Använd \code{Try} och matcha på \code{Success} och \code{Failure}. Testa så att alla metoder i kompanjonsobjektet fungerar.

% \Subtask Gör om implementation så att du i stället använder \code{Try} och \code{getOrElse}. Testa så att det fungerar som innan. Vilken implementation är smidigast?
% \begin{CodeSmall}
% object Cell {
%   import scala.util.{Try, Success, Failure}

%   /** Ger en Num om BigDecimal(s) lyckas annars en Str. */
%   def apply(s: String): Cell =  ???

%   def apply(i: Int): Num = Num(BigDecimal(i))

%   def empty: Str = Str("")

%   def zero: Num = Num(BigDecimal(0))
% }
% \end{CodeSmall}

% \Subtask I given kod och nedan finns en nästan färdig klass för tabelldatahantering. Implementera de saknade delarna enligt beskrivning i dokumentationskommentarerna. Testa så att dina implementationer fungerar och försök förstå hur övriga delar av \code{Table} fungerar.

% \scalainputlisting[numbers=left,basicstyle=\ttfamily\fontsize{9}{11.5}\selectfont]{../workspace-old/w13_tabular/src/main/scala/tabular/Table.scala}

% \noindent Tips vid färdigställande av \code{Table}:
% \begin{itemize}[leftmargin=*]
%   \item Nyckel-värde-tabeller har en metod \code{withDefaultValue} som är smidig om man vill undvika undantag vid uppslagning med nyckel som inte finns och det i stället för undantag är möjligt/lämpligt att erbjuda ett vettigt defaultvärde.
%   \item Metoderna \code{getOrElse} och \code{toOption} på en \code{Try} är smidiga när man vill ge resultat som beror av om det är \code{Success} eller \code{Failure} utan att man behöver göra en \code{match}.
% \item Skiss på implementation av \code{load} i kompanjonsobjektet:
% \begin{CodeSmall}
% def load(fileOrUrl: String, separator: Char): Table = {
%   val source = fileOrUrl match {
%     case /* använd gard och startsWith*/ => scala.io.Source.fromURL(url)
%     case path  => scala.io.Source.fromFile(path)
%   }
%   val lines = try source.getLines.toVector finally source.close
%   val rows = ??? // kör split(separator).toVector på alla rader i lines
%   Table(rows.head, rows.tail.map(_.map(Cell.apply)), separator)
% }
% \end{CodeSmall}
% En webbadress börjar med \code{http}.
% Med \code{try sats1 finally sats2} så kan man garantera att \code{sats2} alltid görs även om \code{sats1} ger undantag. Detta används typiskt för att frigöra resurser som annars förblir allokerade vid undantag. I koden ovan används det för att undvika att filer inte stängs även om något går fel under läsningen.
% \end{itemize}
% \SOLUTION


% \TaskSolved \what

% \SubtaskSolved ''Translates the decimal String representation of a BigDecimal into a BigDecimal.''

% \SubtaskSolved Eftersom \code{Cell} är förseglad med \code{sealed} så kan inga andra subtyper finnas och vi behöver inte kolla efter andra subtyper när vi matchar. Kompilatorn varnar också om vi glömmer matcha på någon av subtyperna.

% \SubtaskSolved
% \begin{REPL}
% scala> val xs = Seq[Cell](Str("!"), Num(BigDecimal("100000000.000000001")))
% xs: Seq[Cell] = List(Str(!), Num(100000000.000000001))

% scala> val ys = xs.map(_ match { case Num(n) => Some(n) case _ => None })
% ys: Seq[Option[BigDecimal]] = List(None, Some(100000000.000000001))

% scala> val b = ys.flatten.headOption.getOrElse(BigDecimal(0))
% b: BigDecimal = 100000000.000000001
% \end{REPL}

% \SubtaskSolved
% \begin{Code}
%   def apply(s: String): Cell = Try(BigDecimal(s)) match {
%     case Success(num) => Num(num)
%     case Failure(_)   => Str(s)
%   }
% \end{Code}

% \SubtaskSolved
% \begin{Code}
%   def apply(s: String): Cell = Try(Num(BigDecimal(s))).getOrElse(Str(s))
% \end{Code}

% \SubtaskSolved \emph{Lämnas som egen laborationsförberedelse.}

% \QUESTEND


\AdvancedTasks %%%%%%%%%%%%%%%%%%%



\WHAT{Använda matchning eller dynamisk bindning?}

\QUESTBEGIN

\Task  \what~ Man kan åstadkomma urskiljningen av de ätbara grönsakerna i uppgift \ref{task:match-caseclass} med dynamisk bindning i stället för \code{match}.

\Subtask Gör en ny variant av ditt program enligt nedan riktlinjer och spara den modifierade koden i filen \texttt{vegopoly.scala} och kompilera och kör.
\begin{itemize}[noitemsep]
\item Ta bort predikatet \code{ärÄtvärd} i objektet \code{Main} och inför i stället en abstrakt metod \code{def ärÄtbar: Boolean} i traiten \code{Grönsak}.
\item Inför konkreta \code{val}-medlemmar i respektive grönsak som definierar ätbarheten.
\item Ändra i huvudprogrammet i enlighet med ovan ändringar så att \code{ärÄtvärd} anropas som en metod på de skördade grönsaksobjekten när de ätvärda ska filtreras ut.
\end{itemize}

\Subtask Lägg till en ny grönsak \code{case class Broccoli} och definiera dess ätbarhet. Ändra i slump-funktionerna så att broccoli blir ovanligare än gurka.

\Subtask Jämför lösningen med \code{match} i uppgift \ref{task:match-caseclass} och lösningen ovan med polymorfism. Vilka är för- och nackdelarna med respektive lösning? Diskutera två olika situationer på ett hypotetiskt företag som utvecklar mjukvara för jordbrukssektorn: 1) att uppsättningen grönsaker inte ändras särskilt ofta medan definitionerna av ätbarhet ändras väldigt ofta och 2) att uppsättningen grönsaker ändras väldigt ofta men att ätbarhetsdefinitionerna inte ändras särskilt ofta.



\SOLUTION


\TaskSolved \what


\SubtaskSolved
\begin{Code}
package vegopoly

trait Grönsak:
	def vikt: Int
	def ärRutten: Boolean
	def ärÄtbar: Boolean

case class Gurka(vikt: Int, ärRutten: Boolean) extends Grönsak:
  val ärÄtbar: Boolean = (!ärRutten && vikt > 100)

case class Tomat(vikt: Int, ärRutten: Boolean) extends Grönsak:
  val ärÄtbar: Boolean = (!ärRutten && vikt > 50)

object Main:
	def slumpvikt: Int = (math.random()*500 + 100).toInt

	def slumprutten: Boolean = math.random() > 0.8

	def slumpgurka: Gurka = Gurka(slumpvikt, slumprutten)

	def slumptomat: Tomat = Tomat(slumpvikt, slumprutten)

	def slumpgrönsak: Grönsak =
    if math.random() > 0.2 then slumpgurka else slumptomat

	def main(args: Array[String]): Unit = 
		val skörd = Vector.fill(args(0).toInt)(slumpgrönsak)
		val ätvärda = skörd.filter(_.ärÄtbar)
		println("Antal skördade grönsaker: " + skörd.size)
		println("Antal ätvärda grönsaker: " + ätvärda.size)
\end{Code}

\SubtaskSolved
Följande \code{case class} läggs till:
\begin{Code}
case class Broccoli(vikt: Int, ärRutten: Boolean) extends Grönsak:
  val ärÄtbar: Boolean = (!ärRutten && vikt > 80)
\end{Code}
~\\
Därefter läggs följande till i \code{object Main} innan \code{def slumpgrönsak}:

\begin{Code}
def slumpbroccoli: Broccoli = Broccoli(slumpvikt, slumprutten)
\end{Code}
~\\
Slutligen ändras \code{def slumpgrönsak} till följande:

\begin{Code}
def slumpgrönsak: Grönsak =     // välj t.ex. denna fördelning:
  val rnd = math.random()
  if rnd > 0.5 then slumpgurka      // 50% sannolikhet för gurka
  else if rnd > 0.2 then slumptomat // 30% sannolikhet för tomat
  else slumpbroccoli             // 20% sannolikhet för broccoli

\end{Code}

\SubtaskSolved  Fördelarna med \code{match}-versionen, och mönstermatchning i sig, är att det är väldigt lätt att göra ändringar på hur matchningen sker. Detta innebär att det skulle vara väldigt lätt att ändra definitionen för ätbarheten. Skulle dock dessa inte ändras ofta utan snarare grönsaksutbudet så kan det polyformistiska alternativet vara att föredra. Detta eftersom det skulle implementeras och ändras lättare än mönstermatchningen vid byte av grönsaker.



\QUESTEND





\WHAT{Metoden \code{equals}.}

\QUESTBEGIN

\Task  \what~   Om man överskuggar den befintliga metoden \code{equals} så kommer metoden \code{==} att fungera annorlunda. Man kan då själv åstadkomma innehållslikhet i stället för referenslikhet. Vi börjar att studera den befintliga equals med referenslikhet.

\Subtask \label{subtask:refequals} Vad händer nedan? Undersök parametertyp och returvärdestyp för  \code{equals}.
\begin{REPL}
scala> class Gurka(val vikt: Int, val ärÄtbar: Boolean)
scala> val g1 = new Gurka(42, true)
scala> val g2 = g1
scala> val g3 = new Gurka(42, true)
scala> g1 == g2
scala> g1 == g3
scala> g1.equals  // tryck ENTER för att se funktionstyp
\end{REPL}

\Subtask Rita minnessituationen efter rad 4.

\Subtask \emph{Överskugga metoderna \code{equals} och \code{hashCode}.}

\begin{Background}
Det visar sig förvånande komplicerat att implementera innehållslikhet med metoden \code{equals} så att den ger bra resultat under alla speciella omständigheter. Till exempel måste man även överskugga en metod vid namn \code{hashCode} om man överskuggar \code{equals}, eftersom dessa båda används gemensamt av effektivitetsskäl för att skapa den interna lagringen av objekten i vissa samlingar. Om man missar det kan objekt bli ''osynliga'' i \code{hashCode}-baserade samlingar -- men mer om detta i senare kurser. Om objekten ingår i en öppen arvshierarki blir det också mer komplicerat; det är enklare om man har att göra med finala klasser. Dessutom krävs speciella hänsyn om klassen har en typparameter.
\end{Background}

\noindent Definera klassen nedan i REPL med överskuggade \code{equals} och \code{hashCode}; den ärver inte något och är final.

\begin{Code}
// fungerar fint om klassen är final och inte ärver något
final class Gurka(val vikt: Int, val ärÄtbar: Boolean):
  override def equals(other: Any): Boolean = other match
    case that: Gurka => vikt == that.vikt && ärÄtbar == that.ärÄtbar
    case _ => false
  override def hashCode: Int = (vikt, ärÄtbar).## //förklaras sen
\end{Code}
\Subtask Vad händer nu nedan, där \code{Gurka} nu har en överskuggad \code{equals} med innehållslikhet?
\begin{REPL}
scala> val g1 = new Gurka(42, true)
scala> val g2 = g1
scala> val g3 = new Gurka(42, true)
scala> g1 == g2
scala> g1 == g3
\end{REPL}
\Subtask Hur märker man ovan att den överskuggade \code{equals} medför att \code{==} nu ger innehållslikhet? Jämför med deluppgift \ref{subtask:refequals}.

I uppgift \ref{task:equals:Complex} får du prova på att följa det fullständiga receptet i 8 steg för att överskugga en \code{equals} enligt konstens alla regler. I efterföljande kurs kommer mer träning i att hantera innehållslikhet och hash-koder. I Scala får man ett objekts hash-kod med metoden \code{##}.%
\footnote{Om du är nyfiken på hash-koder, läs mer här:
\href{https://en.wikipedia.org/wiki/Hash_function}
{en.wikipedia.org/wiki/Hash\_function}
}


\SOLUTION


\TaskSolved \what


\SubtaskSolved  \begin{enumerate}
\item En klass \code{Gurka} skapas med parametrarna \code{vikt} av typen \code{Int} och ärÄtbar av typen \code{Boolean}.
\item \code{g1} tilldelas en instans av \code{Gurka}-klassen med \code{vikt = 42} och \code{ärÄtbar = true}.
\item \code{g2} tilldelas samma \code{Gurka}-objekt som g1.
\item \code{g3} tilldelas en ny instans av \code{Gurka}-klassen med motsvarande parametrar som g1.
\item \code{==}(\code{equals})-metoden jämför g1 med g2 och returnerar \code{true}.
\item \code{==}(\code{equals})-metoden jämför g1 med g3 och returnerar \code{false}.
\item \code{def equals(x\$1: Any): Boolean}
\end{enumerate}
Som kan ses ovan är elementet som jämförs i \code{equals} av typen \code{Any}. Eftersom programmet inte känner till klassen så används \code{Any.equals} vid jämförelsen. Till skillnad från de primitiva datatyperna som vid jämförelse med \code{equals} jämför innehållslikhet, så jämförs referenslikheten hos klasser om inget annat är specificerat. \code{g1} och \code{g2} refererar till samma objekt medan \code{g3} pekar på ett eget sådant vilket innebär att \code{g1} och \code{g3} inte har referenslikhet.

\SubtaskSolved  \\
\vspace{1em}
\tikzstyle{mybox} = [draw=red, fill=blue!20, very thick,
    rectangle, rounded corners, inner sep=10pt, inner ysep=20pt]
\begin{tikzpicture}[
	font=\large\sffamily,
	varname/.style={node distance=0.2cm},
	varbox/.style={draw, node distance=0.2cm},
	objcloud/.style={cloud, cloud puffs=15.7, cloud ignores aspect, align=center, draw},
]

\node [varname] (g1var) {\texttt{g1}};
\node [varbox, right = of g1var] (g1ref) {\phantom{abc}};
\filldraw[black] (g1ref) circle (3pt) node[] (g1dot) {};
\node [objcloud, right = of g1ref, yshift=1.3cm, scale =0.8] (g1obj) {
	\texttt{\textbf{Gurka}} \\~\\ \texttt{vikt} \framebox{42} ~ \texttt{ärÄtvärd} \framebox{true}
};
\draw [arrow] (g1dot) -- (g1obj);

\node [varname, below = of g1var] (g2var) {\texttt{g2}};
\node [varbox, right = of g2var] (g2ref) {\phantom{abc}};
\filldraw[black] (g2ref) circle (3pt) node[] (g2dot) {};
\node [objcloud, right = of g2ref, yshift=-1.3cm, scale =0.8] (g2obj) {
	\texttt{\textbf{Gurka}} \\~\\ \texttt{vikt} \framebox{42} ~ \texttt{ärÄtvärd} \framebox{true}
};
\draw [arrow] (g2dot) -- (g1obj);
\node [varname, below = of g2var] (g3var) {\texttt{g3}};
\node [varbox, right = of g3var] (g3ref) {\phantom{abc}};
\filldraw[black] (g3ref) circle (3pt) node[] (g3dot) {};
\draw [arrow] (g3dot) -- (g2obj);

\end{tikzpicture}

\SubtaskSolved  -

\SubtaskSolved  I de första 3 raderna sker samma som i deluppgift \textit{a}. När nu dessa jämförelser görs mellan \code{Gurka}-objekten så överskuggas \code{Any.equals} av den \code{equals} som är specificerad för just \code{Gurka}. Eftersom båda objekten \code{g1} jämförs med också är av typen \code{Gurka} så matchar den med \code{case that: Gurka}. Denna i sin tur jämför vikterna hos de båda gurkorna och returnerar en \code{Boolean} huruvida de är lika eller inte, vilket de i båda fallen är.

\SubtaskSolved  I deluppgift a gav \code{g1 == g3 false} trots innehållslikhet. Efter skuggningen ger dock detta uttryck \code{true} vilket påvisar jämförelse av innehållslikhet.



\QUESTEND






\WHAT{Polynom.}

\QUESTBEGIN

\Task \label{task:polynomial} \what~   Med hjälp av koden nedan, kan man göra följande:
\begin{REPL}
scala> import polynomial.*

scala> Const(1) * x
res0: polynomial.Term = x

scala> (x*5)^2
res1: polynomial.Prod = 25x^2

scala> Poly(x*(-5), y^4, (z^2)*3)
res2: polynomial.Poly = -5x + y^4 + 3z^2

\end{REPL}

\Subtask Förklara vad som händer ovan genom att studera koden nedan\footnote{Koden finns även här:\\ \href{https://github.com/lunduniversity/introprog/tree/master/compendium/examples/polynomial}{github.com/lunduniversity/introprog/tree/master/compendium/examples/polynomial}}.

\scalainputlisting[numbers=left,basicstyle=\ttfamily\fontsize{10.5}{13}\selectfont]{examples/polynomial/polynomial.scala}

\Subtask Bygg vidare på \code{object polynomial} och implementera addition mellan olika termer.


\SOLUTION


\TaskSolved \what


\SubtaskSolved \TODO

\SubtaskSolved \TODO



\QUESTEND






\WHAT{\code{Option} som en samling.}

\QUESTBEGIN

\Task  \what~Studera dokumentationen för \code{Option} här och se om du känner igen några av metoderna som också finns på samlingen \code{Vector}:\\ \href{http://www.scala-lang.org/api/current/scala/Option.html}{www.scala-lang.org/api/current/scala/Option.html}
\\Förklara hur metoden \code{contains} på en \code{Option} fungerar med hjälp av dokumentationens exempel.



\SOLUTION


\TaskSolved \what 

Exempel på metoder som finns både för \code{Vector} och \{Option}:
\code{foreach}, \code{filter}, \code{fold} etc.

Contains returnerar en \code{Boolean} som visar om den har ett värde eller ej.


\QUESTEND






\WHAT{Fånga undantag med \code{catch} i Java och Scala.}

\QUESTBEGIN

\Task  \what~ Gör motsvarande program i Scala som visas i uppgift \ref{task:javatry}, men utnyttja att Scalas \code{try}-\code{catch} är ett uttryck. Kompilera och kör och testa så att de ur användarens synvinkel fungerar precis på samma sätt. Notera de viktigaste skillnaderna mellan de båda programmen.


\SOLUTION


\TaskSolved \what \TODO


\QUESTEND



\WHAT{Polynom, fortsättning: reducering.}

\QUESTBEGIN

\Task  \what~ Bygg vidare på \code{object polynomial} i uppgift \ref{task:polynomial} på sidan \pageref{task:polynomial} och implementera metoden \code{def reduce: Poly} i case-klassen \code{Poly} som förenklar polynom om flera \code{Prod}-termer kan adderas.

\SOLUTION


\TaskSolved \what



\QUESTEND




% \WHAT{Hash-koder.}

% \QUESTBEGIN

% \Task  \what~ Läs om hash-funktioner här: \href{https://en.wikipedia.org/wiki/Hash_function}{en.wikipedia.org/wiki/Hash_function} \\
% Vad ger metoden \code{##} i scala.Any för resultat? Läs dokumentationen här: \\ \href{http://www.scala-lang.org/api/current/scala/Any.html}{www.scala-lang.org/api/current/scala/Any.html}

% \SOLUTION

% \TaskSolved \what I Scala får man ett objekts hash-kod med metoden \code{##}.

% \QUESTEND






\WHAT{Typsäker innehållstest med metoden \code{===}.}

\QUESTBEGIN

\Task  \what~  Metoderna \code{equals} och \code{==} tillåter jämförelse med vad som helst. Ibland vill man ha en typsäker innehållsjämförelse som bara tillåter jämförelse av objekt av en mer specifik typ och ger kompileringsfel annars. Man brukar då definiera en metod \code{===} som har en parameter \code{that} som har en så specifik typ som önskas. Inför nedan abstrakta metod \code{===} i traiten \code{polynomial.Term} i uppgift \ref{task:polynomial} på sidan \pageref{task:polynomial} och överskugga den sedan i alla subklasser till Term. Testa så att du får kompileringsfel om du försöker jämföra en \code{Term} med något helt annat, t.ex. en \code{String} eller \code{Vector}.
\begin{Code}
  def ===(that: Term): Boolean
\end{Code}


\SOLUTION


\TaskSolved \what



\QUESTEND






\WHAT{Överskugga \code{equals} med innehållslikhet även för icke-finala klasser.}

\QUESTBEGIN

\Task \label{task:equals:Complex} \what~   Nedan visas delar av klassen \code{Complex} som representerar ett komplext tal med realdel och imaginärdel. I stället för att, som man ofta gör i Scala, använda en case-klass och en \code{equals}-metod som automatiskt ger innehållslikhet, ska du träna på att implementera en egen \code{equals}.
\begin{Code}
class Complex(val re: Double, val im: Double):
  def abs: Double = math.hypot(re, im)
  override def toString = s"Complex($re, $im)"
  def canEqual(other: Any): Boolean = ???
  override def hashCode: Int  = ???
  override def equals(other: Any): Boolean = ???

case object Complex:
  def apply(re: Double, im: Double): Complex = new Complex(re, im)
\end{Code}
Följ detta \textbf{recept}\footnote{Detta recept bygger på \url{http://www.artima.com/pins1ed/object-equality.html}} i 8 steg för att överskugga \code{equals} med innehållslikhet som fungerar även för klasser som inte är \code{final}:

\begin{enumerate}[leftmargin=*]
\item Inför denna metod: \code{ def canEqual(other: Any): Boolean}\\Observera att typen på parametern ska vara \code{Any}. Om detta görs i en subklass till en klass som redan implementerat \code{canEqual}, behövs även \code{override}.

\item Metoden \code{canEqual} ska ge \code{true} om \code{other} är av samma typ som \code{this}, alltså till exempel: \\
\code{def canEqual(other: Any): Boolean = other.isInstanceOf[Complex]}

\item Inför metoden \code{equals} och var noga med att parametern har typen \code{Any}: \\ \code{override def equals(other: Any): Boolean}

\item Implementera metoden \code{equals} med ett match-uttryck som börjar så här: \\
%\code|other match { ... } |
\code|other match |

\item Match-uttrycket ska ha två grenar. Den första grenen ska ha ett typat mönster för den klass som ska jämföras: \\ \code{  case that: Complex =>}

\item Om du implementerar \code{equals} i den klass som inför \code{canEqual}, börja uttrycket med: \\ \code{(that canEqual this) &&} \\
och skapa därefter en fortsättning som baseras på innehållet i klassen, till exempel: \code{this.re == that.re && this.im == that.im} \\
Om du överskuggar en \textit{annan} equals än den standard-equals som finns i \code{AnyRef}, vill du förmodligen börja det logiska uttrycket med att anropa superklassens equals-metod:
 \code{super.equals(that) && } men du får fundera noga på vad likhet av underklasser egentligen ska innebära i ditt speciella fall.

\item Den andra grenen i matchningen ska vara:
\code{case _ => false}

\item Överskugga \code{hashCode}, till exempel genom att göra en tupel av innehållet i klassen och anropa metoden \code{##} på tupeln så får du i en bra hashcode: \\
\code{override def hashCode: Int  = (re, im).## }

\end{enumerate}


\SOLUTION


\TaskSolved \what



\QUESTEND






\WHAT{Överskugga equals vid arv.}

\QUESTBEGIN

\Task  \what~ Bygg vidare på exemplet nedan och överskugga equals vid arv, genom att följa receptet i uppgift \ref{task:equals:Complex}.
\begin{Code}
trait Number:
  override def equals(other: Any): Boolean = ???

class Complex(re: Double, im: Double) extends Number:
  override def equals(other: Any): Boolean = ???

class Rational(numerator: Int, denominator: Int) extends Number:
  override def equals(other: Any): Boolean = ???
\end{Code}


\SOLUTION


\TaskSolved \what



\QUESTEND






\WHAT{Speciella matchningar.}

\QUESTBEGIN

\Task  \what~ Läs om användning av speciella matchningar här: \\
\href{https://dotty.epfl.ch/docs/reference/changed-features/vararg-splices.html}{dotty.epfl.ch/docs/reference/changed-features/vararg-splices.html}

\Subtask Prova variabelbinding med \texttt{@} i en matchning i REPL.

\Subtask Prova sekvensmönster med \texttt{\_} och \texttt{\_*} i en matching i REPL.

\SOLUTION


\TaskSolved \what \TODO



\QUESTEND






\WHAT{Extraktorer.}

\QUESTBEGIN

\Task \label{task:extractor} \what~  Läs mer om extraktorer här: \\ \href{https://dotty.epfl.ch/docs/reference/changed-features/pattern-matching.html}{dotty.epfl.ch/docs/reference/changed-features/pattern-matching.html} \\
Skapa ditt eget extraktor-objekt för http-addresser som i t.ex.: \\
\texttt{http://my.host.domain/path/to/this} \\ extraherar \texttt{my.host.domain} och \texttt{path/to/this} med metoden \texttt{unapply} och testa i en matchning.

%\Task \TODO \emph{flatten och flatMap med Option och Try}
%Ska detta vara ordinarie uppgift eller fördjupning???


%\Task \TODO \emph{partiella funktioner och metoderna collect och collectFirst på samlingar}
%Ska detta vara ordinarie uppgift eller fördjupning???

\SOLUTION


\TaskSolved \what \TODO



\QUESTEND




\WHAT{Polynom, fortsättning: polynomdivision.}

\QUESTBEGIN

\Task  \what~ Implementera polynomdivision på lämpligt sätt genom att bygga vidare på  \code{object polynomial} i  uppgift \ref{task:polynomial} på sidan \pageref{task:polynomial}.  \\ Läs mer om polynomdivision här: \href{https://sv.wikipedia.org/wiki/Polynomdivision}{sv.wikipedia.org/wiki/Polynomdivision}

\SOLUTION


\TaskSolved \what \TODO

\QUESTEND

%!TEX encoding = UTF-8 Unicode

%!TEX root = ../compendium1.tex

\Lab{\LabWeekSIX}

\begin{Goals}
%\item Kunna skapa en klass utifrån en textuell beskrivning. % av dess medlemmar.
%\item Kunna skapa en klass utifrån ofärdig kod och dokumentationskommentarer.
%\item Kunna införa privata attribut med lämpliga namn som representerar instansers förändringsbara tillstånd.
\item Kunna förklara skillnader och likheter mellan ett singelobjekt och objekt som är instanser av klasser.
\item Kunna förklara skillnaden mellan förändringsbara och oföränderliga objekt.
\item Kunna definiera och instansiera klasser och case-klasser, samt kunna beskriva när en case-klass är lämpligast och ge några exempel på vad en sådan erbjuder utöver en vanlig klass.
\item Kunna skapa och använda klasser vars instanser innehåller referenser till andra instanser (aggregering).
\item Förstå innebörden av instansreferensen \code{this}.
\item Kunna skapa enkla match-uttryck.
\end{Goals}

\begin{Preparations}
\item \DoExercise{\ExeWeekFIVE}{05}, speciellt uppgift \ref{exe:classes:labprep}.
\item \DoExercise{\ExeWeekSIX}{06}.
\item Läs igenom hela laborationen och planera ditt arbete.
\item Hämta given kod via \href{https://github.com/lunduniversity/introprog/tree/master/workspace/}{kursen github-plats} eller via hemsidan under \href{https://cs.lth.se/pgk/download/}{Download}.

\end{Preparations}

\subsection{Bakgrund}

{\raggedright%
\begin{minipage}{0.42\textwidth}
\begin{figure}[H]
  %\centering
  \includegraphics[width=1.0\textwidth]{../img/blockbattle.png}
  \caption{En duell om blockmaskar mellan två lundensiska blockmullvader fångade på bild under intensivt grävanade.}
  \label{lab:blockbattle:fig:game}
\end{figure}
\end{minipage}%
}%
\newlength{\currentparskip}%
\newlength{\currentparindent}%
{
\setlength{\currentparskip}{\parskip}% save the value
\setlength{\currentparindent}{\parindent}% save the value
\hfill%
\begin{minipage}{0.47\textwidth}
\setlength{\parskip}{\currentparskip}% restore the value
\setlength{\parindent}{\currentparindent}% restore the value
\noindent Under denna laboration ska du träna på att deklarera klasser och skapa flera instanser av samma klass. Du tränar även på att bygga ett större program från grunden.

Du ska utveckla ett spel för två spelare som sitter vid samma tangentbord, där den vänstra spelaren styr en blockmullvad med tangenterna A,S,D,W, och den högra spelaren styr en annan blockmullvad med piltangenterna.

I bilden till vänster ser du hur spelet kan se ut. Det finns en ljusbrun och en mörkbrun mullvad. Poängräkningen visas överst i himlen. Det finns fyra rosa blockmaskar (se uppgift \ref{lab:blockmole:task:blockworm} i laboration \code{blockmole}) som mullvadarna tävlar om att försöka fånga. När en blockmask teleporterar sig till en ny slumpmässig position lämnar den jord efter sig. När en mullvad gräver sig upp till gräsytan blir det hål i gräset.
Det ger poäng att gräva tunnlar och att fånga blockmaskar.

Du bestämmer själv hur poängsättningen ska ske och kriteriet för när spelet är slut etc.
\end{minipage}%
}



\subsection{Obligatoriska krav}

Följande funktionella krav ska uppfyllas av ditt program:
\begin{itemize}[nosep, label={$\square$},]
%\item Det ska finnas två blockmullvadar, en för vänster spelare och en för höger spelare, som styrs med ASDW resp. piltangerna.
\item Varje mullvad rör sig i sin aktuella riktning tills användaren ändrar riktning genom att trycka på ''sin'' motsvarande knapp, t.ex. W eller pil-upp.
\item Då en mullvad går i mörkbrun jord ska ljusbruna tunnlar grävas.
\item Då en blockmullvad når fönstrets kant eller himlen ska dess riktning reverseras.
\item Det ska ge poäng att gräva tunnlar.
\item Varje spelares poäng ska visas under spelets gång.
\item Ett spel ska avslutas och \emph{Game over} visas när något valfritt kriterium uppfyllts.
%\item Vid \emph{Game over} ska man kunna välja att avsluta programmet eller spela igen.
\end{itemize}

\noindent Din kod ska utformas enligt dessa design-krav:
\begin{itemize}[nosep, label={$\square$}]
\item Ett \code{Game} skapas i huvudprogrammet med metoden \code{start} som kör igång spelet.
\item Konstanter ska namnges och placeras i lämpligt kompanjonsobjekt.
\item Varje klass med ev. tillhörande kompanjonsobjekt ska finnas i en egen kodfil och tillhöra paketet \code{blockbattle}.
\item Du ska utgå från klasserna som du implementerat i uppgift \ref{exe:classes:labprep} i övning \texttt{\ExeWeekFIVE}.
\item Klassen \code{BlockWindow} omvandlar till interna fönsterkoordinater. Övriga klasser ska använda block-koordinater.
\end{itemize}


\subsection{Valbara krav -- välj minst ett}

Du ska implementera minst ett (gärna flera) av dessa krav:
\begin{itemize}[nosep, label={$\square$}]
\item Det ska finnas lagom många blockmaskar (se labb \code{blockmole} uppg. \ref{lab:blockmole:task:blockworm},  sid. \pageref{lab:blockmole:task:blockworm}).
\item Blockmullvadarna ska även ha ett attribut som representerar hälsan, t.ex. ett numeriskt värde mellan 0 och 100. Hälsan ska försvagas något när man gräver tunnlar. Hälsan ska synas i spelfönstret, t.ex. som en sekvens med röda block i himlen som indikerar andelen av maxhälsan för resp. spelare.
\item Att springa på gräset ska påverka poäng och/eller hälsa.
\item Att fånga blockmask ska påverka poäng och/eller hälsa.
\item Det ska finnas gula blockdiamanter som ger många poäng om man tar dem först.
\item Det ska vid spelstart gå att välja namn på respektive blockmullvad och namnet ska synas i spelet vid poängutskriften.
\item Det ska gå fortare att gå i gångar jämfört med att gräva i jord.
\item Om en blockmullvad fångar en blockmask ska dess grävhastighet öka.
\item Om en blockmullvad krockar med en annan blockmullvad ska något hända, t.ex. att dess riktning reverseras.
\item Visa \emph{highscore} vid \emph{Game Over}.  Highscore sparas med \code{introprog.IO} i en fil som skapas om den inte finns annars läses in vid uppstart om den finns och uppdateras vid behov. Spara hela highscore-listan eller bara högsta poäng hittills.
\end{itemize}

\subsection{Förebredelser inför redovisningen}
\Checkpoint\noindent Innan du redovisar din implementation ska du muntligt kunna redogöra för följande:
\begin{itemize}[nosep, label={$\square$}]
  \item Studera någon annans spel och ge din kamrat minst ett tips om hur kodens läsbarhet kan förbättras. Skriv ner dina tips och beskriv dem vid redovisningen.
  \item Beskriv vilka åtgärder du gjort för att din kod ska vara lätt att läsa och förstå.
  \item Beskriv hur du stegvis utvecklat ditt program från enklare till mer avancerad funktionalitet, samt vilka buggar du upptäckt och fixat.
  \item Beskriv vilket eller vilka valfria krav som din implementation uppfyller.
  \item Beskriv hur du hade behövt ändra i klassen \code{Mole} för att det ska gå att skriva\\\code{new Mole().move().move().reverseDir().move()}
\end{itemize}

\subsection{Tips och förslag}

\begin{enumerate}[leftmargin=*]
  \item \textbf{Många små steg.} Kör kompilering under ändringsbevakning med \code{--watch} i ett eget terminalfönster, så att du vid varje ändring kan rätta ev. kompileringsfel. Kör och testa ditt program ett annat terminalfönster.

  \item \textbf{Inför bra namn}. Din kod blir lättare att läsa och ändra i om du hittar på bra namn på medlemmar och lägger dem på lämpligt ställe. T.ex. kan du samla globala spel-konstanter i kompanjonsobjektet till klassen \code{Game}. Du kan bygga vidare på nedan kod och lägga till medlemmar allteftersom du upptäcker att de behövs. Nedan finns exempelvis en funktion som ger bakgrundsfärgen för en viss y-koordinat, vilken är användbar när du ska återställa bakgrunden efter att en mullvad har flyttat sig.
\scalainputlisting[basicstyle=\ttfamily\fontsize{10}{12}\selectfont]{../workspace/w06_blockbattle/Game.scala}
% \begin{CodeSmall}
% object Game {
%   val windowSize = (30, 50)
%   val windowTitle = "EPIC BLOCK BATTLE"
%   val blockSize = 14
%   val skyRange    = 0 to 7
%   val grassRange  = 8 to 8
%   object Color { ??? }
%   def backgroundColorAtDepth(y: Int): java.awt.Color = ???
% }
%
% class Game(
%   val leftPlayerName: String  = "LEFT",
%   val rightPlayerName: String = "RIGHT"
% ) {
%  import Game.* // direkt tillgång till namn på medlemmar i kompanjon
%
%  val window    = new BlockWindow(windowSize, windowTitle, blockSize)
%  val leftMole  = ???
%  val rightMole = ???
%
%  def drawWorld(): Unit = ???
%
%  def eraseBlocks(x1: Int, y1: Int, x2: Int, y2: Int): Unit = ???
%
%  def update(mole: Mole): Unit = ???  // update, draw new, erase old
%
%  def gameLoop(): Unit = ???
%
%  def start(): Unit = {
%    println("Start digging!")
%    println(s"$leftPlayerName ${leftMole.keyControl}")
%    println(s"$rightPlayerName ${rightMole.keyControl}")
%    drawWorld()
%    gameLoop()
%  }
% }
% \end{CodeSmall}

 \item \textbf{Dela upp din kod i funktioner.} Din kod blir lättare att läsa och ändra i om du delar upp den i många små funktioner med bra namn. I \code{Game}-klassen ovan finns exempel på några användbara funktioner. Allteftersom du utvidgar ditt program kan du lägga till fler funktioner som t.ex. heter \code{showPoints}, \code{gameOver}, etc.

\item \textbf{Utformning av \texttt{gameLoop()}}. I ett spel behövs en s.k. spel-loop \Eng{game loop} som upprepar den kod som ska köras vid varje ny skärmbild, ofta kallad \emph{frame}. I varje runda i spel-loopen sker uppdatering av data och ritning i spelfönstret, samt en lämplig fördröjning. En skiss på en typisk spel-loop visas nedan:
\begin{CodeSmall}
  var quit = false
  val delayMillis = 80

  def gameLoop(): Unit = {
    while(!quit) {
      val t0 = System.currentTimeMillis
      handleEvents()    // ändrar riktning vid tangenttryck etc.
      update(leftMole)  // flyttar, ritar, suddar, etc.
      update(rightMole)

      val elapsedMillis = (System.currentTimeMillis - t0).toInt
      Thread.sleep((delayMillis - elapsedMillis) max 0)
    }
  }
\end{CodeSmall}

\item \textbf{Hantering av händelser.} Ett \code{BlockWindow}, som du implementerade i uppgift \ref{exe:classes:labprep} i övning \texttt{\ExeWeekFIVE}, kan via anrop av \code{nextEvent} ge   \code{KeyPressed(key)} vid knapptryck och \code{WindowClosed} vid fönsterstängning. Om ingen händelse finns att behandla returneras \code{Undefined}. Använd en loop som betar av alla händelser tills \code{Undefined} påträffas, enligt nedan:

\begin{CodeSmall}
  def handleEvents(): Unit = {
    var e = window.nextEvent()
    while (e != BlockWindow.Event.Undefined) {
      e match {
        case BlockWindow.Event.KeyPressed(key) =>
          ???  // ändra riktning på resp. mullvad

        case BlockWindow.Event.WindowClosed =>
          ???  // avsluta spel-loopen
      }
      e = window.nextEvent()
    }
  }
\end{CodeSmall}

\item \textbf{Flimmerfri grafik.} För att minska mängden flimmer \Eng{flicker} är det bäst att i varje iteration i spel-loopen (1) bara rita om det som ändrats för att minimera tiden som spenderas på att rita, och (2) vid ändringar rita nya delar före att gamla delar raderas. För att slippa mullvadsflimmer kan du ''\emph{rita först -- sudda sen}'' enligt nedan.\footnote{Inom spelutveckling använder man oftast istället så kallad \emph{double buffering} (eller till och med \emph{triple buffering}) för att få helt flimmerfri grafik. Det ligger dock bortom kursen och stöds inte av \code{PixelWindow}.}

% \begin{CodeSmall}
% val oldMolePos = mole.pos                  // save
% mole.move()                                // update
% window.setBlock(mole.pos, mole.color)      // draw new
% window.setBlock(oldMolePos, Color.tunnel)  // erase old
% \end{CodeSmall}

\begin{CodeSmall}
window.setBlock(mole.nextPos, mole.color) // draw new
window.setBlock(mole.pos, Color.tunnel)   // erase old
mole.move()                               // update
\end{CodeSmall}

\end{enumerate}


%!TEX encoding = UTF-8 Unicode

%!TEX root = ../compendium1.tex

%!TEX encoding = UTF-8 Unicode
\chapter{Arv}\label{chapter:W07}
Koncept du ska lära dig denna vecka:
\begin{multicols}{2}\begin{itemize}[nosep,label={$\square$},leftmargin=*]
\item arv
\item polymorfism
\item asInstanceOf
\item klasshierarkin i Scala: Any AnyRef Object AnyVal Nothing Null
\item referensklasser vs värdeklasser
\item klasshierarkin i Scalas samlingar
\item Shape som basklass till Point och Rectangle
\item accessregler vid arv
\item protected
\item final
\item abstrakt klass
\item trait
\item inmixning
\item klass vs trait
\item case-object
\item typer med uppräknade värden\end{itemize}\end{multicols}

\clearpage\section{Teori}
%!TEX encoding = UTF-8 Unicode
%!TEX root = ../lect-w07.tex

\Subsection{Vad är en sekvens?}

\ifkompendium\else
{
  \setbeamercolor{background canvas}{bg=black}
  \frame[plain]{\centering\Huge\textbf{\color{pink}{ORDNINGEN}\\SPELAR\\ROLL}}
}
\fi


\begin{Slide}{Vad är en sekvens?}  
\begin{itemize}
\item En sekvens är en \Emph{följd av element} som
  \begin{itemize}
  \item har \Alert{ordningsnummer} (t.ex. numrerade från noll)
  \item är av en viss \Alert{typ} (t.ex. heltal).
  \end{itemize}
  \pause
\item En sekvens kan innehålla flera element som är lika.
\item En sekvens kan vara \Alert{tom} och har då längden noll.
\item Exempel på en icke-tom sekvens med dubbletter:
\begin{REPLnonum}
scala> val xs = Vector(42, 0, 42, -9, 0, 5)
xs: scala.collection.immutable.Vector[Int] =
  Vector(42, 0, 42, -9, 0, 5)
\end{REPLnonum}
\pause
\item \Emph{Indexering} ger ett element via dess ordningsnummer:
\begin{REPLnonum}
scala> xs(2)
res0: Int = 42

scala> xs.apply(2)
res1: Int = 42
\end{REPLnonum}
\end{itemize}
\end{Slide}

\begin{Slide}{Exempel: En sträng är en sekvens av tecken}
\begin{REPLnonum}
scala> "haj po daj"
\end{REPLnonum}
Längd? 
Vad ligger på första platsen?
Elementtyp?
Dubbletter?
\pause
\begin{REPLnonum}
scala> "haj po daj".length
res1: Int = 10

scala> "haj po daj".apply(0)
res2: Char = h

scala> "haj po daj"(0)
res3: Char = h

scala> "haj po daj".distinct
res4: String = haj pod
\end{REPLnonum}

\end{Slide}


\begin{Slide}{Iterera över element i en sekvens}
\begin{itemize}
\item Att \Emph{iterera} \Eng{iterate}, ä.k. traversera \Eng{traverse}, innebär att \Alert{gå igenom} och behandla element i en samling. 
\item Exempel på iterering med \code{foreach}, \code{map}, \code{for}:
\begin{REPLnonum}
scala> val xs = Vector(1,2,3)
val xs: Vector[Int] = Vector(1, 2, 3)

scala> xs.foreach(x => println(x + 1)) 
2
3
4

scala> xs.map(_ + 1)
val res0: Vector[Int] = Vector(2, 3, 4)

scala> for x <- xs yield x - 1
val res1: Vector[Int] = Vector(0, 1, 2)

\end{REPLnonum}
\end{itemize}
\end{Slide}

\begin{Slide}{Lägg till i början och i slutet av en sekvens}
  \begin{itemize}
  \item Med metoderna \code{+:} och \code{:+} kan du skapa en ny sekvens med nya element tillagda i början resp. i slutet.
  \item Minnesregel: ''\Alert{Colon on the collection side}''
  \begin{REPLnonum}
  scala> val xs = Vector(1,2,3)
  scala> 42 +: xs         // ger ny Vector(42, 1, 2, 3)
  scala> xs :+ 42         // ger ny Vector(1, 2, 3, 42)
  \end{REPLnonum}
  \pause
  \item Semantik: operatornotation med operatorer som \Emph{slutar med kolon} är \Alert{högerassociativa}
  \item Anropet \code{42 +: xs} skrivs av kompilatorn om till \code{xs.+:(42)}
  \begin{REPL}
  scala> xs.+:(42)
  res4: scala.collection.immutable.Vector[Int] = Vector(42, 1, 2, 3)
  \end{REPL}
  \pause
  \item Konkatenering (sammanfogning) av sekvenser: \code{xs ++ ys}
  
  \end{itemize}
\end{Slide}


\begin{Slide}{Egenskaper hos några sekvenssamlingar i Scala}
\vspace{-0.5em}
\begin{itemize}\SlideFontSmall

\item \code{Vector}
  \begin{itemize}\SlideFontSmall
  \item \Emph{Oföränderlig}. Snabb på att skapa kopior med små förändringar.
  \item Allsidig prestanda: \Emph{bra till det mesta}.
  \end{itemize}

\item \code{List}
  \begin{itemize}\SlideFontSmall
  \item \Emph{Oföränderlig}. Snabb på att skapa kopior med små förändringar.
  \item Snabb indexering \& uppdatering \Emph{i början}.
  \item Smidig \& snabb vid \Emph{rekursiva} algoritmer.
  \item Långsam vid upprepad \Alert{indexering} på godtyckliga ställen.
  \end{itemize}

\item \code{ArrayBuffer}
  \begin{itemize}\SlideFontSmall
  \item \Alert{Föränderlig}: \Emph{snabb indexering \& uppdatering}.
  \item Kan \Emph{ändra storlek} efter allokering. Snabb att indexera överallt.
  \end{itemize}

\item \code{ListBuffer}
  \begin{itemize}\SlideFontSmall
  \item \Alert{Föränderlig}: snabb indexering \& uppdatering \Emph{i början}.
  \item Snabb om du bygger upp sekvens genom många tillägg i början.
  \end{itemize}

\item \code{Array} (använd fr.o.m. Scala 2.13 hellre \code{ArraySeq})
  \begin{itemize}\SlideFontSmall
  \item \Alert{Föränderlig}: \Emph{snabb indexering \& uppdatering}.
  \item Kan \Alert{ej ändra storlek}; storlek ges vid allokering.
  \item Har särställning i JVM: ger snabb allokering och access.
  \end{itemize}

\end{itemize}
\end{Slide}

\begin{Slide}{I stället för Array: \texttt{ArraySeq} (Scala >2.13)}
Nytt i Scala 2.13:
\begin{itemize}
  \item \code{collection.mutable.ArraySeq} fungerar lika bra som, eller bättre än \code{Array}
  \item \url{https://stackoverflow.com/questions/5028551/array-vs-arrayseq-comparison}
  \item \code{collection.immutable.ArraySeq} fungerar som en Array som inte kan ändras
  \item \code{WrappedArray} är från 2.13 \textit{deprecated} (på väg bort)
\end{itemize}
{\SlideFontTiny Fördjupning: Sammanfattning av prestanda för olika samlingar: \href{https://docs.scala-lang.org/overviews/collections-2.13/performance-characteristics.html}{docs.scala-lang.org/overviews/collections-2.13/performance-characteristics.html}
}
\end{Slide}

\begin{Slide}{Vilken sekvenssamling ska jag välja?}\SlideFontSmall
\vspace{-0.5em}
\begin{itemize}
\item Välj \code{Vector} om ...
  \begin{itemize}\SlideFontTiny
  \item[a)] du vill ha oföränderlighet: \code{val xs = Vector[Int](1,2,3)}
  \item[b)] du behöver föränderlighet (notera \code{var}):\\ \code{var xs = Vector.empty[Int]}
  \item[c)] du ännu inte vet vilken sekvenssamling som är bäst; du kan alltid ändra efter att du mätt prestanda och kollat flaskhalsar vid upprepade körningar.
  \end{itemize}

\item Välj \code{List} om ...
  \begin{itemize}\SlideFontTiny
  \item[] du har en \Alert{rekursiv} sekvensalgoritm och/eller \Alert{mestadels jobbar i början}.
  \end{itemize}

\item Välj \code{ArrayBuffer} om ...
  \begin{itemize}\SlideFontTiny
  \item[] det behövs av prestandaskäl och du \Alert{inte} vet storlek vid allokering:\\
  \code{val xs = scala.collection.mutable.ArrayBuffer.empty[Int]}
  \end{itemize}

\item Välj \code{ListBuffer} om ...
  \begin{itemize}\SlideFontTiny
  \item[] det behövs av prestandaskäl och du bara behöver lägga till i början:\\ \code{val xs = scala.collection.mutable.ListBuffer.empty[Int]}
  \end{itemize}

\item Välj \code{Array} eller \code{ArraySeq} om ...
  \begin{itemize}\SlideFontTiny
  \item[] det verkligen behövs av prestandaskäl och du \Alert{vet} storlek vid allokering:\\
  \code{val xs = Array.fill(initSize)(initValue)}
  \end{itemize}

\end{itemize}
\end{Slide}

\begin{Slide}{Några konstigheter med Array}
\begin{itemize}
\item Referenslikhet istf innehållslikhet: 
\begin{REPLnonum}
scala> Vector(1,2,3) == Vector(1,2,3)
val res0: Boolean = true

scala> Array(1,2,3) == Array(1,2,3)
val res1: Boolean = false  // aaargh!!
\end{REPLnonum}
\item Special-syntax för att skapa utan explicit initialisering: \\
{\SlideFontSmall\code{val xs = new Array[String](1000)  // 1000 null-referenser}}
\item Fungerar inte lika bra med generiska typer:
\begin{REPLnonum}
scala> def box[T](x: T) = Vector[T](x)  //funkar fint

scala> def abox[T](x: T) = Array[T](x)
  error: No ClassTag available for T
\end{REPLnonum}
\end{itemize}
\end{Slide}

\begin{Slide}{Oföränderlig eller förändringsbar?}
\begin{itemize}
\item \Emph{Oföränderlig}:  Kan ej ändra elementreferenserna, men effektiv på att skapa kopia som är (delvis) förändrad \Emph{Vector} eller \Emph{List}

\item \Alert{Förändringsbar}: kan ändra elementreferenserna
  \begin{itemize}
  \item Kan \Alert{ej ändra storlek} efter allokering: \\ \Emph{Array} eller \Emph{ArraySeq}: indexera och uppdatera varsomhelst
  \item Kan även ändra storlek efter allokering:
  \\\Alert{ArrayBuffer} eller \Alert{ListBuffer}
  %\\ Java: \Alert{ArrayList} eller \Alert{LinkedList}
  \end{itemize}
\item \Emph{Ofta funkar oföränderlig sekvenssamling utmärkt}, men om man \Alert{efter prestandamätning} upptäcker en flaskhals kan man ändra från \Emph{Vector} till t.ex. \Emph{ArrayBuffer}.
\end{itemize}
\end{Slide}



\Subsection{Vad är en sekvensalgoritm?}



\begin{Slide}{Vad är en sekvensalgoritm?}\SlideFontTiny
\begin{itemize}
\item En algoritm är en stegvis beskrivning av lösningen på ett problem.
\item En \textbf{sekvensalgoritm} är en algoritm där \Emph{element i sekvens} utgör en viktig del av \Alert{problembeskrivningen} och/eller \Alert{lösningen}.
\item Exempelproblem: sortera en sekvens av personer efter deras ålder.
\pause
\item \Alert{Sju} ofta återkommande programmeringsproblem som löses med en sekvensalgoritm:
\begin{itemize}\SlideFontTiny
\item \Emph{Kopiering} av alla element i en sekvens till en \Alert{ny} sekvens
\item \Emph{Uppdatering} av sekvensen: ta bort, lägga till, ändra \Emph{enskilda} element
\item \Emph{Transformering}: applicera en \Alert{funktion} på \Emph{alla} element   
\item \Emph{Filtrering}: urval av vissa element som uppfyller ett \Alert{villkor}
\item \Emph{Sökning} efter ett element som uppfyller ett \Alert{sökkriterium}
\item \Emph{Sortering} enligt någon \Alert{ordning}
\item \Emph{Registrering} av \Alert{frekvens} av element med vissa \Alert{egenskaper}
\end{itemize}
\end{itemize}
\href{https://youtu.be/0ArlUSVDQIw?t=27s}{\textbf{KUT FSSR}} 
\end{Slide}

\ifkompendium\else
{
  \setbeamercolor{background canvas}{bg=black}
  \frame[plain]{\centering\Huge\textbf{\color{pink}{ORDNINGEN\\SPELAR\\ROLL}\\\color{red}{KUT FSSR}}}
}
\fi



\Subsection{Använda färdiga sekvenssamlingsmetoder}


\begin{Slide}{Använda färdiga sekvenssamlingsmetoder}
\begin{itemize}\SlideFontSmall
\item Standardbiblioteket i Scala innehåller flera olika samlingar som har sekvensegenskaper, t.ex. \code{Vector} och \code{ArrayBuffer}, som erbjuder olika möjligheter och har olika prestanda beroende på vad man vill göra.
\item Scala 3 använder samma samlingsbibliotek som i Scala 2.13, se intro:  \url{https://docs.scala-lang.org/overviews/collections-2.13/introduction.html} 

\item Ofta kan man implementera sekvensalgoritmer genom anrop av en eller flera \Alert{färdiga} metoder.

\item Dessa färdiga metoder är \Emph{optimerade och vältestade} och är att föredra om möjligt.

\item Studera Scalas api-dokumentation och kursens quickref för att se vad man kan göra med färdiga metoder.

\item Det är \Emph{lärorikt} att ''\Alert{uppfinna hjulet}'' och implementera några sekvensalgoritmer \Emph{själv} för bättre förståelse, även om de redan finns färdiga i Scalas samlingsbibliotek.

\end{itemize}
\end{Slide}



\begin{Slide}{Några användbara samlingsmetoder vid implementation av sekvensalgoritmer}
\SlideFontTiny
\begin{tabular}{@{}l l}
\code|xs.map(f)|           & transformering, motsv. \code|for x <- xs yield f(x)| \\
\code|xs.map(x => x)|    & kopiering, motsv. \code|for x <- xs yield x| \\
\code|xs.filter(p)|        & filtrering, ta med x om p(x)\\
\code|xs.filterNot(p)|     & filtrering, ta med x om !p(x)\\
\code|xs.distinct|        & filtrering, ta bort dubbletter \\
\code|xs.take(n)|          & ny sekvens med de första n elements, resten skippade\\ 
\code|xs.drop(n)|          & ny sekvens där de första n elements är skippade\\ 
\code|xs.takeWhile(p)|     & filtrera, ta med i början så länge p(x)  \\
\code|xs.dropWhile(p)|     & filtrera, skippa i början så länge p(x)  \\
\code|xs.find(p)|       & sök framifrån efter första element x där p(x) är sant\\
\code|xs.indexOf(x)|       & sök framifrån efter index för element som är samma som x \\
\code|xs.lastIndexOf(x)|   & sök bakifrån efter index för element som är samma som x \\
\code|xs.sorted|          & sortera med inbyggd (implicit given) ordning \\
\code|xs.sorted.reverse| & sortera i omvänd ordning \\
\code|xs.sortBy(f)|        & sortera i ordning enligt f(x)\\
\code|xs.sortWith(lt)|     & sortera enligt ''less than''-funktionen \code|lt: (A, A) => Boolean|\\
\code|xs.count(p)|         & räkna antalet element där p(x) är sant
\end{tabular}

\vspace{0.5em}%
\Emph{Lär dig fler smidiga metoder i} \Alert{quickref}
\end{Slide}

\begin{Slide}{Uppdaterad sekvens med kraftfulla metoden \texttt{patch}}
  Metoden \code{patch} kan användas så: \code{xs.patch(fromPos, ys, nbrReplaced)} \\för att skapa en \Alert{ny} sekvens där \Emph{ett} eller \Emph{flera} element i xs är...  
  \begin{itemize}
    \item utbytta \Eng{replaced}
    \item borttagna \Eng{removed}
    \item tillagda \Eng{inserted}
  \end{itemize}
  .. med nya element ur \code{ys} 
\begin{REPL}
scala> val xs = Vector(1,2,3)

scala> xs.patch(2, Vector(-1), 1)     // replaced one elem
res0: scala.collection.immutable.Vector[Int] = Vector(1, 2, -1)

scala> xs.patch(1, Vector(42), 0)     // inserted one elem
res11: scala.collection.immutable.Vector[Int] = Vector(1, 42, 2, 3)

scala> xs.patch(0, Vector(), 2)       // removed two elems
res2: scala.collection.immutable.Vector[Int] = Vector(3)
 
\end{REPL}
\end{Slide}

\begin{Slide}{Använda for-uttryck för filtrering med hjälp av gard}
I ett for-uttryck kan man ha en \Emph{gard} \Eng{guard} i form av ett booleskt uttryck efter nyckelordet \code{if}. Då kommer uttrycket efter \code{yield} bara göras om gard-uttrycket är sant.

\vspace{1em}

Syntaxen är så här: (parenteser behövs ej runt gard-uttrycket)
\begin{Code}[basicstyle=\ttfamily\SlideFontSize{12}{14}]
for x <- xs if uttryck1 yield uttryck2
\end{Code}
\pause
Exempel:
\begin{REPLnonum}
scala> val udda = for x <- 1 to 6 if x % 2 == 1 yield x
\end{REPLnonum}
\pause
\code{udda} blir \code{Vector(1, 3, 5)}
\end{Slide}


\begin{Slide}{Använda samlingsmetoden \texttt{filter} för filtrering}
Alla samlingar i \code{scala.collection} har metoden \code{filter}. Den har ett predikat som parameter \code{p: T => Boolean} och ger en ny samling med de element för vilka predikatet är sant.
\begin{Code}[basicstyle=\ttfamily\SlideFontSize{12}{14}]
xs.filter(p)
\end{Code}
\pause
Exempel: Antag att \code{xs} är \code{(1 to 6).toVector}
\begin{REPLnonum}
xs.filter(_ % 2 == 1)
\end{REPLnonum}
\pause
uttryckets resultat blir \code{Vector(1, 3, 5)}, vilket motsvarar:
\begin{Code}[basicstyle=\ttfamily\SlideFontSize{10}{13}]
for x <- xs if x % 2 == 1 yield x
\end{Code}
\pause
I själva verket skriver Scala-kompilatorn om for-uttryck med gard till anrop av metoden \code{filter} före kodgenerering sker.
\end{Slide}


\begin{Slide}{Vanliga sekvensproblem som funktionshuvuden}
Indata och utdata för några vanliga sekvensproblem:
\begin{Code}
def copy(xs: Vector[Int]): Vector[Int] = ???

def filter(xs: Vector[Int], p: Int => Boolean): Vector[Int] = ???

def findIndices(xs: Vector[Int], p: Int => Boolean): Vector[Int] = ???

def sort(xs: Vector[Int]): Vector[Int] = ???

def freq(xs: Vector[Int]): Vector[(Int, Int)] = ???  // (heltal, frekvens)
\end{Code}
Övning: Hur implementera dessa med \code{for}-uttryck och/eller färdiga samlingsmetoder?\\
\Emph{Tips:} För \code{sort}\&\code{freq} se \code{sorted}, \code{distinct}, \code{count} i \href{http://cs.lth.se/pgk/quickref/}{quickref}
\end{Slide}


\begin{Slide}{Implementation av sekvensproblem med \texttt{for}-uttryck och/eller färdiga samlingsmetoder}
\begin{Code}
def copy(xs: Vector[Int]): Vector[Int] = for x <- xs yield x

def filter(xs: Vector[Int], p: Int => Boolean): Vector[Int] =
  for x <- xs if p(x) yield x

def findIndices(xs: Vector[Int], p: Int => Boolean): Vector[Int] =
  (for i <- xs.indices if p(xs(i)) yield i).toVector

def sort(xs: Vector[Int]): Vector[Int] = xs.sorted // mer om sortering sen

def freq(xs: Vector[Int]): Vector[(Int, Int)] = // mer om registrering snart
  for x <- xs.distinct yield x -> xs.count(_ == x)
\end{Code}
Övning: Hur implementera dessa med \code{map} och \code{filter} och/eller andra färdiga samlingsmetoder?
\end{Slide}

\begin{Slide}{Implementation av sekvensproblem med \texttt{map}, \texttt{filter}}
\begin{Code}
def copy(xs: Vector[Int]): Vector[Int] = xs.map(x => x)

def filter(xs: Vector[Int], p: Int => Boolean): Vector[Int] = xs.filter(p)

def findIndices(xs: Vector[Int], p: Int => Boolean): Vector[Int] =
  xs.indices.filter(i => p(xs(i))).toVector

def sort(xs: Vector[Int]): Vector[Int] = xs.sorted // mer om sortering sen

def freq(xs: Vector[Int]): Vector[(Int, Int)] = // mer om registrering snart
  xs.distinct.map(x => x -> xs.count(_ == x))
\end{Code}
%Fördjupningsövning: Hur göra dessa metoder generiska för alla sekvenssamlingar av typen \code{Seq[T]}?
\end{Slide}


\begin{Slide}{Hierarki av samlingstyper i \texttt{scala.collection} v2.13}

  \begin{multicols}{2}
  \begin{tikzpicture}[sibling distance=5.0em,->,>=stealth', inner sep=3pt, %scale=0.5,
    every node/.style = {shape=rectangle, draw, align=center,font=\small\ttfamily},
    class/.style = {fill=blue!20},
    trait/.style = {rounded corners, fill=red!20}]
    \node[trait] {Iterable}
        child { node[trait] {Seq} }
        child { node[trait] {Set} }
        child { node[trait] {Map} }
      ;
  \end{tikzpicture}
  
  \columnbreak
  
  {\SlideFontTiny
  
  \code{Iterable} har metoder som är implementerade med hjälp av: \\
  \code{def foreach[U](f: Elem => U): Unit}\\
  \code{def iterator: Iterator[A] }
  
}

\begin{itemize}\SlideFontTiny
  \item[] \code{Seq}: ordnade i sekvens
  \item[] \code{Set}: unika element
  \item[] \code{Map}: par av (nyckel, värde)
  \end{itemize}
  
  
  \end{multicols}
  
  {\SlideFontSmall Samlingen \Emph{\texttt{Vector}} är en \code{Seq} som är en \code{Iterable}. \\ \vspace{0.5em}\pause
  De konkreta samlingarna är uppdelade i dessa paket:\\
  \code{scala.collection.immutable} \hfill där flera är \Emph{automatiskt} importerade\\
  \code{scala.collection.mutable}  \hfill som \Alert{måste importeras} explicit\\\pause
  (undantag: primitiva \code{scala.Array})
  }
\end{Slide}
  


\begin{Slide}{Lämna det öppet: använd \texttt{Seq}}\SlideFontSmall
Typen \Emph{\code{collection.immutable.Seq}} är supertyp till alla sekvenssamlingar i \code{collection.immutable}.
\pause Exempel: kopiering av sekvens:
\begin{itemize}
\item Kopiering av \Alert{specifik} heltalssekvens:
\begin{Code}
def copyIntVector(xs: Vector[Int]): Vector[Int] = for x <- xs yield x
\end{Code}

\item Kopiering som fungerar för alla oföränderliga heltalssekvenser:
\begin{Code}
def copyIntSeq(xs: Seq[Int]): Seq[Int] = for x <- xs yield x
\end{Code}
\end{itemize}
\pause
\begin{REPL}
scala> val xs = Vector(1,2,3)
xs: Vector[Int] = Vector(1, 2, 3)

scala> val ys = copyIntVector(xs)
ys: Vector[Int] = Vector(1, 2, 3)

scala> val zs = copyIntSeq(xs)
val zs: Seq[Int] = Vector(1, 2, 3)
\end{REPL}
%  Någon lämplig specifik samling som är subtyp till \code{Seq[T]} väljs automatiskt. \\
% (Mer om typparametrar och supertyper/subtyper senare i kursen.)
% \begin{Code}[basicstyle=\ttfamily]
% def varannanBaklänges[T](xs: Seq[T]): Seq[T] =
%   for i <- xs.indices.reverse by -2 yield xs(i)
% \end{Code}
% Fungerar med alla sekvenssamlingar:
% \begin{REPLnonum}
% scala> varannanBaklänges(Vector(1,2,3,4,5))
% res0: Seq[Int] = Vector(5, 3, 1)
%
% scala> varannanBaklänges(List(1,2,3,4,5))
% res1: Seq[Int] = List(5, 3, 1)
%
% scala> varannanBaklänges(collection.mutable.ListBuffer(1,2))
% res2: Seq[Int] = Vector(2)
% \end{REPLnonum}
% Scalas standardbibliotek returnerar ofta lämpligaste specifika sekvenssamlingen som är subtyp till \texttt{Seq[T]}.
\end{Slide}
  
\begin{Slide}{Implementation med generiska funktioner}\SlideFontSmall
Genom att generalisera funktionshuvudena blir våra lösningar användbara för \Alert{alla} sekvenser av typen \code{Seq[T]}, där den obundna \Emph{typparametern} \code{T} vid anrop kan bindas till godtycklig typ. (Mer om typparametrar senare.)
\begin{Code}
def copy[T](xs: Seq[T]): Seq[T] = xs.map(x => x)

def filter[T](xs: Seq[T], p: T => Boolean): Seq[T] = xs.filter(p)

def findIndices[T](xs: Seq[T], p: T => Boolean): Seq[Int] =
  xs.indices.filter(i => p(xs(i))).toVector

def sort[T: Ordering](xs: Seq[T]): Seq[T] = xs.sorted // mer om Ordering sen

def freq[T](xs: Seq[T]): Seq[(T, Int)] =
  xs.distinct.map((_, xs.count(_ == x)))
\end{Code}
\pause
Standardbibliotekets metoder försöker ordna så att det blir samma konkreta typ in som ut, men ibland väljs annan lämplig konkret samling, t.ex. kan en \code{Array} bli en \code{ArrayBuffer}. 
\end{Slide}

\begin{Slide}{Använda Java-samlingar i Scala med \texttt{CollectionConverters}}\SlideFontSmall
Med hjälp av \code{import scala.jdk.CollectionConverters.*} \\
får man smidig \Emph{interoperabilitet} med Java och dess standardbibliotek, \\
speciellt metoderna \Alert{\code{asJava}} och \Alert{\code{asScala}}:
\begin{REPL}
scala> import scala.jdk.CollectionConverters.*

scala> Vector(1,2,3).asJava
res0: java.util.List[Int] = [1, 2, 3]

scala> val xs = new java.util.ArrayList[String]()
xs: java.util.ArrayList[String] = []

scala> xs.add("hej")
res1: Boolean = true

scala> xs.asScala
res2: scala.collection.mutable.Buffer[String] = Buffer(hej)
\end{REPL}

\noindent Läs mer här: %
\ifkompendium\\\fi%
\scriptsize%
\url{https://docs.scala-lang.org/overviews/collections-2.13/conversions-between-java-and-scala-collections.html}

\end{Slide}


\begin{Slide}{Fördjupning: Skapa generisk Array}\SlideFontTiny
\begin{itemize}
\item I JVM bytekod går det tyvärr \Alert{inte} att skapa en primitiv generisk array.

\item Maskinkoden måste istället skapa en array av den mest generella referenstypen \code{Object} 
och sedan \Alert{typtesta och typkonvertera under körtid}.\\ Se t.ex. Java-implementationen av \code{ArrayList}:\\\href{http://developer.classpath.org/doc/java/util/ArrayList-source.html}{http://developer.classpath.org/doc/java/util/ArrayList-source.html} %på rad 119
\item[]
\item Men det \Emph{\emph{går}} att skapa en generisk array i Scala (men inte i Java). Då behövs en \code{reflect.ClassTag} som möjliggör typinformation vid körtid för arrayer. \\
\begin{REPLsmall}
scala> def fyll[T](n: Int, x: T): Array[T] = Array.fill(n)(x)
-- Error:
1 |def fyll[T](n: Int, x: T): Array[T] = Array.fill(n)(x)
  |                                                      ^
  |  No ClassTag available for T

scala> def fyll[T: reflect.ClassTag](n: Int, x: T): Array[T] = Array.fill(n)(x)

scala> fyll(42, "hej")
res2: Array[String] = Array(hej, hej, hej, hej, hej, hej, hej, hej, hej, hej, hej, hej, hej, hej, hej, hej, hej, hej, hej, hej, hej, hej, hej, hej, hej, hej, hej, hej, hej, hej, hej, hej, hej, hej, hej, hej, hej, hej, hej, hej, hej, hej)

\end{REPLsmall}
\item Kompilatorn skapar då maskinkod som automatiskt gör typkonverteringarna.

\end{itemize}
\end{Slide}

%!TEX encoding = UTF-8 Unicode
%!TEX root = ../lect-w07.tex

%%%

\Subsection{Repeterade parametrar}

\begin{Slide}{Repeterade parametrar blir sekvens}\SlideFontSmall
Med en asterisk efter parametertypen kan antalet argument variera:
\begin{Code}[basicstyle=\fontsize{10}{12}\selectfont\ttfamily]
def sumSizes(xs: String*): Int = xs.map(_.length).sum
\end{Code}
\begin{REPLnonum}
scala> sumSizes("Zaphod")
res0: Int = 6

scala> sumSizes("Zaphod","Beeblebrox")
res1: Int = 16

scala> sumSizes("Zaphod","Beeblebrox","Ford","Prefect")
res3: Int = 27

scala> sumSizes()
res4: Int = 0
\end{REPLnonum}
Repeterade parametrar \Eng{repeated parameters} blir en sekvens av typen \code{Seq} och som mer specifikt är en \code{ArraySeq}
\end{Slide}


\begin{Slide}{Sekvenssamling som argument till repeterade parametrar}
\begin{Code}[basicstyle=\fontsize{10}{12}\selectfont\ttfamily]
def sumSizes(xs: String*): Int = xs.map(_.size).sum

val veg = Vector("gurka","tomat")
\end{Code}
Om du \emph{redan har} en sekvenssamling så kan du applicera den på en funktion
som har repeterade parametrar med hjälp av en asterisk \code{*} \\
\vspace{1em}Den ska skrivas direkt \Alert{efter} den sekvenssamling, som du vill att kompilatorn ska tolka som en sekvens av argument, så här:
\begin{REPLnonum}
scala> sumSizes(veg*)
res5: Int = 10
\end{REPLnonum}

\end{Slide}

%!TEX encoding = UTF-8 Unicode
%!TEX root = ../lect-w07.tex

%%%

\Subsection{Enumerationer}
\SlideFontSmall
\begin{Slide}{Enumerationer har en ordning}
En uppräkning av färger i en kortlek med \code{enum}:
\begin{Code}
enum Suit:
  case Spade, Heart, Club, Diamond 
\end{Code}
Användbara metoder för att hantera elementens \Emph{ordningen}:
\begin{REPLsmall}
scala> Suit.Spade.ordinal      // från element till heltal
val res0: Int = 0

scala> Suit.Club.ordinal
val res1: Int = 2

scala> Suit.fromOrdinal(3)    // från heltal till element
val res2: Suit = Diamond

scala> Suit.values            // alla element i ordning
val res3: Array[Suit] = Array(Spade, Heart, Club, Diamond)

scala> Suit.valueOf("Spade")  // från sträng till element
val res4: Suit = Spade
\end{REPLsmall}
\end{Slide}

\begin{Slide}{Enumerationer kan ha parametrar och medlemmar}
En \code{enum} kan ha parametrar. Använd \code{val} för extern synlighet:  
\begin{Code}
enum Color(val consoleColor: String): 
  case Black extends Color(Console.BLUE) //Blå färg syns på svart bakgrund
  case Red   extends Color(Console.RED)
\end{Code}
I \code{enum}-kroppen kan du ha medlemmar, tex metoder:
\begin{Code}
enum Suit(val color: Color):
  def show(isConsoleColor: Boolean = true): String = 
    if isConsoleColor then color.consoleColor + toString + Console.RESET
    else toString

  case Spade   extends Suit(Color.Black)
  case Heart   extends Suit(Color.Red)
  case Club    extends Suit(Color.Black) 
  case Diamond extends Suit(Color.Red)
\end{Code}
\begin{REPLsmall}
scala> println(Suit.Club.show(isConsoleColor = false)) 
Club
\end{REPLsmall}
\end{Slide}

\begin{Slide}{Enum kan bli fullfjädrade case-klasser}
Vill du kunna göra mönster-matching på enum-värden så behövs parametrar på alternativen för att det ska bli motsvarande case-klasser: 
\begin{Code}
enum Veg:
  def taste: String
  case Tomato(taste: String)
  case Banana(taste: String)
\end{Code}
Ovan expanderas automatiskt av kompilatorn till motsvarande detta:
\begin{Code}
sealed trait Veg:
  def taste: String
object Veg:
  case class Tomato(taste: String) extends Veg
  case class Banana(taste: String) extends Veg
\end{Code}
\end{Slide}

\begin{Slide}{Enum och mönster-matchning}
\SlideFontSmall
Med parametrar på varje fall och en abstrakt medlem för varje attribut... 
\begin{Code}
enum Veg:
  def taste: String
  case Tomato(taste: String)
  case Banana(taste: String)
\end{Code}
...så gör den automatiska expansionen till case-klasser att detta fungerar fint: 
\begin{REPLsmall}
scala> val v: Veg = Veg.Tomato("najs") // notera typen : Veg
val v: Veg = Tomato(najs)

scala> v.taste  // funkar eftersom Veg har en taste
val res0: String = najs

scala> v match { case Veg.Tomato(t) => t case _ => "bad!" }
val res1: String = najs
\end{REPLsmall}
Den abstrakta medlemmen \code{def taste: String} behövs för att attributet ska synas via referenser som är av den mindre specifika typen \code{Veg}.\\(Mer om abstrakta medlemmar i veckan om arv.)

\end{Slide}


\begin{Slide}{Fördelar med \texttt{\textbf{enum}} jämfört med uppräkning med heltal}
Varför inte bara så här?
\begin{Code}
val (Spade, Heart, Club, Diamond) = (0, 1, 2, 3)  
\end{Code}  
Alla element har samma specifika typ enligt \code{enum}-deklarationen:  
\begin{REPL}
scala> Suit.Heart              // alla element är av typen Suit 
val res5: Suit = Heart
\end{REPL}

\begin{itemize}
\item Detta är säkrare jämfört med att bara använda heltalsvärden: kompilatorn kan hjälpa dig att skilja på element av olika typ och ge felmeddelande om du använder fel typ oavsiktligt. 
\item Ej tillåtna värden kan inte representeras (jmf alla möjliga heltal, där bara några är relevanta).
\end{itemize}  
Detta får du prova på veckans labb: först använda heltal sedan \code{enum}.
\end{Slide}

%!TEX encoding = UTF-8 Unicode
%!TEX root = ../lect-w07.tex

%%%

\Subsection{Registrering}

\begin{Slide}{Registrering}
\begin{itemize}
\item \Emph{Registrering} innefattar algoritmer för att kategorisera eller räkna antalet förekomster av element med vissa specifika egenskaper.

\item Exempel:
\\\vspace{0.5em}Utfallsfrekvens vid kast med en tärning 1000 gånger:

\vspace{1em}\begin{tabular}{r c l}
utfall & & antal \\ \hline
1 & $\rightarrow$ & 178 \\
2 & $\rightarrow$ & 187 \\
3 & $\rightarrow$ & 167 \\
4 & $\rightarrow$ & 148 \\
5 & $\rightarrow$ & 155 \\
6 & $\rightarrow$ & 165 \\
\end{tabular}
\end{itemize}
\end{Slide}

\begin{Slide}{Registrering av tärningskast i \code{Array}}
Vi låter plats 0 representera antalet ettor, plats 1 representerar antalet tvåor etc. \Alert{Övning}: implementera \code{???}
\begin{REPLnonum}
scala> def rollDice(): Int = ???  //dra slumptal 1-6

scala> val reg = ??? //skapa heltalsarray med 6 platser
reg: Array[Int] = Array(0, 0, 0, 0, 0, 0)

scala> for k <- 1 to 1000 do 
         ??? //kasta tärning, räkna ut rätt index 
         ??? //registrera

scala> for i <- 1 to 6 do println(s"$i: ${reg(i - 1)}")
1: 178
2: 187
3: 167
4: 148
5: 155
6: 165
\end{REPLnonum}
\end{Slide}


\begin{Slide}{Registrering av tärningskast i \code{Array}}
\Emph{Lösning}:\\~
\begin{REPLnonum}
scala> def rollDice() = scala.util.Random.nextInt(6) + 1

scala> val reg = new Array[Int](6)
reg: Array[Int] = Array(0, 0, 0, 0, 0, 0)

scala> for k <- 1 to 1000 do
         val i = rollDice() - 1 
         reg(i) = reg(i) + 1    // eller: reg(i) += 1

scala> for i <- 1 to 6 do println(s"$i: ${reg(i - 1)}")
1: 178
2: 187
3: 167
4: 148
5: 155
6: 165
\end{REPLnonum}
\end{Slide}

% \begin{Slide}{Registrering av tärningskast i \code{Map}, imperativ lösning}
% Vi registrerar antalet i en Map[Int, Int] där nyckeln är antalet tärningsögon och värdet är frekvensen.
% \begin{REPLnonum}
% scala> val rnd = new java.util.Random(42L)
% rnd: java.util.Random = java.util.Random@6d946eee
%
% scala> var reg = (1 to 6).map(i => i -> 0).toMap
% reg: scala.collection.immutable.Map[Int,Int] =
%   Map(5 -> 0, 1 -> 0, 6 -> 0, 2 -> 0, 3 -> 0, 4 -> 0)
%
% scala> for i <- 1 to 1000 do 
%          val t = rnd.nextInt(6) + 1
%          reg = reg + ((t, reg(t) + 1))
%
% scala> reg
% res0:scala.collection.immutable.Map[Int,Int]= Map(5 -> 155,
% 1 -> 178, 6 -> 165, 2 -> 187, 3 -> 167, 4 -> 148)
% \end{REPLnonum}
% \end{Slide}
%
% \begin{Slide}{Registrering av tärningskast i \code{collection.mutable.Map}, imperativ lösning}
% Om vi är bekymrade över prestanda:
% \begin{REPL}
% scala> val rnd = new java.util.Random(42L)
% rnd: java.util.Random = java.util.Random@6d946eee
%
% scala> val initPairs = (1 to 6).map(i => i -> 0)
% initPairs: scala.collection.immutable.IndexedSeq[(Int, Int)] =
% Vector((1,0), (2,0), (3,0), (4,0), (5,0), (6,0))
%
% scala> var reg = scala.collection.mutable.Map(initPairs: _*)
%
% scala> for i <- 1 to 1000 do {
%          val t = rnd.nextInt(6) + 1
%          reg(t) = reg(t) + 1
%
% scala> reg
% res0: scala.collection.mutable.Map[Int,Int] =
% Map(2 -> 187, 5 -> 155, 4 -> 148, 1 -> 178, 3 -> 167, 6 -> 165)
%
% \end{REPL}
% \end{Slide}
%
% \begin{Slide}{Registrering av tärningskast i \code{Map}, funktionell lösning}
% Oföränderlighet: Skapa nya samlingar utan att ändra något.
% \begin{REPLnonum}
% scala> val rnd = new java.util.Random(42L)
% rnd: java.util.Random = java.util.Random@6d946eee
%
% scala> val dice = (1 to 1000).map(i => rnd.nextInt(6) + 1)
%
% scala> dice.groupBy(i => i).map(p => p._1 -> p._2.size)
% res0:scala.collection.immutable.Map[Int,Int]= Map(5 -> 155,
% 1 -> 178, 6 -> 165, 2 -> 187, 3 -> 167, 4 -> 148)
% \end{REPLnonum}
% Övn. för den nyfikne: mät prestanda för de olika lösningarna.
% \end{Slide}

% \ifkompendium\else
% \begin{SlideExtra}{Syresättning av hjärnan med registreringslek}\SlideFontTiny

% \vspace{-0.65em}\scalainputlisting[numbers=left,numberstyle=,basicstyle=\SlideFontSize{6}{8}\ttfamily\selectfont]{../compendium/examples/sequences/RegisterToggleWinner.scala}
% Övning: Vad gör koden?
% % \begin{Code}
% % object StandUpSleepyBrain {
% %   def randomChar: Char = (scala.util.Random.nextInt('Z' - 'A' + 1) + 'A').toChar
% %   val reg = new Array[Int]('Z' - 'A' + 1)
% %   def tab: Seq[(Char, Int)] = reg.indices.map(i => ((i + 'A').toChar, reg(i)))
% %   def winner = "\n ** Toggle Winner: **" + (reg.indexOf(reg.max)+'A').toChar
% %   def report = "Registreringsrapport:\n" + tab.mkString("\n") + winner
% %
% %   def toggle(n: Int = 10): Unit = {
% %     println(s"Toggla (sitt upp/sitt ner) om ditt förnamn börjar på: ")
% %     var ready = false
% %     while(!ready){
% %       val chars = Vector.fill(n)(randomChar).distinct.sorted
% %       println("\n" + chars.mkString(" "))
% %       chars.foreach(ch => reg(ch - 'A') += 1)
% %       ready = scala.io.StdIn.readLine("\n").length > 0
% %     }
% %     println(report)
% %   }
% % }
% % \end{Code}
% \vspace{-0.5em}Kör koden och lyssna på: \href{https://youtu.be/ZVrgj3A0_BY}{https://youtu.be/ZVrgj3A0\_BY}
% \end{SlideExtra}
% \fi
%!TEX encoding = UTF-8 Unicode
%!TEX root = ../lect-w07.tex

\Subsection{Skapa lösningar på sekvensproblem från grunden}

\begin{Slide}{Skapa lösningar på sekvensproblem från grunden}
  \begin{itemize}
    \item Normalt använder man färdiga samlingsmetoder
    \item Det finns ofta en färdig metod som gör det man vill
    \item Annars kan man ofta göra det man vill genom att kombinera flera färdiga samlingsmetoder
    \item[] \pause
    \item Vi ska nu i lärosyfte implementera några egna varianter av uppdatering från grunden.  
  \end{itemize}

{\SlideFontSmall  För problem av typen KUTFSSR ingår det i kursen att kunna 1) lösa dessa med färdiga samlingsmetoder, och 2) implementera egna lösningar med hjälp av sekvens, alternativ, repetition, abstraktion (\textbf{SARA}).
}
\end{Slide}

\begin{Slide}{Skapa ny sekvenssamling eller ändra på plats?}
Två olika principer vid sekvensalgoritmkonstruktion:
\begin{itemize}
\item Skapa \Emph{ny sekvens} utan att förändra insekvensen
\item Ändra \Emph{på plats} \Eng{in-place} i \Alert{förändringsbar} sekvens
\end{itemize}
\pause
\vspace{1em}
Välja mellan att skapa ny sekvens eller ändra på plats?
\begin{itemize}
\item Ofta är det \Emph{lättast att skapa ny samling} och kopiera över elementen efter eventuella förändringar medan man loopar.
\item Om man har mycket stora samlingar kan man behöva ändra på plats för att spara tid/minne.
\end{itemize}
\end{Slide}

\begin{Slide}{Algoritm: SEQ-COPY}
\Emph{Pseudokod} för algoritmen SEQ-COPY som kopierar en sekvens, här en Array med heltal:\\
\noindent\hrulefill
\begin{algorithm}[H]
\SetKwInOut{Input}{Indata}\SetKwInOut{Output}{Utdata}
\Input{Heltalsarray $xs$}
\Output{En ny heltalsarray som är en kopia av $xs$. \\ \vspace{1em}}
$result \leftarrow$ en ny array med plats för $xs.length$ element\\
$i \leftarrow 0$  \\
\While{$i < xs.length$}{
$result(i) \leftarrow xs(i)$\\
$i \leftarrow i + 1$\\
}
$result$
\end{algorithm}
\noindent\hrulefill
\end{Slide}

\begin{Slide}{Implementation av SEQ-COPY med \texttt{while}}
\lstinputlisting[numbers=left]{../compendium/examples/workspace/w05-seqalg/src/seqCopy.scala}
\code{xs.sameElements(ys)} behövs då \code{==} på en \code{Array} ger referenslikhet.
\end{Slide}

% \begin{Slide}{Implementation av SEQ-COPY med \texttt{for}}
% \lstinputlisting[numbers=left]{../compendium/examples/workspace/w05-seqalg/src/seqCopyFor.scala}
% \end{Slide}
%
% \begin{Slide}{Implementation av SEQ-COPY med \texttt{for-yield}}
% \lstinputlisting[numbers=left]{../compendium/examples/workspace/w05-seqalg/src/seqCopyForYield.scala}
% \end{Slide}

%!TEX encoding = UTF-8 Unicode
%!TEX root = ../lect-w07.tex

%%%


\Subsection{Implementera insert, remove, append}


\begin{Slide}{Typ-alias för att abstrahera typnamn}\SlideFontSmall
Med hjälp av nyckelordet \code{type} kan man deklarera ett \Emph{typ-alias} för att ge ett \Alert{alternativt} namn till en viss typ. Exempel:
\begin{REPL}
scala> type Pt = (Int, Int)            // typalias
scala> type Pts = Vector[Pt]           // nästlad typalias

scala> def distToOrigo(pt: Pt): Double = math.hypot(pt._1, pt._2)

scala> val xs: Pts = Vector((1,1), (2,2), (3,4))
val xs: Pts = Vector((1,1), (2,2), (3,4))

scala> xs.head
val res0: Pt = (1,1)

scala> xs.map(distToOrigo)                                                                  
val res1: Vector[Double] = Vector(1.4142135623730951, 2.8284271247461903, 5.0)
\end{REPL}

Typ-alias kan vara bra när:
\begin{itemize}
\item man har en lång och krånglig typ och vill använda ett kortare namn,

\item man vill kunna lätt byta implementation senare\\(t.ex. om man vill använda en case-klass i stället för en tupel).
\end{itemize}
\end{Slide}


\begin{Slide}{Exempel: SEQ-INSERT/REMOVE-COPY}
Nu ska vi ''uppfinna hjulet'' och som träning implementera \Emph{insättning} och \Emph{borttagning} till en \Alert{ny} sekvens utan användning av sekvenssamlingsmetoder (förutom \code{length} och \code{apply}):
\begin{Code}
object PointSeqUtils:
  type Pt = (Int, Int)  // a type alias to make the code more concise

  def primitiveInsertCopy(pts: Array[Pt], pos: Int, pt: Pt): Array[Pt] = ???

  def primitiveRemoveCopy(pts: Array[Pt], pos: Int): Array[Pt] = ???
\end{Code}
\end{Slide}




\begin{Slide}{Pseudo-kod för SEQ-INSERT-COPY}\SlideFontSmall
\begin{algorithm}[H]
 \SetKwInOut{Input}{Indata}\SetKwInOut{Output}{Resultat}

 \Input{\texttt{pts: Array[Pt], pt: Pt, pos: Int}} ~\\

 \Output{En kopia av $pts$ men där $pt$ är infogat på plats $pos$}~\\


 \noindent\hrulefill

  $result \leftarrow$ en ny \texttt{Array[Pt]} med plats för $pts.length + 1$ element \\
 \For{$i \leftarrow 0$ \KwTo $pos - 1$}{
  $result(i) \leftarrow pts(i)$
 }
 $result(pos) \leftarrow pt$ \\
 \For{$i \leftarrow pos + 1$ \KwTo $xs.length$}{
  $result(i) \leftarrow xs(i - 1)$
 }

 \Return $result$

  \noindent\hrulefill\\
\end{algorithm}
\pause\vspace{0.5em}\Emph{Övning}: Skriv pseudo-kod för SEQ-REMOVE-COPY
\end{Slide}

\begin{Slide}{Insättning/borttagning i kopia av primitiv Array}
\vspace{-0.6em}\scalainputlisting[numbers=left,numberstyle=,basicstyle=\fontsize{6}{7.2}\ttfamily\selectfont]{../compendium/examples/sequences/PointSeqUtils.scala}

\pause
\SlideFontSmall Man gör \Alert{mycket lätt fel} på gränser/specialfall: +-1, to/until, tom sekvens etc.
\end{Slide}

% \begin{Slide}{Exempel: Test av SEQ-INSERT/REMOVE-COPY}
% \vspace{-0.6em}\scalainputlisting[numbers=left,numberstyle=,basicstyle=\fontsize{6.5}{8}\ttfamily\selectfont]{../compendium/examples/workspace/w05-seqalg/src/polygonTest2.scala}
% \end{Slide}

% \begin{Slide}{Exempel: Göra insättning med take/drop}\SlideFontSmall
% Om du inte vill ''uppfinna hjulet'' och inte använda \code{patch} kan du göra så här: \\Använd \code{take} och \code{drop} tillsammans med \code{:+} och \code{++} \\Du kan också göra insättningen generiskt användbar för alla sekvenser:
% \begin{REPLnonum}
% scala> val xs = Vector(1,2,3)
% xs: scala.collection.immutable.Vector[Int] =
%   Vector(1, 2, 3)
%
% scala> val ys = (xs.take(2) :+ 42) ++ xs.drop(2)
% ys: scala.collection.immutable.Vector[Int] =
%   Vector(1, 2, 42, 3)
%
% scala> def insertCopy[T](xs: Seq[T], elem: T, pos: Int) =
%         (xs.take(pos) :+ elem) ++ xs.drop(pos)
%
% scala> insertCopy(xs, 42, 2)
% res0: Seq[Int] = Vector(1, 2, 42, 3)
%
% \end{REPLnonum}
% \Emph{Övning}: Implementera \code{insertCopy[T]} med \code{patch} istället.
% \end{Slide}

%!TEX encoding = UTF-8 Unicode
%!TEX root = ../lect-w07.tex

%%%

\Subsection{Exempel: Polygon}


\begin{Slide}{Exempel: PolygonWindow}\SlideFontTiny
\setlength{\leftmargini}{0pt}
\begin{itemize}
\item En polygon kan representeras som en punktsekvens, där varje punkt är ett heltalspar.

\item \code{PolygonWindow} nedan är ett fönster som kan rita en polygon.
\end{itemize}

\vspace{-0.5em}\scalainputlisting[numbers=left,numberstyle=,basicstyle=\fontsize{6.5}{8}\ttfamily\selectfont]{../compendium/examples/sequences/PolygonWindow.scala}
\pause
\vspace{-0em}\scalainputlisting[numbers=left,numberstyle=,basicstyle=\fontsize{6.5}{8}\ttfamily\selectfont]{../compendium/examples/sequences/PolygonTest.scala}
\end{Slide}



\begin{Slide}{Implementera Polygon}
\begin{itemize}
\item En polygon kan representeras som en sekvens av punkter.
\item Vi vill kunna lägga till punkter, samt ta bort punkter.
\item En polygon kan implementeras på många olika sätt:
\pause
\begin{itemize}
\item \Alert{Förändringsbar} \Eng{mutable}
\begin{itemize}
\item Med punkterna i en \Alert{\texttt{Array}}
\item Med punkterna i en \Alert{\texttt{ArrayBuffer}}
\item Med punkterna i en \Alert{\texttt{ListBuffer}}
\item Med punkterna i en \Emph{\texttt{Vector}}
\item Med punkterna i en \Emph{\texttt{List}}
\end{itemize}
\item \Emph{Oföränderlig} \Eng{immutable}
\begin{itemize}
\item Som en case-klass med en oföränderlig \Emph{\texttt{Vector}} som returnerar nytt objekt vid uppdatering. Vi kan låta datastrukturen vara \Emph{publik} eftersom allt är oföränderligt.
\item Som en ''vanlig'' klass med någon lämplig \Alert{privat} datastruktur där vi \Alert{inte} möjliggör förändring av efter initialisering och där vi returnerar nytt objekt vid uppdatering.
\end{itemize}
\end{itemize}
\end{itemize}
\pause
Val av implementation \Alert{beror på} sammanhang \& användning!
\end{Slide}




\begin{Slide}{Exempel: PolygonArray, ändring på plats}
\vspace{-0.6em}\scalainputlisting[numbers=left,numberstyle=,basicstyle=\fontsize{6.5}{7.7}\ttfamily\selectfont]{../compendium/examples/sequences/PolygonArray.scala}
\end{Slide}

% \begin{Slide}{Test av PolygonArray, ändring på plats}
% \vspace{0em}\scalainputlisting[numbers=left,numberstyle=,basicstyle=\fontsize{6.5}{8}\ttfamily\selectfont]{../compendium/examples/workspace/w05-seqalg/src/polygonTest3.scala}
% \end{Slide}


\begin{Slide}{Exempel: PolygonVector, variabel referens till oföränderlig datastruktur}
\vspace{-0.6em}\scalainputlisting[numbers=left,numberstyle=,basicstyle=\fontsize{6.5}{7.7}\ttfamily\selectfont]{../compendium/examples/workspace/w05-seqalg/src/PolygonVector.scala}
\end{Slide}

% \begin{Slide}{Test av PolygonVector, variabel referens till oföränderlig datastruktur}
% \vspace{0em}\scalainputlisting[numbers=left,numberstyle=,basicstyle=\fontsize{6.5}{8}\ttfamily\selectfont]{../compendium/examples/workspace/w05-seqalg/src/polygonTest4.scala}
% \end{Slide}


\begin{Slide}{Exempel: Polygon som oföränderlig case class}
\vspace{-0.6em}\scalainputlisting[numbers=left,numberstyle=,basicstyle=\fontsize{6.5}{7.7}\ttfamily\selectfont]{../compendium/examples/sequences/Polygon.scala}
% \begin{itemize}\SlideFontTiny
% \item Nu är attributet points en publik \code{val} som vi kan dela med oss av eftersom datastrukturen \code{Vector} är oföränderlig.
%
% \item Vi behöver inte införa ett beroende till \code{PolygonWindow} här då vi ger tillgång till sekvensen av punkter som kan användas vid anrop av \code{PolygonWindow.draw}
%
% \item Att ändra implementationen till något annat än \code{Vector} blir lätt om klientkoden använder typ-alias \code{Polygon.Pts} i stället för \code{Vector[(Int, Int)]}.
% \end{itemize}
\end{Slide}

% \begin{Slide}{Test av Polygon som oföränderlig case class}
% \vspace{0em}\scalainputlisting[numbers=left,numberstyle=,basicstyle=\fontsize{6.5}{8}\ttfamily\selectfont]{../compendium/examples/workspace/w05-seqalg/src/polygonTest5.scala}
% \end{Slide}

%!TEX encoding = UTF-8 Unicode
%!TEX root = ../lect-w07.tex

%%%

\Subsection{Jämföra strängar}

\begin{Slide}{Att sortera och jämföra strängar lexikografiskt}\SlideFontSmall
Teckenstandard \href{https://sv.wikipedia.org/wiki/UTF-8}{UTF-8}: Alla stora bokstäver är \href{https://www.youtube.com/watch?v=MijmeoH9LT4}{''mindre''} än alla små:
\begin{REPLnonum}
scala> Array("hej","Hej","gurka").sorted
\end{REPLnonum}
\pause\vspace{-1.2em}
\begin{REPLnonum}
res0: Array[String] = Array(Hej, gurka, hej)\end{REPLnonum}
\pause
\begin{itemize}
\item Antag att vi vill lösa detta problem ''från scratch'': \\ \Emph{att sortera en sekvens med strängar}
\item Följdfrågor:
\begin{itemize}\SlideFontTiny
 \item Vad betyder det att två strängar är ''lika''?
\item Vad betyder det att en sträng är ''mindre'' än en annan?
\end{itemize}
\item För att sortera en strängsekvens behöver vi lösa dessa delproblemen:
\begin{itemize}\SlideFontTiny
\item \Emph{att} \Alert{jämföra strängar}
\item \Emph{sökning i sekvenser}
\item \Emph{SWAP} (om på-plats-sortering i förändringsbar sekvens)
\end{itemize}
\end{itemize}
\pause {\SlideFontTiny Vi använder här strängjämförelse, sökning och sortering för att illustrera typiska \Emph{imperativa algoritmer}. \Alert{Normalt} använder man \Emph{färdiga lösningar} på dessa problem!}

\end{Slide}

\begin{Slide}{Jämföra strängar: likhet}\SlideFontSmall
Antag att vi inte kan göra \code{s1 == s2} utan bara kan jämföra strängar tecken för tecken,
t.ex. så här: \code{s1(i) == s2(i)}. Antag också att vi inte har tillgång till annat än metoderna \code{length} och \code{apply} på strängar, samt  \code{while} och variabler av grundtyp. \Emph{Lös problemet att \emph{avgöra om två strängar är lika}.}

\pause
\begin{itemize}
\item Indata: två strängar
\item Utdata: \code{true} om lika annars \code{false}
\end{itemize}
\begin{enumerate}
\item Klura ut din lösningsidé
\item Formulera algoritmen i pseudokod
\item Implementera algoritmen i Scala: \\\code{def isEqual(s1: String, s2: String): Boolean} = ???
\end{enumerate}
\end{Slide}

\begin{Slide}{Algoritmexempel: stränglikhet, pseudokod}
\begin{Code}
def isEqual(s1: String, s2: String): Boolean = 
  if (/* lika längder */) then
    var foundDiff = false
    var i = /* första index */
    while !foundDiff && /* i inom indexgräns */ do
      if /* tecken på plats i är olika */ then foundDiff = true
      else i = /* nästa index */
    end while
    !foundDiff
  else false
end isEqual
\end{Code}

\pause\noindent Detta är en variant av s.k. \Emph{linjärsökning} där vi söker från början i en sekvens till vi hittar det vi söker efter (här söker vi efter tecken som skiljer sig åt).
\\\pause\vspace{1em}

\noindent Hur ser implementationen i exekverbar Scala ut?
\end{Slide}

\begin{Slide}{Algoritmexempel: stränglikhet, implementation}\SlideFontSmall
\begin{Code}
def isEqual(s1: String, s2: String): Boolean = 
  if s1.length == s2.length then
    var foundDiff = false
    var i = 0
    while !foundDiff && i < s1.length do
      if s1(i) != s2(i) then foundDiff = true
      else i += 1
    end while
    !foundDiff
  else false
end isEqual
\end{Code}

% \pause
% {\SlideFontTiny \emph{Fördjupning:} Jämför ovan med implementationen av \code{String.equals} här:\\
% \href{http://hg.openjdk.java.net/jdk8u/jdk8u60/jdk/file/935758609767/src/share/classes/java/lang/String.java#l976}{hg.openjdk.java.net/jdk8u/jdk8u60/jdk/file/935758609767/src/share/classes/java} \\ och använd \code{timed} nedan och jämför prestanda med \code{isEqual} ovan.\\
% Obs! Mät efter flera körningar då JVM optimerar bytekoden efter ett tag (s.k. ''uppvärmning'').}

% \vspace{-0.25em}\begin{Code}
% def timed[T](block: => T): (Double, T) = {
%   val (t, res) = (System.nanoTime, block)
%   ((System.nanoTime - t) / 1e9, res)
% }
% \end{Code}

\end{Slide}
 
% \begin{Slide}{Algoritmexempel: stränglikhet, prestanda}
% \begin{REPL}
% scala> val enMiljon = 1000000

% scala> val s = Array.fill(enMiljon)('x').mkString

% scala> val t = s.updated(enMiljon - 1, 'y')

% scala> timed { s == t }
% res42: (Double, Boolean) = (3.76459E-4,false)

% scala> timed { isEqual(s,t) }
% res43: (Double, Boolean) = (3.31597E-4,false)
% \end{REPL}
% Ovan är kört efter ''uppvärmning'' på i7-4790K CPU @ 4.00GHz \\
% Skillnaden inom mätfelmarginalen!
% \end{Slide}



\begin{Slide}{Jämföra strängar: ''mindre än''}\SlideFontSmall
Med \code{s1 < s2} menar vi att strängen s1 ska sorteras före strängen s2 enligt hur de enskilda tecknen är ordnade med uttrycket \code{s1(i) < s2(i)}. \\
Antag också att vi inte har tillgång till annat än metoderna \code{length} och \code{apply} på strängar, samt  \code{while} och variabler av grundtyp, samt \code{math.min}
\\\Emph{Lös problemet att \emph{avgöra om en sträng är ''mindre'' än en annan}.}\\
\begin{itemize}
\item Indata: två strängar, s1, s2
\item Utdata: \code{true} om s1 ska sorteras före s2 annars \code{false}
\end{itemize}
\begin{enumerate}
\item Klura ut din lösningsidé
\item Formulera algoritmen i pseudokod
\item Implementera algoritmen i Scala: \\\code{def isLessThan(s1: String, s2: String): Boolean} = ???
\end{enumerate}
\end{Slide}

\begin{Slide}{Jämföra strängar: ''mindre än''}\SlideFontSmall
Pseudokod:
\begin{Code}
def isLessThan(s1: String, s2: String): Boolean = 
  val minLength = /* minimum av längderna på s1 och s2 */

  def firstDiff(s1: String, s2: String): Int =
    /* index för första skillnaden (om de börjar lika: minLength) */

  val diffIndex = firstDiff(s1, s2)
  if diffIndex == minLength then /* s1 är kortare än s2 */
  else /* tecknet s1(diffIndex) är mindre än tecknet s2(diffIndex) */
\end{Code}
\end{Slide}

\begin{Slide}{Jämföra strängar: ''mindre än''}\SlideFontSmall
\begin{Code}
def isLessThan(s1: String, s2: String): Boolean = 
  val minLength = math.min(s1.length, s2.length)

  def firstDiff(s1: String, s2: String): Int = 
    var foundDiff = false
    var i = 0
    while !foundDiff && i < minLength do
      if (s1(i) != s2(i)) foundDiff = true
      else i += 1
    end while
    i
  end firstDiff  

  val diffIndex = firstDiff(s1, s2)
  if diffIndex == minLength then s1.length < s2.length
  else s1(diffIndex) < s2(diffIndex)
end isLessThan
\end{Code}
\end{Slide}


%!TEX encoding = UTF-8 Unicode
%!TEX root = ../lect-w07.tex

%%%

\Subsection{Sökning och sortering}


\begin{Slide}{Sökning}\SlideFontSmall

\begin{itemize}
\item \Emph{Sökning} återkommer i många skepnader: \\ i en datastruktur, vilken det än må vara, vill man ofta kunna \\ \Emph{hitta ett element med en viss egenskap}.

\pause
Några färdiga linjärsökningar i Scalas standardbibliotek:

\begin{REPL}
scala> Vector("gurka","tomat","broccoli").indexOf("tomat")
res0: Int = 1

scala> Vector("gurka","tomat","broccoli").indexWhere(_.contains("o"))
res1: Int = 1

scala> Vector("gurka","tomat","broccoli").find(_.contains("o"))
res2: Option[String] = Some(tomat)
\end{REPL}

\pause
\item Sökning efter ett visst index i en sekvens:

\begin{itemize}\SlideFontTiny
\item Indata: en sekvens och ett \Emph{sökkriterium}
\item Utdata: index för första eftersökta element, annars -1
\end{itemize}

\pause
\item Två typiska varianter av sökning i en sekvens:
\begin{itemize}\SlideFontTiny
\item Linjärsökning: börja från början och sök tills ett eftersökt element är funnet
\item Binärsökning: antag sorterad sekvensen; börja i mitten, välj rätt halva ... 
\end{itemize}
\end{itemize}
\end{Slide}

%\Subsection{Linjärsökning}

% \begin{Slide}{Linjärsökning: hitta index för elementet 42}
% Skriv pseudokod för:\\ \code{def indexOf42(xs: Vector[Int]): Int = ???}
% \pause
% \begin{Code}
% def indexOf42(xs: Vector[Int]): Int = {
%   var i = /* index för första elementet */
%   var found = false
%   while (!found && /* index inom indexgräns */) {
%     if (/* element på plats i är 42 */) found = true
%     else i = /* nästa index */
%   }
%   if (/* hittat */) i else -1
% }
% \end{Code}
% \end{Slide}
%
% \begin{Slide}{Linjärsökning: hitta index för elementet 42}
% Implementera:\\ \code{def indexOf42(xs: Vector[Int]): Int = ???}
% \pause
% \begin{Code}
% def indexOf42(xs: Vector[Int]): Int = {
%   var i = 0
%   var found = false
%   while (!found && i < xs.length) {
%     if (xs(i) == 42) found = true
%     else i += 1
%   }
%   if (found) i else -1
% }
% \end{Code}
% \end{Slide}

\begin{Slide}{Linjärsökning: hitta index för elementet x}
Implementera \code{indexOf}:
\begin{Code}
def indexOf(xs: Vector[Int], x: Int): Int = ???
\end{Code}
Utdata: index \code{i} där \code{xs(i) == x}\\Om värde saknas. returnera \code{-1}
\pause
\begin{Code}
def indexOf(xs: Vector[Int], x: Int): Int = {
  var i = 0
  var found = false
  while (!found && i < xs.length) {
    if (xs(i) == x) found = true
    else i += 1
  }
  if (found) i else -1
}
\end{Code}
(Är du nyfiken på binärsökning, se fördjupningsveckan.)
\end{Slide}

% \begin{Slide}{Linjärsökning: hitta index för elementet p(x)}\SlideFontSmall
% Implementera:\\ \code{def indexWhere(xs: Vector[Int], p: Int => Boolean): Int = ???}
% \pause
% \begin{Code}
% def indexWhere(xs: Vector[Int], p: Int => Boolean): Int = {
%   var i = 0
%   var found = false
%   while (!found && i < xs.length) {
%     if (p(xs(i))) found = true
%     else i += 1
%   }
%   if (found) i else -1
% }
% \end{Code}
% \end{Slide}
%
% \begin{Slide}{Linjärsökning: generalisera till godtycklig typ}\SlideFontSmall
% Implementera:\\ \code{def indexWhere[T](xs: Vector[T], p: T => Boolean): Int = ???}
% \pause
% \begin{Code}
% def indexWhere[T](xs: Vector[T], p: T => Boolean): Int = {
%   var i = 0
%   var found = false
%   while (!found && i < xs.length) {
%     if (p(xs(i))) found = true
%     else i += 1
%   }
%   if (found) i else -1
% }
% \end{Code}
% \end{Slide}
%
% \begin{Slide}{Linjärsökning: generalisera till godtycklig typ}\SlideFontSmall
% Typinferensen fungerar bättre om stegad funktion:\\
% \code{def indexWhere[T](xs: Vector[T])(p: T => Boolean): Int}
% \begin{Code}
% def indexWhere[T](xs: Vector[T])(p: T => Boolean): Int = {
%   var i = 0
%   var found = false
%   while (!found && i < xs.length) {
%     if (p(xs(i))) found = true
%     else i += 1
%   }
%   if (found) i else -1
% }
% \end{Code}
% \end{Slide}
%
%
% \begin{Slide}{Linjärsökning: generalisera till godtycklig sekvens}\SlideFontSmall
% Implementera:\\ \code{def indexWhere[T](xs: Seq[T])(p: T => Boolean): Int = ???}
% \pause
% \begin{Code}
% def indexWhere[T](xs: Seq[T])(p: T => Boolean): Int = {
%   var i = 0
%   var found = false
%   while (!found && i < xs.length) {
%     if (p(xs(i))) found = true
%     else i += 1
%   }
%   if (found) i else -1
% }
% \end{Code}
% \end{Slide}
%
% \begin{Slide}{Linjärsökning: generalisera till godtycklig sekvens}\SlideFontSmall
% Implementera:\\ \code{def find[T](xs: Seq[T])(p: T => Boolean): Option[T] = ???}
% \pause
% \begin{Code}
% def find[T](xs: Seq[T])(p: T => Boolean): Option[T] = {
%   var i = 0
%   var found = false
%   while (!found && i < xs.length) {
%     if (p(xs(i))) found = true
%     else i += 1
%   }
%   if (found) Some(xs(i)) else None
% }
% \end{Code}
% \end{Slide}

\begin{Slide}{Sortering}
\Emph{Problem}: Vi har en osorterad sekvens med heltal. Vi vill ordna denna osorterade sekvens i en sorterad sekvens från minst till störst.
\pause

\vspace{1em}\noindent
En \emph{generalisering} av problement: \\ \vspace{1em}

\noindent Vi har många element av godtycklig typ och en \Emph{ordningsrelation} som säger vad vi menar med att ett element är \emph{mindre än} eller \emph{större än eller lika med} ett annat element. \\ 

\vspace{1em}\noindent Vi vill lösa problemet att ordna elementen i sekvens så att för varje element på plats $i$ så är efterföljande element på plats $i + 1$ större eller lika med elementet på plats $i$.

% \end{Slide}

% \begin{Slide}{Insättningssortering \& Urvalssortering}
\begin{itemize}
\item Insättningssortering \Emph{lösningsidé}: Ta ett element i taget från den osorterade listan och \Alert{sätt in} det på \Alert{rätt plats} i den sorterade listan och upprepa till det inte finns fler osorterade element.
% \pause
% \item Urvalsssortering \Emph{lösningsidé}: \Alert{Välj ut} det minsta kvarvarande elementet i den osorterade listan och placera det \Alert{sist} i den sorterade listan och upprepa till det inte finns fler osorterade element.
\end{itemize}
\end{Slide}

\begin{Slide}{Det finns många olika sorteringsalgoritmer}
\begin{itemize}
\item Visualisering av 15 olika sorteringsalgoritmer på 6 min:\\{\SlideFontSmall\url{https://www.youtube.com/watch?v=kPRA0W1kECg}}
\item Olika sorteringsalgoritmer har olika tids- \& minneskomplexitet: i bästa fall, i värsta fall, i medeltal, för nästan sorterad, etc.
\\{\SlideFontSmall\url{https://en.wikipedia.org/wiki/Sorting_algorithm}}
\item Olika sorteringsalgoritmer lämpar sig olika väl för parallellisering på många kärnor.
\end{itemize}
\end{Slide}

\begin{Slide}{Bogo sort}
\begin{Code}
def bogoSort(xs: Vector[Int]) = 
  var result = xs
  while result != result.sorted do
    result = scala.util.Random.shuffle(result)
  end while
  result
}
\end{Code}
När blir denna färdig? \pause \\~\\
Antal jämförelser i medeltal vid $n$ element: $ n \cdot n!$ \\~\\
\url{https://en.wikipedia.org/wiki/Bogosort}

\end{Slide}


\begin{Slide}{Sortera till ny vektor med insättningssortering: pseudo-kod}

{\SlideFontSmall Det är nog lättare att förstå \Emph{insertion sort} om man sorterar till en ny vektor. \\ Vi ska sedan se hur man sorterar ''på plats'' \Eng{in place} i en  array.\\} \vspace{0.5em}

\noindent \Emph{Indata}: en osorterad vektor med heltal \\
\Emph{Utdata}: en ny, sorterad vektor med heltal
\begin{Code}
def insertionSort(xs: Vector[Int]): Vector[Int] = 
  val sorted = /* tom ArrayBuffer */
  for /* alla element i xs */ do
     /* linjärsök rätt position i sorted */
     /* sätt in element på rätt plats i sorted */
  end for
  sorted.toVector
\end{Code}
\end{Slide}


\begin{Slide}{Sortera till ny vektor med insättningssortering: implementation} % i Scala}
\begin{Code}
def insertionSort(xs: Vector[Int]): Vector[Int] = 
  val sorted = scala.collection.mutable.ArrayBuffer.empty[Int]
  for elem <- xs do
     // linjärsök rätt position i sorted:
     var pos = 0
     while pos < sorted.length && sorted(pos) < elem do
       pos += 1
     end while
     // sätt in element på rätt plats i sorted:
     sorted.insert(pos, elem)
  end for
  sorted.toVector
end insertionSort
\end{Code}
\end{Slide}

\begin{Slide}{Sorter till ny samling med godtyckligt ordningspredikat}
\begin{CodeSmall}
def sortWith(xs: Vector[Int])(lt: (Int, Int) => Boolean ): Vector[Int] = 
  val sorted = scala.collection.mutable.ArrayBuffer.empty[Int]
  for elem <- xs do  // insertion sort using lt as "less than"
     var pos = 0
     while pos < sorted.length && lt(sorted(pos), elem) do
       pos += 1
     end while
     sorted.insert(pos, elem)
  end for
  sorted.toVector
end sortWith
\end{CodeSmall}
\pause
\begin{REPL}
scala> val xs = Vector(1,2,1,2,12,42,1)

scala> sortWith(xs)(_ < _)
val res0: Vector[Int] = Vector(1, 1, 1, 2, 2, 12, 42)

scala> sortWith(xs)(_ > _)
val res1: Vector[Int] = Vector(42, 12, 2, 2, 1, 1, 1)
\end{REPL}
\end{Slide}


\begin{Slide}{Insättningssortering på plats -- pseudo-kod}
\Emph{Indata:} en array med heltal\\
\Emph{Utdata:} samma array, men nu sorterad\\
\begin{Code}
def insertionSortInPlace(xs: Array[Int]): Unit = 
  for i <- 1 until xs.length do  //från ANDRA till sista
    var j = i
    while j > 0 && xs(j - 1) > xs(j) do
      /* byt plats på xs(j) och xs(j - 1) */
      j -= 1;  // stega bakåt
\end{Code}
\pause
Se animering här: \href{https://sv.wikipedia.org/wiki/Ins\%C3\%A4ttningssortering}{Insättningssortering på wikipedia}\\
Gå igenom alla specialfall och kolla så att detta fungerar!
\end{Slide}

\begin{Slide}{Insättningssortering på plats -- implementation} %, Scala}
\begin{Code}
def insertionSortInPlaceSwap(xs: Array[Int]): Unit = 
  def swap(i: Int, j: Int): Unit = 
    val temp = xs(i)
    xs(i) = xs(j)
    xs(j) = temp
  end swap 

  for i <- 1 until xs.length do  //från ANDRA till sista
    var j = i
    while j > 0 && xs(j - 1) > xs(j) do
      swap(j, j - 1)
      j -= 1;  // stega bakåt
    end while
  end for
end insertionSortInPlaceSwap
\end{Code}
\end{Slide}

\begin{Slide}{Checklista vid granskning av sekvensalgoritmer}
\begin{itemize}
\item Fungerar algoritmen för en tom sekvens?
\item Fungerar algoritmen för en sekvens med endast ett element?
\item Fungerar algoritmen för för osorterad sekvens med (minst) två element?
\item Vad händer om sekvensen redan är sorterad?
\end{itemize}
\end{Slide}
%!TEX encoding = UTF-8 Unicode
%!TEX root = ../lect-w07.tex

\ifkompendium\else

\Subsection{Uppgifter denna vecka}

\begin{Slide}{Denna veckas övning: \texttt{sequences}}
\begin{itemize}\SlideFontTiny
%!TEX encoding = UTF-8 Unicode
%!TEX root = ../compendium2.tex

\item Kunna läsa och skriva pseudokod för sekvensalgoritmer och implementera sekvensalgoritmer enligt pseudokod.

\item Kunna implementera sekvensalgoritmer, både genom kopiering till ny sekvens och genom förändring på plats i befintlig sekvens.

\item Kunna använda inbyggda metoder för uppdatering av, linjärsökning i, och sortering av sekvenssamlingar.

\item Kunna beskriva skillnaden i användningen av föränderliga och oföränderliga sekvenser, speciellt vid uppdatering.

\item Förstå hur sorteringsordningen är definierad för strängar.

\item Kunna sortera sekvenssamlingar innehållande objekt av grundtyper med hjälp av inbyggda och egendefinierade sorteringsordningar med metoderna \code{sorted}, \code{sortBy} och \code{sortWith}.

\item Kunna implementera linjärsökning enligt olika sökkriterier.


\item Kunna beskriva egenskaperna hos sekvenssamlingarna \code{Vector}, \code{List}, \code{Array}, \code{ArrayBuffer} och \code{ListBuffer}.

\item Förstå bieffekter av uppdatering av delade referenser till föränderliga element.

\item Kunna använda funktioner med repeterade parametrar.

\item Känna till hur man implementerar funktioner med repeterade parametrar.

\item Kunna implementera heltalsregistrering i en heltalsarray.

%\item Kunna beskriva skillnader i syntax mellan arrayer i Scala och Java.

%\item Kunna beskriva skillnader i syntax och semantik mellan enkla for-satser i Scala och Java.


%\item Känna till hur klassen \code{java.util.Scanner} kan användas för att skapa heltalssekvenser ur strängsekvenser.

\end{itemize}
\end{Slide}

\begin{Slide}{Denna veckas laboration: \texttt{shuffle}}
\begin{itemize}\SlideFontSmall
\input{../compendium/modules/w07-sequences-lab-goals.tex}
\end{itemize}
\end{Slide}
\fi



%%!TEX encoding = UTF-8 Unicode
\chapter{Arv}\label{chapter:W07}
Koncept du ska lära dig denna vecka:
\begin{multicols}{2}\begin{itemize}[nosep,label={$\square$},leftmargin=*]
\item arv
\item polymorfism
\item asInstanceOf
\item klasshierarkin i Scala: Any AnyRef Object AnyVal Nothing Null
\item referensklasser vs värdeklasser
\item klasshierarkin i Scalas samlingar
\item Shape som basklass till Point och Rectangle
\item accessregler vid arv
\item protected
\item final
\item abstrakt klass
\item trait
\item inmixning
\item klass vs trait
\item case-object
\item typer med uppräknade värden\end{itemize}\end{multicols}


%!TEX encoding = UTF-8 Unicode
%!TEX root = ../exercises.tex

\ifPreSolution


\Exercise{\ExeWeekSEVEN}\label{exe:W07}

\begin{Goals}
%!TEX encoding = UTF-8 Unicode
%!TEX root = ../compendium2.tex

\item Kunna läsa och skriva pseudokod för sekvensalgoritmer och implementera sekvensalgoritmer enligt pseudokod.

\item Kunna implementera sekvensalgoritmer, både genom kopiering till ny sekvens och genom förändring på plats i befintlig sekvens.

\item Kunna använda inbyggda metoder för uppdatering av, linjärsökning i, och sortering av sekvenssamlingar.

\item Kunna beskriva skillnaden i användningen av föränderliga och oföränderliga sekvenser, speciellt vid uppdatering.

\item Förstå hur sorteringsordningen är definierad för strängar.

\item Kunna sortera sekvenssamlingar innehållande objekt av grundtyper med hjälp av inbyggda och egendefinierade sorteringsordningar med metoderna \code{sorted}, \code{sortBy} och \code{sortWith}.

\item Kunna implementera linjärsökning enligt olika sökkriterier.


\item Kunna beskriva egenskaperna hos sekvenssamlingarna \code{Vector}, \code{List}, \code{Array}, \code{ArrayBuffer} och \code{ListBuffer}.

\item Förstå bieffekter av uppdatering av delade referenser till föränderliga element.

\item Kunna använda funktioner med repeterade parametrar.

\item Känna till hur man implementerar funktioner med repeterade parametrar.

\item Kunna implementera heltalsregistrering i en heltalsarray.

%\item Kunna beskriva skillnader i syntax mellan arrayer i Scala och Java.

%\item Kunna beskriva skillnader i syntax och semantik mellan enkla for-satser i Scala och Java.


%\item Känna till hur klassen \code{java.util.Scanner} kan användas för att skapa heltalssekvenser ur strängsekvenser.

\end{Goals}

\begin{Preparations}
\item \StudyTheory{07}
\end{Preparations}

\else

\ExerciseSolution{\ExeWeekSEVEN}

\fi


\BasicTasks %%%%%%%%%%%



\WHAT{Para ihop begrepp med beskrivning.}

\QUESTBEGIN

\Task \what

\vspace{1em}\noindent Koppla varje begrepp med den (förenklade) beskrivning som passar bäst:

\begin{ConceptConnections}
  mängd & 1 & & A & egenskapen att finnas kvar efter programmets avslut \\ 
  nyckel-värde-tabell & 2 & & B & unika identifierare, associerade med ett enda värde \\ 
  nyckelmängd & 3 & & C & unika element, kan snabbt se om element finns \\ 
  persistens & 4 & & D & koda objekt till avkodningsbar sekvens av symboler \\ 
  serialisera & 5 & & E & för att snabbt hitta tillhörande värde \\ 
  de-serialisera & 6 & & F & avkoda symbolsekvens och återskapa objekt i minnet \\ 
\end{ConceptConnections}

\SOLUTION

\TaskSolved \what

\begin{ConceptConnections}
  mängd & 1 & ~~\Large$\leadsto$~~ &  C & unika element, kan snabbt se om element finns \\ 
  nyckel-värde-tabell & 2 & ~~\Large$\leadsto$~~ &  E & för att snabbt hitta tillhörande värde \\ 
  nyckelmängd & 3 & ~~\Large$\leadsto$~~ &  B & unika identifierare, associerade med ett enda värde \\ 
  persistens & 4 & ~~\Large$\leadsto$~~ &  A & egenskapen att finnas kvar efter programmets avslut \\ 
  serialisera & 5 & ~~\Large$\leadsto$~~ &  D & koda objekt till avkodningsbar sekvens av symboler \\ 
  de-serialisera & 6 & ~~\Large$\leadsto$~~ &  F & avkoda symbolsekvens och återskapa objekt i minnet \\ 
\end{ConceptConnections}

\QUESTEND



\WHAT{Olika sekvenssamlingar.}

\QUESTBEGIN

\Task \what~Koppla varje sekvenssamling med den (förenklade) beskrivning som passar bäst:

\begin{ConceptConnections}
\input{generated/quiz-w07-seq-collections-taskrows-generated.tex}
\end{ConceptConnections}

\SOLUTION

\TaskSolved \what

\begin{ConceptConnections}
\input{generated/quiz-w07-seq-collections-solurows-generated.tex}
\end{ConceptConnections}

\QUESTEND



% This task has been removed because it didn't make much sense anymore after the removal of Traversable in Scala 2.13. https://github.com/lunduniversity/introprog/issues/497
%
%\WHAT{Typer i hierarkin av sekvenssamlingar.}
%
%\QUESTBEGIN
%
%\Task \what~Koppla varje typ i hierarkin av sekvenssamling %med den (förenklade) beskrivning som passar bäst:
%
%\begin{ConceptConnections}
%\input{generated/quiz-w07-abstract-collections-taskrows-generated.tex}
%\end{ConceptConnections}
%
%\SOLUTION
%
%\TaskSolved \what
%
%\begin{ConceptConnections}
%\input{generated/quiz-w07-abstract-collections-solurows-generated.tex}
%\end{ConceptConnections}
%
%\QUESTEND


\WHAT{Använda sekvenssamlingar.}

\QUESTBEGIN

\Task \what~Antag att nedan variabler finns synliga i aktuell namnrymd:
\begin{Code}
val xs: Vector[Int] = Vector(1, 2, 3)
val x: Int = 0
\end{Code}

\Subtask Koppla varje uttryck till vänster med motsvarande resultat till höger. Om du är osäker på resultatet, läs i snabbreferensen och testa i REPL. \\\emph{Tips: ''colon on the collection side''}.

\begin{ConceptConnections}
  \code|x +: xs         | & 1 & & A & \code|true                                    | \\ 
  \code|xs +: x         | & 2 & & B & \code|Vector(2, 2, 3)                         | \\ 
  \code|xs :+ x         | & 3 & & C & \code|1                                       | \\ 
  \code|xs ++ xs        | & 4 & & D & \code|value tail is not a member of Int       | \\ 
  \code|xs.indices      | & 5 & & E & \code|Range 0 until 3                         | \\ 
  \code|xs apply 0      | & 6 & & F & \code|Vector(1, 2, 3)                         | \\ 
  \code|xs(3)           | & 7 & & G & \code|Vector(0, 1, 2, 3)                      | \\ 
  \code|xs.length       | & 8 & & H & \code|false                                   | \\ 
  \code|xs.take(4)      | & 9 & & I & \code|java.lang.IndexOutOfBoundsException     | \\ 
  \code|xs.drop(2)      | & 10 & & J & \code|Vector(1, 2, 3, 0)                      | \\ 
  \code|xs.updated(0, 2)| & 11 & & K & \code|Vector(3)                               | \\ 
  \code|xs.tail.head    | & 12 & & L & \code|value +: is not a member of Int         | \\ 
  \code|xs.head.tail    | & 13 & & M & \code|Vector(1, 2, 3, 1, 2, 3)                | \\ 
  \code|xs.isEmpty      | & 14 & & N & \code|2                                       | \\ 
  \code|xs.nonEmpty     | & 15 & & O & \code|3                                       | \\ 
\end{ConceptConnections}

\Subtask Vid tre tillfällen blir det fel. Varför? Är det kompileringsfel eller exekveringsfel?

\begin{framed}
\noindent\emph{Tips inför fortsättningen:}
Scalas standardbibliotek har många användbara samlingar med enhetlig metoduppsättning. Om du lär dig de viktigaste samlingsmetoderna får du en kraftfull verktygslåda. Läs mer här:

    \begin{itemize}%[nolistsep]
      \item snabbreferensen (enda tentahjälpmedel): \\{\small\url{http://cs.lth.se/pgk/quickref}}
      \item översikt (av Prof. Martin Odersky, uppfinnare av Scala, m.fl.): \\
       {\small\url{http://docs.scala-lang.org/overviews/collections/introduction.html}}
      \item api-dokumentation:\\  {\small\url{https://www.scala-lang.org/api/current/scala/collection/}}
    \end{itemize}
\end{framed}

\SOLUTION

\TaskSolved \what

\SubtaskSolved

\begin{ConceptConnections}
  \code|x +: xs         | & 1 & ~~\Large$\leadsto$~~ &  G & \code|Vector(0, 1, 2, 3)                      | \\ 
  \code|xs +: x         | & 2 & ~~\Large$\leadsto$~~ &  L & \code|value +: is not a member of Int         | \\ 
  \code|xs :+ x         | & 3 & ~~\Large$\leadsto$~~ &  J & \code|Vector(1, 2, 3, 0)                      | \\ 
  \code|xs ++ xs        | & 4 & ~~\Large$\leadsto$~~ &  M & \code|Vector(1, 2, 3, 1, 2, 3)                | \\ 
  \code|xs.indices      | & 5 & ~~\Large$\leadsto$~~ &  E & \code|Range 0 until 3                         | \\ 
  \code|xs apply 0      | & 6 & ~~\Large$\leadsto$~~ &  C & \code|1                                       | \\ 
  \code|xs(3)           | & 7 & ~~\Large$\leadsto$~~ &  I & \code|java.lang.IndexOutOfBoundsException     | \\ 
  \code|xs.length       | & 8 & ~~\Large$\leadsto$~~ &  O & \code|3                                       | \\ 
  \code|xs.take(4)      | & 9 & ~~\Large$\leadsto$~~ &  F & \code|Vector(1, 2, 3)                         | \\ 
  \code|xs.drop(2)      | & 10 & ~~\Large$\leadsto$~~ &  K & \code|Vector(3)                               | \\ 
  \code|xs.updated(0, 2)| & 11 & ~~\Large$\leadsto$~~ &  B & \code|Vector(2, 2, 3)                         | \\ 
  \code|xs.tail.head    | & 12 & ~~\Large$\leadsto$~~ &  N & \code|2                                       | \\ 
  \code|xs.head.tail    | & 13 & ~~\Large$\leadsto$~~ &  D & \code|value tail is not a member of Int       | \\ 
  \code|xs.isEmpty      | & 14 & ~~\Large$\leadsto$~~ &  H & \code|false                                   | \\ 
  \code|xs.nonEmpty     | & 15 & ~~\Large$\leadsto$~~ &  A & \code|true                                    | \\ 
\end{ConceptConnections}

\SubtaskSolved

\noindent\renewcommand*{\arraystretch}{1.2}\begin{tabular}{p{5cm} l p{6cm}}

~\\ \emph{fel} & \emph{typ} & \emph{förklaring} \\\hline

\code|value +: is not| \code|a member of Int|
& kompileringsfel
& Operatorer som slutar med kolon är högerassociativa. Metodanropet \code|xs +: x| motsvarar med punktnotation \code|x.+:(xs)| och det finns ingen metod med namnet \code|+:| på heltal.\\\hline

\code|IndexOutOfBoundsException|
& körtidsfel & Det finns bara 3 element och index räknas från 0 i sekvenssamlingar.\\\hline

\code|value tail is not| \code|a member of Int|
& kompileringsfel
& Metoden \code|head| ger första elementet och heltal saknar sekvenssamlingsmetoden \code|tail|.\\\hline

\end{tabular}


\QUESTEND


\WHAT{Kopiering av sekvenser.}

\QUESTBEGIN

\Task \what~ %\code{map} \code{toArray} \code{copyToArray}
Klassen \code{Mutant} nedan kan användas för att skapa förändringsbara instanser med heltal.\footnote{Om den inbyggda grundtypen Int, i likhet med \code{Mutant}, knasigt nog  kunnat användas för att skapa förändringsbara instanser hade heltalsmatematiken i Scala omvandlats till ett skrämmande kaos.
%\\Lär mer om fem här: \url{https://www.youtube.com/watch?v=dpdOUEe9mm4}
}

\noindent\begin{minipage}{0.6\textwidth}
\begin{Code}[basicstyle=\ttfamily\large\selectfont]
class Mutant(var int: Int = 0)
\end{Code}
\end{minipage}
\hfill\begin{minipage}{0.38\textwidth}
%https://www.1001freedownloads.com/free-clipart/mutant
\centering\includegraphics[width=3.4cm]{../img/mutant.png}
\captionof{figure}{En instans av klassen Mutant där \code{int} kanske är 5.}
%https://tex.stackexchange.com/questions/55337/how-to-use-figure-inside-a-minipage
\end{minipage}

\vspace{1em}\noindent Kör nedan i REPL efter studier av detta:  \url{https://youtu.be/dpdOUEe9mm4}
\begin{REPL}
scala> val fem = new Mutant(5)
scala> val xs = Vector(fem, fem, fem)
scala> val ys = xs.toArray    // kopierar referenserna till ny Array
scala> val zs = xs.map(x => new Mutant(x.int)) // djupkopierar till ny Vector
scala> xs(0).int = (new Mutant).int
\end{REPL}
\Subtask Fyll i tabellen nedan genom att till höger skriva värdet av varje uttryck till vänster. Förklara vad som händer. \emph{Tips:} Metoden \code{eq} jämför alltid referenser (ej innehåll).

\renewcommand{\arraystretch}{2.0}
\vspace{1em}\noindent\begin{tabular}{@{} l | p{5.5cm}}\hline
\code|xs(0)         | & \\\hline
\code|ys(0).int| & \\\hline
\code|zs(0).int| & \\\hline
\code|xs(0) eq ys(0)| & \\\hline
\code|xs(0) eq zs(0)| & \\\hline
\code|(ys.toBuffer :+ new Mutant).apply(0).int| & \\\hline
\end{tabular}

\Subtask Implementera med hjälp av en \code{while}-sats funktionen \code{deepCopy} nedan som gör \emph{djup} kopiering, d.v.s skapar en ny array med nya, innehållskopierade mutanter.
\begin{Code}
def deepCopy(xs: Array[Mutant]): Array[Mutant] = ???
\end{Code}
Använd denna algoritm:

\begin{algorithm}[H]
 \SetKwInOut{Input}{Indata}\SetKwInOut{Output}{Resultat}

 \Input{ ~En mutantarray $xs$}
 \Output{ ~En djup kopia av $xs$}
 $result \leftarrow$ en ny mutantarray med plats för lika många element som i $xs$\\
 $i \leftarrow 0$  \\
 \While{$i$ mindre än antalet element}{
  skapa en kopia av elementet $xs(i)$ och lägg kopian i $result$ på platsen $i$ \\
  öka $i$ med 1
 }
 \Return $result$
\end{algorithm}

\Subtask Testa att din funktion och kolla så att inga läskiga muteringar genom delade referenser går att göra, så som med \code|xs| och \code|ys| i första deluppgiften.

\Subtask Är det vanligt att man, för säkerhets skull, gör djupkopiering av alla element i oföränderliga samlingar som enbart innehåller oföränderliga element?

\SOLUTION

\TaskSolved \what~

\SubtaskSolved

\renewcommand{\arraystretch}{1.5}
\vspace{1em}\noindent\begin{tabular}{@{} p{0.4\textwidth} p{0.6\textwidth}}\hline
\code|xs(0)| & \code|rs$line5$Mutant@66d766b9 | nya instanser får nya hexkoder \\ \hline 
\code|ys(0).int               | & \code|0 | eftersom \code|ys| innehåller samma instans som \code|xs|\\ \hline
\code|zs(0).int               | & \code|5 | eftersom \code|!(xs(0) eq zs(0))| \\ \hline
\code|xs(0) eq ys(0)          | & \code|true |  eftersom samma instans \\ \hline
\code|xs(0) eq zs(0)          | & \code|false | eftersom olika instanser\\ \hline
\code|(ys.toBuffer :+ |
\code|  new Mutant).apply(0).int| & \code|0 | eftersom den ej djupkopierade kopian av typen \code|ArrayBuffer| refererar samma instans på första platsen som både \code|ys| och \code|xs| och \code|x(0).int| blev noll i en tilldelning på rad 5 i REPL-körningen\\ \hline
\end{tabular}

\vspace{0.5em}\noindent Observera alltså att kopiering med \code{toArray}, \code{toVector}, \code{toBuffer}, etc. \emph{inte är djup}, d.v.s. det är bara instansreferenserna som kopieras och inte själva instanserna.


\SubtaskSolved
\begin{CodeSmall}
def deepCopy(xs: Array[Mutant]): Array[Mutant] =
  val result = Array.ofDim[Mutant](xs.length) //fylld med null-referenser
  var i = 0
  while i < xs.length do
    result(i) = new Mutant(xs(i).int) //kopia med samma innehåll på samma plats
    i += 1
  result
\end{CodeSmall}
Det går också bra att skapa resultatarrayen med \code{new Array[Mutant](xs.length)}.
Du kan också använda \code{size} i stället för \code{length}.

\SubtaskSolved
\begin{REPL}
scala> class Mutant(var int: Int = 0)
// defined class Mutant

scala> def deepCopy(xs: Array[Mutant]): Array[Mutant] =
     |   val result = Array.ofDim[Mutant](xs.length)
     |   var i = 0
     |   while i < xs.length do
     |     result(i) = new Mutant(xs(i).int)
     |     i += 1
     |   result

scala> val xs = Array.fill(3)(new Mutant)
xs: Array[Mutant] = Array(rs$line$2$Mutant@46a123e4, rs$line$2$Mutant@44bc2449,
rs$line2$Mutant@3c28e5b6)

scala> val ys = deepCopy(xs)
ys: Array[Mutant] = Array(rs$line$2$Mutant@14b8a751, rs$line2$Mutant@7345f97d,
rs$line$2$Mutant@554566a8)

scala> xs(0).int = 5

scala> ys(0).int
val res0: Int = 0
\end{REPL}

\SubtaskSolved Nej, eftersom elementen inte kan förändras kan man utan problem dela referenser mellan samlingar. Det finns inte någon möjlighet att det kan ske förändringar som påverkar flera samlingar samtidigt.
Dock gör man vanligen (ofta tidsödande) djupkopieringar av samlingar med förändringsbara element för att kunna vara säker på att den ursprungliga samlingen inte förändras.

\QUESTEND



\ifPreSolution
\begin{framed}
\noindent\emph{Tips inför fortsättningen:} Ofta kan du lösa grundläggande delproblem med inbyggda samlingsmetoder ur standardbiblioteket. Till exempel kan ju kopieringen i \code{deepCopy} i föregående uppgift enkelt göras med hjälp av samlingsmetoden \code{map}.

Men det är mycket bra för din förståelse om du kan implementera grundläggande sekvensalgoritmer själv även om det normalt är bättre att använda färdiga, vältestade  metoder. I kommande uppgifter ska du därför göra egna implementationer av några sekvensalgoritmer som redan finns i standardbiblioteket.
\end{framed}
\fi



\WHAT{Uppdatering av sekvenser.}

\QUESTBEGIN

\Task \what~Deklarera dessa variabler i REPL:

\begin{Code}
val xs = (1 to 4).toVector
val buf = xs.toBuffer
\end{Code}

\Subtask Uttrycken till vänster evalueras uppifrån och ned. Para ihop med rätt resultat.

\begin{ConceptConnections}
  \code|{ buf(0) = -1; buf(0) }   | & 1 & & A & {\small\code|value update is not a member|} \\ 
  \code|{ xs(0) = -1; xs(0) }| & 2 & & B & \code|Vector(5, 2, 3, 4)| \\ 
  \code|buf.update(1, 5)          | & 3 & & C & \code|ArrayBuffer(-1, 5, 3, 4, 5)| \\ 
  \code|xs.updated(0, 5)          | & 4 & & D & \code|-1| \\ 
  \code|buf += 5                 | & 5 & & E & \code|Vector(1, -1, 5)| \\ 
  \code|xs += 5                  | & 6 & & F & \code|(): Unit| \\ 
  \code|xs.patch(1,Vector(-1,5),3)| & 7 & & G & {\small\code|value += is not a member|} \\ 
  \code|xs                        | & 8 & & H & \code|Vector(1, 2, 3, 4)|
\end{ConceptConnections}

\smallskip
\emph{Tips:} Läs om metoderna i snabbreferensen och undersök i REPL. Exempel:
\begin{REPL}
scala> Vector(1,2,3,4).patch(from = 1, other = Vector(0,0), replaced = 3)
val res0: Vector[Int] = Vector(1, 0, 0)
\end{REPL}

\Subtask Implementera funktionen \code{insert} nedan med hjälp av sekvenssamlingsmetoden \code{patch}. \emph{Tips:} Ge argumentet \code{0} till parametern \code{replaced}.
\begin{Code}
/** Skapar kopia av xs men med elem insatt på plats pos. */
def insert(xs: Array[Int], elem: Int, pos: Int): Array[Int] = ???
\end{Code}

\Subtask Skriv pseduokod för en algoritm som implementerar \code{insert} med hjälp av \code{while}.

\Subtask Implementera \code{insert} enligt din pseudokod. Testa i REPL och se vad som händer om \code{pos} är negativ? Vad händer om \code{pos} är precis ett steg bortom sista platsen i \code{xs}? Vad händer om \code{pos} är flera steg bortom sista platsen?

\SOLUTION

\TaskSolved \what~

\SubtaskSolved

\begin{ConceptConnections}
  \code|{ buf(0) = -1; buf(0) }   | & 1 & ~~\Large$\leadsto$~~ &  D & \code|-1| \\ 
  \code|{ xs(0) = -1; xs(0) }| & 2 & ~~\Large$\leadsto$~~ &  A & {\small\code|value update is not a member|} \\ 
  \code|buf.update(1, 5)          | & 3 & ~~\Large$\leadsto$~~ &  F & \code|(): Unit| \\ 
  \code|xs.updated(0, 5)          | & 4 & ~~\Large$\leadsto$~~ &  B & \code|Vector(5, 2, 3, 4)| \\ 
  \code|buf += 5                | & 5 & ~~\Large$\leadsto$~~ &  C & \code|ArrayBuffer(-1, 5, 3, 4, 5)| \\ 
  \code|xs += 5                 | & 6 & ~~\Large$\leadsto$~~ &  G & {\small\code|value += is not a member|} \\ 
  \code|xs.patch(1,Vector(-1,5),3)| & 7 & ~~\Large$\leadsto$~~ &  E & \code|Vector(1, -1, 5)| \\ 
  \code|xs                        | & 8 & ~~\Large$\leadsto$~~ &  H & \code|Vector(1, 2, 3, 4)| 
\end{ConceptConnections}

\SubtaskSolved

\begin{Code}
def insert(xs: Array[Int], elem: Int, pos: Int): Array[Int] =
  xs.patch(from = pos, other = Array(elem), replaced = 0)
\end{Code}

\SubtaskSolved Pseudokoden nedan är skriven så att den kompilerar fast den är ofärdig.
\begin{Code}
def insert(xs: Array[Int], elem: Int, pos: Int): Array[Int] = 
  val result = ??? /* ny array med plats för ett element mer än i xs */
  var i = 0
  while(???){/* kopiera elementen före plats pos och öka i */}
  if i < result.length then /* lägg elem i result på plats i */
  while(???){/* kopiera över resten */}
  result

\end{Code}

\SubtaskSolved
\begin{Code}
def insert(xs: Array[Int], elem: Int, pos: Int): Array[Int] = 
  val result = new Array[Int](xs.length + 1)
  var i = 0
  while i < pos && i < xs.length do  { result(i) = xs(i); i += 1}
  if i < result.length then { result(i) = elem; i += 1 }
  while i < result.length do { result(i) = xs(i - 1); i += 1}
  result

\end{Code}
\begin{REPL}
scala> insert(Array(1, 2), 0, pos = -1)
val res2: Array[Int] = Array(0, 1, 2)

scala> insert(Array(1, 2), 0, pos = 0)
val res3: Array[Int] = Array(0, 1, 2)

scala> insert(Array(1, 2), 0, pos = 1)
val res4: Array[Int] = Array(1, 0, 2)

scala> insert(Array(1, 2), 0, pos = 2)
val res5: Array[Int] = Array(1, 2, 0)

scala> insert(Array(1, 2), 0, pos = 42)
val res7: Array[Int] = Array(1, 2, 0)
\end{REPL}

\QUESTEND




\ifPreSolution
\begin{framed}
\noindent\emph{Tips inför fortsättningen:} Det är inte lätt att få rätt på alla specialfall även i små algoritmer så som \code{insert} ovan. Det är därför viktigt att noga tänka igenom sin sekvensalgoritm med avseende på olika specialfall. Använd denna checklista:
\begin{enumerate}[noitemsep]
  \item Vad händer om sekvensen är tom?
  \item Fungerar det för exakt ett element?
  \item Kan index bli negativt?
  \item Kan index bli mer än längden minus ett?
  \item Kan det bli en oändlig loop, t.ex. p.g.a. saknad loopvariabeluppräkning?
\end{enumerate}
Ibland vill man att vettiga undantag ska kastas vid ogiltig indata eller andra feltillstånd och då är \code{require} eller \code{assert} bra att använda. I andra fall vill man att resultatet t.ex. ska bli en tom sekvenssamling om indata är ogiltigt. Sådana beteenden behöver dokumenteras så att andra som använder dina algoritmer (eller du själv efter att du glömt hur det var) förstår vad som händer i olika fall.


\end{framed}
\fi

\WHAT{Jämföra strängar i Scala.}

\QUESTBEGIN

\Task \label{task:string-order-operators} \what~  I Scala kan strängar jämföras med operatorerna \code{==}, \code{!=}, \code{<}, \code{<=}, \code{>}, \code{>=},  där likhet/olikhet avgörs av om alla tecken i strängen är lika eller inte, medan större/mindre avgörs av sorteringsordningen i enlighet med varje teckens Unicode-värde.\footnote{Överkurs: Alla tecken i en \code{java.lang.String} representeras enligt UTF-16-standarden (\href{https://en.wikipedia.org/wiki/UTF-16}{https://en.wikipedia.org/wiki/UTF-16}), vilket innebär att varje Unicode-kodpunkt \Eng{code point} lagras som antingen ett eller två 16-bitars heltal. Strängjämförelse i Scala och Java jämför egentligen inte varje tecken, utan varje 16-bitars heltal. Denna skillnad har ingen betydelse när en sträng bara innehåller tecken som tar upp ett 16-bitars heltal var, och praktiskt nog är nästan alla tecken som används vardagligen av den typen. De flesta tecken som kräver två 16-bitars heltal är sällsynta kinesiska tecken, sällsynta symboler, tecken från utdöda språk och emoji. Vi kommer att bortse från sådana tecken i den här kursen.}

\Subtask Vad ger följande jämförelser för värde?
\begin{REPL}
scala> 'a' < 'b'
scala> "aaa" < "aaaa"
scala> "aaa" < "bbb"
scala> "AAA" < "aaa"
scala> "ÄÄÄ" < "ÖÖÖ"
scala> "ÅÅÅ" < "ÄÄÄ"
\end{REPL}
Tyvärr så följer ordningen av ÄÅÖ inte svenska regler, men det ignorerar vi i fortsättningen för enkelhets skull; om du är intresserad av hur man kan fixa  detta, gör uppgift \ref{task:swedish-letter-ordering}.

\Subtask\Pen Vilken av strängarna $s1$ och $s2$ kommer först (d.v.s. är ''mindre'') om $s1$ utgör början av $s2$ och $s2$ innehåller fler tecken än $s1$?


\SOLUTION


\TaskSolved \what


\SubtaskSolved
\begin{REPL}
true
true
true
true
true
false
\end{REPL}

\SubtaskSolved
\emph{s1} kommer först.


\QUESTEND




\WHAT{Linjärsökning enligt olika sökkriterier.}

\QUESTBEGIN

\Task \what~Linjärsökning innebär att man letar tills man hittar det man söker efter i en sekvens. Detta delproblem återkommer ofta! Vanligen börjar linjärsökning från början och håller på tills man hittar något element som uppfyller kriteriet. Beroende på vad som finns i sekvensen och hur kriteriet ser ut kan det hända att man måste gå igenom alla element utan att hitta det som söks.

\Subtask Linjärsökning med inbyggda sekvenssamlingsmetoder.
\begin{Code}
val xs = ((1 to 5).reverse ++ (0 to 5)).toVector
\end{Code}
Deklarera ovan variabel i REPL och para ihop uttrycken nedan med rätt värden. Förklara vad som händer.

\begin{ConceptConnections}
\input{generated/quiz-w07-seq-find-taskrows-generated.tex}
\end{ConceptConnections}

\Subtask Implementera linjärsökning i strängvektor med strängpredikat.
\begin{Code}
/** Returns first index where p is true. Returns -1 if not found. */
def indexOf(xs: Vector[String], p: String => Boolean): Int = ???
\end{Code}
Ett strängpredikat \code{p: String => Boolean} är en funktion som tar en sträng som indata och ger ett booleskt värde som resultat. Implementera \code{indexOf} med hjälp av en \code{while}-sats. Du kan t.ex. använda en lokal boolesk variabel \code{found} för att hålla reda på om du har hittat det som eftersöks enligt predikatet.

När element som uppfyller predikatet saknas måste man bestämma vad som ska hända. Kravet på din implementation i detta fall ges av dokumentationskommentaren ovan.

Din funktion ska fungera enligt nedan:
\begin{REPL}
scala> val xs = Vector("hej", "på", "dej")
val xs: Vector[String] = Vector(hej, på, dej)

scala> indexOf(xs, _.contains('p'))
val res0: Int = 1

scala> indexOf(xs, _.contains('q'))
val res1: Int = -1

scala> indexOf(Vector(), _.contains('q'))
val res2: Int = -1

scala> indexOf(Vector("q"), _.length == 1)
val res3: Int = 0
\end{REPL}

\SOLUTION

\TaskSolved \what~

\SubtaskSolved

\begin{ConceptConnections}
\input{generated/quiz-w07-seq-find-solurows-generated.tex}
\end{ConceptConnections}

\SubtaskSolved Med en boolesk variabel \code{found}:

\begin{Code}
def indexOf(xs: Vector[String], p: String => Boolean): Int = 
  var found = false
  var i = 0
  while i < xs.length && !found do
      found = p(xs(i))
      i += 1
  if found then i - 1 else -1
\end{Code}
Eller utan \code{found}:
\begin{Code}
def indexOf(xs: Vector[String], p: String => Boolean): Int = 
  var i = 0
  while i < xs.length && !p(xs(i)) do i += 1
  if i == xs.length then -1 else i
\end{Code}
Eller så kanske man vill börja bakifrån; lösningen nedan är nog enklare att fatta (?) och definitivt mer koncis, men uppfyller \emph{inte} kravet att returnera index för \emph{första} förekomsten som det står i uppgiften. Men om sammanhanget tillåter att vi returnerar \emph{något} index för vilket predikatet gäller, eller om man faktiskt har kravet att leta bakifrån, så funkar detta:
\begin{Code}
def indexOf(xs: Vector[String], p: String => Boolean): Int = 
  var i = xs.length - 1
  while i >= 0 && !p(xs(i)) do i -= 1
  i
\end{Code}
Eller så kan man göra på flera andra sätt. När du ska implementera algoritmer, både på programmeringstentan och i yrkeslivet som systemutvecklare, finns det ofta många olika sätt att lösa uppgiften på som har olika egenskaper, fördelar och nackdelar. Det viktiga är att lösningen fungerar så gott det går enligt kraven, att koden är begriplig för människor och att implementationen inte är så ineffektiv att användarna tröttnar i sin väntan på resultatet...

\QUESTEND




\WHAT{Labbförberedelse: Implementera heltalsregistrering i Array.}

\QUESTBEGIN

\Task \what~Registrering innebär att man räknar antalet förekomster av olika värden. Varje gång ett nytt värde förekommer behöver vi räkna upp en frekvensräknare. Det behövs en räknare för varje värde som ska registreras. Vi ska fortsätta räkna ända tills alla värden är registrerade.

På veckans laboration ska du registrera förekomsten av olika kortkombinationer i kortspelet poker. I denna övning ska du som träning inför laborationen lösa ett liknande registreringsproblem:  frekvensanalys av många tärningskast. Vid tärningsregistrering behövs sex olika räknare. Man kan med fördel då använda en sekvenssamling med plats för sex heltal. Man kan t.ex. låta  plats \code{0} håller reda på antalet ettor, plats \code{1} hålla reda på antalet tvåor, etc.

\Subtask Implementera nedan algoritm enligt pseudokoden:
\begin{Code}
def registreraTärningskast(xs: Seq[Int]): Vector[Int] = 
  val result = ??? /* Array med 6 nollor */
  xs.foreach{ x =>
    require(x >= 1 && x <= 6, "tärningskast ska vara mellan 1 & 6")
    ??? /* räkna förekomsten av x */
  }
  result.toVector
\end{Code}

\Subtask Använd funktionen \code{kasta} nedan när du testar din registreringsalgoritm med en sekvenssamling innehållande minst $1000$ tärningskast.
\begin{Code}
def kasta(n: Int) = Vector.fill(n)(util.Random.nextInt(6) + 1)
\end{Code}

\SOLUTION

\TaskSolved \what~

\SubtaskSolved
\begin{Code}
def registreraTärningskast(xs: Seq[Int]): Vector[Int] = 
  val result = Array.fill(6)(0)
  xs.foreach{ x =>
    require(x >= 1 && x <= 6, "tärningskast ska vara mellan 1 & 6")
    result(x - 1) += 1
  }
  result.toVector
\end{Code}

\SubtaskSolved
\begin{REPL}
scala> registreraTärningskast(kasta(1000))
val res0: Vector[Int] = Vector(171, 163, 166, 152, 184, 164)

scala> registreraTärningskast(kasta(1000))
val res1: Vector[Int] = Vector(163, 161, 158, 174, 161, 183)
\end{REPL}

\QUESTEND




\WHAT{Inbyggda metoder för sortering.}

\QUESTBEGIN

\Task \what~Det finns fler olika sätt att ordna sekvenser efter olika kriterier. För  grundtyperna \code{Int}, \code{Double}, \code{String}, etc., finns inbyggda ordningar som gör att sekvenssamlingsmetoden \code{sorted} fungerar utan vidare argument (om du är nöjd med den inbyggda ordningsdefinitionen). Det finns också metoderna \code{sortBy} och \code{sortWith} om du vill ordna en sekvens med element av någon grundtyp efter egna ordningsdefinitioner eller om du har egna klasser i din sekvens.
\begin{Code}
val xs = Vector(1, 2, 1, 3, -1)
val ys = Vector("abra", "ka", "dabra").map(_.reverse)
val zs = Vector('a', 'A', 'b', 'c').sorted

case class Person(förnamn: String, efternamn: String)

val ps = Vector(Person("Kim", "Ung"), Person("kamrat", "Clementin"))
\end{Code}
Deklarera ovan i REPL och para ihop uttryck nedan med rätt resultat.
\\\emph{Tips:} Stora bokstäver sorteras före små bokstäver i den inbyggda ordningen för grundtyperna \code{String} och \code{Char}. Dessutom har svenska tecken knasig ordning.\footnote{Ordningen kommer ursprungligen från föråldrade teckenkodningsstandarder:    \url{https://sv.wikipedia.org/wiki/ASCII}}
\\Läs om sorteringsmetoderna i snabbreferensen och prova i REPL.

\begin{ConceptConnections}
  \code|'a' < 'A'                  | & 1 & & A & \code|"ka"| \\ 
  \code|"AÄÖö" < "AÅÖö"        | & 2 & & B & \code|1| \\ 
  \code|xs.sorted.head             | & 3 & & C & \code|-1| \\ 
  \code|xs.sorted.reverse.head     | & 4 & & D & \code|error: ...| \\ 
  \code|ys.sorted.head             | & 5 & & E & \code|false| \\ 
  \code|zs.indexOf('a')            | & 6 & & F & \code|0| \\ 
  \code|ps.sorted.head.förnamn.take(2)| & 7 & & G & \code|3| \\ 
  \code|ps.sortBy(_.förnamn).apply(1).förnamn.take(2)| & 8 & & H & \code|true| \\ 
  \code|xs.sortWith((x1, x2) => x1 > x2).indexOf(3)| & 9 & & I & \code|"ak"| 
\end{ConceptConnections}
Vi ska senare i kursen implementera egna sorteringsalgoritmer som träning, men i normala fall använder man inbyggda sorteringar som är effektiva och vältestade. Dock är det inte ovanligt att man vill definiera egna ordningar för egna klasser, vilket vi ska undersöka senare i kursen.

\SOLUTION

\TaskSolved \what

\begin{ConceptConnections}
  \code|'a' < 'A'                  | & 1 & ~~\Large$\leadsto$~~ &  E & \code|false| \\ 
  \code|"AÄÖö" < "AÅÖö"        | & 2 & ~~\Large$\leadsto$~~ &  H & \code|true| \\ 
  \code|xs.sorted.head             | & 3 & ~~\Large$\leadsto$~~ &  C & \code|-1| \\ 
  \code|xs.sorted.reverse.head     | & 4 & ~~\Large$\leadsto$~~ &  G & \code|3| \\ 
  \code|ys.sorted.head             | & 5 & ~~\Large$\leadsto$~~ &  I & \code|"ak"| \\ 
  \code|zs.indexOf('a')            | & 6 & ~~\Large$\leadsto$~~ &  B & \code|1| \\ 
  \code|ps.sorted.head.förnamn.take(2)| & 7 & ~~\Large$\leadsto$~~ &  D & \code|error: ...| \\ 
  \code|ps.sortBy(_.förnamn).apply(1).förnamn.take(2)| & 8 & ~~\Large$\leadsto$~~ &  A & \code|"ka"| \\ 
  \code|xs.sortWith((x1, x2) => x1 > x2).indexOf(3)| & 9 & ~~\Large$\leadsto$~~ &  F & \code|0| 
\end{ConceptConnections}
Det blir fel i uttrycket ovan som försöker sortera en sekvens med instanser av \code{Person} direkt med metoden \code{sorted}:
\begin{REPL}
scala> ps.sorted
No implicit Ordering defined for Person.
\end{REPL}
Det blir fel eftersom kompilatorn inte hittar någon ordningsdefinition för dina egna klasser. Senare i kursen ska vi se hur vi kan skapa egna ordningar om man vill få \code{sorted} att fungera på sekvenser med instanser av egna klasser, men ofta räcker det fint med \code{sortBy} och \code{sortWith}.
\QUESTEND


\WHAT{Inbyggd metod för blandning.}

\QUESTBEGIN
\Task \what~På veckans laboration ska du implementera en egen blandningsalgoritm och använda den för att blanda en kortlek. Det finns redan en inbygg metod \code{shuffle} i singelobjektet \code{Random} i paketet \code{scala.util}.

\Subtask Sök upp dokumentationen för \code{Random.shuffle} och studera funktionshuvudet. Det står en hel del invecklade saker om \code{CanBuildFrom} etc. Detta smarta krångel, som vi inte går närmare in på i denna kurs, är till för att metoden ska kunna returnera lämplig typ av samling. När du ser ett sådant funktionshuvud kan du anta att metoden fungerar fint med flera olika typer av lämpliga samlingar i Scalas standardbibliotek.

Klicka på \code{shuffle}-dokumentationen så att du ser hela texten. Vad säger dokumentationen om resultatet? Är det blandning på plats eller blandning till ny samling?

\Subtask Prova upprepade blandningar av olika typer av sekvenser med olika typer av element i REPL.

\SOLUTION

\TaskSolved \what~

\SubtaskSolved \code{Random.shuffle} returnerar en ny blandad sekvenssamling av samma typ. Ordningen i den ursprungliga samlingen påverkas inte.

\SubtaskSolved Exempel på användning av \code{random.shuffle}:
\begin{REPL}
scala> import scala.util.Random

scala> val xs = Vector("Sten", "Sax", "Påse")
val xs: Vector[String] = Vector(Sten, Sax, Påse)

scala> (1 to 10).foreach(_ => println(Random.shuffle(xs).mkString(" ")))
Sax Påse Sten
Sten Påse Sax
Sten Sax Påse
Sten Sax Påse
Sten Påse Sax
Sten Påse Sax
Sax Sten Påse
Sten Påse Sax
Sax Påse Sten
Sax Påse Sten

scala> (1 to 5).map(_ => Random.shuffle(1 to 6))
val res1: IndexedSeq[IndexedSeq[Int]] =
  Vector(Vector(5, 2, 1, 4, 3, 6), Vector(6, 5, 4, 2, 1, 3),
  Vector(3, 1, 4, 6, 5, 2), Vector(3, 2, 6, 5, 1, 4),
  Vector(5, 3, 4, 6, 1, 2))

scala> (1 to 1000).map(_ => Random.shuffle(1 to 6).head).count(_ == 6)
val res2: Int = 168
\end{REPL}

\QUESTEND



\WHAT{Repeterade parametrar.}

\QUESTBEGIN

\Task  \what~  Det går att deklarera en funktion som tar en argumentsekvens av godtycklig längd, ä.k. \emph{varargs}. Syntaxen består av en asterisk \code{*} efter typen. Funktion sägs då ha repeterade parametrar \Eng{repeated parameters}. I funktionskroppen får man tillgång till argumenten i en sekvenssamling. Argumenten anges godtyckligt många med komma emellan. Exempel:
\begin{Code}
/** Ger en vektor med stränglängder för godtyckligt antal strängar. */
def stringSizes(xs: String*): Vector[Int] = xs.map(_.size).toVector
\end{Code}

\Subtask Deklarera och använd \code{stringSizes} i REPL. Vad händer om du anropar \code{stringSizes} med en tom argumentlista?

\Subtask Det händer ibland att man redan har en sekvenssamling, t.ex. \code{xs}, och vill skicka med varje element som argument till en varargs-funktion. Syntaxen för detta är \code{xs: _* } vilket gör att kompilatorn omvandlar sekvenssamlingen till en argumentsekvens av rätt typ.

Prova denna syntax genom att ge en \code{xs} av typen \code{Vector[String]} som argument till \code{stringSizes}. Fungerar det även om \code{xs} är en sekvens av längden 0?

\SOLUTION

\TaskSolved \what

\SubtaskSolved

\begin{REPL}
scala> def stringSizes(xs: String*): Vector[Int] = xs.map(_.size).toVector
def stringSizes(xs: String*): Vector[Int]

scala> stringSizes("hej")
val res0: Vector[Int] = Vector(3)

scala> stringSizes("hej", "på", "dej", "")
val res1: Vector[Int] = Vector(3, 2, 3, 0)

scala> stringSizes()
val res2: Vector[Int] = Vector()
\end{REPL}

\noindent Anrop med tom argumentlista ger en tom heltalssekvens.

\SubtaskSolved

\begin{REPL}
scala> val xs = Vector("hej","på","dej", "")
val xs: Vector[String] = Vector(hej, på, dej, "")

scala> stringSizes(xs: _*)
val res0: Vector[Int] = Vector(3, 2, 3, 0)

scala> stringSizes(Vector(): _*)
val res1: Vector[Int] = Vector()
\end{REPL}
Ja, det funkar fint med tom sekvens.

\QUESTEND



\clearpage

\ExtraTasks %%%%%%%%%%%%%%%%%%%%%%%%%%%%%%%%%%%%%%%%%%%%%%%%%%%%%%%%%%%%%%%%%%%%



\WHAT{Registrering av booleska värden. Singla slant.}

\QUESTBEGIN

\Task \what~

\Subtask Implementera en funktion som registrerar många slantsinglingar enligt nedan funktionshuvud. Indata är en sekvens av booleska värden där krona kodas som \code{true} och klave kodas som \code{false}. För registreringen ska du använda en lokal \code{Array[Int]}. I resultatet ska antalet utfall av \code{krona} ligga på första platsen i 2-tupeln och på andra platsen ska antalet utfall av \code{klave} ligga.

\begin{Code}
def registerCoinFlips(xs: Seq[Boolean]): (Int, Int) = ???
\end{Code}

\Subtask Skapa en funktion \code{flips(n)} som ger en boolesk \code{Vector} med $n$ stycken slantsinglingar och använd den när du testar din slantsinglingsregistreringsalgoritm.

\SOLUTION

\TaskSolved \what~

\SubtaskSolved
\begin{Code}
def registerCoinFlips(xs: Seq[Boolean]): (Int, Int) = 
  val result = Array.fill(2)(0)
  xs.foreach(x => if (x) result(0) += 1 else result(1) += 1)
  (result(0), result(1))
\end{Code}

\SubtaskSolved

\QUESTEND


\WHAT{Kopiering och tillägg på slutet.}

\QUESTBEGIN

\Task \what~
Skapa funktionen \code{copyAppend} som implementerar nedan algoritm, \emph{efter} att du rättat de \textbf{\color{red}{två buggarna}} nedan:

\begin{algorithm}[H]
 \SetKwInOut{Input}{Indata}\SetKwInOut{Output}{Resultat}

 \Input{Heltalsarray $xs$ och heltalet $x$}
 \Output{En ny heltalsarray som som är en kopia av $xs$ men med $x$ tillagt på slutet som extra element.}
 $ys \leftarrow$ en ny array med plats för ett element mer än i $xs$\\
 $i \leftarrow 0$  \\
 \While{$i \leq xs.length$}{
  $ys(i) \leftarrow xs(i)$
 }
lägg $x$ på sista platsen i $ys$
\end{algorithm}

\noindent Granska din kod enligt checklistan i tidigare tipsruta. Testa din funktion för de olika fallen: tom sekvens, sekvens med exakt ett element, sekvens med många element.


\SOLUTION

\TaskSolved \what~

\begin{Code}
def copyAppend(xs: Array[Int], x: Int): Array[Int] = 
  val ys = new Array[Int](xs.length + 1)
  var i = 0
  while i < xs.length do
    ys(i) = xs(i)
    i += 1
  ys(xs.length) = x
  ys
\end{Code}
De två buggarna i algoritmen finns (1) i villkoret som ska vara strikt mindre än och (2) inne i loopen där uppräkningen av loppvariabeln saknas.

\QUESTEND



% \WHAT{Välja sekvenssamling.}
%
% \QUESTBEGIN
%
% \Task  \what~Vilken sekvenssamling är lämpligast i respektive situation nedan? Välj mellan \code{Vector}, \code{ArrayBuffer} och \code{ListBuffer}.
%
% \Subtask Det asociala mediet ZuckerBok ska lagra statusuppdateringar från sina användare. Dessa lagras i en förändringsbar sekvens där nya poster läggs till först. Indexering mitt i sekvensen är mycket ovanligt eftersom de flesta användarna sällan läser vad andra skriver, utan mest skriver nya inlägg om sig själv.
%
% \Subtask ZuckerBok försöker öka sina intäkter och börjar frenetiskt indexera i kors och tvärs i sekvensen med statusuppdaringar för att söka efter lämpliga spamoffer.
%
% \Subtask ZuckerBok bestämmer sig för att lagra födelsedatum för alla ca $10^7$ medborgare i Sverige i en oföränderlig sekvens för att kunna förmedla specialreklam på födelsedagar.
%
% \SOLUTION
%
% \TaskSolved \what
%
% \SubtaskSolved  \code{ListBuffer} som är snabb på fröändringar i början av sekvensen.
%
% \SubtaskSolved  \code{ArrayBuffer} som är snabb på både storleksförändringar och godtycklig indexering.
%
% \SubtaskSolved  \code{Vector} eftersom ofränderlighet efterfrågas.
%
% \QUESTEND



\WHAT{Kopiera och reversera sekvens.}

\QUESTBEGIN

\Task  \what~  Implementera \code{seqReverseCopy} enligt:

\begin{algorithm}[H]
 \SetKwInOut{Input}{Indata}\SetKwInOut{Output}{Resultat}

 \Input{Heltalsarray $xs$}
 \Output{En ny heltalsarray med elementen i $xs$ i omvänd ordning.}
 $n \leftarrow$ antalet element i $xs$ \\
 $ys \leftarrow$ en ny heltalsarray med plats för $n$ element\\
 $i \leftarrow 0$  \\
 \While{$i < n$}{
  $ys(n - i - 1) \leftarrow xs(i)$ \\
  $i \leftarrow i + 1$
 }
 \Return $ys$
\end{algorithm}

\Subtask Använd en \code{while}-sats på samma sätt som i algoritmen. Prova din implementation i REPL och kolla så att den fungerar i olika fall.

\Subtask Gör en ny implementation som i stället använder en \code{for}-sats som börjar bakifrån. Kör din implementation i REPL och kolla så att den fungerar i olika fall.

\SOLUTION

\TaskSolved \what

\SubtaskSolved  \begin{Code}
def seqReverseCopy(xs: Array[Int]): Array[Int] =
  val n = xs.length
  val ys = new Array[Int](n)
  var i = 0
  while i < n do
    ys(n - i - 1) = xs(i)
    i += 1
  ys
\end{Code}

\SubtaskSolved  \begin{Code}
def seqReverseCopy(xs: Array[Int]): Array[Int] = 
  val n = xs.length
  val ys = new Array[Int](n)
  for i <- (n - 1) to 0 by -1 do
    ys(n - i - 1) = xs(i)
  ys
\end{Code}


\QUESTEND




\WHAT{Kopiera alla utom ett.}

\QUESTBEGIN

\Task  \what~  Implementera kopiering av en array \emph{utom} ett element på en viss angiven plats.
Skriv först pseudokod innan du implementerar:
\begin{Code}
def removeCopy(xs: Array[Int], pos: Int): Array[Int]
\end{Code}

\SOLUTION


\TaskSolved \what

\begin{algorithm}[H]
 \SetKwInOut{Input}{Indata}\SetKwInOut{Output}{Resultat}

 \Input{En sekvens $xs$ av typen \texttt{Array[Int]} och $pos$}
 \Output{En ny sekvens av typen \texttt{Array[Int]} som är en kopia av $xs$ fast med elementet på plats $pos$ borttaget}
 $n \leftarrow$ antalet element $xs$\\
 $ys \leftarrow$ en ny \texttt{Array[Int]} med plats för $n-1$ element \\
 \For{$i \leftarrow 0$ \KwTo $pos - 1$}{
  $ys(i) \leftarrow xs(i)$
 }
 $ys(pos) \leftarrow x$ \\
 \For{$i \leftarrow pos+1$ \KwTo $n - 1$}{
  $ys(i - 1) \leftarrow xs(i)$
 }
 \Return $ys$
\end{algorithm}

\begin{Code}
def removeCopy(xs: Array[Int], pos: Int): Array[Int] =
  val n = xs.size
  val ys = Array.fill(n - 1)(0)
  for i <- 0 until pos do
    ys(i) = xs(i)
  for i <- (pos + 1) until n do
    ys(i - 1) = xs(i)
  ys
\end{Code}

\QUESTEND




\WHAT{Borttagning på plats i array.}

\QUESTBEGIN

\Task  \what~  Ibland vill man ta bort ett element på en viss position i en array utan att kopiera alla element, utom ett, till en ny samling. Ett sätt att göra detta är att flytta alla efterföljande element ett steg mot lägre index och fylla ut sista positionen med ett utfyllnadsvärde, t.ex. $0$.
Skriv först pseudokod för en sådan algoritm. Implementera sedan algoritmen i en funktion med denna signatur:
\begin{Code}
def removeAndPad(xs: Array[Int], pos: Int, pad: Int = 0): Unit
\end{Code}

\SOLUTION

\TaskSolved \what

\begin{algorithm}[H]
 \SetKwInOut{Input}{Indata}\SetKwInOut{Output}{Resultat}

 \Input{En sekvens $xs$ av typen \texttt{Array[Int]}, en position $pos$ och ett utfyllnadsvärde $pad$}
 \Output{En uppdaterad sekvens av $xs$ där elementet på plats $pos$ tagits bort och efterföljande element flyttas ett steg mot lägre index med ett sista elementet som tilldelats värdet av $pad$}
 $n \leftarrow$ antalet element $xs$\\
 \For{$i \leftarrow pos+1$ \KwTo $n - 1$}{
  $xs(i - 1) \leftarrow xs(i)$
 }
 $xs(n - 1) \leftarrow pad$ \\
\end{algorithm}

\begin{Code}
def remove(xs: Array[Int], pos: Int, pad: Int = 0): Unit =
  val n = xs.size
  for i <- (pos + 1) until n do
    xs(i - 1) = xs(i)
  xs(n - 1) = pad
\end{Code}

\QUESTEND




\WHAT{Kopiering och insättning.}

\QUESTBEGIN

\Task  \what~

\Subtask Implementera en funktion med detta huvud enligt efterföljande algoritm:
\begin{Code}
def insertCopy(xs: Array[Int], x: Int, pos: Int): Array[Int]
\end{Code}


\begin{algorithm}[H]
 \SetKwInOut{Input}{Indata}\SetKwInOut{Output}{Resultat}

 \Input{En sekvens $xs$ av typen \texttt{Array[Int]} och heltalen $x$ och $pos$}
 \Output{En ny sekvens av typen \texttt{Array[Int]} som är en kopia av $xs$ men där $x$ är infogat på plats $pos$}
 $n \leftarrow$ antalet element $xs$\\
 $ys \leftarrow$ en ny \texttt{Array[Int]} med plats för $n+1$ element \\
 \For{$i \leftarrow 0$ \KwTo $pos - 1$}{
  $ys(i) \leftarrow xs(i)$
 }
 $ys(pos) \leftarrow x$ \\
 \For{$i \leftarrow pos$ \KwTo $n - 1$}{
  $ys(i + 1) \leftarrow xs(i)$
 }
 \Return $ys$
\end{algorithm}


\Subtask Vad måste \code{pos} vara för att det ska fungera med en tom array som argument?

\Subtask Vad händer om din funktion anropas med ett negativt argument för \code{pos}?

\Subtask Vad händer om din funktion anropas med \code{pos} lika med \code{xs.size}?

\Subtask Vad händer om din funktion anropas med \code{pos} större än \code{xs.size}?

\SOLUTION

\TaskSolved \what

\SubtaskSolved  \begin{Code}
def insertCopy(xs: Array[Int], x: Int, pos: Int): Array[Int] =
  val n = xs.size
  val ys = Array.ofDim[Int](n + 1)
  for i <- 0 until pos do
    ys(i) = xs(i)
  ys(pos) = x
  for i <- pos until n do
    ys(i + 1) = xs(i)
  ys
\end{Code}

\SubtaskSolved  \code{pos} måste vara \code{0}.

\SubtaskSolved  \begin{REPL}
java.lang.ArrayIndexOutOfBoundsException: -1
\end{REPL}

\SubtaskSolved  Elementet \code{x} läggs till på slutet av arrayen, alltså kommer den returnerande arrayen vara större än den som skickades in.

\SubtaskSolved  \begin{REPL}
java.lang.ArrayIndexOutOfBoundsException: 5
\end{REPL}
Man får \code{ArrayIndexOutOfBoundsException} då indexeringen är utanför storleken hos arrayen.

\QUESTEND




\WHAT{Insättning på plats i array.}

\QUESTBEGIN

\Task  \what~  Ett sätt att implementera insättning i en array, utan att kopiera alla element till en ny array med en plats extra, är att alla elementen efter \code{pos} flyttas fram ett steg till högre index, så att plats bereds för det nya elementet. Med denna lösning får det sista elementet ''försvinna'' genom brutal överskrivning eftersom arrayer inte kan ändra storlek.

Skriv först en sådan algoritm i pseudokod och implementera den sedan i en procedur med detta huvud:
\begin{Code}
def insertDropLast(xs: Array[Int], x: Int, pos: Int): Unit
\end{Code}

\SOLUTION

\TaskSolved \what

\begin{algorithm}[H]
 \SetKwInOut{Input}{Indata}\SetKwInOut{Output}{Resultat}

 \Input{En sekvens $xs$ av typen \texttt{Array[Int]} och heltalen $x$ och $pos$}
 \Output{En uppdaterad sekvens av $xs$ där elementet $x$ har satts in på platsen $pos$ och efterföljande element flyttas ett steg där sista elementet försvinner}
 $n \leftarrow$ antalet element i $xs$\\
 $ys \leftarrow$ en klon av $xs$\\
 $xs(pos) \leftarrow x$\\
 \For{$i \leftarrow pos+1$ \KwTo $n - 1$}{
  $xs(i) \leftarrow ys(i - 1)$
 }
\end{algorithm}

\begin{Code}
def insertDropLast(xs: Array[Int], x: Int, pos: Int): Unit =
  val n = xs.size
  val ys = xs.clone
  xs(pos) = x
  for i <- pos + 1 until n do
    xs(i) = ys(i - 1)
\end{Code}

\QUESTEND


\WHAT{Fler inbyggda metoder för linjärsökning.}

\QUESTBEGIN

\Task \what~

\Subtask Läs i snabbreferensen om metoderna \code{lastIndexOf}, \code{indexOfSlice}, \code{segmentLength} och \code{maxBy} och beskriv vad var och en kan användas till.

\Subtask Testa metoderna i REPL.

\SOLUTION

\TaskSolved \what~

\SubtaskSolved

\begin{itemize}[noitemsep]
  \item \code{lastIndexOf} är bra om man vill leta bakifrån i stället för framifrån; utan denna hade man annars då behövt använda \code{xs.reverse.indexOf(e)}
  \item \code{indexOfSlice(ys)} letar efter index där en hel sekvens \code{ys} börjar, till skillnad från \code{indexOf(e)} som bara letar efter ett enstaka element.
  \item \code{segmentLength(p, i)} ger längden på den längsta sammanhängande sekvens där alla element uppfyller predikatet \code{p} och sökningen efter en sådan sekvens börjar på plats \code{i}
  \item \code{xs.maxBy(f)} kör först funktionen \code{f} på alla element i \code{xs} och letar sedan upp det största värdet; motsvarande \code{minBy(f)} ger minimum av \code{f(e)} över alla element \code{e} i \code{xs}
\end{itemize}

\SubtaskSolved --

\QUESTEND



\clearpage

\AdvancedTasks %%%%%%%%%%%%%%%%%%%%%%%%%%%%%%%%%%%%%%%%%%%%%%%%%%%%%%%%%%%%%%%%%

\WHAT{Fixa svensk sorteringsordning av ÄÅÖ.}

\QUESTBEGIN

\Task \label{task:swedish-letter-ordering} \what~   Svenska bokstäver kommer i, för svenskar, konstig ordning om man inte vidtar speciella åtgärder. Med hjälp av klassen \code{java.text.Collator} kan man få en \code{Comparator} för strängar som följer lokala regler för en massa språk på planeten jorden.

\Subtask Verifiera att sorteringsordningen blir rätt i REPL enligt nedan.

\begin{REPL}
scala> val fel = Vector("ö","å","ä","z").sorted
scala> val svColl = java.text.Collator.getInstance(new java.util.Locale("sv"))
scala> val svOrd = Ordering.comparatorToOrdering(svColl)
scala> val rätt = Vector("ö","å","ä","z").sorted(svOrd)
\end{REPL}
\Subtask Använd metoden ovan för att skriva ett program som skriver ut raderna i en textfil i korrekt svensk sorteringsordning. Programmet ska kunna köras med kommandot:\\\texttt{scala sorted -sv textfil.txt}

\Subtask Läs mer här: \\
\noindent{\href{http://stackoverflow.com/questions/24860138/sort-list-of-string-with-localization-in-scala}{\small stackoverflow.com/questions/24860138/sort-list-of-string-with-localization-in-scala}}



\SOLUTION


\TaskSolved \what



\QUESTEND



\WHAT{Fibonacci-sekvens med ListBuffer.}

\QUESTBEGIN

\Task  \what~ Samlingen \code{ListBuffer} är en förändringsbar sekvens som är snabb och minnessnål vid tillägg i början \Eng{prepend}. Undersök vad som händer här:
\begin{REPL}
scala> val xs = scala.collection.mutable.ListBuffer.empty[Int]
scala> xs.prependAll(Vector(1, 1))
scala> while xs.head < 100 do {xs.prepend(xs.take(2).sum); println(xs)}
scala> xs.reverse.toList
\end{REPL}
Talen i sekvensen som produceras på rad 4 ovan kallas Fibonacci-tal  \footnote{\href{https://sv.wikipedia.org/wiki/Fibonaccital}{sv.wikipedia.org/wiki/Fibonaccital}} och blir snabbt mycket stora.

\Subtask Definera och testa följande funktion. Den ska internt använda förändringsbara \code{ListBuffer} men returnera en sekvens av oföränderliga \code{List}.

\begin{Code}
/** Ger en lista med tal ur Fibonacci-sekvensen 1, 1, 2, 3, 5, 8 ...
  * där det största talet är mindre än max. */
def fib(max: Long): List[Long] = ???
\end{Code}


\Subtask
Hur lång ska en Fibonacci-sekvens vara för att det sista elementet ska vara så nära \code{Int.MaxValue} som möjligt?


\Subtask Implementera \code{fibBig} som använder \code{BigInt} i stället för \code{Long} och låt din dator få använda sitt stora minne medan planeten värms upp en aning.

\SOLUTION

\TaskSolved \what


\SubtaskSolved

\begin{Code}
def fib(max: Long): List[Long] = 
  val xs = scala.collection.mutable.ListBuffer.empty[Long]
  xs.prependAll(Vector(1, 1))
  while xs.head < max do xs.prepend(xs.take(2).sum)
  xs.reverse.drop(1).toList
\end{Code}

\SubtaskSolved

\begin{REPL}
scala> fib(Int.MaxValue).size
val res0: Int = 46
\end{REPL}

\SubtaskSolved

\begin{Code}
def fibBig(max: BigInt): List[BigInt] =
  val xs = scala.collection.mutable.ListBuffer.empty[BigInt]
  xs.prependAll(Vector(BigInt(1), BigInt(1)))
  while xs.head < max do xs.prepend(xs.take(2).sum)
  xs.reverse.drop(1).toList
\end{Code}

\begin{REPL}
scala> fibBig(Long.MaxValue).size
val res0: Int = 92

scala> fibBig(BigInt(Long.MaxValue).pow(64)).size
val res1: Int = 5809

scala> fibBig(BigInt(Long.MaxValue).pow(128)).last
val res2: BigInt = 466572805528355449194553611102863153950720005186045547177525242118545194247268198196024304108711020686545660707513547993668927474420737702772726410095432646683782038269206733583562623144723659044965174192994997081915291671203135284448809948278794870130243195729759407652514927641622448506112336858244040087748168546825439555497978038066584506772917257705338472345660520902622305735366348690501583267086607109594118454398543160294999638070938386822164561738531661786873174424857409631803971069795886028284195109247953151499404937810249349132907101567724032186422592145774126660328936577771749713614176045435526886758975994177511201005911748503347657112775964769397750819976041389533451539207673441658345632507479241970993525868183091563469584756527454807108...

scala> fibBig(BigInt(Long.MaxValue).pow(128)).last.toString.size
val res3: Int = 2428

scala> fibBig(BigInt(Long.MaxValue).pow(256)).last.toString.size
val res4: Int = 4856

scala> fibBig(BigInt(Long.MaxValue).pow(1024)).last.toString.size
java.lang.OutOfMemoryError: Java heap space

\end{REPL}

\QUESTEND



\WHAT{Omvända sekvens på plats.}

\QUESTBEGIN

\Task \what~Implementera nedan algoritm i funktionen \code{reverseChars} och testa så att den fungerar för olika fall i REPL.


\begin{algorithm}[H]
 \SetKwInOut{Input}{Indata}\SetKwInOut{Output}{Resultat}

 \Input{En array $xs$ med tecken}
 \Output{Samma array med tecknen i omvänd ordning}
 $n \leftarrow$ antalet element i $xs$\\
 \For{$i \leftarrow 0$ \KwTo $\frac{n}{2} - 1$}{
  $temp \leftarrow xs(i)$ \\
  $xs(i) \leftarrow xs(n - i - 1)$ \\
  $xs(n - i - 1) \leftarrow temp$ \\
 }
\end{algorithm}

\SOLUTION

\TaskSolved \what~
\begin{Code}
def reverseChars(xs: Array[Char]): Unit =
  val n = xs.length
  for i <- 0 to (n/2 - 1) do
    val temp = xs(i)
    xs(i) = xs(n - i - 1)
    xs(n - i - 1) = temp
\end{Code}

\QUESTEND



\WHAT{Palindrompredikat.}

\QUESTBEGIN

\Task  \what~ En palindrom\footnote{\url{https://sv.wikipedia.org/wiki/Palindrom}} är ett ord som förblir oförändrat om man läser det baklänges. Exempel på palindromer: kajak, dallassallad.

Ett sätt att implementera ett palindrompredikat visas nedan:
\begin{Code}
def isPalindrome(s: String): Boolean = s == s.reverse
\end{Code}

\Subtask Implementationen ovan kan innebära att alla tecken i strängen gås igenom två gånger och behöver minnesutrymme för dubbla antalet tecken. Varför?

\Subtask Skapa ett palindromtest som går igenom elementen max en gång och som inte behöver extra minnesutrymme för en kopia av strängen. \emph{Lösningsidé:} Jämför parvis första och sista, näst första och näst sista, etc.

\SOLUTION

\TaskSolved \what

\SubtaskSolved Omvändning med \code{reverse} kan kräva genomgång av hela strängen en gång samt minnesutrymme för kopian. Innehållstestet kräver ytterligare en genomgång. (Detta är i och för sig inget stort problem eftersom världens längsta palindrom inte är längre än 19 bokstäver och är ett obskyrt finskt ord som inte ofta yttras i dagligt tal. Vilket?)

\SubtaskSolved

\begin{Code}
def isPalindrome(s: String): Boolean =
  val n = s.length
  var foundDiff = false
  var i = 0
  while i < n/2 && !foundDiff do
    foundDiff = s(i) != s(n - i - 1)
    i += 1
  !foundDiff
\end{Code}

\QUESTEND



\WHAT{Fler användbara sekvenssamlingsmetoder.}

\QUESTBEGIN

\Task \what~Sök på webben och läs om dessa metoder och testa dem i REPL:
\begin{itemize}[noitemsep]
  \item \code{xs.tabulate(n)(f)}
  \item \code{xs.forall(p)}
  \item \code{xs.exists(p)}
  \item \code{xs.count(p)}
  \item \code{xs.zipWithIndex}
\end{itemize}

\SOLUTION

\TaskSolved \what~
\begin{REPL}
scala> val xs = Vector.tabulate(10)(i => math.pow(2, i).toInt)
xs: Vector[Int] = Vector(1, 2, 4, 8, 16, 32, 64, 128, 256, 512)

scala> xs.forall(_ < 1024)
val res0: Boolean = true

scala> xs.exists(_ == 3)
val res1: Boolean = false

scala> xs.count(_ > 64)
val res2: Int = 3

scala> xs.zipWithIndex.take(5)
val res3: Vector[(Int, Int)] = Vector((1,0), (2,1), (4,2), (8,3), (16,4))
\end{REPL}
\QUESTEND






\WHAT{Arrays don't behave, but \code{ArraySeq}s do!}

\QUESTBEGIN

\Task \what~Även om \code{Array} är primitiv så finns smart krångel ''under huven'' i Scalas samlingsbibliotek för att arrayer ska bete sig nästan som ''riktiga'' samlingar. Därmed behöver man inte ägna sig åt olika typer av specialhantering, t.ex. s.k. boxning, wrapperklasser och typomvandlingar \Eng{type casting}, vilket man ofta behöver kämpa med som Java-programmerare.

Dock finns fortfarande begränsningar och anomalier vad gäller till exempel likhetstest. Om du vill att en array ska bete sig som andra samlingar kan du enkelt ''wrappa'' den med metoden \code{toSeq} som vid anrop på arrayer ger en \code{ArraySeq}. Denna beter sig som en helt vanlig oföränderlig sekvenssamling utan att offra snabbheten hos en primitiv array.
\begin{Code}
val as = Array(1,2,3)
val xs = as.toSeq
\end{Code}
\Subtask Hur fungerar likhetstest mellan två \code{ArraySeq}s? Vad har \code{xs} ovan för typ? Går det att uppdatera en wrappad array?

\Subtask Vilken typ av argumentsekvens får du tillgång till i kroppen för en funktion med repeterande parametrar?

\Subtask\Uberkurs Läs här:
\url{http://docs.scala-lang.org/overviews/collections/arrays.html}
och ge ett exempel på vad mer man inte kan göra med en array, förutom innehållslikhetstest.



\SOLUTION

\TaskSolved \what~

\SubtaskSolved \code{xs} erbjuder innehållslikhet och har typen \code{Seq[Int]} med den underliggande typen \code{ArraySeq[Int]}. Det går inte att göra tilldelning av element i en \code{ArraySeq} eftersom metoden \code{update} saknas, och den är oföränderlig. Den uppdateras därför inte när den urspringliga arrayen uppdateras.

\begin{REPL}
scala> val as1 = Array(1,2,3)
val as1: Array[Int] = Array(1, 2, 3)

scala> val as2 = Array(1,2,3)
val as2: Array[Int] = Array(1, 2, 3)


scala> val (xs1, xs2) = (as1.toSeq, as2.toSeq)
val xs1: Seq[Int] = ArraySeq(1, 2, 3)
val xs2: Seq[Int] = ArraySeq(1, 2, 3)

scala> as1 == as2
val res0: Boolean = false

scala> xs1 == xs2
val res1: Boolean = true

scala> as1(0) = 42

scala> xs1
val res2: Seq[Int] = ArraySeq(1, 2, 3)

scala> xs1(0) = 42
value update is not a member of Seq[Int]
\end{REPL}

\SubtaskSolved Vid repeterade parametrar får man en \code{ArraySeq}.

\begin{REPL}
scala> def f(xs: Int*) = xs
def f(xs: Int*): Seq[Int]

scala> println(f(1,2,3))
ArraySeq(1, 2, 3)
\end{REPL}


\SubtaskSolved Det går inte att ha en generisk array som funktionsresultat utan att bifoga kontextgränsen \code{ClassTag} i typparametern för att kompilatorn ska kunna generera kod för den typkonvertering som krävs under runtime av JVM. Se exempel här:\\
\url{http://docs.scala-lang.org/overviews/collections/arrays.html}


\QUESTEND




\WHAT{List eller Vector?}

\QUESTBEGIN

\Task\Uberkurs  \what~ Jämför tidskomplexitet mellan List och Vector vid hantering i början och i slutet, baserat på efterföljande REPL-session i din egen körmiljö.  Körningen nedan gjordes på en AMD Ryzen 7 5800X (16) @ 3.800GHz under Arch Linux 5.12.8-arch1-1 med Scala 3.0.1 och openjdk 11.0.11, men du ska använda det du har på din dator.

Hur snabbt går nedan på din dator? När är List snabbast och när är Vector snabbast? Hur stor är skillnaderna i prestanda?
\footnote{Denna typ av mätningar lär du dig mer om i LTH-kursen ''Utvärdering av programvarusystem'', som ges i slutet av årskurs 1 för Datateknikstudenter.}
%sudo lshw -class processor


\begin{CodeSmall}
> head -5 /proc/cpuinfo
processor    : 0
vendor_id    : AuthenticAMD
cpu family    : 25
model        : 33
model name    : AMD Ryzen 7 5800X 8-Core Processor

scala> def time(n: Int)(block: => Unit): Double =                  
     |   def now = System.nanoTime
     |   var timestamp = now
     |   var sum = 0L
     |   var i = 0
     |   while i < n do
     |     block
     |     sum = sum + (now - timestamp)
     |     timestamp = now
     |     i = i + 1
     |   val average = sum.toDouble / n
     |   println("Average time: " + average + " ns")
     |   average


// Exiting paste mode, now interpreting.

time: (n: Int)(block: => Unit)Double


scala> val n = 100000
scala> val l = List.fill(n)(math.random())
scala> val v = Vector.fill(n)(math.random())

scala> (for i <- 1 to 20 yield time(n){l.take(10)}).min
average time: 97.66952 ns
average time: 91.90033 ns
average time: 79.88311 ns
average time: 69.5963 ns
average time: 69.69892 ns
average time: 69.8033 ns
average time: 69.7705 ns
average time: 69.68491 ns
average time: 69.54222 ns
average time: 69.66051 ns
average time: 69.73661 ns
average time: 69.54112 ns
average time: 69.69141 ns
average time: 69.46341 ns
average time: 69.4098 ns
average time: 61.34162 ns
average time: 41.1333 ns
average time: 40.97051 ns
average time: 40.9075 ns
average time: 41.12321 ns
val res0: Double = 40.9075

scala> (for i <- 1 to 20 yield time(n){v.take(10)}).min
average time: 84.56978 ns
average time: 75.20167 ns
average time: 57.16529 ns
average time: 34.84469 ns
average time: 34.38478 ns
average time: 34.77709 ns
average time: 34.77179 ns
average time: 35.0506 ns
average time: 34.7967 ns
average time: 35.04258 ns
average time: 34.82559 ns
average time: 36.3673 ns
average time: 34.91029 ns
average time: 34.87239 ns
average time: 34.51958 ns
average time: 34.83949 ns
average time: 34.56169 ns
average time: 34.80719 ns
average time: 34.84459 ns
average time: 34.89468 ns
val res1: Double = 34.38478

scala> (for i <- 1 to 20 yield time(1000){l.takeRight(10)}).min
average time: 131365.106 ns
average time: 118632.787 ns
average time: 118440.066 ns
average time: 118687.567 ns
average time: 118428.487 ns
average time: 118871.686 ns
average time: 118964.797 ns
average time: 119030.236 ns
average time: 119262.534 ns
average time: 119228.344 ns
average time: 119226.494 ns
average time: 119310.933 ns
average time: 119352.854 ns
average time: 119121.913 ns
average time: 119133.664 ns
average time: 119015.193 ns
average time: 119276.674 ns
average time: 119224.882 ns
average time: 119301.771 ns
average time: 119444.401 ns
val res2: Double = 118428.487

scala> (for i <- 1 to 20 yield time(1000){v.takeRight(10)}).min
average time: 805.989 ns
average time: 365.219 ns
average time: 225.49 ns
average time: 125.92 ns
average time: 124.98 ns
average time: 130.689 ns
average time: 139.86 ns
average time: 128.29 ns
average time: 132.59 ns
average time: 125.729 ns
average time: 125.46 ns
average time: 130.59 ns
average time: 122.03 ns
average time: 121.9 ns
average time: 119.69 ns
average time: 120.48 ns
average time: 125.239 ns
average time: 126.09 ns
average time: 125.92 ns
average time: 125.91 ns
val res3: Double = 119.69

\end{CodeSmall}

\noindent Varför går olika rundor i for-loopen olika snabbt även om varje runda gör samma sak?

\SOLUTION

\TaskSolved
Sekvenssamlingen \code{List} är nästan dubbelt så snabb vid bearbetning i början men ungefär 1000 gånger långsammare vid bearbetning i slutet av en sekvens med 100000 element.


Olika körningar går olika snabbt på JVM bl.a. p.g.a optimeringar som sker när JVM-en ''värms upp'' och den så kallade Just-In-Time-kompileringen gör sitt mäktiga jobb. Det går ibland plötsligt väsentligt långsammare när skräpsamlaren tvingas göra tidsödande storstädning av minnet.

\QUESTEND






\WHAT{Tidskomplexitet för olika samlingar i Scalas standardbibliotek.}

\QUESTBEGIN

\Task\Uberkurs  \what~\\
Studera skillnader i tidskomplexitet mellan olika samlingar här: \\ \href{http://docs.scala-lang.org/overviews/collections/performance-characteristics.html}{docs.scala-lang.org/overviews/collections/performance-characteristics.html} \\
Läs även kritiken av förenklingar i ovan beskrivning här:\\
\href{http://www.lihaoyi.com/post/ScalaVectoroperationsarentEffectivelyConstanttime.html}{www.lihaoyi.com/post/ScalaVectoroperationsarentEffectivelyConstanttime.html}
\\
Läs denna grundliga empirisk genomgång av prestanda i Scalas samlingsbibliotek:\\
\href{http://www.lihaoyi.com/post/BenchmarkingScalaCollections.html}{www.lihaoyi.com/post/BenchmarkingScalaCollections.html}
\\Du får lära dig mer om hur man resonerar kring komplexitet i kommande kurser.


\SOLUTION

\TaskSolved --

\QUESTEND

%!TEX encoding = UTF-8 Unicode

%!TEX root = ../compendium1.tex

\Lab{\LabWeekSEVEN}

\begin{Goals}
\input{modules/w07-sequences-lab-goals.tex}
\end{Goals}

\begin{Preparations}
\item \DoExercise{\ExeWeekSEVEN}{07}
\item Läs igenom hela laborationen och säkerställ att du förstår hur SHUFFLE-algoritmen nedan fungerar.

\item Studera den givna koden i kursens workspace på GitHub här:\\
\url{https://github.com/lunduniversity/introprog/tree/master/workspace}

\end{Preparations}

\subsection{Bakgrund}\label{knuth-shuffle}

Denna uppgift handlar om kortblandning. Att blanda kort så att varje möjlig permutation (ordning som korten ligger i) är lika sannolik är icke-trivialt; en osystematisk blandning leder till en skev fördelning.

Givet en bra slumpgenerator går det att blanda en kortlek genom att lägga alla kort i en hög och sedan ta ett slumpvist kort från högen och lägga det överst i leken, tills alla kort ligger i leken. Fisher-Yates-algoritmen\footnote{\href{https://en.wikipedia.org/wiki/Fisher\%E2\%80\%93Yates_shuffle}{https://en.wikipedia.org/wiki/Fisher\%E2\%80\%93Yates\_shuffle}} (även kallad Knuth-shuffle), fungerar på det sättet. Här benämner vi denna algoritm SHUFFLE. Den återfinns i pseudokod nedan:

\begin{algorithm}[H]
 \SetKwInOut{Input}{Indata}
 \SetKwInOut{Output}{Utdata}
 \Input{Array $xs$ med $n$ st värden som ska blandas ''på plats''}
 \Output{$xs$ med sina värden omflyttade i slumpmässig ordning}
 \For{$i \leftarrow (n - 1)$ \KwTo $0$}{
  $r \leftarrow$ slumptal mellan $0$ och $i$ \\
  byt plats på $xs(i)$ och $xs(r)$
%  $temp \leftarrow xs(i)$ \\
%  $xs(i) \leftarrow xs(r)$ \\
%  $xs(r) \leftarrow temp$ \\
 }
\end{algorithm}

En kortlek \Eng{deck} har 52 kort, vart och ett med olika valör \Eng{rank} och färg (eng. \emph{suit}, på svenska även svit). Kortspelet poker handlar om att dra kort och få upp vissa kombinationer av kort, s.k. ''händer''\footnote{\href{https://sv.wikipedia.org/wiki/Pokerhand}{https://sv.wikipedia.org/wiki/Pokerhand}}. Dessa är ordnade från bättre till sämre; den spelare som får bäst hand vinner.
Det är därför intressant att veta med vilken sannolikhet en viss hand dyker upp vid dragning från en blandad kortlek.

De vanliga pokerhänderna är, i fallande värde, färgstege (straight flush), fyrtal, kåk (full house), färg (flush), stege (straight), triss, tvåpar och par. Dessa finns illustrerade i avsnitt \ref{shuffle:hands}.
Det finns ytterligare en hand, s.k. ''royal (straight) flush'' som betecknar en färgstege med ess som högsta kort, men dess sannolikhet är för låg för att man vid simulering kan förväntas påträffa den inom rimlig tid.

Under laborationen ska du börja med att göra klar den ofärdiga klassen \code{Deck} som visas nedan, och återfinns i workspace på GitHub.



Labbinstruktionerna i avsnitt \ref{subsection:lab:shuffle:tasks} ger tips om hur du ska ersätta \code{???} i givna kodskelett med med dina lösningar.
Med hjälp av klasserna \code{TestHand} och \code{TestDeck} kan du testa så att dina implementationer fungerar.

\begin{figure}
\scalainputlisting[numbers=left,basicstyle=\ttfamily\fontsize{10}{12}\selectfont]{../workspace/w07_shuffle/src/main/scala/cards/Card.scala}
\caption{Den färdigimplementerade, oföränderliga case-klassen \code{Card}.}
\label{shuffle:fig-card}
\end{figure}




När dina implementationer av metoderna \code{full} och \code{shuffle} fungerar ska du använda \code{Deck} i singelobjektet \code{PokerProbability} för att ta reda på sannolikheter för att olika pokerhänder uppkommer när man delar ut 5 kort ur en bra blandad kortlek.

Till din hjälp har du nedan kodfiler, där några har ofärdig kod som du ska färdigställa. All kod  ligger i ett paket med namnet \code{cards}.\footnote{Du kan bläddra bland klasserna i paketet cards här: \\
\href{https://github.com/lunduniversity/introprog/tree/master/workspace/w07_shuffle/src/main/scala/cards}{\mbox{\fontsize{9}{11}\selectfont  https://github.com/lunduniversity/introprog/tree/master/workspace/w07\_shuffle/src/main/scala/cards}}}

\begin{itemize}
\item \code{Card.scala} i fig. \ref{shuffle:fig-card} innehåller den färdigimplementerade case-klassen \code{Card} som representerar ett kort och har en koncis \code{toString} med valör och svit (färg).

\item \code{Deck.scala} i fig. \ref{shuffle:fig-deck} innehåller den förändringsbara klassen \code{Deck}, där du ska implementera kortblandning i metoden \code{shuffle}. Kompanjonsobjektet har metoder för att skapa kortlekar. Du ska implementera metoden \code{full} som skapar en fullständig kortlek med de 52 korten ordnade efter valör och färg.

\item \code{Hand.scala} i fig. \ref{shuffle:fig-hand} innehåller en case-klass \code{Hand} som representerar en pokerhand och har metoder för att avgöra vilken hand det är. I kompanjonsobjektet finns fabriksmetoder som kan skapa en ny hand från enskilda kort eller genom att dra kort ur en kortlek. Du ska implementera \code{tally} som registrerar antalet kort av en viss valör.

\item \code{PokerProbability.scala} i fig. \ref{shuffle:fig-pokerprob}  har en main-metod som räknar ut pokersannolikheter, samt hjälpmetoden \code{register} som du ska implementera.

\item \code{TestDeck.scala} ska du använda för att testa din implementation av \code{shuffle} med en kortlek som endast innehåller tre kort. Upprepade blandningar görs och förekomsten av varje möjlig permutation  registreras.

\item \code{TestHand.scala} har en \code{main}-metod som testar klassen \code{Hand}.

%\item \code{AsciiBarGraph.scala} innehåller enbart en metod som skapar ett stapeldiagram åt \code{TestingDeck}
\end{itemize}

\begin{figure}
\scalainputlisting[numbers=left,basicstyle=\ttfamily\fontsize{10}{12}\selectfont]{../workspace/w07_shuffle/src/main/scala/cards/Deck.scala}
\caption{Den ofärdiga klassen \code{Deck} med förändringsbar kortlek.}
\label{shuffle:fig-deck}
\end{figure}



\begin{figure}
\scalainputlisting[numbers=left,basicstyle=\ttfamily\fontsize{9}{10.5}\selectfont]{../workspace/w07_shuffle/src/main/scala/cards/Hand.scala}
\caption{Den ofärdiga, oföränderliga klassen \code{Hand} som representerar en pokerhand.}
\label{shuffle:fig-hand}
\end{figure}

\begin{figure}
\scalainputlisting[numbers=left,basicstyle=\ttfamily\fontsize{10}{12}\selectfont]{../workspace/w07_shuffle/src/main/scala/cards/PokerProbability.scala}
\caption{Det ofärdiga singelobjektet \code{PokerProbability} som tar reda på sannolikheter för olika pokerhänder.}
\label{shuffle:fig-pokerprob}
\end{figure}


\subsection{Obligatoriska uppgifter}\label{subsection:lab:shuffle:tasks}


\Task Implementera algoritmen SHUFFLE.

\Subtask Skapa din egen implementation av metoden \code{shuffle} i klassen \code{Deck}. Följ den givna algoritmen i stycke \ref{knuth-shuffle} noga. Du kan använd \code{cards.length} för att få fram längden på kortleken, men du kan gärna istället använda \code{cards.indices.reverse}. Implementera och använd metoden  \code{swap}.

\Subtask Kör \code{testShuffle} i \code{TestDeck} som kontrollerar att blandningen är jämnt fördelad genom att blanda en kortlek med tre kort och räkna hur ofta varje möjlig permutation dyker upp. Du bör få en utskrift med sex ($3!$) procentsatser som ska vara nästan lika.


\Task Skapa en fullständig, ordnad kortlek.

\Subtask Implementera metoden \code{full} som skapar en 52-korts standardlek ordnad efter färg och valör. Använd \code{Range}-värdena i kompanjonsobjektet \code{Card}.

\Subtask Kör \code{testCreate} i \code{TestDeck} och kontrollera så att du får kort av alla fyra färger, samt både ess och kungar.


\Task Gör färdigt och testa \code{Hand}.

\Subtask Implementera \code{tally} som ska ge en indexerbar sekvens med 14 platser där plats 1-13 innehåller antalet av respektive valör. (Plats 0 ska inte användas.)

\Subtask Testa klassen \code{Hand} med hjälp av \code{TestHand}.


\Task Ta fram sannolikheterna för ''straight flush'', ''straight'' och ''flush''.

\Subtask Implementera metoden \code{register} i \code{PokerProbability}. Använd \code{from} och \code{category} i \code{Hand} för att skapa och kategorisera en hand från en kortlek. Lagra frekvenserna i en lokal array som du, när resultatet är färdigt, gör om till en vektor med \code{toVector}.

\Subtask Kör \code{PokerProbability}, förslagsvis med en miljon iterationer. Du bör få ungefär dessa sannolikheter\footnote{\url{http://www.forum.gpcdata.se/pdf/poker.pdf}}:
\begin{figure}[H]\centering
\begin{tabular}{r|l}
\emph{hand} & $\emph{sannolikhet}$ \\ \hline
Straight flush & 0.00154\%  \\
Flush          & 0.197\%    \\
Straight       & 0.39\%     \\
\end{tabular}
\end{figure}

\Task Kopiera hela din lösning till en ny katalog och ändra implementationen så att du drar nytta av uppräknade datatyper med \code{enum} i stället för heltal och strängsekvenser. Diskutera med handledare för- och nackdelar med de två olika implementationerna. 

\subsection{Frivilliga extrauppgifter}


\Task Förbättra programmet så att simuleringen registrerar alla handkategorier utom Royal Flush. Kör sedan \code{PokerProbability} igen och notera sannolikheterna.

\Task Gör om alla metoder i case-klassen \code{Hand} till \code{lazy val} och undersök hur det påverkar exekveringstiden. Varför förändras prestanda med denna åtgärd? Hade denna optimering varit lämplig om klassen \code{Hand} vore förändringsbar? Varför?

\Task Gör så att även sannolikheten för Royal Flush kan simuleras. Det krävs i storleksordningen $10^8$ iterationer för en noggrannhet på 2 värdesiffror. Detta kan ta ca 5 minuter på en någorlunda snabb dator, så det kan vara läge före en paus under simuleringen...

\Task Implementera ett interaktivt kortspel, t.ex. någon enkel pokervariant. Börja med något mycket enkelt, till exempel högst-kort-vinner, och bygg vidare med sådant som du tycker verkar roligt.

Du kan t.ex. skapa en metod \code{def compareTo(other: Hand): Int} i case-klassen \code{Hand} som ger -1 om \code{other} är sämre, 0 om händerna är lika bra, och +1 om \code{other} är bättre. Du kan steg för steg göra så att det går att jämföra fler och fler händer enligt de specialregler som gäller för när olika händer anses bättre eller lika. Läs om reglerna här: \url{https://en.wikipedia.org/wiki/List_of_poker_hands}



\subsection{Bilder med exempel på olika pokerhänder}\label{shuffle:hands}

Figurerna \ref{lab:shuffle:first-picture} -- \ref{lab:shuffle:last-picture} visar bilder på olika korthänder i poker.

\newcommand{\CardWidth}{0.5\textwidth}
\newcommand{\CardCaptionWidth}{0.3\textwidth}

\begin{figure}[H]
 \begin{minipage}[c]{\CardWidth}
  \includegraphics[width=\textwidth]{../img/w05-hands/pair.png}
 \end{minipage}
 \begin{minipage}[c]{\CardCaptionWidth}
  \caption{Par: två kort har samma valör.}
   \label{lab:shuffle:first-picture}
 \end{minipage}
\end{figure}

\begin{figure}[H]
 \begin{minipage}[c]{\CardWidth}
  \includegraphics[width=\textwidth]{../img/w05-hands/twopair.png}
 \end{minipage}
 \begin{minipage}[c]{\CardCaptionWidth}
  \caption{Två par: handen har två \emph{olika} par.}
 \end{minipage}
\end{figure}

\begin{figure}[H]
 \begin{minipage}[c]{\CardWidth}
  \includegraphics[width=\textwidth]{../img/w05-hands/trips.png}
 \end{minipage}
 \begin{minipage}[c]{\CardCaptionWidth}
  \caption{Triss: tre kort har samma valör.}
 \end{minipage}
\end{figure}

\begin{figure}[H]
 \begin{minipage}[c]{\CardWidth}
  \includegraphics[width=\textwidth]{../img/w05-hands/straight.png}
 \end{minipage}
 \begin{minipage}[c]{\CardCaptionWidth}
  \caption{Stege: kortens valörer bildar en följd, ess kan vara antingen 1 eller 14.}
 \end{minipage}
\end{figure}

\begin{figure}[H]
 \begin{minipage}[c]{\CardWidth}
  \includegraphics[width=\textwidth]{../img/w05-hands/flush.png}
 \end{minipage}
 \begin{minipage}[c]{\CardCaptionWidth}
  \caption{Färg: alla kort har samma färg.}
 \end{minipage}
\end{figure}

\begin{figure}[H]
 \begin{minipage}[c]{\CardWidth}
  \includegraphics[width=\textwidth]{../img/w05-hands/fullhouse.png}
 \end{minipage}
 \begin{minipage}[c]{\CardCaptionWidth}
  \caption{Kåk: både triss och par.}
 \end{minipage}
\end{figure}

\begin{figure}[H]
 \begin{minipage}[c]{\CardWidth}
  \includegraphics[width=\textwidth]{../img/w05-hands/fours.png}
 \end{minipage}
 \begin{minipage}[c]{\CardCaptionWidth}
  \caption{Fyrtal: fyra kort har samma valör.}
 \end{minipage}
\end{figure}

\begin{figure}[H]
 \begin{minipage}[c]{\CardWidth}
  \includegraphics[width=\textwidth]{../img/w05-hands/straightflush.png}
 \end{minipage}
 \begin{minipage}[c]{\CardCaptionWidth}
  \caption{Färgstege: både stege och färg.}
 \end{minipage}
\end{figure}

\begin{figure}[H]
 \begin{minipage}[c]{\CardWidth}
  \includegraphics[width=\textwidth]{../img/w05-hands/none.png}
 \end{minipage}
 \begin{minipage}[c]{\CardCaptionWidth}
  \caption{Högt kort: inget mönster finns.}
 \label{lab:shuffle:last-picture}
  \end{minipage}
\end{figure}


\part{Lösningar}

\setcounter{chapter}{11} %L in \Alph
\renewcommand\thechapter{\Alph{chapter}}
\chapter{Lösningar till övningarna}\label{chapter:solutions}

\PreSolutionfalse

\let\QUESTBEGIN\ifPreSolution  % to mark formatting and numbering of exercises
\let\SOLUTION\else  % to mark solutions in the same file as questions
\let\QUESTEND\fi    % to mark end of exercise

%!TEX encoding = UTF-8 Unicode
%!TEX root = ../exercises.tex

\ifPreSolution
\Exercise{\ExeWeekONE}\label{exe:W01}

\begin{Goals}
%!TEX encoding = UTF-8 Unicode

\item Förstå vad som händer när satser exekveras och uttryck evalueras.
\item Förstå sekvens, alternativ och repetition.
\item Känna till litteralerna för enkla värden, deras typer och omfång.
\item Kunna deklarera och använda variabler och tilldelning, samt kunna rita bilder av minnessituationen då variablers värden förändras.
\item Förstå skillnaden mellan olika numeriska typer, kunna omvandla mellan dessa och vara medveten om noggrannhetsproblem som kan uppstå.
\item Förstå booleska uttryck och värdena \code{true} och \code{false}, samt kunna förenkla booleska uttryck.
\item Förstå skillnaden mellan heltalsdivision och flyttalsdivision, samt användning av rest vid heltalsdivision.
\item Förstå precedensregler och användning av parenteser i uttryck.
\item Kunna använda \code{if}-satser och \code{if}-uttryck.
\item Kunna använda \code{for}-satser och \code{while}-satser.
\item Kunna använda \code{math.random()} för att generera slumptal i olika intervaller.
\item Kunna beskriva skillnader och likheter mellan en procedur och en funktion.

\end{Goals}

\begin{Preparations}
\item \StudyTheory{01}
\item Du behöver en dator med Scala och Kojo installerad, se appendix~\ref{appendix:compile} och  \ref{appendix:kojo}.
\end{Preparations}

\else

\ExerciseSolution{\ExeWeekONE}

\fi  %%% END \ifPreSolution


\BasicTasks
%%%% TODO Strukturera övningen annorlunda: atomer, sammansatta uttryck, funktiner, kojo ??}


\def\what{\emph{Para ihop begrepp med beskrivning.}}

\QUESTBEGIN

\Task \what

\vspace{1em}\noindent Koppla varje begrepp med den (förenklade) beskrivning som passar bäst: 

\begin{ConceptConnections}
  litteral & 1 & & A & att översätta kod till exekverbar form \\ 
  sträng & 2 & & B & anger ett specifikt datavärde \\ 
  sats & 3 & & C & decimaltal med begränsad noggrannhet \\ 
  uttryck & 4 & & D & bra då antalet repetitioner ej är bestämt i förväg \\ 
  funktion & 5 & & E & vid anrop sker (sido)effekt; returvärdet är tomt \\ 
  procedur & 6 & & F & sker innan exekveringen startat \\ 
  exekveringsfel & 7 & & G & bra då antalet repetitioner är bestämt i förväg \\ 
  kompileringsfel & 8 & & H & för att ändra en variabels värde \\ 
  abstrahera & 9 & & I & sker medan programmet kör \\ 
  kompilera & 10 & & J & beskriver vad data kan användas till \\ 
  typ & 11 & & K & antingen sann eller falsk \\ 
  for-sats & 12 & & L & vid anrop beräknas ett returvärde \\ 
  while-sats & 13 & & M & en kodrad som gör något; kan särskiljas med semikolon \\ 
  tilldelning & 14 & & N & kombinerar värden och funktioner till ett nytt värde \\ 
  flyttal & 15 & & O & en sekvens av tecken \\ 
  boolesk & 16 & & P & att införa nya begrepp som förenklar kodningen \\ 
\end{ConceptConnections}

\SOLUTION

\TaskSolved \what

\begin{ConceptConnections}
  litteral & 1 & ~~\Large$\leadsto$~~ &  C & anger ett specifikt datavärde \\ 
  sträng & 2 & ~~\Large$\leadsto$~~ &  G & en sekvens av tecken \\ 
  sats & 3 & ~~\Large$\leadsto$~~ &  K & en kodrad som gör något; kan särskiljas med semikolon \\ 
  uttryck & 4 & ~~\Large$\leadsto$~~ &  N & kombinerar värden och funktioner till ett nytt värde \\ 
  funktion & 5 & ~~\Large$\leadsto$~~ &  L & vid anrop beräknas ett returvärde \\ 
  procedur & 6 & ~~\Large$\leadsto$~~ &  H & vid anrop sker (sido)effekt; returvärdet är tomt \\ 
  exekveringsfel & 7 & ~~\Large$\leadsto$~~ &  P & kan inträffa medan programmet kör \\ 
  kompileringsfel & 8 & ~~\Large$\leadsto$~~ &  A & kan inträffa innan exekveringen startat \\ 
  abstrahera & 9 & ~~\Large$\leadsto$~~ &  F & att införa nya begrepp som förenklar kodningen \\ 
  kompilera & 10 & ~~\Large$\leadsto$~~ &  B & att översätta kod till exekverbar form \\ 
  typ & 11 & ~~\Large$\leadsto$~~ &  D & beskriver vad data kan användas till \\ 
  for-sats & 12 & ~~\Large$\leadsto$~~ &  O & bra då antalet repetitioner är bestämt i förväg \\ 
  while-sats & 13 & ~~\Large$\leadsto$~~ &  E & bra då antalet repetitioner ej är bestämt i förväg \\ 
  tilldelning & 14 & ~~\Large$\leadsto$~~ &  I & för att ändra en variabels värde \\ 
  flyttal & 15 & ~~\Large$\leadsto$~~ &  M & decimaltal med begränsad noggrannhet \\ 
  boolesk & 16 & ~~\Large$\leadsto$~~ &  J & antingen sann eller falsk \\ 
\end{ConceptConnections}

\QUESTEND






\def\what{\emph{Utskrift i Scala REPL.}}

\QUESTBEGIN

\Task \what 

\vspace{1em}\noindent Starta Scala REPL \Eng{Read-Evaluate-Print-Loop}.

\begin{REPLnonum}
$ scala
Welcome to Scala version 2.11.8 (Java HotSpot(TM) 64-Bit Server VM, Java 1.8).
Type in expressions to have them evaluated.
Type :help for more information.

scala> 
\end{REPLnonum}

\Subtask Skriv efter prompten \code{scala>} en sats som skriver ut en valfri (bruklig/knasig) hälsningsfras, genom anrop av proceduren \code{println} med något strängargument. Tryck på \textit{Enter} så att satsen kompileras och exekveras. 

\Subtask Skriv samma sats igen (eller tryck pil-upp) men ''glöm bort'' att skriva högerparentesen efter argumentet innan du trycker på \textit{Enter}. Vad händer?

\begin{framed}
\noindent\emph{Tips inför fortsättningen:} Det finns många användbara kortkommandon och andra trix för att jobba snabbt i REPL. Be gärna någon som kan dessa trix att visa dig hur man kan jobba snabbare. Läs appendix \ref{appendix:compile:REPL} och prova sedan att kopiera och klistra in text. Använd piltangenterna för att bläddra i historiken och Ctrl+A för att komma till början av raden, Ctrl+K för att radera resten av raden, etc.
\end{framed}



\SOLUTION 
\TaskSolved \what

\SubtaskSolved Till exempel:
\begin{REPLnonum}
scala> println("hejsan svejsan")
\end{REPLnonum}

\SubtaskSolved Om högerparentes fattas får man fortsätta skriva på nästa rad. Detta indikeras med vertikalstreck i början av varje ny rad:
\begin{REPLnonum}
scala> println("hejsan svejsan"
     | + "!" 
     | )
hejsan svejsan!
\end{REPLnonum}

\QUESTEND



\def\what{\emph{Konkatenering av strängar.}}

\QUESTBEGIN

\Task \what

\Subtask Skriv ett uttryck som konkatenerar två strängar, t.ex. \code{"gurk"} och \code{"burk"}, med hjälp av operatorn \code{+} och studera resultatet. Vad har uttrycket för värde och typ? Vilken siffra står efter ordet \code{res} i variabeln som lagrar resultatet?

\Subtask Använd resultatet från konkateneringen, t.ex. \code{res0} (byt ev. ut \code{0}:an mot siffran efter \code{res} i utskriften från förra evalueringen), och skriv ett uttryck med hjälp av operatorn \code{*} som upprepar resultatet från förra deluppgiften 42 gånger. 


\SOLUTION

\TaskSolved \what

\SubtaskSolved 
\begin{REPLnonum}
scala> "gurk" + "burk"
res1: String = gurkburk
\end{REPLnonum}
värde: \code{"gurkburk"}, typ:  \code{String}

\SubtaskSolved
\begin{REPLnonum}
scala> res1 * 42
res2: String = gurkatomatgurkatomatgurkatomatgurkatomatgurkatomatgurkatomatgurkatomatgurkatomatgurkatomatgurkatomatgurkatomatgurkatomatgurkatomatgurkatomatgurkatomatgurkatomatgurkatomatgurkatomatgurkatomatgurkatomatgurkatomatgurkatomatgurkatomatgurkatomatgurkatomatgurkatomatgurkatomatgurkatomatgurkatomatgurkatomatgurkatomatgurkatomatgurkatomatgurkatomatgurkatomatgurkatomatgurkatomatgurkatomatgurkatomatgurkatomatgurkatomatgurkatomat
\end{REPLnonum}

\QUESTEND




\def\what{\emph{När upptäcks felet?}}

\QUESTBEGIN

\Task \what 

\Subtask Vad har uttrycket \code{ "hej" * 3 } för typ och värde? Testa i REPL.

\Subtask Byt ut 3:an ovan mot ett så pass stort heltal så att minnet blir fullt. Hur börjar felmeddelandet? Är detta ett körtidsfel eller ett kompileringsfel?

\Subtask Välj ett värde på argumentet efter operatorn \code{*} så att ett typfel genereras. Hur börjar felmeddelandet? Är detta ett körtidsfel eller ett kompileringsfel?

\begin{framed}
\noindent\emph{Tips inför fortsättningen:} Gör gärna fel när du kodar så lär du dig mer! Träna på att tolka olika felmeddelanden och fråga någon om hjälp om du inte förstår. Kompilatorns utskrifter kan vara till stor hjälp, men är ibland kryptiska. Om du kör fast och inte kommer vidare själv så be om hjälp, \emph{men be om tips snarare än färdiga lösningar} så att du behåller initiativet själv och tar kontroll över nästa steg i ditt lärande.
\end{framed}


\SOLUTION

\TaskSolved \what

\SubtaskSolved Typ: \code{String}, värde: \code{"hejhejhej"}

\SubtaskSolved Körtiddsfel:
\begin{REPLnonum}
scala> "hej" * Int.MaxValue
java.lang.OutOfMemoryError: Java heap space
\end{REPLnonum}

\SubtaskSolved Kompileringsfel: (indikeras av texten \code{<console> ... error:})
\begin{REPLnonum}
scala> "hej" * true
<console>:12: error: type mismatch;
 found   : Boolean(true)
 required: Int
       "hej" * true
\end{REPLnonum}


\QUESTEND




\def\what{\emph{Litteraler och typer.}}

\QUESTBEGIN

\Task \what

\Subtask Ta hjälp av REPL-kommadot \verb+:type+ (kan förkortas \code{:t}) vid behov för att para ihop nedan litteraler med rätt typ. 

\begin{ConceptConnections}[0.35\textwidth]
  \code|1    | & 1 & & A & \code|Char   | \\ 
  \code|1L   | & 2 & & B & \code|Double | \\ 
  \code|1.0  | & 3 & & C & \code|Boolean| \\ 
  \code|1D   | & 4 & & D & \code|Int    | \\ 
  \code|1F   | & 5 & & E & \code|Boolean| \\ 
  \code|'1'  | & 6 & & F & \code|Double | \\ 
  \code|"1"| & 7 & & G & \code|Long   | \\ 
  \code|true | & 8 & & H & \code|Float  | \\ 
  \code|false| & 9 & & I & \code|String | \\ 
  \code|()   | & 10 & & J & \code|Unit   | \\ 
%\Connect{\code|1      |}  {\code|Int    |}
%\Connect{\code|1L     |}  {\code|Long   |}
%\Connect{\code|1.0    |}  {\code|Double |}
%\Connect{\code|1D     |}  {\code|Double |}
%\Connect{\code|1F     |}  {\code|Float  |}
%\Connect{\code|'1'    |}  {\code|Char   |}
%\Connect{\code|\"1\"  |}  {\code|String |}
%\Connect{\code|true   |}  {\code|Boolean|} 
%\Connect{\code|false  |}  {\code|Boolean|} 
%\Connect{\code|()     |}  {\code|Unit   |} 
\end{ConceptConnections}

\Subtask Vad händer om du adderar 1 till det största möjliga värdet av typen \code{Int}? 
\\\emph{Tips:} se snabbreferensen \footnote{\url{http://cs.lth.se/pgk/quickref/}} under rubriken ''The Scala type system'' avsnitt ''Methods on numbers''.

\Subtask Vad är skillnaden mellan typerna \code{Long} och \code{Int}?

\Subtask Vad är skillnaden mellan typerna \code{Double} och \code{Float}?


\SOLUTION

\TaskSolved \what

\SubtaskSolved 

\begin{ConceptConnections}
  \code|1    | & 1 & ~~\Large$\leadsto$~~ &  D & \code|Int    | \\ 
  \code|1L   | & 2 & ~~\Large$\leadsto$~~ &  J & \code|Long   | \\ 
  \code|1.0  | & 3 & ~~\Large$\leadsto$~~ &  G & \code|Double | \\ 
  \code|1D   | & 4 & ~~\Large$\leadsto$~~ &  C & \code|Double | \\ 
  \code|1F   | & 5 & ~~\Large$\leadsto$~~ &  B & \code|Float  | \\ 
  \code|'1'  | & 6 & ~~\Large$\leadsto$~~ &  H & \code|Char   | \\ 
  \code|"1"| & 7 & ~~\Large$\leadsto$~~ &  A & \code|String | \\ 
  \code|true | & 8 & ~~\Large$\leadsto$~~ &  E & \code|Boolean| \\ 
  \code|false| & 9 & ~~\Large$\leadsto$~~ &  F & \code|Boolean| \\ 
  \code|()   | & 10 & ~~\Large$\leadsto$~~ &  I & \code|Unit   | \\ 
%\ConnectSolved{\code|1      |}  {\code|Int    |}
%\ConnectSolved{\code|1L     |}  {\code|Long   |}
%\ConnectSolved{\code|1.0    |}  {\code|Double |}
%\ConnectSolved{\code|1D     |}  {\code|Double |}
%\ConnectSolved{\code|1F     |}  {\code|Float  |}
%\ConnectSolved{\code|'1'    |}  {\code|Char   |}
%\ConnectSolved{\code|\"1\"  |}  {\code|String |}
%\ConnectSolved{\code|true   |}  {\code|Boolean|} 
%\ConnectSolved{\code|false  |}  {\code|Boolean|} 
\end{ConceptConnections}

\SubtaskSolved Värdet går över gränsen för vad som får plats i ett 32 bitars heltal och ''börjar om'' på det minsta möjliga heltalet \code{Int.MinValue}
\begin{REPL}
scala> Int.MaxValue + 1
res3: Int = -2147483648

scala> Int.MinValue
res4: Int = -2147483648
\end{REPL}

\SubtaskSolved Båda är heltal men \code{Long} kan representera större tal än \code{Int}.

\SubtaskSolved Båda är flyttal men \code{Double} har dubbel precision och kan representera större tal med fler decimaler.



\QUESTEND





\def\what{\emph{Matematiska funktioner. Scaladoc.}}

\QUESTBEGIN

\Task \what

\Subtask Antag att du har ett schackbräde med 64 rutor. Tänk dig att du börjar med ett enda riskorn på första rutan och sedan lägger dubbelt så många riskorn i en ny hög för varje efterföljande ruta: 1, 2, 4, 8, ...  etc. Hur många riskorn\footnote{\url{https://en.wikipedia.org/wiki/Wheat_and_chessboard_problem}} blir det då i den sextiofjärde rishögen?

\emph{Tips:} Du ska beräkna $2^{64} - 1$. Om du skriver \code{math.} i REPL och trycker TAB får du se inbyggda matematiska funktioner i Scalas standardbibliotek:
\begin{REPL}
scala> math.    // Tryck TAB direkt efter punkten och betrakta listan
\end{REPL}
Använd funktionen \code{math.pow} och lämpliga argument. Om du skriver \code{math.pow} och trycker TAB \emph{två gånger} får du se funktionshuvudet med parameterlistan. 

Om du surfar till \url{http://www.scala-lang.org/api/current/} och skriver \code{math} i sökrutan och sedan, efter att du klickat på \textbf{\textsf{\small scala.math}}, skriver \textbf{\textsf{\small pow}} i rutan längre ner, så filtreras sidan och du hittar dokumentationen av \code{ def pow } som du kan klicka på och läsa mer om.   

\Subtask Definiera funktionen \code{omkrets} nedan i REPL. Går det bra att utelämna returtyp-annoteringen? Varför? Finns det anledning att ha den kvar?
\begin{Code}
def omkrets(radie: Double): Double = 2 * math.Pi * radie
\end{Code}

\Subtask Jordens (genomsnittliga) diameter (vid ekvatorn) är ca $12 750$ $km$. Anropa funktionen \code{omkrets} ovan för att beräkna hur många kilometer per dag man ungefär måste färdas om man vill åka jorden runt på 80 dagar. 

\SOLUTION

\TaskSolved \what

\SubtaskSolved Ja, returtyp-annoteringen \code{: Double} kan utelämnas. 

\begin{itemize}
\item Varför kan returtyp utelämnas?\\Eftersom kompilatorns typhärledning kan härleda returtypen. 
\item Varför kan man vilja utelämna den?\\Det blir kortare att skriva utan. 
\item Anledningar att ange returtyp: 
\begin{itemize}
\item  Med explicit returtyp får du hjälp av kompilatorn att redan under kompileringen kontrollera att uttrycket till höger om likhetstecknet har den typ som förväntas. 

\item Genom att du anger returtypen explicit får de som enbart läser metodhuvudet (och inte implementationen)
 tydligt se vad som returneras.
\end{itemize}
\end{itemize}	


\SubtaskSolved Beräkning av $2^{64} - 1$ med \code{math.pow} enligt nedan ger ungefär $1.8 \cdot 10^{19}$
\begin{REPL}
scala> math.pow(2, 64) - 1
res0: Double = 1.8446744073709552E19
\end{REPL}


\SubtaskSolved Ca $500$ $km$.
\begin{REPL}
scala> omkrets(12750 / 2) / 80
res0: Double = 500.6913291658733
\end{REPL}

\QUESTEND




\def\what{\emph{Förändringsbara variabler och tilldelning.}}

\QUESTBEGIN

\Task \what~Rita en \emph{ny} bild av datorns minne efter \emph{varje} exekverad rad 1--6 nedan. Varje bild ska visa alla variabler som finns i minnet och deras variabelnamn, typ och värde.

\begin{REPL}[numbers=left, numberstyle=\color{black}\ttfamily\scriptsize\selectfont]
scala> var a = 13
scala> var b = a + 1
scala> var c = (a + b) * 2.0
scala> b = 0
scala> a = 0
scala> c = c + 1
\end{REPL}
Efter första raden ser minnessituationen ut så här:

\MEM{a}{Int}{13}

\SOLUTION

\TaskSolved \what

\begin{tabular}{l l l}
\MEM{{\it Efter rad1:~~~~} a}{Int}{13}\\
\MEM{{\it Efter rad2:~~~~} a}{Int}{13} & \MEM{b}{Int}{14}\\
\MEM{{\it Efter rad3:~~~~} a}{Int}{13} & \MEM{b}{Int}{14} & \MEM{c}{Double}{54.0}\\
\MEM{{\it Efter rad4:~~~~} a}{Int}{13} & \MEM{b}{Int}{0} & \MEM{c}{Double}{54.0}\\
\MEM{{\it Efter rad5:~~~~} a}{Int}{0} & \MEM{b}{Int}{0} & \MEM{c}{Double}{54.0}\\
\MEM{{\it Efter rad6:~~~~} a}{Int}{0} & \MEM{b}{Int}{0} & \MEM{c}{Double}{55.0}\\
\end{tabular}

\QUESTEND


\def\what{\emph{Slumptal med \code{math.random}.}}

\QUESTBEGIN

\Task\label{exercise:expressions:roll} \what

\Subtask Vad ger funktionen \code{math.random} för resultatvärde? Vilken typ? Vad är största och minsta möjliga värde?
\\\emph{Tips:} Se scaladoc här: \Scaladoc och prova i REPL.

\Subtask Deklarera den parameterlösa funktionen \code{def roll: Int = ???} som ska representera ett tärningskast och ge ett slumpmässigt heltal mellan 1 och 6. Testa funktionen genom att anropa den många gånger. \\\emph{Tips:} Använd \code{math.random} och multiplicera och addera med lämpliga heltal. Omge beräkningen med parenteser och avsluta med \code{.toInt} för att avkorta decimaler och omvandla typen från \code{Double} till \code{Int}.

\SOLUTION

\TaskSolved \what

\SubtaskSolved Ur dokumentationen:
\begin{Code}
/** Returns a Double value with a positive sign, 
 *  greater than or equal to 0.0 and less than 1.0.
 */
def random(): Double
\end{Code}


\SubtaskSolved 
\begin{REPL}
scala> def roll: Int = (math.random * 6 + 1).toInt

scala> roll
res0: Int = 4

scala> roll
res1: Int = 1
\end{REPL}

\QUESTEND




\def\what{\emph{Repetition med \code{for}, \code{foreach} och \code{while}.}}

\QUESTBEGIN

\Task \what

\Subtask Så här kan en \code{for}-sats ser ut: 
\begin{Code}
for (i <- 1 to 10) print(i + ", ")
\end{Code}
Använd en \code{for}-sats för att skriva ut resultatet av 100 tärningskast med funktionen \code{roll} från uppgift \ref{exercise:expressions:roll}. 

\Subtask Så här kan en \code{foreach}-sats ser ut: 
\begin{Code}
(1 to 10).foreach { i => print(i + ",") }
\end{Code}
Använd en \code{foreach}-sats för att skriva ut resultatet av 100 tärningskast med funktionen \code{roll} från uppgift \ref{exercise:expressions:roll}. 

\Subtask Så här kan en \code{while}-sats ser ut: 
\begin{Code}
var i = 1
while (i <= 10) { print(i + ","); i = i + 1 }
\end{Code}
Använd en \code{while}-sats för att skriva ut resultatet av 100 tärningskast med funktionen \code{roll} från uppgift \ref{exercise:expressions:roll}. Vad händer om du glömmer \code{i = i + 1} ?


\SOLUTION

\TaskSolved \what

\SubtaskSolved \TODO

\QUESTEND


\def\what{\emph{Alternativ med \code{if}-sats och \code{if}-uttryck.}}

\QUESTBEGIN

\Task \what

\Subtask Så här kan en \code{if}-sats se ut (notera dubbla likhetstecken):
\begin{Code}
if (roll == 3) println("TRE") else println("INTE TRE") 
\end{Code}
Testa ovan i REPL. Skriv sedan en \code{for}-sats som kastar 100 tärningar och skriver ut strängen \code{"GRATTIS!"} om det blir en sexa, annars en ledsen smiley: \code{":("} 

\Subtask Så här kan ett \code{if}-uttryck se ut:
\begin{Code}
if (roll < 6) 0 else 1 
\end{Code}
Testa ovan i REPL. Skriv sedan en \code{while}-sats som kastar 100 tärningar och räknar antalet sexor. 

\SOLUTION

\TaskSolved \what

\SubtaskSolved \TODO

\QUESTEND



\def\what{\emph{Sekvens, sats och procedur.}}

\QUESTBEGIN

\Task \what

\Subtask Vad gör dessa satser? 
\begin{REPLnonum}
scala> def p = { print("san"); print("!"); println("hej")}
scala> p;p;p;p
\end{REPLnonum}

\Subtask
Använd pil-upp för att få tillbaka raden du skrev med definitionen av proceduren \code{p}. Byt plats på strängarna i utskriftsanropen i proceduren \code{p} så att utskriften blir: 
\begin{REPLnonum}
hejsan!
hejsan!
hejsan!
hejsan!
\end{REPLnonum}

\Subtask Hur tolkar kompilatorn klammerparenteser och semikolon?

\SOLUTION

\TaskSolved \what

\SubtaskSolved 
Satserna skapar denna utskrift:
\begin{REPLnonum}
san!hej
san!hej
san!hej
san!hej
\end{REPLnonum}

\SubtaskSolved 
\begin{REPLnonum}
scala> def p = { print("hej"); print("san"); println("!")}
scala> p;p;p;p
\end{REPLnonum}

\SubtaskSolved 
\begin{itemize}
\item Klammerparenteser används för att gruppera flera satser. Klammerparenteser behövs om man vill definiera en funktion som består av mer än en sats.  

\item Semikolon särskiljer flera satser. Semikolon behövs om man vill skriva många satser på samma rad.


\end{itemize}

\QUESTEND




\def\what{\emph{Heltalsdivision.}}

\QUESTBEGIN

\Task \what~Vilket värde och vilken typ hör till vilket uttryck?  Är du osäker på svaret, testa i REPL.

\begin{ConceptConnections}[0.3\textwidth]
  \code| 4 / 42      | & 1 & & A & \code|true : Boolean  | \\ 
  \code| 42.0 / 2    | & 2 & & B & \code|    2: Int      | \\ 
  \code| 42 / 4      | & 3 & & C & \code| 10.5: Double   | \\ 
  \code| 42 % 4      | & 4 & & D & \code|   10: Int      | \\ 
  \code| 4 % 42      | & 5 & & E & \code|    0: Int      | \\ 
  \code| 40 % 4 == 0 | & 6 & & F & \code|false: Boolean  | \\ 
  \code| 42 % 4 == 0 | & 7 & & G & \code|    4: Int      | \\ 
\end{ConceptConnections}

\SOLUTION

\TaskSolved \what

\begin{ConceptConnections}[0.3\textwidth]
  \code| 4 / 42      | & 1 & ~~\Large$\leadsto$~~ &  A & \code|    0: Int      | \\ 
  \code| 42.0 / 2    | & 2 & ~~\Large$\leadsto$~~ &  G & \code| 10.5: Double   | \\ 
  \code| 42 / 4      | & 3 & ~~\Large$\leadsto$~~ &  E & \code|   10: Int      | \\ 
  \code| 42 % 4      | & 4 & ~~\Large$\leadsto$~~ &  C & \code|    2: Int      | \\ 
  \code| 4 % 42      | & 5 & ~~\Large$\leadsto$~~ &  F & \code|    4: Int      | \\ 
  \code| 40 % 4 == 0 | & 6 & ~~\Large$\leadsto$~~ &  D & \code|true : Boolean  | \\ 
  \code| 42 % 4 == 0 | & 7 & ~~\Large$\leadsto$~~ &  B & \code|false: Boolean  | \\ 
\end{ConceptConnections}

\QUESTEND





\def\what{\emph{Booleska värden.}}

\QUESTBEGIN

\Task \what~Vilket värde har dessa uttryck?  % Uppgift 13

\Subtask \code{true && true}

\Subtask \code{false && true}

\Subtask \code{true || true}

\Subtask \code{false || true}

\Subtask \code{false || false}

\Subtask \code{true == true}

\Subtask \code{true != false}

\Subtask \code{true > false}

\Subtask \code{true && (1 / 0 > 1)}

\Subtask \code{false && (1 / 0 > 1)}

\SOLUTION

\TaskSolved \what

\SubtaskSolved \code{true}

\SubtaskSolved \code{false}

\SubtaskSolved \code{false}

\SubtaskSolved \code{true}

\SubtaskSolved \code{true}

\SubtaskSolved \code{false}

\SubtaskSolved \code{true}

\SubtaskSolved \code{true}

\SubtaskSolved Undantag kastas: \code{java.lang.ArithmeticException: / by zero}

\SubtaskSolved \code{false}

\QUESTEND





\def\what{\emph{Booleska variabler.}}

\QUESTBEGIN

\Task \what~Vad skrivs ut på rad 2 och 4 nedan?

\begin{REPL}
scala> var monster = false
scala> if (monster) println("akta dig!!!")
scala> monster = true
scala> if (monster) println("akta dig!!!")
\end{REPL}

\SOLUTION

\TaskSolved \what

\begin{itemize}
\item[Rad 2:] Ingenting skrivs ut.
\item[Rad 4:] \code{akta dig!!!}
\end{itemize}


\QUESTEND






\def\what{\emph{Turtle graphics med Kojo.}}

\QUESTBEGIN

\Task \what~På veckans laboration ska du använda Kojo för att verifiera att du kan använda sekvens, alternativ, repetition och abstraktion. Med Kojo kan du rita färgglada figurer med hjälp av ett lättanvänt Scala-bibliotek för \emph{turtle graphics}\footnote{\url{https://en.wikipedia.org/wiki/Turtle_graphics}}. 

Starta Kojo (se appendix \ref{appendix:kojo}). Om du inte redan har svenska menyer: välj svenska i språkmenyn och starta om Kojo.  Skriv in nedan program och tryck på den \emph{gröna} play-knappen. Notera kopplingen mellan satssekvensen och vad som händer i ritfönstret.

\begin{Code}
sudda

fram; höger
fram; vänster
färg(grön)
fram
\end{Code}
\noindent


\Subtask Vad händer om du \emph{inte} börjar programmet med \code{sudda} och kör samma program upprepade gånger? Varför är det bra att börja programmet med \code{sudda}?

\Subtask Skriv kod som ritar en kvadrat enligt bilden nedan.
\vspace{1em}\\\includegraphics[width=0.47\textwidth]{../img/kojo/kvadrat}

\noindent Prova gärna olika sätt att skriva din kod \emph{utan} att resultatet ändras: skriv satser i sekvens på flera rader eller satser i sekvens på samma rad med semikolon emellan; använd blanktecken och blanka rader i koden. Hur vill du gruppera dina satser så att de är lätta för en människa att läsa?
%Prova att ändra på \emph{ordningen} mellan satserna och studera hur resultatet påverkas. Använd den \emph{gula} play-knappen  (programspårning) för att studera exekveringen i detalj. Vad händer du klickar på satser i ditt program och på rutor i programspårningen?


\Subtask Rita en trappa enligt bilden nedan.

\includegraphics[width=0.3\textwidth]{../img/kojo/stairs}

\Subtask Rita valfri bild på valfri bakgrund med hjälp av några av procedurerna i tabellen nedan. Du kan till exempel rita en rosa triangel med lila konturer mot svart bakgrund. % \ref{lab:kojo:kojo-procedures}. 
Försök att underlätta läsbarheten av din kod med hjälp av lämpliga radbrytningar och gruppering av satser. 


\begin{table}[H]
\begin{longtable}{l l}\small
\code|fram(100)| & Paddan går framåt 100 steg (25 om argument saknas).\\
\code|färg(rosa)| & Sätter pennans färg till rosa. \\
\code|fyll(lila)| & Sätter ifyllnadsfärgen till lila. \\
\code|fyll(genomskinlig)| & Gör så att paddan \emph{inte} fyller i något när den ritar. \\
\code|bredd(20)| & Gör så att pennan får bredden 20. \\
\code|bakgrund(svart)| & Bakgrundsfärgen blir svart. \\
\code|bakgrund2(grön,gul)| & Bakgrund med övergång från grönt till gult. \\
\code|pennaNer|  & Sätter ner paddans penna så att den ritar när den går. \\
\code|pennaUpp|  & Sänker paddans penna så att den \emph{inte} ritar när den går. \\
\code|höger(45)|   & Paddan vrider sig 45 grader åt höger. \\
\code|vänster(45)| & Paddan vrider sig 45 grader åt vänster. \\
\code|hoppa|       & Paddan hoppar 25 steg utan att rita. \\
\code|hoppa(100)|  & Paddan hoppar 100 steg utan att rita. \\
\code|hoppaTill(100, 200)| & Paddan hoppar till läget (100, 200) utan att rita. \\
\code|gåTill(100, 200)|    & Paddan vrider sig och går till läget (100, 200). \\
\code|öster|   & Paddan vrider sig så att nosen pekar åt höger. \\
\code|väster|  & Paddan vrider sig så att nosen pekar åt vänster. \\
\code|norr|    & Paddan vrider sig så att nosen pekar uppåt. \\
\code|söder|   & Paddan vrider sig så att nosen pekar neråt. \\
\code|mot(100,200)|   & Paddan vrider sig så att nosen pekar mot läget (100, 200) \\
\code|sättVinkel(90)| & Paddan vrider nosen till vinkeln 90 grader. \\
\end{longtable}
%\label{lab:kojo:kojo-procedures}
%\caption{Några användbara procedurer i Kojo.}
\end{table}

\begin{framed}
\noindent\emph{Tips inför fortsättningen:} Ha gärna både REPL och Kojo igång samtidigt. Då kan du undersöka hur olika kodkonstruktioner fungerar i REPL, medan du stegvis skapar allt större program i editorn i Kojo. Detta sätt att jobba har du nytta av under resten av kursen, både om du använder en texteditor och kompilerar i terminalen, och om du använder en professionell integrerad utvecklingsmiljö. Oavsett vilka andra verktyg du kör är det användbart att ha REPL igång i ett eget fönster som hjälp i den kreativa processen, medan du jagar buggar och medan du lär dig nya koncept. Så fort du undrar hur något fungerar i Scala: fram med REPL och testa!
\end{framed}


\SOLUTION

\TaskSolved \what
 
\SubtaskSolved Genom att börja din Kojo-program med \code{sudda} så startar du exekveringen i samma utgångsläge: en tom rityta \Eng{canvas} där paddan pekar uppåt, pennan är nere och pennans färg är röd.  Då blir det lättare att resonera om vad programmet gör från början till slut, jämfört med om exekveringen beror på resultatet av tidigare exekveringar.


\SubtaskSolved
\begin{Code}
sudda

fram; vänster
fram; vänster
fram; vänster
fram; vänster
\end{Code}


\SubtaskSolved
\begin{Code}
sudda

fram; vänster
fram; höger

fram; vänster
fram; höger

fram; vänster
fram; höger

fram; vänster
\end{Code}


\QUESTEND









\clearpage

\ExtraTasks %%%%%%%%%%%%%%%%%% EXTRAUPPGIFTER



\def\what{\emph{Typ och värde.}}

\QUESTBEGIN

\Task \what~Vilket värde och vilken typ hör till vilket uttryck?  Är du osäker på svaret, testa i REPL.

\begin{ConceptConnections}[0.3\textwidth]
  \code|1.0 + 18          | & 1 & & A & \code|" ": String   | \\ 
  \code|(41 + 1).toDouble | & 2 & & B & \code|19.0: Double    | \\ 
  \code|1.042e42 + 1      | & 3 & & C & \code|57: Int         | \\ 
  \code|12E6.toLong       | & 4 & & D & \code|42.0: Double    | \\ 
  \code|32.toChar.toString| & 5 & & E & \code|48: Int         | \\ 
  \code|'A'.toInt         | & 6 & & F & \code|0: Int          | \\ 
  \code|0.toInt           | & 7 & & G & \code|1.042E42: Double| \\ 
  \code|'0'.toInt         | & 8 & & H & \code|'*': Char       | \\ 
  \code|'9'.toInt         | & 9 & & I & \code|12000000: Long  | \\ 
  \code|'A' + '0'         | & 10 & & J & \code|65: Int         | \\ 
  \code|('A' + '0').toChar| & 11 & & K & \code|'q': Char       | \\ 
  \code|"*!%#".charAt(0)| & 12 & & L & \code|113: Int        | \\ 
\end{ConceptConnections}

\SOLUTION

\TaskSolved \what

\begin{ConceptConnections}
  \code|1.0 + 18          | & 1 & ~~\Large$\leadsto$~~ &  B & \code|19.0: Double    | \\ 
  \code|(41 + 1).toDouble | & 2 & ~~\Large$\leadsto$~~ &  D & \code|42.0: Double    | \\ 
  \code|1.042e42 + 1      | & 3 & ~~\Large$\leadsto$~~ &  G & \code|1.042E42: Double| \\ 
  \code|12E6.toLong       | & 4 & ~~\Large$\leadsto$~~ &  I & \code|12000000: Long  | \\ 
  \code|32.toChar.toString| & 5 & ~~\Large$\leadsto$~~ &  A & \code|" ": String   | \\ 
  \code|'A'.toInt         | & 6 & ~~\Large$\leadsto$~~ &  J & \code|65: Int         | \\ 
  \code|0.toInt           | & 7 & ~~\Large$\leadsto$~~ &  F & \code|0: Int          | \\ 
  \code|'0'.toInt         | & 8 & ~~\Large$\leadsto$~~ &  E & \code|48: Int         | \\ 
  \code|'9'.toInt         | & 9 & ~~\Large$\leadsto$~~ &  C & \code|57: Int         | \\ 
  \code|'A' + '0'         | & 10 & ~~\Large$\leadsto$~~ &  L & \code|113: Int        | \\ 
  \code|('A' + '0').toChar| & 11 & ~~\Large$\leadsto$~~ &  K & \code|'q': Char       | \\ 
  \code|"*!%#".charAt(0)| & 12 & ~~\Large$\leadsto$~~ &  H & \code|'*': Char       | \\ 
\end{ConceptConnections}

%\Subtask \code{1.0 + 18}
%
%\Subtask \code{(41 + 1).toDouble}
%
%\Subtask \code{1.042e42 + 1}
%
%\Subtask \code{12E6.toLong}
%
%\Subtask \code{"gurk" + 'a'}
%
%\Subtask \code{32.toChar.toString}
%
%\Subtask \code{'A'.toInt}
%
%\Subtask \linebreak[0] \code{'0'.toInt}
%
%\Subtask \code{'0'.toInt}
%
%\Subtask \code{'9'.toInt}
%
%\Subtask \code{'A' + '0'}
%
%\Subtask \code{('A' + '0').toChar}
%
%\Subtask \code{"*!%#".charAt(0)}
%%%%%%%%%%%%%%%%%%%%%%%%%%%%%%%%%%%%%%%%%%%%%%%%
%\SubtaskSolved \code{Double, 19}
%
%\SubtaskSolved \code{Double, 42}
%
%\SubtaskSolved \code{Double, 1.042E42}
%
%\SubtaskSolved \code{Long, 12000000}
%
%\SubtaskSolved \code{String, gurka}
%
%\SubtaskSolved \code{String, " "}
%
%\SubtaskSolved \code{Int, 65}
%
%\SubtaskSolved \code{Int, 48}
%
%\SubtaskSolved \code{Int,49}
%
%\SubtaskSolved \code{Int,57}
%
%\SubtaskSolved \code{Int, 113}
%
%\SubtaskSolved \code{Char, 'q'}
%
%\SubtaskSolved \code{Char, '*'}


\QUESTEND




\def\what{\emph{Satser och uttryck.}}

\QUESTBEGIN

\Task \what

\Subtask Vad är det för skillnad på en sats och ett uttryck?

\Subtask Ge exempel på satser som inte är uttryck?

\Subtask Förklara vad som händer för varje evaluerad rad:
\begin{REPL}
scala> def värdeSaknas = ()
scala> värdeSaknas
scala> värdeSaknas.toString
scala> println(värdeSaknas)
scala> println(println("hej"))
\end{REPL}

\Subtask Vilken typ har literalen \code{()}?

\Subtask Vilken returtyp har \code{println}?

\SOLUTION

\TaskSolved \what

\SubtaskSolved  Ett utryck kan evalueras och resulterar då i ett användbart värde. En sats \emph{gör} något (t.ex. skriver ut något), men resulterat inte i något användbart värde.

\SubtaskSolved \code{println()}

\SubtaskSolved 

 Värdesaknas innehåller Unit

 Skriver ut \code{Unit}

 Skriver ut \code{"()"}

 Skriver ut \code{"()"}

 Skriver först ut hej med det innersta anropet och sen \code{()} med det yttre anropet

\SubtaskSolved  \code{Unit}

\SubtaskSolved  \code{Unit}

\QUESTEND



\def\what{\emph{Procedur med parameter.} \TODO}

\QUESTBEGIN

\Task \what~En procedur är en funktion som orsakar en effekt, till exempel en utskrift eller en variabeltilldelning, men som inte returnerar något intressant resultatvärde. \footnote{I Scala är procedurer funktioner som returnerar det \emph{tomma värdet}, vilket skrivs \code{()} och är av typen \code{Unit}. I Java och flera andra språk finns inget tomt värde och man har en specialsyntax för procedurer som använder nyckelordet \code{void}. }

\Subtask Deklarera en förändringsbar variabel \code{highscore} som initieras till 0.

\Subtask Deklarera en procedur \code{updateHighscore} som tar en parameter \code{points} och tilldelar \code{highscore} \TODO ...


\SOLUTION

\TaskSolved \what

\SubtaskSolved 

\QUESTEND





\def\what{\emph{\code{if}\textit{-sats}.}}

\QUESTBEGIN

\Task \what~För varje rad nedan, beskriv vad som skrivs ut.  % Uppgift 18
\begin{REPL}
scala> if (!true) println("sant") else println("falskt")
scala> if (!false) println("sant") else println("falskt")
scala> def singlaSlant = if (math.random > 0.5) "krona" else "klave"
scala> for (i <- 1 to 5) print(s"$i:$singlaSlant ")
\end{REPL}

\SOLUTION

\TaskSolved \what

\begin{enumerate}
\item Utskrift: \code{falskt}
\item Utskrift: \code{sant}
\item Inget skrivs ut, funktionen deklareras men körs ej.
\item Utskrift: code{1:krona 2:klave 3:krona 4:krona 5:klave }
\end{enumerate}

\QUESTEND





\def\what{\emph{\code{if}\textit{-uttryck}.}}

\QUESTBEGIN

\Task  Deklarera följande variabler med nedan initialvärden:  

\begin{REPLnonum}
scala> var grönsak = "gurka"
scala> var frukt = "banan"
\end{REPLnonum}

Ange för varje rad nedan vad uttrycket har för värde och typ:
\begin{REPLnonum}
scala> if (grönsak == "tomat") "gott" else "inte gott" 
scala> if (frukt == "banan") "gott" else "inte gott" 
scala> if (true) grönsak else 42 
scala> if (false) grönsak else 42 
\end{REPLnonum}

\SOLUTION


\TaskSolved \what~Notera typen \code{Any} på de sista två uttrycken.

\begin{REPLnonum}
scala> if (grönsak == "tomat") "gott" else "inte gott"
res0: String = inte gott

scala> if (frukt == "banan") "gott" else "inte gott"
res1: String = gott

scala> if (true) grönsak else 42
res2: Any = gurka

scala> if (false) grönsak else 42
res3: Any = 42
\end{REPLnonum}


\QUESTEND






\def\what{\emph{QUESTTEMPLATE}}

\QUESTBEGIN

\Task \what

\Subtask

\SOLUTION

\TaskSolved \what

\SubtaskSolved 

\QUESTEND




\clearpage

\AdvancedTasks   %%%%%%%%%%%%%%%%%%% FÖRDJUPNINGSUPPGIFTER




\def\what{\emph{Stränginterpolatorn \code{s}.}}

\QUESTBEGIN

\Task \what~Med ett \code{s} framför en strängliteral får man hjälp av kompilatorn att, på ett typsäkert sätt, infoga variabelvärden i en sträng. 
Variablernas namn ska föregås med ett dollartecken, t.ex. \code{s"Hej $namn"}.  
Om man vill evaluera ett uttryck placeras detta inom klammer direkt efter dollartecknet, t.ex.
\code/s"Dubbla längden: ${namn.size * 2}"/  

\Subtask Vad skrivs ut nedan?
\begin{REPL}
scala> val f = "Kim"
scala> val e = "Finkodare"
scala> println(s"Namnet '$f $e' har ${f.size + e.size} bokstäver.")
\end{REPL}

\Subtask Skapa följande utskrifter med hjälp av stränginterpolatorn \code{s} och variablerna \code{f} och \code{e} i föregående deluppgift.
\begin{REPL}
Kim har 3 bokstäver.
Finkodare har 9 bokstäver.
\end{REPL}

\SOLUTION

\TaskSolved \what

\SubtaskSolved 
\begin{REPLnonum}
Namnet 'Kim Finkodare' har 12 bokstäver.
\end{REPLnonum}

\SubtaskSolved 
\begin{REPLnonum}
println(s"$f har  ${f.size} bokstäver.")
println(s"$e har  ${e.size} bokstäver.")
\end{REPLnonum}

\QUESTEND






\def\what{\emph{Flyttalsaritmetik}}

\QUESTBEGIN

\Task \what

\Subtask Vilket är det minsta positiva värdet av typen \code{Double}?

\Subtask Vad är värdet av detta uttryck? Varför blir det så?
\begin{REPL}
scala> Double.MaxValue + Double.MinPositiveValue == Double.MaxValue
\end{REPL}

\SOLUTION

\TaskSolved \what

\SubtaskSolved 

\begin{REPL}
scala> Double.MinPositiveValue
res0: Double = 4.9E-324
\end{REPL}

\SubtaskSolved 

\begin{REPL}
scala> Double.MaxValue + Double.MinPositiveValue == Double.MaxValue
res2: Boolean = true
\end{REPL}

\QUESTEND




\def\what{\emph{Stora tal.}}

\QUESTBEGIN

\Task \what~Om vi vill beräkna $2^{64} -1$ som ett exakt heltal\footnote{\url{https://en.wikipedia.org/wiki/Wheat_and_chessboard_problem}} blir det större än \code{Int.MaxValue}, så vi kan tyvärr inte använda snabba \code{Int}. Till vår räddning: \code{BigInt} 

\Subtask Läs om \code{BigInt} och \code{BigDecimal} på \Scaladoc \\ Notera vad de kan användas till. 

\Subtask Du skapar ett \code{BigInt}-heltal med \code{BigInt(2)} och kan anropa funktionen \code{pow} på en \code{BigInt} med punktnotation. Beräkna $2^{64} -1$ som ett exakt heltal.

\Subtask Vilka nackdelar finns med \code{BigInt} och \code{BigDecimal}?

\SOLUTION

\TaskSolved \what

\SubtaskSolved \code{BigInt} kan användas i stället för \code{Int} vid mycket stora heltal. \code{BigDecimal} kan användas i stället för \code{Double} vid mycket stora decimaltal.

\SubtaskSolved 
\begin{REPL}
scala> BigInt(2).pow(64)
res0: scala.math.BigInt = 18446744073709551616
\end{REPL}

\SubtaskSolved Beräkningar går mycket långsammare och de är lite krångligare att använda.

\QUESTEND





\def\what{\emph{Precedensregler}}

\QUESTBEGIN

\Task \what~Evalueringsordningen kan styras med parenteser. Vilket värde och vilken typ har följande uttryck? 

\Subtask \code{23 + 2 * 2 + (23 + 2) * 2}

\Subtask \code{(-(2 - 42)) / (1 + 1 + 1)}

\Subtask \code{(-(2 - 42)) / (-1)/(1 + 1 + 1)}

\SOLUTION

\TaskSolved \what

\SubtaskSolved \code{77:  Int}

\SubtaskSolved \code{13: Int}

\SubtaskSolved \code{-13: Int}

\QUESTEND






\def\what{\emph{QUESTTEMPLATE}}

\QUESTBEGIN

\Task \what

\Subtask

\SOLUTION

\TaskSolved \what

\SubtaskSolved 

\QUESTEND




\subsection{TODO}

\TODO{SAKERNA NEDAN SKA FLYTTAS/UPPDATERAS/TAS BORT???} 
%%%%%%%%%%%%%%%%%%%%%%%%%%%%%%%%%%%%%%%%%%%%%%%%
%%%%%%%%%%%%%%%%%%%%%%%%%%%%%%%%%%%%%%%%%%%%%%%%
%%%%%%%%%%%%%%%%%%%%%%%%%%%%%%%%%%%%%%%%%%%%%%%%
%%%%%%%%%%%%%%%%%%%%%%%%%%%%%%%%%%%%%%%%%%%%%%%%
%%%%%%%%%%%%%%%%%%%%%%%%%%%%%%%%%%%%%%%%%%%%%%%%
%%%%%%%%%%%%%%%%%%%%%%%%%%%%%%%%%%%%%%%%%%%%%%%%
%%%%%%%%%%%%%%%%%%%%%%%%%%%%%%%%%%%%%%%%%%%%%%%%
%%%%%%%%%%%%%%%%%%%%%%%%%%%%%%%%%%%%%%%%%%%%%%%%
%%%%%%%%%%%%%%%%%%%%%%%%%%%%%%%%%%%%%%%%%%%%%%%%


\ifPreSolution  %%% TODO remove \fi at end of file and break sultions into pieces





\Task Klassen \code{java.lang.Math} och paketobjektet \code{scala.math}. % Uppgift 11
Genom att trycka på tab tagenten kan man se vad som finns i olika paket.

\begin{REPL}
scala> java.    //tryck TAB efter punkten
applet   awt   beans   io   lang   math   net   nio   rmi   security   sql

scala>
\end{REPL}

\Subtask Undersök genom att trycka på Tab-tangenten, vilka funktioner som finns i \code{Math} och \code{math}. Vad heter konstanten $\pi$ i java.lang.Math respektive scala.math?

\begin{REPL}
scala> java.lang.Math.    //tryck TAB efter punkten
scala> scala.math.        //tryck TAB efter punkten
\end{REPL}

\Subtask Undersök dokumentationen för klassen \code{java.lang.Math} här: \\ \url{https://docs.oracle.com/javase/8/docs/api/java/lang/Math.html} \\
Vad gör \code{java.lang.Math.hypot}?

\Subtask Undersök dokumentationen för paketobjektet \code{scala.math} här: \\
\url{http://www.scala-lang.org/api/current/#scala.math.package} \\
Ge exempel på någon funktion i \code{java.lang.Math} som inte finns i \code{scala.math}.

%\TaskSection{Noggrannhet och undantag i aritmetiska uttryck}

\Task Vad händer här? Notera undantag \Eng{exceptions} och noggrannhetsproblem. % Uppgift 12

\Subtask \code{Int.MaxValue} + 1

\Subtask \code{1 / 0}

\Subtask \code{1E8 + 1E-8}

\Subtask \code{1E9 + 1E-9}

\Subtask \code{math.pow(math.hypot(3,6), 2)}

\Subtask \code{1.0 / 0}

\Subtask \code{(1.0 / 0).toInt}

\Subtask \code{math.sqrt(-1)}

\Subtask \code{math.sqrt(Double.NaN)}

\Subtask \code{throw new Exception("PANG!!!")}





\Task \textit{Deklarationer: \code{var}, \code{val}, \code{def}}. Evaluera varje rad nedan i tur och ordning i Scala REPL.  % Uppgift 15
\begin{REPL}[numbers=left, numberstyle=\color{black}\ttfamily\scriptsize\selectfont]
scala> var x = 30
scala> x + 1
scala> x
scala> x = x + 1
scala> x
scala> x == x + 1
scala> val y = 20
scala> y = y + 1
scala> var z = {println("gurka"); 10}
scala> def w = {println("gurka"); 10}
scala> z
scala> z
scala> z = z + 1
scala> w
scala> w
scala> w = w + 1
\end{REPL}

\Subtask För varje rad ovan: förklara för vad som händer.

\Subtask Vilka rader ger kompileringsfel och i så fall vilket och varför?

\Subtask\Pen Vad är det för skillnad på \code{var}, \code{val} och \code{def}?

\Subtask\Pen Tilldela variabeln \code{val even } värdet av ett uttryck som med modulo-operatorn \code
och olikhetsoperatorn \code{!=} testar om ett tal \code{n} är udda.


\Task\Pen \emph{Tilldelningsoperatorer.} Man kan förkorta en tilldelningssats som förändrar en variabel, t.ex. \code{x = x + 1}, genom att använda så kallade tilldelningsoperatorer och skriva \code{x += 1} som betyder samma sak. Rita en ny bild av datorns minne efter varje evaluerad rad nedan. Bilderna ska visa variablers namn, typ och värde.  % Uppgift 16

\begin{REPL}
scala> var a = 40
scala> var b = a + 40
scala> a += 10
scala> b -= 10
scala> a *= 2
scala> b /= 2
\end{REPL}



\Task \emph{Stränginterpolatorn \code{s}.} Man behöver ofta skapa strängar som innehåller variabelvärden. Med ett \code{s} framför en strängliteral får man hjälp av kompilatorn att, på ett typsäkert sätt, infoga variabelvärden i en sträng. Variablernas namn ska föregås med ett dollartecken, t.ex. \code{s"Hej $namn"}.  Om man vill evaluera ett uttryck placeras detta inom klammer direkt efter dollartecknet, t.ex.
\code/s"Dubbla längden: ${namn.size * 2}"/  % Uppgift 17

\begin{REPL}
scala> val f = "Kim"
scala> val e = "Finkodare"
scala> val tot = f.size + e.size
scala> println(s"Namnet '$f $e' har $tot bokstäver.")
scala> println(s"Efternamnet '$e' har ${e.size} bokstäver.")
\end{REPL}

\Subtask Vad skrivs ut ovan?

\Subtask Skapa följande utskrifter med hjälp av stränginterpolatorn \code{s} och lämpliga variabler.
\begin{REPL}
Namnet 'Kim' har 3 bokstäver.
Namnet 'Finkodare' har 9 bokstäver.
\end{REPL}



\Task \code{if}\textit{-sats}.För varje rad nedan; förklara vad som händer.  % Uppgift 18
\begin{REPL}
scala> if (true) println("sant") else println("falskt")
scala> if (false) println("sant") else println("falskt")
scala> if (!true) println("sant") else println("falskt")
scala> if (!false) println("sant") else println("falskt")
scala> def singlaSlant =
scala> 	 if (math.random > 0.5) print(" krona") else print(" klave")
scala> singlaSlant; singlaSlant; singlaSlant
\end{REPL}


\Task \code{if}\textit{-uttryck}. Deklarera följande variabler med nedan initialvärden:  % Uppgift 19

\begin{REPLnonum}
scala> var grönsak = "gurka"
scala> var frukt = "banan"
\end{REPLnonum}

Vad har följande uttryck för värden och typ?

\Subtask \code{if (grönsak == "tomat") "gott" else "inte gott" }

\Subtask \code{if (frukt == "banan") "gott" else "inte gott" }

\Subtask \code{if (frukt.size == grönsak.size) "lika stora" else "olika stora" }

\Subtask \code{if (true) grönsak else frukt }

\Subtask \code{if (false) grönsak else frukt }


\Task \code{for}\textit{-sats}.  Med bakåtpilen \texttt{<-} kan man i en \code{for}-sats ange vilka värden som ska gås igenom i sekvens. Vid varje runda i loopen får en lokal variabel ett nytt värde i sekvensen. % Uppgift 20

\Subtask Vad ger nedan \code{for}-satser för utskrift?

\begin{REPL}
scala> for (i <- 1 to 10) print(i + ", ")
scala> for (i <- 1 until 10) print(i + ", ")
scala> for (i <- 1 to 5) print((i * 2) + ", ")
scala> for (i <- 1 to 92 by 10) print(i + ", ")
scala> for (i <- 10 to 1 by -1) print(i + ", ")
\end{REPL}

\Subtask Skriv en \code{for}-sats som ger följande utskrift:
\begin{REPLnonum}
A1, A4, A7, A10, A13, A16, A19, A22, A25, A28, A31, A34, A37, A40, A43,
\end{REPLnonum}

\Task Repetition med metoden \code{foreach}. Efter framåtpilen \texttt{=>} (se nedan) anges vad som ska hända för varje element som gås igenom sekventiellt. Vid varje runda i loopen får en lokal variabel ett nytt värde i sekvensen.   % Uppgift 21

\Subtask Vad ger nedan satser för utskrifter?

\begin{REPL}
scala> (9 to 19).foreach{i => print(i + ", ")}
scala> (1 until 20).foreach{i => print(i + ", ")}
scala> (0 to 33 by 3).foreach{i => print(i + ", ")}
\end{REPL}

\Subtask Använd \code{foreach} och skriv ut följande:
\begin{REPLnonum}
B33, B30, B27, B24, B21, B18, B15, B12, B9, B6, B3, B0,
\end{REPLnonum}

\Task \code{while}\textit{-sats}. En sats eller ett block med satser upprepas så länge ett villkor är sant.  % Uppgift 22

\Subtask Vad ger nedan satser för utskrifter?
\begin{REPL}
scala> var i = 0
scala> while (i < 10) { println(i); i = i + 1 }
scala> var j = 0; while (j <= 10) { println(j); j = j + 2 }; println(j)
\end{REPL}

\Subtask Skriv en \code{while}-sats som ger följande utskrift. Använd en variabel \code{k} som initialiseras till 1.
\begin{REPLnonum}
A1, A4, A7, A10, A13, A16, A19, A22, A25, A28, A31, A34, A37, A40, A43,
\end{REPLnonum}

\Subtask\Pen Vilken av \code{for}, \code{while} och \code{foreach} är kortast att skriva om man vill repetera mer än en sats 100 gånger? Vilken tycker du är lättast att läsa?

\Task \textit{Slumptal}. Undersök vad dokumentationen säger om funktionen \code{scala.math.random}:\\  % Uppgift 23
\url{http://www.scala-lang.org/api/current/#scala.math.package}

\Subtask\Pen Vilken typ har värdet som returneras av funktionen \code{random}?

\Subtask\Pen Vilket är det minsta respektive största värde som kan returneras?

\Subtask\Pen Är \code{random} en \textit{äkta} funktion \Eng{pure function} i matematisk mening?

\Subtask Anropa funktionen \code{math.random} upprepade gånger och notera vad som händer. Använd pil-upp-tangenten.
\begin{REPLnonum}
scala> math.random
\end{REPLnonum}


\Subtask Vad händer? Använd \textit{pil-upp} och kör nedan \code{for}-sats flera gånger. Förklara vad som sker.

\begin{REPLnonum}
scala> for (i <- 1 to 20) println((math.random * 3 + 1).toInt)
\end{REPLnonum}

\Subtask Skriv en \code{for}-sats som skriver ut 100 slumpmässiga heltal från 0 till och med 9 på var sin rad.

\begin{REPLnonum}
scala> for (i <- 1 to 100) println(???)
\end{REPLnonum}

\Subtask Skriv en \code{for}-sats som skriver ut 100 slumpmässiga heltal från 1 till och med 6 på samma rad.

\begin{REPLnonum}
scala> for (i <- 1 to 100) print(???)
\end{REPLnonum}


\Subtask Använd \textit{pil-upp} och kör nedan \code{while}-sats flera gånger. Förklara vad som sker.

\begin{REPLnonum}
scala> while (math.random > 0.2) println("gurka")
\end{REPLnonum}

\Subtask Ändra i \code{while}-satsen ovan så att sannolikheten ökar att riktigt många strängar ska skrivas ut.

\Subtask Förklara vad som händer nedan.
\begin{REPL}
scala> var slumptal = math.random
scala> while (slumptal > 0.2) { println(slumptal); slumptal = math.random }
\end{REPL}

\Task\Pen \textit{Logik och De Morgans Lagar}. Förenkla följande uttryck. Antag att \code{poäng} och \code{highscore} är heltalsvariabler medan \code{klar} är av typen \code{Boolean}.
  % Uppgift 24

\Subtask \code{poäng > 100 && poäng > 1000}

\Subtask \code{poäng > 100 || poäng > 1000}

\Subtask \code{!(poäng > highscore)}

\Subtask \code{!(poäng > 0 && poäng < highscore) }

\Subtask \code{!(poäng < 0 || poäng > highscore) }

\Subtask \code{klar == true}

\Subtask \code{klar == false}


\clearpage

\ExtraTasks

\Task \textit{Slumptal}.

\Subtask Ersätt \code{???} nedan med literaler så att \code{tärning} returnerar ett slumpmässigt heltal mellan 1 och 6.
\begin{REPLnonum}
scala> def tärning = (math.random * ??? + ???).toInt
\end{REPLnonum}

\Subtask Ersätt \code{???} med literaler så att \code{rnd} blir ett decimaltal med max en decimal mellan 0.0 och 1.0.
\begin{REPLnonum}
scala> def rnd = math.round(math.random * ???) / ???
\end{REPLnonum}

\Subtask Vad blir det för skillnad om \code{math.round} ersätts med \code{math.floor} ovan? (Se dokumentationen av \code{java.lang.Math.round} och \code{java.lang.Math.floor}.)

\Task Undersök vad som finns i paketet \code{scala.math} genom att studera dess dokumentation: \href{http://www.scala-lang.org/api/current/#scala.math.package}{www.scala-lang.org/api/current/\#scala.math.package} och gör några matematiska beräkningar i REPL som använder olika funktioner i \code{math}-paketet.

\Task\Pen Antag att du byter plats mellan satsen efter villkoret och satsen efter \code{else} i \code{if}-satsen nedan. Hur kan du ändra i villkoret så att det ändå skrivs ut samma sak som före bytet?
\begin{Code}
if (x == 42) println("the meaning of it all") else println(":(")
\end{Code}

\Task\Pen Rita en ny bild av datorns minne efter varje evaluerad rad nedan. Bilderna ska visa variablers namn, typ och värde.
\begin{REPL}
scala> var x = 42
scala> var y = x + 1
scala> x += -1
scala> y -= 1
\end{REPL}

\Task Skapa med hjälp av \code{while} några olika oändliga loopar som skriver ut olika saker vid varje loop-runda.

\Task Hitta på några egna övningar för att träna mer på De Morgans lagar.



\clearpage

\AdvancedTasks

\Task Läs om moduloräkning här \href{https://en.wikipedia.org/wiki/Modulo\_operation}{en.wikipedia.org/wiki/Modulo\_operation} och undersök hur det blir med olika tecken (positivt resp. negativt) på divisor och dividend.



\Task Läs om identifierare i Scala och speciellt \emph{literal identifiers} här: \url{http://www.artima.com/pins1ed/functional-objects.html#6.10}.

\Subtask Förklara vad som händer nedan:
\begin{REPLnonum}
scala> val `konstig val` = 42
scala> println(`konstig val`)
\end{REPLnonum}

\Subtask Scala och Java har olika uppsättningar med reserverade ord. På vilket sätt kan ''backticks'' vara använbart med anledning av detta?


\Task Sök upp dokumentationen för \code{java.lang.Integer}.

\Subtask Undersök i REPL hur metoderna \code{toBinaryString} och \code{toHexString} fungerar.

\Subtask Vad betyder literalen \code{0x2a}?

\Task Typannoteringar skapas genom att i ett uttryck placera ett kolon följt av en typ, vid behov  omslutet av en parentes. Skapa ett större uttryck med typannoteringar och försök få kompilatorn att kontrollera typen på intressanta ställen. Märk att typannoteringar också ibland kan användas för att konvertera mellan numeriska typer.


\Task Förklara vad som händer nedan:
\begin{REPL}
scala> var i = 42
scala> i += 1
scala> i *= 2
scala> i /= 3
\end{REPL}


\Task Läs om BigInt och BigDecimal här: \href{http://alvinalexander.com/scala/how-to-use-large-integer-decimal-numbers-in-scala-bigint-bigdecimal}{alvinalexander.com/scala/how-to-use-large-integer-decimal-numbers-in-scala-bigint-bigdecimal} och prova att skapa riktigt stora tal med hjälp av metoden \code{pow} på BigInt och tal med riktigt många decimaler med BigDecimal dess metod \code{pow}.

\Task Sök upp dokumentationtionen för \code{java.lang.Math.multiplyExact} och läs om vad den metoden gör.

\Subtask Vad händer här?
\begin{REPLnonum}
scala> Math.multiplyExact(2, 42)
scala> Math.multiplyExact(Int.MaxValue, Int.MaxValue)
\end{REPLnonum}

\Subtask\Pen Varför kan man vilja använda \code{java.lang.Math.multiplyExact} i stället för ''vanlig'' multiplikation?



\Subtask\Pen Sök med Ctrl+F i webbläsaren och efter förekomster av texten \textit{''overflow''} i javadoc för klassen \code{java.lang.Math} i JDK 8. Vad är ''overflow''? Vilka metoder finns i \code{java.lang.Math} som hjälper dig att upptäcka om det blir overflow?

\Task Använda Scala REPL för att undersöka konstanterna nedan. Vilket av dessa värden är negativt? Vad kan man ha för praktisk nytta av dessa värden i ett program som gör flyttalsberäkningar?

\Subtask \code{java.lang.Double.MIN_VALUE}

\Subtask \code{scala.Double.MinValue}

\Subtask \code{scala.Double.MinPositiveValue}

\Task För typerna \code{Byte}, \code{Short}, \code{Char}, \code{Int}, \code{Long}, \code{Float}, \code{Double}: Undersök hur många bitar som behövs för att representera varje typs omfång? \\*
\textit{Tips:} Några användbara uttryck: \\*
 \code{Integer.toBinaryString(Int.MaxValue + 1).size} \\*
 \code{Integer.toBinaryString((math.pow(2,16) - 1).toInt).size} \\*
 \code{1 + math.log(Long.MaxValue)/math.log(2)}
Se även språkspecifikationen för Scala, kapitlet om heltalsliteraler: \\
\url{http://www.scala-lang.org/files/archive/spec/2.11/01-lexical-syntax.html#integer-literals}

\Subtask Undersök källkoden för paketobjektet \code{scala.math} här: \\
\url{https://github.com/scala/scala/blob/v2.11.7/src/library/scala/math/package.scala} \\
Hur många olika överlagrade varianter av funktionen \code{abs} finns det och för vilka parametertyper är den definierad?

\Task Läs mer om stränginterpolatorer här:\\ \href{http://docs.scala-lang.org/overviews/core/string-interpolation.html}{docs.scala-lang.org/overviews/core/string-interpolation.html} \\ Hur kan du använda \code{f}-interpolatorn för att göra följande utskrift i REPL? Byt ut \code{???} mot lämpliga tecken.
\begin{REPLnonum}
scala> val g: Double = 1 / 3.0
scala> val s: String = f"Gurkan är ??? meter lång"
scala> println(s)
Gurkan är 0.333 meter lång
\end{REPLnonum}

\fi %%% TODO fix solutions





%!TEX encoding = UTF-8 Unicode
%!TEX root = ../exercises.tex

\ifPreSolution

\Exercise{\ExeWeekTWO}\label{exe:W02}
\begin{Goals}
%!TEX encoding = UTF-8 Unicode
%!TEX root = ../exercises.tex

\item Kunna skapa, kompilera och köra en enkel applikation i terminalen.
\item Kunna skapa samlingarna Range, Array och Vector med heltal och strängar.
\item Kunna indexera i en indexerbar samling, t.ex. Array och Vector.
\item Kunna anropa operationerna size, mkString, sum, min, max på samlingar som innehåller heltal.
\item Känna till skillnader och likheter mellan samlingarna Range, Array och Vector.
\item Förstå skillnaden mellan en while-sats och ett for-uttryck.
\item Kunna skapa samlingar med heltalsvärden som resultat av enkla for-uttryck.
\item Förstå skillnaden mellan en algoritm i pseudo-kod och dess implementation.
\item Kunna implementera algoritmerna SUM, MIN, MAX med en indexerbar samling och en while-sats.

\end{Goals}

\begin{Preparations}
\item \StudyTheory{02}
\item Bekanta dig med grundläggande terminalkommandon, se appendix~\ref{appendix:terminal}.
\item Bekanta dig med den editor du vill använda, se appendix~\ref{appendix:compile}.
\end{Preparations}

\else

\ExerciseSolution{\ExeWeekTWO}

\fi


% terminalkommando
% scalac -> hello world; scala som script; javac
% paket, import, jar, main,


\BasicTasksNoLab %%%%%%%%%%%%%%%%




\WHAT{Para ihop begrepp med beskrivning.}

\QUESTBEGIN

\Task \what

\vspace{1em}\noindent Koppla varje begrepp med den (förenklade) beskrivning som passar bäst: 

\begin{ConceptConnections}
  kompilerad & 1 & & A & där exekveringen av kompilerad app startar \\ 
  skript & 2 & & B & en samling som representerar ett intervall av heltal \\ 
  objekt & 3 & & C & maskinkod sparad och kan köras igen utan kompilering \\ 
  main & 4 & & D & en oföränderlig, indexerbar sekvenssamling \\ 
  programargument & 5 & & E & applicerar en funktion på varje element i en samling \\ 
  datastruktur & 6 & & F & stegvis beskrivning av en lösning på ett problem \\ 
  samling & 7 & & G & maskinkod sparas ej utan skapas vid varje körning \\ 
  sekvenssamling & 8 & & H & samlar variabler och funktioner \\ 
  Array & 9 & & I & överförs via parametern args i main \\ 
  Vector & 10 & & J & en specifik realisering av en algoritm \\ 
  Range & 11 & & K & används i for-uttryck för att skapa ny samling \\ 
  yield & 12 & & L & en förändringsbar, indexerbar sekvenssamling \\ 
  map & 13 & & M & datastruktur med element av samma typ \\ 
  algoritm & 14 & & N & många olika element i en helhet; elementvis åtkomst \\ 
  implementation & 15 & & O & datastruktur med element i en viss ordning \\ 
\end{ConceptConnections}

\SOLUTION

\TaskSolved \what

\begin{ConceptConnections}
  kompilerad & 1 & ~~\Large$\leadsto$~~ &  C & maskinkod sparad och kan köras igen utan kompilering \\ 
  skript & 2 & ~~\Large$\leadsto$~~ &  G & maskinkod sparas ej utan skapas vid varje körning \\ 
  objekt & 3 & ~~\Large$\leadsto$~~ &  H & samlar variabler och funktioner \\ 
  main & 4 & ~~\Large$\leadsto$~~ &  A & där exekveringen av kompilerad app startar \\ 
  programargument & 5 & ~~\Large$\leadsto$~~ &  I & överförs via parametern args i main \\ 
  datastruktur & 6 & ~~\Large$\leadsto$~~ &  N & många olika element i en helhet; elementvis åtkomst \\ 
  samling & 7 & ~~\Large$\leadsto$~~ &  M & datastruktur med element av samma typ \\ 
  sekvenssamling & 8 & ~~\Large$\leadsto$~~ &  O & datastruktur med element i en viss ordning \\ 
  Array & 9 & ~~\Large$\leadsto$~~ &  L & en förändringsbar, indexerbar sekvenssamling \\ 
  Vector & 10 & ~~\Large$\leadsto$~~ &  D & en oföränderlig, indexerbar sekvenssamling \\ 
  Range & 11 & ~~\Large$\leadsto$~~ &  B & en samling som representerar ett intervall av heltal \\ 
  yield & 12 & ~~\Large$\leadsto$~~ &  K & används i for-uttryck för att skapa ny samling \\ 
  map & 13 & ~~\Large$\leadsto$~~ &  E & applicerar en funktion på varje element i en samling \\ 
  algoritm & 14 & ~~\Large$\leadsto$~~ &  F & stegvis beskrivning av en lösning på ett problem \\ 
  implementation & 15 & ~~\Large$\leadsto$~~ &  J & en specifik realisering av en algoritm \\ 
\end{ConceptConnections}

\QUESTEND






%%%%%%%%%%%%%%%%%%% SKA FIXAS:




\WHAT{Datastrukturen \code+Range+.}

\QUESTBEGIN

\Task  \what~Evaluera nedan uttryck i Scala REPL. Vad har respektive uttryck för värde och typ?

\Subtask \code{Range(1, 10)}

\Subtask \code{Range(1, 10).inclusive}

\Subtask \code{Range(0, 50, 5)}

\Subtask \code{Range(0, 50, 5).size}

\Subtask \code{Range(0, 50, 5).inclusive}

\Subtask \code{Range(0, 50, 5).inclusive.size}

\Subtask \code{0.until(10)}

\Subtask \code{0 until (10)}

\Subtask \code{0 until 10}

\Subtask \code{0.to(10)}

\Subtask \code{0 to 10}

\Subtask \code{0.until(50).by(5)}

\Subtask \code{0 to 50 by 5}

\Subtask \code{(0 to 50 by 5).size}

\Subtask \code{(1 to 1000).sum}


\SOLUTION


\TaskSolved \what
 

\SubtaskSolved  värde: \code{Range(1,2,3,4,5,6,7,8,9)}

typ: \code{scala.collection.immutable.Range}

\SubtaskSolved  värde: \code{Range(1,2,3,4,5,6,7,8,9,10)}

typ: \code{scala.collection.immutable.Range}

\SubtaskSolved  värde: \code{Range(0,5,10,15,20,25,30,35,40,45)}

 typ: \code{scala.collection.immutable.Range}

\SubtaskSolved  värde: \code{10}, typ: \code{Int}

\SubtaskSolved  värde: \code{Range(0,5,10,15,20,25,30,35,40,45,50)}

typ: \code{scala.collection.immutable.Range}

\SubtaskSolved  värde: \code{11}, typ: \code{Int}

\SubtaskSolved  värde: \code{Range(0,1,2,3,4,5,6,7,8,9)}

typ: \code{scala.collection.immutable.Range}

\SubtaskSolved  värde: \code{Range(0,1,2,3,4,5,6,7,8,9)}

typ: \code{scala.collection.immutable.Range}

\SubtaskSolved  värde: \code{Range(0,1,2,3,4,5,6,7,8,9)}

typ: \code{scala.collection.immutable.Range}

\SubtaskSolved  värde: \code{Range(0,1,2,3,4,5,6,7,8,9,10)}

typ: \code{scala.collection.immutable.Range.Inclusive}

\SubtaskSolved  värde: \code{Range(0,1,2,3,4,5,6,7,8,9,10)}

typ: \code{scala.collection.immutable.Range.Inclusive}

\SubtaskSolved  värde: \code{Range(0,5,10,15,20,25,30,35,40,45)}

typ: \code{scala.collection.immutable.Range}

\SubtaskSolved  värde: \code{Range(0,5,10,15,20,25,30,35,40,45,50)}

typ: \code{scala.collection.immutable.Range}

\SubtaskSolved  värde: \code{11}, typ: \code{Int}

\SubtaskSolved  värde: \code{500500}, typ: \code{Int}




\QUESTEND




%%<AUTOEXTRACTED by mergesolu>%%      %Uppgift 2




\WHAT{Datastrukturen \code+Array+.}

\QUESTBEGIN

\Task \label{task:array} \what~   Kör nedan kodrader i Scala REPL. Beskriv vad som händer.

\Subtask \code{val xs = Array("hej","på","dej", "!")}

\Subtask \code{xs(0)}

\Subtask \code{xs(3)}

\Subtask \code{xs(4)}

\Subtask \code{xs(1) + " " + xs(2)}

\Subtask \code{xs.mkString}

\Subtask \code{xs.mkString(" ")}

\Subtask \code{xs.mkString("(", ",", ")")}

\Subtask \code{xs.mkString("Array(", ", ", ")")}

\Subtask \code{xs(0) = 42}

\Subtask \code{xs(0) = "42"; println(xs(0))}

\Subtask \code{val ys = Array(42, 7, 3, 8)}

\Subtask \code{ys.sum}

\Subtask \code{ys.min}

\Subtask \code{ys.max}

\Subtask \code{val zs = Array.fill(10)(42)}

\Subtask \code{zs.sum}

\Subtask\Pen Datastrukturen \code{Range} håller reda på start- och slutvärde, samt stegstorleken för en uppräkning, men alla talen i uppräkningen genereras inte förrän så behövs. En \code{Int} tar 4 bytes i minnet. Ungefär hur mycket plats i minnet tar de objekt som variablerna \code{r} respektive \code{a} refererar till nedan?
\begin{REPL}
scala> val r = (1 to Int.MaxValue by 2)
scala> val a = r.toArray
\end{REPL}
\emph{Tips:} Använd uttrycket \code{ BigInt(Int.MaxValue) * 2 } i dina beräkningar.

\SOLUTION


\TaskSolved \what
 

\SubtaskSolved  Ett objekt av typen \code{Array[String]} skapas med värdet 

\code{Array(hej, på, dej, !)} och med namnet \code{xs}.

\SubtaskSolved  Returnerar en sträng med värdet \code{hej}.

\SubtaskSolved  Returnerar en sträng med värdet \code{!}.

\SubtaskSolved  Ett exception genereras. Skriver ut:

\code{java.lang.ArrayIndexOutOfBoundsException: 4}

\SubtaskSolved  Returnerar en sträng med värdet \code{på dej}.

\SubtaskSolved  Returnerar en sträng med värdet \code{hejpådej!}.

\SubtaskSolved  Returnerar en sträng med värdet \code{hej på dej !}.

\SubtaskSolved  Returnerar en sträng med värdet \code{(hej,på,dej,!)}.

\SubtaskSolved  Returnerar en sträng med värdet \code{Array(hej,på,dej,!)}.

\SubtaskSolved  Ett fel uppstår av typen \code{type mismatch}. Konsollen talar om för oss vad den fick, dvs värdet \code{42} av typen \code{Int}. Den talar även om för oss vad den ville ha, dvs något värde av typen \code{String}. Till sist skriver den ut vår kodrad och pekar ut felet.

\SubtaskSolved  Det första elementet i \code{xs} ändras till värdet \code{42}. Därefter skrivs det första värdet i \code{xs} ut.

\SubtaskSolved  Ett objekt av typen \code{Array[Int]} skapas med värdet \code{Array(42, 7, 3, 8)} och med namnet \code{ys}.

\SubtaskSolved  Returnerar summan av elementen i \code{ys}. Resultatet är \code{60}.

\SubtaskSolved  Returnerar det minsta värdet i \code{ys}. Resultatet är \code{3}.

\SubtaskSolved  Returnerar det största värdet i \code{ys}. Resultatet är \code{42}.

\SubtaskSolved  Ett nytt värde av typen \code{Array[Int]} skapas med \code{10} stycken element, alla med värdet \code{42}.

\SubtaskSolved  Returnerar summan av elementen i \code{zs}. Resultatet blir 420 (42 multiplicerat med 10).

\SubtaskSolved  \code{r} tar upp 12 bytes. \code{a} tar upp ca 4 miljarder bytes.



\QUESTEND




%%<AUTOEXTRACTED by mergesolu>%%      %Uppgift 3




\WHAT{Datastrukturen \code+Vector+.}

\QUESTBEGIN

\Task  \what~  Kör nedan kodrader i Scala REPL. Beskriv vad som händer.

\Subtask \code{val words = Vector("hej","på","dej", "!")}

\Subtask \code{words(0)}

\Subtask \code{words(3)}

\Subtask \code{words.mkString}

\Subtask \code{words.mkString(" ")}

\Subtask \code{words.mkString("(", ",", ")")}

\Subtask \code{words.mkString("Ord(", ", ", ")")}

\Subtask \code{words(0) = "42"}

\Subtask \code{val numbers = Vector(42, 7, 3, 8)}

\Subtask \code{numbers.sum}

\Subtask \code{numbers.min}

\Subtask \code{numbers.max}

\Subtask \code{val moreNumbers = Vector.fill(10000)(42)}

\Subtask \code{moreNumbers.sum}

\Subtask\Pen Jämför med uppgift \ref{task:array}. Vad kan man göra med en \code{Array} som man inte kan göra med en \code{Vector}?

\SOLUTION


\TaskSolved \what
 

\SubtaskSolved  Ett objekt av typen \code{scala.collection.immutable.Vector[String]} initieras med värdet \code{Vector(hej, på dej, !)}.

\SubtaskSolved  Returnerar det nollte elementet i \code{words}, dvs strängen \code{hej}.

\SubtaskSolved  Returnerar det tredje elementet i \code{words}, dvs strängen \code{!}.

\SubtaskSolved  Omvandlar vektorn till en Sträng.

\SubtaskSolved  Samma som ovan, fast den här gången används mellanrum för att seperera elementen.

\SubtaskSolved  Samma som ovan, fast den här gången sepereras elementen av kommatecken istället för mellanrum och dessutom börjar och slutar den resulterande strängen med parenteser.

\SubtaskSolved  Samma som ovan, fast med ordet \code{Ord} tillagt i början av den resulterande strängen.

\SubtaskSolved  Ett fel uppstår. Typen \code{Vector} är immutable. Dess element kan alltså inte bytas ut.

\SubtaskSolved  En ny \code{Vector[Int]} skapas med värdet \code{Vector(42, 7, 3, 8)}. 

\SubtaskSolved  Returnerar summan av vektorn \code{numbers}.

\SubtaskSolved  Returnerar vektorns minsta element.

\SubtaskSolved  Returnerar vektorns största element. 

\SubtaskSolved  En ny vektor skapas innehållandes tiotusen 42or.

\SubtaskSolved  Returnerar summan av vektorns element.

\SubtaskSolved  Byta ut element.



\QUESTEND




%%<AUTOEXTRACTED by mergesolu>%%      %Uppgift 4




\WHAT{\code+for+-uttryck}

\QUESTBEGIN

\Task  \what~ . Evaluera nedan uttryck i Scala REPL. Vad har respektive uttryck för värde och typ?

\Subtask \code{for (i <- Range(1,10)) yield i}

\Subtask \code{for (i <- 1 until 10) yield i}

\Subtask \code{for (i <- 1 until 10) yield i + 1}

\Subtask \code{for (i <- Range(1,10).inclusive) yield i}

\Subtask \code{for (i <- 1 to 10) yield i}

\Subtask \code{for (i <- 1 to 10) yield i + 1}

\Subtask \code{(for (i <- 1 to 10) yield i + 1).sum}

\Subtask \code{for (x <- 0.0 to 2 * math.Pi by math.Pi/4) yield math.sin(x)}


\SOLUTION


\TaskSolved \what
 

\SubtaskSolved  typ: \code{scala.collection.immutable.IndexedSeq[Int]}

värde: \code{Vector(1, 2, 3, 4, 5, 6, 7, 8, 9)}

\SubtaskSolved  typ: \code{scala.collection.immutable.IndexedSeq[Int]}

värde: \code{Vector(1, 2, 3, 4, 5, 6, 7, 8, 9)}

\SubtaskSolved  typ: \code{scala.collection.immutable.IndexedSeq[Int]}

värde: \code{Vector(2, 3, 4, 5, 6, 7, 8, 9, 10)}

\SubtaskSolved  typ: \code{scala.collection.immutable.IndexedSeq[Int]}

värde: \code{Vector(1, 2, 3, 4, 5, 6, 7, 8, 9, 10)}

\SubtaskSolved  typ: \code{scala.collection.immutable.IndexedSeq[Int]}

värde: \code{Vector(1, 2, 3, 4, 5, 6, 7, 8, 9, 10)}

\SubtaskSolved  typ: \code{scala.collection.immutable.IndexedSeq[Int]}

värde: \code{Vector(2, 3, 4, 5, 6, 7, 8, 9, 10, 11)}

\SubtaskSolved  typ: \code{Int}, värde: \code{Vector(65)}

\SubtaskSolved  typ: \code{scala.collection.immutable.IndexedSeq[Int]}

värde: \code{Vector(0.0, 0.707, 1.0, 0.707, 0.0, -0.707, -1.0, -0.707)}



\QUESTEND




%%<AUTOEXTRACTED by mergesolu>%%      %Uppgift 5




\WHAT{Metoden \code+map+ på en samling.}

\QUESTBEGIN

\Task  \what~  Evaluera nedan uttryck i Scala REPL. Vad har respektive uttryck för värde och typ?

\Subtask \code{Range(0,10).map(i => i + 1)}

\Subtask \code{(0 until 10).map(i => i + 1)}

\Subtask \code{(1 to 10).map(i => i * 2)}

\Subtask \code{(1 to 10).map(_ * 2)}

\Subtask \code{Vector.fill(10000)(42).map(_ + 43)}

\SOLUTION


\TaskSolved \what
 

\SubtaskSolved  typ: \code{scala.collection.immutable.IndexedSeq[Int]}

värde: \code{Vector(1, 2, 3, 4, 5, 6, 7, 8, 9, 10)}

\SubtaskSolved  typ: \code{scala.collection.immutable.IndexedSeq[Int]}

värde: \code{Vector(1, 2, 3, 4, 5, 6, 7, 8, 9, 10)}

\SubtaskSolved  typ: \code{scala.collection.immutable.IndexedSeq[Int]}

värde: \code{Vector(2, 4, 6, 8, 10, 12, 14, 16, 18, 20)}

\SubtaskSolved  typ: \code{scala.collection.immutable.IndexedSeq[Int]}

värde: \code{Vector(2, 4, 6, 8, 10, 12, 14, 16, 18, 20)}

\SubtaskSolved  typ: \code{scala.collection.immutable.Vector[Int]}

värde: En vector av tiotusen 85or (85 = 42 + 43).



\QUESTEND




%%<AUTOEXTRACTED by mergesolu>%%      %Uppgift 6




\WHAT{Metoden \code+foreach+ på en samling.}

\QUESTBEGIN

\Task  \what~  Kör nedan satser i Scala REPL. Vad händer?

\Subtask \code{Range(0,10).foreach(i => println(i))}

\Subtask \code{(0 until 10).foreach(i => println(i))}

\Subtask \code|(1 to 10).foreach{i => print("hej"); println(i * 2)}|

\Subtask \code{(1 to 10).foreach(println)}

\Subtask \code{Vector.fill(10000)(math.random).foreach(r => }\\
           \code{      if (r > 0.99) print("pling!"))}


\SOLUTION


\TaskSolved \what
 

\SubtaskSolved  En \code{Range} skapas och dess element skrivs ut ett och ett.

\SubtaskSolved  Samma sak händer.

\SubtaskSolved  De tio första jämna talen (noll ej inräknat) skrivs ut med ett "hej" framför.

\SubtaskSolved  Talen 1 till 10 skrivs ut.

\SubtaskSolved  Tiotusen slumptal mellan 0 och 1 genereras. Varje gång ett tal är större än 0.99 kommer det ett pling.



\QUESTEND




%%<AUTOEXTRACTED by mergesolu>%%      %Uppgift 7




\WHAT{Algoritm: SWAP.}

\QUESTBEGIN

\Task  \what~ 

\Subtask Skriv med \emph{pseudo-kod} algoritmen SWAP. Beskriv på vanlig svenska, steg för steg, hur en variabel $temp$ används för mellanlagring vid värdebytet:

\emph{Indata:} två heltalsvariabler $x$ och $y$

\emph{???}

\emph{Utdata:} variablerna $x$ och $y$ vars värden har bytt plats.

\Subtask Implementerar algoritmen SWAP. Ersätt \code{???} nedan med satser separerade av semikolon:

\begin{REPL}
scala> var (x, y) = (42, 43)
scala> ???
scala> println("x är " + x + ", y är " + y)
x är 43, y är 42
\end{REPL}



\SOLUTION


\TaskSolved \what
 

\SubtaskSolved  Pseudokoden kan se ut såhär:

Skapa heltalsvariabel temp. 
Flytta värdet från x till temp. 
Flytta värdet från y till x. 
Flytta värdet från temp till y.

\SubtaskSolved 
\begin{REPLnonum}
scala> var (x, y) = (42, 43)
x: Int = 42
y: Int = 43
scala> var temp = x; x = y; y = temp;
temp: Int = 42
x: Int = 43
y: Int = 42
scala> println("x är " + x + ", y är " + y)
x är 43, y är 42
\end{REPLnonum}



\QUESTEND




%%<AUTOEXTRACTED by mergesolu>%%      %Uppgift 8




\WHAT{Skript.}

\QUESTBEGIN

\Task  \what~  Skapa en fil med namn \texttt{hello-script.scala} med hjälp av en editor som innehåller denna enda rad:
\begin{Code}
println("hej skript")
\end{Code}
Spara filen och kör kommandot \code{scala hello-script.scala} i terminalen:
\begin{REPLnonum}
> scala hello-script.scala
\end{REPLnonum}

\Subtask Vad händer?

\Subtask Ändra i filen så att högerparentesen saknas. Spara och kör skriptfilen igen. Vad händer?

\Subtask Lägg till en sats sist i skriptet som skriver ut summan av de ett tusen stycken heltalen från och med 2 till och med 1001, så som visas nedan.
\begin{REPL}
> scala hello-script.scala
hej skript
501500
\end{REPL}

\Subtask Ändra i hello-script.scala genom att införa \code{val n = args(0).toInt} och använd \code{n} som övre gräns för summeringen av de n första heltalen.
\begin{REPL}
> scala hello-script.scala 5001
hej skript
12507501
\end{REPL}

\Subtask Vad blir det för felmeddelande om du glömmer ge programmet ett argument?


\SOLUTION


\TaskSolved \what
 

\SubtaskSolved  Skriver ut "hej skript".

\SubtaskSolved  Ett felmeddelande skrivs ut.

\SubtaskSolved  Lägg till raden:
\code{println((2 to 1001).sum)} 
eller motsvarande.

\SubtaskSolved  Filen ska se ut ungefär såhär: \\
\begin{Code} 
val n = args(0).toInt 
println("hej skript") 
println((1 to n).sum)
\end{Code}

\SubtaskSolved  \code{java.lang.ArrayIndexOutOfBoundsException: 0}



\QUESTEND




%%<AUTOEXTRACTED by mergesolu>%%      %Uppgift 9




\WHAT{Applikation med \code+main+-metod.}

\QUESTBEGIN

\Task  \what~  Skapa med hjälp av en editor en fil med namn \texttt{hello-app.scala}.
\begin{REPLnonum}
> gedit hello-app.scala
\end{REPLnonum}
Skriv dessa rader i filen:


\scalainputlisting{examples/hello-app.scala}

\Subtask Kompilera med \code{scalac hello-app.scala} och kör koden med \code{scala Hello}.
\begin{REPLnonum}
> scalac hello-app.scala
> ls
> scala Hello
\end{REPLnonum}
Vad heter filerna som kompilatorn skapar?

\Subtask Ändra i din kod så att kompilatorn ger följande felmeddelande: \\
\texttt{Missing closing brace}

\Subtask\Pen Varför behövs \code{main}-metoden?

\Subtask\Pen Vilket alternativ går snabbast att köra igång, ett skript eller en kompilerad applikation? Varför? Vilket alternativ kör snabbast när väl exekveringen är igång?


\SOLUTION


\TaskSolved \what
 

\SubtaskSolved  Hello.class och Hello\$.class

\SubtaskSolved  Ta bort en av krullparenteserna i slutet.

\SubtaskSolved  I ett skript behöver man inte skriva någon main-metod. Kompilatorn lägger till en automatiskt precis när koden ska köras. I en applikation behöver man däremot det. För att göra en applikation definierar vi ett objekt som vi i det här fallet kallar för \code{Hello}. Från början gör inte objekt någonting. De bara finns. För att objekt ska kunna göra något behövs det metoder. I vanliga fall utförs inte metoder förrän en annan metod "ropar" på metoden. main-metoden ropas dock automatiskt när en applikation startas. Annars hade ju ingenting hänt, eftersom alla metoderna väntar på att någon annan metod ska börja. \\
\SubtaskSolved  Första gången man ska köra en applikation måste den först kompileras innan den exekveras. Skript kompileras automatiskt samtidigt som de exekveras, vilket totalt sett görs på kortare tid. Därför tar det längre tid att starta en applikation första gången än att starta ett skript första gånge. När en applikation väl har kompileras och kan exekveras, går det dock mycket fortare. Fördelen med applikationer är att de kan exekveras flera gånger utan att kompileras om.



\QUESTEND




%%<AUTOEXTRACTED by mergesolu>%%      %Uppgift 10




\WHAT{Java-applikation.}

\QUESTBEGIN

\Task \label{task:java} \what~   Skapa med hjälp av en editor en fil med namn \texttt{Hi.java}.
\begin{REPLnonum}
> gedit Hi.java
\end{REPLnonum}
Skriv dessa rader i filen:

\javainputlisting{examples/Hi.java}

\noindent Kompilera med \code{javac Hi.java} och kör koden med \code{java Hi}.
\begin{REPLnonum}
> javac Hi.java
> ls
> java Hi
\end{REPLnonum}

\Subtask\Pen Vad heter filen som kompilatorn skapat?

\Subtask\Pen Jämför signaturen för Java-programmets main-metod med signaturen för Scala-programmets main-metod. De betyder samma sak men syntaxen är olika. Beskriv skillnader och likheter i syntaxen.

\Subtask\Pen Vad blir det för felmeddelande om källkodsfilen och klassnamnet inte överensstämmer i ett Java-program?


\SOLUTION


\TaskSolved \what
 

\SubtaskSolved  Hi.class

\SubtaskSolved  I javas syntax börjar man med orden \code{public static}. I scala uteblir dessa. I scala är alla metoder automatiskt publika om inget annat används. Därför behövs aldrig ordet \code{public} i scala. I scala finns det tekniskt sett inga statiska metoder. Men i praktiken fungerar vanliga metoder i ett scala-objekt på ungefär samma sätt som statiska metoder i en java-klass. I scala används ordet \code{def} varje gång en funktion ska definieras. I java slipper man det. I java skriver man returtypen (\code{void}) innan parametrarna. I scala kommer istället metodens returtyp (\code{Unit}) i slutet. Javas \code{void} motsvarar scalas \code{Unit}. I scalas syntax kommer parameterns namn (\code{args}) före parameterns typ (\code{Array[String]}), separerat med ett kolon. I java kommer typen (\code{String[]}) först och sen kommer namnet (\code{args}). \code{String[]} i java betyder ungefär samma sak som \code{Array[String]} i scala.

\SubtaskSolved  -



\QUESTEND




%%<AUTOEXTRACTED by mergesolu>%%      %Uppgift 11




\WHAT{Algoritm: SUMBUG}

\QUESTBEGIN

\Task  \what~ . Nedan återfinns pseudo-koden för SUMBUG.

\begin{algorithm}[H]
 \SetKwInOut{Input}{Indata}\SetKwInOut{Output}{Resultat}

 \Input{heltalet $n$}
 \Output{utskrift av summan av de första $n$ heltalen }
 $sum \leftarrow 0$ \\
 $i \leftarrow 1$  \\
 \While{$i \leq n$}{
  $sum \leftarrow sum + 1$
 }
 skriv ut $sum$
\end{algorithm}

\Subtask\Pen Kör algoritmen steg för steg med penna och papper, där du skriver upp hur värdena för respektive variabel ändras. Det finns två buggar i algoritmen. Vilka? Rätta buggarna och test igen genom att ''köra'' algoritmen med penna på papper och kontrollera så att algoritmen fungerar för $n=0$, $n=1$, och $n=5$. Vad händer om $n=-1$?

\Subtask Skapa med hjälp av en editor filen \code{sumn.scala}. Implementera algoritmen SUM enligt den rättade pseudokoden och placera implementationen i en main-metod i ett objekt med namnet \code{sumn}. Du kan skapa indata \code{n} till algoritmen med denna deklaration i början av din main-metod: \\ \code{val n = args(0).toInt} \\ Vad ger applikationen för utskrift om du kör den med argumentet 8888?

\begin{REPLnonum}
> scalac sumn.scala
> scala sumn 8888
\end{REPLnonum}

\Subtask Kontrollera att din implementation räknar rätt genom att jämföra svaret med detta uttrycks värde, evaluerat i Scala REPL:
\begin{REPLnonum}
scala> (1 to 8888).sum
\end{REPLnonum}

\Subtask Implementera algoritmen SUM enligt pseudokoden ovan, men nu i Java. Skapa filen \code{SumN.java} och använd koden från uppgift \ref{task:java} som mall för att deklarera den publika klassen \code{SumN} med en main-metod. Några tips om Java-syntax och standarfunktioner i Java:

\begin{itemize}[noitemsep, nolistsep]
\item Alla satser i Java måste avslutas med semikolon.
\item Heltalsvariabler deklareras med nyckelordet \lstinline[language=Java]{int} (litet i).
\item Typnamnet ska stå \emph{före} namnet på variabeln. Exempel: \\ \lstinline[language=Java]{int sum = 0;}
\item Indexering i en array görs i Java med hakparenteser: \code{args[0]}
\item I stället för Scala-uttrycket \code{args(0).toInt}, använd Java-uttrycket: \\ \code{Integer.parseInt(args[0])}
\item \code{while}-satser i Scala och Java har samma syntax.
\item Utskrift i Java görs med \code{System.out.println}
\end{itemize}


\SOLUTION


\TaskSolved \what
 

\SubtaskSolved  Bugg: Eftersom \code{i} inte ökar, fastnar programmet i en oändlig loop. Fix: Lägg till en sats i slutet av while-blocket som ökar värdet på i med 1.
Bugg: Eftersom man bara ökar summan med 1 varje gång, kommer resultatet att bli summan av n stycken 1or, inte de n första heltalen. Fix: Ändra så att summan ökar med \code{i} varje gång, istället för 1.
För -1, blir resultatet 0. Förklaring: i börjar på 1 och är alltså aldrig mindre än n som ju är -1. while-blocket genomförs alltså noll gånger, och efter att \code{sum} får sitt ursprungsvärde förändras den aldrig.
\SubtaskSolved  39502716
\SubtaskSolved  -
\SubtaskSolved  Såhär kan implementationen se ut:
\begin{Code}
public class SumN {
  public static void main(String[] args) {
    int n = Integer.parseInt(args[0]);
    int sum = 0;
    int i = 1;
    while(i <= n){
      sum = sum + i;
      i = i + 1;
      }
    }
    System.out.println(sum);
}
\end{Code}



\QUESTEND




%%<AUTOEXTRACTED by mergesolu>%%      %Uppgift 12




\WHAT{Algoritm: MAXBUG}

\QUESTBEGIN

\Task  \what~ . Nedan återfinns pseudo-koden för MAXBUG.

\begin{algorithm}[H]
 \SetKwInOut{Input}{Indata}\SetKwInOut{Output}{Resultat}

 \Input{Array $args$ med strängar som alla innehåller heltal}
 \Output{utskrift av största heltalet }
 $max \leftarrow$ det minsta heltalet som kan uppkomma  \\
 $n \leftarrow $ antalet heltal \\
 $i \leftarrow 0$ \\
 \While{$i < n$}{
   $x \leftarrow args(i).toInt$ \\
   \If{( x > $max$)}{$max \leftarrow x$}
  % $i \leftarrow i + 1$
 }
 skriv ut $max$
\end{algorithm}

\Subtask\Pen Kör med penna och papper. Det finns en bugg i algoritmen ovan. Vilken? Rätta buggen.

\Subtask Implementera algoritmen MAX (utan bugg) som en Scala-applikation. Tips:
\begin{itemize}[noitemsep, nolistsep]
\item Det minsta \code{Int}-värdet som någonsin kan uppkomma: \code{Int.MinValue}
\item Antalet element i $args$ ges av: \code{args.size}
\end{itemize}

\begin{REPL}
> gedit maxn.scala
> scalac maxn.scala
> scala maxn 7 42 1 -5 9
42
\end{REPL}

\Subtask\Pen \label{subtask:arg0} Skriv om algoritmen så att variabeln $max$ initialiseras med det första talet i sekvensen.

\Subtask Implementera den nya algoritmvarianten från uppgift \ref{subtask:arg0} och prova programmet. Vad händer om $args$ är tom?

\SOLUTION


\TaskSolved \what
 

\SubtaskSolved  Bugg: i ökar aldrig. Programmet fastnar i en oändlig loop. Fix: Lägg till en sats som ökar i med 1, i slutet av while-blocket.

\SubtaskSolved  Så här kan implementationen se ut:
\begin{Code}
object Max {
  def main(args: Array[String]): Unit = {
    var max = Int.MinValue
    val n = args.size
    var i = 0
    while(i < n) {
      val x = args(i).toInt
      if(x > max) {
        max = x
      }
      i = i + 1
    }
    println(max)
  }
}
\end{Code}
\SubtaskSolved  Raden där max initieras ändras till \code{var max = args(0).toInt} 

\SubtaskSolved  \code{java.lang.ArrayIndexOutOfBoundsException: 0}



\QUESTEND




%%<AUTOEXTRACTED by mergesolu>%%      %Uppgift 13




\WHAT{Block, namnsynlighet, namnöverskuggning}

\QUESTBEGIN

\Task  \what~ . Kör nedan kod i Scala REPL eller i Kojo. Vad händer nedan? Varför?

\Subtask \code|val a = {1 + 1; 2 + 2; 3 + 3; 4 + 4}; println(a)|

\Subtask \code|val b = {1; 2; 3; {val b = 4; b + b; b + 1}}; println(b)|

\Subtask \code|{val a = 42; println(a)}|

\Subtask \code|{val a = 42}; println(a)|

\Subtask \code|{val a = 42; {val a = 43; println(a)}; println(a)}|

\Subtask \code|{var a = 42; {a = a + 1}; var a = 43}|

\Subtask \code|{var a = 42; {a = a + b; var b = 43}; println(a)}|

\Subtask \code|{var a = 42; {var b = 43; a = a + b}; println(a)}|

\Subtask \code|{var a = 42; {a = a + b; def b = 43}; println(a)}|

\Subtask \code|{object a{var b=42;object a{var a=43}};println(a.b+a.a.a)}|

\Subtask

\begin{Code}
{
  object a {
    var b = 42
    object a {
      var a = 43
    }
  }
  println(a.b + a.a.a)
}
\end{Code}

\Subtask Vad är fördelen med att namn deklarerade inne i ett block är lokala i stället för globala?


\SOLUTION


\TaskSolved \what


\SubtaskSolved  Skriver ut talet 8. \code{a} får värdet \code{4 + 4} eftersom detta är den sista satsen i blocket. Man får också tre stycken varningar. Detta beror på att det förekommer tre satser i blocket som inte gör någon skillnad.

\SubtaskSolved  Skriver ut talet 5. De tre första satserna i det yttre blocket ignoreras. \code{b} får värdet som returneras av det yttre blocket. Det yttre blocket returnerar värdet som returneras i den sista satsen i blocket, som i sin tur är ett block. I det inre blocket skapas en ny \code{val} som också får namnet \code{b}. Notera att detta alltså inte är samma värde, även om det har samma namn. Den andra satsen räknar summan av \code{b} med sig själv. Eftersom vi nu befinner oss i det block där det andra \code{b}et precis har definieras så är det detta \code{b} som används och summan blir alltså åtta. Detta är dock helt irrelevant eftersom resultatet inte sparas någonstans. I den sista satsen blir resultatet 5 (eftersom \code{b} är fyra och vi adderar ett). Detta resultatet returneras från det innre blocket och vidare ur det yttre blocket.

\SubtaskSolved  Skriver ut talet 42. Blockets satser exekveras i ordning. 

\SubtaskSolved  Skriver inte ut 42. I blocket skapas ett \code{val} med namnet \code{a} och värdet \code{42}. Detta värde finns inte utanför blocket och kommer därför inte att skrivas ut. Om du däremot definierat \code{a} som något annat tidigare så kommer istället det värdet att skrivas ut.

\SubtaskSolved  Skriver först ut \code{43} och sedan \code{42}. Förklaring:

\code{a} initieras med värdet \code{42}. Ett nytt värde som också har namnet \code{a} initieras med värdet \code{43}. Eftersom detta sker innanför ett nytt block, befinner vi oss i ett annat "namespace" och det gör alltså inget att vi använder samma namn. \code{a} skrivs ut. Eftersom vi befinner oss i det inre blocket är det \code{43} som skrivs ut, inte \code{42}. Scala kollar först efter värden som heter \code{a} i det inre "namespacet". Det är först i andra hand som den skulle upptäcka att det finns ett \code{a} i det yttre blocket. Till sist körs den sista satsen i det yttre blocket. Då skrivs \code{a} ut. Eftersom vi nu befinner oss i det yttre blocket, vet inte ens scala om att det andra \code{a}:et existerar. Resultatet av den här utskriften blir alltså \code{42}.

\SubtaskSolved  Ett fel uppstår. Variabeln \code{a} initieras två gånger i samma namespace. Förklaring till felet:

I det yttre blockets första sats initieras variablen \code{a} med värdet \code{42}. I det yttre blockets tredje sats försöker vi definiera en ny variabel med samma namn. I och med att vi befinner oss i samma namespace, krockar namnen.

Förklaring till vad som händer i sats två:

I det inre blocket har vi inte definierat någon variabel \code{a}. Till en början hittar alltså inte scala något sådant. Då letar scala vidare i det namespace som finns utanför det inre blocket och hittar variabeln som vi definierade i det yttre blockets första sats. Denna variabel får sitt värde förändrat.

\SubtaskSolved  Fel. Framåtreferens. Förklaring:

Det är inte tillåtet att referera till variabler som initieras senare i koden.

\SubtaskSolved  Skriver ut \code{85}. Förklaring:

I och med att vi den här gången initierade variabeln \code{b} och gav den ett värde innan vi använder oss av den, slipper vi problemet ovan.

\SubtaskSolved  Skriver ut \code{85}. Förklaring:

Det är tillåtet att referera till funktioner som definieras senare i koden.

\SubtaskSolved  Skriver ut \code{85}. Förklaring:

\code{a.b} refererar till variabeln \code{b} som ingår i objektet \code{a}.
\code{a.a.a} refererar till variabeln \code{a}, som ingår i ett objekt som heter \code{a} som i sin tur befinner sig i ett annat objekt som också heter \code{a}.

\SubtaskSolved  Skriver ut \code{85}. Förklaring:

Koden är identisk med förra deluppgiften förutom att ny rad används istället för semikolon.

\SubtaskSolved  I stora projekt med mycket kod, kan det vara svårt att hitta unika namn till alla sina variabler. Då är det en fördel om man kan hålla sina variabler i begränsade namespaces, så att de bara är tillgängliga precis när de behöver användas. 



\QUESTEND




%%<AUTOEXTRACTED by mergesolu>%%      %Uppgift 14??? NUMMER I KOMMENTAR STÄMMER EJ MED GENERERAT NUMMER




\WHAT{Paket, \code{import} och klassfilstrukturer.}

\QUESTBEGIN

\Task \label{task:package} \what~   Med Java-8-plattformen kommer 4240 färdiga klasser, som är organiserade i 217 olika paket.\footnote{Se Stackoverflow: \href{http://stackoverflow.com/questions/3112882/how-many-classes-are-there-in-java-standard-edition}{how-many-classes-are-there-in-java-standard-edition}}

\Subtask Vilka paket finns i paketet javax som börjar på s?

\begin{REPLnonum}
scala> javax.s   //tryck på TAB-tangenten
\end{REPLnonum}

\Subtask Kör raderna nedan i REPL. Beskriv vad som händer för varje rad.
\begin{REPL}[numbers=left, numberstyle=\color{black}\ttfamily\scriptsize\selectfont]
scala> import javax.swing.JOptionPane
scala> def msg(s: String) = JOptionPane.showMessageDialog(null, s)
scala> msg("Hej på dej!")
scala> def input(msg: String) = JOptionPane.showInputDialog(null, msg)
scala> input("Vad heter du?")
scala> import JOptionPane.{showOptionDialog => optDlg}
scala> def inputOption(msg: String, opt: Array[Object]) =
         optDlg(null, msg, "Option", 0, 0, null, opt, opt(0))
scala> inputOption("Vad väljer du?", Array("Sten", "Sax", "Påse"))
\end{REPL}

\Subtask\Pen Vad hade du behövt ändra på efterföljande rader om import-satsen på rad 1 ovan ej hade gjorts?

\Subtask Skapa med en editor filen paket.scala och kompilera. Rita en bild av hur katalogstrukturen ser ut.

\begin{Code}
package gurka.tomat.banan

package p1 {
  package p11 {
    object hello {
      def hello = println("Hej paket p1.p11!")
    }
  }
  package p12 {
    object hello {
      def hello = println("Hej paket p1.p12!")
    }
  }
}

package p2 {
  package p21 {
    object hello {
      def hello = println("Hej paket p2.p21!")
    }
  }
}

object Main {
  def main(args: Array[String]): Unit = {
    import p1._
    p11.hello.hello
    p12.hello.hello
    import p2.{p21 => apelsin}
    apelsin.hello.hello
  }
}
\end{Code}

\begin{REPL}
> gedit paket.scala
> scalac paket.scala
> scala gurka.tomat.banan.Main
> ls -R
\end{REPL}

\SOLUTION


\TaskSolved \what
 

\SubtaskSolved  \code{script   security   smartcardio   sound   sql   swing}

\SubtaskSolved  Radernas funktion i ordning:

1. Importerar JOptionPane från javax.swing

2. Definierar en metod som tar en sträng och öppnar en dialogruta med strängen.

3. Testar funktionen med argumentet "Hej på dej!". En dialogruta öppnas med texten "Hej på dej!".

4. Definierar en metod som tar emot en sträng som argument och öppnar en input-dialogruta med strängen.

5. Testar funktionen med argumentet "Vad heter du?". En dialogruta öppnas med texten "Vad heter du?". I ett fält kan man fylla i sitt namn. Funktionen returnerar namnet.

6. Importerar showOptionDialog från JOptionPane under namnet optDlg.

7. Definierar en metod som tar emot en sträng och en Array som argument och öppnar en flervalsdialog. Strängen ska innehålla frågan som flervalsdialogen visar upp. Arrayn ska innehålla alternativen som användaren ska välja mellan.

8.Testar funktionen med argumenten \code{"Vad väljer du?"} och \\ \code{Array("Sten, "Sax", "Påse")}. En dialogruta kommer upp och man får möjlighet att välja sten sax eller påse. Funktionen returnerar valet som man gör.

\SubtaskSolved  På alla ställen där \code{JOptionPane} förekommer, hade man istället fått skriva \code{javax.swing.JOptionPane}.

\SubtaskSolved  -



\QUESTEND




%%<AUTOEXTRACTED by mergesolu>%%      %Uppgift 15




\WHAT{Skapa \code{jar}-filer och använda classpath}

\QUESTBEGIN

\Task  \what~ 

\Subtask Skriv kommandot \code{jar} i terminalen och undersök vad som finns för optioner. Se speciellt ''Example 1.'' i hjälputskriften. Vilket kommando ska du använda för att packa ihop flera filer i en enda jar-fil?

\Subtask Som en fortsättning på uppgift \ref{task:package}, packa ihop biblioteket \code{gurka} i en jar-fil med nedan kommando, samt kör igång REPL med jar-filen på classpath.

\begin{REPL}
> jar cvf mittpaket.jar gurka
> scala -cp mittpaket.jar
scala> gurka.tomat.banan.Main.main(Array())
\end{REPL}


\SOLUTION


\TaskSolved \what
 

\SubtaskSolved  jar cvf [namn på skapad fil] [namn på input-filer]

\SubtaskSolved  -



\QUESTEND




%%<AUTOEXTRACTED by mergesolu>%%      %Uppgift 16




\WHAT{Skapa dokumentation med \code{scaladoc}-kommandot}

\QUESTBEGIN

\Task  \what~ 

\Subtask Som en fortsättning på uppgift \ref{task:package}, kör nedan kommando i terminalen:

\begin{REPL}
> scaladoc paket.scala
> ls
> firefox index.html   # eller öppna index.html i valfri webbläsare
\end{REPL}

Vad händer?

\Subtask Lägg till några fler metoder i något av objekten i filen \code{paket.scala} och lägg även till några dokumentationskommentarer. Kompilera om och kör. Generera om dokumentationen.

\begin{verbatim}
//... ändra i filen paket.scala

/** min paketdokumentationskommentar p2 */
package p2 {
  /** min paketdokumentationskommentar p21 */
  package p21 {
    /** ett hälsningsobjekt */
    object hello {
      /** en hälsningsmetod i p2.p21 */
      def hello = println("Hej paket p2.p21!")

      /** en metod som skriver ut tiden */
      def date = println(new java.util.Date)
    }
  }
}

\end{verbatim}

\begin{REPL}
> gedit paket.scala
> scalac paket.scala
> jar cvf mittpaket.jar gurka
> scala -cp mittpaket.jar
scala> gurka.tomat.banan.p2.p21.hello.date
scala> :q
> scaladoc paket.scala
> firefox index.html
\end{REPL}

\newpage

\ExtraTasks %%%%%%%%%%%%%%%%%%%

\SOLUTION


\TaskSolved \what
 

\SubtaskSolved  -

\SubtaskSolved  -
\QUESTEND






\WHAT{NEEDS A TOPIC DESCRIPTION}

\QUESTBEGIN

\Task \label{task:minindex} \what~  Implementera algoritmen MININDEX som söker index för minsta heltalet i en sekvens. Pseudokod för algoritmen MININDEX:

\begin{algorithm}[H]
 \SetKwInOut{Input}{Indata}\SetKwInOut{Output}{Utdata}

 \Input{Sekvens $xs$ med $n$ st heltal.}
 \Output{Index för det minsta talet eller $-1$ om $xs$ är tom.  }
 $minPos \leftarrow 0 $\\
 $i \leftarrow 1$ \\
 \While{$i < n$}{
   \If{xs(i) < $xs(minPos)$}{$minPos \leftarrow i$}
   $i \leftarrow i + 1$
 }
 \eIf{$n > 0$}{\Return{$minPos$}}{\Return{$-1$}}
\end{algorithm}

\Subtask Prova algoritmen med penna och papper på sekvensen $(1, 2, -1, 4)$ och rita minnessituationen efter varje runda i loopen. Vad blir skillnaden i exekveringsförloppet om loopvariablen $i$  initialiserats till $0$ i stället för $1$?

\Subtask Implementera algoritmen MININDEX i Scala i en funktion med denna signatur:
\begin{Code}
def indexOfMin(xs: Array[Int]): Int = ???
\end{Code}
Testa för olika fall: tom sekvens; sekvens med endast ett tal; lång sekvens med det minsta talet först, någonstans mitt i, samt sist.

\begin{Code}
// kod till facit
def indexOfMin(xs: Array[Int]): Int = {
  var minPos = 0
  var i = 1
  while (i < xs.size) {
    if (xs(i) < xs(minPos)) minPos = i
    i += 1
  }
  if (xs.size > 0) minPos else -1
}


\end{Code}

\newpage

\AdvancedTasks %%%%%%%%%%%%%%%%%


\SOLUTION


\QUESTEND






\WHAT{NEEDS A TOPIC DESCRIPTION}

\QUESTBEGIN

\Task  \what~ Läs om krullparenteser och vanliga parenteser på stack overflow: \\ \href{http://stackoverflow.com/questions/4386127/what-is-the-formal-difference-in-scala-between-braces-and-parentheses-and-when}{stackoverflow.com/questions/4386127/what-is-the-formal-difference-in-scala-between-braces-and-parentheses-and-when} och prova själv i REPL hur du kan blanda dessa olika slags parenteser på olika vis.

\SOLUTION


\QUESTEND






\WHAT{Tips:}

\QUESTBEGIN

\Task  \what~ Gör jämförande studier av Scalas api-dokumentation för \code{ArrayBuffer}, \code{Array} och \code{Vector}. Ge exempel på metoder som finns på objekt av typen \code{Array} och \code{ArrayBuffer} men inte på objekt av typen \code{Vector}.  Kolla efter metoder som returnerar \code{Unit}. Prova några muterande metoder på \code{Array} och \code{ArrayBuffer} i REPL.

\SOLUTION


\QUESTEND






\WHAT{Tips:}

\QUESTBEGIN

\Task  \what~ Bygg vidare på koden nedan och gör ett Sten-Sax-Påse-spel\footnote{\href{https://sv.wikipedia.org/wiki/Sten,\_sax,\_p\%C3\%A5se}{sv.wikipedia.org/wiki/Sten,\_sax,\_p\%C3\%A5se}} som även meddelar vem som vinner. Koden fungerar att köra som den är, men funktionen \code{winnerMsg} är ej klar.  Du kan använda modulo-räkning med \code{%}-operatorn för att avgöra vem som vinner.

\begin{Code}[basicstyle=\ttfamily\footnotesize\selectfont]]
object Rock {
  import javax.swing.JOptionPane
  import JOptionPane.{showOptionDialog => optDlg}

  def inputOption(msg: String, opt: Vector[String]) =
    optDlg(null, msg, "Option", 0, 0, null, opt.toArray[Object], opt(0))

  def msg(s: String) = JOptionPane.showMessageDialog(null, s)

  val opt =  Vector("Sten", "Sax", "Påse")

  def userChoice = inputOption("Vad väljer du?", opt)

  def computerChoice = (math.random * 3).toInt

  def winnerMsg(user: Int, computer: Int) = "??? vann!"

  def main(args: Array[String]): Unit = {
    var keepPlaying = true
    while (keepPlaying) {
      val u = userChoice
      val c = computerChoice
      msg("Du valde " + opt(u) + "\n" +
          "Datorn valde " + opt(c) + "\n" +
          winnerMsg(u, c))
      if (u != c) keepPlaying = false
    }
  }
}
\end{Code}\SOLUTION


\QUESTEND


%!TEX encoding = UTF-8 Unicode
%!TEX root = ../compendium1.tex

\ifPreSolution

\Exercise{\ExeWeekTHREE}\label{exe:W03}
\begin{Goals}
%!TEX encoding = UTF-8 Unicode
%!TEX root = ../exercises.tex

\item Kunna skapa och använda funktioner med en eller flera parametrar, default-argument, namngivna argument, och uppdelad parameterlista.
\item Kunna använda funktioner som äkta värden.
\item Kunna skapa och använda anonyma funktioner (ä.k. lambda-funktioner).
\item Kunna applicera en funktion på element i en samling.
\item Förstå skillnader och likheter mellan en funktion och en procedur.
\item Förstå vad ett block och en lokal variabel är.
\item Kunna skapa och använda lokala funktioner och förklara nyttan med dessa.
\item Förstå skillnader och likheter mellan värdeanrop och namnanrop.
\item Kunna skapa en enkel kontrollstruktur med fördröjd evaluering av ett block.
\item Förstå skillnaden mellan äkta funktioner och funktioner med sidoeffekter.
%\item Kunna skapa och använda variabler med fördröjd initialisering och förstå när de är användbara.
\item Kunna förklara hur nästlade funktionsanrop sker med   aktiveringsposter.
\item Känna till rekursion och kunna förklara hur rekursiva funktioner fungerar.
\item Känna till att det går att partiellt applicera argument på funktioner med uppdelad parameterlista för att skapa s.k. stegade funktioner (ä.k. curry-funktioner).

%\item Känna till svansrekursion och att svansrekursiva funktioner kan optimeras till loopar.

\end{Goals}

\begin{Preparations}
\item \StudyTheory{03}
\end{Preparations}

\BasicTasks %%%%%%%%%%%%%%%%

\else

\ExerciseSolution{\ExeWeekTHREE}

\fi





\WHAT{Para ihop begrepp med beskrivning.}

\QUESTBEGIN

\Task \what~Koppla varje begrepp med den (förenklade) beskrivning som passar bäst:

\begin{ConceptConnections}
  funktionshuvud & 1 & & A & har parameterlista och eventuellt en returtyp \\ 
  funktionskropp & 2 & & B & beskriver namn och typ på parametrar \\ 
  parameterlista & 3 & & C & argumentet evalueras innan anrop \\ 
  block & 4 & & D & en funktion som anropar sig själv \\ 
  namngivna argument & 5 & & E & gör att argument kan utelämnas \\ 
  defaultargument & 6 & & F & koden som exekveras vid funktionsanrop \\ 
  värdeanrop & 7 & & G & gör att en funktion kan flera resultatvärden \\ 
  namnanrop & 8 & & H & gör att argument kan ges i valfri ordning \\ 
  tupel & 9 & & I & fördröjd evaluering av argument \\ 
  tupelreturtyp & 10 & & J & kan ha lokala namn; sista raden ger värdet \\ 
  äkta funktion & 11 & & K & funktion utan namn; kallas även lambda \\ 
  predikat & 12 & & L & ger alltid samma resultat om samma argument \\ 
  slumptalsfrö & 13 & & M & lista med bestämt antal (heterogena) värden \\ 
  anonym funktion & 14 & & N & ger återupprepningsbar sekvens av pseudoslumptal \\ 
  rekursiv funktion & 15 & & O & en funktion som ger ett booleskt värde \\ 
\end{ConceptConnections}

\SOLUTION

\TaskSolved \what

\begin{ConceptConnections}
  funktionshuvud & 1 & ~~\Large$\leadsto$~~ &  K & har parameterlista och eventuellt returtyp \\ 
  funktionskropp & 2 & ~~\Large$\leadsto$~~ &  M & koden som exekveras vid funktionsanrop \\ 
  parameterlista & 3 & ~~\Large$\leadsto$~~ &  I & beskriver namn och typ på parametrar \\ 
  parameter & 4 & ~~\Large$\leadsto$~~ &  N & namn i funktionshuvud; binds till argument \\ 
  argument & 5 & ~~\Large$\leadsto$~~ &  E & uttryck som är invärde vid funktionsanrop \\ 
  block & 6 & ~~\Large$\leadsto$~~ &  G & kan ha lokala namn; sista raden ger värdet \\ 
  namngivna argument & 7 & ~~\Large$\leadsto$~~ &  H & gör att argument kan ges i valfri ordning \\ 
  default-argument & 8 & ~~\Large$\leadsto$~~ &  L & gör att argument kan utelämnas \\ 
  värdeanrop & 9 & ~~\Large$\leadsto$~~ &  B & argumentet evalueras innan anrop \\ 
  namnanrop & 10 & ~~\Large$\leadsto$~~ &  A & fördröjd evaluering av argument \\ 
  tupel & 11 & ~~\Large$\leadsto$~~ &  J & lista med bestämt antal (heterogena) värden \\ 
  tupelreturtyp & 12 & ~~\Large$\leadsto$~~ &  D & gör att en funktion kan flera resultatvärden \\ 
  anonym funktion & 13 & ~~\Large$\leadsto$~~ &  F & funktion utan namn; kallas även lambda \\ 
  rekursiv funktion & 14 & ~~\Large$\leadsto$~~ &  C & en funktion som anropar sig själv \\ 
\end{ConceptConnections}

\QUESTEND





\WHAT{Definiera och anropa funktioner.}

\QUESTBEGIN

\Task \label{task:funcall} \what~
En funktion med en parameter definieras med följande syntax i Scala:
\vspace{0.5em} \\
\texttt{\code{def} \textit{namn}(\textit{parameter}: \textit{Typ} = \textit{defaultArgument}): \textit{Returtyp} = \textit{returvärde}}

% En funktion med två parametrar definieras med följande syntax i Scala: \vspace{0.5em} \\  \texttt{\code{def} \textit{namn}(\textit{parameter1}: \textit{Typ1}, \textit{parameter2}: \textit{Typ2}): \textit{Returtyp} = \textit{returvärde}}

\Subtask Definiera funktionen \code{öka} som har en heltalsparameter \code{x} och vars returvärde är argumentet plus 1. Defaultargument ska vara 1. Ange returtypen explicit.

\Subtask Vad har uttrycket \code{öka(öka(öka(öka())))} för värde?

\Subtask Definiera funktionen \code{minska} som har en heltalsparameter \code{x} och vars returvärde är argumentet minus 1. Defaultargument ska vara 1. Ange returtypen explicit.

\Subtask Vad är värdet av uttrycket \code{öka(minska(öka(öka(minska(minska())))))}

\Subtask Vad är det för skillnad mellan parameter och argument?

\SOLUTION

\TaskSolved \what

\SubtaskSolved
\begin{Code}
def öka(x: Int = 1): Int = x + 1
\end{Code}

\SubtaskSolved  \code{5}

\SubtaskSolved
\begin{Code}
def minska(x: Int = 1): Int = x - 1
\end{Code}

\SubtaskSolved  \code{1}

\SubtaskSolved
\begin{itemize}
  \item \emph{Kort, förenklad förklaring:} Parametern i funktionshuvudet är ett lokalt namn på indata som kan användas i funktionskroppen, medan argumentet är själva värdet på parametern som skickas med vid anrop.
  \item \emph{Längre, mer exakt förklaring:} En \textbf{parameter} är en deklaration av en oföränderlig variabel i ett funktionshuvud vars namn finns tillgängligt lokalt i funktionskroppen. Vid anrop \emph{binds} parameternamnet till ett specifikt argument. Ett \textbf{argument} är ett uttryck som  appliceras på en funktion vid anrop. Normalt evalueras argumentet innan anropet sker, men om parametertypen föregås av \code{=>} fördröjs evalueringen av argumentet och sker i stället \emph{varje gång} parameternamnet förekommer i funktionskroppen.
\end{itemize}

\QUESTEND



\WHAT{Implementera funktion på olika sätt.}

\QUESTBEGIN

\Task \label{task:funcsumfirst} \what~
Skapa en funktion som kan summera de första \code{n} positiva heltalen.

\Subtask Skriv först funktionshuvudet med \code{???} som funktionskropp. Ge funktionen ett bra namn. Ange returtyp. Kontrollera att din funktion kompilerar utan kompileringsfel innan du går vidare.

\Subtask Implementera funktionen med hjälp av ett intervall och metoden \code{sum}. Testa så att funktionen fungerar. Vad händer om du ger ett negativt argument?

\Subtask Implementera funktionen med hjälp av \code{while}-\code{do}. Vad händer om du ger ett negativt argument?

\SOLUTION

\TaskSolved \what

\SubtaskSolved
\begin{Code}
def sumFirst(n: Int): Int = ???
\end{Code}

\SubtaskSolved
\begin{Code}
def sumFirst(n: Int): Int = (1 to n).sum
\end{Code}
\begin{REPL}
scala> sumFirst(-1)
val res0: Int = 0
\end{REPL}

\SubtaskSolved
\begin{Code}
def sumFirst(n: Int): Int = 
  var result = 0
  var i = 1
  while i <= n do 
    result += i
    i += 1
  end while
  result
end sumFirst
\end{Code}
\begin{REPL}
scala> sumFirst(-1)
val res1: Int = 0
\end{REPL}

\QUESTEND




\WHAT{Textspelet AliensOnEarth.}

\QUESTBEGIN

\Task  \what~Ladda ner spelet nedan \footnote{
\url{https://raw.githubusercontent.com/lunduniversity/introprog/master/compendium/examples/AliensOnEarth.scala}} och studera koden.

\scalainputlisting[basicstyle=\ttfamily\fontsize{10}{12}\selectfont,numbers=left]{examples/AliensOnEarth.scala}

% def randomDistribution(weights: Vector[Int]): Int = {
%   require(weights.size > 0)
%   require(weights.forall(_ >= 0))
%
%   val probabilities = for (w <- weights) yield w / weights.sum.toDouble
%   val rnd = math.random()
%   var i = 0
%   var sum = probabilities(i)
%   while (i < probabilities.size - 1 && rnd > sum) {
%     i += 1
%     sum += probabilities(i)
%   }
%   i
% }

\Subtask Medan du läser koden, försök lista ut vilket som är bästa strategin för att få så mycket poäng som möjligt. Kompilera och kör spelet i terminalen med ditt favoritnamn som argument. Vilket av de tre objekten på planeten jorden har störst sannolikhet att vara bästa alternativet?

\Subtask Para ihop kodsnuttarna nedan med bästa beskrivningen.\footnote{Gör så gott du kan även om allt inte är solklart. Vissa saker kommer vi att gå igenom i detalj först under senare kursmoduler.}

\begin{ConceptConnections}
  \code|options.indices| & 1 & & A & fångar undantag för att förhindra krasch \\ 
  \code|"1X2".toLowercase| & 2 & & B & gör om en sträng till små bokstäver \\ 
  \code|Random.nextInt(n)| & 3 & & C & slumptal i intervallet \code|0 until n| \\ 
  \code|try { } catch { }| & 4 & & D & sträng som kan sträcka sig över flera kodrader \\ 
  \code|""" ... """| & 5 & & E & heltalssekvens med alla index i en sekvens \\ 
  \code|s.stripMargin| & 6 & & F & tar bort marginal till och med vertikalstreck \\ 
  \code|e.printStackTrace| & 7 & & G & skriver ut information om ett undantag \\ 
\end{ConceptConnections}

\noindent\emph{Tips:} Med hjälp av REPL kan du ta reda på hur olika delar fungerar, t.ex.:

\begin{REPL}
scala> val xs = Vector("p", "w", "a")
scala> xs.indices
scala> xs.indices.foreach(i => println(i))
scala> xs.indexOf("w")
scala> xs.indexOf("gurka")
scala> Vector("hej", "hejsan", "hej").indexOf("hej")
scala> try 1 / 0 catch case e: Exception => println(e)
\end{REPL}
%Kolla även dokumentationen för \code{nextInt}, \code{readLine}, m.fl genom att söka här: \\ \url{http://www.scala-lang.org/api/current/index.html}


%\begin{framed}
\noindent\emph{Tips inför fortsättningen:}

\begin{itemize}[nolistsep]
  \item När jag hittade på \code{AliensOnEarth} började jag med ett mycket litet program med en enkel \code{main}-funktion som bara skrev ut något kul. Sedan byggde jag vidare på programmet steg för steg och kompilerade och testade efter varje liten ändring.

  \item När jag kodar har jag REPL igång i ett eget terminalfönster och min kodeditor i ett annat fönster. I ett tredje fönster har jag en terminal med kompilering i \textit{watch mode}, se appendix \ref{appendix:build-scala-cli-watch-mode}. Fråga en handledare om hur du kan arbeta effektivt med stegvisa experimentering i REPL för att bygga upp ett allt större program i små steg.

  \item Detta arbetssätt tar ett tag att komma in i, men är ett bra sätt att uppfinna allt större och bättre program. Ett stort program byggs lättast i små steg och felsökning blir mycket lättare om man bara gör små tillägg åt gången.

  \item Du får också det mycket lättare att förstå ditt program om du delar upp koden i många korta funktioner med bra namn. Du kan sedan lättare hitta på mer avancerade funktioner genom att återanvända befintliga.

  \item Under veckans laboration ska du utveckla ditt eget textspel. Då har du nytta av att återanvända funktionerna för indata och slumpdragning från exempelprogrammet \code{AliensOnEarth}.
\end{itemize}

%\end{framed}


\SOLUTION

\TaskSolved \what~

\SubtaskSolved \code{"penguin"} är bästa alternativ med sannolikheten $\frac{1}{2} + \frac{1}{2}\cdot\frac{1}{3} = \frac{2}{3}$

\SubtaskSolved

\begin{ConceptConnections}
    \code|options.indices| & 1 & ~~\Large$\leadsto$~~ &  F & heltalssekvens med alla index i en sekvens \\ 
  \code|"1X2".toLowercase| & 2 & ~~\Large$\leadsto$~~ &  C & gör om en sträng till små bokstäver \\ 
  \code|Random.nextInt(n)| & 3 & ~~\Large$\leadsto$~~ &  D & slumptal i intervallet \code|0 until n| \\ 
  \code|try { } catch { }| & 4 & ~~\Large$\leadsto$~~ &  B & fångar undantag för att förhindra krasch \\ 
  \code|""" ... """| & 5 & ~~\Large$\leadsto$~~ &  G & sträng som kan sträcka sig över flera kodrader \\ 
  \code|s.stripMargin| & 6 & ~~\Large$\leadsto$~~ &  A & tar bort marginal till och med vertikalstreck \\ 
  \code|e.printStackTrace| & 7 & ~~\Large$\leadsto$~~ &  E & skriver ut information om ett undantag \\ 
\end{ConceptConnections}

\QUESTEND



\WHAT{Äkta funktioner.}

\QUESTBEGIN

\Task  \what~  En äkta funktion%
\footnote{Äkta funktioner uppfyller per definition  \textit{referentiell transparens} \Eng{referential transparency} som du kan läsa mer om här:  \href{https://simple.wikipedia.org/wiki/Referential_transparency}{simple.wikipedia.org/wiki/Referential\_transparency}}
\Eng{pure function} ger alltid samma resultat med samma argument (så som vi är vana vid inom matematiken) och har inga externt observerbara sidoeffekter (till exempel utskrifter).

Vilka funktioner nedan är äkta funktioner?
\begin{Code}
var x = 0
val y = x

def inc(i: Int) = i + 1

def nöff(i: Int) = 
  x = x + i
  "nöff " * x
end nöff

def addX(i: Int) = x + i

def addY(i: Int) = y + i

def isPalindrome(s: String) = s == s.reverse

def rnd(min: Int, max: Int) = math.random() * max + min
\end{Code}


\noindent\emph{Tips:} Skriv av och testa funktionerna i REPL en och en, så att du förstår exakt vad som händer.

\SOLUTION

\TaskSolved \what

\begin{itemize}
  \item Funktionerna  \code{inc}, \code{addY} och \code{isPalindrome} är äkta. Notera att \code{y}-variablen initialiseras till \code{0} och kan sedan inte ändras eftersom den är deklarerad med nyckelordet \code{val}.
\end{itemize}

\QUESTEND


\WHAT{Applicera funktion på varje element i en samling. Funktion som argument.}

\QUESTBEGIN

\Task  \what~

\noindent Deklarera funktionen \code{öka} och variabeln \code{xs} enligt nedan i REPL:
\begin{REPL}
scala> def öka(x: Int) = x + 1
scala> val xs = Vector(3, 4, 5)
\end{REPL}
\noindent Para ihop nedan uttryck till vänster med det uttryck till höger som har samma värde. Om du undrar något, testa uttrycken och olika varianter av dem i REPL.

\begin{ConceptConnections}
  \code|for (i <- 1 to 3) yield öka(i)| & 1 & & A & \code|Vector(5, 6, 7)| \\ 
  \code|Vector(2, 3, 4).map(i => öka(i))| & 2 & & B & \code|Vector(4, 5, 6)| \\ 
  \code|xs.map(öka)| & 3 & & C & \code|Vector(2, 3, 4)| \\ 
  \code|xs.map(öka).map(öka)| & 4 & & D & \code|()| \\ 
  \code|xs.foreach(öka)| & 5 & & E & \code|xs| \\ 
\end{ConceptConnections}

\SOLUTION

\TaskSolved \what

\begin{ConceptConnections}
    \code|for (i <- 1 to 3) yield öka(i)| & 1 & ~~\Large$\leadsto$~~ &  D & \code|Vector(2, 3, 4)| \\ 
  \code|Vector(2, 3, 4).map(i => öka(i))| & 2 & ~~\Large$\leadsto$~~ &  C & \code|xs| \\ 
  \code|xs.map(öka)| & 3 & ~~\Large$\leadsto$~~ &  E & \code|Vector(4, 5, 6)| \\ 
  \code|xs.map(öka).map(öka)| & 4 & ~~\Large$\leadsto$~~ &  A & \code|Vector(5, 6, 7)| \\ 
  \code|xs.foreach(öka)| & 5 & ~~\Large$\leadsto$~~ &  B & \code|()| \\ 
\end{ConceptConnections}

\QUESTEND




\WHAT{Anonyma funktioner.}

\QUESTBEGIN

\Task  \what~  Vi har flera gånger sett syntaxen \code{i => i + 1}, till exempel i en loop \code{(1 to 10).map(i => i + 1)} där funktionen \code{i => i + 1} appliceras på alla heltal från 1 till och med 10 och resultatet blir en ny sekvenssamling.

Syntaxen \code{(i: Int) => i + 1} är en litteral för att skapa ett \emph{funktionsvärde} (kallas även \emph{anonym funktion} eller \emph{lambda-uttryck}). Syntaxen liknar den för funktionsdeklarationer, men nyckelordet \code{def} saknas i funktionshuvudet och i stället för likhetstecken används \code{=>} för att avskilja parameterlistan från funktionskroppen.
Om kompilatorn kan härleda typen ur sammanhanget kan kortformen \code{i => i + 1} användas.

Det finns ett \emph{ännu} kortare sätt att skriva en anonym funktion \emph{om} typen kan härledas \emph{och} den bara använder sin parameter \emph{en enda gång}; då går funktionslitteraler att skriva med s.k. \emph{platshållarsyntax} som använder understreck, till exempel \code{ _ + 1} och som automatiskt expanderas av kompilatorn till \code{ngtnamn => ngtnamn + 1} (namnet på parametern spelar ingen roll; kompilatorn väljer något eget, internt namn).

Para ihop uttryck till vänster med uttryck till höger som har samma värde:

\begin{ConceptConnections}
\input{generated/quiz-w03-lambda-taskrows-generated.tex}
\end{ConceptConnections}

\noindent
Funktionslitteraler kallas \textit{anonyma funktioner}, eftersom de inte har något namn, till skillnad från t.ex. \code{def öka(i: Int): Int = i + 1}, som ju heter \code{öka}. Ett annat vanligt namn är \textit{lambda-uttryck} efter det datalogiska matematikverktyget \href{https://sv.wikipedia.org/wiki/Lambdakalkyl}{lambdakalkyl}.

\SOLUTION

\TaskSolved \what

\begin{ConceptConnections}
    \code|(0 to 2).map(i => i + 1)           | & 1 & ~~\Large$\leadsto$~~ &  B & \code|(2 to 4).map(i => i - 1)| \\ 
  \code|(1 to 3).map(_ + 1)                | & 2 & ~~\Large$\leadsto$~~ &  D & \code|Vector(2, 3, 4)         | \\ 
  \code|(2 to 4).map(math.pow(2, _))       | & 3 & ~~\Large$\leadsto$~~ &  A & \code|Vector(4.0, 8.0, 16.0)  | \\ 
  \code|(3 to 5).map(math.pow(_, 2))       | & 4 & ~~\Large$\leadsto$~~ &  C & \code|Vector(9.0, 16.0, 25.0) | \\ 
  \code|(4 to 6).map(_.toDouble).map(_ / 2)| & 5 & ~~\Large$\leadsto$~~ &  E & \code|Vector(2.0, 2.5, 3.0)   | \\ 
\end{ConceptConnections}

\QUESTEND




\WHAT{Skapa din egen kontrollstruktur med hjälp av namnanrop.}\label{func:upprepa}

\QUESTBEGIN

\Task  \what~Namnanrop skrivs med en raket efter kolon före parametertypen och innebär att argumentet evalueras på plats varje gång.

\Subtask Använd namnanrop i kombination med en uppdelad parameterlista och skapa din egen kontrollstruktur enligt nedan.\footnote{Det är så loopen \code{upprepa} i Kojo är definierad.}
\begin{Code}
def upprepa(n: Int)(block: => Unit): Unit =
  var i = 0
  while i < n do 
    ???
  end while
\end{Code}

\Subtask
Testa din kontrollstruktur i REPL. Låt upprepa 100 gånger att ett slumptal mellan 1 och 6 dras och sedan skrivs ut. Prova även att använda färre klammerparenteser med hjälp av kolon.

\Subtask
Varför behövs namnanrop här?

\SOLUTION

\TaskSolved \what

\SubtaskSolved
\begin{Code}
def upprepa(n: Int)(block: => Unit): Unit =
  var i = 0
  while i < n do
    block
    i += 1
  end while
\end{Code}

\SubtaskSolved
\begin{Code}
upprepa(100):
  val tärningskast = (math.random() * 6 + 1).toInt
  print(s"\$tärningskast ")
\end{Code}

\SubtaskSolved Om parametern \code{block} inte vore deklarerad med namnanrop så hade argumentet evaluerats en gång innan anropet och sedan hade det blivit samma resultat vid varje iteration. Med namnanrop kan block innehålla kod som t.ex. uppdaterar en variabel som vi vill ska ske vid varje iteration. Namn-anrop liknar att koden för argumentet ''klistras in'' på varje plats i funktionskroppen där parameternamnet förekommer. 

\QUESTEND



\WHAT{Lär dig läsa en stack trace.}

\QUESTBEGIN

\Task  \what~  Skriv ett program i filen \texttt{fel.scala} som orsakar ett \emph{körtidsfel} och kör igång det i terminalen med \code{scala-cli run fel.scala}. Studera den stack trace som skrivs ut. Vad innehåller en \code{stack trace}? Diskutera med handledare hur du kan ha nytta av en stack trace när du felsöker.

\SOLUTION

\TaskSolved \what En stack trace innehåller följande information:
\begin{enumerate}
  \item ett felmeddelande
  \item namn på alla funktioner som anropats vid tiden för körtidsfelet, enligt alla aktiveringsposter som ligger på anropsstacken 
  \item aktuell namnrymnd för varje funktionen, alltså paket/singelobjekt
  \item namnet på kodfilen för varje funktion
  \item radnummer i varje funktion 
  \item den funktion som kommer först är den funktion där felet inträffade
  \item eventuellt kan felet inträffa i standardbibliotekets funktioner och då är din egen funktion tidigare i anropskedjan
\end{enumerate}

Exempel på en stack trace:
\begin{REPLnonum}
> cat fel.scala 
@main def run = 
  println("Hej Scala!" + Vector().head)
> scala-cli run fel.scala
Compiling project (Scala 3.3.0, JVM)
Compiled project (Scala 3.3.0, JVM)
Exception in thread "main" java.util.NoSuchElementException: empty.head
	at scala.collection.immutable.Vector.head(Vector.scala:279)
	at fel$package$.run(fel.scala:2)
	at run.main(fel.scala:1)
>
\end{REPLnonum}

\QUESTEND


\ExtraTasks %%%%%%%%%%%%%%%%%%%%%%%%%%%%%%%%%%%%%%%%%%%%%%%%%%%%%%%%%%



\WHAT{Funktion med flera parametrar.}

\QUESTBEGIN

\Task  \what~  

\Subtask Definiera i REPL två funktioner \code{sum} och \code{diff} med två heltalsparametrar som returnerar summan respektive differensen av argumenten:
\begin{Code}
def sum(x: Int, y: Int): Int = ???

def diff(x: Int, y: Int): Int = ???
\end{Code}
Vad har nedan uttryck för värden? Förklara vad som händer.

\Subtask \code{diff(0, 100)}

\Subtask \code{diff(100, sum(42, 43))}

\Subtask \code{sum(sum(42, 43), diff(100, sum(0, 0)))}

\Subtask \code{sum(diff(Byte.MaxValue, Byte.MinValue), 1)}

\SOLUTION

\TaskSolved \what

\SubtaskSolved
\begin{Code}
  def sum(x: Int, y: Int): Int = x + y
  
  def diff(x: Int, y: Int): Int = x - y
\end{Code}
  

\SubtaskSolved  Det blir \code{-100} efter som \code{0 - 100 == -100} 

\SubtaskSolved  Det blir \code{15} eftersom det nästlade anropet motsvarar \\\code{diff(100, 42 + 43) == (100 - 85)}

\SubtaskSolved  Det blir \code{185} eftersom det nästlade anropet motsvarar \\\code{sum(42 + 43, 100 - 0) == (85 + 100)}

\SubtaskSolved  Det blir \code{256} eftersom \code{Byte.MaxValue == 127} och \  code{Byte.MinValue == -128} och \code{sum(127 + 128, 1) == 256}

\QUESTEND



\WHAT{Medelvärde.}

\QUESTBEGIN

\Task  \what~ Skriv och testa en funktion \code{avg} som räknar ut medelvärdet mellan två heltal och returnerar en \code{Double}.

\SOLUTION

\TaskSolved \what

\begin{Code}
def avg(x: Int, y: Int): Double = (x + y) / 2.0
\end{Code}

\QUESTEND




\WHAT{Funktionsanrop med namngivna argument.}

\QUESTBEGIN

\Task  \what~
\begin{REPL}
scala> def skrivNamn(efternamn: String, förnamn: String) =
         println(s"Namn: $efternamn, $förnamn")
scala> skrivNamn(förnamn = "Stina", efternamn = "Triangelsson")
scala> skrivNamn(efternamn = "Oval", "Viktor")

\end{REPL}

\Subtask Vad skrivs ut efter rad 3 resp. rad 4 ovan?

\Subtask Nämn tre fördelar med namngivna argument.

\SOLUTION

\TaskSolved \what~

\SubtaskSolved
\begin{REPL}
Namn: Triangelsson, Stina
Namn: Oval, Viktor
\end{REPL}

\SubtaskSolved
\begin{itemize}
  \item Anroparen kan själv välja ordning.
  \item Koden blir lättare att begripa om parameternamnen är självbeskrivande.
  \item Hjälper till att förhindra buggar som beror på förväxlade parametrar.
\end{itemize}

\QUESTEND



\WHAT{Funktion som äkta värde.}

\QUESTBEGIN

\Task  \what~  Funktioner är \emph{äkta värden} i Scala%\footnote{I likhet med t.ex. Javascript, men till skillnad från t.ex. Java.}
. Det betyder att variabler kan ha funktioner som värden och funktionsvärden kan vara argument till funktioner som har funktionsparametrar. Funktioner som tar funktioner som argument kallas \emph{högre ordningens funktioner}.

En funktion som har en heltalsparameter och ett heltalsresultat är av funktionstypen \code{Int => Int} (uttalas \emph{int-till-int}) och värdet av funktionen utgör ett objekt som har en metod som heter \code{apply} med motsvarande funktionstyp.

\Subtask \label{subtask:funcval} Deklarera nedan funktioner och variabler i REPL. Para sedan ihop nedan uttryck till vänster med det uttryck till höger som skapar samma utskrift. Om du undrar något, testa uttrycken och olika varianter av dem i REPL.

\begin{REPL}
scala> def hälsa(): Unit = println("Hej!")
scala> def fleraAnrop(antal: Int, f: () => Unit): Unit =
         for _ <- 1 to antal do f()
scala> val f1 = () => hälsa()
scala> var f2 = (s: String) => println(s)
scala> val f3 = () => f2("Thunk")
\end{REPL}

\begin{ConceptConnections}
  \code| fleraAnrop(1, hälsa) | & 1 & & A & \code| f2("Hej!\nHej!")| \\ 
  \code| fleraAnrop(3, hälsa) | & 2 & & B & \code| fleraAnrop(3, f1)  | \\ 
  \code| fleraAnrop(2, f1)    | & 3 & & C & \code| f3()               | \\ 
  \code| fleraAnrop(1, f3)    | & 4 & & D & \code| f2("Hej!")       | \\ 
\end{ConceptConnections}


\Subtask Vilka typer har variablerna \code{f1}, \code{f2} och \code{f3}?

\Subtask Funkar detta? Varför? \code{f2 = f1}

\Subtask Funkar detta? Varför? \code{val f4 = fleraAnrop}

\Subtask Funkar detta? Varför? \code{val f4 = hälsa}

\Subtask Funkar detta? Varför? \code{val f4: () => Unit = hälsa}

\SOLUTION

\TaskSolved \what

\SubtaskSolved

\begin{ConceptConnections}
    \code| fleraAnrop(1, hälsa) | & 1 & ~~\Large$\leadsto$~~ &  D & \code| f2("Hej!")       | \\ 
  \code| fleraAnrop(3, hälsa) | & 2 & ~~\Large$\leadsto$~~ &  B & \code| fleraAnrop(3, f1)  | \\ 
  \code| fleraAnrop(2, f1)    | & 3 & ~~\Large$\leadsto$~~ &  A & \code| f2("Hej!\nHej!")| \\ 
  \code| fleraAnrop(1, f3)    | & 4 & ~~\Large$\leadsto$~~ &  C & \code| f3()               | \\ 
\end{ConceptConnections}

\SubtaskSolved \code{f1} och \code{f3} är av typen \code{() => Unit} och \code{f2} av typen \code{String => Unit}.

\SubtaskSolved  Nej. \code{f1} och \code{f2} är av två olika funktionstyper.

\SubtaskSolved  Ja, det går fint.

\SubtaskSolved  Nej. När funktionen inte har någon parameter behöver kompilatorn mer information för att vara säker på att det är ett funktionsvärde du vill ha.

\SubtaskSolved Ja! Nu med typinformationen på plats är kompilatorn säker på vad du vill göra.

\QUESTEND



\WHAT{Bortkastade resultatvärden och returtypen \code{Unit}.}

\QUESTBEGIN

\Task  \what~ Undersök nedan kod i REPL och förklara vad som händer.

\Subtask
\begin{REPL}
scala> def tom = println("")
scala> println(tom)
\end{REPL}

\Subtask
\begin{REPL}
scala> def bortkastad: Unit = 1 + 1
scala> println(bortkastad)
\end{REPL}

\Subtask
\begin{REPL}
scala> def bortkastad2 = { val x = 1 + 1 }
scala> println(bortkastad2)
\end{REPL}

\Subtask Varför är det bra att explicit ange \code{Unit} som returtyp för procedurer?

\SOLUTION

\TaskSolved \what

\SubtaskSolved Procedurer returnerar tomma värdet och \code{println} är en procedur. När tomma värdet skrivs ut visas \code{()}.

\SubtaskSolved Procedurer returnerar tomma värdet. Om du anger returtyp \code{Unit} explicit, har du bättre chans att kompilatorn kan ge varning då uträkningar kommer att kastas bort. En varning avbryter inte exekveringen, utan är ett sätt för kompilatorn att ge dig tips om saker som kan behöva fixas till i din kod.

\SubtaskSolved I Scala är variabeldeklaration, precis som en tilldelningssats, och inte ett uttryck och saknar värde.

\SubtaskSolved  Koden blir lättare att läsa och kompilatorn får bättre möjlighet att hjälpa till med varningar om resultatvärden riskerar att bli bortkastade.

\QUESTEND


\WHAT{Namnanrop.}

\QUESTBEGIN

\Task  \what~

Deklarera denna procedur i REPL:
\begin{Code}
def görDettaTvåGånger(b: => Unit): Unit = { b; b }
\end{Code}

Anropa \code{görDettaTvåGånger} med ett block som parameter. Blocket ska innehålla en utskriftssats. Förklara vad som händer.

\SOLUTION

\TaskSolved \what

Blocket är ett uttryck som har värdet \code{(): Unit}. Evalueringen av blocket sker där namnet \code{b} förekommer i procedurkroppen, vilket är två gånger.
\begin{REPL}
scala> görDettaTvåGånger { println("goddag") }
goddag
goddag
\end{REPL}

\QUESTEND




\clearpage

\AdvancedTasks %%%%%%%%%%%%%%%%%%%%%%%%%%%%%%%%%%%%%%%%%%%%%%%%%%%%%%%%%%%




\WHAT{Föränderlighet av parametrar.}

\QUESTBEGIN

\Task \what~Vad tror du om detta: Är en parameter förändringsbar i funktionskroppen ...

\Subtask ... i Scala?  (Ja/Nej)

\Subtask ... i Java?  (Ja/Nej)

\Subtask ... i Python?  (Ja/Nej)


\SOLUTION

\TaskSolved \what~

\Subtask Nej, i Scala är parametern oföränderlig och det blir kompileringsfel om man försöker tilldela den ett nytt värde i funktionskroppen.

\Subtask \Subtask Ja det går utmärkt i både Java och Python att ändra värdet på parametern i funktionskroppen med tilldelning, men koden riskerar att bli förvirrande.\\
\url{https://stackoverflow.com/questions/2970984}

\QUESTEND



\WHAT{Värdeanrop och namnanrop.}

\QUESTBEGIN

\Task  \what~Normalt sker i Scala (och i Java) s.k. \emph{värdeanrop} vid anrop av funktioner, vilket innebär att argumentuttrycket evalueras \emph{före} bindningen till parameternamnet sker.

Man kan också i Scala (men inte i Java) med syntaxen \code{=>} framför parametertypen deklarera att \emph{namnanrop} ska ske, vilket innebär att evalueringen av argumentuttrycket \emph{fördröjs} och sker \emph{varje gång} namnet används i metodkroppen.

Deklarera nedan funktioner i REPL.

\begin{Code}
def snark: Int = { print("snark "); Thread.sleep(1000); 42 }
def callByValue(x: Int):   Int = x + x
def callByName(x: => Int): Int = x + x
lazy val zzz = snark
\end{Code}

\noindent Förklara vad som händer när nedan uttryck evalueras.

\Subtask \code{snark + snark}

\Subtask \code{callByValue(snark)}

\Subtask \code{callByName(snark)}

\Subtask \code{callByName(zzz)}

\SOLUTION

\TaskSolved \what

\SubtaskSolved Vid varje anrop av \code{snark} sker en utskrift och en fördröjnig innan $42$ returneras. \\\code{42 + 42 == 84} vilket blir värdet av uttrycket.
\begin{REPL}
scala> snark + snark
snark snark val res1: Int = 84
\end{REPL}

\SubtaskSolved Uttrycket \code{snark} evalueras direkt vid anropet och parametern \code{x} binds till värdet $42$ och i funktionskroppen beräknas $42+42$. Utskriften sker bara en gång.
\begin{REPL}
callByValue(snark)
snark val res2: Int = 84
\end{REPL}

\SubtaskSolved Evalueringen av uttrycket \code{snark} fördröjs tills varje förekomst av parametern \code{x} i funktionskroppen. Utskriften sker två gånger.
\begin{REPL}
callByName(snark)
snark snark val res3: Int = 84
\end{REPL}

\SubtaskSolved Evalueringen av uttrycket \code{zzz} fördröjs tills varje förekomst av parametern \code{x} i funktionskroppen. Utskriften sker en gång eftersom \code{val}-variabler tilldelas sitt värde en gång för alla vid den fördröjda initialiseringen.
\begin{REPL}
callByName(zzz)
snark val res4: Int = 84
\end{REPL}

\QUESTEND



\WHAT{Skapa egen kontrollstruktur för iteration med loop-variabel.}

\QUESTBEGIN

\Task  \what~

\Subtask Fördelen med \code{upprepa} i uppgift \ref{func:upprepa} är att den är koncis och lättanvänd. Men den är inte lika lätt att använda om man behöver tillgång till en loopvariabel. Implementera därför nedan kontrollstruktur.

\begin{Code}
def repeat(n: Int)(p: Int => Unit): Unit = 
  var i = 0
  while i < n do
    ??? 
\end{Code}

\Subtask Använd \code{repeat} för att 100 gånger skriva ut loopvariabeln och ett slumpdecimaltal mellan 0 och 1.


\SOLUTION

\TaskSolved \what

\SubtaskSolved
\begin{Code}
def repeat(n: Int)(p: Int => Unit): Unit = 
  var i = 0
  while i < n do
    p(i)
    i += 1
  end while
end repeat
\end{Code}

\SubtaskSolved

\begin{Code}
repeat(100){ i =>
  print("i ")
  println(math.random())
}
\end{Code}
Du kan använda färre klammerparenteser med hjälp av kolon:
\begin{Code}
repeat(100): i =>
  print("i ")
  println(math.random())
\end{Code}

\QUESTEND






\WHAT{Uppdelad parameterlista och stegade funktioner.}

\QUESTBEGIN

\Task \what~Man kan dela upp parametrarna till en funktion i flera parameterlistor. Funktionen \code{add1} nedan har en parameterlista med två parametrar medan \code{add2} har två parameterlistor med en parameter vardera:
\begin{Code}
  def add1(a: Int, b: Int) = a + b
  def add2(a: Int)(b: Int) = a + b
\end{Code}

\Subtask  När man anropar funktionen \code{add2} ska argumenten skrivas inom två olika parentespar. Hur kan du använda \code{add2} för att räkna ut \code{1 + 1}?

\Subtask En fördel med uppdelade parameterlistor är att man kan skapa s.k. \emph{stegade funktioner}\footnote{Kallas även Curry-funktioner efter matematikern och logikern Haskell Brooks Curry.} där argumenten är partiellt applicerade. Prova det stegade funktionsvärdet \code{singLa} nedan. Vad skrivs ut på efter raderna 3 och 5?

\begin{REPL}
scala> def repeat(s: String)(n: Int): String = s * n
scala> val song = repeat("doremi ")(3)
scala> println(song)
scala> val singLa = repeat("la")
scala> println(singLa(7))
\end{REPL}

\SOLUTION

\TaskSolved \what

\SubtaskSolved
\begin{REPL}
scala> def add2(a: Int)(b: Int) = a + b
def add2(a: Int)(b: Int): Int

scala> add2(1)(1)
val res0: Int = 2
\end{REPL}

\SubtaskSolved
\begin{itemize}

\item Rad 3:
\begin{REPLnonum}
doremi doremi doremi 
\end{REPLnonum}

\item Rad 5:
\begin{REPLnonum}
lalalalalalala
\end{REPLnonum}

\end{itemize}


\QUESTEND




\WHAT{Rekursion.}

\QUESTBEGIN

\Task\Uberkurs  \what~  En rekursiv funktion anropar sig själv.

\Subtask Förklara vad som händer nedan.

\begin{REPL}
scala> def countdown(x: Int): Unit = 
         if x > 0 then {println(x); countdown(x - 1)}
scala> countdown(10)
scala> countdown(-1)
scala> def finalCountdown(x: Byte): Unit =
         {println(x); Thread.sleep(100); finalCountdown((x-1).toByte); 1 / x}
scala> finalCountdown(Byte.MaxValue)
\end{REPL}

\Subtask Vad händer om du gör satsen som riskerar division med noll \emph{före} det rekursiva anropet i funktionen \code{finalCountdown} ovan?

\Subtask Förklara vad som händer nedan. Varför tar sista raden längre tid än näst sista raden?
\begin{REPL}
scala> def signum(a: Int): Int = if a >= 0 then 1 else -1
scala> def add(x: Int, y: Int): Int =
         if y == 0 then x else add(x + 1, y - signum(y))
scala> add(100, 100)
scala> add(Int.MaxValue, 0)
scala> add(0, Int.MaxValue)
\end{REPL}

\SOLUTION

\TaskSolved \what

\SubtaskSolved
\code{countdown} skriver ut x och gör ett rekursivt anrop med \code{x - 1} som argument, men bara om basvillkoret \code{x > 0} är uppfyllt. Resultatet blir en ändlig  repetition.
\code{finalCountdown} anropar sig själv rekursivt men saknar ett basvillkor som kan avbryta rekursionen, vilket genererar en oändlig repetition. Vid -128 blir det \emph{overflow} eftersom bitarna inte räcker till för större negativa tal och räkningen börjar om på 127. (Om minskar fördröjningen till \code{Thread.sleep(1)} blir det ganska snabbt \emph{stack overflow})

\SubtaskSolved
Eftersom vi hade \code{1/x} \emph{efter} det rekursiva anropet i föregående deluppgift, så kom vi aldrig till denna (potentiellt ödesdigra) beräkning, utan lade bara aktiveringsposter på hög på stacken vid varje anrop. Om vi placerar \code{1/x} \emph{före} det rekursiva anropet, så når vi detta uttryck direkt och det kastas ett undantag p.g.a. division med noll.

\SubtaskSolved
Den sista raden leder till många fler rekursiva anrop, så som basvillkoret och det rekursiva anropet är konstruerade. Lägg gärna in en \code{println}-sats före det rekursiva anropet och undersök i detalj vad som sker.

\QUESTEND



\WHAT{Undersök svansrekursion genom att kasta undantag.}

\QUESTBEGIN

\Task\Uberkurs  \what~  Förklara vad som händer. Kan du hitta bevis för att kompilatorn kan optimera rekursionen till en vanlig loop?

\begin{REPL}
scala> def explode = throw Exception("BANG!!!")
scala> explode
scala> def countdown(n: Int): Unit =
         if n == 0 then explode else countdown(n-1)
scala> countdown(10)
scala> countdown(10000)
scala> def countdown2(n: Int): Unit =
         if n == 0 then explode else {countdown2(n-1); print("no tailrec")}
scala> countdown2(10)
scala> countdown2(10000)
\end{REPL}

\SOLUTION

\TaskSolved \what~\code{countdown} är svansrekursiv eftersom det rekursiva anropet står \emph{sist} och kan då optimeras till en \code{while}-loop av kompilatorn. Det går fint att köra ända till det exploderar, även med 10000 anrop, och i felmeddelandet finns det endast ett anrop till \code{countdown}.

\code{countdown2} är inte svansrekursiv eftersom den har ett uttryck \code{efter} det rekursiva anropet. I felutskriften syns alla rekursiva anrop till \code{countdown2} innan basvillkoret inträffade. Vid \code{countdown2(10000)} uppfylls inte basvillkoret innan det blir \code{StackOverflowError}.

\QUESTEND



\WHAT{\code{@tailrec}-annotering.}

\QUESTBEGIN

\Task\Uberkurs  \what~  Du kan be kompilatorn att ge felmeddelande om den inte kan optimera koden till en motsvarande while-loop. Detta kan användas i de fall man vill vara helt säker på att kompilatorn kan optimera koden och det inte kan finnas risk för en överfull stack \Eng{stack overflow} på grund av för djup anropsnästling.

Prova nedan rader i REPL och förklara vad som händer.
\begin{REPL}
scala> def countNoTailrec(n: Long): Unit =
         if n <= 0L then println("Klar! " + n) else {countNoTailrec(n-1L); ()}
scala> countNoTailrec(1000L)
scala> countNoTailrec(100000L)
scala> import scala.annotation.tailrec
scala> @tailrec def countNoTailrec(n: Long): Unit =
         if n <= 0L then println("Klar! " + n) else {countNoTailrec(n-1L); ()}
scala> @tailrec def countTailrec(n: Long): Unit =
         if n <= 0L then println("Klar! " + n) else countTailrec(n-1L)
scala> countTailrec(1000L)
scala> countTailrec(100000L)
scala> countTailrec(Int.MaxValue.toLong * 2L)
\end{REPL}

\SOLUTION

\TaskSolved \what~Första gången \code{countNoTailrec(100000L)} anropas blir det \code{StackOverflowError}. Med annoteringen \code{@tailrec} får vi ett kompileringsfel eftersom kompilatorn inte kan optimera en icke svansrekursiv funktion. Om funktionen skrivs om kan kompilatorn optimera funktionen så att rekursionen byts ut mot en \code{while}-loop och vi kan köra så länge vi orkar utan att stacken flödar över. Och himla snabbt går det!!

\QUESTEND

%!TEX encoding = UTF-8 Unicode
%!TEX root = ../compendium2.tex

\Exercise{\ExeWeekFOUR}\label{exe:W04}
\begin{Goals}
%!TEX encoding = UTF-8 Unicode
%!TEX root = ../compendium2.tex

\item Kunna skapa och använda objekt som moduler.
\item Känna till att funktioner är objekt med en \code{apply}-metod.
\item Förstå begreppen synlighet, \code{private}, \code{import}, namnrymd och skuggning.
\item \TODO{FLER MÅL OM OBJEKT HÄR}

%\item Känna till svansrekursion och att svansrekursiva funktioner kan optimeras till loopar.

\end{Goals}

\begin{Preparations}
\item \StudyTheory{04}
\end{Preparations}

\BasicTasks %%%%%%%%%%%%%%%%

\TODO{ÖVNINGAR OM OBJEKT}


%!TEX encoding = UTF-8 Unicode
%!TEX root = ../exercises.tex

\ifPreSolution

\Exercise{\ExeWeekFIVE}\label{exe:W05}

\begin{Goals}
%!TEX encoding = UTF-8 Unicode

%!TEX root = ../compendium2.tex

\item Kunna deklarera klasser med klassparametrar.
\item Kunna skapa objekt med \code{new} och konstruktorargument.
\item Förstå innebörden av referensvariabler och värdet \code{null}.
\item Förstå innebörden av begreppen instans och referenslikhet.
\item Kunna använda nyckelordet \code{private} för att styra synlighet i primärkonstruktor.
\item Förstå i vilka sammanhang man kan ha nytta av en privat konstruktor.
\item Kunna implementera en klass utifrån en specifikation.
\item Förstå skillnaden mellan referenslikhet och strukturlikhet.
\item Känna till hur case-klasser hanterar likhet.
\item Förstå nyttan med att möjliggöra framtida förändring av attributrepresentation.
\item Känna till begreppen getters och setters.
\item Känna till accessregler för kompanjonsobjekt.
\item Känna till skillnaden mellan \code{==} och \code{eq}, samt \code{!=} versus \code{ne}.

\end{Goals}

\begin{Preparations}
\item \StudyTheory{05}
\end{Preparations}

\else


\ExerciseSolution{\ExeWeekFIVE}


\fi


\BasicTasks %%%%%%%%%%%


\WHAT{Para ihop begrepp med beskrivning.}

\QUESTBEGIN

\Task \what

\vspace{1em}\noindent Koppla varje begrepp med den (förenklade) beskrivning som passar bäst:

\begin{ConceptConnections}
  klass & 1 & & A & ett värde som ej refererar till någon instans \\ 
  instans & 2 & & B & nyckelord vid direkt instansiering av klass \\ 
  konstruktor & 3 & & C & ser privata medlemmar i klass med samma namn \\ 
  klassparameter & 4 & & D & binds till argument som ges vid konstruktion \\ 
  referenslikhet & 5 & & E & indirekt åtkomst av attributvärde \\ 
  innehållslikhet & 6 & & F & slipper skriva new; automatisk innehållslikhet \\ 
  case-klass & 7 & & G & instanser anses lika om de har samma tillstånd \\ 
  getter & 8 & & H & indirekt tilldelning av attributvärde \\ 
  setter & 9 & & I & hjälpfunktion för indirekt konstruktonsanrop \\ 
  kompanjonsobjekt & 10 & & J & upplaga av ett objekt med eget tillståndsminne \\ 
  fabriksmetod & 11 & & K & skapar instans, allokerar plats för tillståndsminne \\ 
  \code|null| & 12 & & L & en mall för att skapa flera instanser av samma typ \\ 
  \code|new| & 13 & & M & instanser anses olika även om tillstånden är lika \\ 
\end{ConceptConnections}

\SOLUTION

\TaskSolved \what

\begin{ConceptConnections}
  klass & 1 & ~~\Large$\leadsto$~~ &  I & en mall för att skapa flera instanser av samma typ \\ 
  instans & 2 & ~~\Large$\leadsto$~~ &  F & upplaga av ett objekt med eget tillståndsminne \\ 
  konstruktor & 3 & ~~\Large$\leadsto$~~ &  E & skapar instans, allokerar plats för tillståndsminne \\ 
  klassparameter & 4 & ~~\Large$\leadsto$~~ &  M & binds till argument som ges vid konstruktion \\ 
  referenslikhet & 5 & ~~\Large$\leadsto$~~ &  L & instanser anses olika även om tillstånden är lika \\ 
  innehållslikhet & 6 & ~~\Large$\leadsto$~~ &  J & instanser anses lika om de har samma tillstånd \\ 
  case-klass & 7 & ~~\Large$\leadsto$~~ &  D & slipper skriva new; automatisk innehållslikhet \\ 
  getter & 8 & ~~\Large$\leadsto$~~ &  A & indirekt åtkomst av attributvärde \\ 
  setter & 9 & ~~\Large$\leadsto$~~ &  B & indirekt tilldelning av attributvärde \\ 
  kompanjonsobjekt & 10 & ~~\Large$\leadsto$~~ &  H & ser privata medlemmar i klass med samma namn \\ 
  fabriksmetod & 11 & ~~\Large$\leadsto$~~ &  G & hjälpfunktion för indirekt konstruktonsanrop \\ 
  \code|null| & 12 & ~~\Large$\leadsto$~~ &  K & ett värde som ej refererar till någon instans \\ 
  \code|new| & 13 & ~~\Large$\leadsto$~~ &  C & nyckelord vid direkt instansiering av klass \\ 
\end{ConceptConnections}

\QUESTEND


\WHAT{Klass och instans.}

\QUESTBEGIN

\Task \what~Du har i övning \texttt{\ExeWeekFOUR}~sett hur singelobjekt i en egen namnrymd  kan samla funktioner (metoder) och ha tillstånd (attribut). Men singelobjekt finns bara i en upplaga.

Vill du kunna skapa många objekt av samma typ behöver du en \emph{klass}. En objektupplaga som skapats ur en klass kallas en \emph{instans} av klassen. Varje instans har sitt eget tillstånd.

Deklarera singelobjektet och klassen nedan i REPL.

\begin{Code}
object Singelpunkt { var x = 1; var y = 2 }
class  Punkt       { var x = 3; var y = 2 }
\end{Code}

\Subtask  Antag att uttrycken till vänster evalueras uppifrån och ned. Vilket resultat till höger hör ihop med respektive uttryck? Prova i REPL om du är osäker.\footnote{Strängen efter \code{@}-tecknet är en hexadecimal representation av det heltal som tillordnas varje objekt för att systemet ska kunna särskilja olika instanser. \url{https://stackoverflow.com/questions/4712139}}


\begin{ConceptConnections}
  \code|Singelpunkt.x               | & 1 & & A & \code|2| \\ 
  \code|Punkt.x                     | & 2 & & B & \verb|p2: Punkt = Punkt@51ab04bd| \\ 
  \code|val p  = new Singelpunkt    | & 3 & & C & \verb|p1: Punkt = Punkt@27a1a53c| \\ 
  \code|val p1 = new Punkt          | & 4 & & D & \verb|error: not found: type| \\ 
  \code|val p2 = new Punkt          | & 5 & & E & \code|java.lang.NullPointerException| \\ 
  \code|{ p1.x = 1; p2.x }          | & 6 & & F & \code|1| \\ 
  \code|(new Punkt).y               | & 7 & & G & \code|3| \\ 
  \code|{ val p: Punkt = null; p.x }| & 8 & & H & \code|error: not found: value| \\ 
\end{ConceptConnections}

\Subtask Vid tre tillfällen blir det fel. Varför? Är det kompileringsfel eller exekveringsfel?

\SOLUTION

\TaskSolved \what

\SubtaskSolved

\begin{ConceptConnections}
  \code|Singelpunkt.x               | & 1 & ~~\Large$\leadsto$~~ &  E & \code|1| \\ 
  \code|Punkt.x                     | & 2 & ~~\Large$\leadsto$~~ &  A & \code|error: not found: value| \\ 
  \code|val p  = new Singelpunkt    | & 3 & ~~\Large$\leadsto$~~ &  F & \verb|error: not found: type| \\ 
  \code|val p1 = new Punkt          | & 4 & ~~\Large$\leadsto$~~ &  D & \code|p1: Punkt = Punkt@27a1a53c| \\ 
  \code|val p2 = new Punkt          | & 5 & ~~\Large$\leadsto$~~ &  C & \code|p2: Punkt = Punkt@51ab04bd| \\ 
  \code|{ p1.x = 1; p2.x }          | & 6 & ~~\Large$\leadsto$~~ &  G & \code|3| \\ 
  \code|(new Punkt).y               | & 7 & ~~\Large$\leadsto$~~ &  B & \code|2| \\ 
  \code|{ val p: Punkt = null; p.x }| & 8 & ~~\Large$\leadsto$~~ &  H & \code|java.lang.NullPointerException| \\ 
\end{ConceptConnections}

\SubtaskSolved

\noindent\begin{tabular}{l l p{5cm}}

~\\ \emph{fel} & \emph{typ} & \emph{förklaring} \\\hline

\code|error: not found: value|
& kompileringsfel & det finns ingen instans med namnet \code|Punkt|\\

\verb|error: not found: type|
& kompileringsfel & det finns ingen klass som heter \code|Singelpunkt|\\

\code|NullPointerException|
& körtidsfel & det går inte att referera attribut i en instans som inte finns\\

\end{tabular}

\QUESTEND



\WHAT{Klassparametrar.}

\QUESTBEGIN

\Task \what~Klassen punkt i föregående uppgift är inte så smidig att använda eftersom man först \emph{efter} instansiering kan ge attributen \code{x} och \code{y} de koordinatvärden man önskar och detta måste ske med explicita tilldelningssatser.

Detta problem kan du lösa med \emph{klassparametrar} som låter dig initialisera attributen med konstruktionsargument och på så sätt ange ett initialtillstånd direkt i samband med instansiering.

Deklarera klassen nedan i REPL.

\begin{Code}
class Point(var x: Int, var y: Int)
\end{Code}


\Subtask  Antag att uttrycken till vänster evalueras uppifrån och ned. Vilket resultat till höger hör ihop med respektive uttryck? Prova i REPL om du är osäker.

\begin{ConceptConnections}
  \code|val p1 = Point(1, 2)        | & 1 & & A & \code|1| \\ 
  \code|val p2 = new Point          | & 2 & & B & \verb|p2: Point = Point@218cf600| \\ 
  \code|val p1 = new Point(1, 2)    | & 3 & & C & \code|error: not found: value| \\ 
  \code|val p2 = new Point(3, 4)    | & 4 & & D & \code|error: too many arguments| \\ 
  \code|p2.x - p1.x                 | & 5 & & E & \code|error: not enough arguments| \\ 
  \code|(new Point(0, 1)).y         | & 6 & & F & \code|2| \\ 
  \code|new Point(0, 1, 2)          | & 7 & & G & \verb|p1: Point = Point@30ef773e| \\ 
\end{ConceptConnections}

\Subtask Vid tre tillfällen blir det fel. Varför? Är det kompileringsfel eller exekveringsfel?

\SOLUTION

\TaskSolved \what

\SubtaskSolved

\begin{ConceptConnections}
  \code|val p1 = Point(1, 2)        | & 1 & ~~\Large$\leadsto$~~ &  C & \verb|p1: Point = Point@30ef773e| \\
  \code|val p2 = new Point          | & 2 & ~~\Large$\leadsto$~~ &  B & \verb|missing argument for parameter| \\
  \code|val p2 = new Point(3, 4)    | & 3 & ~~\Large$\leadsto$~~ &  D & \verb|p2: Point = Point@218cf600| \\
  \code|p2.x - p1.x                 | & 4 & ~~\Large$\leadsto$~~ &  F & \code|2| \\
  \code|(new Point(0, 1)).y         | & 5 & ~~\Large$\leadsto$~~ &  A & \code|1| \\
  \code|new Point(0, 1, 2)          | & 6 & ~~\Large$\leadsto$~~ &  E & \verb|too many arguments for constructor|

\end{ConceptConnections}

\SubtaskSolved

\noindent\begin{tabular}{l l p{5cm}}

  ~\\ \emph{fel} & \emph{typ} & \emph{förklaring} \\\hline

  \code|error: not found: value|
  & kompileringsfel & det finns ingen instans med namnet \code|Point|\\

  \verb|error: not enough arguments|
  & kompileringsfel  & du måste ge argument vid konstruktion av klassen \code|Point| \\

  \code|error: too many arguments|
  & kompileringsfel & antalet argument stämmer ej överens med antalet klassparametrar\\

\end{tabular}

\QUESTEND



\WHAT{Oföränderlig klass med defaultargument.}

\QUESTBEGIN

\Task \what~Det du tidigare lärt dig om parametrar och argument är tillämpligt även på klassparametrar, t.ex. defaultargument och namngivna argument. Man kan dessutom framför klassparametrar använda synlighetsmodifieraren \code{private} och nyckelorden \code{var} och \code{val}.

Om inget anges framför en klassparameter är det \code{private val} som gäller\footnote{För case-klasser, som vi ska se snart, är det i stället \code{val} som gäller (alltså inte \code{private}).}.

Deklarera nedan klass i REPL.

\begin{Code}
class Point3D(val x: Int = 0, val y: Int = 0, z: Int = 0)
\end{Code}

\Subtask Antag att uttrycken till vänster evalueras uppifrån och ned. Vilket resultat till höger hör ihop med respektive uttryck? Prova i REPL om du är osäker.

\begin{ConceptConnections}
  \code|val p1 = Point3D()          | & 1 & & A & \code|false| \\ 
  \code|val p2 = Point3D(y = 1)     | & 2 & & B & \code|Reassignment to val| \\ 
  \code|Point3D(z = 2).z            | & 3 & & C & \verb|p1: Point3D = Point3D@2eb37eee| \\ 
  \code|p2.y = 0                    | & 4 & & D & \code|true| \\ 
  \code|p2.y == 0                   | & 5 & & E & \code|value cannot be accessed| \\ 
  \code|p1.x == Point3D().x         | & 6 & & F & \verb|p2: Point3D = Point3D@65a9e8d7| \\ 
\end{ConceptConnections}

\Subtask Vad är problemet med ovan klass om man vill använda den för att representera punkter i 3 dimensioner?

\SOLUTION

\TaskSolved \what~

\SubtaskSolved

\begin{ConceptConnections}
  \code|val p1 = new Point3D        | & 1 & ~~\Large$\leadsto$~~ &  A & \verb|p1: Point3D = Point3D@2eb37eee| \\ 
  \code|val p2 = new Point3D(y = 1) | & 2 & ~~\Large$\leadsto$~~ &  B & \verb|p2: Point3D = Point3D@65a9e8d7| \\ 
  \code|(new Point3D(z = 2)).z      | & 3 & ~~\Large$\leadsto$~~ &  C & \code|error: not found: value| \\ 
  \code|p2.y = 0                    | & 4 & ~~\Large$\leadsto$~~ &  D & \code|error: reassignment to val| \\ 
  \code|p2.y == 0                   | & 5 & ~~\Large$\leadsto$~~ &  F & \code|false| \\ 
  \code|p1.x == (new Point3D).x     | & 6 & ~~\Large$\leadsto$~~ &  E & \code|true| \\ 
\end{ConceptConnections}

\SubtaskSolved Problemet är att så som klassen \code{Point3D} är deklarerad går det inte att avläsa \code{z}-koordinaten efter att en instans konstruerats. Det vore bättre om även \code{z}-attributet är \code{val}.

\QUESTEND



\WHAT{Case-klass.}

\QUESTBEGIN

\Task \what~\TODO

\begin{Code}
case class Pt(x: Int = 0, y: Int = 0) {
  def moved(dx: Int = 0, dy: Int = 0): Pt = Pt(x + dx, y + dy)
}

class MutablePt(private var p: (Int, Int) = (0, 0)) {
  def x: Int = p._1
  def y: Int = p._2
  def move(dx: Int = 0, dy: Int = 0) = { p = (x + dx, y+ dy); this }
}
\end{Code}

\Subtask Vilken returtyp kommer kompilatorn härleda för funktionen MutablePt.move?

\Subtask Implementera en fabriksmetod \code{apply} i ett kompanjonsobjekt till klassen \code{MutablePt} som gör att du inte behöver skriva \code{new} när du skapar instanser.

\SOLUTION

\TaskSolved \what~\TODO


\QUESTEND


\WHAT{Skapa en punktklass att använda på veckans laboration.}

\QUESTBEGIN

\Task \what~
Du ska som förberedelse till laborationen skapa den oföränderliga case-klassen \code{Point} som ska beskriva en koordinat i ett kartesiskt koordinatsystem\footnote{\url{https://sv.wikipedia.org/wiki/Kartesiskt_koordinatsystem}}. Skapa kod med hjälp av en editor, t.ex. \code{atom}, i filen  \code{Point.scala} enligt följande riktlinjer:
\begin{enumerate}[noitemsep]
\item \code{Point} ska ligga i paketet \code{graphics}.

\item \code{Point} ska ha följande två publika, oföränderliga klassparametrar:
\begin{itemize}[nolistsep, noitemsep]
\item \code{x: Double} för x-koordinaten.
\item \code{y: Double} för y-koordinaten.
\end{itemize}

\item \code{Point} ska ha följande publika medlemmar (två oföränderliga attribut och två metoder):
\begin{itemize}[nolistsep, noitemsep]
\item \code{val r: Double} ska ge motsvarande polära kordinatens%
\footnote{\url{https://sv.wikipedia.org/wiki/Pol\%C3\%A4ra\_koordinater}}
 avstånd till origo.
\item \code{val theta: Double} ska ge polära kordinatens vinkel i radianer.
\item \code{def negY: Point} ska ge en ny punkt med y-koordinaten negerad.
\item \code{def +(p: Point): Point} ska ge en ny punkt vars koordinat är summan av x- respektive y-kordinaterna för denna instans och punkten \code{p}.
\end{itemize}

\item \code{Point} ska ha ett kompanjonsobjekt med en metod som konstruerar en punkt från polära koortdinater. Metoden ska ha detta huvud: \\\code{def polar(r: Double, theta: Double): Point}

\end{enumerate}

\noindent Tips vid implementation och senare användning:
\begin{itemize}
\item Du har nytta av metoderna \code{math.hypot(x, y)} och \code{math.atan2(y, x)} vid omvandling till polära koordinater.

\item Du har nytta av metoderna \code{math.cos(x)} och \code{math.sin(y)} vid omvandling från polära koordinater.

\item Attributet \code{negY} kommer att underlätta för dig när du i metoden \code{draw} i klassen \code{Turtle} ska omvandla en punkt till fönsterkoordinater där y-axeln är omvänd jämfört med kartesiska koordinater.

\item Notera att klassens attribut är av typen \code{Double} och inte \code{Int}, trots att vi senare ska använda punkten för att beskriva en diskret pixelposition. Anledningen till detta är att det kan uppstå avrundningsfel vid numeriska beräkningar. Detta blir särskilt märkbart vid upprepad räkning med små värden, t.ex. när man ritar en approximerad cirkel med många linjesegment.
\end{itemize}

\SOLUTION

\TaskSolved \what~\TODO

\QUESTEND



\AdvancedTasks %%%%%%%%%%%%%%%%%%%%%%%%%%%%%%%%%%%%%%%%%%%%%%%%%%%%%%%%%%%%%%%%%

\WHAT{Ändra attributrepresentation. Privat konstruktor}

\QUESTBEGIN

\Task \what~Kim Kodkunnig skapade för länge sedan denna punktklass som används på många ställen i befintlig kod:

\begin{Code}
class Point private (val x: Int, val y: Int)
object Point {
  def apply(x: Int = 0, y: Int = 0): Point = new Point(x, y)
  def origo = apply()
}
\end{Code}

\Subtask Vad händer om du försöker instansiera Kim Kodkunnigs klass i din egen kod direkt med nyckelordet \code{new}?

\Subtask Varför använder Kim Kodkunnig ett kompanjonsobjekt med en fabriksmetod? Vilka accessregler gäller mellan ett kompanjonsobjekt och klassen med samma namn?

\Subtask Hjälp Kim Kodkunnig att ändra attributrepresentationen så att det oföränderliga tillståndet utgörs av en 2-tupel \code{val p: (Int, Int)} i stället. Befintlig kod ska inte behöva ändras och klassen \code{Point} ska bete sig från ''utsidan'' precis som innan.

\SOLUTION

\TaskSolved \what~

\SubtaskSolved Det blir kompileringsfel eftersom konstruktorn är privat.
\begin{REPL}
scala> :paste

class Point private (val x: Int, val y: Int)
object Point {
  def apply(x: Int = 0, y: Int = 0): Point = new Point(x, y)
  def origo = apply()
}

scala> new Point(0,0)
<console>:14: error: constructor Point in class Point cannot be accessed
\end{REPL}

\SubtaskSolved
\begin{itemize}
  \item Genom att ha en privat konstruktor och bara göra indirekt instansiering via fakriksmetod är det möjligt att ändra attributrepresentation i framtiden utan att befintlig kod behöver ändras.

  \item Med en \code{apply}-metod i kompansjonsobjektet kan man instansiera genom att skriva \code{Point(1, 2)} utan new.

  \item Accessreglerna för kompanjonsobjekt är sådana att kompanjoner ser varandras privata delar.
\end{itemize}

\SubtaskSolved

\begin{Code}
class Point private (private val p: (Int, Int)) {
  def x: Int = p._1
  def y: Int = p._2
}
object Point {
  def apply(x: Int = 0, y: Int = 0): Point = new Point(x, y)
  def origo = apply()
}
\end{Code}

\QUESTEND


\subsection{\TODO värdera nedan uppgifter}


\WHAT{Instansiering med \code{new} och värdet \code{null}.}

\QUESTBEGIN

\Task  \what~  Man skapar instanser av klasser med \code{new}. Då anropas konstruktorn och plats reserveras i datorns minne för objektet. Variabler av referenstyp som inte refererar till något objekt har värdet \code{null}.

\Subtask Vad händer nedan? Vilka rader ger felmeddelande och i så fall hur lyder felmeddelandet?

\begin{REPL}
scala> class Gurka(val vikt: Int)
scala> var g: Gurka = null
scala> g.vikt
scala> g = new Gurka(42)
scala> g.vikt
scala> g = null
scala> g.vikt
\end{REPL}

\Subtask\Pen Rita minnessituationen efter raderna 2, 4, 6.

\SOLUTION


\TaskSolved \what


\SubtaskSolved  Rad 3 och 7 ger båda felmeddelandet "java.lang.NullPointerException". Detta eftersom \code{g} i båda fallen inte innehåller en referens till en \code{Gurka} utan pekar på inget -- "null".

\SubtaskSolved  \includegraphics[scale=0.6]{../img/w06-solutions/1b}


\QUESTEND




%%<AUTOEXTRACTED by mergesolu>%%      % uppgift 1




\WHAT{Klasser och instanser.}

\QUESTBEGIN

\Task  \what~

\Subtask Vad händer nedan?
\begin{REPL}
scala> :pa
class Arm(val ärTillVänster: Boolean)
class Ben(val ärTillVänster: Boolean)
class Huvud(val harHår: Boolean)
class Rymdvarelse {
  var arm1 = new Arm(true)
  var arm2 = new Arm(false)
  var ben1 = new Ben(true)
  var ben2 = new Ben(false)
  var huvud1 = new Huvud(false)
  var huvud2 = new Huvud(true)
  def ärSkallig = !huvud1.harHår && !huvud2.harHår
}
scala> val alien = new Rymdvarelse
scala> alien.ärSkallig
scala> val predator = new Rymdvarelse
scala> predator.ärSkallig
scala> predator.huvud2 = alien.huvud1
scala> predator.ärSkallig
\end{REPL}

\Subtask\Pen Rita minnessituationen efter rad 18.

\Subtask\Pen Vad händer så småningom med det ursprungliga huvud2-objektet i predator efter tilldelningen på rad 18? Går det att referera till detta objekt på något sätt?


\SOLUTION


\TaskSolved \what


\SubtaskSolved  Vi skapar två rymdvarelser, \code{alien} och \code{predator}, med två ben, två armar samt två huvuden (där det ena är skalligt och det andra har hår) vardera. Efter det är varken \code{alien} eller \code{predator} skallig eftersom båda har ett huvud med hår. Sen låter man referensen till \code{predator}s huvud med hår referera till aliens huvud utan hår. Nu är predator helt skallig.

\SubtaskSolved   \includegraphics[scale=0.7]{../img/w06-solutions/2b}

\SubtaskSolved  Eftersom det inte längre finns någon referens som pekar på det objektet kommer Garbage Collector ta hand om det och kommer förr eller senare skrivas över av något annat som behöver sparas. Nej, det går inte att komma åt.

% uppgift 3

\QUESTEND




%%<AUTOEXTRACTED by mergesolu>%%      % uppgift 2




\WHAT{Synlighet i primärkonstruktorer.}

\QUESTBEGIN

\Task  \what~  Undersök nedan vad nyckelorden \code{val} och \code{private} får för konsekvenser. Förklara vad som händer. Vilka rader ger vilka felmeddelanden?

\begin{REPL}
scala> class Gurka1(vikt: Int)
scala> new Gurka1(42).vikt
scala> class Gurka2(val vikt: Int)
scala> new Gurka2(42).vikt
scala> class Gurka3(private val vikt: Int)
scala> new Gurka3(42).vikt
scala> class Gurka4(private val vikt: Int, kompis: Gurka4){
         def kompisVikt = kompis.vikt
       }
scala> val ingenGurka: Gurka4 = null
scala> new Gurka4(42, ingenGurka).kompisVikt
scala> new Gurka4(42, new Gurka4(84, null)).kompisVikt
scala> class Gurka5(private[this] val vikt: Int, kompis: Gurka5){
         def kompisVikt = kompis.vikt
       }
scala> class Gurka6 private (vikt: Int)
scala> new Gurka6(42)
scala> :pa
class Gurka7 private (var vikt: Int)
object Gurka7 {
  def apply(vikt: Int) = {
    require(vikt >= 0, s"negativ vikt: $vikt")
    new Gurka7(vikt)
  }
}
scala> new Gurka7(-42)
scala> Gurka7(-42)
scala> val g = Gurka7(42)
scala> g.vikt
scala> g.vikt = -1
scala> g.vikt
\end{REPL}


\SOLUTION


\TaskSolved \what
 Rad 2:
\begin{REPL}
	error: value vikt is not a member of Gurka1
\end{REPL}
Detta eftersom om man varken väljer att skriva \code{val} eller \code{var} skapar inte scala någon getter eller setter (metoder för att läsa/ändra en variabel) och därför ser det ut som att vikt inte finns för kompilatorn.

Rad 4: Denna rad skapar inte en error eftersom om man skriver \code{val} innan variabeln skapas en getter automatiskt och man kan därför komma åt \code{vikt}.

Rad 6:
\begin{REPL}
	error: value vikt in class Gurka3 cannot be accessed in Gurka3
\end{REPL}
I detta fallet skapas en \code{getter} men eftersom accessnivån sätts till \code{private} vet kompilatorn att man inte får komma åt variabeln utifrån.

Rad 11:
\begin{REPL}
	java.lang.NullPointerException
\end{REPL}
Detta eftersom \code{kompis} är \code{ingenGurka} som inte pekar på något objekt och när man då försöker komma åt ett attribut från den kommer det inte funka.

Rad 12: Kommer inte generera en error eftersom när man kallar \code{kompisVikt} (som är \code{public}) försöker den komma åt \code{Gurka4(84, null).vikt}. \code{vikt} är \code{private val} vilket innebär att det har en getter och eftersom huvudobjektet också är av typen \code{Gurka4} är accessnivån tillräckligt hög.

Rad 13:
\begin{REPL}
	error: value vikt is not a member of Gurka5
\end{REPL}
När man sätter ett attribut till \code{private[this]} tillåts inte ens objekt av samma typ att komma åt variabeln och därför får man en error som säger att den inte finns.

Rad 17:
\begin{REPL}
	error: constructor Gurka6 in class Gurka6 cannot be accessed in object
\end{REPL}
Eftersom man satt klassparametrarna till \code{private} kan man inte komma åt konstruktorn och därför får man en error.

Rad 26:
\begin{REPL}
	error: constructor Gurka7 in class Gurka7 cannot be accessed in object
\end{REPL}
Samma anledning som på rad 17.

Rad 27:
\begin{REPL}
	java.lang.IllegalArgumentException: requirement failed: negativ vikt: -42
\end{REPL}
Kompanjonsobjektet har en requirement på att \code{vikt >= 0} vilket innebär att om det inte stämmer kommer man få en error av typen \code{IllegalArgumentException}.

Rad 30: Anledningen till att man kan sätta vikten till något negativt är att checken om det är negativt endast görs när man skapar \code{Gurka7} vilket innebär att i efterhand kan man ändra den till vilket värde som helst (av typen \code{Int}).


\QUESTEND






\WHAT{Egendefinierad setter kombinerat med privat konstruktor.}

\QUESTBEGIN

\Task  \what~

\Subtask Förklara vad som händer nedan. Vilka rader ger vilka felmeddelanden?
\begin{REPL}
scala> :pa
class Gurka8 private (private var _vikt: Int) {
  def vikt = _vikt
  def vikt_=(v: Int): Unit = {
    require(v >= 0, s"negativ vikt: $v")
    _vikt = v
  }
}

object Gurka8 {
  def apply(vikt: Int) = {
    require(vikt >= 0, s"negativ vikt: $vikt")
    new Gurka8(vikt)
  }
}
scala> val g = Gurka8(-42)
scala> val g = Gurka8(42)
scala> g.vikt
scala> g.vikt = 0
scala> g.vikt = -1
scala> g.vikt += 42
scala> g.vikt -= 1000
\end{REPL}

\Subtask\Pen Vad är fördelen med möjligheten att skapa egendefinierade setters?

\SOLUTION


\TaskSolved \what


\SubtaskSolved  Rad 16:
\begin{REPL}
	java.lang.IllegalArgumentException: requirement failed: negativ vikt: -42
\end{REPL}
Kompanjonsobjektet har en requirement på att \code{vikt >= 0} vilket innebär att om det inte stämmer kommer man få en error.

Rad 20:
\begin{REPL}
	java.lang.IllegalArgumentException: requirement failed: negativ vikt: -1
\end{REPL}
Eftersom settern har implementerat ett krav på att vikten måste vara större eller lika med 0 får man en error när man försöker sätta den till -1.

Rad 22:
\begin{REPL}
	java.lang.IllegalArgumentException: requirement failed: negativ vikt: -958
\end{REPL}
Eftersom 42-1000 är mindre än noll får man en error.

\SubtaskSolved  Man kan sätta egna mer specifika krav på vad som får göras med värdena så man har större koll på att inget oväntat händer.

% uppgift 5

\QUESTEND




%%<AUTOEXTRACTED by mergesolu>%%      % uppgift 4




\WHAT{En oföränderlig kvadrat med alternativ fabriksmetod.}

\QUESTBEGIN

\Task \label{task:Square} \what~

\Subtask Implementera klassen \code{Square} enligt nedan specifikation. Gör  implementationen i en kodeditor, så som \code{gedit}, och klistra in klassen i Scala REPL efter kommandot \code{:pa} (förkortning av \code{:paste}). På så sätt blir \code{object Square} ett kompanjonsobjekt till \code{class Square}.

\begin{ScalaSpec}{Square}
/** A class representing a square object with position and side. */
class Square(val x: Int, val y: Int, val side: Int) {
  /** The area of this Square */
  val area: Int = ???

  /** Creates a new Square moved to position (x + dx, y + dy) */
  def move(dx: Int, dy: Int): Square = ???

  /** Tests if this Square has equal size as that Square */
  def isEqualSizeAs(that: Square): Boolean = ???

  /** Multiplies the side with factor and rounded to nearest integer */
  def scale(factor: Double): Square = ???

  /** A string representation of this Square */
  override def toString: String = ???
}

object Square {
  /** A square placed in origin with size 1 */
  val unit: Square = ???

  /** Constructs a new Square object at (x, y) with size side */
  def apply(x: Int, y: Int, side: Int): Square = ???

  /** Constructs a new Square object at (0, 0) with side 1 */
  def apply(): Square = ???
}
\end{ScalaSpec}

\Subtask Testa din kvadrat enligt nedan. Förklara vad som händer.

\begin{REPL}
scala> val (s1, s2) = (Square(), Square(1, 10, 1))
scala> val s3 = s1.move(1,-5)
scala> s1 isEqualSizeAs s3
scala> s2 isEqualSizeAs s1
scala> s1 isEqualSizeAs Square.unit
scala> s2.scale(math.Pi) isEqualSizeAs s2
scala> s2.scale(math.Pi) isEqualSizeAs s2.scale(math.Pi)
\end{REPL}

\SOLUTION


\TaskSolved \what
 \begin{CodeSmall}
	class Square(val x: Int, val y: Int, val side: Int) {
		val area: Int = side*side

		def move(dx: Int, dy: Int): Square = new Square(x + dx, y + dy, side)

		def isEqualSizeAs(that: Square): Boolean = this.side == that.side

		def scale(factor: Double): Square = new Square(x, y, (side*factor).toInt)

		override def toString: String = s"Square(x: $x, y: $y, side: $side)"
	}

	object Square {
		val unit: Square = new Square(0, 0, 1)

		def apply(x: Int, y: Int, side: Int): Square = new Square(x, y, side)

		def apply(): Square = new Square(0, 0, 1)
	}
\end{CodeSmall}

Eftersom \code{s1}, \code{s2}, \code{s3} och \code{Square.unit} alla har en sida med längden 1 så kommer rad 3-5 returnera \code{true}. Rad 6 kommer returnera \code{false} eftersom \code{s2.scale(math.Pi)} sida är $\pi$ och \code{s2} fortfarande har sidan 1. Rad 7 kommer däremot returnera \code{true} då båda har sidan $\pi$.


\QUESTEND






\WHAT{Referenslikhet versus strukturlikhet.}

\QUESTBEGIN

\Task  \what~  Metoden \code{==} på case-klasser ger \textbf{strukturlikhet} (även kallad innehållslikhet) så att \emph{innehållet} i klassens klassparametrar jämförs om de har lika värde, medan för vanliga klasser ger metoden \code{==} \textbf{referenslikhet} där olika objekt är olika även om de har samma innehåll (om man inte överskuggar metoden \code{equals} som anropas av \code{==} vilket vi ska titta närmare på i kapitel \ref{chapter:W08}).

\begin{REPL}
scala> class GurkaRef(val vikt: Int)
scala> case class GurkaStrukt(val vikt: Int)
scala> val a = new GurkaRef(42)
scala> val b = new GurkaRef(42)
scala> val c = new GurkaStrukt(42)
scala> val d = new GurkaStrukt(42)
scala> a == b
scala> c == d
\end{REPL}

\Subtask Förklara vad som händer ovan.

\Subtask Istället för \code{==}, prova metoden \code{eq} på objekten ovan. Metoden \code{eq} ger alltid referenslikhet (även om byter ut metoden \code{equals}).

\SOLUTION


\TaskSolved \what


\SubtaskSolved  Variablerna \code{a} och \code{b} är båda objekt av en vanlig klass vilket kommer innebära att de jämförs med referenslikhet och eftersom de inte är samma objekt retunerar \code{==} \code{false}. \code{c} och \code{d} är däremot objekt av en case klass så de jämförs med strukturlikhet och eftersom de har samma vikt returnerar \code{==} \code{true}.

\SubtaskSolved  Både \code{a eq b} och \code{c eq d} ska returnera \code{false} eftersom de alla är olika objekt och det är referenslikhetsom gäller.


\QUESTEND




%%<AUTOEXTRACTED by mergesolu>%%      % uppgift 6




\WHAT{Klassen \code{Point} med case-klass.}

\QUESTBEGIN

\Task \label{task:Point} \what~

\Subtask Implementera klassen \code{Point} som en oföränderlig case-klass med heltalsattributen \code{x} och \code{y}.

\Subtask Lägg till metoden \code{distanceTo(that: Point): Double} som räknar ut avståndet till en annan punkt med hjälp av \code{math.hypot}.

\Subtask Lägg till metoden \code{distanceTo(x: Int, y: Int): Double} som räknar ut avståndet till koordinaterna x och y med hjälpa av metoden i föregående deluppgift.

\Subtask Lägg till metoden \code{move(dx: Int, dy: Int): Point} som skapar en ny punkt på translaterad position enligt delta-koordinaterna \code{dx} och {dy}.

\Subtask Lägg till ett kompanjonsobjekt med medlemmen \code{val origin} som ger en punkt i origo.

\Subtask Undersök metoderna \code{==}, \code{!=}, \code{eq} och \code{ne} och förklara vad som händer nedan:
\begin{REPL}
scala> Point(1, 2) == Point(1, 3)
scala> Point(1, 2) != Point(1, 3)
scala> Point(1, 2) == Point(1, 2)
scala> Point(1, 2) != Point(1, 2)
scala> Point.origin.move(1, 1) == Point.origin.move(1, 1)
scala> Point.origin.move(1, 1).move(1, 1) != Point(2, 2)
scala> Point(0, 0) eq Point(0, 0)
scala> Point(0, 0) ne Point(0, 0)
scala> Point.origin eq Point.origin
scala> Point.origin ne Point.origin
scala> val p1 = Point(0, 0)
scala> val p2 = p1
scala> p1 eq p2
\end{REPL}

\Subtask Vad ger \code{Point.origin eq Point.origin} för resultat om \code{origin} istället  implementeras som \code{def origin: Point = Point(0, 0)}

\Subtask\Pen Vad är det för skillnad på strukturlikhet och referenslikhet?

\SOLUTION


\TaskSolved \what


\SubtaskSolved  se e) för komplett lösning

\SubtaskSolved  se e) för komplett lösning

\SubtaskSolved  se e) för komplett lösning

\SubtaskSolved  se e) för komplett lösning

\SubtaskSolved  \begin{CodeSmall}
case class Point(x: Int, y: Int) {

	def distanceTo(that: Point): Double = math.hypot(that.x - x, that.y -y)

	def distanceTo(x: Int, y: Int): Double = distanceTo(Point(x, y))

	def move(dxdy: (Int, Int)): Point = Point(dxdy._1 + x, dxdy._2 + y)
}

object Point {
	//val origin: Point = new Point(0, 0)
	def origin: Point = Point(0, 0)
}
\end{CodeSmall}

\SubtaskSolved  \code{==} och \code{!=} kollar strukturlikhet så om två objekt innehåller samma värden kommer \code{==} returnera \code{true} och \code{!=} \code{false} och vise versa. \code{eq} och \code{ne} kollar referenslikhet så om två variabler pekar på samma objekt kommer \code{eq} returnera \code{true} och \code{ne} \code{false} och vise versa.

\SubtaskSolved  \code{false}. Detta eftersom om origin implementeras som en metod som returnerar en ny \code{Point} varje gång den kallas kommer \code{Point.origin} inte peka på samma objekt varje gång metoden kallas (\code{eq} är referenslikhet).

\SubtaskSolved  Sturkturlikhet bryr sig endast om innehållet i objekten och jämför det. Det kvittar alltså om det är samma objekt eller två olika så länge de innehåller samma värden. Referenslikhet kollar endast på om det är samma objekt variablerna pekar på och struntar fullständigt i om de innehåller samma värden.

% uppgift 8

\QUESTEND




%%<AUTOEXTRACTED by mergesolu>%%      % uppgift 7




\WHAT{NEEDS A TOPIC DESCRIPTION}

\QUESTBEGIN

\Task \label{task:PointSquare} \what~ Ändra representationen av positionen i klassen \code{Square} från deluppgift \ref{task:Square} till att vara en \code{Point} från deluppgift \ref{task:Point}.


\SOLUTION


\TaskSolved \what
 \begin{CodeSmall}
class Square(val p: Point, val side: Int) {
	val area: Int = side*side

	def move(dx: Int, dy: Int): Square = new Square(p.move(dx, dy), side)

	def isEqualSizeAs(that: Square): Boolean = this.side == that.side

	def scale(factor: Double): Square = new Square(p, (side*factor).toInt)

	override def toString: String = s"Square(p: $p, side: $side)"
}

object Square {
	val unit: Square = new Square(new Point(0, 0), 1)

	def apply(x: Int, y: Int, side: Int): Square =
		new Square(new Point(x, y), side)

	def apply(): Square = new Square(new Point(0, 0), 1)
}
\end{CodeSmall}



\QUESTEND




%%<AUTOEXTRACTED by mergesolu>%%      % uppgift 9




\WHAT{Case-klassen \code{Point} med 2-tupel.}

\QUESTBEGIN

\Task \label{task:PointTuple} \what~   I ett utvecklingsprojekt vill man ändra representationen av positionen i den gamla klassen  \\ \code{case class Point(x: Int, y: Int)} så att positionen istället i den uppdaterade klassen representeras av en 2-tupel. Man kan då vid konstruktion utnyttja att n-tupler som parameter även kan skrivas som en parameterlista med n argument, varför både \code{Point(1,2)} och \code{Point((1,2))} fungerar fint. Samtidigt vill man att befintlig kod som fortfarande använder \code{x} och \code{y} ska fungera utan ändringar.  Implementera den nya \code{Point} enligt specifikationen nedan.
\begin{ScalaSpec}{Point}
/** A 2-dimensional immutable position p in an integer coordinate system */
case class Point(p:(Int, Int)) {
  /** The x-axis position of this Point */
  val x: Int = ???

  /** The y-axis position of this Point */
  val y: Int = ???

  /** The distance to another Point that */
  def distanceTo(that: Point): Double = ???

  /** The distance to another 2-tuple that representing (x, y). */
  def distanceTo(that: (Int, Int)): Double = ???

  /** A new Point that is moved (dx, dy) */
  def move(dxdy: (Int, Int)): Point = ???
}

object Point {
  /** A Point object at position (0, 0) */
  val origin: Point = ???
}
\end{ScalaSpec}

\SOLUTION


\TaskSolved \what
  \begin{CodeSmall}
case class Point(p:(Int,Int)) {
	val x: Int = p._1

	val y: Int = p._2

	def distanceTo(that: Point): Double = math.hypot(that.x - x, that.y -y)

	def distanceTo(that: (Int, Int)): Double = distanceTo(Point(that))
	def move(dx: Int, dy: Int): Point = Point(x + dx, y + dy)
}

object Point {
	val origin: Point = new Point(0, 0)
}
\end{CodeSmall}



\QUESTEND




%%<AUTOEXTRACTED by mergesolu>%%      % uppgift 10




\WHAT{NEEDS A TOPIC DESCRIPTION}

\QUESTBEGIN

\Task  \what~\Pen Vad behöver du ändra i klassen \code{Square} från uppgift \ref{task:PointSquare} för att den ska fungera med en \code{Point} med 2-tupel från uppgift \ref{task:PointTuple}?

\SOLUTION


\TaskSolved \what
 Inget! Eftersom både \code{Point(1,2)} och \code{Point((1,2))} är okej sätt att komma åt den nya klassen så kommer det se likadant utifrån och därför behöver man inte ändra något i \code{Square}.


\QUESTEND






\WHAT{Objekt med föränderligt tillstånd \Eng{mutable state}.}

\QUESTBEGIN

\Task  \what~  Du ska implementera en modell av en hoppande groda som uppfyller följande krav:
\begin{enumerate}[nolistsep, noitemsep]
\item Varje grodobjekt ska hålla reda på var den är.
\item Varje grodobjekt ska hålla reda på hur långt grodan hoppat totalt.
\item Varje grodobjekt ska kunna beräkna hur långt det är mellan grodans nuvarande position och utgångsläget.
\item Alla grodor börjar sitt hoppande i origo.
\item En groda kan hoppa enligt två metoder:
  \begin{itemize} [nolistsep, noitemsep]
  \item relativ förflyttning enligt parametrarna \code{dx} och \code{dy},
  \item slumpmässig förflyttning $[1, 10]$ i x-led och $[1, 10]$ i y-led.
  \end{itemize}
\end{enumerate}

\Subtask Implementera klassen \code{Frog} enligt nedan specifikation och ovan krav. \\  \emph{Tips:}
  \begin{itemize} [nolistsep, noitemsep]
  \item Om namnet man vill ge ett privat föränderligt attribut ''krockar'' med ett metodnamn, är det vanligt att man börjar attributets namn med understreck, t.ex. \code{private var _x } för att på så sätt undkomma namnkonflikten.
  \item Inför en metod i taget och klistra in den nya grodan i REPL efter varje utvidgning och testa.
  \end{itemize}

\begin{ScalaSpec}{Frog}
class Frog private (initX: Int = 0, initY: Int = 0) {
  def jump(dx: Int, dy: Int): Unit = ???
  def x: Int = ???
  def y: Int = ???
  def randomJump: Unit = ???
  def distanceToStart: Double = ???
  def distanceJumped: Double = ???
  def distanceTo(that: Frog): Double = ???
}
object Frog {
  def spawn(): Frog = ???
}
\end{ScalaSpec}

\Subtask Skriv ett testhuvudprogram som kontrollerar så att alla krav är uppfyllda och att alla metoder fungerar som de ska.

\Subtask\Pen Vad kallas en metod som enbart returnerar värdet av ett privat attribut?

\Subtask\Pen Hur kan man från en metods signatur få en ledtråd om att ett objekt har föränderligt tillstånd \Eng{mutable state}?

\Subtask Inför setters för attributen som håller reda på x- och y-postitionen. Förändringar av positionen i x- eller y-led ska räknas som ett hopp och alltså registreras i det attribut som håller reda på det ackumulerade hoppavståndet.

\Subtask Simulera ett massivt grodhoppande med krockdetektering genom att skapa 100 grodor som till att börja med är placerade på x-axeln med avståndet $8$ längdenheter mellan sig. Låt grodorna i en \code{while}-sats hoppa slumpmässigt tills någon groda befinner sig närmare än $0.5$ längdenheter som är definitionen på att de har krockat. Räkna hur många looprundor som behövs innan något grodpar krockar och skriv ut antalet. \\ \emph{Tips:} Börja med pseudokod på papper. Använd en grodvektor.

\clearpage

\ExtraTasks %%%%%%%%%%%%%%%%%%%

\SOLUTION


\TaskSolved \what


\SubtaskSolved  \begin{CodeSmall}
class Frog private (initX: Int = 0, initY: Int = 0) {
	private var _x: Int = initX
	private var _y: Int = initY
	private var _distanceJumped: Double = 0

	def jump(dx: Int, dy: Int): Unit = {
		_x += dx
		_y += dy
		_distanceJumped += Math.hypot(dx, dy)
	}

	def x: Int = _x
	def y: Int = _y

	def randomJump: Unit = {
		val r = scala.util.Random
		val xtmp = r.nextInt(10)+1
		val ytmp = r.nextInt(10)+1
		_x += xtmp
		_y += ytmp
		_distanceJumped += Math.hypot(xtmp, ytmp)
	}

	def distanceToStart: Double = Math.hypot(_x,_y)
	def distanceJumped: Double = _distanceJumped
	def distanceTo(that: Frog): Double = Math.hypot(_x - that.x, _y - that.y)
}

object Frog {
	def spawn(): Frog = new Frog()
}
\end{CodeSmall}

\SubtaskSolved  \begin{CodeSmall}
val f1 = Frog.spawn()
//test requirement 1 and 4
assert(f1.x == 0 && f1.y == 0, "Either x or y isn't 0")

f1.jump(4,3)
//test requirement 1 and 5
assert(f1.x == 4 && f1.y == 3, "Either x isn't 4 or y isn't 3")

f1.jump(4,3)
//test requirement 2
var text = "distanceJumped is " + f1.distanceJumped + ". Should be 10"
assert(f1.distanceJumped == 10, text)

f1.jump(-4,-3)
//test requirement 3
text = "distanceToStart is " + f1.distanceJumped + ". Should be 5"
assert(f1.distanceToStart == 5, text)

var f2 = Frog.spawn()
for (x <- 1 to 1000) {
	f2.randomJump
	//test requirement 5
	text = "Either x or y isn't in [1,10]. x:" + f2.x + ", y: " + f2.y
	assert(f2.x > 0 && f2.x <= 10 && f2.y > 0 && f2.y <= 10, text)
	f2 = Frog.spawn()
}

val f3 = Frog.spawn()
f3.jump(1,1)
val f4 = Frog.spawn()
f4.jump(4,5)
// Test distanceT()
text = "distanceTo is " + f3.distanceTo(f4) + ". Should be 5"
assert(f3.distanceTo(f4) == 5, text)
\end{CodeSmall}

\SubtaskSolved  Getter

\SubtaskSolved  Om metoden har parametrar och retur-typen \code{Unit}. Det betyder troligen att parametrarna ändrar något istället för att skapa något nytt.

\SubtaskSolved  \begin{CodeSmall}
class Frog private (initX: Int = 0, initY: Int = 0) {
	private var _x: Int = initX
	private var _y: Int = initY
	private var _distanceJumped: Double = 0

	def jump(dx: Int, dy: Int): Unit = {
		_x += dx
		_y += dy
		_distanceJumped += Math.hypot(dx, dy)
	}

	def x: Int = _x
	def y: Int = _y

	def x_= (newX: Int): Unit = {
		_distanceJumped += Math.abs(_x - newX)
		_x = newX
	}
	def y_= (newY: Int): Unit = {
		_distanceJumped += Math.abs(_y - newY)
		_y = newY
	}

	def randomJump: Unit = {
		val r = scala.util.Random
		val xtmp = r.nextInt(10)+1
		val ytmp = r.nextInt(10)+1
		_x += xtmp
		_y += ytmp
		_distanceJumped += Math.hypot(xtmp, ytmp)
	}

	def distanceToStart: Double = Math.hypot(_x,_y)
	def distanceJumped: Double = _distanceJumped
	def distanceTo(that: Frog): Double = Math.hypot(_x - that.x, _y - that.y)
}

object Frog {
	def spawn(): Frog = new Frog()
}
\end{CodeSmall}

\SubtaskSolved  \begin{CodeSmall}
var noCollision = true
var counter = 0
val numberOfFrogs = 100
val distanceBetweenFrogs = 8
val frogArray = Array.fill(numberOfFrogs){Frog.spawn()}
(0 until numberOfFrogs).foreach(i => frogArray(i).x(i*distanceBetweenFrogs))
while (noCollision) {
	frogArray.foreach(frog => frog.randomJump)
	for (frog <- frogArray) {
		for (frog2 <- frogArray) {
			if (frog != frog2 && frog.distanceTo(frog2) < 0.5) {
				noCollision = false
			}
		}
	}
	counter += 1
}
print(counter)
\end{CodeSmall}


\clearpage

\ExtraTasks %%%%%%%%%%%%


\QUESTEND




%%<AUTOEXTRACTED by mergesolu>%%      % uppgift 11




\WHAT{En kvadratklass med föränderligt tillstånd \Eng{mutable state}.}

\QUESTBEGIN

\Task  \what~  Webbshoppen UberSquare säljer flyttbara kvadrater. I affärsmodellen ingår att ta betalt per förflyttning. Du ska hjälpa UberSquare med att utveckla en enkel systemprototyp.

\Subtask Implementera \code{Square} enligt nedan specifikation, under uppfyllandet av följande krav:

\begin{enumerate}[nolistsep, noitemsep]
\item Till skillnad från uppgift \ref{task:Square} ska du nu göra en kvadrat med föränderligt tillstånd \Eng{mutable state}. I stället för att vid förflyttning returnera ett nytt kvadratobjekt, returneras \code{Unit} i samband med att privata attribut uppdateras.
\item Du ska införa funktionalitet som räknar antalet förflyttningar som gjorts för varje kvadrat som skapats och även räkna ut det totala antalet förflyttningar som någonsin gjorts.
\item Varje gång förflyttning sker adderas en kostnad till den ackumulerade kostnaden för respektive kvadrat. Kostnaden för varje förflyttning är avståndet till ursprungsläget multiplicerat med storleken på kvadraten.
\end{enumerate}

\begin{ScalaSpec}{Square}
/** A mutable and expensive Square. */
class Square private (val initX: Int, val initY: Int, val initSide: Int) {

  private var nMoves = 0;
  private var sumCost = 0.0;
  private var _x = initX;
  private var _y = initY;
  private var _side = initSide;

  private def addCost: Unit = {
   sumCost += ???
  }

  /** The current position on the x axis */
  def x: Int = ???

  /** The current position on the y axis */
  def y: Int = ???

  /** The size of this Square */
  def side = ???

  /** Scales the side of this square and rounds it to nearest integer */
  def scale(factor: Double): Unit = ???

  /** Moves this square to position (x + xd, y + dy) */
  def move(dx: Int, dy: Int): Unit = ???

  /** Moves this square to position (x, y) */
  def moveTo(x: Int, y: Int): Unit = ???

  /** The accumulated cost of this Square */
  def cost: Double = ???

  /** Reset the cost of this Square */
  def pay: Unit = ???

  /** A string representation of this Square */
  override def toString: String =
    s"Square[($x, $y), side: $side, #moves: $nMoves times, cost: $sumCost]"
}

object Square {
  private var created = Vector[Square]()

  /** Constructs a new Square object at (x, y) with size side */
  def apply(x: Int, y: Int, side: Int): Square = {
    require(side >= 0, s"side must be positive: $side")
    ???
  }

  /** Constructs a new Square object at (0, 0) with side 1 */
  def apply(): Square = apply(0, 0, 1)

  /** The total number of moves that have been made for all squares. */
  def totalNumberOfMoves: Int = ???

  /** The total cost of all squares. */
  def totalCost: Double = ???
}
\end{ScalaSpec}

\Subtask Testa din kvadratprototyp i REPL enligt nedan:
\begin{REPL}
scala> val xs = Vector.fill(10)(Square())
scala> xs.foreach(_.move(2,3))
scala> xs.foreach(_.scale(2.9))
scala> val (m, c) = (Square.totalNumberOfMoves, Square.totalCost)
m: Int = 10
c: Double = 36.055512754639885
\end{REPL}


\clearpage

\AdvancedTasks %%%%%%%%%%%%%%%%%


\SOLUTION


\TaskSolved \what


\vspace{1em} %tweak pagination

\begin{CodeSmall}
/** A mutable and expensive Square. */
class Square private (val initX: Int, val initY: Int, val initSide: Int) {

  private var nMoves = 0;
  private var sumCost = 0.0;
  private var _x = initX;
  private var _y = initY;
  private var _side = initSide;

  private def addCost: Unit = {
   sumCost += math.hypot(x - initX, y - initY) * side
  }

  /** The current position on the x axis */
  def x: Int = _x

  /** The current position on the y axis */
  def y: Int = _y

  /** The size of the side */
  def side = _side

  /** Scales the size of this square and rounds it to nearest integer */
  def scale(factor: Double): Unit = { _side = (_side * factor).round.toInt }

  /** Moves this square to position (x + xd, y + dy) */
  def move(dx: Int, dy: Int): Unit = {
    _x += dx; _y += dy;
    nMoves += 1
    addCost
  }

  /** Moves this square to position (x, y) */
  def moveTo(x: Int, y: Int): Unit = {
    _x = x; _y = y;
    nMoves += 1
    addCost
  }

  /** The accumulated cost of this Square */
  def cost: Double = sumCost

  /** Reset the cost of this Square */
  def pay: Unit = {sumCost = 0}

  /** A string representation of this Square */
  override def toString: String =
    s"Square[($x, $y), side: $side, #moves: $nMoves times, cost: $sumCost]"
}

object Square {
  private var created = Vector[Square]()

  /** Constructs a new Square object at (x, y) with size side */
  def apply(x: Int, y: Int, side: Int): Square = {
    require(side >= 0, s"side must be positive: $side")
    val sq = (new Square(x, y, side))
    created :+= sq
    sq
  }

  /** Constructs a new Square object at (0, 0) with side 1 */
  def apply(): Square = apply(0, 0, 1)

  /** The total number of moves that have been made for all squares. */
  def totalNumberOfMoves: Int = created.map(_.nMoves).sum

  /** The total cost of all squares. */
  def totalCost: Double = created.map(_.cost).sum
}
\end{CodeSmall}

\AdvancedTasks %%%%%%%%%

\TODO
\QUESTEND






\WHAT{Hjälpkonstruktor.}

\QUESTBEGIN

\Task \label{task:aux-constructor} \what~   I uppgift \ref{task:Square} erbjöds ett alternativt sätt att skapa \code{Square} med en extra fabriksmetod med namnet \code{apply} i kompanjonsobjektet. Ett annat sätt att göras detta på, som i Scala är mindre vanligt (men i Java är desto vanligare), är att definiera flera konstruktorer innuti klassen. I Scala kallas en sådan extra konstruktor för \textbf{hjälpkonstruktor} \Eng{auxiliary constructor}.

En hjälpkonstruktor skapar man i Scala genom att definiera en metod som har det speciella namnet \code{this}, alltså en deklaration \code{def this(...) = ...} Hjälponstruktorer måste börja med att anropa en annan konstruktor, antingen den primära konstruktorn eller en tidigare definierad  hjälpkonstruktor.

\Subtask Läs mer om hjälpkonstruktorer här: \\ \href{http://www.artima.com/pins1ed/functional-objects.html#6.7}{www.artima.com/pins1ed/functional-objects.html\#6.7}

\Subtask Hitta på en egen uppgift med hjälpkonstruktorer, baserat på någon av klasserna i tidigare övningar.


%\Task \TODO \\ \code{class Rational private (numerator: BigInt, denominator: BigInt)} \\
%Inspirerat av Rational i pins1ed med GCD\SOLUTION


\QUESTEND


%!TEX encoding = UTF-8 Unicode
%!TEX root = ../exercises.tex

\ifPreSolution



\Exercise{\ExeWeekSIX}\label{exe:W06}

\begin{Goals}
\item Kunna skapa och använda \code{match}-uttryck med konstanta värden, garder och mönstermatchning med case-klasser.
\item Kunna skapa och använda case-objekt för matchningar på uppräknade värden.
\item Kunna hantera saknade värden med hjälp av typen \code{Option} och mönstermatchning på \code{Some} och \code{None}.
\item Kunna fånga undantag med \code{scala.util.Try}.
\item Känna till \code{try}, \code{catch} och \code{throw}.
%\item Känna till \jcode{switch}-satser i Java.
\item Känna till nyckelordet \code{sealed} och förstå nyttan med förseglade typer.
%\item Känna till relationen mellan \code{hashCode} och \code{equals}.
%\item Kunna skapa partiella funktioner med case-uttryck.
%\item Känna till betydelsen av små och stora begynnelsebokstäver i case-grenar i en matchning, samt förstå hur namn binds till värden in en case-gren.
%\item Kunna använda \code{flatMap} tillsammans med \code{Option} och \code{Try}.
%\item Känna till skillnaderna mellan \code{try}-\code{catch} i Scala och java.
%\item Känna till att metoden \code{unapply} används vid mönstermatchning.
%\item Kunna implementera \code{equals} med hjälp av en \code{match}-sats, som fungerar för finala klasser utan arv.
%\item Känna till \code{null}.
\end{Goals}

\begin{Preparations}
\item \StudyTheory{06}
\end{Preparations}

\BasicTasks %%%%%%%%%%%%%%%%

\else



\ExerciseSolution{\ExeWeekSIX}

\BasicTasks %%%%%%%%%%%

\fi




\WHAT{Matcha på konstanta värden.}

\QUESTBEGIN

\Task \label{task:vegomatch} \what~   % I Scala finns ingen \jcode{switch}-sats. I stället har Scala ett \code{match}-uttryck som är mer kraftfullt. Dock saknar Scala nyckelordet \jcode{break} och Scalas \code{match}-uttryck kan inte ''falla igenom'' som skedde i uppgift \ref{task:switch}\ref{subtask:break}.

\Subtask \label{subtask:vegomatch} Skriv nedan program med en kodeditor och spara i filen \texttt{Match.scala}. Kompilera och kör och och ge som argument din favoritgrönsak. Vad händer? Förklara hur ett \code{match}-uttryck fungerar.

\scalainputlisting[numbers=left,basicstyle=\ttfamily\fontsize{11}{12}\selectfont]{examples/Match.scala}

\Subtask Vad blir det för felmeddelande om du tar bort case-grenen för defaultvärden och indata väljs så att inga case-grenar matchar? Är det ett exekveringsfel eller ett kompileringsfel?

% \Subtask Beskriv några skillnader i syntax och semantik mellan Javas flervalssats \jcode{switch} och Scalas flervalsuttryck \code{match}.



\SOLUTION


\TaskSolved \what


\SubtaskSolved  %Svaret blir identiskt mot föregående uppgiften i Java.\\
Scalas \code{match}-uttryck jämför stegvis värdet med varje \code{case} för att sedan returnera ett värde tillhörande motsvarande \code{case}.

\SubtaskSolved  \begin{REPL}
scala.MatchError 
\end{REPL}
Exekveringsfel, uppstår av en viss input under körningen.

% \SubtaskSolved  Scalas \code{match} ersätter kolonet (:) i \jcode{switch} med Scalas högerpil (=>).\\
% \code{match} returnerar ett värde till skillnad från \jcode{switch} som inte returnerar något.\\
% \code{match} kan inte $"$falla igenom$"$ så ett \jcode{break} efter varje \jcode{case} är inte nödvändigt.\\
% Till skillnad från \jcode{switch}-satsen kastar \code{match} ett \code{MatchError} om ingen matchning skulle ske.



\QUESTEND






\WHAT{Gard i case-grenar.}

\QUESTBEGIN

\Task  \what~  Med hjälp en gard \Eng{guard} i en case-gren kan man begränsa med ett villkor om grenen ska väljas.

Utgå från koden i uppgift \ref{task:vegomatch}\ref{subtask:vegomatch} och byt ut case-grenen för \code{'g'}-matchning till nedan variant med en gard med nyckelordet \code{if} (notera att det inte behövs parenteser runt villkoret):
\begin{Code}
    case 'g' if math.random() > 0.5 => "gurka är gott ibland..."
\end{Code}
Kompilera om och kör programmet upprepade gånger med olika indata tills alla grenar i \code{match}-uttrycket har exekverats. Förklara vad som händer.

\SOLUTION


\TaskSolved \what

Garden som införts vid \code{case 'g'} slumpar fram ett tal mellan 0 och 1 och om talet inte är större än $0.5$ så blir det ingen matchning med \code{case 'g'} och programmet testar vidare tills default-caset.\\
Gardens krav måste uppfyllas för att det ska matcha som vanligt.



\QUESTEND






\WHAT{Mönstermatcha på attributen i case-klasser.}

\QUESTBEGIN

%\Task \label{task:match-caseclass} \what~   Scalas \code{match}-uttryck är extra kraftfulla om de används tillsammans med \code{case}-klasser: då kan attribut extraheras automatiskt och bindas till lokala variabler direkt i case-grenen som nedan exempel visar (notera att \code{v} och \code{rutten} inte behöver deklareras explicit). Detta kallas för \textbf{mönstermatchning}.

\Task \label{task:match-caseclass} \what~   Scalas \code{match}-uttryck är extra kraftfulla om de används tillsammans med \code{case}-klasser: då kan attribut extraheras automatiskt och bindas till lokala variabler direkt i case-grenen som nedan exempel visar (notera att \code{v} och \code{rutten} inte behöver deklareras explicit). Detta kallas för \textbf{mönstermatchning}. 
Vad skrivs ut nedan? Varför? Prova att byta namn på \code{v} och \code{rutten}.
%\Subtask \label{subtask:autobinding-match} Vad skrivs ut nedan? Varför? Prova att byta namn på \code{v} och \code{rutten}.
\begin{REPL}
scala> case class Gurka(vikt: Int, ärRutten: Boolean)
scala> val g = Gurka(100, true)
scala> g match { case Gurka(v,rutten) => println("G" + v + rutten) }
\end{REPL}

%\TODO %Tab två gånger fungerar inte i scala3-repl, issue #536
%\Subtask Skriv sedan nedan i REPL och tryck TAB två gånger efter punkten. Vad har \code{unapply}-metoden för resultattyp?
%\begin{REPL}
%scala> Gurka.unapply   // Tryck TAB två gånger
%\end{REPL}
%\begin{Background}
%Case-klasser får av kompilatorn automatiskt ett kompanjonsobjekt \Eng{companion object}, i detta fallet \code{object Gurka}. Det objektet får av kompilatorn automatiskt en \code{unapply}-metod. Det är \code{unapply} som anropas ''under huven'' när case-klassernas attribut extraheras vid mönstermatchning, men detta sker alltså automatiskt och man behöver inte explicit nyttja \code{unapply} om man inte själv vill implementera s.k. extraherare \Eng{extractors}; om du är nyfiken på detta, se fördjupningsuppgift \ref{task:extractor}.
%\end{Background}

%\Subtask Anropa \code{unapply}-metoden enligt nedan. Vad blir resultatet?
%\begin{REPL}
%scala> Gurka.unapply(g)
%\end{REPL}
%Vi ska i senare uppgifter undersöka hur typerna \code{Option} och \code{Some} fungerar och hur man kan ha nytta av dessa i andra sammanhang.

% \Subtask Spara programmet nedan i filen \texttt{vegomatch.scala} och kompilera och kör med \code{scala vegomatch.Main 1000} i terminalen. Förklara hur predikatet \code{ärÄtvärd} fungerar.
% \scalainputlisting[numbers=left,basicstyle=\ttfamily\fontsize{11}{12}\selectfont]{examples/vegomatch.scala}
%

\SOLUTION


\TaskSolved \what \\
G100true. Vid byte av plats: Gtrue100.\\
\code{match} testar om kompanjonsobjektet \code{Gurka} är av typen \code{Gurka} med två parametervärden. De angivna parametrarna tilldelas namn, \code{vikt} får namnet \code{v} och \code{ärRutten} namnet \code{rutten} och skrivs sedan ut. Byts namnen dessa ges skrivs de ut i den omvända ordningen.

%\TODO % TAB+TAB fungerar inte i scala3-repl så svaret till uppgiften är felaktig
%\SubtaskSolved  \code{Option[(Int, Boolean)]}

%\SubtaskSolved	\code{Gurka(100, true)}

% \SubtaskSolved  \code{ärÄtvärd} testar om \code{Grönsak g} är av typen \code{Gurka(v, rutten)} eller \code{Tomat}. Dessa har sedan garder.\\ \code{Gurka} måste ha \code{vikt} över 100 och \code{ärRutten} vara \code{false} för att \code{case Gurka} ska returnera \code{true}.\\
% \code{Tomat} måste ha \code{vikt} över 50 och \code{ärRutten} vara \code{false} för att \code{case Tomat} ska returnera \code{true}.\\
% Matchas inte \code{Grönsak g} med någon av dessa returneras default-värdet \code{false}.



\QUESTEND







\WHAT{Matcha på case-objekt och nyttan med \code{sealed}.}

\QUESTBEGIN

\Task	\label{task:match-sealedtrait} \what~	Skriv nedan kodrader i en REPL en för en. Notera nyckelordet \code{sealed} som används för att försegla en typ. En \textbf{förseglad typ} måste ha alla sina subtyper i en och samma kodfil.
\begin{REPL}
scala> sealed trait Färg
scala> case object Spader extends Färg
\end{REPL}
\Subtask Hur lyder felmeddelandet och varför sker det? Är det ett kompileringsfel eller ett körtidsfel?

\Subtask  \label{subtask:match-sealedtrait-caseobject}
Skapa nu nedan kod i en editor och klistra in i REPL.
\begin{Code}
object Kortlek:
  sealed trait Färg
  object Färg:
      val values = Vector(Spader, Hjärter, Ruter, Klöver)
  case object Spader extends Färg
  case object Hjärter extends Färg
  case object Ruter extends Färg
  case object Klöver extends Färg
\end{Code}

\Subtask \label{subtask:match-sealedtrait-function}
Skapa en funktion \code{def parafärg(f: Färg): Färg} i en editor, som med hjälp av ett match-uttryck returnerar parallellfärgen till en färg. Parallellfärgen till \code{Hjärter} är \code{Ruter} och vice versa, medan parallellfärgen till \code{Klöver} är \code{Spader} och vice versa. Klistra in funktionen i REPL. Passa även på att skriva en \code{import}-sats för det yttre objektet \textbf{Kortlek}, så medlemmarna av objektet kan nås enkelt.
\begin{REPL}
scala> parafärg(Spader)
scala> val xs = Vector.fill(5)(Färg.values((math.random() * 4).toInt))
scala> xs.map(parafärg)
\end{REPL}

\Subtask \label{subtask:match-forgetcase}
Vi ska nu undersöka vad som händer om man glömmer en av case-grenarna i matchningen i \code{parafärg}. ''Glöm'' alltså avsiktligt en av case-grenarna och klistra in den nya \code{parafärg} med den ofullständiga matchningen. Hur lyder varningen? Kommer varningen vid körtid eller vid kompilering?

\Subtask Anropa \code{parafärg} med den ''glömda'' färgen. Hur lyder felmeddelandet? Är det ett kompileringsfel eller ett körtidsfel?

\Subtask Förklara vad nyckelordet \code{sealed} innebär och vilken nytta man kan ha av att \textbf{försegla} en supertyp.


\SOLUTION


\TaskSolved \what

\SubtaskSolved
\begin{REPL}
Cannot extend sealed trait Färg in a different source file
\end{REPL}
Felmeddelandet fås av att REPL:en behandlar varje inmatning individuellt och tillåter därför inte att subtypen \code{Spader} ärver från \Eng{extends} supertypen \code{Färg} eftersom denna var förseglad \Eng{sealed}. Mer om detta senare i kursen...

\SubtaskSolved
-

\SubtaskSolved
Förusatt att \code{import Kortlek._} har skrivits...
\begin{Code}
def parafärg(f: Färg): Färg = f match
  case Spader  => Klöver
  case Hjärter => Ruter
  case Ruter   => Hjärter
  case Klöver  => Spader
\end{Code}

\SubtaskSolved
\begin{REPL}
<console>:17: warning: match may not be exhaustive.
It would fail on the following input: Ruter
\end{REPL}
Varningen kommer redan vid kompilering.

\SubtaskSolved
\begin{REPL}
scala.MatchError: Ruter (of class Ruter)
  at .parafärg(<console>:17)
\end{REPL}
Detta är ett körtidsfel.

\SubtaskSolved  Om en klass är \code{sealed} innebär det att om ett element ska matchas och är en subtyp av denna klass så ger Scala varning redan vid kompilering om det finns en risk för ett \code{MatchError}, alltså om \code{match}-uttrycket inte är uttömmande och det finns fall som inte täcks av ett \code{case}.\\
En förseglad supertyp innebär att programmeraren redan vid kompileringstid får en varning om ett fall inte täcks och i sånt fall vilket av undertyperna, liksom annan hjälp av kompilatorn. Detta kräver dock att alla subtyperna delar samma fil som den förseglade klassen.



\QUESTEND


\WHAT{Mönstermatcha enumeration.}

\QUESTBEGIN
%\TODO %Se gärna över denna frågan samt facit.
\Task	\what~ Vi ska nu undersöka och jämföra skillnad mellan nyckelorden \code{enum} och \code{sealed trait}. Skriv nedan kod i en REPL.
\begin{Code}
enum Färg:
  case Spader, Hjärter, Ruter, Klöver
\end{Code}

\Subtask Skapa med hjälp av en editor igen en funktion \code{def parafärg(f: Färg): Färg}, nästintill likadan som den som vi skapade i deluppgift \ref{task:match-sealedtrait}\ref{subtask:match-sealedtrait-function}. Funktionen ska återigen utnyttja match-uttryck för att returnera paralellfärgen till argumentet som ges. Tänk på att denna gången är \code{Färg} inget \code{sealed trait}, utan istället en enumeration (\code{enum}). Klistra in funktionen i REPL.
\begin{REPL}
scala> parafärg(Färg.Ruter)
scala> val xs = Vector.fill(5)(Färg.values((math.random() * 4).toInt))
scala> xs.map(parafärg)
\end{REPL}


\Subtask
Fundera på skillnader och likheter mellan att utnyttja \code{sealed trait} ihop med \code{case}-objekt gentemot att använda sig av \code{enum} vid mönstermatchning.


\SOLUTION


\TaskSolved \what
\SubtaskSolved
\begin{Code}
def parafärg(f: Färg): Färg = f match
  case Färg.Spader  => Färg.Klöver
  case Färg.Hjärter => Färg.Ruter
  case Färg.Ruter   => Färg.Hjärter
  case Färg.Klöver  => Färg.Spader
\end{Code}
Likt uppgift \ref{task:match-sealedtrait}\ref{subtask:match-sealedtrait-function} så kan även här en \code{import}-sats skrivas för att nå medlemmarna i \code{Färg} utan punktnotation.
Det är dock inte alltid fördelaktigt att importera medlemmar till den globala namnrymden, då det kan förekomma namnkrockar. Anta ett exempel där vi jobbar på ett program med grafiskt användargränssnitt där vi har en färg \code{Red} definerad.
Anta också att vi nu till vårt program vill importera ytterligare en röd färg för kulörerna hjärter och ruter, denna också namngiven \code{Red}. I detta scenario hade det uppstått en namnkrock då \code{Red} redan är definerad så importeringen hade ej kunnat ske.

\SubtaskSolved
Vid mönstermatchning så fungerar \code{sealed trait} ihop med \code{case}-objekt i praktiken likadant som att använda sig av \code{enum}.
Vi såg att i deluppgift \ref{task:match-sealedtrait}\ref{subtask:match-forgetcase} så varnade REPL redan vid kompilering att denna matchning inte var uttömmande \Eng{exhaustive}. Detta gäller även vid användning av \code{enum}.

\QUESTEND



\WHAT{Betydelsen av små och stora begynnelsebokstäver vid matchning.}

\QUESTBEGIN

\Task  \what~  För att åstadkomma att namn kan bindas till variabler vid matchning utan att de behöver deklareras i förväg (som vi såg i uppgift \ref{task:match-caseclass}) så har identifierare med liten begynnelsebokstav fått speciell betydelse: den tolkas av kompilatorn som att du vill att en variabel  binds till ett värde vid matchningen. En identifierare med stor begynnelsebokstav tolkas däremot som ett konstant värde (t.ex. ett case-objekt eller ett case-klass-mönster).

\Subtask \emph{En case-gren som fångar allt}. En case-gren med en identifierare med liten begynnelsebokstav som saknar gard kommer att matcha allt. Prova nedan i REPL, men försök lista ut i förväg vad som kommer att hända. Vad händer?
\begin{REPL}
scala> val x = "urka"
scala> x match
         case str if str.startsWith("g") => println("kanske gurka")
         case vadsomhelst => println("ej gurka: " + vadsomhelst)
scala> val g = "gurka"
scala> g match
         case str if str.startsWith("g") => println("kanske gurka")
         case vadsomhelst => println("ej gurka: " + vadsomhelst)
\end{REPL}

\Subtask \emph{Fallgrop med små begynnelsbokstäver.} Innan du provar nedan i REPL, försök gissa vad som kommer att hända. Vad händer? Hur lyder varningarna och vad innebär de?
\begin{REPL}
scala> val any: Any = "varken tomat eller gurka"
scala> case object Gurka
scala> case object tomat
scala> any match
         case Gurka => println("gurka")
         case tomat => println("tomat")
         case _ => println("allt annat")
\end{REPL}

\Subtask \emph{Använd backticks för att tvinga fram match på konstant värde.} Det finns en utväg om man inte vill att kompilatorn ska skapa en ny lokal variabel: använd specialtecknet \emph{backtick}, som skrivs \`{} och kräver speciella tangentbordstryck.\footnote{Fråga någon om du inte hittar hur man gör backtick \`{} på ditt tangentbord.}  Gör om föregående uppgift men omgärda nu identifieraren \code{tomat} i tomat-case-grenen med backticks, så här: \code{  case `tomat` => ...}



\SOLUTION


\TaskSolved \what


\SubtaskSolved  Både \code{str} och \code{vadsomhelst} matchar med inputen, oavsett vad denna är på grund av att de har en liten begynnelsebokstav.\\
 \code{str} har dock en gard att strängen måste börja med $g$ vilket gör så endast \code{val g = "gurka"} matchar med denna. \code{val x = "urka"} plockas dock upp av \code{vadsomhelst} som är utan gard.

\SubtaskSolved
\begin{REPL}
<console>:16: warning: patterns after a variable pattern cannot match (SLS 8.1
.1)
\end{REPL}
och
\begin{REPL}
<console>:17: warning: unreachable code due to variable patter 'tomat' on line
16
\end{REPL}
Trots att en klass \code{tomat} existerar så tolkar Scalas \code{match} den som en \code{case}-gren som fångar allt på grund av en liten begynnelsebokstav. Detta gör så alla objekt som inte är av typen \code{Gurka} kommer ge utskriften \textit{tomat} och att sista caset inte kan nås.

\SubtaskSolved
\begin{Code}
case `tomat` => println("tomat")
\end{Code}



\QUESTEND





\WHAT{Matcha på innehåll i en Vector.}

\QUESTBEGIN

\Task \what ~ Kör nedan i REPL. Vad skrivs ut? Förklara vad som händer.
\begin{REPL}
scala> val xss = Vector(Vector("hej"),Vector("på", "dej"),Vector("4","x","2"))
scala> xss.map( _ match
  case Vector() => "tom"
  case Vector(a) => a.reverse
  case Vector(_, b) => b.reverse
  case Seq(a, "x", b) => a + b
  case _ => "ANNARS DETTA"
  ).foreach(println)
\end{REPL}


\SOLUTION

\TaskSolved \what

\begin{REPL}
jeh
jed
42
\end{REPL}
För varje element i \code{xss} görs en matching som resulterar i en sträng. Vad som händer i varje gren förklaras nedan.
\begin{enumerate}
  \item Första match-grenen aktiveras aldrig eftersom \code{xss} ej innehåller någon tom vektor.
  \item Andra grenen passar med \code{Vector("hej")} och variablen \code{a} binds till \code{"hej"}.
  \item Tredje grenen matchar \code{Vector("på", "dej")} där första värdet binds inte till någon variabel eftersom understreck finns på motsvarande plats, medan andra värdet binds till \code{b}.
  \item Fjärde grenen matchar en sekvens med tre värden där mittenvärdet är \code{"x"}. Den sista grenen aktiveras inte i detta exempel men hade matchat allt som inte fångas av tidigare grenar.
\end{enumerate}

\QUESTEND




\WHAT{Använda \code{Option} och matcha på värden som kanske saknas.}

\QUESTBEGIN

\Task  \what~  Man behöver ofta skriva kod för att hantera värden som eventuellt saknas, t.ex. saknade telefonnummer i en persondatabas. Denna situation är så pass vanlig att många språk har speciellt stöd för saknande värden.

I Java\footnote{Scala har också \code{null} men det behövs bara vid samverkan med Java-kod.} används värdet \code{null} för att indikera att en referens saknar värde. Man får då komma ihåg att testa om värdet saknas varje gång sådana värden ska behandlas, t.ex. med \code+if (ref != null) { ...} else { ... }+. Ett annat vanligt trick är att låta \code{-1} indikera saknade positiva heltal, till exempel saknade index, som får behandlas med \code+if (i != -1) { ...} else { ... }+.

I Scala finns en speciell typ \code{Option} som möjliggör smidig och typsäker hantering av saknade värden. Om ett kanske saknat värde packas in i en \code{Option} \Eng{wrapped in an Option}, finns det i en speciell slags samling som bara kan innehålla \emph{inget} eller \emph{något} värde, och alltså har antingen storleken \code{0} eller \code{1}.

\Subtask Förklara vad som händer nedan.
\begin{REPL}
scala> var kanske: Option[Int] = None
scala> kanske.size
scala> kanske = Some(42)
scala> kanske.size
scala> kanske.isEmpty
scala> kanske.isDefined
scala> def ökaOmFinns(opt: Option[Int]): Option[Int] = opt match
         case Some(i) => Some(i + 1)
         case None    => None
scala> val annanKanske = ökaOmFinns(kanske)
scala> def öka(i: Int) = i + 1
scala> val merKanske = kanske.map(öka)
\end{REPL}

\Subtask Mönstermatchingen ovan är minst lika knölig som en \code{if}-sats, men tack vare att en \code{Option} är en slags (liten) samling finns det smidigare sätt. Förklara vad som händer nedan.
\begin{REPL}
val meningen = Some(42)
val ejMeningen = Option.empty[Int]
meningen.map(_ + 1)
ejMeningen.map(_ + 1)
ejMeningen.map(_ + 1).orElse(Some("saknas")).foreach(println)
meningen.map(_ + 1).orElse(Some("saknas")).foreach(println)
\end{REPL}

\Subtask \emph{Samlingsmetoder som ger en \code{Option}.} Förklara för varje rad nedan vad som händer. En av raderna ger ett felmeddelande; vilken rad och vilket felmeddelande?
\begin{REPL}
val xs = (42 to 84 by 5).toVector
val e = Vector.empty[Int]
xs.headOption
xs.headOption.get
xs.headOption.getOrElse(0)
xs.headOption.orElse(Some(0))
e.headOption
e.headOption.get
e.headOption.getOrElse(0)
e.headOption.orElse(Some(0))
Vector(xs, e, e, e)
Vector(xs, e, e, e).map(_.lastOption)
Vector(xs, e, e, e).map(_.lastOption).flatten
xs.lift(0)
xs.lift(1000)
e.lift(1000).getOrElse(0)
xs.find(_ > 50)
xs.find(_ < 42)
e.find(_ > 42).foreach(_ => println("HITTAT!"))
\end{REPL}

\Subtask Vilka är fördelerna med \code{Option} jämfört med \code{null} eller \code{-1} om man i sin kod glömmer hantera saknade värden?

\SOLUTION


\TaskSolved \what


\SubtaskSolved  \begin{enumerate}
\item \code{var kanske} blir en \code{Option} som håller \code{Int} men är utan något värde, kallas då \code{None}.
\item Eftersom \code{var kanske} är utan värde är storleken av den 0.
\item \code{var kanske} tilldelas värdet 42 som förvaras i en \code{Some} som visar att värde finns.
\item Eftersom \code{var kanske} nu innehåller ett värde är storleken 1.
\item Eftersom \code{var kanske} innehåller ett värde är den inte tom.
\item Eftersom \code{var kanske} innehåller ett värde är den definierad.
\item \code{def ökaOmFinns} matchar en \code{Option[Int]} med dess olika fall.\\
Finns ett värde, alltså \code{opt: Option[Int]} är en \code{Some}, så returneras en \code{Some} med ursprungliga värdet plus 1.\\
Finns inget värde, alltså \code{opt: Option[Int]} är en \code{None}, så returneras en \code{None}.
\item -
\item -
\item -
\item \code{def ökaOmFinns} appliceras på \code{kanske} och returnerar en \code{Some} med värdet hos \code{kanske} plus 1, alltså 43.
\item \code{def öka} tar emot värdet av en \code{Int} och returnerar värdet av denna plus 1.
\item \code{map} applicerar \code{def öka} till det enda elementen i \code{kanske}, 42. Denna funktion returnerar en \code{Some} med värdet 43 som tilldelas \code{merKanske}.
\end{enumerate}

\SubtaskSolved  \begin{enumerate}
\item \code{val meningen} blir en \code{Some} med värdet 42.
\item \code{val ejMeningen} blir en \code{Option[Int]} utan något värde, en \code{None}.
\item \code{map(_ + 1)} appliceras på \code{meningen} och ökar det existerande värdet med 1 till 43.
\item \code{map(_ + 1)} appliceras på \code{ejMening} men eftersom inget värde existerar fortsätter denna vara \code{None}.
\item \code{map(_ + 1)} appliceras ännu en gång på \code{ejMening} men denna gång inkluderas metoden \code{orElse}. Om ett värde inte existerar hos en \code{Option}, alltså är av typen \code{None}, så utförs koden i \code{orElse}-metoden som i detta fall skriver ut \textit{saknas} för värdet som saknas.
\item Samma anrop från föregående rad utförs denna gång på \code{meningen} och eftersom ett värde finns utförs endast första biten som ökar detta värde med 1.
\end{enumerate}
Denna metod kan användas i stället för \code{match}-versionen i föregående exempel i och med dennas simplare form. En \code{Option} innehåller ju antingen ett värde eller inte så ett längre \code{match}-uttryck är inte nödvändigt.

\SubtaskSolved \begin{enumerate}
\item En vektor \code{xs} skapas med var femte tal från 42 till 82.
\item En tom \code{Int}-vektor \code{e} skapas.
\item \code{headOption} tar ut första värdet av vektorn \code{xs} och returnerar den sparad i en \code{Option}, \code{Some(42)}.
\item Första värdet i vektorn \code{xs} sparas i en \code{Option} och hämtas sedan av \code{get}-metoden, 42.
\item Som i föregående rad men denna gång används \code{getOrElse} som om den \code{Option} som returneras saknar ett värde, alltså är av typen \code{None}, returnerar 0 istället.\\
 Eftersom \code{xs} har minst ett värde så är den \code{Option} som returneras inte \code{None} och ger samma värde som i föregående, 42.
\item Som föregående rad fast istället för att returnera 0 om värde saknas så returneras en \code{Option[Int]} med 0 som värde.
\item \code{headOption} försöker ta ut första värdet av vektorn \code{e} men eftersom denna saknar värden returneras en \code{None}.
\item \begin{REPL}
java.util.NoSuchElementException: None.get
\end{REPL}
Liksom föregående rad returnerar \code{headOption} på den tomma vektorn \code{e} en \code{None}. När  \code{get}-metoden försöker hämta ett värde från en \code{None} som saknar värde ger detta upphov till ett körtidsfel.
\item Liksom i föregående returneras \code{None}  av \code{headOption} men eftersom \code{getOrElse}-metoden används på denna \code{None} returneras 0 istället.
\item Liksom föregående används \code{getOrElse}-metoden på den \code{None} som returneras. Denna gång returneras dock en \code{Option[Int]} som håller värdet 0.
\item En vektor innehållandes elementen \code{xs}-vektorn och 3 \code{e}-vektorer skapas.
\item \code{map} använder metoden \code{lastOption} på varje delvektor från vektorn på föregående rad. Detta sammanställer de sista elementen från varje delvektor i en ny vektor. Eftersom vektor \code{e} är tom returneras \code{None} som element från denna.
\item Samma sker som i föregående rad men \code{flatten}-metoden appliceras på slutgiltiga vektorn som rensar vektorn på \code{None} och lämnar endast faktiska värden.
\item \code{lift}-metoden hämtar det eventuella värdet på plats 0  i \code{xs} och returnerar den i en \code{Option} som blir \code{Some(42)}.
\item \code{lift}-metoden försöker hämta elementet på plats 1000 i \code{xs}, eftersom detta inte existerar returneras \code{None}.
\item  Samma sker som i föregående fast applicerat på vektorn \code{e}. Sedan appliceras \code{getOrElse(0)} som, eftersom \code{lift}-metoden returnerar \code{None}, i sin tur returnerar 0.
\item \code{find}-metoden anropas på \code{xs}-vektorn. Den letar upp första talet över 50 och returnerar detta värde i en \code{Option[Int]}, alltså \code{Some(52)}.
\item \code{find}-metoden anropas på \code{xs}-vektorn. Den letar upp första värdet under 42 men eftersom inget värde existerar under 42 i \code{xs} returneras \code{None} istället.
\item \code{find}-metoden anropas på \code{e}-vektorn och skriver ut \textit{HITTAT!} om ett element under 42 hittas. Eftersom \code{e}-vektorn är tom returneras \code{None} vilket \code{foreach} inte räknar som element och därav inte utförs på.
\end{enumerate}

\SubtaskSolved  Användning av -1 som returvärde vid fel eller avsaknad på värde kan ge upphov till körtidsfel som är svåra att upptäcka. \jcode{null} kan i sin tur orsaka kraschar om det skulle bli fel under körningen. \code{Option} har inte samma problem som dessa, används ett \code{getOrElse}-uttryck eller dylikt så kraschar inte heller programmet.\\
Dessutom behöver inte en funktion som returnerar en \code{Option} samma dokumentation av returvärdena. Istället för att skriva kommentarer till koden på vilka värden som kan returneras och vad dessa betyder så syns det direkt i koden.\\
Slutgiltligen är \code{Option} mer typsäkert än \code{null}. När du returnerar en \code{Option} så specificeras typen av det värde som den kommer innehålla, om den innehåller något, vilket underlättar att förstå och begränsar vad den kan returnera.



\QUESTEND






\WHAT{Kasta undantag.}

\QUESTBEGIN

\Task  \what~  Om man vill signalera att ett fel eller en onormal situtation uppstått så kan man \textbf{kasta} \Eng{throw} ett \textbf{undantag} \Eng{exception}. Då avbryts programmet direkt med ett felmeddelande, om man inte väljer att \textbf{fånga} \Eng{catch} undantaget.
\Subtask Vad händer nedan?
\begin{REPL}
scala> throw new Exception("PANG!")
scala> java.lang.   // Tryck TAB efter punkten
scala> throw new IllegalArgumentException("fel fel fel")
scala> val carola = 
         try 
           throw new Exception("stormvind!")
           42
         catch 
           case e: Throwable => 
             println("Fångad av en " + e)
             -1
\end{REPL}
\Subtask Nämn ett par undantag som finns i paketet \code{java.lang} som du kan gissa vad de innebär och i vilka situationer de kastas.

\Subtask Vilken typ har variabeln \code{carola} ovan? Vad hade typen blivit om catch-grenen hade returnerat en sträng i stället?

\SOLUTION


\TaskSolved \what


\SubtaskSolved  \begin{enumerate}
\item Ett \code{Exception} kastas med felmeddelandet \textit{PANG!}.
\item Flera olika typer av \code{Exception} visas.
\item En typ av \code{Exception}, \code{IllegalArgumentException}, kastas med felmeddelandet \textit{fel fel fel}.
\item Ett undantag med felmeddelandet \code{stormvind!} kastas och fångas av \code{catch}-uttrycket. Ett \code{match}-uttryck undersöker undantaget och skriver ut meddelandet, samt returnerar -1.
\end{enumerate}

\SubtaskSolved  Exempelvis: \\
\code{OutOfMemoryError}, om programmet får slut på minne.\\
\code{IndexOutOfBoundsException}, om en vektorposition som är större än vad som finns hos vektorn försöker nås.\\
\code{NullPointerException}, om en metod eller dylikt försöker användas hos ett objekt som inte finns och därav är en nullreferens.

\SubtaskSolved  om både try-grenen och catch-grenen har samma typ, här \code{Int}, så härleder kompilatorn samma typ för hela uttrycket. 
Skulle \code{catch}-grenen returnera ett värde av en helt annan typ istället, t.ex. \code{String}, så blir den mest precisa typen som kompilatorn kan härleda för hela uttrycket \code{Matchable}, som är en direkt subtyp till den mest generella typen \code{Any}.



\QUESTEND










\WHAT{Fånga undantantag med \code{scala.util.Try}.}

\QUESTBEGIN

\Task  \what~  I paketet \code{scala.util} finns typen \code{Try} med stort T som är som en slags samling som kan innehålla antingen ett ''lyckat'' eller ''misslyckat'' värde. Om beräkningen av värdet lyckades och inga undantag kastas blir värdet inkapslat i en \code{Success}, annars blir undantaget inkapslat i en \code{Failure}. Man kan extrahera värdet, respektive undantaget, med mönstermatchning, men det är oftast smidigare att använda samlingsmetoderna \code{map} och \code{foreach}, i likhet med hur \code{Option} används. Det finns även en smidig metod \code{recover} på objekt av typen \code{Try} där man kan skicka med kod som körs om det uppstår en undantagssituation.

\Subtask Förklara vad som händer nedan.
\begin{REPL}
scala> def pang = throw new Exception("PANG!")
scala> import scala.util.{Try, Success, Failure}
scala> Try{pang}
scala> Try{pang}.recover{case e: Throwable =>   "desarmerad bomb: " + e}
scala> Try{"tyst"}.recover{case e: Throwable => "desarmerad bomb: " + e}
scala> def kanskePang = if math.random() > 0.5 then "tyst" else pang
scala> def kanskeOk = Try{kanskePang}
scala> val xs = Vector.fill(100)(kanskeOk)
scala> xs(13) match
         case Success(x) => ":)"
         case Failure(e) => ":( " + e
scala> xs(13).isSuccess
scala> xs(13).isFailure
scala> xs.count(_.isFailure)
scala> xs.find(_.isFailure)
scala> val badOpt = xs.find(_.isFailure)
scala> val goodOpt = xs.find(_.isSuccess)
scala> badOpt
scala> badOpt.get
scala> badOpt.get.get
scala> badOpt.map(_.getOrElse("bomben desarmerad!")).get
scala> goodOpt.map(_.getOrElse("bomben desarmerad!")).get
scala> xs.map(_.getOrElse("bomben desarmerad!")).foreach(println)
scala> xs.map(_.toOption)
scala> xs.map(_.toOption).flatten
scala> xs.map(_.toOption).flatten.size
\end{REPL}


\Subtask Vad har funktionen \code{pang} för returtyp?

\Subtask Varför får funktionen \code{kanskePang} den härledda returtypen \code{String}?

\SOLUTION


\TaskSolved \what


\SubtaskSolved  \begin{enumerate}
\item \code{def pang} skapas som kastar ett \code{Exception} med felmeddelandet \textit{PANG!}.
\item Scalas verktyg \code{Try}, \code{Success} och \code{Failure} importeras.
\item \code{def pang} anropas i \code{Try} som fångar undantaget och kapslar in den i en \code{Failure}.
\item Metoden \code{recover} matchar undantaget i \code{Failure} från föregående rad med ett \code{case} och gör om föredetta \code{Failure} till \code{Success} vid matchning, liknande \code{catch}.
\item Strängen \textit{tyst} körs i föregående test men eftersom inget undantag kastas blir den inkapslad i en \code{Success} och \code{recover} behöver inte göra något. Den tar endast hand om undantag.
\item \code{def kanskePang} skapas som har lika stor chans att returnera strängen \textit{tyst} såsom anropa \code{def pang}.
\item \code{def kanskeOk} skapas som testar \code{def kanskePang} med \code{Try}.
\item En vektor \code{xs} fylls med resultaten, \code{Success} och \code{Failure}, från 100 körningar av \code{kanskeOk}.
\item Elementet på plats 13 i vektor \code{xs} matchas med något av 2 \code{case}. Om det är en \code{Success} skrivs \textit{:)} ut, om en \code{Failure} skrivs \textit{:(} plus felmeddelandet ut.
\item -
\item -
\item Metoden \code{isSuccess} testar om elementet på plats 13 i \code{xs} är en \code{Success} och returnerar \code{true} om så är fallet.
\item Metoden \code{isFailure} testar om elementet på plats 13 i \code{xs} är en \code{Failure} och returnerar \code{true} om så är fallet.
\item Metoden \code{count} räknar med hjälp av \code{isFailure} hur många av elementen i \code{xs} som är \code{Failure} och returnerar detta tal.
\item Metoden \code{find} letar upp med hjälp av \code{isFailure} ett element i \code{xs} som är \code{Failure} och returnerar denna i en \code{Option}.
\item \code{badOpt} tilldelas den första \code{Failure} som hittas i \code{xs}.
\item \code{goodOpt} tilldelas den första \code{Success} som hittas i \code{xs}.
\item Resultatet badOpt skrivs ut, \code{Option[scala.util.Try[String]] =}\\
\code{Some(Failure(java.lang.Exception: PANG!))}
\item Metoden \code{get} hämtar från \code{badOpt} den \code{Failure} som förvaras i en \code{Option}.
\item Metoden \code{get} anropas ännu en gång på resultatet från föregående rad, alltså en \code{Failure}, som hämtar undantaget från denna och som då i sin tur kastas.
\item Metoden \code{getOrElse} anropas på den \code{Failure} som finns i \code{badOpt}. Eftersom detta är en \code{Exception} utförs \code{orElse}-biten istället för att undantaget försöker hämtas. Då returneras strängen \textit{bomben desarmerad!}.
\item Metoden \code{getOrElse} anropas på den \code{Success} som finns i \code{goodOpt}. Eftersom detta är en \code{Success} med en normal sträng sparad i sig returneras denna sträng, \textit{tyst}.
\item Metoden från föregående används denna gång på alla element i \code{xs} där resultatet skrivs ut för varje.
\item Metoden \code{toOption} appliceras på alla \code{Success} och \code{Failure} i \code{xs}. De med ett exception, alltså \code{Failure}, blir en \code{None} medan de med värden i \code{Success} ger en \code{Some} med strängen \textit{tyst} i sig.
\item Metoden \code{flatten} appliceras på vektorn fylld med \code{Option} från föregående rad för att ta bort alla \code{None}-element.
\item Metoden \code{size} används på slutgiltiga listan från föregående rad för att räkna ut hur många \code{Some} som resultatet innehåller. Den har alltså beräknat antalet element i \code{xs} som var av typen \code{Success} med hjälp av \code{Option}-typen.
\end{enumerate}

\SubtaskSolved  \code{pang} har returtypen \code{Nothing}, en specialtyp inom Scala som inte är kopplad till \code{Any}, och som inte går att returnera.

\SubtaskSolved  Typen \code{Nothing} är en subtyp av varenda typ i Scalas hierarki. Detta innebär att den även är en subtyp av \code{String} vilket implicerar att \code{String} inkluderar både strängar och \code{Nothing} och därav blir returtypen.


\QUESTEND




% \WHAT{Laborationsförberedelse.}

% \QUESTBEGIN

% \Task  \what~ \label{task:labprep-patterns-tabular} På veckans laboration ska du hantera data som finns i tabeller med celler som kan bestå av decimaltal eller strängar. Studera den givna koden som du ska utgå ifrån; uppgifterna nedan berör \code{Cell.scala} och \code{Table.scala} här:
% \url{https://github.com/lunduniversity/introprog/tree/master/workspace-old/w13_tabular/src/main/scala/tabular}

% Bastypen \code{Cell} i koden nedan har två subtyper \code{Str} och \code{Num}.

% \begin{CodeSmall}
% sealed trait Cell { def value: String }
% case class Str(value: String) extends Cell
% case class Num(num: BigDecimal) extends Cell { def value = num.toString }
% \end{CodeSmall}
% \code{BigDecimal} används för att representera decimaltal med bättre precision än vanliga flyttal av typen \code{Double}.

% \Subtask Studera dokumentationen för \code{BigDecimal}: \url{https://www.scala-lang.org/api}\\
% Vad gör fabriksmetoden \code{def apply(x: String): BigDecimal} (se kompanjonsobj.).


% \Subtask Vad är fördelen med att \code{Cell} är förseglad?

% \Subtask Kör igång REPL med koden för \code{Cell}-hierarkin tillgänglig på classpath, t.ex. med \code{sbt console}. Vad ger koden nedan för resultat? Ange värde och typ för varje rad.

% \begin{REPL}
% scala> val xs = Seq[Cell](Str("!"), Num(BigDecimal("100000000.000000001")))
% scala> val ys = xs.map(_ match { case Num(n) => Some(n) case _ => None })
% scala> val b = ys.flatten.headOption.getOrElse(BigDecimal(0))
% \end{REPL}

% \Subtask Lägg till ett kompanjonsobjekt enligt nedan. Gör klart den saknade implementationen. Använd \code{Try} och matcha på \code{Success} och \code{Failure}. Testa så att alla metoder i kompanjonsobjektet fungerar.

% \Subtask Gör om implementation så att du i stället använder \code{Try} och \code{getOrElse}. Testa så att det fungerar som innan. Vilken implementation är smidigast?
% \begin{CodeSmall}
% object Cell {
%   import scala.util.{Try, Success, Failure}

%   /** Ger en Num om BigDecimal(s) lyckas annars en Str. */
%   def apply(s: String): Cell =  ???

%   def apply(i: Int): Num = Num(BigDecimal(i))

%   def empty: Str = Str("")

%   def zero: Num = Num(BigDecimal(0))
% }
% \end{CodeSmall}

% \Subtask I given kod och nedan finns en nästan färdig klass för tabelldatahantering. Implementera de saknade delarna enligt beskrivning i dokumentationskommentarerna. Testa så att dina implementationer fungerar och försök förstå hur övriga delar av \code{Table} fungerar.

% \scalainputlisting[numbers=left,basicstyle=\ttfamily\fontsize{9}{11.5}\selectfont]{../workspace-old/w13_tabular/src/main/scala/tabular/Table.scala}

% \noindent Tips vid färdigställande av \code{Table}:
% \begin{itemize}[leftmargin=*]
%   \item Nyckel-värde-tabeller har en metod \code{withDefaultValue} som är smidig om man vill undvika undantag vid uppslagning med nyckel som inte finns och det i stället för undantag är möjligt/lämpligt att erbjuda ett vettigt defaultvärde.
%   \item Metoderna \code{getOrElse} och \code{toOption} på en \code{Try} är smidiga när man vill ge resultat som beror av om det är \code{Success} eller \code{Failure} utan att man behöver göra en \code{match}.
% \item Skiss på implementation av \code{load} i kompanjonsobjektet:
% \begin{CodeSmall}
% def load(fileOrUrl: String, separator: Char): Table = {
%   val source = fileOrUrl match {
%     case /* använd gard och startsWith*/ => scala.io.Source.fromURL(url)
%     case path  => scala.io.Source.fromFile(path)
%   }
%   val lines = try source.getLines.toVector finally source.close
%   val rows = ??? // kör split(separator).toVector på alla rader i lines
%   Table(rows.head, rows.tail.map(_.map(Cell.apply)), separator)
% }
% \end{CodeSmall}
% En webbadress börjar med \code{http}.
% Med \code{try sats1 finally sats2} så kan man garantera att \code{sats2} alltid görs även om \code{sats1} ger undantag. Detta används typiskt för att frigöra resurser som annars förblir allokerade vid undantag. I koden ovan används det för att undvika att filer inte stängs även om något går fel under läsningen.
% \end{itemize}
% \SOLUTION


% \TaskSolved \what

% \SubtaskSolved ''Translates the decimal String representation of a BigDecimal into a BigDecimal.''

% \SubtaskSolved Eftersom \code{Cell} är förseglad med \code{sealed} så kan inga andra subtyper finnas och vi behöver inte kolla efter andra subtyper när vi matchar. Kompilatorn varnar också om vi glömmer matcha på någon av subtyperna.

% \SubtaskSolved
% \begin{REPL}
% scala> val xs = Seq[Cell](Str("!"), Num(BigDecimal("100000000.000000001")))
% xs: Seq[Cell] = List(Str(!), Num(100000000.000000001))

% scala> val ys = xs.map(_ match { case Num(n) => Some(n) case _ => None })
% ys: Seq[Option[BigDecimal]] = List(None, Some(100000000.000000001))

% scala> val b = ys.flatten.headOption.getOrElse(BigDecimal(0))
% b: BigDecimal = 100000000.000000001
% \end{REPL}

% \SubtaskSolved
% \begin{Code}
%   def apply(s: String): Cell = Try(BigDecimal(s)) match {
%     case Success(num) => Num(num)
%     case Failure(_)   => Str(s)
%   }
% \end{Code}

% \SubtaskSolved
% \begin{Code}
%   def apply(s: String): Cell = Try(Num(BigDecimal(s))).getOrElse(Str(s))
% \end{Code}

% \SubtaskSolved \emph{Lämnas som egen laborationsförberedelse.}

% \QUESTEND


\AdvancedTasks %%%%%%%%%%%%%%%%%%%



\WHAT{Använda matchning eller dynamisk bindning?}

\QUESTBEGIN

\Task  \what~ Man kan åstadkomma urskiljningen av de ätbara grönsakerna i uppgift \ref{task:match-caseclass} med dynamisk bindning i stället för \code{match}.

\Subtask Gör en ny variant av ditt program enligt nedan riktlinjer och spara den modifierade koden i filen \texttt{vegopoly.scala} och kompilera och kör.
\begin{itemize}[noitemsep]
\item Ta bort predikatet \code{ärÄtvärd} i objektet \code{Main} och inför i stället en abstrakt metod \code{def ärÄtbar: Boolean} i traiten \code{Grönsak}.
\item Inför konkreta \code{val}-medlemmar i respektive grönsak som definierar ätbarheten.
\item Ändra i huvudprogrammet i enlighet med ovan ändringar så att \code{ärÄtvärd} anropas som en metod på de skördade grönsaksobjekten när de ätvärda ska filtreras ut.
\end{itemize}

\Subtask Lägg till en ny grönsak \code{case class Broccoli} och definiera dess ätbarhet. Ändra i slump-funktionerna så att broccoli blir ovanligare än gurka.

\Subtask Jämför lösningen med \code{match} i uppgift \ref{task:match-caseclass} och lösningen ovan med polymorfism. Vilka är för- och nackdelarna med respektive lösning? Diskutera två olika situationer på ett hypotetiskt företag som utvecklar mjukvara för jordbrukssektorn: 1) att uppsättningen grönsaker inte ändras särskilt ofta medan definitionerna av ätbarhet ändras väldigt ofta och 2) att uppsättningen grönsaker ändras väldigt ofta men att ätbarhetsdefinitionerna inte ändras särskilt ofta.



\SOLUTION


\TaskSolved \what


\SubtaskSolved
\begin{Code}
package vegopoly

trait Grönsak:
	def vikt: Int
	def ärRutten: Boolean
	def ärÄtbar: Boolean

case class Gurka(vikt: Int, ärRutten: Boolean) extends Grönsak:
  val ärÄtbar: Boolean = (!ärRutten && vikt > 100)

case class Tomat(vikt: Int, ärRutten: Boolean) extends Grönsak:
  val ärÄtbar: Boolean = (!ärRutten && vikt > 50)

object Main:
	def slumpvikt: Int = (math.random()*500 + 100).toInt

	def slumprutten: Boolean = math.random() > 0.8

	def slumpgurka: Gurka = Gurka(slumpvikt, slumprutten)

	def slumptomat: Tomat = Tomat(slumpvikt, slumprutten)

	def slumpgrönsak: Grönsak =
    if math.random() > 0.2 then slumpgurka else slumptomat

	def main(args: Array[String]): Unit = 
		val skörd = Vector.fill(args(0).toInt)(slumpgrönsak)
		val ätvärda = skörd.filter(_.ärÄtbar)
		println("Antal skördade grönsaker: " + skörd.size)
		println("Antal ätvärda grönsaker: " + ätvärda.size)
\end{Code}

\SubtaskSolved
Följande \code{case class} läggs till:
\begin{Code}
case class Broccoli(vikt: Int, ärRutten: Boolean) extends Grönsak:
  val ärÄtbar: Boolean = (!ärRutten && vikt > 80)
\end{Code}
~\\
Därefter läggs följande till i \code{object Main} innan \code{def slumpgrönsak}:

\begin{Code}
def slumpbroccoli: Broccoli = Broccoli(slumpvikt, slumprutten)
\end{Code}
~\\
Slutligen ändras \code{def slumpgrönsak} till följande:

\begin{Code}
def slumpgrönsak: Grönsak =     // välj t.ex. denna fördelning:
  val rnd = math.random()
  if rnd > 0.5 then slumpgurka      // 50% sannolikhet för gurka
  else if rnd > 0.2 then slumptomat // 30% sannolikhet för tomat
  else slumpbroccoli             // 20% sannolikhet för broccoli

\end{Code}

\SubtaskSolved  Fördelarna med \code{match}-versionen, och mönstermatchning i sig, är att det är väldigt lätt att göra ändringar på hur matchningen sker. Detta innebär att det skulle vara väldigt lätt att ändra definitionen för ätbarheten. Skulle dock dessa inte ändras ofta utan snarare grönsaksutbudet så kan det polyformistiska alternativet vara att föredra. Detta eftersom det skulle implementeras och ändras lättare än mönstermatchningen vid byte av grönsaker.



\QUESTEND





\WHAT{Metoden \code{equals}.}

\QUESTBEGIN

\Task  \what~   Om man överskuggar den befintliga metoden \code{equals} så kommer metoden \code{==} att fungera annorlunda. Man kan då själv åstadkomma innehållslikhet i stället för referenslikhet. Vi börjar att studera den befintliga equals med referenslikhet.

\Subtask \label{subtask:refequals} Vad händer nedan? Undersök parametertyp och returvärdestyp för  \code{equals}.
\begin{REPL}
scala> class Gurka(val vikt: Int, val ärÄtbar: Boolean)
scala> val g1 = new Gurka(42, true)
scala> val g2 = g1
scala> val g3 = new Gurka(42, true)
scala> g1 == g2
scala> g1 == g3
scala> g1.equals  // tryck ENTER för att se funktionstyp
\end{REPL}

\Subtask Rita minnessituationen efter rad 4.

\Subtask \emph{Överskugga metoderna \code{equals} och \code{hashCode}.}

\begin{Background}
Det visar sig förvånande komplicerat att implementera innehållslikhet med metoden \code{equals} så att den ger bra resultat under alla speciella omständigheter. Till exempel måste man även överskugga en metod vid namn \code{hashCode} om man överskuggar \code{equals}, eftersom dessa båda används gemensamt av effektivitetsskäl för att skapa den interna lagringen av objekten i vissa samlingar. Om man missar det kan objekt bli ''osynliga'' i \code{hashCode}-baserade samlingar -- men mer om detta i senare kurser. Om objekten ingår i en öppen arvshierarki blir det också mer komplicerat; det är enklare om man har att göra med finala klasser. Dessutom krävs speciella hänsyn om klassen har en typparameter.
\end{Background}

\noindent Definera klassen nedan i REPL med överskuggade \code{equals} och \code{hashCode}; den ärver inte något och är final.

\begin{Code}
// fungerar fint om klassen är final och inte ärver något
final class Gurka(val vikt: Int, val ärÄtbar: Boolean):
  override def equals(other: Any): Boolean = other match
    case that: Gurka => vikt == that.vikt && ärÄtbar == that.ärÄtbar
    case _ => false
  override def hashCode: Int = (vikt, ärÄtbar).## //förklaras sen
\end{Code}
\Subtask Vad händer nu nedan, där \code{Gurka} nu har en överskuggad \code{equals} med innehållslikhet?
\begin{REPL}
scala> val g1 = new Gurka(42, true)
scala> val g2 = g1
scala> val g3 = new Gurka(42, true)
scala> g1 == g2
scala> g1 == g3
\end{REPL}
\Subtask Hur märker man ovan att den överskuggade \code{equals} medför att \code{==} nu ger innehållslikhet? Jämför med deluppgift \ref{subtask:refequals}.

I uppgift \ref{task:equals:Complex} får du prova på att följa det fullständiga receptet i 8 steg för att överskugga en \code{equals} enligt konstens alla regler. I efterföljande kurs kommer mer träning i att hantera innehållslikhet och hash-koder. I Scala får man ett objekts hash-kod med metoden \code{##}.%
\footnote{Om du är nyfiken på hash-koder, läs mer här:
\href{https://en.wikipedia.org/wiki/Hash_function}
{en.wikipedia.org/wiki/Hash\_function}
}


\SOLUTION


\TaskSolved \what


\SubtaskSolved  \begin{enumerate}
\item En klass \code{Gurka} skapas med parametrarna \code{vikt} av typen \code{Int} och ärÄtbar av typen \code{Boolean}.
\item \code{g1} tilldelas en instans av \code{Gurka}-klassen med \code{vikt = 42} och \code{ärÄtbar = true}.
\item \code{g2} tilldelas samma \code{Gurka}-objekt som g1.
\item \code{g3} tilldelas en ny instans av \code{Gurka}-klassen med motsvarande parametrar som g1.
\item \code{==}(\code{equals})-metoden jämför g1 med g2 och returnerar \code{true}.
\item \code{==}(\code{equals})-metoden jämför g1 med g3 och returnerar \code{false}.
\item \code{def equals(x\$1: Any): Boolean}
\end{enumerate}
Som kan ses ovan är elementet som jämförs i \code{equals} av typen \code{Any}. Eftersom programmet inte känner till klassen så används \code{Any.equals} vid jämförelsen. Till skillnad från de primitiva datatyperna som vid jämförelse med \code{equals} jämför innehållslikhet, så jämförs referenslikheten hos klasser om inget annat är specificerat. \code{g1} och \code{g2} refererar till samma objekt medan \code{g3} pekar på ett eget sådant vilket innebär att \code{g1} och \code{g3} inte har referenslikhet.

\SubtaskSolved  \\
\vspace{1em}
\tikzstyle{mybox} = [draw=red, fill=blue!20, very thick,
    rectangle, rounded corners, inner sep=10pt, inner ysep=20pt]
\begin{tikzpicture}[
	font=\large\sffamily,
	varname/.style={node distance=0.2cm},
	varbox/.style={draw, node distance=0.2cm},
	objcloud/.style={cloud, cloud puffs=15.7, cloud ignores aspect, align=center, draw},
]

\node [varname] (g1var) {\texttt{g1}};
\node [varbox, right = of g1var] (g1ref) {\phantom{abc}};
\filldraw[black] (g1ref) circle (3pt) node[] (g1dot) {};
\node [objcloud, right = of g1ref, yshift=1.3cm, scale =0.8] (g1obj) {
	\texttt{\textbf{Gurka}} \\~\\ \texttt{vikt} \framebox{42} ~ \texttt{ärÄtvärd} \framebox{true}
};
\draw [arrow] (g1dot) -- (g1obj);

\node [varname, below = of g1var] (g2var) {\texttt{g2}};
\node [varbox, right = of g2var] (g2ref) {\phantom{abc}};
\filldraw[black] (g2ref) circle (3pt) node[] (g2dot) {};
\node [objcloud, right = of g2ref, yshift=-1.3cm, scale =0.8] (g2obj) {
	\texttt{\textbf{Gurka}} \\~\\ \texttt{vikt} \framebox{42} ~ \texttt{ärÄtvärd} \framebox{true}
};
\draw [arrow] (g2dot) -- (g1obj);
\node [varname, below = of g2var] (g3var) {\texttt{g3}};
\node [varbox, right = of g3var] (g3ref) {\phantom{abc}};
\filldraw[black] (g3ref) circle (3pt) node[] (g3dot) {};
\draw [arrow] (g3dot) -- (g2obj);

\end{tikzpicture}

\SubtaskSolved  -

\SubtaskSolved  I de första 3 raderna sker samma som i deluppgift \textit{a}. När nu dessa jämförelser görs mellan \code{Gurka}-objekten så överskuggas \code{Any.equals} av den \code{equals} som är specificerad för just \code{Gurka}. Eftersom båda objekten \code{g1} jämförs med också är av typen \code{Gurka} så matchar den med \code{case that: Gurka}. Denna i sin tur jämför vikterna hos de båda gurkorna och returnerar en \code{Boolean} huruvida de är lika eller inte, vilket de i båda fallen är.

\SubtaskSolved  I deluppgift a gav \code{g1 == g3 false} trots innehållslikhet. Efter skuggningen ger dock detta uttryck \code{true} vilket påvisar jämförelse av innehållslikhet.



\QUESTEND






\WHAT{Polynom.}

\QUESTBEGIN

\Task \label{task:polynomial} \what~   Med hjälp av koden nedan, kan man göra följande:
\begin{REPL}
scala> import polynomial.*

scala> Const(1) * x
res0: polynomial.Term = x

scala> (x*5)^2
res1: polynomial.Prod = 25x^2

scala> Poly(x*(-5), y^4, (z^2)*3)
res2: polynomial.Poly = -5x + y^4 + 3z^2

\end{REPL}

\Subtask Förklara vad som händer ovan genom att studera koden nedan\footnote{Koden finns även här:\\ \href{https://github.com/lunduniversity/introprog/tree/master/compendium/examples/polynomial}{github.com/lunduniversity/introprog/tree/master/compendium/examples/polynomial}}.

\scalainputlisting[numbers=left,basicstyle=\ttfamily\fontsize{10.5}{13}\selectfont]{examples/polynomial/polynomial.scala}

\Subtask Bygg vidare på \code{object polynomial} och implementera addition mellan olika termer.


\SOLUTION


\TaskSolved \what


\SubtaskSolved \TODO

\SubtaskSolved \TODO



\QUESTEND






\WHAT{\code{Option} som en samling.}

\QUESTBEGIN

\Task  \what~Studera dokumentationen för \code{Option} här och se om du känner igen några av metoderna som också finns på samlingen \code{Vector}:\\ \href{http://www.scala-lang.org/api/current/scala/Option.html}{www.scala-lang.org/api/current/scala/Option.html}
\\Förklara hur metoden \code{contains} på en \code{Option} fungerar med hjälp av dokumentationens exempel.



\SOLUTION


\TaskSolved \what 

Exempel på metoder som finns både för \code{Vector} och \{Option}:
\code{foreach}, \code{filter}, \code{fold} etc.

Contains returnerar en \code{Boolean} som visar om den har ett värde eller ej.


\QUESTEND






\WHAT{Fånga undantag med \code{catch} i Java och Scala.}

\QUESTBEGIN

\Task  \what~ Gör motsvarande program i Scala som visas i uppgift \ref{task:javatry}, men utnyttja att Scalas \code{try}-\code{catch} är ett uttryck. Kompilera och kör och testa så att de ur användarens synvinkel fungerar precis på samma sätt. Notera de viktigaste skillnaderna mellan de båda programmen.


\SOLUTION


\TaskSolved \what \TODO


\QUESTEND



\WHAT{Polynom, fortsättning: reducering.}

\QUESTBEGIN

\Task  \what~ Bygg vidare på \code{object polynomial} i uppgift \ref{task:polynomial} på sidan \pageref{task:polynomial} och implementera metoden \code{def reduce: Poly} i case-klassen \code{Poly} som förenklar polynom om flera \code{Prod}-termer kan adderas.

\SOLUTION


\TaskSolved \what



\QUESTEND




% \WHAT{Hash-koder.}

% \QUESTBEGIN

% \Task  \what~ Läs om hash-funktioner här: \href{https://en.wikipedia.org/wiki/Hash_function}{en.wikipedia.org/wiki/Hash_function} \\
% Vad ger metoden \code{##} i scala.Any för resultat? Läs dokumentationen här: \\ \href{http://www.scala-lang.org/api/current/scala/Any.html}{www.scala-lang.org/api/current/scala/Any.html}

% \SOLUTION

% \TaskSolved \what I Scala får man ett objekts hash-kod med metoden \code{##}.

% \QUESTEND






\WHAT{Typsäker innehållstest med metoden \code{===}.}

\QUESTBEGIN

\Task  \what~  Metoderna \code{equals} och \code{==} tillåter jämförelse med vad som helst. Ibland vill man ha en typsäker innehållsjämförelse som bara tillåter jämförelse av objekt av en mer specifik typ och ger kompileringsfel annars. Man brukar då definiera en metod \code{===} som har en parameter \code{that} som har en så specifik typ som önskas. Inför nedan abstrakta metod \code{===} i traiten \code{polynomial.Term} i uppgift \ref{task:polynomial} på sidan \pageref{task:polynomial} och överskugga den sedan i alla subklasser till Term. Testa så att du får kompileringsfel om du försöker jämföra en \code{Term} med något helt annat, t.ex. en \code{String} eller \code{Vector}.
\begin{Code}
  def ===(that: Term): Boolean
\end{Code}


\SOLUTION


\TaskSolved \what



\QUESTEND






\WHAT{Överskugga \code{equals} med innehållslikhet även för icke-finala klasser.}

\QUESTBEGIN

\Task \label{task:equals:Complex} \what~   Nedan visas delar av klassen \code{Complex} som representerar ett komplext tal med realdel och imaginärdel. I stället för att, som man ofta gör i Scala, använda en case-klass och en \code{equals}-metod som automatiskt ger innehållslikhet, ska du träna på att implementera en egen \code{equals}.
\begin{Code}
class Complex(val re: Double, val im: Double):
  def abs: Double = math.hypot(re, im)
  override def toString = s"Complex($re, $im)"
  def canEqual(other: Any): Boolean = ???
  override def hashCode: Int  = ???
  override def equals(other: Any): Boolean = ???

case object Complex:
  def apply(re: Double, im: Double): Complex = new Complex(re, im)
\end{Code}
Följ detta \textbf{recept}\footnote{Detta recept bygger på \url{http://www.artima.com/pins1ed/object-equality.html}} i 8 steg för att överskugga \code{equals} med innehållslikhet som fungerar även för klasser som inte är \code{final}:

\begin{enumerate}[leftmargin=*]
\item Inför denna metod: \code{ def canEqual(other: Any): Boolean}\\Observera att typen på parametern ska vara \code{Any}. Om detta görs i en subklass till en klass som redan implementerat \code{canEqual}, behövs även \code{override}.

\item Metoden \code{canEqual} ska ge \code{true} om \code{other} är av samma typ som \code{this}, alltså till exempel: \\
\code{def canEqual(other: Any): Boolean = other.isInstanceOf[Complex]}

\item Inför metoden \code{equals} och var noga med att parametern har typen \code{Any}: \\ \code{override def equals(other: Any): Boolean}

\item Implementera metoden \code{equals} med ett match-uttryck som börjar så här: \\
%\code|other match { ... } |
\code|other match |

\item Match-uttrycket ska ha två grenar. Den första grenen ska ha ett typat mönster för den klass som ska jämföras: \\ \code{  case that: Complex =>}

\item Om du implementerar \code{equals} i den klass som inför \code{canEqual}, börja uttrycket med: \\ \code{(that canEqual this) &&} \\
och skapa därefter en fortsättning som baseras på innehållet i klassen, till exempel: \code{this.re == that.re && this.im == that.im} \\
Om du överskuggar en \textit{annan} equals än den standard-equals som finns i \code{AnyRef}, vill du förmodligen börja det logiska uttrycket med att anropa superklassens equals-metod:
 \code{super.equals(that) && } men du får fundera noga på vad likhet av underklasser egentligen ska innebära i ditt speciella fall.

\item Den andra grenen i matchningen ska vara:
\code{case _ => false}

\item Överskugga \code{hashCode}, till exempel genom att göra en tupel av innehållet i klassen och anropa metoden \code{##} på tupeln så får du i en bra hashcode: \\
\code{override def hashCode: Int  = (re, im).## }

\end{enumerate}


\SOLUTION


\TaskSolved \what



\QUESTEND






\WHAT{Överskugga equals vid arv.}

\QUESTBEGIN

\Task  \what~ Bygg vidare på exemplet nedan och överskugga equals vid arv, genom att följa receptet i uppgift \ref{task:equals:Complex}.
\begin{Code}
trait Number:
  override def equals(other: Any): Boolean = ???

class Complex(re: Double, im: Double) extends Number:
  override def equals(other: Any): Boolean = ???

class Rational(numerator: Int, denominator: Int) extends Number:
  override def equals(other: Any): Boolean = ???
\end{Code}


\SOLUTION


\TaskSolved \what



\QUESTEND






\WHAT{Speciella matchningar.}

\QUESTBEGIN

\Task  \what~ Läs om användning av speciella matchningar här: \\
\href{https://dotty.epfl.ch/docs/reference/changed-features/vararg-splices.html}{dotty.epfl.ch/docs/reference/changed-features/vararg-splices.html}

\Subtask Prova variabelbinding med \texttt{@} i en matchning i REPL.

\Subtask Prova sekvensmönster med \texttt{\_} och \texttt{\_*} i en matching i REPL.

\SOLUTION


\TaskSolved \what \TODO



\QUESTEND






\WHAT{Extraktorer.}

\QUESTBEGIN

\Task \label{task:extractor} \what~  Läs mer om extraktorer här: \\ \href{https://dotty.epfl.ch/docs/reference/changed-features/pattern-matching.html}{dotty.epfl.ch/docs/reference/changed-features/pattern-matching.html} \\
Skapa ditt eget extraktor-objekt för http-addresser som i t.ex.: \\
\texttt{http://my.host.domain/path/to/this} \\ extraherar \texttt{my.host.domain} och \texttt{path/to/this} med metoden \texttt{unapply} och testa i en matchning.

%\Task \TODO \emph{flatten och flatMap med Option och Try}
%Ska detta vara ordinarie uppgift eller fördjupning???


%\Task \TODO \emph{partiella funktioner och metoderna collect och collectFirst på samlingar}
%Ska detta vara ordinarie uppgift eller fördjupning???

\SOLUTION


\TaskSolved \what \TODO



\QUESTEND




\WHAT{Polynom, fortsättning: polynomdivision.}

\QUESTBEGIN

\Task  \what~ Implementera polynomdivision på lämpligt sätt genom att bygga vidare på  \code{object polynomial} i  uppgift \ref{task:polynomial} på sidan \pageref{task:polynomial}.  \\ Läs mer om polynomdivision här: \href{https://sv.wikipedia.org/wiki/Polynomdivision}{sv.wikipedia.org/wiki/Polynomdivision}

\SOLUTION


\TaskSolved \what \TODO

\QUESTEND


%!TEX encoding = UTF-8 Unicode
%!TEX root = ../exercises.tex

\ifPreSolution


\Exercise{\ExeWeekSEVEN}\label{exe:W07}

\begin{Goals}
%!TEX encoding = UTF-8 Unicode
%!TEX root = ../compendium2.tex

\item Kunna läsa och skriva pseudokod för sekvensalgoritmer och implementera sekvensalgoritmer enligt pseudokod.

\item Kunna implementera sekvensalgoritmer, både genom kopiering till ny sekvens och genom förändring på plats i befintlig sekvens.

\item Kunna använda inbyggda metoder för uppdatering av, linjärsökning i, och sortering av sekvenssamlingar.

\item Kunna beskriva skillnaden i användningen av föränderliga och oföränderliga sekvenser, speciellt vid uppdatering.

\item Förstå hur sorteringsordningen är definierad för strängar.

\item Kunna sortera sekvenssamlingar innehållande objekt av grundtyper med hjälp av inbyggda och egendefinierade sorteringsordningar med metoderna \code{sorted}, \code{sortBy} och \code{sortWith}.

\item Kunna implementera linjärsökning enligt olika sökkriterier.


\item Kunna beskriva egenskaperna hos sekvenssamlingarna \code{Vector}, \code{List}, \code{Array}, \code{ArrayBuffer} och \code{ListBuffer}.

\item Förstå bieffekter av uppdatering av delade referenser till föränderliga element.

\item Kunna använda funktioner med repeterade parametrar.

\item Känna till hur man implementerar funktioner med repeterade parametrar.

\item Kunna implementera heltalsregistrering i en heltalsarray.

%\item Kunna beskriva skillnader i syntax mellan arrayer i Scala och Java.

%\item Kunna beskriva skillnader i syntax och semantik mellan enkla for-satser i Scala och Java.


%\item Känna till hur klassen \code{java.util.Scanner} kan användas för att skapa heltalssekvenser ur strängsekvenser.

\end{Goals}

\begin{Preparations}
\item \StudyTheory{07}
\end{Preparations}

\else

\ExerciseSolution{\ExeWeekSEVEN}

\fi


\BasicTasks %%%%%%%%%%%



\WHAT{Para ihop begrepp med beskrivning.}

\QUESTBEGIN

\Task \what

\vspace{1em}\noindent Koppla varje begrepp med den (förenklade) beskrivning som passar bäst:

\begin{ConceptConnections}
  mängd & 1 & & A & egenskapen att finnas kvar efter programmets avslut \\ 
  nyckel-värde-tabell & 2 & & B & unika identifierare, associerade med ett enda värde \\ 
  nyckelmängd & 3 & & C & unika element, kan snabbt se om element finns \\ 
  persistens & 4 & & D & koda objekt till avkodningsbar sekvens av symboler \\ 
  serialisera & 5 & & E & för att snabbt hitta tillhörande värde \\ 
  de-serialisera & 6 & & F & avkoda symbolsekvens och återskapa objekt i minnet \\ 
\end{ConceptConnections}

\SOLUTION

\TaskSolved \what

\begin{ConceptConnections}
  mängd & 1 & ~~\Large$\leadsto$~~ &  C & unika element, kan snabbt se om element finns \\ 
  nyckel-värde-tabell & 2 & ~~\Large$\leadsto$~~ &  E & för att snabbt hitta tillhörande värde \\ 
  nyckelmängd & 3 & ~~\Large$\leadsto$~~ &  B & unika identifierare, associerade med ett enda värde \\ 
  persistens & 4 & ~~\Large$\leadsto$~~ &  A & egenskapen att finnas kvar efter programmets avslut \\ 
  serialisera & 5 & ~~\Large$\leadsto$~~ &  D & koda objekt till avkodningsbar sekvens av symboler \\ 
  de-serialisera & 6 & ~~\Large$\leadsto$~~ &  F & avkoda symbolsekvens och återskapa objekt i minnet \\ 
\end{ConceptConnections}

\QUESTEND



\WHAT{Olika sekvenssamlingar.}

\QUESTBEGIN

\Task \what~Koppla varje sekvenssamling med den (förenklade) beskrivning som passar bäst:

\begin{ConceptConnections}
\input{generated/quiz-w07-seq-collections-taskrows-generated.tex}
\end{ConceptConnections}

\SOLUTION

\TaskSolved \what

\begin{ConceptConnections}
\input{generated/quiz-w07-seq-collections-solurows-generated.tex}
\end{ConceptConnections}

\QUESTEND



% This task has been removed because it didn't make much sense anymore after the removal of Traversable in Scala 2.13. https://github.com/lunduniversity/introprog/issues/497
%
%\WHAT{Typer i hierarkin av sekvenssamlingar.}
%
%\QUESTBEGIN
%
%\Task \what~Koppla varje typ i hierarkin av sekvenssamling %med den (förenklade) beskrivning som passar bäst:
%
%\begin{ConceptConnections}
%\input{generated/quiz-w07-abstract-collections-taskrows-generated.tex}
%\end{ConceptConnections}
%
%\SOLUTION
%
%\TaskSolved \what
%
%\begin{ConceptConnections}
%\input{generated/quiz-w07-abstract-collections-solurows-generated.tex}
%\end{ConceptConnections}
%
%\QUESTEND


\WHAT{Använda sekvenssamlingar.}

\QUESTBEGIN

\Task \what~Antag att nedan variabler finns synliga i aktuell namnrymd:
\begin{Code}
val xs: Vector[Int] = Vector(1, 2, 3)
val x: Int = 0
\end{Code}

\Subtask Koppla varje uttryck till vänster med motsvarande resultat till höger. Om du är osäker på resultatet, läs i snabbreferensen och testa i REPL. \\\emph{Tips: ''colon on the collection side''}.

\begin{ConceptConnections}
  \code|x +: xs         | & 1 & & A & \code|true                                    | \\ 
  \code|xs +: x         | & 2 & & B & \code|Vector(2, 2, 3)                         | \\ 
  \code|xs :+ x         | & 3 & & C & \code|1                                       | \\ 
  \code|xs ++ xs        | & 4 & & D & \code|value tail is not a member of Int       | \\ 
  \code|xs.indices      | & 5 & & E & \code|Range 0 until 3                         | \\ 
  \code|xs apply 0      | & 6 & & F & \code|Vector(1, 2, 3)                         | \\ 
  \code|xs(3)           | & 7 & & G & \code|Vector(0, 1, 2, 3)                      | \\ 
  \code|xs.length       | & 8 & & H & \code|false                                   | \\ 
  \code|xs.take(4)      | & 9 & & I & \code|java.lang.IndexOutOfBoundsException     | \\ 
  \code|xs.drop(2)      | & 10 & & J & \code|Vector(1, 2, 3, 0)                      | \\ 
  \code|xs.updated(0, 2)| & 11 & & K & \code|Vector(3)                               | \\ 
  \code|xs.tail.head    | & 12 & & L & \code|value +: is not a member of Int         | \\ 
  \code|xs.head.tail    | & 13 & & M & \code|Vector(1, 2, 3, 1, 2, 3)                | \\ 
  \code|xs.isEmpty      | & 14 & & N & \code|2                                       | \\ 
  \code|xs.nonEmpty     | & 15 & & O & \code|3                                       | \\ 
\end{ConceptConnections}

\Subtask Vid tre tillfällen blir det fel. Varför? Är det kompileringsfel eller exekveringsfel?

\begin{framed}
\noindent\emph{Tips inför fortsättningen:}
Scalas standardbibliotek har många användbara samlingar med enhetlig metoduppsättning. Om du lär dig de viktigaste samlingsmetoderna får du en kraftfull verktygslåda. Läs mer här:

    \begin{itemize}%[nolistsep]
      \item snabbreferensen (enda tentahjälpmedel): \\{\small\url{http://cs.lth.se/pgk/quickref}}
      \item översikt (av Prof. Martin Odersky, uppfinnare av Scala, m.fl.): \\
       {\small\url{http://docs.scala-lang.org/overviews/collections/introduction.html}}
      \item api-dokumentation:\\  {\small\url{https://www.scala-lang.org/api/current/scala/collection/}}
    \end{itemize}
\end{framed}

\SOLUTION

\TaskSolved \what

\SubtaskSolved

\begin{ConceptConnections}
  \code|x +: xs         | & 1 & ~~\Large$\leadsto$~~ &  G & \code|Vector(0, 1, 2, 3)                      | \\ 
  \code|xs +: x         | & 2 & ~~\Large$\leadsto$~~ &  L & \code|value +: is not a member of Int         | \\ 
  \code|xs :+ x         | & 3 & ~~\Large$\leadsto$~~ &  J & \code|Vector(1, 2, 3, 0)                      | \\ 
  \code|xs ++ xs        | & 4 & ~~\Large$\leadsto$~~ &  M & \code|Vector(1, 2, 3, 1, 2, 3)                | \\ 
  \code|xs.indices      | & 5 & ~~\Large$\leadsto$~~ &  E & \code|Range 0 until 3                         | \\ 
  \code|xs apply 0      | & 6 & ~~\Large$\leadsto$~~ &  C & \code|1                                       | \\ 
  \code|xs(3)           | & 7 & ~~\Large$\leadsto$~~ &  I & \code|java.lang.IndexOutOfBoundsException     | \\ 
  \code|xs.length       | & 8 & ~~\Large$\leadsto$~~ &  O & \code|3                                       | \\ 
  \code|xs.take(4)      | & 9 & ~~\Large$\leadsto$~~ &  F & \code|Vector(1, 2, 3)                         | \\ 
  \code|xs.drop(2)      | & 10 & ~~\Large$\leadsto$~~ &  K & \code|Vector(3)                               | \\ 
  \code|xs.updated(0, 2)| & 11 & ~~\Large$\leadsto$~~ &  B & \code|Vector(2, 2, 3)                         | \\ 
  \code|xs.tail.head    | & 12 & ~~\Large$\leadsto$~~ &  N & \code|2                                       | \\ 
  \code|xs.head.tail    | & 13 & ~~\Large$\leadsto$~~ &  D & \code|value tail is not a member of Int       | \\ 
  \code|xs.isEmpty      | & 14 & ~~\Large$\leadsto$~~ &  H & \code|false                                   | \\ 
  \code|xs.nonEmpty     | & 15 & ~~\Large$\leadsto$~~ &  A & \code|true                                    | \\ 
\end{ConceptConnections}

\SubtaskSolved

\noindent\renewcommand*{\arraystretch}{1.2}\begin{tabular}{p{5cm} l p{6cm}}

~\\ \emph{fel} & \emph{typ} & \emph{förklaring} \\\hline

\code|value +: is not| \code|a member of Int|
& kompileringsfel
& Operatorer som slutar med kolon är högerassociativa. Metodanropet \code|xs +: x| motsvarar med punktnotation \code|x.+:(xs)| och det finns ingen metod med namnet \code|+:| på heltal.\\\hline

\code|IndexOutOfBoundsException|
& körtidsfel & Det finns bara 3 element och index räknas från 0 i sekvenssamlingar.\\\hline

\code|value tail is not| \code|a member of Int|
& kompileringsfel
& Metoden \code|head| ger första elementet och heltal saknar sekvenssamlingsmetoden \code|tail|.\\\hline

\end{tabular}


\QUESTEND


\WHAT{Kopiering av sekvenser.}

\QUESTBEGIN

\Task \what~ %\code{map} \code{toArray} \code{copyToArray}
Klassen \code{Mutant} nedan kan användas för att skapa förändringsbara instanser med heltal.\footnote{Om den inbyggda grundtypen Int, i likhet med \code{Mutant}, knasigt nog  kunnat användas för att skapa förändringsbara instanser hade heltalsmatematiken i Scala omvandlats till ett skrämmande kaos.
%\\Lär mer om fem här: \url{https://www.youtube.com/watch?v=dpdOUEe9mm4}
}

\noindent\begin{minipage}{0.6\textwidth}
\begin{Code}[basicstyle=\ttfamily\large\selectfont]
class Mutant(var int: Int = 0)
\end{Code}
\end{minipage}
\hfill\begin{minipage}{0.38\textwidth}
%https://www.1001freedownloads.com/free-clipart/mutant
\centering\includegraphics[width=3.4cm]{../img/mutant.png}
\captionof{figure}{En instans av klassen Mutant där \code{int} kanske är 5.}
%https://tex.stackexchange.com/questions/55337/how-to-use-figure-inside-a-minipage
\end{minipage}

\vspace{1em}\noindent Kör nedan i REPL efter studier av detta:  \url{https://youtu.be/dpdOUEe9mm4}
\begin{REPL}
scala> val fem = new Mutant(5)
scala> val xs = Vector(fem, fem, fem)
scala> val ys = xs.toArray    // kopierar referenserna till ny Array
scala> val zs = xs.map(x => new Mutant(x.int)) // djupkopierar till ny Vector
scala> xs(0).int = (new Mutant).int
\end{REPL}
\Subtask Fyll i tabellen nedan genom att till höger skriva värdet av varje uttryck till vänster. Förklara vad som händer. \emph{Tips:} Metoden \code{eq} jämför alltid referenser (ej innehåll).

\renewcommand{\arraystretch}{2.0}
\vspace{1em}\noindent\begin{tabular}{@{} l | p{5.5cm}}\hline
\code|xs(0)         | & \\\hline
\code|ys(0).int| & \\\hline
\code|zs(0).int| & \\\hline
\code|xs(0) eq ys(0)| & \\\hline
\code|xs(0) eq zs(0)| & \\\hline
\code|(ys.toBuffer :+ new Mutant).apply(0).int| & \\\hline
\end{tabular}

\Subtask Implementera med hjälp av en \code{while}-sats funktionen \code{deepCopy} nedan som gör \emph{djup} kopiering, d.v.s skapar en ny array med nya, innehållskopierade mutanter.
\begin{Code}
def deepCopy(xs: Array[Mutant]): Array[Mutant] = ???
\end{Code}
Använd denna algoritm:

\begin{algorithm}[H]
 \SetKwInOut{Input}{Indata}\SetKwInOut{Output}{Resultat}

 \Input{ ~En mutantarray $xs$}
 \Output{ ~En djup kopia av $xs$}
 $result \leftarrow$ en ny mutantarray med plats för lika många element som i $xs$\\
 $i \leftarrow 0$  \\
 \While{$i$ mindre än antalet element}{
  skapa en kopia av elementet $xs(i)$ och lägg kopian i $result$ på platsen $i$ \\
  öka $i$ med 1
 }
 \Return $result$
\end{algorithm}

\Subtask Testa att din funktion och kolla så att inga läskiga muteringar genom delade referenser går att göra, så som med \code|xs| och \code|ys| i första deluppgiften.

\Subtask Är det vanligt att man, för säkerhets skull, gör djupkopiering av alla element i oföränderliga samlingar som enbart innehåller oföränderliga element?

\SOLUTION

\TaskSolved \what~

\SubtaskSolved

\renewcommand{\arraystretch}{1.5}
\vspace{1em}\noindent\begin{tabular}{@{} p{0.4\textwidth} p{0.6\textwidth}}\hline
\code|xs(0)| & \code|rs$line5$Mutant@66d766b9 | nya instanser får nya hexkoder \\ \hline 
\code|ys(0).int               | & \code|0 | eftersom \code|ys| innehåller samma instans som \code|xs|\\ \hline
\code|zs(0).int               | & \code|5 | eftersom \code|!(xs(0) eq zs(0))| \\ \hline
\code|xs(0) eq ys(0)          | & \code|true |  eftersom samma instans \\ \hline
\code|xs(0) eq zs(0)          | & \code|false | eftersom olika instanser\\ \hline
\code|(ys.toBuffer :+ |
\code|  new Mutant).apply(0).int| & \code|0 | eftersom den ej djupkopierade kopian av typen \code|ArrayBuffer| refererar samma instans på första platsen som både \code|ys| och \code|xs| och \code|x(0).int| blev noll i en tilldelning på rad 5 i REPL-körningen\\ \hline
\end{tabular}

\vspace{0.5em}\noindent Observera alltså att kopiering med \code{toArray}, \code{toVector}, \code{toBuffer}, etc. \emph{inte är djup}, d.v.s. det är bara instansreferenserna som kopieras och inte själva instanserna.


\SubtaskSolved
\begin{CodeSmall}
def deepCopy(xs: Array[Mutant]): Array[Mutant] =
  val result = Array.ofDim[Mutant](xs.length) //fylld med null-referenser
  var i = 0
  while i < xs.length do
    result(i) = new Mutant(xs(i).int) //kopia med samma innehåll på samma plats
    i += 1
  result
\end{CodeSmall}
Det går också bra att skapa resultatarrayen med \code{new Array[Mutant](xs.length)}.
Du kan också använda \code{size} i stället för \code{length}.

\SubtaskSolved
\begin{REPL}
scala> class Mutant(var int: Int = 0)
// defined class Mutant

scala> def deepCopy(xs: Array[Mutant]): Array[Mutant] =
     |   val result = Array.ofDim[Mutant](xs.length)
     |   var i = 0
     |   while i < xs.length do
     |     result(i) = new Mutant(xs(i).int)
     |     i += 1
     |   result

scala> val xs = Array.fill(3)(new Mutant)
xs: Array[Mutant] = Array(rs$line$2$Mutant@46a123e4, rs$line$2$Mutant@44bc2449,
rs$line2$Mutant@3c28e5b6)

scala> val ys = deepCopy(xs)
ys: Array[Mutant] = Array(rs$line$2$Mutant@14b8a751, rs$line2$Mutant@7345f97d,
rs$line$2$Mutant@554566a8)

scala> xs(0).int = 5

scala> ys(0).int
val res0: Int = 0
\end{REPL}

\SubtaskSolved Nej, eftersom elementen inte kan förändras kan man utan problem dela referenser mellan samlingar. Det finns inte någon möjlighet att det kan ske förändringar som påverkar flera samlingar samtidigt.
Dock gör man vanligen (ofta tidsödande) djupkopieringar av samlingar med förändringsbara element för att kunna vara säker på att den ursprungliga samlingen inte förändras.

\QUESTEND



\ifPreSolution
\begin{framed}
\noindent\emph{Tips inför fortsättningen:} Ofta kan du lösa grundläggande delproblem med inbyggda samlingsmetoder ur standardbiblioteket. Till exempel kan ju kopieringen i \code{deepCopy} i föregående uppgift enkelt göras med hjälp av samlingsmetoden \code{map}.

Men det är mycket bra för din förståelse om du kan implementera grundläggande sekvensalgoritmer själv även om det normalt är bättre att använda färdiga, vältestade  metoder. I kommande uppgifter ska du därför göra egna implementationer av några sekvensalgoritmer som redan finns i standardbiblioteket.
\end{framed}
\fi



\WHAT{Uppdatering av sekvenser.}

\QUESTBEGIN

\Task \what~Deklarera dessa variabler i REPL:

\begin{Code}
val xs = (1 to 4).toVector
val buf = xs.toBuffer
\end{Code}

\Subtask Uttrycken till vänster evalueras uppifrån och ned. Para ihop med rätt resultat.

\begin{ConceptConnections}
  \code|{ buf(0) = -1; buf(0) }   | & 1 & & A & {\small\code|value update is not a member|} \\ 
  \code|{ xs(0) = -1; xs(0) }| & 2 & & B & \code|Vector(5, 2, 3, 4)| \\ 
  \code|buf.update(1, 5)          | & 3 & & C & \code|ArrayBuffer(-1, 5, 3, 4, 5)| \\ 
  \code|xs.updated(0, 5)          | & 4 & & D & \code|-1| \\ 
  \code|buf += 5                 | & 5 & & E & \code|Vector(1, -1, 5)| \\ 
  \code|xs += 5                  | & 6 & & F & \code|(): Unit| \\ 
  \code|xs.patch(1,Vector(-1,5),3)| & 7 & & G & {\small\code|value += is not a member|} \\ 
  \code|xs                        | & 8 & & H & \code|Vector(1, 2, 3, 4)|
\end{ConceptConnections}

\smallskip
\emph{Tips:} Läs om metoderna i snabbreferensen och undersök i REPL. Exempel:
\begin{REPL}
scala> Vector(1,2,3,4).patch(from = 1, other = Vector(0,0), replaced = 3)
val res0: Vector[Int] = Vector(1, 0, 0)
\end{REPL}

\Subtask Implementera funktionen \code{insert} nedan med hjälp av sekvenssamlingsmetoden \code{patch}. \emph{Tips:} Ge argumentet \code{0} till parametern \code{replaced}.
\begin{Code}
/** Skapar kopia av xs men med elem insatt på plats pos. */
def insert(xs: Array[Int], elem: Int, pos: Int): Array[Int] = ???
\end{Code}

\Subtask Skriv pseduokod för en algoritm som implementerar \code{insert} med hjälp av \code{while}.

\Subtask Implementera \code{insert} enligt din pseudokod. Testa i REPL och se vad som händer om \code{pos} är negativ? Vad händer om \code{pos} är precis ett steg bortom sista platsen i \code{xs}? Vad händer om \code{pos} är flera steg bortom sista platsen?

\SOLUTION

\TaskSolved \what~

\SubtaskSolved

\begin{ConceptConnections}
  \code|{ buf(0) = -1; buf(0) }   | & 1 & ~~\Large$\leadsto$~~ &  D & \code|-1| \\ 
  \code|{ xs(0) = -1; xs(0) }| & 2 & ~~\Large$\leadsto$~~ &  A & {\small\code|value update is not a member|} \\ 
  \code|buf.update(1, 5)          | & 3 & ~~\Large$\leadsto$~~ &  F & \code|(): Unit| \\ 
  \code|xs.updated(0, 5)          | & 4 & ~~\Large$\leadsto$~~ &  B & \code|Vector(5, 2, 3, 4)| \\ 
  \code|buf += 5                | & 5 & ~~\Large$\leadsto$~~ &  C & \code|ArrayBuffer(-1, 5, 3, 4, 5)| \\ 
  \code|xs += 5                 | & 6 & ~~\Large$\leadsto$~~ &  G & {\small\code|value += is not a member|} \\ 
  \code|xs.patch(1,Vector(-1,5),3)| & 7 & ~~\Large$\leadsto$~~ &  E & \code|Vector(1, -1, 5)| \\ 
  \code|xs                        | & 8 & ~~\Large$\leadsto$~~ &  H & \code|Vector(1, 2, 3, 4)| 
\end{ConceptConnections}

\SubtaskSolved

\begin{Code}
def insert(xs: Array[Int], elem: Int, pos: Int): Array[Int] =
  xs.patch(from = pos, other = Array(elem), replaced = 0)
\end{Code}

\SubtaskSolved Pseudokoden nedan är skriven så att den kompilerar fast den är ofärdig.
\begin{Code}
def insert(xs: Array[Int], elem: Int, pos: Int): Array[Int] = 
  val result = ??? /* ny array med plats för ett element mer än i xs */
  var i = 0
  while(???){/* kopiera elementen före plats pos och öka i */}
  if i < result.length then /* lägg elem i result på plats i */
  while(???){/* kopiera över resten */}
  result

\end{Code}

\SubtaskSolved
\begin{Code}
def insert(xs: Array[Int], elem: Int, pos: Int): Array[Int] = 
  val result = new Array[Int](xs.length + 1)
  var i = 0
  while i < pos && i < xs.length do  { result(i) = xs(i); i += 1}
  if i < result.length then { result(i) = elem; i += 1 }
  while i < result.length do { result(i) = xs(i - 1); i += 1}
  result

\end{Code}
\begin{REPL}
scala> insert(Array(1, 2), 0, pos = -1)
val res2: Array[Int] = Array(0, 1, 2)

scala> insert(Array(1, 2), 0, pos = 0)
val res3: Array[Int] = Array(0, 1, 2)

scala> insert(Array(1, 2), 0, pos = 1)
val res4: Array[Int] = Array(1, 0, 2)

scala> insert(Array(1, 2), 0, pos = 2)
val res5: Array[Int] = Array(1, 2, 0)

scala> insert(Array(1, 2), 0, pos = 42)
val res7: Array[Int] = Array(1, 2, 0)
\end{REPL}

\QUESTEND




\ifPreSolution
\begin{framed}
\noindent\emph{Tips inför fortsättningen:} Det är inte lätt att få rätt på alla specialfall även i små algoritmer så som \code{insert} ovan. Det är därför viktigt att noga tänka igenom sin sekvensalgoritm med avseende på olika specialfall. Använd denna checklista:
\begin{enumerate}[noitemsep]
  \item Vad händer om sekvensen är tom?
  \item Fungerar det för exakt ett element?
  \item Kan index bli negativt?
  \item Kan index bli mer än längden minus ett?
  \item Kan det bli en oändlig loop, t.ex. p.g.a. saknad loopvariabeluppräkning?
\end{enumerate}
Ibland vill man att vettiga undantag ska kastas vid ogiltig indata eller andra feltillstånd och då är \code{require} eller \code{assert} bra att använda. I andra fall vill man att resultatet t.ex. ska bli en tom sekvenssamling om indata är ogiltigt. Sådana beteenden behöver dokumenteras så att andra som använder dina algoritmer (eller du själv efter att du glömt hur det var) förstår vad som händer i olika fall.


\end{framed}
\fi

\WHAT{Jämföra strängar i Scala.}

\QUESTBEGIN

\Task \label{task:string-order-operators} \what~  I Scala kan strängar jämföras med operatorerna \code{==}, \code{!=}, \code{<}, \code{<=}, \code{>}, \code{>=},  där likhet/olikhet avgörs av om alla tecken i strängen är lika eller inte, medan större/mindre avgörs av sorteringsordningen i enlighet med varje teckens Unicode-värde.\footnote{Överkurs: Alla tecken i en \code{java.lang.String} representeras enligt UTF-16-standarden (\href{https://en.wikipedia.org/wiki/UTF-16}{https://en.wikipedia.org/wiki/UTF-16}), vilket innebär att varje Unicode-kodpunkt \Eng{code point} lagras som antingen ett eller två 16-bitars heltal. Strängjämförelse i Scala och Java jämför egentligen inte varje tecken, utan varje 16-bitars heltal. Denna skillnad har ingen betydelse när en sträng bara innehåller tecken som tar upp ett 16-bitars heltal var, och praktiskt nog är nästan alla tecken som används vardagligen av den typen. De flesta tecken som kräver två 16-bitars heltal är sällsynta kinesiska tecken, sällsynta symboler, tecken från utdöda språk och emoji. Vi kommer att bortse från sådana tecken i den här kursen.}

\Subtask Vad ger följande jämförelser för värde?
\begin{REPL}
scala> 'a' < 'b'
scala> "aaa" < "aaaa"
scala> "aaa" < "bbb"
scala> "AAA" < "aaa"
scala> "ÄÄÄ" < "ÖÖÖ"
scala> "ÅÅÅ" < "ÄÄÄ"
\end{REPL}
Tyvärr så följer ordningen av ÄÅÖ inte svenska regler, men det ignorerar vi i fortsättningen för enkelhets skull; om du är intresserad av hur man kan fixa  detta, gör uppgift \ref{task:swedish-letter-ordering}.

\Subtask\Pen Vilken av strängarna $s1$ och $s2$ kommer först (d.v.s. är ''mindre'') om $s1$ utgör början av $s2$ och $s2$ innehåller fler tecken än $s1$?


\SOLUTION


\TaskSolved \what


\SubtaskSolved
\begin{REPL}
true
true
true
true
true
false
\end{REPL}

\SubtaskSolved
\emph{s1} kommer först.


\QUESTEND




\WHAT{Linjärsökning enligt olika sökkriterier.}

\QUESTBEGIN

\Task \what~Linjärsökning innebär att man letar tills man hittar det man söker efter i en sekvens. Detta delproblem återkommer ofta! Vanligen börjar linjärsökning från början och håller på tills man hittar något element som uppfyller kriteriet. Beroende på vad som finns i sekvensen och hur kriteriet ser ut kan det hända att man måste gå igenom alla element utan att hitta det som söks.

\Subtask Linjärsökning med inbyggda sekvenssamlingsmetoder.
\begin{Code}
val xs = ((1 to 5).reverse ++ (0 to 5)).toVector
\end{Code}
Deklarera ovan variabel i REPL och para ihop uttrycken nedan med rätt värden. Förklara vad som händer.

\begin{ConceptConnections}
\input{generated/quiz-w07-seq-find-taskrows-generated.tex}
\end{ConceptConnections}

\Subtask Implementera linjärsökning i strängvektor med strängpredikat.
\begin{Code}
/** Returns first index where p is true. Returns -1 if not found. */
def indexOf(xs: Vector[String], p: String => Boolean): Int = ???
\end{Code}
Ett strängpredikat \code{p: String => Boolean} är en funktion som tar en sträng som indata och ger ett booleskt värde som resultat. Implementera \code{indexOf} med hjälp av en \code{while}-sats. Du kan t.ex. använda en lokal boolesk variabel \code{found} för att hålla reda på om du har hittat det som eftersöks enligt predikatet.

När element som uppfyller predikatet saknas måste man bestämma vad som ska hända. Kravet på din implementation i detta fall ges av dokumentationskommentaren ovan.

Din funktion ska fungera enligt nedan:
\begin{REPL}
scala> val xs = Vector("hej", "på", "dej")
val xs: Vector[String] = Vector(hej, på, dej)

scala> indexOf(xs, _.contains('p'))
val res0: Int = 1

scala> indexOf(xs, _.contains('q'))
val res1: Int = -1

scala> indexOf(Vector(), _.contains('q'))
val res2: Int = -1

scala> indexOf(Vector("q"), _.length == 1)
val res3: Int = 0
\end{REPL}

\SOLUTION

\TaskSolved \what~

\SubtaskSolved

\begin{ConceptConnections}
\input{generated/quiz-w07-seq-find-solurows-generated.tex}
\end{ConceptConnections}

\SubtaskSolved Med en boolesk variabel \code{found}:

\begin{Code}
def indexOf(xs: Vector[String], p: String => Boolean): Int = 
  var found = false
  var i = 0
  while i < xs.length && !found do
      found = p(xs(i))
      i += 1
  if found then i - 1 else -1
\end{Code}
Eller utan \code{found}:
\begin{Code}
def indexOf(xs: Vector[String], p: String => Boolean): Int = 
  var i = 0
  while i < xs.length && !p(xs(i)) do i += 1
  if i == xs.length then -1 else i
\end{Code}
Eller så kanske man vill börja bakifrån; lösningen nedan är nog enklare att fatta (?) och definitivt mer koncis, men uppfyller \emph{inte} kravet att returnera index för \emph{första} förekomsten som det står i uppgiften. Men om sammanhanget tillåter att vi returnerar \emph{något} index för vilket predikatet gäller, eller om man faktiskt har kravet att leta bakifrån, så funkar detta:
\begin{Code}
def indexOf(xs: Vector[String], p: String => Boolean): Int = 
  var i = xs.length - 1
  while i >= 0 && !p(xs(i)) do i -= 1
  i
\end{Code}
Eller så kan man göra på flera andra sätt. När du ska implementera algoritmer, både på programmeringstentan och i yrkeslivet som systemutvecklare, finns det ofta många olika sätt att lösa uppgiften på som har olika egenskaper, fördelar och nackdelar. Det viktiga är att lösningen fungerar så gott det går enligt kraven, att koden är begriplig för människor och att implementationen inte är så ineffektiv att användarna tröttnar i sin väntan på resultatet...

\QUESTEND




\WHAT{Labbförberedelse: Implementera heltalsregistrering i Array.}

\QUESTBEGIN

\Task \what~Registrering innebär att man räknar antalet förekomster av olika värden. Varje gång ett nytt värde förekommer behöver vi räkna upp en frekvensräknare. Det behövs en räknare för varje värde som ska registreras. Vi ska fortsätta räkna ända tills alla värden är registrerade.

På veckans laboration ska du registrera förekomsten av olika kortkombinationer i kortspelet poker. I denna övning ska du som träning inför laborationen lösa ett liknande registreringsproblem:  frekvensanalys av många tärningskast. Vid tärningsregistrering behövs sex olika räknare. Man kan med fördel då använda en sekvenssamling med plats för sex heltal. Man kan t.ex. låta  plats \code{0} håller reda på antalet ettor, plats \code{1} hålla reda på antalet tvåor, etc.

\Subtask Implementera nedan algoritm enligt pseudokoden:
\begin{Code}
def registreraTärningskast(xs: Seq[Int]): Vector[Int] = 
  val result = ??? /* Array med 6 nollor */
  xs.foreach{ x =>
    require(x >= 1 && x <= 6, "tärningskast ska vara mellan 1 & 6")
    ??? /* räkna förekomsten av x */
  }
  result.toVector
\end{Code}

\Subtask Använd funktionen \code{kasta} nedan när du testar din registreringsalgoritm med en sekvenssamling innehållande minst $1000$ tärningskast.
\begin{Code}
def kasta(n: Int) = Vector.fill(n)(util.Random.nextInt(6) + 1)
\end{Code}

\SOLUTION

\TaskSolved \what~

\SubtaskSolved
\begin{Code}
def registreraTärningskast(xs: Seq[Int]): Vector[Int] = 
  val result = Array.fill(6)(0)
  xs.foreach{ x =>
    require(x >= 1 && x <= 6, "tärningskast ska vara mellan 1 & 6")
    result(x - 1) += 1
  }
  result.toVector
\end{Code}

\SubtaskSolved
\begin{REPL}
scala> registreraTärningskast(kasta(1000))
val res0: Vector[Int] = Vector(171, 163, 166, 152, 184, 164)

scala> registreraTärningskast(kasta(1000))
val res1: Vector[Int] = Vector(163, 161, 158, 174, 161, 183)
\end{REPL}

\QUESTEND




\WHAT{Inbyggda metoder för sortering.}

\QUESTBEGIN

\Task \what~Det finns fler olika sätt att ordna sekvenser efter olika kriterier. För  grundtyperna \code{Int}, \code{Double}, \code{String}, etc., finns inbyggda ordningar som gör att sekvenssamlingsmetoden \code{sorted} fungerar utan vidare argument (om du är nöjd med den inbyggda ordningsdefinitionen). Det finns också metoderna \code{sortBy} och \code{sortWith} om du vill ordna en sekvens med element av någon grundtyp efter egna ordningsdefinitioner eller om du har egna klasser i din sekvens.
\begin{Code}
val xs = Vector(1, 2, 1, 3, -1)
val ys = Vector("abra", "ka", "dabra").map(_.reverse)
val zs = Vector('a', 'A', 'b', 'c').sorted

case class Person(förnamn: String, efternamn: String)

val ps = Vector(Person("Kim", "Ung"), Person("kamrat", "Clementin"))
\end{Code}
Deklarera ovan i REPL och para ihop uttryck nedan med rätt resultat.
\\\emph{Tips:} Stora bokstäver sorteras före små bokstäver i den inbyggda ordningen för grundtyperna \code{String} och \code{Char}. Dessutom har svenska tecken knasig ordning.\footnote{Ordningen kommer ursprungligen från föråldrade teckenkodningsstandarder:    \url{https://sv.wikipedia.org/wiki/ASCII}}
\\Läs om sorteringsmetoderna i snabbreferensen och prova i REPL.

\begin{ConceptConnections}
  \code|'a' < 'A'                  | & 1 & & A & \code|"ka"| \\ 
  \code|"AÄÖö" < "AÅÖö"        | & 2 & & B & \code|1| \\ 
  \code|xs.sorted.head             | & 3 & & C & \code|-1| \\ 
  \code|xs.sorted.reverse.head     | & 4 & & D & \code|error: ...| \\ 
  \code|ys.sorted.head             | & 5 & & E & \code|false| \\ 
  \code|zs.indexOf('a')            | & 6 & & F & \code|0| \\ 
  \code|ps.sorted.head.förnamn.take(2)| & 7 & & G & \code|3| \\ 
  \code|ps.sortBy(_.förnamn).apply(1).förnamn.take(2)| & 8 & & H & \code|true| \\ 
  \code|xs.sortWith((x1, x2) => x1 > x2).indexOf(3)| & 9 & & I & \code|"ak"| 
\end{ConceptConnections}
Vi ska senare i kursen implementera egna sorteringsalgoritmer som träning, men i normala fall använder man inbyggda sorteringar som är effektiva och vältestade. Dock är det inte ovanligt att man vill definiera egna ordningar för egna klasser, vilket vi ska undersöka senare i kursen.

\SOLUTION

\TaskSolved \what

\begin{ConceptConnections}
  \code|'a' < 'A'                  | & 1 & ~~\Large$\leadsto$~~ &  E & \code|false| \\ 
  \code|"AÄÖö" < "AÅÖö"        | & 2 & ~~\Large$\leadsto$~~ &  H & \code|true| \\ 
  \code|xs.sorted.head             | & 3 & ~~\Large$\leadsto$~~ &  C & \code|-1| \\ 
  \code|xs.sorted.reverse.head     | & 4 & ~~\Large$\leadsto$~~ &  G & \code|3| \\ 
  \code|ys.sorted.head             | & 5 & ~~\Large$\leadsto$~~ &  I & \code|"ak"| \\ 
  \code|zs.indexOf('a')            | & 6 & ~~\Large$\leadsto$~~ &  B & \code|1| \\ 
  \code|ps.sorted.head.förnamn.take(2)| & 7 & ~~\Large$\leadsto$~~ &  D & \code|error: ...| \\ 
  \code|ps.sortBy(_.förnamn).apply(1).förnamn.take(2)| & 8 & ~~\Large$\leadsto$~~ &  A & \code|"ka"| \\ 
  \code|xs.sortWith((x1, x2) => x1 > x2).indexOf(3)| & 9 & ~~\Large$\leadsto$~~ &  F & \code|0| 
\end{ConceptConnections}
Det blir fel i uttrycket ovan som försöker sortera en sekvens med instanser av \code{Person} direkt med metoden \code{sorted}:
\begin{REPL}
scala> ps.sorted
No implicit Ordering defined for Person.
\end{REPL}
Det blir fel eftersom kompilatorn inte hittar någon ordningsdefinition för dina egna klasser. Senare i kursen ska vi se hur vi kan skapa egna ordningar om man vill få \code{sorted} att fungera på sekvenser med instanser av egna klasser, men ofta räcker det fint med \code{sortBy} och \code{sortWith}.
\QUESTEND


\WHAT{Inbyggd metod för blandning.}

\QUESTBEGIN
\Task \what~På veckans laboration ska du implementera en egen blandningsalgoritm och använda den för att blanda en kortlek. Det finns redan en inbygg metod \code{shuffle} i singelobjektet \code{Random} i paketet \code{scala.util}.

\Subtask Sök upp dokumentationen för \code{Random.shuffle} och studera funktionshuvudet. Det står en hel del invecklade saker om \code{CanBuildFrom} etc. Detta smarta krångel, som vi inte går närmare in på i denna kurs, är till för att metoden ska kunna returnera lämplig typ av samling. När du ser ett sådant funktionshuvud kan du anta att metoden fungerar fint med flera olika typer av lämpliga samlingar i Scalas standardbibliotek.

Klicka på \code{shuffle}-dokumentationen så att du ser hela texten. Vad säger dokumentationen om resultatet? Är det blandning på plats eller blandning till ny samling?

\Subtask Prova upprepade blandningar av olika typer av sekvenser med olika typer av element i REPL.

\SOLUTION

\TaskSolved \what~

\SubtaskSolved \code{Random.shuffle} returnerar en ny blandad sekvenssamling av samma typ. Ordningen i den ursprungliga samlingen påverkas inte.

\SubtaskSolved Exempel på användning av \code{random.shuffle}:
\begin{REPL}
scala> import scala.util.Random

scala> val xs = Vector("Sten", "Sax", "Påse")
val xs: Vector[String] = Vector(Sten, Sax, Påse)

scala> (1 to 10).foreach(_ => println(Random.shuffle(xs).mkString(" ")))
Sax Påse Sten
Sten Påse Sax
Sten Sax Påse
Sten Sax Påse
Sten Påse Sax
Sten Påse Sax
Sax Sten Påse
Sten Påse Sax
Sax Påse Sten
Sax Påse Sten

scala> (1 to 5).map(_ => Random.shuffle(1 to 6))
val res1: IndexedSeq[IndexedSeq[Int]] =
  Vector(Vector(5, 2, 1, 4, 3, 6), Vector(6, 5, 4, 2, 1, 3),
  Vector(3, 1, 4, 6, 5, 2), Vector(3, 2, 6, 5, 1, 4),
  Vector(5, 3, 4, 6, 1, 2))

scala> (1 to 1000).map(_ => Random.shuffle(1 to 6).head).count(_ == 6)
val res2: Int = 168
\end{REPL}

\QUESTEND



\WHAT{Repeterade parametrar.}

\QUESTBEGIN

\Task  \what~  Det går att deklarera en funktion som tar en argumentsekvens av godtycklig längd, ä.k. \emph{varargs}. Syntaxen består av en asterisk \code{*} efter typen. Funktion sägs då ha repeterade parametrar \Eng{repeated parameters}. I funktionskroppen får man tillgång till argumenten i en sekvenssamling. Argumenten anges godtyckligt många med komma emellan. Exempel:
\begin{Code}
/** Ger en vektor med stränglängder för godtyckligt antal strängar. */
def stringSizes(xs: String*): Vector[Int] = xs.map(_.size).toVector
\end{Code}

\Subtask Deklarera och använd \code{stringSizes} i REPL. Vad händer om du anropar \code{stringSizes} med en tom argumentlista?

\Subtask Det händer ibland att man redan har en sekvenssamling, t.ex. \code{xs}, och vill skicka med varje element som argument till en varargs-funktion. Syntaxen för detta är \code{xs: _* } vilket gör att kompilatorn omvandlar sekvenssamlingen till en argumentsekvens av rätt typ.

Prova denna syntax genom att ge en \code{xs} av typen \code{Vector[String]} som argument till \code{stringSizes}. Fungerar det även om \code{xs} är en sekvens av längden 0?

\SOLUTION

\TaskSolved \what

\SubtaskSolved

\begin{REPL}
scala> def stringSizes(xs: String*): Vector[Int] = xs.map(_.size).toVector
def stringSizes(xs: String*): Vector[Int]

scala> stringSizes("hej")
val res0: Vector[Int] = Vector(3)

scala> stringSizes("hej", "på", "dej", "")
val res1: Vector[Int] = Vector(3, 2, 3, 0)

scala> stringSizes()
val res2: Vector[Int] = Vector()
\end{REPL}

\noindent Anrop med tom argumentlista ger en tom heltalssekvens.

\SubtaskSolved

\begin{REPL}
scala> val xs = Vector("hej","på","dej", "")
val xs: Vector[String] = Vector(hej, på, dej, "")

scala> stringSizes(xs: _*)
val res0: Vector[Int] = Vector(3, 2, 3, 0)

scala> stringSizes(Vector(): _*)
val res1: Vector[Int] = Vector()
\end{REPL}
Ja, det funkar fint med tom sekvens.

\QUESTEND



\clearpage

\ExtraTasks %%%%%%%%%%%%%%%%%%%%%%%%%%%%%%%%%%%%%%%%%%%%%%%%%%%%%%%%%%%%%%%%%%%%



\WHAT{Registrering av booleska värden. Singla slant.}

\QUESTBEGIN

\Task \what~

\Subtask Implementera en funktion som registrerar många slantsinglingar enligt nedan funktionshuvud. Indata är en sekvens av booleska värden där krona kodas som \code{true} och klave kodas som \code{false}. För registreringen ska du använda en lokal \code{Array[Int]}. I resultatet ska antalet utfall av \code{krona} ligga på första platsen i 2-tupeln och på andra platsen ska antalet utfall av \code{klave} ligga.

\begin{Code}
def registerCoinFlips(xs: Seq[Boolean]): (Int, Int) = ???
\end{Code}

\Subtask Skapa en funktion \code{flips(n)} som ger en boolesk \code{Vector} med $n$ stycken slantsinglingar och använd den när du testar din slantsinglingsregistreringsalgoritm.

\SOLUTION

\TaskSolved \what~

\SubtaskSolved
\begin{Code}
def registerCoinFlips(xs: Seq[Boolean]): (Int, Int) = 
  val result = Array.fill(2)(0)
  xs.foreach(x => if (x) result(0) += 1 else result(1) += 1)
  (result(0), result(1))
\end{Code}

\SubtaskSolved

\QUESTEND


\WHAT{Kopiering och tillägg på slutet.}

\QUESTBEGIN

\Task \what~
Skapa funktionen \code{copyAppend} som implementerar nedan algoritm, \emph{efter} att du rättat de \textbf{\color{red}{två buggarna}} nedan:

\begin{algorithm}[H]
 \SetKwInOut{Input}{Indata}\SetKwInOut{Output}{Resultat}

 \Input{Heltalsarray $xs$ och heltalet $x$}
 \Output{En ny heltalsarray som som är en kopia av $xs$ men med $x$ tillagt på slutet som extra element.}
 $ys \leftarrow$ en ny array med plats för ett element mer än i $xs$\\
 $i \leftarrow 0$  \\
 \While{$i \leq xs.length$}{
  $ys(i) \leftarrow xs(i)$
 }
lägg $x$ på sista platsen i $ys$
\end{algorithm}

\noindent Granska din kod enligt checklistan i tidigare tipsruta. Testa din funktion för de olika fallen: tom sekvens, sekvens med exakt ett element, sekvens med många element.


\SOLUTION

\TaskSolved \what~

\begin{Code}
def copyAppend(xs: Array[Int], x: Int): Array[Int] = 
  val ys = new Array[Int](xs.length + 1)
  var i = 0
  while i < xs.length do
    ys(i) = xs(i)
    i += 1
  ys(xs.length) = x
  ys
\end{Code}
De två buggarna i algoritmen finns (1) i villkoret som ska vara strikt mindre än och (2) inne i loopen där uppräkningen av loppvariabeln saknas.

\QUESTEND



% \WHAT{Välja sekvenssamling.}
%
% \QUESTBEGIN
%
% \Task  \what~Vilken sekvenssamling är lämpligast i respektive situation nedan? Välj mellan \code{Vector}, \code{ArrayBuffer} och \code{ListBuffer}.
%
% \Subtask Det asociala mediet ZuckerBok ska lagra statusuppdateringar från sina användare. Dessa lagras i en förändringsbar sekvens där nya poster läggs till först. Indexering mitt i sekvensen är mycket ovanligt eftersom de flesta användarna sällan läser vad andra skriver, utan mest skriver nya inlägg om sig själv.
%
% \Subtask ZuckerBok försöker öka sina intäkter och börjar frenetiskt indexera i kors och tvärs i sekvensen med statusuppdaringar för att söka efter lämpliga spamoffer.
%
% \Subtask ZuckerBok bestämmer sig för att lagra födelsedatum för alla ca $10^7$ medborgare i Sverige i en oföränderlig sekvens för att kunna förmedla specialreklam på födelsedagar.
%
% \SOLUTION
%
% \TaskSolved \what
%
% \SubtaskSolved  \code{ListBuffer} som är snabb på fröändringar i början av sekvensen.
%
% \SubtaskSolved  \code{ArrayBuffer} som är snabb på både storleksförändringar och godtycklig indexering.
%
% \SubtaskSolved  \code{Vector} eftersom ofränderlighet efterfrågas.
%
% \QUESTEND



\WHAT{Kopiera och reversera sekvens.}

\QUESTBEGIN

\Task  \what~  Implementera \code{seqReverseCopy} enligt:

\begin{algorithm}[H]
 \SetKwInOut{Input}{Indata}\SetKwInOut{Output}{Resultat}

 \Input{Heltalsarray $xs$}
 \Output{En ny heltalsarray med elementen i $xs$ i omvänd ordning.}
 $n \leftarrow$ antalet element i $xs$ \\
 $ys \leftarrow$ en ny heltalsarray med plats för $n$ element\\
 $i \leftarrow 0$  \\
 \While{$i < n$}{
  $ys(n - i - 1) \leftarrow xs(i)$ \\
  $i \leftarrow i + 1$
 }
 \Return $ys$
\end{algorithm}

\Subtask Använd en \code{while}-sats på samma sätt som i algoritmen. Prova din implementation i REPL och kolla så att den fungerar i olika fall.

\Subtask Gör en ny implementation som i stället använder en \code{for}-sats som börjar bakifrån. Kör din implementation i REPL och kolla så att den fungerar i olika fall.

\SOLUTION

\TaskSolved \what

\SubtaskSolved  \begin{Code}
def seqReverseCopy(xs: Array[Int]): Array[Int] =
  val n = xs.length
  val ys = new Array[Int](n)
  var i = 0
  while i < n do
    ys(n - i - 1) = xs(i)
    i += 1
  ys
\end{Code}

\SubtaskSolved  \begin{Code}
def seqReverseCopy(xs: Array[Int]): Array[Int] = 
  val n = xs.length
  val ys = new Array[Int](n)
  for i <- (n - 1) to 0 by -1 do
    ys(n - i - 1) = xs(i)
  ys
\end{Code}


\QUESTEND




\WHAT{Kopiera alla utom ett.}

\QUESTBEGIN

\Task  \what~  Implementera kopiering av en array \emph{utom} ett element på en viss angiven plats.
Skriv först pseudokod innan du implementerar:
\begin{Code}
def removeCopy(xs: Array[Int], pos: Int): Array[Int]
\end{Code}

\SOLUTION


\TaskSolved \what

\begin{algorithm}[H]
 \SetKwInOut{Input}{Indata}\SetKwInOut{Output}{Resultat}

 \Input{En sekvens $xs$ av typen \texttt{Array[Int]} och $pos$}
 \Output{En ny sekvens av typen \texttt{Array[Int]} som är en kopia av $xs$ fast med elementet på plats $pos$ borttaget}
 $n \leftarrow$ antalet element $xs$\\
 $ys \leftarrow$ en ny \texttt{Array[Int]} med plats för $n-1$ element \\
 \For{$i \leftarrow 0$ \KwTo $pos - 1$}{
  $ys(i) \leftarrow xs(i)$
 }
 $ys(pos) \leftarrow x$ \\
 \For{$i \leftarrow pos+1$ \KwTo $n - 1$}{
  $ys(i - 1) \leftarrow xs(i)$
 }
 \Return $ys$
\end{algorithm}

\begin{Code}
def removeCopy(xs: Array[Int], pos: Int): Array[Int] =
  val n = xs.size
  val ys = Array.fill(n - 1)(0)
  for i <- 0 until pos do
    ys(i) = xs(i)
  for i <- (pos + 1) until n do
    ys(i - 1) = xs(i)
  ys
\end{Code}

\QUESTEND




\WHAT{Borttagning på plats i array.}

\QUESTBEGIN

\Task  \what~  Ibland vill man ta bort ett element på en viss position i en array utan att kopiera alla element, utom ett, till en ny samling. Ett sätt att göra detta är att flytta alla efterföljande element ett steg mot lägre index och fylla ut sista positionen med ett utfyllnadsvärde, t.ex. $0$.
Skriv först pseudokod för en sådan algoritm. Implementera sedan algoritmen i en funktion med denna signatur:
\begin{Code}
def removeAndPad(xs: Array[Int], pos: Int, pad: Int = 0): Unit
\end{Code}

\SOLUTION

\TaskSolved \what

\begin{algorithm}[H]
 \SetKwInOut{Input}{Indata}\SetKwInOut{Output}{Resultat}

 \Input{En sekvens $xs$ av typen \texttt{Array[Int]}, en position $pos$ och ett utfyllnadsvärde $pad$}
 \Output{En uppdaterad sekvens av $xs$ där elementet på plats $pos$ tagits bort och efterföljande element flyttas ett steg mot lägre index med ett sista elementet som tilldelats värdet av $pad$}
 $n \leftarrow$ antalet element $xs$\\
 \For{$i \leftarrow pos+1$ \KwTo $n - 1$}{
  $xs(i - 1) \leftarrow xs(i)$
 }
 $xs(n - 1) \leftarrow pad$ \\
\end{algorithm}

\begin{Code}
def remove(xs: Array[Int], pos: Int, pad: Int = 0): Unit =
  val n = xs.size
  for i <- (pos + 1) until n do
    xs(i - 1) = xs(i)
  xs(n - 1) = pad
\end{Code}

\QUESTEND




\WHAT{Kopiering och insättning.}

\QUESTBEGIN

\Task  \what~

\Subtask Implementera en funktion med detta huvud enligt efterföljande algoritm:
\begin{Code}
def insertCopy(xs: Array[Int], x: Int, pos: Int): Array[Int]
\end{Code}


\begin{algorithm}[H]
 \SetKwInOut{Input}{Indata}\SetKwInOut{Output}{Resultat}

 \Input{En sekvens $xs$ av typen \texttt{Array[Int]} och heltalen $x$ och $pos$}
 \Output{En ny sekvens av typen \texttt{Array[Int]} som är en kopia av $xs$ men där $x$ är infogat på plats $pos$}
 $n \leftarrow$ antalet element $xs$\\
 $ys \leftarrow$ en ny \texttt{Array[Int]} med plats för $n+1$ element \\
 \For{$i \leftarrow 0$ \KwTo $pos - 1$}{
  $ys(i) \leftarrow xs(i)$
 }
 $ys(pos) \leftarrow x$ \\
 \For{$i \leftarrow pos$ \KwTo $n - 1$}{
  $ys(i + 1) \leftarrow xs(i)$
 }
 \Return $ys$
\end{algorithm}


\Subtask Vad måste \code{pos} vara för att det ska fungera med en tom array som argument?

\Subtask Vad händer om din funktion anropas med ett negativt argument för \code{pos}?

\Subtask Vad händer om din funktion anropas med \code{pos} lika med \code{xs.size}?

\Subtask Vad händer om din funktion anropas med \code{pos} större än \code{xs.size}?

\SOLUTION

\TaskSolved \what

\SubtaskSolved  \begin{Code}
def insertCopy(xs: Array[Int], x: Int, pos: Int): Array[Int] =
  val n = xs.size
  val ys = Array.ofDim[Int](n + 1)
  for i <- 0 until pos do
    ys(i) = xs(i)
  ys(pos) = x
  for i <- pos until n do
    ys(i + 1) = xs(i)
  ys
\end{Code}

\SubtaskSolved  \code{pos} måste vara \code{0}.

\SubtaskSolved  \begin{REPL}
java.lang.ArrayIndexOutOfBoundsException: -1
\end{REPL}

\SubtaskSolved  Elementet \code{x} läggs till på slutet av arrayen, alltså kommer den returnerande arrayen vara större än den som skickades in.

\SubtaskSolved  \begin{REPL}
java.lang.ArrayIndexOutOfBoundsException: 5
\end{REPL}
Man får \code{ArrayIndexOutOfBoundsException} då indexeringen är utanför storleken hos arrayen.

\QUESTEND




\WHAT{Insättning på plats i array.}

\QUESTBEGIN

\Task  \what~  Ett sätt att implementera insättning i en array, utan att kopiera alla element till en ny array med en plats extra, är att alla elementen efter \code{pos} flyttas fram ett steg till högre index, så att plats bereds för det nya elementet. Med denna lösning får det sista elementet ''försvinna'' genom brutal överskrivning eftersom arrayer inte kan ändra storlek.

Skriv först en sådan algoritm i pseudokod och implementera den sedan i en procedur med detta huvud:
\begin{Code}
def insertDropLast(xs: Array[Int], x: Int, pos: Int): Unit
\end{Code}

\SOLUTION

\TaskSolved \what

\begin{algorithm}[H]
 \SetKwInOut{Input}{Indata}\SetKwInOut{Output}{Resultat}

 \Input{En sekvens $xs$ av typen \texttt{Array[Int]} och heltalen $x$ och $pos$}
 \Output{En uppdaterad sekvens av $xs$ där elementet $x$ har satts in på platsen $pos$ och efterföljande element flyttas ett steg där sista elementet försvinner}
 $n \leftarrow$ antalet element i $xs$\\
 $ys \leftarrow$ en klon av $xs$\\
 $xs(pos) \leftarrow x$\\
 \For{$i \leftarrow pos+1$ \KwTo $n - 1$}{
  $xs(i) \leftarrow ys(i - 1)$
 }
\end{algorithm}

\begin{Code}
def insertDropLast(xs: Array[Int], x: Int, pos: Int): Unit =
  val n = xs.size
  val ys = xs.clone
  xs(pos) = x
  for i <- pos + 1 until n do
    xs(i) = ys(i - 1)
\end{Code}

\QUESTEND


\WHAT{Fler inbyggda metoder för linjärsökning.}

\QUESTBEGIN

\Task \what~

\Subtask Läs i snabbreferensen om metoderna \code{lastIndexOf}, \code{indexOfSlice}, \code{segmentLength} och \code{maxBy} och beskriv vad var och en kan användas till.

\Subtask Testa metoderna i REPL.

\SOLUTION

\TaskSolved \what~

\SubtaskSolved

\begin{itemize}[noitemsep]
  \item \code{lastIndexOf} är bra om man vill leta bakifrån i stället för framifrån; utan denna hade man annars då behövt använda \code{xs.reverse.indexOf(e)}
  \item \code{indexOfSlice(ys)} letar efter index där en hel sekvens \code{ys} börjar, till skillnad från \code{indexOf(e)} som bara letar efter ett enstaka element.
  \item \code{segmentLength(p, i)} ger längden på den längsta sammanhängande sekvens där alla element uppfyller predikatet \code{p} och sökningen efter en sådan sekvens börjar på plats \code{i}
  \item \code{xs.maxBy(f)} kör först funktionen \code{f} på alla element i \code{xs} och letar sedan upp det största värdet; motsvarande \code{minBy(f)} ger minimum av \code{f(e)} över alla element \code{e} i \code{xs}
\end{itemize}

\SubtaskSolved --

\QUESTEND



\clearpage

\AdvancedTasks %%%%%%%%%%%%%%%%%%%%%%%%%%%%%%%%%%%%%%%%%%%%%%%%%%%%%%%%%%%%%%%%%

\WHAT{Fixa svensk sorteringsordning av ÄÅÖ.}

\QUESTBEGIN

\Task \label{task:swedish-letter-ordering} \what~   Svenska bokstäver kommer i, för svenskar, konstig ordning om man inte vidtar speciella åtgärder. Med hjälp av klassen \code{java.text.Collator} kan man få en \code{Comparator} för strängar som följer lokala regler för en massa språk på planeten jorden.

\Subtask Verifiera att sorteringsordningen blir rätt i REPL enligt nedan.

\begin{REPL}
scala> val fel = Vector("ö","å","ä","z").sorted
scala> val svColl = java.text.Collator.getInstance(new java.util.Locale("sv"))
scala> val svOrd = Ordering.comparatorToOrdering(svColl)
scala> val rätt = Vector("ö","å","ä","z").sorted(svOrd)
\end{REPL}
\Subtask Använd metoden ovan för att skriva ett program som skriver ut raderna i en textfil i korrekt svensk sorteringsordning. Programmet ska kunna köras med kommandot:\\\texttt{scala sorted -sv textfil.txt}

\Subtask Läs mer här: \\
\noindent{\href{http://stackoverflow.com/questions/24860138/sort-list-of-string-with-localization-in-scala}{\small stackoverflow.com/questions/24860138/sort-list-of-string-with-localization-in-scala}}



\SOLUTION


\TaskSolved \what



\QUESTEND



\WHAT{Fibonacci-sekvens med ListBuffer.}

\QUESTBEGIN

\Task  \what~ Samlingen \code{ListBuffer} är en förändringsbar sekvens som är snabb och minnessnål vid tillägg i början \Eng{prepend}. Undersök vad som händer här:
\begin{REPL}
scala> val xs = scala.collection.mutable.ListBuffer.empty[Int]
scala> xs.prependAll(Vector(1, 1))
scala> while xs.head < 100 do {xs.prepend(xs.take(2).sum); println(xs)}
scala> xs.reverse.toList
\end{REPL}
Talen i sekvensen som produceras på rad 4 ovan kallas Fibonacci-tal  \footnote{\href{https://sv.wikipedia.org/wiki/Fibonaccital}{sv.wikipedia.org/wiki/Fibonaccital}} och blir snabbt mycket stora.

\Subtask Definera och testa följande funktion. Den ska internt använda förändringsbara \code{ListBuffer} men returnera en sekvens av oföränderliga \code{List}.

\begin{Code}
/** Ger en lista med tal ur Fibonacci-sekvensen 1, 1, 2, 3, 5, 8 ...
  * där det största talet är mindre än max. */
def fib(max: Long): List[Long] = ???
\end{Code}


\Subtask
Hur lång ska en Fibonacci-sekvens vara för att det sista elementet ska vara så nära \code{Int.MaxValue} som möjligt?


\Subtask Implementera \code{fibBig} som använder \code{BigInt} i stället för \code{Long} och låt din dator få använda sitt stora minne medan planeten värms upp en aning.

\SOLUTION

\TaskSolved \what


\SubtaskSolved

\begin{Code}
def fib(max: Long): List[Long] = 
  val xs = scala.collection.mutable.ListBuffer.empty[Long]
  xs.prependAll(Vector(1, 1))
  while xs.head < max do xs.prepend(xs.take(2).sum)
  xs.reverse.drop(1).toList
\end{Code}

\SubtaskSolved

\begin{REPL}
scala> fib(Int.MaxValue).size
val res0: Int = 46
\end{REPL}

\SubtaskSolved

\begin{Code}
def fibBig(max: BigInt): List[BigInt] =
  val xs = scala.collection.mutable.ListBuffer.empty[BigInt]
  xs.prependAll(Vector(BigInt(1), BigInt(1)))
  while xs.head < max do xs.prepend(xs.take(2).sum)
  xs.reverse.drop(1).toList
\end{Code}

\begin{REPL}
scala> fibBig(Long.MaxValue).size
val res0: Int = 92

scala> fibBig(BigInt(Long.MaxValue).pow(64)).size
val res1: Int = 5809

scala> fibBig(BigInt(Long.MaxValue).pow(128)).last
val res2: BigInt = 466572805528355449194553611102863153950720005186045547177525242118545194247268198196024304108711020686545660707513547993668927474420737702772726410095432646683782038269206733583562623144723659044965174192994997081915291671203135284448809948278794870130243195729759407652514927641622448506112336858244040087748168546825439555497978038066584506772917257705338472345660520902622305735366348690501583267086607109594118454398543160294999638070938386822164561738531661786873174424857409631803971069795886028284195109247953151499404937810249349132907101567724032186422592145774126660328936577771749713614176045435526886758975994177511201005911748503347657112775964769397750819976041389533451539207673441658345632507479241970993525868183091563469584756527454807108...

scala> fibBig(BigInt(Long.MaxValue).pow(128)).last.toString.size
val res3: Int = 2428

scala> fibBig(BigInt(Long.MaxValue).pow(256)).last.toString.size
val res4: Int = 4856

scala> fibBig(BigInt(Long.MaxValue).pow(1024)).last.toString.size
java.lang.OutOfMemoryError: Java heap space

\end{REPL}

\QUESTEND



\WHAT{Omvända sekvens på plats.}

\QUESTBEGIN

\Task \what~Implementera nedan algoritm i funktionen \code{reverseChars} och testa så att den fungerar för olika fall i REPL.


\begin{algorithm}[H]
 \SetKwInOut{Input}{Indata}\SetKwInOut{Output}{Resultat}

 \Input{En array $xs$ med tecken}
 \Output{Samma array med tecknen i omvänd ordning}
 $n \leftarrow$ antalet element i $xs$\\
 \For{$i \leftarrow 0$ \KwTo $\frac{n}{2} - 1$}{
  $temp \leftarrow xs(i)$ \\
  $xs(i) \leftarrow xs(n - i - 1)$ \\
  $xs(n - i - 1) \leftarrow temp$ \\
 }
\end{algorithm}

\SOLUTION

\TaskSolved \what~
\begin{Code}
def reverseChars(xs: Array[Char]): Unit =
  val n = xs.length
  for i <- 0 to (n/2 - 1) do
    val temp = xs(i)
    xs(i) = xs(n - i - 1)
    xs(n - i - 1) = temp
\end{Code}

\QUESTEND



\WHAT{Palindrompredikat.}

\QUESTBEGIN

\Task  \what~ En palindrom\footnote{\url{https://sv.wikipedia.org/wiki/Palindrom}} är ett ord som förblir oförändrat om man läser det baklänges. Exempel på palindromer: kajak, dallassallad.

Ett sätt att implementera ett palindrompredikat visas nedan:
\begin{Code}
def isPalindrome(s: String): Boolean = s == s.reverse
\end{Code}

\Subtask Implementationen ovan kan innebära att alla tecken i strängen gås igenom två gånger och behöver minnesutrymme för dubbla antalet tecken. Varför?

\Subtask Skapa ett palindromtest som går igenom elementen max en gång och som inte behöver extra minnesutrymme för en kopia av strängen. \emph{Lösningsidé:} Jämför parvis första och sista, näst första och näst sista, etc.

\SOLUTION

\TaskSolved \what

\SubtaskSolved Omvändning med \code{reverse} kan kräva genomgång av hela strängen en gång samt minnesutrymme för kopian. Innehållstestet kräver ytterligare en genomgång. (Detta är i och för sig inget stort problem eftersom världens längsta palindrom inte är längre än 19 bokstäver och är ett obskyrt finskt ord som inte ofta yttras i dagligt tal. Vilket?)

\SubtaskSolved

\begin{Code}
def isPalindrome(s: String): Boolean =
  val n = s.length
  var foundDiff = false
  var i = 0
  while i < n/2 && !foundDiff do
    foundDiff = s(i) != s(n - i - 1)
    i += 1
  !foundDiff
\end{Code}

\QUESTEND



\WHAT{Fler användbara sekvenssamlingsmetoder.}

\QUESTBEGIN

\Task \what~Sök på webben och läs om dessa metoder och testa dem i REPL:
\begin{itemize}[noitemsep]
  \item \code{xs.tabulate(n)(f)}
  \item \code{xs.forall(p)}
  \item \code{xs.exists(p)}
  \item \code{xs.count(p)}
  \item \code{xs.zipWithIndex}
\end{itemize}

\SOLUTION

\TaskSolved \what~
\begin{REPL}
scala> val xs = Vector.tabulate(10)(i => math.pow(2, i).toInt)
xs: Vector[Int] = Vector(1, 2, 4, 8, 16, 32, 64, 128, 256, 512)

scala> xs.forall(_ < 1024)
val res0: Boolean = true

scala> xs.exists(_ == 3)
val res1: Boolean = false

scala> xs.count(_ > 64)
val res2: Int = 3

scala> xs.zipWithIndex.take(5)
val res3: Vector[(Int, Int)] = Vector((1,0), (2,1), (4,2), (8,3), (16,4))
\end{REPL}
\QUESTEND






\WHAT{Arrays don't behave, but \code{ArraySeq}s do!}

\QUESTBEGIN

\Task \what~Även om \code{Array} är primitiv så finns smart krångel ''under huven'' i Scalas samlingsbibliotek för att arrayer ska bete sig nästan som ''riktiga'' samlingar. Därmed behöver man inte ägna sig åt olika typer av specialhantering, t.ex. s.k. boxning, wrapperklasser och typomvandlingar \Eng{type casting}, vilket man ofta behöver kämpa med som Java-programmerare.

Dock finns fortfarande begränsningar och anomalier vad gäller till exempel likhetstest. Om du vill att en array ska bete sig som andra samlingar kan du enkelt ''wrappa'' den med metoden \code{toSeq} som vid anrop på arrayer ger en \code{ArraySeq}. Denna beter sig som en helt vanlig oföränderlig sekvenssamling utan att offra snabbheten hos en primitiv array.
\begin{Code}
val as = Array(1,2,3)
val xs = as.toSeq
\end{Code}
\Subtask Hur fungerar likhetstest mellan två \code{ArraySeq}s? Vad har \code{xs} ovan för typ? Går det att uppdatera en wrappad array?

\Subtask Vilken typ av argumentsekvens får du tillgång till i kroppen för en funktion med repeterande parametrar?

\Subtask\Uberkurs Läs här:
\url{http://docs.scala-lang.org/overviews/collections/arrays.html}
och ge ett exempel på vad mer man inte kan göra med en array, förutom innehållslikhetstest.



\SOLUTION

\TaskSolved \what~

\SubtaskSolved \code{xs} erbjuder innehållslikhet och har typen \code{Seq[Int]} med den underliggande typen \code{ArraySeq[Int]}. Det går inte att göra tilldelning av element i en \code{ArraySeq} eftersom metoden \code{update} saknas, och den är oföränderlig. Den uppdateras därför inte när den urspringliga arrayen uppdateras.

\begin{REPL}
scala> val as1 = Array(1,2,3)
val as1: Array[Int] = Array(1, 2, 3)

scala> val as2 = Array(1,2,3)
val as2: Array[Int] = Array(1, 2, 3)


scala> val (xs1, xs2) = (as1.toSeq, as2.toSeq)
val xs1: Seq[Int] = ArraySeq(1, 2, 3)
val xs2: Seq[Int] = ArraySeq(1, 2, 3)

scala> as1 == as2
val res0: Boolean = false

scala> xs1 == xs2
val res1: Boolean = true

scala> as1(0) = 42

scala> xs1
val res2: Seq[Int] = ArraySeq(1, 2, 3)

scala> xs1(0) = 42
value update is not a member of Seq[Int]
\end{REPL}

\SubtaskSolved Vid repeterade parametrar får man en \code{ArraySeq}.

\begin{REPL}
scala> def f(xs: Int*) = xs
def f(xs: Int*): Seq[Int]

scala> println(f(1,2,3))
ArraySeq(1, 2, 3)
\end{REPL}


\SubtaskSolved Det går inte att ha en generisk array som funktionsresultat utan att bifoga kontextgränsen \code{ClassTag} i typparametern för att kompilatorn ska kunna generera kod för den typkonvertering som krävs under runtime av JVM. Se exempel här:\\
\url{http://docs.scala-lang.org/overviews/collections/arrays.html}


\QUESTEND




\WHAT{List eller Vector?}

\QUESTBEGIN

\Task\Uberkurs  \what~ Jämför tidskomplexitet mellan List och Vector vid hantering i början och i slutet, baserat på efterföljande REPL-session i din egen körmiljö.  Körningen nedan gjordes på en AMD Ryzen 7 5800X (16) @ 3.800GHz under Arch Linux 5.12.8-arch1-1 med Scala 3.0.1 och openjdk 11.0.11, men du ska använda det du har på din dator.

Hur snabbt går nedan på din dator? När är List snabbast och när är Vector snabbast? Hur stor är skillnaderna i prestanda?
\footnote{Denna typ av mätningar lär du dig mer om i LTH-kursen ''Utvärdering av programvarusystem'', som ges i slutet av årskurs 1 för Datateknikstudenter.}
%sudo lshw -class processor


\begin{CodeSmall}
> head -5 /proc/cpuinfo
processor    : 0
vendor_id    : AuthenticAMD
cpu family    : 25
model        : 33
model name    : AMD Ryzen 7 5800X 8-Core Processor

scala> def time(n: Int)(block: => Unit): Double =                  
     |   def now = System.nanoTime
     |   var timestamp = now
     |   var sum = 0L
     |   var i = 0
     |   while i < n do
     |     block
     |     sum = sum + (now - timestamp)
     |     timestamp = now
     |     i = i + 1
     |   val average = sum.toDouble / n
     |   println("Average time: " + average + " ns")
     |   average


// Exiting paste mode, now interpreting.

time: (n: Int)(block: => Unit)Double


scala> val n = 100000
scala> val l = List.fill(n)(math.random())
scala> val v = Vector.fill(n)(math.random())

scala> (for i <- 1 to 20 yield time(n){l.take(10)}).min
average time: 97.66952 ns
average time: 91.90033 ns
average time: 79.88311 ns
average time: 69.5963 ns
average time: 69.69892 ns
average time: 69.8033 ns
average time: 69.7705 ns
average time: 69.68491 ns
average time: 69.54222 ns
average time: 69.66051 ns
average time: 69.73661 ns
average time: 69.54112 ns
average time: 69.69141 ns
average time: 69.46341 ns
average time: 69.4098 ns
average time: 61.34162 ns
average time: 41.1333 ns
average time: 40.97051 ns
average time: 40.9075 ns
average time: 41.12321 ns
val res0: Double = 40.9075

scala> (for i <- 1 to 20 yield time(n){v.take(10)}).min
average time: 84.56978 ns
average time: 75.20167 ns
average time: 57.16529 ns
average time: 34.84469 ns
average time: 34.38478 ns
average time: 34.77709 ns
average time: 34.77179 ns
average time: 35.0506 ns
average time: 34.7967 ns
average time: 35.04258 ns
average time: 34.82559 ns
average time: 36.3673 ns
average time: 34.91029 ns
average time: 34.87239 ns
average time: 34.51958 ns
average time: 34.83949 ns
average time: 34.56169 ns
average time: 34.80719 ns
average time: 34.84459 ns
average time: 34.89468 ns
val res1: Double = 34.38478

scala> (for i <- 1 to 20 yield time(1000){l.takeRight(10)}).min
average time: 131365.106 ns
average time: 118632.787 ns
average time: 118440.066 ns
average time: 118687.567 ns
average time: 118428.487 ns
average time: 118871.686 ns
average time: 118964.797 ns
average time: 119030.236 ns
average time: 119262.534 ns
average time: 119228.344 ns
average time: 119226.494 ns
average time: 119310.933 ns
average time: 119352.854 ns
average time: 119121.913 ns
average time: 119133.664 ns
average time: 119015.193 ns
average time: 119276.674 ns
average time: 119224.882 ns
average time: 119301.771 ns
average time: 119444.401 ns
val res2: Double = 118428.487

scala> (for i <- 1 to 20 yield time(1000){v.takeRight(10)}).min
average time: 805.989 ns
average time: 365.219 ns
average time: 225.49 ns
average time: 125.92 ns
average time: 124.98 ns
average time: 130.689 ns
average time: 139.86 ns
average time: 128.29 ns
average time: 132.59 ns
average time: 125.729 ns
average time: 125.46 ns
average time: 130.59 ns
average time: 122.03 ns
average time: 121.9 ns
average time: 119.69 ns
average time: 120.48 ns
average time: 125.239 ns
average time: 126.09 ns
average time: 125.92 ns
average time: 125.91 ns
val res3: Double = 119.69

\end{CodeSmall}

\noindent Varför går olika rundor i for-loopen olika snabbt även om varje runda gör samma sak?

\SOLUTION

\TaskSolved
Sekvenssamlingen \code{List} är nästan dubbelt så snabb vid bearbetning i början men ungefär 1000 gånger långsammare vid bearbetning i slutet av en sekvens med 100000 element.


Olika körningar går olika snabbt på JVM bl.a. p.g.a optimeringar som sker när JVM-en ''värms upp'' och den så kallade Just-In-Time-kompileringen gör sitt mäktiga jobb. Det går ibland plötsligt väsentligt långsammare när skräpsamlaren tvingas göra tidsödande storstädning av minnet.

\QUESTEND






\WHAT{Tidskomplexitet för olika samlingar i Scalas standardbibliotek.}

\QUESTBEGIN

\Task\Uberkurs  \what~\\
Studera skillnader i tidskomplexitet mellan olika samlingar här: \\ \href{http://docs.scala-lang.org/overviews/collections/performance-characteristics.html}{docs.scala-lang.org/overviews/collections/performance-characteristics.html} \\
Läs även kritiken av förenklingar i ovan beskrivning här:\\
\href{http://www.lihaoyi.com/post/ScalaVectoroperationsarentEffectivelyConstanttime.html}{www.lihaoyi.com/post/ScalaVectoroperationsarentEffectivelyConstanttime.html}
\\
Läs denna grundliga empirisk genomgång av prestanda i Scalas samlingsbibliotek:\\
\href{http://www.lihaoyi.com/post/BenchmarkingScalaCollections.html}{www.lihaoyi.com/post/BenchmarkingScalaCollections.html}
\\Du får lära dig mer om hur man resonerar kring komplexitet i kommande kurser.


\SOLUTION

\TaskSolved --

\QUESTEND


%\chapter{Ordlista}

%\chapter{Lösningar till övningarna}\label{chapter:solutions}
%\foreach \n in {1,...,9}{%
%  \input{modules/w0\n-solutions.tex}
%}
%\foreach \n in {10,...,14}{%
%  \input{modules/w\n-solutions.tex}
%}
%
%\chapter{Snabbreferens}\label{chapter:quickref}
%
%Detta appendix innehåller en snabbreferens för Scala och Java. Snabbreferensen är enda tillåtna hjälpmedel under kursens skriftliga tentamen.
%
%Lär dig vad som finns i snabbreferensen så att du snabbt hittar det du behöver och träna på hur du  effektivt kan dra nytta av den när du skriver program med papper och penna utan datorhjälpmedel.
%
%\clearpage
%~
%\clearpage
%
%\includepdf[pages={1-12}, scale=0.77, frame]{../quickref/quickref.pdf}


\end{document}
