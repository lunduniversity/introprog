%!TEX encoding = UTF-8 Unicode
\documentclass[a4paper]{compendium}
%\usepackage{xr} %to crossreference to compendium1.tex
\externaldocument{compendium1}
\usepackage[swedish]{babel}


\addto\captionsswedish{%
  \renewcommand{\appendixname}{Appendix}%
}
%TODO: Glossary
%http://tex.stackexchange.com/questions/5821/creating-a-standalone-glossary/5837#5837

\setlength{\columnsep}{16mm}

\newcommand{\LibVersion}{1.1.5} % latest version of introlib at https://github.com/lunduniversity/introprog-scalalib
\newcommand{\LibJar}{\texttt{introprog\_3-\LibVersion.jar}}
\newcommand{\JDKApiUrl}{\url{https://docs.oracle.com/en/java/javase/11/docs/api/}}
\newcommand{\CurrentYear}{2021}
\newcommand{\VMName}{vm2020} %TODO: update vm
\newcommand{\VMPassword}{pgkBytMig\CurrentYear}
\newcommand{\VirtualBoxVersion}{6.1} %https://www.virtualbox.org/wiki/Downloads
\newcommand{\UbuntuVersion}{20.04}
\newcommand{\ScalaVersion}{3.0.1} %https://www.scala-lang.org/
\newcommand{\SbtVersion}{1.5.3} %https://eed3si9n.com/category/tags/sbt
\newcommand{\JDKVersion}{11} %https://adoptopenjdk.net/
\newcommand{\KojoVersion}{2.9.10} %https://www.kogics.net/kojo-download
\newcommand{\VSCodeVersion}{1.41} %https://code.visualstudio.com/updates
\newcommand{\MetalsVersion}{v1.10.6} %https://marketplace.visualstudio.com/items?itemName=scalameta.metals
\newcommand{\WindowsVersion}{10}
\newcommand{\ScalaIDEVersion}{4.7.0} %%DEPRECATED




\title{
{\vspace{-3.0cm}\bf\sffamily\Huge\selectfont  Introduktion till programmering med Scala}
\\ \vspace{2em}%\hspace*{1.5cm}\inputgraphics[width=0.6\textwidth]{../img/gurka} \\
{\sffamily \textbf{Kompendium 2}\\Andra läsperioden: Modul 8 -- 14}\\\vspace{2cm}
\includegraphics[height=11cm]{../img/glider-blinker-block}
%\includegraphics[height=4cm]{../img/scala-logo.png}
%\includegraphics[height=4cm]{../img/java-logo.png}
%\includegraphics[height=12cm]{cover/gurka.jpg}
}

\author{Björn Regnell}
\date{\raggedbottom%
\vspace{1em}\begin{minipage}{1.0\textwidth}\centering
EDAA45, Lp1-2, HT \CurrentYear\\
Datavetenskap, LTH\\
Lunds universitet\\
~\\
Kompileringsdatum: \today \\
\url{http://cs.lth.se/pgk}
\end{minipage}
}

\usepackage{multicol}

\usepackage{pgffor}  %% http://stackoverflow.com/questions/2561791/iteration-in-latex
%  allows:  \foreach \n in {1,...,4}{ do something with \n }

\usepackage{framed}  %  allows:   \begin{framed}\end{framed}
\FrameSep5pt
\OuterFrameSep0pt

% \newenvironment{Slide}[2][]{%
% \begin{oframed}\setlist{noitemsep}%
% {\vspace{-1.5\topsep}}%tighter frames
% \subsection{#2}%
% }%
% {\end{oframed}} 

\newenvironment{Slide}[2][]{%
%\noindent\rule{\textwidth}{0.4pt}%
\setlist{noitemsep}%
%{\vspace{-1.5\topsep}}%tighter frames
\subsection{#2}%
}%
{~\newline\noindent\rule{\textwidth}{0.4pt}}

% \newcommand{\SlideHeading}[1]{\section*{#1}}

% \usepackage[most]{tcolorbox}
% \newenvironment{Slide}[2][]
%   {\vspace{0.5em}\begin{tcolorbox}[left=1.5em,%width=1.05\textwidth,
%   grow to right by=0.05\textwidth,grow to left by=0.05\textwidth,%
%   %breakable,
%   %frame hidden,
%   colframe=gray!20,
%   enhanced]\setlist{noitemsep}\SlideHeading{#2}}
%   {\end{tcolorbox}\vspace{0.5em}}

\newcommand{\Subsection}[1]{} %ignore slide sections
\newcommand{\SlideOnly}[1]{} %ignore slide font size

\usepackage[framemethod=tikz]{mdframed}


\newif\ifkompendium  % to allow conditional text in slides only showing up in compendium
\kompendiumtrue      % in slides: \kompendiumfalse

\newif\ifPreSolution  % to allow tasks and solutions in same file
\PreSolutiontrue      % in solutions: \PreSolutionfalse

\let\QUESTBEGIN\ifPreSolution  % to mark formatting and numbering of exercises
\let\SOLUTION\else  % to mark solutions in the same file as questions
\let\QUESTEND\fi    % to mark end of exercise

%!TEX encoding = UTF-8 Unicode
\newcommand{\ExeWeekONE}{expressions}
\newcommand{\LabWeekONE}{kojo}

\newcommand{\ExeWeekTWO}{programs}
\newcommand{\LabWeekTWO}{--}

\newcommand{\ExeWeekTHREE}{functions}
\newcommand{\LabWeekTHREE}{irritext}

\newcommand{\ExeWeekFOUR}{objects}
\newcommand{\LabWeekFOUR}{blockmole}

\newcommand{\ExeWeekFIVE}{classes}
\newcommand{\LabWeekFIVE}{turtlegraphics}

\newcommand{\ExeWeekSIX}{sequences}
\newcommand{\LabWeekSIX}{shuffle}

\newcommand{\ExeWeekSEVEN}{sets-maps}
\newcommand{\LabWeekSEVEN}{words}

\newcommand{\ExeWeekEIGHT}{matrices}
\newcommand{\LabWeekEIGHT}{maze}

\newcommand{\ExeWeekNINE}{inheritance}
\newcommand{\LabWeekNINE}{turtlerace-team}

\newcommand{\ExeWeekTEN}{patterns}
\newcommand{\LabWeekTEN}{chords-team}

\newcommand{\ExeWeekELEVEN}{scala-java}
\newcommand{\LabWeekELEVEN}{lthopoly-team}

\newcommand{\ExeWeekTWELVE}{sorting}
\newcommand{\LabWeekTWELVE}{survey}

\newcommand{\ExeWeekTHIRTEEN}{--}
\newcommand{\LabWeekTHIRTEEN}{Projekt}

\newcommand{\ExeWeekFOURTEEN}{threads}
\newcommand{\LabWeekFOURTEEN}{--}


\begin{document}

\pagenumbering{roman}

\frontmatter
\maketitle
%!TEX root = ../compendium.tex

\clearpage\null\thispagestyle{empty}
\vfill

{
\setlength{\parindent}{0pt}
\emph{Editor}: Björn Regnell, Faculty of Engineering LTH, Lund University. \\ 

\emph{Contributors}: 
Björn Regnell,
Per Holm,
Sandra Nilsson,
Patrik Andersson,
Gustav Cedersjö,
Maj Stenmark,
Anna Axelsson,
Roy Andersson,
Markus Borg,
Anton Klarén.
\\

\emph{Repo}: \url{https://github.com/lunduniversity/introprog} \\ \newline

This manuscript is on-going work. Contributions are welcome! \\ 
\emph{Contact}: \url{bjorn.regnell@cs.lth.se}
\\ \newline

\emph{LICENCE}: CC BY-NC-SA 4.0 \\
\url{http://creativecommons.org/licenses/by-nc-sa/4.0/}
\\ \newline
Copyright \copyright~Computer Science, LTH \& Björn Regnell. 2016. Lund. Sweden.\\
}

%!TEX encoding = UTF-8 Unicode
%!TEX root = ../compendium.tex

\ChapterUnnum{Framstegsprotokoll}\label{progress-protocoll}


\section*{Genomförda övningar}

\vspace{1em}\noindent
{Till varje laboration hör en övning med uppgifter som utgör förberedelse inför labben. Du behöver minst behärska grunduppgifterna för att klara labben inom rimlig tid. Om du känner att du behöver öva mer på grunderna, gör då även extrauppgifterna. Om du vill fördjupa dig, gör fördjupningsuppgifterna som är på mer avancerad nivå. Kryssa för nedan vilka övningar du har gjort, så blir det lättare för din handledare att anpassa dialogen till de kunskaper du förvärvat hittills.}

\newcommand{\TickBox}{\raisebox{-.50ex}{\Large$\square$}}
\newcommand{\ExeRow}[1]{\hyperref[section:exe:#1]{\texttt{#1}} & \TickBox  &  \TickBox &  \TickBox  \\ \addlinespace }

\begin{table}[h]
%\centering
\vspace{2em}
\begin{tabular}{lccc}
\toprule \addlinespace
{\sffamily Övning} &
{\sffamily Grund} &
{\sffamily Extra} &
{\sffamily Fördjupning}\\ \addlinespace \midrule \\[-0.7em]
\ExeRow{expressions}
\ExeRow{programs}
\ExeRow{functions}
\ExeRow{data}
\ExeRow{vectors}
\ExeRow{classes}
\ExeRow{traits}
\ExeRow{matching}
\ExeRow{matrices}
\ExeRow{sorting}
\ExeRow{scalajava}
\ExeRow{threads}
\bottomrule
\end{tabular}
\end{table}

\newpage

\section*{Godkända obligatoriska moment}

\vspace{1em}\noindent
För att bli godkänd på laborationsuppgifterna och projektuppgiften måste du lösa deluppgifterna och diskutera dina lösningar med en handledare. Denna diskussion är din möjlighet att få feedback på dina lösningar. Ta vara på den!
Se till att handledaren noterar nedan när du blivit godkänd på respektive obligatorisk moment. Spara detta blad tills du fått slutbetyg i kursen.


\vspace{2.5em}\noindent Namn: \dotfill\\

\vspace{1em}\noindent Namnteckning: \dotfill\\

\newcommand{\LabRow}[1]{\\[-1.1em] \hyperref[section:lab:#1]{\texttt{#1}} & \dotfill &  \dotfill  \\ \addlinespace }

\begin{table}[h]
%\centering
\vspace{1em}
\begin{tabular}{lcc}
\toprule \addlinespace
{\sffamily\bfseries\small Lab} & {\sffamily\small Datum gk} &	
{\sffamily\small Handledares signatur + namnförtydligande}\\ \addlinespace 
%\midrule 
\\[-0.5em]
%!TEX encoding = UTF-8 Unicode
%!TEX root = ../compendium2.tex
\LabRow{kojo}
\LabRow{irritext}
\LabRow{blockmole}
\LabRow{blockbattle}
\LabRow{shuffle}
\LabRow{words}
\LabRow{life}
\LabRow{snake}
\LabRow{tabular}
\LabRow{javatext}
%\toprule
\addlinespace 
%\midrule 
\addlinespace\addlinespace
{\sffamily\small {\bfseries Projektuppgift} (välj en)	} & \dotfill &  \dotfill  \\
\addlinespace\addlinespace %\midrule
{\Large$\square$}\texttt{~~~\hyperref[section:proj:bank]{bank}} &
\multicolumn{2}{c}{\textit{Om egendef., ge kort beskrivning här:}}  \\ \addlinespace
{\Large$\square$}\texttt{~~~\hyperref[section:proj:tabular]{tabular}} \\ \addlinespace
{\Large$\square$}\texttt{~~~\hyperref[section:proj:music]{music}} \\ \addlinespace
{\Large$\square$}\texttt{~~~\hyperref[section:proj:photo]{photo}}  \\ \addlinespace
{\Large$\square$}\texttt{~~~}\textit{egendefinerad}  \\
%\dotfill  \\
\addlinespace\addlinespace
%\midrule
\addlinespace
{\sffamily\small {\bfseries Muntligt prov}} &  & \\
\addlinespace\addlinespace{}
{\Large$\square$}\texttt{~~~} godkänd & \dotfill &  \dotfill \\
\addlinespace\addlinespace\bottomrule
\end{tabular}
\end{table}

%!TEX encoding = UTF-8 Unicode
%!TEX root = ../compendium1.tex


\ChapterUnnum{Förord}

Detta kompendium innehåller övningar och laborationer och övningslösningar för andra läsperioden i LTH:s grundkurs i programmering för civilingenjörsprogrammet Datateknik.


Vi avslutade första läsperioden med en diagnostisk kontrollskrivning där du fick återkoppling på vad du lärt dig hittills. Det är viktigt att du använder dina lärdomar om vad du behöver träna mer på och direkt gör upp en plan för hur du kan befästa din förståelse för begreppen i första läsperioden, så att du hänger med under kommande läsperiod.

Det övergripande målet för den andra läsperioden är att du ska kunna skapa egna program som löser mer omfattande problem än tidigare, genom att kombinera flera abstraktionsmekanismer och begrepp. Vi inför även nya abstraktionsmekanismer (t.ex. arv), nya språkmekanismer (t.ex. mönstermatching), samt jämför och kombinerar Scala och Java. Läsperioden avslutas med ett individuellt projektarbete där du får möjlighet att fördjupa dig enligt dina egna intressen och önskemål.

Kompendiet är framtaget för och av studenter och lärare, och distribueras som öppen källkod. Det får användas fritt så länge erkännande ges och eventuella ändringar publiceras under samma licens som ursprungsmaterialet. På kurshemsidan \href{http://cs.lth.se/pgk}{cs.lth.se/pgk} och i kursrepot \href{http://github.com/lunduniversity/introprog}{github.com/lunduniversity/introprog} finns instruktioner om hur du kan bidra till kursmaterialet.

Välkommen till andra halvlek!

\vspace{1em}\noindent \textit{\hfill Lund, \today, Björn Regnell}


\setcounter{tocdepth}{2} % set headings level in table of contents
\tableofcontents
\mainmatter

\pagenumbering{arabic}


\part{Modulöversikt}

\begin{table}
\noindent\resizebox{1.0\columnwidth}{!}{
\renewcommand{\arraystretch}{2.0}
%!TEX encoding = UTF-8 Unicode
\begin{tabular}{l|l|l|l}
\textit{W} & \textit{Modul} & \textit{Övn} & \textit{Lab} \\ \hline \hline
W01 & Introduktion & expressions & kojo \\
W02 & Kodstrukturer & programs & -- \\
W03 & Funktioner, objekt & functions & blockmole \\
W04 & Datastrukturer & data & pirates \\
W05 & Sekvensalgoritmer & sequences & shuffle \\
W06 & Klasser & classes & turtlegraphics \\
W07 & Arv & traits & turtlerace-team \\
KS & KONTROLLSKRIVN. & -- & -- \\
W08 & Repetition, trösklar, luckor & reboot-init & reboot-check \\
W09 & Mönster, undantag & matching & chords-team \\
W10 & Matriser, typparametrar & matrices & maze \\
W11 & Sökning, sortering & sorting & survey \\
W12 & Scala och Java & scalajava & lthopoly-team \\
W13 & Extra: design, api, trådar, webb & threads & Projekt \\
W14 & Tentaträning & Extenta & -- \\
T & TENTAMEN & -- & -- \\
\end{tabular}

}
\end{table}
\clearpage

\hyphenation{intro-duktion sekvens-algoritmer kod-strukturer data-strukturer}
{\fontsize{11}{12}\selectfont
\renewcommand{\arraystretch}{1.75}
\begin{longtable}{@{}p{.05\textwidth} | >{\hspace{0.1em}\raggedright\bfseries\sffamily}p{.15\textwidth}  >{\raggedleft\arraybackslash\hspace{0.0em}%\fontsize{10.5}{12}\selectfont
}p{0.735\textwidth}}
W01 & Introduktion & sekvens, alternativ, repetition, abstraktion, programmeringsspråk, programmeringsparadigmer, editera-kompilera-exekvera, datorns delar, virtuell maskin, REPL, literal, värde, uttryck, identifierare, variabel, typ, tilldelning, namn, val, var, def, inbyggda grundtyper, Int, Long, Short, Double, Float, Byte, Char, String, println, typen Unit, enhetsvärdet (), stränginterpolatorn s, if, else, true, false, MinValue, MaxValue, aritmetik, slumptal, math.random, logiska uttryck, de Morgans lagar, while-sats, for-sats \\
W02 & Kodstrukturer & iterering, for-uttryck, map, foreach, Range, Array, Vector, algoritm vs implementation, pseudokod, algoritm: SWAP, algoritm: SUM, algoritm: MIN/MAX, algoritm: MININDEX, block, namnsynlighet, namnöverskuggning, lokala variabler, paket, import, filstruktur, jar, dokumentation, programlayout, JDK, main i Java vs Scala, java.lang.System.out.println \\
W03 & Funktioner & definera funktion, anropa funktion, parameter, returtyp, värdeandrop, namnanrop, default-argument, namngivna argument, applicera funktion på alla element i en samling, procedur, värdeanrop vs namnanrop, uppdelad parameterlista, skapa egen kontrollstruktur, funktionsvärde, funktionstyp, äkta funktion, stegad funktion, apply, lazy val, lokala funktioner, anonyma funktioner, lambda, aktiveringspost, anropsstacken, objektheapen, rekursion, cslib.window.SimpleWindow \\
W04 & Objekt & objekt, modul, paket, punktnotation, tillstånd, metod, medlem, funktioner är objekt, cslib.window.SimpleWindow \\
W05 & Klasser & objektorientering, klass, Point, Square, Complex, new, null, this, inkapsling, accessregler, private, private[this], kompanjonsobjekt, getters och setters, klassparameter, primär konstruktor, objektfabriksmetod, överlagring av metoder, referenslikhet vs strukturlikhet, eq vs == \\
W06 & Sekvensalgoritmer & sekvensalgoritm, algoritm: SEQ-COPY, in-place vs copy, algoritm: SEQ-REVERSE, algoritm: SEQ-REGISTER, sekvenser i Java vs Scala, for-sats i Java, java.util.Scanner, scala.collection.mutable.ArrayBuffer, StringBuilder, java.util.Random, slumptalsfrö \\
W07 & Datastrukturer & attribut (fält), medlem, metod, tupel, klass, Any, isInstanceOf, toString, case-klass, samling, scala.collection, föränderlighet vs oföränderlighet, List, Vector, Set, Map, typparameter, generisk samling som parameter, översikt samlingsmetoder, översikt strängmetoder, läsa/skriva textfiler, Source.fromFile, java.nio.file \\
KS & \multicolumn{2}{l}{KONTROLLSKRIVN.} \\
W08 & Matriser, typparametrar & matris, nästlad samling, nästlad for-sats, typparameter, generisk funktion, generisk klass, fri vs bunden typparameter, matriser i Java vs Scala, allokering av nästlade arrayer i Scala och Java \\
W09 & Arv & arv, polymorfism, trait, extends, asInstanceOf, with, inmixning, supertyp, subtyp, bastyp, override, klasshierarkin i Scala: Any AnyRef Object AnyVal Null Nothing, referenstyper vs värdetyper, klasshierarkin i scala.collection, Shape som bastyp till Rectangle och Circle, accessregler vid arv, protected, final, klass vs trait, abstract class, case-object, typer med uppräknade värden, gränssnitt, trait vs interface, programmeringsgränssnitt (api) \\
W10 & Mönster, undantag, likhet & mönstermatchning, match, Option, throw, try, catch, Try, unapply, sealed, flatten, flatMap, partiella funktioner, collect, speciella matchningar: wildcard pattern; variable binding; sequence wildcard; back-ticks, equals, hashcode, exempel: equals för klassen Complex, switch-sats i Java \\
W11 & Scala och Java & syntaxskillnader mellan Scala och Java, klasser i Scala vs Java, referensvariabler vs enkla värden i Java, referenstilldelning vs värdetilldelning i Java, alternativ konstruktor i Scala och Java, for-sats i Java, for-each-sats i Java, java.util.ArrayList, autoboxing i Java, primitiva typer i Java, wrapperklasser i Java, samlingar i Java vs Scala, scala.collection.JavaConverters, namnkonventioner för konstanter \\
W12 & Sökning, sortering, ordning & strängjämförelse, compareTo, implicit ordning, linjärsökning, binärsökning, algoritm: LINEAR-SEARCH, algoritm: BINARY-SEARCH, algoritmisk komplexitet, sortering till ny vektor, sortering på plats, insättningssortering, urvalssortering, algoritm: INSERTION-SORT, algoritm: SELECTION-SORT, Ordering[T], Ordered[T], Comparator[T], Comparable[T] \\
W13 & \multicolumn{2}{l}{Repetition, tentaträning, projekt} \\
W14 & Extra: jämlöpande exekvering & tråd, jämlöpande exekvering, icke-blockerande anrop, callback, java.lang.Thread, java.util.concurrent.atomic.AtomicInteger, scala.concurrent.Future, kort om html+css+javascript+scala.js och webbprogrammering \\
T & \multicolumn{2}{l}{TENTAMEN} \\
\end{longtable}
}

%\renewcommand{\SlideHeading}[1]{\subsection{#1}}  %numbering sections in compendium slides

\part{Moduler}

\setcounter{chapter}{7}

%!TEX encoding = UTF-8 Unicode

%!TEX root = ../compendium2.tex

\chapter{Mönster, Undantag}\label{chapter:W08}
\begin{itemize}[nosep]
\item match
\item Option
\item null
\item try
\item catch
\item Try
\item unapply
\end{itemize}
\clearpage\section{Teori}
%!TEX encoding = UTF-8 Unicode
%!TEX root = ../lect-w08.tex

%%%

\Subsection{Veckans labb: \texttt{life}}

\begin{Slide}{Veckans labb: \texttt{life}}
\begin{minipage}{0.52\textwidth}
  \setlength{\leftmargini}{0pt}

\begin{itemize}
  \SlideFontSmall
\item Universum är en binär matris av \Emph{celler} där \Emph{levande} celler representeras med \code{true} och \Alert{döda} med \code{false}.
\item Följande regler gäller för \Emph{nästa generation} celler i universum:
\begin{itemize}\SlideFontTiny
  \item \textbf{Fortlevnad}: en levande cell med 2 eller 3 grannar \Emph{lever vidare}
  \item \textbf{Död}: en levande cell med färre än 2 eller fler än 3 grannar \Alert{dör}
  \item \textbf{Födelse}: en död cell med exakt tre grannar föds
\end{itemize}
\item Övning \code{matrices} uppgift 5: skapa en generisk \code{case class Matrix[T]}
\item På labben: använd \code{Matrix[Boolean]}
\end{itemize}

\end{minipage}%
\begin{minipage}{0.5\textwidth}
  \includegraphics[width=1.0\textwidth]{../img/glider-blinker-block}

  \begin{itemize}\SlideFontTiny
  \item Du ska simulera \emph{Game of Life} i ett \code{introprog.PixelWindow}
  \item Fördjupning:\\{\SlideFontTiny\url{https://en.wikipedia.org/wiki/Conway%27s_Game_of_Life}}
  \end{itemize}
\end{minipage}%

\end{Slide}






\Subsection{Matriser}

\begin{Slide}{Vad är en matris?}\SlideFontSmall
\begin{itemize}

\item En \Emph{matris} inom \Alert{matematiken} innehåller \Emph{rader} och \Emph{kolumner}\footnote{även kallade \emph{kolonner}} med tal.

\item I en \Alert{matematisk} matris har alla rader \Emph{lika många} element och

\item även alla kolumner har \Emph{lika många} element.

\item En matris av dimension $2\times{}5$ har $2 \cdot 5 = 10$ stycken element.

\item Exempel på en matematisk matris av dimension $2\times{}5$:
\[
M_{2,5}=
  \begin{pmatrix}
    5 & 2 & 42 & 4 & 5 \\
    3 & 4 & 18 & 6 & 7
  \end{pmatrix}
\]
\end{itemize}
\end{Slide}

\begin{Slide}{Indexering i en matris}\SlideFontSmall
\begin{itemize}

  \item En matris av dimension $m\times{}n$ har $m \cdot n$ stycken element.

  \item En matris $A_{m,n}$ av dimension $m\times{}n$ ritas inom matematiken ofta så här:

  \[
  A_{m,n} =
   \begin{pmatrix}
    a_{1,1} & a_{1,2} & \cdots & a_{1,n} \\
    a_{2,1} & a_{2,2} & \cdots & a_{2,n} \\
    \vdots  & \vdots  & \ddots & \vdots  \\
    a_{m,1} & a_{m,2} & \cdots & a_{m,n}
   \end{pmatrix}
  \]


\item Matrisindexering inom matematiken sker ofta från $1$, men ofta från $0$ i datorprogram.

\item Vad har talet $42$ för index i matrisen $M_{2,5}$ nedan?
\begin{itemize}\SlideFontTiny
  \item[--] Inom matematiken?
  \item[--] I Scala och Java och många andra språk?

  \[
  M_{2,5}=
    \begin{pmatrix}
      5 & 2 & 42 & 4 & 5 \\
      3 & 4 & 18 & 6 & 7
    \end{pmatrix}
  \]
\end{itemize}
\end{itemize}
\end{Slide}

\begin{Slide}{Hur skapa matriser?}
  \setlength{\leftmargini}{0pt}

  \begin{itemize}
  \item Inom programmering används ordet \Emph{matris} ofta för att beteckna en \Alert{nästlad struktur} i två dimensioner. Exempel:
  \begin{itemize}
   \item \Emph{Oföränderliga} sekvenser, t.ex. \code{Vector[Vector[Int]]} \\
   \code{val xss = Vector(Vector(0, 0, 0), Vector(0, 0, 0))} eller enklare: \\
      \code{val xss = Vector.fill(2,3)(0)}

    \item \Alert{Föränderliga} sekvens, t.ex. \code{Array[Array[Int]]} \\
    \code{val yss = Array(Array(0, 0, 0), Array(0, 0, 0))} eller enklare: \\
       \code{val yss = Array.fill(2,3)(0)}

  \end{itemize}

\end{itemize}
\end{Slide}

\begin{Slide}{Hur indexera i matriser?}
En matris med array av arrayer:
\begin{REPL}
scala> val xss = Array(Array(5,2,42,4,5),Array(3,4,18,6,7))
xss: Array[Array[Int]] = Array(Array(5, 2, 42, 4, 5), Array(3, 4, 18, 6, 7))
\end{REPL}
\pause
Man indexerar i en nästlad sekvens med upprepad \code{apply}:
\begin{REPL}
scala> xss(0)(2)
res0: ???

scala> xss.apply(0).apply(2)
res1: ???

scala> xss(0)
res2: ???
\end{REPL}
Övning: Vad är typ och värde vid \code{???} ovan?
\end{Slide}

\begin{Slide}{Hur indexera i matriser?}
En matris med array av arrayer:
\begin{REPL}
scala> val xss = Array(Array(5,2,42,4,5),Array(3,4,18,6,7))
xss: Array[Array[Int]] = Array(Array(5, 2, 42, 4, 5), Array(3, 4, 18, 6, 7))
\end{REPL}

Man indexerar i en nästlad sekvens med upprepad \code{apply}:
\begin{REPL}
scala> xss(0)(2)
res0: Int = 42

scala> xss.apply(0).apply(2)
res1: Int = 42

scala> xss(0)
res2: Array[Int] = Array(5, 2, 42, 4, 5)
\end{REPL}
Övning: Rita en bild av minnet som referensen \code{xss} refererar till.

\end{Slide}

\begin{Slide}{Uppdatering av en förändringsbar nästlad struktur}
Man kan förändra en array av arrayer ''på plats'' med tilldelning:
\begin{REPL}
scala> val xss = Array(Array(5,2,42,4,5),Array(3,4,18,6,7))

scala> xss(0)(0) = 100

scala> xss
res0: ???

scala> xss(0)(2) = xss(0)(2) - 1

scala> xss
res1: ???

scala> xss(1) = Array.fill(5)(-1)

scala> xss
res2: ???
\end{REPL}
\end{Slide}

\begin{Slide}{Uppdatering av en förändringsbar nästlad struktur}
Man kan förändra en array av arrayer ''på plats'' med tilldelning:
\begin{REPL}
scala> val xss = Array(Array(5,2,42,4,5),Array(3,4,18,6,7))

scala> xss(0)(0) = 100

scala> xss
res0: Array[Array[Int]]=Array(Array(100, 2, 42, 4, 5), Array(3, 4, 18, 6, 7))

scala> xss(0)(2) = xss(0)(2) - 1

scala> xss
res1: Array[Array[Int]]=Array(Array(100, 2, 41, 4, 5), Array(3, 4, 18, 6, 7))

scala> xss(1) = Array.fill(5)(-1)

scala> xss
res2: Array[Array[Int]]=Array(Array(100, 2, 41, 4, 5), Array(-1,-1,-1,-1,-1))
\end{REPL}
\end{Slide}

\begin{Slide}{Några olika sätt att skapa förändringsbara matriser}\SlideFontSmall
Det jobbiga, primitiva sättet:
\begin{REPL}
scala> val xss = new Array[Array[Int]](2)
xss: Array[Array[Int]] = Array(null, null)

scala> for (i <- xss.indices) {xss(i) = new Array[Int](5)}

scala> xss
res0: Array[Array[Int]] = Array(Array(0, 0, 0, 0, 0), Array(0, 0, 0, 0, 0))

scala> println(xss)
[[I@196a99d0
\end{REPL}
Enklare sätt:
\begin{REPL}
scala> val xss = Array.ofDim[Int](2,5)
xss: Array[Array[Int]] = Array(Array(0, 0, 0, 0, 0), Array(0, 0, 0, 0, 0))
\end{REPL}
Enklare och tydligare sätt, där initialvärdet anges explicit:
\begin{REPL}
scala> val xss = Array.fill(2,5)(0)
xss: Array[Array[Int]] = Array(Array(0, 0, 0, 0, 0), Array(0, 0, 0, 0, 0))
\end{REPL}

\end{Slide}

\begin{Slide}{Exempel på skapande av oföränderlig nästlad struktur}\SlideFontSmall
Om du kan beräkna initialvärde direkt, använd \code{Vector.fill}:\\
{\SlideFontTiny\code{def fill[A](n1: Int, n2: Int)(elem: => A): Vector[Vector[A]]}}
\begin{REPL}
scala> Vector.fill(2,5)(scala.util.Random.nextInt(6) + 1)
res0:
  typ???
  värde???

\end{REPL}
Om du kan beräkna initialvärde ur index, använd \code{Vector.tabulate}:\\
{\SlideFontTiny\code{def tabulate[A](n1: Int, n2: Int)(f: (Int, Int) => A): Vector[Vector[A]]}}
\begin{REPL}
scala> Vector.tabulate(5,2)((x,y) => x + y + 1)
res1:
  typ???
  värde???

\end{REPL}
\end{Slide}

\begin{Slide}{Exempel på skapande av oföränderlig nästlad struktur}\SlideFontSmall
Om du kan beräkna initialvärde direkt, använd \code{Vector.fill}:\\
{\SlideFontTiny\code{def fill[A](n1: Int, n2: Int)(elem: => A): Vector[Vector[A]]}}
\begin{REPL}
scala> Vector.fill(2,5)(scala.util.Random.nextInt(6) + 1)
res0: Vector[Vector[Int]] =
  Vector(Vector(1, 2, 6, 2, 1), Vector(1, 4, 3, 3, 2))

\end{REPL}
Om du kan beräkna initialvärde ur index, använd \code{Vector.tabulate}:\\
{\SlideFontTiny\code{def tabulate[A](n1: Int, n2: Int)(f: (Int, Int) => A): Vector[Vector[A]]}}
\begin{REPL}
scala> Vector.tabulate(5,2)((x,y) => x + y + 1)
res1: Vector[Vector[Int]] =
  Vector(Vector(1,2), Vector(2,3), Vector(3,4), Vector(4,5), Vector(5,	6))

\end{REPL}
\end{Slide}



\begin{Slide}{Uppdatering av en oföränderlig nästlad struktur}\SlideFontSmall
Uppdatering av endimensionell struktur med \code{xs.updated}:\\
{\SlideFontTiny\code{def updated[A](index: Int, elem: A): Vector[A]} }
\begin{REPL}
scala> var xs = Vector.tabulate(5)(x => x + 1)
xs: typ??? = värde???

scala> xs = xs.updated(1, 42)
xs: typ??? = värde???
\end{REPL}

Uppdatering av nästlad struktur i två dimensioner:
\begin{REPL}
scala> var xss = Vector.tabulate(2, 5)((x,y) => x + y + 1)
xss:
  typ??? =
  värde???

scala> xss = xss.updated(0, xss(0).updated(1, 42))
xss:
  typ??? =
  värde???
\end{REPL}

\end{Slide}



\begin{Slide}{Uppdatering av en oföränderlig nästlad struktur}\SlideFontSmall
Uppdatering av endimensionell struktur med \code{xs.updated}:\\
{\SlideFontTiny\code{def updated[A](index: Int, elem: A): Vector[A]} }
\begin{REPL}
scala> var xs = Vector.tabulate(5)(x => x + 1)
xs: Vector[Int] = Vector(1, 2, 3, 4, 5)

scala> xs = xs.updated(1, 42)
xs: Vector[Int] = Vector(1, 42, 3, 4, 5)
\end{REPL}

Uppdatering av nästlad struktur i två dimensioner:
\begin{REPL}
scala> var xss = Vector.tabulate(2, 5)((x,y) => x + y + 1)
xss: Vector[Vector[Int]] =
  Vector(Vector(1, 2, 3, 4, 5), Vector(2, 3, 4, 5, 6))

scala> xss = xss.updated(0, xss(0).updated(1, 42))
xss:
  Vector[Vector[Int]] =
  Vector(Vector(1, 42, 3, 4, 5), Vector(2, 3, 4, 5, 6))
\end{REPL}

\end{Slide}


\begin{Slide}{Iterera över nästlad struktur}\SlideFontSmall
Behandling av nästlade strukturer kräver ofta algoritmer med nästlad iterering. \\
Exempel: iterera med nästlad \code{for}-sats för utskrift av denna matris\\
\code{val xss = Vector.tabulate(2,5)((x,y) => x + y + 1)}
\pause
\begin{REPL}
scala> for ??? do
         for ??? do 
           print(xss(i)(j))
           print(" ")
         println

1 2 3 4 5
2 3 4 5 6
\end{REPL}
Övning: \\Vad ska det stå vid \code{???} för att alla element ska skrivas ut?
\end{Slide}

\begin{Slide}{Iterera över nästlad struktur}\SlideFontSmall
  \vspace{1em}
  Behandling av nästlade strukturer kräver ofta algoritmer med nästlad iterering. \\
  Exempel: iterera med nästlad \code{for}-sats för utskrift av denna matris \\
  \code{val xss = Vector.tabulate(2,5)((x,y) => x + y + 1)}

  \begin{REPL}
scala> for xs <- xss do
         for x <- xs do 
           print(x)
           print(" ")
         end for
         println()
       end for

1 2 3 4 5
2 3 4 5 6
\end{REPL}
Övning: skriv ut matrisen med nästlad \code{foreach}\\
\pause
\begin{Code}
xss.foreach { xs => 
  xs.foreach { x => print(x); print(" ") }
  println()
}
\end{Code}
\end{Slide}


\begin{Slide}{Övningsexempel: Yatzy}\SlideFontSmall
Skapa en funktion \code{roll} som ger utfallet av n st tärningskast:
\begin{REPL}
scala> import scala.util.Random

scala> def roll(n: Int): Vector[Int] = ???
\end{REPL}

Skapa en funktion \code{isYatzy} som ger \code{true} om alla utfall är lika:
\begin{REPL}
scala> def isYatzy(xs: Vector[Int]): Boolean = ???
\end{REPL}
Du kan anta att xs.length > 0\\
Tips: använd metoden xs.forall: \\
\code{def forall[A](p: A => Boolean): Boolean }
\end{Slide}


\begin{Slide}{Övningsexempel: Yatzy}\SlideFontSmall
Skapa en funktion \code{roll} som ger utfallet av n st tärningskast:
\begin{REPL}
scala> import scala.util.Random

scala> def roll(n: Int): Vector[Int] = Vector.fill(n)(Random.nextInt(6) + 1)
\end{REPL}

Skapa en funktion \code{isYatzy} som ger \code{true} om alla utfall är lika:
\begin{REPL}
scala> def isYatzy(xs: Vector[Int]): Boolean = xs.forall(x => x == xs(0))
\end{REPL}
Du kan anta att xs.length > 0\\
Tips: använd metoden xs.forall: \\
\code{def forall[A](p: A => Boolean): Boolean }
\end{Slide}

\begin{Slide}{Iterera över nästlad struktur: for-sats}\SlideFontSmall
Iterera med nästlad for-sats: (vad har xss för typ?)
\begin{REPL}
scala> val xss = Vector.fill(100)(roll(5))

scala> for (i <- ???) do 
         for (j <- ???) do
           print(s"($i)($j): ${xss(i)(j)} ")
         println(s" YATZY: ${isYatzy(xss(i))}")

(0)(0): 3 (0)(1): 6 (0)(2): 4 (0)(3): 4 (0)(4): 6  YATZY: false
(1)(0): 4 (1)(1): 1 (1)(2): 5 (1)(3): 2 (1)(4): 6  YATZY: false
(2)(0): 1 (2)(1): 3 (2)(2): 5 (2)(3): 6 (2)(4): 2  YATZY: false
(3)(0): 2 (3)(1): 1 (3)(2): 1 (3)(3): 5 (3)(4): 4  YATZY: false
(4)(0): 4 (4)(1): 4 (4)(2): 1 (4)(3): 6 (4)(4): 5  YATZY: false
(5)(0): 3 (5)(1): 3 (5)(2): 2 (5)(3): 3 (5)(4): 6  YATZY: false
(6)(0): 3 (6)(1): 6 (6)(2): 1 (6)(3): 1 (6)(4): 4  YATZY: false
(7)(0): 6 (7)(1): 2 (7)(2): 4 (7)(3): 4 (7)(4): 3  YATZY: false
(8)(0): 1 (8)(1): 5 (8)(2): 4 (8)(3): 2 (8)(4): 4  YATZY: false
(9)(0): 1 (9)(1): 1 (9)(2): 3 (9)(3): 6 (9)(4): 6  YATZY: false
(10)(0): 2 (10)(1): 5 (10)(2): 2 (10)(3): 4 (10)(4): 5  YATZY: false
(11)(0): 3 (11)(1): 4 (11)(2): 2 (11)(3): 5 (11)(4): 6  YATZY: false
...
\end{REPL}
\end{Slide}

\begin{Slide}{Iterera över nästlad struktur: for-sats}\SlideFontSmall
Iterera med nästlad for-sats: (xss är en \code{Vector[Vector[Int]]})
\begin{REPL}
scala> val xss = Vector.fill(100)(roll(5))

scala> for (i <- xss.indices) do 
         for (j <- xss(i).indices) do
           print(s"($i)($j): ${xss(i)(j)} ")
         println(s" YATZY: ${isYatzy(xss(i))}")

(0)(0): 3 (0)(1): 6 (0)(2): 4 (0)(3): 4 (0)(4): 6  YATZY: false
(1)(0): 4 (1)(1): 1 (1)(2): 5 (1)(3): 2 (1)(4): 6  YATZY: false
(2)(0): 1 (2)(1): 3 (2)(2): 5 (2)(3): 6 (2)(4): 2  YATZY: false
(3)(0): 2 (3)(1): 1 (3)(2): 1 (3)(3): 5 (3)(4): 4  YATZY: false
(4)(0): 4 (4)(1): 4 (4)(2): 1 (4)(3): 6 (4)(4): 5  YATZY: false
(5)(0): 3 (5)(1): 3 (5)(2): 2 (5)(3): 3 (5)(4): 6  YATZY: false
(6)(0): 3 (6)(1): 6 (6)(2): 1 (6)(3): 1 (6)(4): 4  YATZY: false
(7)(0): 6 (7)(1): 2 (7)(2): 4 (7)(3): 4 (7)(4): 3  YATZY: false
(8)(0): 1 (8)(1): 5 (8)(2): 4 (8)(3): 2 (8)(4): 4  YATZY: false
(9)(0): 1 (9)(1): 1 (9)(2): 3 (9)(3): 6 (9)(4): 6  YATZY: false
(10)(0): 2 (10)(1): 5 (10)(2): 2 (10)(3): 4 (10)(4): 5  YATZY: false
(11)(0): 3 (11)(1): 4 (11)(2): 2 (11)(3): 5 (11)(4): 6  YATZY: false
...
\end{REPL}
\end{Slide}


% \begin{Slide}{Iterera över nästlad struktur med nästlad foreach}\SlideFontSmall
% Iterera med nästlad foreach-sats:
% \begin{REPL}
% scala> val xss = Vector.tabulate(2,5)((x,y) => x + y + 1)

% xss.foreach{ xs => ??? ; println }

% 1 2 3 4 5
% 2 3 4 5 6
% \end{REPL}
% \end{Slide}


% \begin{Slide}{Iterera över nästlad struktur med nästlad foreach}\SlideFontSmall
% Iterera med nästlad foreach-sats:
% \begin{REPL}
% scala> val xss = Vector.tabulate(2,5)((x,y) => x + y + 1)

% xss.foreach{ xs => xs.foreach{ x => print(x + " ") }; println }

% 1 2 3 4 5
% 2 3 4 5 6
% \end{REPL}
% \end{Slide}


\begin{Slide}{Nästlade for-uttryck}\SlideFontSmall
Iterera med \Emph{nästlad for-yield}:\\
%Statisk typ: \code{IndexedSeq[IndexedSeq[[Int]]} \\
%Dynamisk typ: \code{Vector[Vector[[Int]]}

\begin{REPL}
scala> val xss = for (i <- 1 to 2) yield 
                   for (j <- 1 to 5) yield i + j + 1
                 
val xss: IndexedSeq[IndexedSeq[Int]] =
      ???

\end{REPL}
\pause Om man skriver så här får man en endimensionell struktur:
\begin{REPL}
scala> val xs = for (i <- 1 to 2; j <- 1 to 5) yield i + j + 1
val xs: IndexedSeq[Int] =
    ???

\end{REPL}
\end{Slide}

\begin{Slide}{Nästlade for-uttryck}\SlideFontSmall
Iterera med \Emph{nästlad for-yield}:\\
\begin{REPL}
scala> val xss = for (i <- 1 to 2) yield {
                   for (j <- 1 to 5) yield i + j + 1
                 }
val xss: IndexedSeq[IndexedSeq[Int]] =
    Vector(Vector(3, 4, 5, 6, 7), Vector(4, 5, 6, 7, 8))

\end{REPL}
\pause Om man skriver så här får man en endimensionell struktur:
\begin{REPL}
scala> val xs = for (i <- 1 to 2; j <- 1 to 5) yield i + j + 1
val xs: IndexedSeq[Int] =
    Vector(3, 4, 5, 6, 7, 4, 5, 6, 7, 8)

\end{REPL}
\end{Slide}



\begin{Slide}{Nästlade map-uttryck}\SlideFontSmall
Iterera med \Emph{nästlade map-uttryck}:\\
\begin{REPL}
scala> val xss = (1 to 2).map(i => (1 to 5).map(j => i + j + 1))
xss: IndexedSeq[IndexedSeq[Int]] =
      ???
\end{REPL}
\end{Slide}

\begin{Slide}{Nästlade map-uttryck}\SlideFontSmall
Iterera med \Emph{nästlade map-uttryck}:\\
\begin{REPL}
scala> val xss = (1 to 2).map(i => (1 to 5).map(j => i + j + 1))
xss: IndexedSeq[IndexedSeq[Int]] =
      Vector(Vector(3, 4, 5, 6, 7), Vector(4, 5, 6, 7, 8))
\end{REPL}
\end{Slide}



\ifkompendium\else
\begin{Slide}{Fallgrop: likhet av array}
\begin{REPL}
scala> Vector.fill(5, 2)(42) == Vector.fill(5, 2)(42)
val res0: ???

scala> Array.fill(5, 2)(42) == Array.fill(5, 2)(42)
val res1: ???
\end{REPL}
\end{Slide}
\fi

\begin{Slide}{Fallgrop: likhet av array}
\begin{REPL}
scala> Vector.fill(5, 2)(42) == Vector.fill(5, 2)(42)
val res0: Boolean = true

scala> Array.fill(5, 2)(42) == Array.fill(5, 2)(42)
val res1: Boolean = false  // AAAARRGH!!! :(
\end{REPL}
Primitiva arrayer har en equals-metod som ger referenslikhet, \Alert{inte} innehållslikhet. Och det fungerar följaktligen ej heller på nästlade strukturer. 
\end{Slide}

\ifkompendium\else
\begin{Slide}{Övning: Kolla likhet av array (uppfinner hjulet)}
\begin{Code}
def isEqual(xss: Array[Array[Int]], yss: Array[Array[Int]]) = 
  var i = 0
  var foundUnequal = false
  while ??? do                          // VILKET VILLKOR?
    var j = 0
    while ??? do                        // VILKET VILLKOR?
      if xss(i)(j) != yss(i)(j) then ???   // VAD SKA UPPDATERAS? 
      j += 1
    end while
    i += 1
  end while
  !foundUnequal
end isEqual
\end{Code}
\begin{REPL}
scala> val (xss, yss) = (Array.fill(5,2)(42), Array.fill(5,2)(42))

scala> isEqual(xss, yss)

scala> yss(4)(1) = 0

scala> isEqual(xss, yss)
\end{REPL}
\end{Slide}
\fi


\begin{Slide}{Övning: Kolla likhet av array (uppfinner hjulet)}
\begin{Code}
def isEqual(xss: Array[Array[Int]], yss: Array[Array[Int]]) = 
  var i = 0
  var foundUnequal = false
  while i < xss.length && !foundUnequal do
    var j = 0
    while j < xss(i).length && !foundUnequal do
      if xss(i)(j) != yss(i)(j) then foundUnequal = true
      j += 1
    end while
    i += 1
  end while
  !foundUnequal
end isEqual
\end{Code}
\begin{REPL}
scala> val (xss, yss) = (Array.fill(5,2)(42), Array.fill(5,2)(42))

scala> isEqual(xss, yss)  // true

scala> yss(4)(1) = 0

scala> isEqual(xss, yss)  // false
\end{REPL}
\end{Slide}

\begin{Slide}{Använd \texttt{sameElements} för test av innehållslikhet men bara på icke-nästlade arrayer}

  I Scala kan du använda metoden \code{sameElements} på arrayer för innehållslikhet, men det funkar \Alert{INTE} på nästlade strukturer.

\begin{REPL}
scala> val xs = Array(1,2,3)
xs: Array[Int] = Array(1, 2, 3)

scala> val ys = Array(1,2,3)
ys: Array[Int] = Array(1, 2, 3)

scala> xs.sameElements(ys)
res0: Boolean = true

scala> Array(Array(1)) sameElements Array(Array(1))  
res1: Boolean = false

\end{REPL}
\pause Använd i stället: \code{java.util.Arrays.deepEquals(xs, ys)}\\
men det kan då behövas \code{.asInstanceOf[Array[Object]]} på argumenten om kompilatorn inte klarar typkonverteringen.
\end{Slide}

% \begin{Slide}{Matris som Array med Array med heltal i Java}\SlideFontTiny
% \begin{CodeSmall}[language=Java]
% public class ArrayMatrix {

%     public static void showMatrix(int[][] m){
%         System.out.println("\n--- showMatrix ---");
%         for (int row = 0; row < m.length; row++){
%             for (int col = 0; col < m[row].length; col++) {
%                 System.out.print("[" + row + "]");
%                 System.out.print("[" + col + "] = ");
%                 System.out.print(m[row][col] + "; ");
%             }
%             System.out.println();
%         }
%     }

%     public static void main(String[] args) {
%         int[][] xss = new int[10][5];
%         showMatrix(xss);
%     }
% }
% \end{CodeSmall}
% \pause
% Övning: skriv en metod \code{fillRnd} som fyller en heltalsmatris med slumptal 1 till n:\\
% \pause
% \jcode|public static void fillRnd(int[][] m, int n){ /* ??? */ }| \\
% \pause
% Tips: använd en nästlad for-sats och detta uttryck: \\
% \jcode{(int) (Math.random() * n + 1) // (int) motsvarar Scalas asInstanceOf[Int]}

% \end{Slide}

\begin{Slide}{Om veckans övningar}\SlideFontSmall
\begin{itemize}
\item Träna på att iterera över nästlade strukurer

\item Fortsätt jobba med Yatzy-exemplet

\item träna på att skapa \Emph{imperativa} algoritmer: \\
lös \code{isYatzy} med \code{while}-sats 

\item Extrauppgift där du ska bygga ett enkelt yatzy-spel i terminalen (kunde varit en tentauppgift...)

\end{itemize}
\end{Slide}

% \begin{Slide}{Övning extrauppgift, utgör början på labb \code{survey}}\SlideFontSmall
%
% \begin{ScalaSpec}{Table}
% object Table {
%   /** Creates a new Table from fileName with columns split by sep */
%   def fromFile(fileName: String, separator: Char = ';'): Table = ???
% }
% case class Table(
%   data: Vector[Vector[String]],
%   headings: Vector[String],
%   sep: String){
%   /** A 2-tuple with (number of rows, number of columns) in data */
%   val dim: (Int, Int) = ???
%
%   /** The element in row r an column c of data, counting from 0 */
%   def apply(r: Int, c: Int): String = ???
%
%   /** The row-vector r in data, counting from 0 */
%   def row(r: Int): Vector[String]= ???
%
%   /** The column-vector c in data, counting from 0 */
%   def col(c: Int): Vector[String] = ???
%
%   /** A map from heading to index counting from 0 */
%   lazy val indexOfHeading: Map[String, Int] = ???
%
%   /** The column-vector with heading h in data */
%   def col(h: String): Vector[String] = ???
%
%   /** A vector with the distinct, sorted values of col with heading h */
%   def values(h: String): Vector[String] = ???
%
%   /** Headings and data with columns separated by sep */
%   override lazy val toString: String = ???
% }
% \end{ScalaSpec}
% \end{Slide}


% \begin{Slide}{Övn. fördjupn. uppg.: skapa en generisk matris-klass}\SlideFontSmall
% \vspace{-0.7em}
% \begin{Code}[basicstyle=\SlideFontSize{6}{6.8}\ttfamily\selectfont]
% case class Matrix[T](data: Vector[Vector[T]]){
%
%   def foreachRowCol(f: (Int, Int, T) => Unit): Unit =
%     for (r <- data.indices) {
%       for (c <- data(r).indices) {
%         f(r, c, data(r)(c))
%       }
%     }
%
%   def map[U](f: T => U): Matrix[U] = Matrix(data.map(_.map(f)))
%
%   /** The element at row r and column c */
%   def apply(r: Int, c: Int): T = ???
%
%   /** Gives Some[T](element) at index (r, c) if within index bounds, else None */
%   def get(r: Int, c: Int): Option[T] = ???
%
%   /** The row vector of row r */
%   def row(r: Int): Vector[T] = ???
%
%   /** The column vector of column c */
%   def col(c: Int): Vector[T] = ???
%
%   /** A new Matrix with element at row r and col c updated */
%   def updated(r: Int, c: Int, value: T): Matrix[T] = ???
% }
% object Matrix {
%   def fill[T](rowSize: Int, colSize: Int)(init: T): Matrix[T] =
%     new Matrix(Vector.fill(rowSize)(Vector.fill(colSize)(init)))
% }
% \end{Code}
% \end{Slide}

%!TEX encoding = UTF-8 Unicode
%!TEX root = ../lect-w08.tex

\Subsection{Typparametrar}



\begin{Slide}{Exempel: Icke-generisk case-klass med heltalsmatris}
  En \emph{icke-generisk} datastruktur har inga obundna typparametrar; alla typer är \Emph{konkreta} (alltså specifika). \\~\\ En icke-generisk case-class med en \code{Vector[Vector[Int]]}:
  \begin{Code}
  case class Matrix(data: Vector[Vector[Int]]):
    def apply(x: Int, y: Int): Int = data(x)(y)
  \end{Code}

  \begin{REPL}
  scala> Matrix(Vector(Vector(5, 2, 42, 4, 5),Vector(3, 4, 18, 6, 7)))
  res0: Matrix =
    Matrix(Vector(Vector(5, 2, 42, 4, 5), Vector(3, 4, 18, 6, 7)))
  \end{REPL}

\end{Slide}





\begin{Slide}{Exempel: Generisk case-klass med generell matris}
  En \emph{generisk} datastruktur har minst en obunden \Emph{typparameter} som kan bindas  till ett \Alert{konkret} \Emph{typargument}.
  
  \begin{Code}
  case class Matrix[T](data: Vector[Vector[T]]):
    def apply(x: Int, y: Int): T = data(x)(y)
  \end{Code}
  \code{Matrix} i exemplet ovan är en \Emph{generisk} case-class där \code{T} är obunden, eftersom \code{T} är en typparameter deklarerad inom \code{[]} \Alert{efter} klassens namn men \Alert{före} klassparameterlistan. \\

  \vspace{0.5em} Användning där \code{T} binds till \code{Int} via kompilatorns typhärledning:
  \begin{REPL}
  scala> Matrix(Vector(Vector(5, 2, 42, 4, 5),Vector(3, 4, 18, 6, 7)))
  res1: Matrix[Int] =
    Matrix(Vector(Vector(5, 2, 42, 4, 5), Vector(3, 4, 18, 6, 7)))
  \end{REPL}

\end{Slide}




\begin{Slide}{Vad är en typparameter?}\SlideFontSmall
  \setlength{\leftmargini}{0pt}

\begin{itemize}
\item En \Emph{typparameter} gör det möjligt att ge ett \Emph{typargument}.
\item Detta kallas \Emph{parametrisk polymorfism} \Eng{paramteric polymorphism}.
\item Exempel: \Emph{generisk} \Alert{funktion}:
\begin{Code}
def tnirp[A](x: A):Unit = println(x.toString.reverse)
\end{Code}
\pause
\item En \Emph{fri} typparameter kan bindas till vilken typ som helst.
\item Bindingen av typargument till typparametrar sker vid \Alert{kompileringstid}.
\item En typparameter är \Emph{fri} om den \Alert{inte} fått något värde, annars \Emph{bunden}. 
\pause
\item Exempel: \Emph{generisk} \Alert{klass} med \Emph{generiska} \Alert{metoder}:
\begin{Code}
class Cell[A](   // [A] är fri (måste bindas vid användning) 
    var value: A):                              // A är bunden
  def update(a: A): Unit = value = a            // A är bunden
  def replaced[B](b: B): Cell[B] = new Cell(b)  // första [B] är fri
\end{Code}
\pause
\item \Alert{Skuggning kan förekomma}: Om \code{replaced} i \code{Cell} hade använt namnet A på sin typparameter hade den \Emph{skuggat} klassens typparameter och tolkats som en  fri typparameter, alltså en godtycklig typ och \Alert{inte} klassens typparameter. (jämför  namnöverskuggning vid \Emph{lokala} namn i nästlade block)
\end{itemize}

\end{Slide}

\ifkompendium\else
\begin{Slide}{Exempel: Generisk funktion}
Vad händer här?
\begin{REPL}

scala> def skrikBaklänges(x: T): String = x.toString.toUpperCase.reverse
???



scala> def skrikBaklänges[T](x: T): String = x.toString.toUpperCase.reverse

scala> skrikBaklänges("gurka är gott")
val res0: ???

\end{REPL}
\end{Slide}


\begin{Slide}{Exempel: Generisk funktion}
Vad händer här?
\begin{REPL}

scala> def skrikBaklänges(x: T): String = x.toString.toUpperCase.reverse
1 |def skrikBaklänges(x: T): String = x.toString.toUpperCase.reverse
  |                      ^
  |                      Not found: type T
                             ^

scala> def skrikBaklänges[T](x: T): String = x.toString.toUpperCase.reverse

scala> skrikBaklänges("gurka är gott")
val res0: ???
\end{REPL}
\end{Slide}
\fi

\begin{Slide}{Exempel: Generisk funktion}
Vad händer här?
\begin{REPL}

scala> def skrikBaklänges(x: T): String = x.toString.toUpperCase.reverse
1 |def skrikBaklänges(x: T): String = x.toString.toUpperCase.reverse
  |                      ^
  |                      Not found: type T
                             ^

scala> def skrikBaklänges[T](x: T): String = x.toString.toUpperCase.reverse

scala> skrikBaklänges("gurka är gott")
val res0: String = TTOG RÄ AKRUG
\end{REPL}
Om ingen typparameter deklareras inom hakparenteser efter funktionens namns så vet inte kompilatorn vad \code{T} är för en typ. Men med en typparameter \code{[T]} efter funktionsnamnet tolkar kompilatorn funktionen som \Emph{generisk} och typen \code{T} bestäms av argumentets typ \Alert{vid anrop} och \code{T} kan bindas till godtycklig typ.
\end{Slide}


\begin{Slide}{Exempel: Generisk case-klass}
\SlideFontSmall
En generisk klass har en eller flera typparametrar efter klassnamnet:
\begin{Code}
case class Box[A](value: A)  
\end{Code}

Kompilatorn härleder typparameterarnas typ utifrån givna värden. 
\begin{REPL}
scala> Box("gurka")  
val res1: Box[String] = Box(gurka)
\end{REPL}

Du kan också ge typpparametern en typ explicit:
\begin{REPL}
scala> Box[Int](42)  // 
val res3: Box[Int] = Box(42)
\end{REPL}

Om typen inte stämmer får du hjälp av kompilatorn att hitta felet:
\begin{REPL}
scala> Box[String](42)
-- Error:
1 |Box[String](42)
  |            ^^
  |            Found:    (42 : Int)
  |            Required: String
\end{REPL}
\end{Slide}






\begin{Slide}{Fallgrop: Typradering \Eng{type erasure}}\SlideFontSmall
Informationen om typerna i typparametrar raderas innan kodgenerering för JVM av prestandaskäl och \Alert{typparametrar saknas vid runtime} i bytekoden.
\vspace{-0.25em}\begin{REPL}
scala> def isIntVector[T](xs: Vector[T]) = xs.isInstanceOf[Vector[Int]]
-- Warning:
1 |def isIntVector[T](xs: Vector[T]) = xs.isInstanceOf[Vector[Int]]
  |                                    ^^^^^^^^^^^^^^^^^^^^^^^^^^^^
  |                the type test for Vector[Int] cannot be checked at runtime
def isIntVector[T](xs: Vector[T]): Boolean

scala> isIntVector(Vector("hej"))
res42: Boolean = true  // AAAARGHH!! :(
\end{REPL}
Måste ''packa upp'' samlingen och typtesta alla element:
\begin{REPL}
scala> def isIntVector[T](xs: Vector[T]) = xs.forall(_.isInstanceOf[Int])

scala> isIntVector(Vector("hej"))
res43: Boolean = false  // FUNKAR :)

\end{REPL}
Typkontroll vid körtid görs oftast hellre med \code{match}.

\end{Slide}

\Subsection{Upptäcka och åtgärda buggar}

\begin{Slide}{Testning och avlusning}
%\TODO 
\begin{itemize}
\item Läs om testning och avlusning \Eng{debugging} i Appendix D: ''Fixa buggar'' 
\item Träna på println-debugging
\item Prova debuggern i VS code
%\item Visa hur testramverket ska funka som du ska skapa på övning och använda på labb
%\item sbt testOnly och andra sätt att köra testfall
%\item Visa hur en fördröjning kan skapas med en s.k. thunk 
%\item Visa hur printlndebugging
%\item Visa hur debugga i vs code
\end{itemize}
\end{Slide}


% \ifkompendium\else


% \begin{Slide}{Typparametrar på tentan?}
% \begin{itemize}
% \item Det ingår att kunna använda färdiga generiska strukturer med specifika typer, t.ex. \code{Vector[Int]}

% \item Det ingår att kunna skapa abstraktioner med specifika typparametrar, t.ex. metoder eller klasser som tar en vektor med en specifik typ som parameter:\\
% \code{case class X(x: Vector[Int])}


% \item Det ingår \Alert{inte} på tentan att kunna skapa generiska metoder eller klasser, t.ex.: \\
% \code{def f[T](x: Vector[T]): Vector[T] = ???} \\
% Mer om generiska strukturer i fördjupningskursen!
% \end{itemize}
% \end{Slide}

% \fi

%!TEX encoding = UTF-8 Unicode
%!TEX root = ../lect-w08.tex

\Subsection{Upptäcka och åtgärda buggar}

\begin{Slide}{Debugging -- Appendix D}
\begin{itemize}
\item Läs om testning och avlusning \Eng{debugging} i Appendix D: ''Fixa buggar'' 
\item Träna på println-debugging
\item Prova debuggern i VS code

% TODO?
%\item munit ?
%\item sbt testOnly och andra sätt att köra testfall???

\end{itemize}
\end{Slide}


\begin{Slide}{Den första buggen}\SlideFontSmall
En nattfjäril i ett relä i datorn Mark II hittad av Grace Hopper i logg från 1940:

\includegraphics[width=0.70\textwidth]{../img/bug}

\url{https://en.wikipedia.org/wiki/Debugging}

\end{Slide}



\begin{Slide}{Olika sorters fel?}
\begin{itemize}
\item Kravfel  
\item Designfel  
\item Implementeringsfel  
\item Testfel  
\item Operatörsfel  
\item Användarfel  
\end{itemize}
\end{Slide}

\begin{Slide}{När upptäcks fel?}
\begin{itemize}
\item Vid granskning av människor  
\item Kompileringsfel -- tack kompilatorn!
\item Exekveringsfel \pause 
\begin{itemize}
\item Exekveringen ger oönskat resultat 
\begin{itemize}
\item Vid testning -- eller är det fel på testfallet?
\item I produktion -- ledsna användare \code{:(}  
\end{itemize}
\item Exekveringen hänger sig \Eng{hang}
\begin{itemize}
\item oändlig loop  
\item väldigt långsamt  
\item väntar på indata
\item dödläge
\end{itemize}
\item Exekveringen kraschar \Eng{crash}
\begin{itemize}
\item minnet är slut
\item null-referens
\item undantag \Eng{exception}
\end{itemize}
\end{itemize}
\end{itemize}
\end{Slide}

\begin{Slide}{Förebygga fel}
\begin{itemize}
\item Skapa begriplig kod.
\item Tänk ut bra namn.
\item Kontrollera parametrar och variabler.
\item Kontrollera typer.
\item Hantera saknade värden.
\item Hantera undantag.
\item Granska kod.
\item Testa kod.
\item Lär av användarnas upplevelser.
\end{itemize}
\end{Slide}

\begin{Slide}{Hitta felorsaken: debugging (avlusning)}
\begin{itemize}
\item Återskapa buggen med ett minimalt testfall.
\item Formulera och verifiera hypoteser om buggen.
\item Instrumentering med utskrifter, "println-debugging".
\end{itemize}
\end{Slide}

\begin{Slide}{Åtgärda fel}
\begin{itemize}
\item Algoritmen i grunden feltänkt: skapa ny algoritm
\item Undantagsfall hanteras ej korrekt.
\item En knepig algoritm är extra svår att fixa till.
\item Medan man rättar en bug kan man råka att, av misstag, skapa nya buggar.
\item Exekveringstiden växer alltför snabbt ökad datamängd.
\end{itemize}
\end{Slide}

\begin{Slide}{Använda en debugger}
\begin{minipage}{0.42\textwidth}
\begin{itemize}
\item Sätta brytpunkter.
\item Stegad exekvering.
\item Inspektera variabler.
\end{itemize}
\end{minipage}%
\begin{minipage}{0.65\textwidth}
\includegraphics[width=1.0\textwidth]{../img/vscode-debug}
\end{minipage}

\vspace{2em}
Läs mer i Appendix H om debuggern i VSCode.
\end{Slide}



\ifkompendium\else

\begin{SlideExtra}{Om veckans övning: \code{matrices}}
\SlideFontSmall
\begin{itemize}

%!TEX encoding = UTF-8 Unicode
%!TEX root = ../exercises.tex


\item Kunna skapa och använda matriser med nästlade strukturer av \code{Vector}.
\item Kunna iterera över elementen i en matris med nästlade \code{for}-satser och \code{for}-\code{yield}-uttryck, samt nästlad applicering av \code{map} respektive \code{foreach}.
\item Kunna skapa och använda funktioner som tar matriser som parametrar.
\item Kunna skapa en enkel generisk klass och enkla generiska funktioner med hjälp av en typparameter.
\item Kunna beskriva skillnader och likheter mellan Scala och Java vad gäller indexering och iterering i matriser implementerade med nästlade arrayer.
%\item Kunna skapa och använda matriser med hjälp inbyggda arrayer i Java.
%\item Kunna använda nästlade \code{for}-satser i Java för att iterera över elementen i en matris.

\end{itemize}

\end{SlideExtra}

\begin{SlideExtra}{Om veckans labb: \code{life}}
\SlideFontSmall
\begin{itemize}
%!TEX encoding = UTF-8 Unicode
%!TEX root = ../compendium2.tex

\item Kunna skapa och använda matriser med hjälp av en generisk datatyp.
\item Kunna iterera över alla element i en matris.
\item Träna på algoritmkonstruktion.
\item Träna på hantering av både oföränderliga och förändringsbara objekt.
\item Prova på att använda en avlusare \Eng{debugger} i en integrerad utvecklingsmiljö (IDE), t.ex. VS code.

\end{itemize}
\end{SlideExtra}

\fi



%\chapter{Mönster, Undantag}\label{chapter:W08}
\begin{itemize}[nosep]
\item match
\item Option
\item null
\item try
\item catch
\item Try
\item unapply
\end{itemize}

%!TEX encoding = UTF-8 Unicode
%!TEX root = ../exercises.tex

\ifPreSolution

\Exercise{\ExeWeekEIGHT}\label{exe:W08}

\begin{Goals}
\item Kunna skapa och använda matriser med nästlade strukturer av \code{Vector}.
\item Kunna iterera över elementen i en matris med nästlade \code{for}-satser och \code{for}-\code{yield}-uttryck, samt nästlad applicering av \code{map} respektive \code{foreach}.
\item Kunna skapa och använda funktioner som tar matriser som parametrar.
\item Kunna skapa en enkel generisk klass och enkla generiska funktioner med hjälp av en typparameter.
\item Kunna beskriva skillnader och likheter mellan Scala och Java vad gäller indexering och iterering i matriser implementerade med nästlade arrayer.
%\item Kunna skapa och använda matriser med hjälp inbyggda arrayer i Java.
%\item Kunna använda nästlade \code{for}-satser i Java för att iterera över elementen i en matris.
\end{Goals}

\begin{Preparations}
\item \StudyTheory{08}
\end{Preparations}

\BasicTasks

\else

\ExerciseSolution{\ExeWeekEIGHT}

\BasicTasks

\fi



\WHAT{Para ihop begrepp med beskrivning.}

\QUESTBEGIN

\Task \what

\vspace{1em}\noindent Koppla varje begrepp med den (förenklade) beskrivning som passar bäst:

\begin{ConceptConnections}
  matris & 1 & & A & konkret typ, binds till typparameter vid kompilering \\ 
  generisk & 2 & & B & indexerbar datastruktur i två dimensioner \\ 
  typargument & 3 & & C & har abstrakt typparameter, typen är generell \\ 
  typhärledning & 4 & & D & kompilatorn beräknar typ ur sammanhanget \\ 
\end{ConceptConnections}

\SOLUTION

\TaskSolved \what

\begin{ConceptConnections}
  matris & 1 & ~~\Large$\leadsto$~~ &  A & indexerbar datastruktur i två dimensioner \\ 
  radvektor & 2 & ~~\Large$\leadsto$~~ &  F & matris av dimension $1\times{}m$ med $m$ horisontella värden \\ 
  kolumnvektor & 3 & ~~\Large$\leadsto$~~ &  G & matris av dimension $m\times{}1$ med $m$ vertikala värden \\ 
  kolonn & 4 & ~~\Large$\leadsto$~~ &  C & annat ord för kolumn \\ 
  generisk & 5 & ~~\Large$\leadsto$~~ &  B & har abstrakt typparameter, typen är generell \\ 
  typargument & 6 & ~~\Large$\leadsto$~~ &  D & konkret typ, binds till typparameter vid kompilering \\ 
  typhärledning & 7 & ~~\Large$\leadsto$~~ &  E & kompilatorn beräknar typ ur sammanhanget \\ 
\end{ConceptConnections}

\QUESTEND




\WHAT{Skapa matriser med hjälp av nästlade samlingar.}

\QUESTBEGIN

\Task  \what~  Man kan i ett datorprogram, med hjälp av samlingar som innehåller samlingar, skapa nästlade strukturer som kan indexeras i två dimensioner och på så sätt representera en  \textbf{matris}.\footnote{\href{https://sv.wikipedia.org/wiki/Matris}{sv.wikipedia.org/wiki/Matris}}

\Subtask Rita minnessituationen efter tilldelningen på rad 1 nedan. Vad har \code{m} för typ och värde? Vad har \code{m} för dimensioner? Hur sker indexeringen i ett datorprogram jämfört med i matematiken?

\begin{REPL}
scala> val m = Vector((1 to 5).toVector, (3 to 7).toVector)
scala> m.apply(0).apply(1)
scala> m(1)
scala> m(1)(4)
\end{REPL}

\Subtask Vad ger uttrycken på raderna 2, 3 och 4 ovan för värden och typ?

\Subtask Man kan i ett datorprogram mycket väl skapa tvådimensionella, nästlade strukturer där raderna \emph{inte} innehåller samma antal element. Det blir då ingen äkta matris i strikt matematisk mening, men man kallar ofta ändå en sådan struktur för en ''matris''. Vilken typ har variablerna \code{m2}, \code{m3}, \code{m4} och \code{m5} nedan?

\begin{REPL}
scala> val m2 = Vector(Vector(1,2,3),Vector(4,5),Vector(42))
scala> val m3 = Vector(Vector(1,2), Vector(1.0, 2.0, 3.0))
scala> val m4 = m3(1) +: Vector("a") +: m3
scala> val m5 = Vector.fill(42){ m2(1).map(e => (e * math.random()).toInt) }
\end{REPL}

\Subtask Vilken av variablerna \code{m2}, \code{m3}, \code{m4} och \code{m5} ovan representerar en äkta matris i matematisk mening? Vilken är dess dimensioner?

\SOLUTION

\TaskSolved \what

\SubtaskSolved   \includegraphics{../img/w09-solutions/1a} \\
Typ: \code{Vector[Vector[Int]]}\\
Värde: \code{Vector(Vector(1, 2, 3, 4, 5), Vector(3, 4, 5, 6, 7))} \\
Dimensioner: $2 \times 5$\\
Inom matematiken sker indexering enligt konvention med 1 som lägsta index. I scala är lägsta index 0, man använder s.k. 0-indexering. \footnote{Detta är inte fallet i alla programmeringsspråk, vilket du kan läsa mer om på \url{https://en.wikipedia.org/wiki/Array\_data\_type\#Index\_origin}}

\SubtaskSolved
\begin{REPL}
scala> val m = Vector((1 to 5).toVector, (3 to 7).toVector)
m: Vector[Vector[Int]] = Vector(Vector(1, 2, 3, 4, 5), Vector(3, 4, 5, 6, 7))

scala> m.apply(0).apply(1)
res4: Int = 2

scala> m(1)
res5: Vector[Int] = Vector(3, 4, 5, 6, 7)

scala> m(1)(4)
res6: Int = 7
\end{REPL}

\SubtaskSolved  \\
m2: \code{Vector[Vector[Int]]}\\
m3: \code{Vector[Vector[Int | Double]]}\\
m4: \code{Vector[Vector[Int | Double | String]]}\\
m5: \code{Vector[Vector[Int]]}

\SubtaskSolved  m5, $42 \times 2$

\QUESTEND





\WHAT{Skapa och iterera över matriser.}

\QUESTBEGIN

\Task  \label{matrices:task:yatzy} \what~  Du ska skapa matriser där varje rad representerar 5 kast med en tärning i spelet Yatzy.\footnote{\href{https://sv.wikipedia.org/wiki/Yatzy}{sv.wikipedia.org/wiki/Yatzy}}


\Subtask Definiera i REPL en funktion \code{def throwDie: Int = ???} som returnerar ett slumptal mellan 1 och 6.

\Subtask Skapa nedan heltalsmatris i REPL. Vilken dimension får matrisen?
\begin{REPL}
scala> val ds1 = for (i <- 1 to 1000) yield 
            for (j <- 1 to 5) yield throwDie
          
\end{REPL}

\Subtask Man kan också använda nedan varianter för att skapa en heltalsmatris. Vilken av varianterna \code{ds1} ... \code{ds6} tycker du är lättast att läsa och förstå? Prova respektive variant i REPL och ange vilken typ på \code{ds1} ... \code{ds6} som härleds av kompilatorn.
\begin{REPL}
val ds2 = (1 to 1000).map(i => (1 to 5).map(j => throwDie))
val ds3 = (1 to 1000).map(i => Vector.fill(5)(throwDie))
val ds4 = for (i <- 1 to 1000) yield Vector.fill(5)(throwDie)
val ds5 = Vector.fill(1000)(Vector.fill(5)(throwDie))
val ds6 = Vector.fill(1000, 5)(throwDie)
\end{REPL}


\Subtask Definiera en funktion \\ \code{def roll(n: Int): Vector[Int] = ???}\\ som ger en heltalsvektor med $n$ stycken slumpvisa tärningskast. Kasten ska vara sorterade i växande ordning; använd för detta ändamål samlingsmetoden \code{sorted}.


\Subtask \label{matrices:subtask:isyatzyforall} Definera i REPL en funktion \code{isYatzy(xs: Vector[Int]): Boolean = ???} som testar om alla elementen i en heltalsvektor är samma. Använd samlingsmetoden \code{forall}.


\Subtask Skapa en funktion  \\ \code{def diceMatrix(m: Int, n: Int): Vector[Vector[Int]] = ???} \\ som med hjälp av funktionen \code{roll} skapar en matris med \code{m} st vektorer med vardera \code{n} slumpvisa tärningskast.


\Subtask \label{matrices:subtask:diceMatrixToString} Skapa en funktion som returnerar en utskriftsvänlig sträng \\ \code{def diceMatrixToString(xss: Vector[Vector[Int]]): String = ???} \\med hjälp av \code{map} och \code{mkString}, som fungerar enligt nedan.
\begin{REPL}
scala> val dm2s = diceMatrixToString(diceMatrix(4, 5))
val dm2s: String = 1 4 4 6 6
1 1 2 6 6
2 4 4 5 6
1 1 5 6 6

scala> println(dm2s)
1 4 4 6 6
1 1 2 6 6
2 4 4 5 6
1 1 5 6 6
\end{REPL}



\Subtask Implementera funktionen \\ \code{def filterYatzy(xss: Vector[Vector[Int]]): Vector[Vector[Int]]} \\ som filtrerar fram alla yatzy-rader i matrisen \code{xss} enligt nedan. Använd din funktion \code{isYatzy} och samlingsmetoden \code{filter}.
\begin{REPL}
scala> println(diceMatrixToString(filterYatzy(diceMatrix(10000, 5))))
4 4 4 4 4
6 6 6 6 6
4 4 4 4 4
6 6 6 6 6
4 4 4 4 4
4 4 4 4 4
2 2 2 2 2
\end{REPL}



\Subtask Implementera funktionen \\
\code{def yatzyPips(xss: Vector[Vector[Int]]): Vector[Int] = ???}\\
som ska ge en vektor med de tärningsvärden som gav yatzy, för kasten i matrisen \code{xss} enligt nedan. Använd din funktion \code{filterYatzy}.
\begin{REPL}
scala> val dm = Vector(Vector(1,2,3,4,5),Vector(4,4,4,4,4),Vector(3,3,3,3,3))
scala> yatzyPips(dm)
val res42: Vector[Int] = Vector(4, 3)
\end{REPL}

\SOLUTION

\TaskSolved \what

\SubtaskSolved
\begin{Code}
def throwDie: Int = (math.random() * 6).toInt + 1
\end{Code}
Eller:
\begin{Code}
def throwDie: Int = scala.util.Random.nextInt(6) + 1
\end{Code}

\SubtaskSolved  Matrisdimension i matematisk notation: $1000 \times 5$, vilket motsvarar en matris med 1000 rader och 5 kolumner.

\SubtaskSolved
\begin{Code}
ds1: IndexedSeq[IndexedSeq[Int]]
ds2: IndexedSeq[IndexedSeq[Int]]
ds3: IndexedSeq[Vector[Int]]
ds4: IndexedSeq[Vector[Int]]
ds5: Vector[Vector[Int]]
ds6: Vector[Vector[Int]]
\end{Code}
\code{IndexedSeq} och \code{Vector} ovan finns i paketet \code{scala.collection.immutable}

\SubtaskSolved  \begin{Code}
def roll(n: Int) = Vector.fill(n)(throwDie).sorted
\end{Code}

\SubtaskSolved  \begin{Code}
def isYatzy(xs: Vector[Int]): Boolean = xs.forall(_ == xs(0))
\end{Code}



%2.g)
\SubtaskSolved  \begin{Code}
def diceMatrix(m: Int, n: Int): Vector[Vector[Int]] =
  Vector.fill(m)(roll(n))
\end{Code}

\SubtaskSolved  \begin{Code}
def diceMatrixToString(xss: Vector[Vector[Int]]): String =
  xss.map(_.mkString(" ")).mkString("\n")
\end{Code}


%2.j)
\SubtaskSolved
\begin{Code}
def filterYatzy(xss: Vector[Vector[Int]]): Vector[Vector[Int]] =
  xss.filter(isYatzy)
\end{Code}



%2.m)
\SubtaskSolved  \begin{Code}
def yatzyPips(xss: Vector[Vector[Int]]): Vector[Int] =
  filterYatzy(xss).map(_.head)
\end{Code}

\QUESTEND








\WHAT{En oföränderlig, generisk matris-klass till veckans laboration \hyperref[section:lab:\LabWeekEIGHT]{\texttt{\LabWeekEIGHT}}.}

\QUESTBEGIN

\Task\label{exe:matrices:labprep}  \what~Under veckans laboration ska du simulera en enkel form av ''liv'' som består av celler i ett rutnät. För detta ändamål har vi nytta av en matris-klass som du ska implementera steg för steg i denna övning.
Skapa case-klassen nedan med en editor i filen \code{Matrix.scala}. Testa din lösning med hjälp av valfri \hyperref[appendix:ide]{IDE}, t.ex. \code{scalaide} eller \code{idea}.
\begin{Code}
case class Matrix(data: Vector[Vector[String]]){
  def apply(row: Int, col: Int): String = data(row)(col)
}
object Matrix {
  def fill(dim: (Int, Int))(value: String): Matrix =
    Matrix(Vector.fill(dim._1, dim._2)(value))
}
\end{Code}

\begin{REPLnonum}
scala> val m = Matrix.fill(3,4)("hej")
scala> val e = m(2, 2)
\end{REPLnonum}

\Subtask Vad får \code{m} ovan för typ?

\Subtask Vad får \code{e} ovan för typ?

\Subtask På hur många ställen måste du ändra i \code{Matrix} ovan för att den i stället ska representera en matris av heltal?

\Subtask Du ska nu med hjälp av en \textbf{typparameter} göra \code{Matrix} \textbf{generisk} \Eng{generic}, så att den blir en mer användbar matrisklass som kan innehålla element av vilken typ som helst. Genomför följande ändringar i \code{Matrix.scala}:

\begin{itemize}[noitemsep, nolistsep]
  \item Lägg till en typparameter \code{T} inom klammerparenteser efter namnet \code{Matrix} på alla ställen där det förekommer \emph{utom} efter namnet på kompanjonsobjektet\footnote{Singelobjekt kan inte ha typparametrar, men deras medlemmar kan.}.
  \item Byt ut \code{String} mot \code{T} på alla ställen där \code{String} förekommer.
  \item Lägg till en typparameter \code{T} inom klammerparenteser efter \code{def fill}.
\end{itemize}
Testa din generiska klass i REPL genom att skapa en boolesk matris:
\begin{REPLnonum}
scala> val bm = Matrix.fill(3,4)(false)
scala> val be = bm(0, 0)
\end{REPLnonum}

\Subtask Vad får \code{bm} ovan för typ?

\Subtask Vad får \code{be} ovan för typ?

\Subtask Lägg en kodrad i början av klasskroppen som med hjälp av \code{require} garanterar att alla rader i matrisen är lika långa.

\Subtask Lägg till en medlem \code{val dim: (Int, Int)} i klasskroppen efter \code{require}-satsen som ger ett par (alltså en 2-tupel) med antalet rader resp. kolumner i matrisen.

\Subtask Lägg till en metod \code{def updated(row: Int, col: Int)(value: T): Matrix[T]} som ger en ny matris där element på platsen \code{(row, col)} har uppdaterats till \code{value}.

\Subtask Lägg till en metod \code{def foreachIndex(f: (Int, Int) => Unit): Unit} som för varje index i \code{data} applicerar funktionen \code{f}.

\Subtask Lägg till en metod \code{override def toString} som så att en instans av \code{Matrix} visas enligt följande:
\begin{REPLnonum}
scala> val dm = Matrix.fill(3,4)(42.0)
val dm: Matrix[Double] =
Matrix of dim (3,4):
42.0 42.0 42.0 42.0
42.0 42.0 42.0 42.0
42.0 42.0 42.0 42.0
\end{REPLnonum}


\SOLUTION


\TaskSolved \what

\SubtaskSolved Typen på \code{m} blir \code{Matrix}.

\SubtaskSolved Typen på \code{e} blir \code{String}.

\SubtaskSolved Man behöver ändra på 3 ställen från \code{String} till \code{Int}.

\SubtaskSolved Generisk matris \code{Matrix[T]} för element av godtycklig typ \code{T}:

\begin{CodeSmall}
case class Matrix[T](data: Vector[Vector[T]]):
  def apply(row: Int, col: Int): T = data(row)(col)

object Matrix:
  def fill[T](dim: (Int, Int))(value: T): Matrix[T] =
    Matrix[T](Vector.fill(dim._1, dim._2)(value))
\end{CodeSmall}

\SubtaskSolved Tack vare kompilatorns typinferens så får \code{bm} typen \code{Matrix[Boolean]}.

\SubtaskSolved Typen på \code{be} blir \code{Boolean}.

\noindent \SubtaskSolved \SubtaskSolved \SubtaskSolved \SubtaskSolved \SubtaskSolved är alla implementerade i koden nedan: \vspace{-0.5em}
\begin{CodeSmall}
case class Matrix[T](data: Vector[Vector[T]]):
  require(data.forall(row => row.length == data(0).length))

  val dim: (Int, Int) = (data.length, data(0).length)

  def apply(row: Int, col: Int): T = data(row)(col)

  def updated(row: Int, col: Int)(value: T): Matrix[T] =
    Matrix(data.updated(row, data(row).updated(col, value)))

  def foreachIndex(f: (Int, Int) => Unit): Unit =
    for r <- data.indices; c <- data(r).indices do f(r, c)

  override def toString =
    s"""Matrix of dim $dim:\n${ data.map(_.mkString(" ")).mkString("\n") }"""

object Matrix:
  def fill[T](dim: (Int, Int))(value: T): Matrix[T] =
    Matrix[T](Vector.fill(dim._1, dim._2)(value))

\end{CodeSmall}

\QUESTEND


\clearpage

\ExtraTasks %%%%%%%%%%%%%%%%%%%%%%%%%%%%%%%%%%%%%%%%%%%%%%%%%


\WHAT{Imperativa matrisalgoritmer.}

\QUESTBEGIN

\Task  \what~Imperativa angreppssätt är nödvändiga att kunna när du stöter på samlingar och/eller språk som saknar funktionella metoder och/eller funktionsprogrammeringsmöjligheter. Genom att studera imperativa lösningar till de ofta mer koncisa funktionella lösningarna, får du träning i att skapa algoritmer som använder förändring genom tilldelning vid iterering.

\Subtask Implementera \code{isYatzy} från uppgift \ref{matrices:task:yatzy}\ref{matrices:subtask:isyatzyforall} igen, men nu med ett imperativt angreppssätt som använder en \code{while}-sats i stället för funktionella \code{forall}. Ta hjälp av en variabel \code{i} som håller reda på index och en variabel \code{foundDiff} som håller reda på om ett avvikande värde upptäcks. Funktionen kräver ca 9 rader, så det kan vara lämpligt att öppna en editor att skriva i medan du klurar ut lösningen. Börja med att skriva pseudokod, gärna med penna på papper. Prova genom att klistra in i REPL.

\Subtask En imperativ implementation av \code{diceMatrixToString} från uppgift \ref{matrices:task:yatzy}\ref{matrices:subtask:diceMatrixToString} med hjälp av förändringsbara  \code{StringBuilder}\footnote{\url{https://www.scala-lang.org/api/2.12.9/scala/collection/mutable/StringBuilder.html}} visas nedan. Förklara hur nedan kod fungerar. Vad händer om \code{xss} är tom? Vad händer om \code{xss} bara innehåller tomma vektorer? Nämn en fördel och en nackdel med att använda \code{val sb: StringBuilder} och \code{append}, jämfört med en vanlig, oföränderlig \code{var s: String} och \code{+} för tillägg i slutet.
\begin{Code}
def diceMatrixToString(xss: Vector[Vector[Int]]): String = 
  val sb = new StringBuilder()
  for(m <- xss.indices) do
    for(n <- xss(m).indices) do
      sb.append(xss(m)(n).toString)
      if n < xss(m).size - 1 then sb.append(" ")
      else if m < xss.size - 1 then sb.append("\n")
    end for
  end for
  sb.toString
\end{Code}

\Subtask Gör som träning en imperativ implementation av \code{filterYatzy} med en \code{for}-\code{do}-sats (alltså utan att använda \code{filter}, och utan att använda \code{yield}).


\Subtask Förklara hur nedan funktionella implementation av \code{filterYatzy} med \code{for}-\code{yield}-uttryck fungerar. Tycker du din imperativa lösning är lättare eller svårare att läsa och förstå jämfört nedan funktionella lösning?
\begin{CodeSmall}
def filterYatzy(xss: Vector[Vector[Int]]): Vector[Vector[Int]] = 
  (for i <- xss.indices if isYatzy(xss(i)) yield xss(i)).toVector
\end{CodeSmall}


\SOLUTION

\TaskSolved \what

\SubtaskSolved  \begin{Code}
def isYatzy(xs: Vector[Int]): Boolean = 
  var foundDiff = false
  var i = 0
  while (i < xs.size && !foundDiff) do
    foundDiff = xs(i) != xs(0)
    i += 1
  end while
  !foundDiff
\end{Code}


\SubtaskSolved  Funktionen går igenom varje matrisrad, där den i sin tur går igenom
varje element på raden och lägger till i \code{StringBuilder}-objektet. Om det inte är
det sista elementet på raden läggs även ett blanktecken till, annars läggs ett
nyradstecken till. Undantaget är sista raden, där inget nyradstecken läggs till.
Slutligen konverteras \code{StringBuilder}-objektet till en \code{String} som
returneras.


Är \code{xss} tom blir \code{xss.indices} en tom \code{Range} och den yttre \code{for}-loopen hoppas över och en tom sträng returneras.
Är alla rader tomma hoppas i stället de inre \code{for}-looparna över, med samma resultat.

\emph{Fördel:} \code{StringBuilder} är snabbare vid tillägg på slutet vid stora strängar (men här kommer det inte märkas eftersom strängen är så liten).

\emph{Nackdel:} StringBuilder-koden uppfattas av många som svårare att läsa.

\SubtaskSolved
\begin{Code}
def filterYatzy(xss: Vector[Vector[Int]]): Vector[Vector[Int]] = 
  var result: Vector[Vector[Int]] = Vector()
  for i <- xss.indices if isYatzy(xss(i)) do result = result :+ xss(i)
  result
\end{Code}

\SubtaskSolved  Varje looprunda ger en vektor \code{xss(i)} om filtervillkoret är uppfyllt och resultatet av \code{for}-uttrycket blir en vektor med vektorer som är yatzyslag.

\QUESTEND



\WHAT{Strängtabell med kolumnrubriker.}

\QUESTBEGIN

\Task  \what~  %Denna övning utgör en början på laboration \hyperref[section:lab:survey]{\texttt{survey}} i avsnitt \ref{section:lab:survey} på sidan \pageref{section:lab:survey}.

\Subtask Implementera case-klassen \code{Table} enligt specifikationen nedan. Du kan förutsätta att alla rader har lika många kolumner som antalet element i \code{headings}, samt att alla rubrikerna i \code{headings} är unika. Parametern \code{sep} anger det tecken som används för att separera kolumner. Detta förutsätts också gälla för indatafiler som läses in med \code{fromFile}.

\emph{Tips:}
\begin{itemize}%[nolistsep,noitemsep]
\item Värdet \code{indexOfHeading} kan skapas med hjälp av metoden \code{zipWithIndex} som fungerar på alla sekvenssamlingar, samt metoden \code{toMap} som fungerar på sekvenser av 2-tupler. Undersök först hur metoderna fungerar i REPL och sök upp deras dokumentation.
\item Skapa en indatafil som du kan använda för att testa att \code{Table} fungerar.
\end{itemize}


\begin{CodeSmall}
case class Table(
  data: Vector[Vector[String]],
  headings: Vector[String],
  sep: Char
):
  /** A 2-tuple with (number of rows, number of columns) in data */
  val dim: (Int, Int) = ???

  /** The element in row r and column c of data, counting from 0 */
  def apply(r: Int, c: Int): String = ???

  /** The row-vector r in data, counting from 0 */
  def row(r: Int): Vector[String]= ???

  /** The column-vector c in data, counting from 0 */
  def col(c: Int): Vector[String] = ???

  /** A map from heading to index counting from 0 */
  lazy val indexOfHeading: Map[String, Int] = ???

  /** The column-vector with heading h in data */
  def col(h: String): Vector[String] = ???

  /** A vector with the distinct, sorted values of col with heading h */
  def values(h: String): Vector[String] = ???

  /** Headings and data with columns separated by sep */
  override lazy val toString: String = ???

object Table:
  /** Creates a new Table from fileName with columns split by sep */
  def fromFile(fileName: String, sep: Char = ';'): Table = ???
\end{CodeSmall}

\Subtask Skapa med hjälp av \code{Table} ett program som kan köras från terminalen med \texttt{scala run infile.csv ';'} som ger en utskrift av antalet förekomster av olika värden i respektive kolumn (alltså en variant av registrering).



\SOLUTION

\TaskSolved \what

\SubtaskSolved  \begin{CodeSmall}
case class Table(
  data: Vector[Vector[String]],
  headings: Vector[String],
  sep: Char
):

  val dim: (Int, Int) = (data.size, headings.size)

  def apply(r: Int, c: Int): String = data(r)(c)

  def row(r: Int): Vector[String]= data(r)

  def col(c: Int): Vector[String] = data.map(r => r(c))

  lazy val indexOfHeading: Map[String, Int] = headings.zipWithIndex.toMap

  def col(h: String): Vector[String] = col(indexOfHeading(h))

  def values(h: String): Vector[String] = col(h).distinct.sorted

  override def toString: String =
    val s = sep.toString
    headings.mkString(s) + "\n" +data.map(_.mkString(s)).mkString("\n")

object Table:
  def fromFile(fileName: String, sep: Char = ';'): Table = 
    val lines = scala.io.Source.fromFile(fileName).getLines.toVector
    val matrix= lines.map(_.split(sep).toVector)
    new Table(matrix.tail, matrix.head, sep)
\end{CodeSmall}

\SubtaskSolved  \begin{CodeSmall}
@main 
def run(fileName: String, separator: String): Unit = 
  require(separator.length == 1, "separator ska vara exakt ett tecken")
  val t = Table.fromFile(fileName, separator.head)
  val counts: Vector[Vector[String]] =
    (0 until t.dim._2)
      .map(i => t.values(t.headings(i))
      .map(x => s"$x: ${t.col(i).count(_ == x)}"))
      .toVector
  for (i <- 0 until t.dim._2) do
    println(s"\nColumn: ${i + 1}, ${t.headings(i)}:")
    for (j <- 0 until counts(i).length) do
      println(counts(i)(j))
\end{CodeSmall}

\QUESTEND




\WHAT{Skapa ett yatzy-spel för användning i terminalen.}

\QUESTBEGIN

\Task  \what~%
% \Subtask Skapa en yatzy-matris enligt nedan specifikation. Läs om hur de olika predikaten för att kolla olika giltiga kombinationer i Yatzy ska fungera här: \href{https://en.wikipedia.org/wiki/Yahtzee}{en.wikipedia.org/wiki/Yahtzee}. Bygg ett huvudprogram som testar dina funktioner. Kompilera och testa i terminalen allteftersom du lägger till nya funktioner.
%
% \begin{CodeSmall}
% /** En skiss på en klass som kan användas till ett förenklat yatzy-spel */
% case class YatzyRows(val rows: Vector[Vector[Int]]) {
%   /** A new YatzyRows with a new row of 5 dice rolls appended to rows  */
%   def roll: YatzyRows = ???
%
%   /** A new YatzyRows with some indices of the last row re-rolled  */
%   def reroll(indices: Vector[Int]): YatzyRows = ???
% }
%
% object YatzyRows {
%   def isYatzy(xs: Vector[Int]): Boolean = ???
%   def isThreeOfAKind(xs: Vector[Int]): Boolean = ???
%   def isFourOfAKind(xs: Vector[Int]): Boolean = ???
%   def isFullHouse(xs: Vector[Int]): Boolean = ???
%   def isSmallStraight(xs: Vector[Int]): Boolean = ???
%   def isLargeStraight(xs: Vector[Int]): Boolean = ???
% }
% \end{CodeSmall}
%
%
% \Subtask Använd \code{YatzyRows} för att med hjälp av många tärningskast beräkna sannolikheter för några olika giltiga kombinationer. Använd, om du vill, möjligheten som reglerna ger att slå om tärningar i två ytterliggare kast, där de tärningar som slås om väljs slumpmässigt.
%
%\Subtask
Bygg ett förenklat yatzy-spel i terminalen där användaren kan bestämma vilka tärningar som ska slås om. Börja med något riktigt enkelt och bygg sedan vidare på ditt spel genom att införa fler och fler funktioner.

\SOLUTION


\TaskSolved \what
     %starts with: \emph{Skapa ett yatzy-spel för %%%

 --

% \SubtaskSolved   \begin{CodeSmall}
% /** En skiss på en klass som kan användas till ett förenklat yatzy-spel */
% case class YatzyRows(val rows: Vector[Vector[Int]]) {
%
%   private def throwDie: Int = (math.random() * 6).toInt + 1
%
%   /** A new YatzyRows with a new row of 5 dice rolls appended to rows */
%   def roll: YatzyRows = new YatzyRows(rows :+ Vector.fill(5)(throwDie))
%
%   /** A new YatzyRow with some indices of the last row re-rolled */
%   def reroll(indices: Vector[Int]): YatzyRows =
%     new YatzyRows(rows :+ rows(rows.length - 1).zipWithIndex.map {
%       case (x, i) => if (indices.contains(i)) throwDie else x
%     })
% }
% object YatzyRows {
%
%   def isYatzy(xs: Vector[Int]): Boolean = xs.forall(_ == xs(0))
%
%   def isThreeOfAKind(xs: Vector[Int]): Boolean =
%     xs.exists(x => xs.count(_ == x) >= 3)
%
%   def isFourOfAKind(xs: Vector[Int]): Boolean =
%     xs.exists(x => xs.count(_ == x) >= 4)
%
%   def isFullHouse(xs: Vector[Int]): Boolean =
%     xs.exists(x => xs.count(_ == x) == 3) &&
%     xs.exists(x => xs.count(_ == x) == 2)
%
%   def isSmallStraight(xs: Vector[Int]): Boolean =
%     xs.forall(x => xs.count(_ == x) == 1) && !xs.exists(_ == 6)
%
%   def isLargeStraight(xs: Vector[Int]): Boolean =
%     xs.forall(x => xs.count(_ == x) == 1) && !xs.exists(_ == 1)
% }
%
% \end{CodeSmall}
% Observera att fem stycken 2:or uppfyller kraven för Yatzy, men även för triss och fyrtal.
%
% \SubtaskSolved   Slumpen gör att utfallet inte kommer stämma exakt överens med teorin, men för ett stort antal kast bör resultaten hamna ganska nära. De teoretiska sannolikheterna (utan omkast) finns i \ref{yatzyProb}.
% \begin{table}[h]
% \centering
% \caption{Sannolikhet för olika Yatzy-resultat}
% \label{yatzyProb}
% \begin{tabular}{ll}
% Yatzy&  $0,077\%$  \\
% $\geq3$ av samma& $21\%$\\
% $\geq4$ av samma& $2,0\%$\\
% Kåk& $3,9\%$\\
% Liten stege& $1,5\%$\\
% Stor stege& $1,5\%$
% \end{tabular}
% \end{table}
%
% Kodexempel:
% \begin{CodeSmall}
% import YatzyRows._
%
% object YatzyStats extends App {
%   val n = 1000000.0
%   var yr = YatzyRows(Vector(Vector[Int]()))
%   for (i <- 1 to n.toInt) yr = yr.roll
%   println(s"Yatzy: ${yr.rows.count(isYatzy(_)) / n * 100}%")
%   println(s"Three of a kind: ${yr.rows.count(isThreeOfAKind(_)) / n * 100}%")
%   println(s"Four of a kind: ${yr.rows.count(isFourOfAKind(_)) / n * 100}%")
%   println(s"Full house: ${yr.rows.count(isFullHouse(_)) / n * 100}%")
%   println(s"Small straight: ${yr.rows.count(isSmallStraight(_)) / n * 100}%")
%   println(s"Large straight: ${yr.rows.count(isLargeStraight(_)) / n * 100}%")
% }
% \end{CodeSmall}
%
% \SubtaskSolved  --

\QUESTEND






\clearpage

\AdvancedTasks %%%%%%%%%%%%%%%%%


\WHAT{Generiska funktioner.}

\QUESTBEGIN

\Task  \what~  En generisk funktion har (minst) en typparameter inom klammerparenteser efter namnet, till exempel \code{[T]}. Denna typ förekommer sedan som typ på (någon av) parametrarna i parameterlistan. Kompilatorn härleder en konkret typ vid kompileringstid och ersätter typparametern med denna konkreta typ. På så sätt kan en funktion fungera för många olika typer.

\Subtask Förklara för varje rad nedan vad som händer.

\begin{REPL}
scala> def tnirp[T](x: T): Unit = println(x.toString.reverse)
scala> tnirp(42)
scala> tnirp("hej")
scala> case class Gurka(vikt: Int)
scala> tnirp(Gurka(42))
scala> tnirp[String](42)
scala> tnirp[Double](42)
\end{REPL}

\Subtask Man kan kombinera generiska funktioner med funktioner som tar funktioner som parametrar. Det är så \code{map} och \code{foreach} är implementerade. Förklara för varje rad nedan vad som händer.

\begin{REPL}
scala> def compose[A, B, C](f: A => B, g: B => C)(x: A): C = g(f(x))
scala> def inc(x: Int): Int = x + 1
scala> def half(x: Int): Double = x / 2.0
scala> compose(inc, half)(42)
scala> compose(half, inc)(42)
\end{REPL}

\Subtask Hur lyder felmeddelandet på sista raden ovan? Ändra \code{inc} och/eller \code{half} så att typerna passar.

\SOLUTION

\TaskSolved \what
     %starts with: \emph{Generiska funkioner.} En %%%

%4.a)
\SubtaskSolved   \begin{enumerate}
\item --
\item Strängrepresentationen av \code{42} spegelvänds
\item \code{"hej"} spegelvänds - \code{toString} av en sträng ger en likadan sträng
\item --
\item Gurk-objektets strängrepresentation spegelvänds
\item Funktionens typparameter matchar inte parameterns typ: \code{42} är ingen sträng
\item Implicit typkonvertering till \code{Double} sker för att stämma överens med typparametern, vilket ger en strängrepresentation med decimal
\end{enumerate}

%4.b)
\SubtaskSolved   \begin{enumerate}
\item En funktion definieras så att den tar emot två andra funktioner som argument, sätter ihop dem, och matar in ett tredje argument till den den sammansatta funktionen.
\item En funktion som inkrementerar ett heltal med 1 definieras.
\item En funktion som halverar ett flyttal definieras.
\item \code{42} matas in i \code{inc()} och resultatet (\code{43}) matas vidare till \code{half()}. Inuti \code{half()} sker implicit typkonvertering till \code{Double} då talet divideras med ett flyttal (\code{2.0}) och resultatet blir \code{43.0 / 2.0}, alltså \code{21.5}.
\item Resultatet från \code{half()} är av typ \code{Double}, medan \code{inc()} tar emot ett argument av typ \code{Int}. Då flyttal generellt inte kan konverteras till heltal utan informationsförlust sker ingen implicit konvertering, istället sker ett kompileringsfel.
\end{enumerate}

%4.c)
\SubtaskSolved  \begin{Code}
def inc(x: Double): Double = x + 1.0
\end{Code}
Nu ges kompileringsfel på rad 4 istället, vilket kan lösas med följande ändring:
\begin{Code}
def half(x: Double): Double = x / 2.0
\end{Code}

\QUESTEND




\WHAT{Generiska klasser.}

\QUESTBEGIN

\Task  \what~  Även klasser kan vara generiska. En generisk klass har (minst) en typparameter inom klammerparenteser efter klassens namn.

\Subtask Testa nedan generiska klass \code{Cell[T]} i REPL. Skapa instanser av klassen \code{Cell[T]} där typparametern \code{T} binds till olika konkreta typer och förklara vad som händer.

\begin{REPL}
scala> class Cell[T](var value: T):
         override def toString = "Cell(" + value + ")"
       
scala> new Cell(42)
scala> new Cell("hej")
scala> new Cell(new Cell(math.Pi))
scala> new Cell[String](42)
scala> new Cell[Double](42)
\end{REPL}

\Subtask Lägg till metoden \code{def concat[U](that: Cell[U]):Cell[String]} i klassen \code{Cell} som konkatenerar strängrepresentationerna av de båda cellvärdena.

\begin{REPL}
scala> val a = new Cell("hej")
scala> val b = new Cell(42)
scala> a concat b
\end{REPL}

\Subtask Vilken sorts celler kan du konkatenera om du tar bort typparameternamnet \code{U} i \code{concat} samtidigt som du använder \code{Cell[T]} som typ på värdeparametern \code{that}? Vad ger det för konsekvenser för celler av annan typ än \code{Cell[String]}?

\SOLUTION

\TaskSolved \what

%5.a)
\SubtaskSolved  --

%5.b)
\SubtaskSolved  \begin{Code}
class Cell[T](var value: T):
  override def toString = "Cell(" + value + ")"
  def concat[U](that: Cell[U]): Cell[String] = 
    Cell(s"$value${that.value}")
\end{Code}

%5.c)
\SubtaskSolved   Endast celler med samma typparameter kan nu konkateneras. Eftersom \code{concat()} returnerar ett objekt av typ \code{Cell[String]} kan ett ojämnt antal celler med någon annan typparameter än \code{String} alltså inte längre konkateneras. Är antalet jämnt går det att konkatenera dem parvis och sedan konkatenera de returnerade \code{Cell[String]}-objekten, men det är något omständigt.

\QUESTEND

\WHAT{Implementera fler generiska metoder i \code{Matrix[T]}.}

\QUESTBEGIN

\Task \what~ Bygg vidare på uppgift \ref{exe:matrices:labprep} och implementera nedan specifikation. Skapa egna tester som kontrollerar att alla metoder fungerar som förväntat.

\begin{ScalaSpec}{Matrix[T]}
/** En oföränderlig, generisk Matris-klass. */
case class Matrix[T](data: Vector[Vector[T]]):
  require(???)  // garantera att alla rader har lika många kolumner

  /** Ger ett par med antal rader och kolumner. */
  val dim: (Int, Int) = ???

  /** Ger elementet på plats (row, col). */
  def apply(row: Int, col: Int): T = ???

  /** Ger en ny matris där elementet på plats (row, col) har värdet value. */
  def updated(row: Int, col: Int)(value: T): Matrix[T] =  ???

  /** Applicerar f på alla element. */
  def foreach(f: T => Unit): Unit = ???

  /** Applicerar f på alla index. */
  def foreachIndex(f: (Int, Int) => Unit): Unit = ???

  /** Ger en ny matris med resultaten av elementvis applicering av f. */
  def map[U](f: T => U): Matrix[U] = ???

  /** Ger en ny matris med resultaten av applicering av f på varje index. */
  def mapIndex[U](f: (Int, Int) => U): Matrix[U] = ???

  /** Ger en utskriftsvänlig strängrepresentation av matrisen. */
  override def toString = ???

object Matrix:
  /** Ger en matris med dimension dim där alla element har värdet value. */
  def fill[T](dim: (Int, Int))(value: T): Matrix[T] = ???
\end{ScalaSpec}

\SOLUTION


\TaskSolved \what

\begin{CodeSmall}
case class Matrix[T](data: Vector[Vector[T]]):
  require(data.forall(row => row.size == data(0).size))

  val dim: (Int, Int) = (data.length, data(0).length)

  def apply(row: Int, col: Int): T = data(row)(col)

  def updated(row: Int, col: Int)(value: T): Matrix[T] =
    Matrix(data.updated(row, data(row).updated(col, value)))

  def foreach(f: T => Unit): Unit = data.foreach(_.foreach(f))

  def foreachIndex(f: (Int, Int) => Unit): Unit =
    for r <- data.indices; c <- data(r).indices do f(r, c)

  def map[U](f: T => U): Matrix[U] = Matrix(data.map(_.map(f)))

  def mapIndex[U](f: (Int, Int) => U): Matrix[U] =
    var result = Matrix.fill(dim)(f(0,0))
    for 
      r <- data.indices
      c <- data(r).indices 
    do
      result = result.updated(r, c)(f(r, c))
    end for
    result

  override def toString =
    s"""Matrix of dim $dim:\n${ data.map(_.mkString(" ")).mkString("\n") }"""

object Matrix:
  def fill[T](dim: (Int, Int))(value: T): Matrix[T] =
    Matrix[T](Vector.fill(dim._1, dim._2)(value))
\end{CodeSmall}


\QUESTEND





% \WHAT{Skapa en generisk, oföränderlig matrisklass.}
%
% \QUESTBEGIN
%
% \Task \label{task:generic-matrix} \what~   Med hjälp av en typparameter kan vi skapa en matrisklass som kan innehålla vilka element som helst. Implementera nedan specifikation. Testa din matrisklass i REPL för olika typer av element.
%
% \begin{ScalaSpec}{Matrix[T]}
% case class Matrix[T](data: Vector[Vector[T]]){
%
%   def foreachRowCol(f: (Int, Int, T) => Unit): Unit =
%     for (r <- 0 until data.size) {
%       for (c <- 0 until data(r).size) {
%         f(r, c, data(r)(c))
%       }
%     }
%
%   def map[U](f: T => U): Matrix[U] = Matrix(data.map(_.map(f)))
%
%   /** The element at row r and column c */
%   def apply(r: Int, c: Int): T = ???
%
%   /** Gives Some[T](element) at row r and column c
%    *  if r and c are within index bounds, else None */
%   def get(r: Int, c: Int): Option[T] = ???
%
%   /** The row vector of row r */
%   def row(r: Int): Vector[T] = ???
%
%   /** The column vector of column c */
%   def col(c: Int): Vector[T] = ???
%
%   /** A new Matrix with element at row r and col c updated */
%   def updated(r: Int, c: Int, value: T): Matrix[T] = ???
% }
% object Matrix {
%   def fill[T](rowSize: Int, colSize: Int)(init: T): Matrix[T] =
%     new Matrix(Vector.fill(rowSize)(Vector.fill(colSize)(init)))
% }
% \end{ScalaSpec}
%
% \SOLUTION
%
%
% \TaskSolved \what
%      %%%TODO number  8 %%%starts with: \label{task:generic-matrix} \em%%%
%
% \SubtaskSolved  -- %%%TODO in task 8 %%%
%
%
%
% \QUESTEND
%

% \clearpage
%
% \WHAT{Skapa en Sprite-editor.}
%
% \QUESTBEGIN
%
% \Task  \what~ Använd matrisklassen från uppgift \ref{task:generic-matrix} för att göra en SpriteEditor med JColorChoser enligt nedan skiss.
%
% \begin{Code}
% object ColorChooser {
%   import java.awt.Color
%   import javax.swing.JColorChooser
%
%   var title = "Pick Color"
%   private val chooser = new JColorChooser(Color.BLACK)
%   private val dialog = JColorChooser.
%     createDialog(null, title, true, jcs, null, null)
%
%   def getColor(initColor: Color = Color.BLACK): Color = {
%     chooser.setColor(initColor)
%     dialog.setVisible(true)
%     chooser.getColor
%   }
% }
%
% class Sprite(// en bild med många lager av pixlar i olika färger
%   val id: String,
%   val size: (Int, Int),
%   val pixels: Matrix[Int],   // färg i colors, -1 betyder genomskinlig
%   var scale: Int,            // uppskalning av storlek i pixlar
%   var colors: Vector[Color], // tillgängliga färger
%   var pos: (Int, Int, Int)   // (row, col, layer)
% ){
%   def row = pos._1
%   def col = pos._2
%   def layer = pos._3
% }
%
% class SpriteEditor(
%     rows: Int = 64, cols: Int = 64,
%     scale: Int = 16, nColors: Int = 16) {
%   private val w = new SimpleWindow(???)
%   def edit: Unit = ???
% }
%
% \end{Code}
%
%
%
% \SOLUTION
%
%
% \TaskSolved \what
%      %%%TODO number  9 %%%starts with: \TODO \emph{Klasser för täta oc%%%
%
% \SubtaskSolved  -- %%%TODO in task 9 %%%
%
% \SubtaskSolved  -- %%%TODO in task 9 %%%
%
% \SubtaskSolved  -- %%%TODO in task 9 %%%
%
% \SubtaskSolved  -- %%%TODO in task 9 %%%
%
% \SubtaskSolved  -- %%%TODO in task 9 %%%
%
% \SubtaskSolved  -- %%%TODO in task 9 %%%
%
%
%
% \QUESTEND




% \WHAT{Klasser för täta och glesa matematiska matriser med flyttal.}
%
% \QUESTBEGIN
%
% \Task  \what~   Läs om matrisräkning här: \href{https://sv.wikipedia.org/wiki/Matris}{sv.wikipedia.org/wiki/Matris}
%
% \Subtask Skapa en oföränderlig klass \code{DenseMatrix} för matematiska matriser med dubbelprecisionsflyttal. \code{DenseMatrix} ska internt lagra elementen i en privat \emph{endimensionell} array av flyttal av typen \code{Array[Double]}.
%
% Klassen ska inte vara en case-klass. Det ska gå att skapa matriser med uttryck så som  \code{DenseMatrix.ofDim(3,7)(1.0,42,3.2,1.0,2.2,3)} tack vare ett kompanjonsobjekt med lämplig fabriksmetod som anropar den privata konstruktorn.  Om antalet element är för litet i förhållande till den angivna dimensionen så fyll på med nollor.
%
% \Subtask Överskugga metoderna equals och hashcode och ge \code{DenseMatrix} innehållslikhet i stället för referenslikhet.
%
% \Subtask Implementera egna innehålllikhetsmetoder med namnet \code{===} på \code{DenseMatrix} som är typsäker, d.v.s. bara tillåter innehållsjämförelse mellan täta matriser.
%
% \Subtask Läs om glesa matriser här: \href{https://sv.wikipedia.org/wiki/Gles_matris}{https://sv.wikipedia.org/wiki/Gles\_matris} och implementera \code{SparseMatrix} med ett privat attribut av typen \\ \code{mutable.Map[(Int, Int), Double]} som bara lagrar index som inte är noll.
%
% \Subtask Skapa ett \code{trait Matrix} som både \code{DenseMatrix} och \code{SparseMatrix} ärver, med lämpliga abstrakta och konkreta medlemmar. Implementera addition, subtraktion och multiplikation av täta och glesa matriser.
%
% %\Task \emph{Matriser med \jcode{ArrayList} i Java.} Om man i Java inte vet antalet element i matrisen från början kan man använda en lista av typen \jcode{ArrayList}, där varje element i sin tur innehåller en lista av typen\jcode{ArrayList}. Javas \jcode{ArrayList} är en generisk samling som motsvaras av Scalas \code{ArrayBuffer}. Generiska samlingar i Java kan endast innehålla referenstyper; vill man ha en primitiv typ, t.ex. \jcode{int}, behöver man packa in denna i en s.k. wrapper-klass, t.ex.  klassen \jcode{Integer}. Det finns en wrapper-klass för varje primitiv typ i Java. Matristypen för en heltalstyp i Java skrivs \jcode{ArrayList<ArrayList<Integer>>} där alltså \code{<T>} motsvarar Scalas hakparenteser \code{[T]} för typparametern T.
% %
% %
%
% \SOLUTION
%
% \TaskSolved \what
%      %%%TODO number  10 %%%starts with: \emph{Matriser med \jcode{Array%%%
%
% \SubtaskSolved  -- %%%TODO in task 10 %%%
% \QUESTEND

%!TEX encoding = UTF-8 Unicode
%!TEX root = ../compendium2.tex

\Lab{\LabWeekEIGHT}

\begin{Goals}
\item Kunna skapa och använda matriser.
\item Kunna iterera över matriser med nästlade for-loopar.
\item Träna på algoritmkonstruktion.
\item Använda en integrerad utvecklingsmiljö (IDE).
\end{Goals}

\begin{Preparations}
\item Gör övning {\tt \ExeWeekEIGHT} i kapitel \ref{chapter:W08}, speciellt övning \ref{exe:matrices:labprep}.

\item Läs igenom hela laborationen och studera den befintliga koden i \TODO \code{workspace}.

\item Läs appendix \ref{appendix:ide} och välj vilken IDE du ska använda (ScalaIDE/Eclipse eller IntelliJ IDEA). Säkerställ att du får igång en av dessa utvecklingsmijöer genom att köra hello-world-exemplet och sedan ladda ner och importera kursens \TODO workspace enligt instruktionerna i appendix \ref{appendix:ide}.
\end{Preparations}

\subsection{Bakgrund}



\subsection{Obligatoriska krav}



\subsection{Valbara krav -- välj minst ett}


%!TEX encoding = UTF-8 Unicode

%!TEX root = ../compendium2.tex

%!TEX encoding = UTF-8 Unicode
\chapter{Arv}\label{chapter:W09}
Begrepp som ingår i denna veckas studier:
\begin{itemize}[noitemsep,label={$\square$},leftmargin=*]
\item arv
\item polymorfism
\item trait
\item extends
\item asInstanceOf
\item with
\item inmixning
\item supertyp
\item subtyp
\item bastyp
\item override
\item klasshierarkin i Scala: Any AnyRef Object AnyVal Null Nothing
\item referenstyper vs värdetyper
\item klasshierarkin i scala.collection
\item Shape som bastyp till Rectangle och Circle
\item accessregler vid arv
\item protected
\item final
\item klass vs trait
\item abstract class
\item case-object
\item typer med uppräknade värden
\item gränssnitt
\item trait vs interface
\item programmeringsgränssnitt (api)\end{itemize}

\clearpage\section{Teori}
%!TEX encoding = UTF-8 Unicode
%!TEX root = ../lect-w09.tex


\Subsection{\texttt{scala.collection}} %%%%%%%%%%%%%%%%%%%%%%%%%%%%%%%%%%%



% \begin{Slide}{Typparameter möjliggör generiska samlingar}\SlideFontSmall
%
% \begin{itemize}
%   \item Med \Emph{generisk} \Eng{generic} kod menar man att koden kan hantera data av \Alert{godtycklig} typ.
%   \item Funktioner och klasser kan, förutom vanliga parametrar, även ha \Emph{typparametrar} som skrivs i en \Alert{egen} parameterlista med \Alert{hakparenteser} i stället för vanliga parenteser.
%
%   \item En typparameter gör så att funktioner och datastrukturer blir \Emph{generiska}.
%
%   \item Exempel: Funktionerna \code{baklänges} 1--4 nedan är ordnade från specifik typ till mer generell typ.
%
% \begin{Code}
% def baklänges1(xs: Vector[Int]): Vector[Int] = xs.reverse
%
% def baklänges2[T](xs: Vector[T]): Vector[T] = xs.reverse
%
% def baklänges3(xs: Seq[T]): Vector[T] = xs.reverse.toVector
%
% def baklänges4(xs: Seq[T]): Seq[T] = xs.reverse  //reverse avgör samling
% \end{Code}
% \item Mer om typparametrar i w08.
% \end{itemize}
% \end{Slide}



\begin{Slide}{Hierarki av samlingstyper i \texttt{scala.collection} v2.13}

\begin{multicols}{2}
\begin{tikzpicture}[sibling distance=5.0em,->,>=stealth', inner sep=3pt, %scale=0.5,
  every node/.style = {shape=rectangle, draw, align=center,font=\small\ttfamily},
  class/.style = {fill=blue!20},
  trait/.style = {rounded corners, fill=red!20}]
  \node[trait] {Iterable}
      child { node[trait] {Seq} }
      child { node[trait] {Set} }
      child { node[trait] {Map} }
  ;
\end{tikzpicture}

\columnbreak

{\SlideFontTiny

\code{Iterable} har metoder som är implementerade med hjälp av: \\
\code{def foreach[U](f: Elem => U): Unit}\\
\code{def iterator: Iterator[A] }

}

\begin{itemize}\SlideFontTiny
\item[] \code{Seq}: ordnade i sekvens
\item[] \code{Set}: unika element
\item[] \code{Map}: par av (nyckel, värde)
\end{itemize}


\end{multicols}

{\SlideFontSmall Samlingen \Emph{\texttt{Vector}} är en \code{Seq} som är en \code{Iterable}. \\ \vspace{0.5em}%\pause
De konkreta samlingarna är uppdelade i dessa paket:\\
\code{scala.collection.immutable} \hfill där flera är \Emph{automatiskt} importerade\\
\code{scala.collection.mutable}  \hfill som \Alert{måste importeras} explicit\\%\pause
(undantag: primitiva förändringsbara \code{scala.Array} är automatiskt synlig)
}
\end{Slide}




\begin{Slide}{Metoden \texttt{iterator} ger en ''engångs-iterator''}\SlideFontSmall
Med \code{iterator} kan man iterera med \code{while}, men endast \Alert{en   gång}; sedan är iteratorn ''förbrukad''. (Men man kan be om en ny.) Används ''under huven'' i samlingsbiblioteket för att implementera andra metoder.
\begin{REPL}
scala> val xs = Vector(1,2,3,4)
val xs: Vector[Int] = Vector(1, 2, 3, 4)

scala> val it = xs.iterator
val it: Iterator[Int] = <iterator>

scala> while it.hasNext do print(it.next)
1234

scala> it.hasNext
val res0: Boolean = false

scala> it.next
java.util.NoSuchElementException: next on empty iterator
\end{REPL}
\Emph{Normalt} behöver man \Alert{inte} använda \code{iterator}: det finns oftast färdiga metoder som gör det man vill, till exempel \code{foreach}, \code{map}, \code{sum}, \code{min} etc.
\end{Slide}




% \ifkompendium
% \else
% \begin{Slide}{Hierarki av samlingar i scala.collection v2.12}\SlideFontTiny
% \includegraphics[width=0.6\textwidth]{../img/collection/collection-traits}\\
% %\noindent Läs mer om Scalas samlingar här: \\
% \url{https://docs.scala-lang.org/overviews/collections/overview.html}
% \end{Slide}
% \fi



\begin{Slide}{Mer specifika samlingstyper i \texttt{scala.collection}}
Det finns \Alert{mer specifika} \Emph{subtyper} av \code{Seq}, \code{Set} och \code{Map}:
\\ \vspace{1em}

\begin{tikzpicture}[sibling distance=5.8em,->,>=stealth', inner sep=3pt, %scale=0.5,
  every node/.style = {shape=rectangle, draw, align=center,font=\small\ttfamily},
  class/.style = {fill=blue!20},
  trait/.style = {rounded corners, fill=red!20}]
  \node[trait] {Iterable}
      child { node[trait, xshift=-2.4cm] {Seq}
        child { node[trait] {IndexedSeq} }
        child { node[trait] {LinearSeq} }
       }
      child { node[trait, yshift=-0.0cm] {Set}
        child { node[trait] {SortedSet} }
        child { node[trait] {BitSet} }
      }
      child { node[trait, xshift=1.0cm] {Map}
        child { node[trait] {SortedMap} }
    };
\end{tikzpicture}

\pause\vspace{0.5em}
\Emph{\texttt{Vector}} är en \Alert{\texttt{IndexedSeq}} medan
\Emph{\texttt{List}} är en \Alert{\texttt{LinearSeq}}.

\pause\vspace{1em}{\SlideFontSmall
\href
{https://docs.scala-lang.org/overviews/collections-2.13/overview.html}
{docs.scala-lang.org/overviews/collections-2.13/overview.html}
}
\end{Slide}

\begin{Slide}{Några oföränderliga och förändringsbara sekvenssamlingar}\SlideFontSmall
\begin{tabular}{r l l}
\texttt{scala.collection.\Emph{immutable}.Seq.} & & \\
 & \code|IndexedSeq.| & \\
 & & \Emph{\texttt{Vector}} \\
 & & \Emph{\texttt{Range}} \\
 & \code|LinearSeq.| & \\
 & & \Emph{\texttt{List}} \\
   & & \Emph{\texttt{Queue}} \\

\texttt{scala.collection.\Alert{mutable}.Seq.} & & \\
 & \code|IndexedSeq.| & \\
 & & \Alert{\texttt{ArrayBuffer}} \\
 & & \Alert{\texttt{StringBuilder}} \\
 & \code|LinearSeq.| & \\
 & & \Alert{\texttt{ListBuffer}} \\
   & & \Alert{\texttt{Queue}} \\
\end{tabular}

{\SlideFontTiny Fördjupning: Studera samlingars prestanda-egenskaper här:\\ \href{https://docs.scala-lang.org/overviews/collections-2.13/performance-characteristics.html}{docs.scala-lang.org/overviews/collections-2.13/performance-characteristics.html}}
\end{Slide}



\begin{Slide}{Några användbara metoder på samlingar}\SlideFontTiny
\begin{tabular}{r r l}\hline
\texttt{\Emph{Iterable}}
  & \code|xs.size| & antal elementet \\
  & \code|xs.head| & första elementet \\
  & \code|xs.last| & sista elementet \\
  & \code|xs.take(n)| & ny samling med de första n elementet \\
  & \code|xs.drop(n)| & ny samling utan de första n elementet \\
  & \code|xs.foreach(f)| & gör \code|f| på alla element, returtyp \code|Unit|\\
  & \code|xs.map(f)| & gör \code|f| på alla element, ger ny samling \\
  & \code|xs.filter(p)| & ny samling med bara de element där p är sant\\
  & \code|xs.groupBy(f)| & ger en \code|Map| som grupperar värdena enligt f\\
  & \code|xs.mkString(",")| & en kommaseparerad sträng med alla element\\ 
  & \code|xs.zip(ys)| & ny samling med par (x, y); ''zippa ihop'' xs och ys \\
  & \code|xs.zipWithIndex| & ger en \code|Map| med par (x, index för x) \\
  & \code|xs.sliding(n)| & ny samling av samlingar genom glidande ''fönster''\\ \hline

\texttt{\Emph{Seq}}
  & \code|xs.length| & samma som \code|xs.size| \\
  & \code|xs :+ x| & ny samling med x sist efter xs \\
  & \code|x +: xs| & ny samling med x före xs \\ \hline

\end{tabular}
Prova fler samlingsmetoder ur snabbreferensen: ~~\url{http://cs.lth.se/quickref}

\vspace{0.5em}\Emph{Minnesregel} för \code{+:} och \code{:+  } \Alert{Colon on the collection side}

\end{Slide}



% \ifkompendium\else

% \begin{Slide}{scala.collection.immutable}
% \includegraphics[width=0.67\textwidth]{../img/collection/collection-immutable}~~%
% \includegraphics[width=0.3\textwidth]{../img/collection/collection-legend}
% \end{Slide}


% \begin{Slide}{scala.collection.mutable}
% \includegraphics[width=1.05\textwidth]{../img/collection/collection-mutable}
% \end{Slide}

% \fi



% \begin{Slide}{\texttt{Vector} eller \texttt{List}?}\SlideFontTiny
% {\href{http://stackoverflow.com/questions/6928327/when-should-i-choose-vector-in-scala}{stackoverflow.com/questions/6928327/when-should-i-choose-vector-in-scala}}
%
% \begin{enumerate}
% \item If we only need to transform sequences by operations like map, filter, fold etc: basically it does not matter, we should program our algorithm generically and might even benefit from accepting parallel sequences. For sequential operations List is probably a bit faster. But you should benchmark it if you have to optimize.
%
% \item If we need a lot of random access and different updates, so we should use vector, list will be prohibitively slow.
%
% \item If we operate on lists in a classical functional way, building them by prepending and iterating by recursive decomposition: use list, vector will be slower by a factor 10-100 or more.
%
% \item If we have an performance critical algorithm that is basically imperative and does a lot of random access on a list, something like in place quick-sort: use an imperative data structure, e.g. ArrayBuffer, locally and copy your data from and to it.
% \end{enumerate}
% {\href{http://stackoverflow.com/questions/20612729/how-does-scalas-vector-work}{stackoverflow.com/questions/20612729/how-does-scalas-vector-work}}\\
% Mer om tids- och minneskomplexitet i fördjupningskursen och senare kurser.
% \end{Slide}








\Subsection{Repetition: sekvens}

\begin{Slide}{Repetition: Vad är en sekvens?}
\begin{itemize}
\item En sekvens är en \Emph{följd av element} som
  \begin{itemize}
   \item är \Alert{numrerade} (t.ex. från noll), och
   \item är av en viss \Alert{typ} (t.ex. heltal).
  \end{itemize}
  \pause
\item En sekvens kan innehålla \Alert{dubbletter}.
\item En sekvens kan vara \Alert{tom} och ha längden noll.
\item Exempel på en icke-tom sekvens med dubbletter:
\begin{REPLnonum}
scala> val xs = Vector(42, 0, 42, -9, 0, 13, 7)
val xs: Vector[Int] = Vector(42, 0, 42, -9, 0, 13, 7)
\end{REPLnonum}
\pause
\item \Emph{Indexering} ger ett element via dess ordningsnummer:
\begin{REPL}
scala> xs(2)
val res0: Int = 42

scala> xs.apply(2)
val res1: Int = 42
\end{REPL}
\end{itemize}
\end{Slide}


\begin{Slide}{En sträng är också en \texttt{IndexedSeq[Char]}}\SlideFontSmall
Det sker vid behov \Emph{implicit konvertering} från \code{String} till \code{IndexedSeq[Char]}.
\begin{REPLnonum}
scala> val x: IndexedSeq[Char] = "hej"
val x: IndexedSeq[Char] = hej
\end{REPLnonum}

Detta gör att \Alert{alla samlingsmetoder på \texttt{Seq} även funkar på strängar} och även flera andra smidiga strängmetoder erbjuds \Alert{utöver} de som finns i \href{https://www.scala-lang.org/api/current/scala/collection/StringOps.html}{\code{java.lang.String}} genom klassen \href{http://www.scala-lang.org/api/current/scala/collection/immutable/StringOps.html}{\code{StringOps}}.

\vspace{0.5em}
\begin{REPLnonum}
scala> "hej".  //tryck på TAB och se alla strängmetoder
JLine: do you wish to see all 248 possibilities (42 lines)?
\end{REPLnonum}
Detta är en stor fördel med Scala jämfört med många andra språk, som har strängar som inte kan allt som andra sekvenssamlingar kan.
\end{Slide}
  

\begin{Slide}{Konvertera mellan olika samlingstyper}
\begin{itemize}
\item För vanligt förekommande konverteringar finns metoderna \code{toVector}, \code{toList},  \code{toArray}, \code{toBuffer}, \code{toMap}, \code{toSeq}, \code{toIndexedSeq}, \code{toSet},  \code{toString}
\item Metoden \texttt{to} (ny från Scala 2.13) tar ett \Emph{kompanjonsobjekt} ur samlingsbiblioteket som argument och kan användas för konvertering till godtycklig samlingstyp.
\item Detta kräver kopiering om underliggande representation är olika och samlingen är förändringsbar.
\item Kan användas för att t.ex. konvertera mellan oföränderlig och förändringsbar samling:
\end{itemize}
\begin{REPLnonum}
scala> val ms = Set(1,2,3).to(collection.mutable.Set)
val ms: scala.collection.mutable.Set[Int] = HashSet(1, 2, 3)
\end{REPLnonum}
\end{Slide}


\Subsection{Mängd} %%%%%%%%%%%%%%%%%%%%%%%%%%%%%%%%%%%%%%%%%%%%%%%%%%%%%%


\begin{Slide}{Vad är en mängd?}\SlideFontSmall
\begin{itemize}
\item En \Emph{mängd} är en samling \Alert{unika} element av en viss \Alert{typ}.
\item En mängd kan alltså inte innehålla dubbletter:
\begin{REPLnonum}
scala> Set(1,1,2,2,3,3,4,4,5,5)
val res0: Set[Int] = HashSet(5, 1, 2, 3, 4)
\end{REPLnonum}
\pause
\item En mängd är \Alert{inte}  en sekvens: du kan inte utgå från att elementen ligger i någon viss ordning, t.ex. den ordning som de ges vid konstruktion; en mängd har ej längd, men en \Emph{storlek}; metoden \code{size} ger antalet element men metoden \code{length} saknas.
\item En mängd kan vara \Alert{tom} och har då storleken \code{0}.
\pause
\item Man kan gå igenom element i \Emph{någon} ordning (exakt vilken är ej def.), med till exempel \code{xs.map(f)} eller \code{for (x <- xs) yield f(x)}
\pause
\item Det går \Alert{inte} att indexera i en mängd med \code{apply}, som i stället ger \Emph{innehållstest}: \code{Set(1,2,3).apply(3) == true}
\item En mängd \code{Set[T]} med element av typen \code{T} kan således ses som ett \Emph{predikat för innehållstest}: alltså en funktion \code{T => Boolean} som är \code{true} om elementet finns annars \code{false}
\end{itemize}
\end{Slide}


\begin{Slide}{Exempel: Oföränderlig mängd}
\setlength{\leftmargini}{1em}
\begin{itemize}
\item \Emph{Skapa}:
\begin{REPLnonum}
scala> var xs = Set("gurka", "tomat", "banan", "pingvin")
\end{REPLnonum}

\item \Emph{Läsa}: avgöra medlemskap
\begin{REPLnonum}
scala> xs("gurka")
val res1: Boolean = true
\end{REPLnonum}

\item \Emph{Uppdatera}: lägg till element (händer inget om redan finns)
\begin{REPLnonum}
scala> xs = xs + "jordekorre"
\end{REPLnonum}

\item \Emph{Ta bort}: (om finns, annars händer inget)
\begin{REPLnonum}
scala> xs = xs - "gurka"
\end{REPLnonum}
\end{itemize}
{\SlideFontTiny\code{SLUT} = Skapa, Läsa, Uppdatera, Ta bort \hfill\code{CRUD} = Create, Read, Update, Delete}
\end{Slide}


\begin{Slide}{Mysteriet med de försvunna elementen}
Vad händer här?
\begin{REPLnonum}
scala> val xs1 = Vector(1,2,3,4,5,6)
scala> xs1.map(_ % 2).count(_ == 0)
val res0: Int = 3                          // antalet jämna tal
scala> val xs2 = Set(1,2,3,4,5,6)
scala> xs2.map(_ % 2).count(_ == 0)
val res1: Int = 1                          // varför?
\end{REPLnonum}
\pause
Mängdegenskaper ger att \code{xs2.map(_ % 2) == Set(0, 1)}\\
Fundera alltid noga på om du \Alert{riskerar att förlora duplikat} som du egentligen hade velat behålla!\\
\pause
Använd \code{toSeq} på mängd om du behöver sekvensegenskaper:
\begin{REPLnonum}
scala> xs2.toSeq.map(_ % 2).count(_ == 0)
val res1: Int = 3         // med toSeq blir det som vi ville
\end{REPLnonum}

\end{Slide}
  
  




\begin{Slide}{Exempel: Förändringsbar mängd}\SlideFontSmall
Med en \Alert{förändringsbar} mängd kan man stegvis utöka på plats.
\begin{REPL}
scala> val mängd = scala.collection.mutable.Set.empty[Int]

scala> for i <- 1 to 1_000_000 do mängd += i

scala> mängd.contains(-1)   // samma som mängd(-1) eller mängd.apply(-1)
\end{REPL}
En \Emph{mängd} är \Alert{snabb} på att avgöra om ett element \Alert{finns eller inte} i mängden. Ingen linjärsökning krävs eftersom den smarta implementationen av datastrukturen medger snabb uppslagning \Eng{lookup} av ett element.
\pause
\\\vspace{0.5em}Men i en sekvens krävs linjärsökning vid innehållstest:
\begin{REPL}
scala> val sekvens = (1 to 1_000_000).toVector

scala> sekvens.contains(-1)   // kräver linjärsökning ända till slutet
\end{REPL}
\pause\SlideFontTiny Övning: Testa själv att mäta tidsskillnaden med hjälp av:
\begin{Code}
def nanos(b: => Unit) = { val t0 = System.nanoTime; b; System.nanoTime - t0 }
\end{Code}

\end{Slide}






\Subsection{Nyckel-värde-tabell} %%%%%%%%%%%%%%%%%%%%%%%%%%%%%%%%%%%%%%%%


\begin{Slide}{Vad är en nyckel-värde-tabell?}\SlideFontSmall
\begin{itemize}
\item En \Emph{nyckel-värde-tabell} är en samling element som är \Alert{par} med:\\
en \Emph{nyckel} av någon typ \code{K} och ett \Emph{värde} av någon typ \code{V}.
\item En sådan tabell kan skapas ur en sekvens av par \code{(k, v)}\\
där \code{k} är en nyckel och \code{v} är ett värde:
\begin{REPL}
scala> val ålder = Map("Björn" -> 42, "Sandra" -> 35, "Kim" -> 19)
val ålder: Map[String, Int] = Map(Björn -> 42, Sandra -> 35, Kim -> 19)
\end{REPL}
\item Tabellens nycklar utgör en mängd som ges av metoden \code{keySet};\\
nycklarna är \Alert{unika}.
\item Elementen utgör \Alert{inte en sekvens} och har ingen speciell ordning;
\\en nyckel-värde-tabell har ej längd, men en \Emph{storlek};\\metoden \code{size} ger antalet element.
\pause
\item En tabell kan ses som en uppslagsfunktion \Eng{dictionary}:\\alltså en funktion \code{K => V} som ger ett värde givet en nyckel.
\end{itemize}
\end{Slide}


\begin{Slide}{Den fantastiska nyckel-värde-tabellen \texttt{Map}}\SlideFontSmall
\begin{itemize}
\item En \Emph{nyckel-värde-tabell} \Eng{key-value table} är en slags generaliserad vektor där man kan ''indexera'' med godtycklig typ.

\item Kallas även \href{https://sv.wikipedia.org/wiki/Hashtabell}{\Emph{hashtabell}} \Eng{hash table}, \Emph{lexikon} \Eng{Dictionary} eller \Emph{mapp} \Eng{Map} (det blir lätt sammanblandning med metoden \code{map}).

\item Om man vet nyckeln kan man slå upp värdet \Alert{snabbt}, på liknande sätt som indexering sker i en vektor givet heltalsindex.

\item Denna datastruktur är \Alert{mycket användbar} och fungerar som en slags databas i kombination med filtrering, registrering, etc.
\end{itemize}
\end{Slide}


\begin{Slide}{Exempel: Oföränderlig nyckel-värde-tabell}
\setlength{\leftmargini}{1em}
\begin{itemize}
\item \Emph{Skapa}: ge par till metoden \code{apply} 
\begin{REPLsmall}
scala> var födelse = Map("C" -> 1972,  "C++" -> 1983, "C#" -> 2000,
  "Scala" -> 2004, "Java" -> 1995, "Javascript" -> 1995, "Python" -> 1991)
\end{REPLsmall}

\item \Emph{Läsa}: slå upp ett värde med hjälp av en nyckel
\begin{REPLsmall}
scala> val year = födelse.apply("Scala")
val year: Int = 2004
\end{REPLsmall}

\item \Emph{Uppdatera}: lägga till ett par, ersätta ett par
\begin{REPLsmall}
scala> födelse = födelse + ("Kotlin" -> 2011)
födelse: Map[String, Int] = HashMap(Scala -> 2004, C# -> 2000, Python -> 1991, 
Javascript -> 1995, C -> 1972, C++ -> 1983, Kotlin -> 2011, Java -> 1995)
\end{REPLsmall}

\item \Emph{Ta bort} ett par via nyckeln (om finns, annars händer inget)
\begin{REPLsmall}
scala> födelse = födelse - "Python"
födelse: Map[String, Int] = HashMap(Scala -> 2004, C# -> 2000, 
Javascript -> 1995, C -> 1972, C++ -> 1983, Kotlin -> 2011, Java -> 1995)
\end{REPLsmall}
\end{itemize}
\end{Slide}


\begin{Slide}{Fler exempel nyckel-värde-tabell}\SlideFontSmall
Några ofta förekommande metoder på tabeller:
\begin{itemize}
\item \code{xs.keySet} ger en mängd av alla nycklar
\item \code{xs.map(f)} mappar funktionen f på alla par av (key, value)
\item \code{xs.map((k, v) => k -> f(v))} mappar funktionen f på alla värden
\end{itemize}
\begin{REPLsmall}
scala> val färg = Map("gurka" -> "grön", "tomat"->"röd", "aubergine"->"lila")
val färg: Map[String, String] = 
  Map(gurka -> grön, tomat -> röd, aubergine -> lila)

scala> färg("gurka")
val res0: String = grön

scala> färg.keySet
val res1: Set[String] = Set(gurka, tomat, aubergine)

scala> val ärGrönSak = färg.map((k,v) => (k, v == "grön"))
val ärGrönSak: Map[String,Boolean] = 
  Map(gurka -> true, tomat -> false, aubergine -> false)

scala> val baklängesFärg = färg.map((k, v) => k -> v.reverse)
val baklängesFärg: Map[String,String] = 
  Map(gurka -> nörg, tomat -> dör, aubergine -> alil)
\end{REPLsmall}

\end{Slide}



\begin{Slide}{Från sekvens av par till tabell}
\begin{REPL}
scala> val xs = Vector(("Kim",42), ("Pam", 42), ("Kim", 50), ("Pam", 50))
val xs: Vector[(String, Int)] = 
  Vector((Kim,42), (Pam,42), (Kim,50), (Pam,50))

scala> xs.toMap
val res0: Map[String, Int] = 
  Map(Kim -> 50, Pam -> 50) // inga dublettnycklar

scala> val grupperaEfterNamn = xs.groupBy(_._1)
grupperaEfterNamn: Map[String,Vector[(String, Int)]] =
  Map(Kim -> Vector((Kim,42), (Kim,50)), Pam -> Vector((Pam,42), (Pam,50)))

scala> val grupperaEfterÅlder = xs.groupBy(_._2)
grupperaEfterÅlder: Map[Int,Vector[(String, Int)]] =
  Map(50 -> Vector((Kim,50), (Pam,50)), 42 -> Vector((Kim,42), (Pam,42)))
\end{REPL}
\end{Slide}
  
  
\begin{Slide}{Övning: Implementera en \texttt{Multimap}}
\begin{itemize}
\item En \Emph{multimap} är en speciell nyckel-värde-tabell där värdena utgör en samling (ofta en mängd).
\item En multimap samlar alla värden som har samma nyckel.
\item Om du lägger till ett värde så ersätts inte värdet; i stället utökas samlingen av värden.
\end{itemize}
\begin{REPL}
scala> val m = Map(1 -> 2, 1 -> 3, 2 -> 1, 2 -> 2)  // senaste värdet gäller
val m: Map[Int, Int] = Map(1 -> 3, 2 -> 2)

scala> val mm = Multimap(1 -> 2, 1 -> 3, 2 -> 1, 2 -> 2)  // värdena samlas
val mm: Multimap[Int, Int] = Multimap(1 -> Set(2, 3), 2 -> Set(1, 2))
\end{REPL}
Övning: Implementera en multimap som fungerar som ovan, med hjälp av en case-klass med attributet \code{toMap} som är en oföränderlig nyckel-värde-tabell där värdena är en mängd. \\Tips: Använd~\code{groupBy}
\end{Slide}

\begin{Slide}{Lösning: \texttt{Multimap}}
\begin{CodeSmall}
case class Multimap[K, V] private (toMap: Map[K,Set[V]]):
  def apply(k: K): Set[V] = toMap(k)
  
  def +(kv: (K, V)): Multimap[K, V] = kv match 
    case (k, v) if toMap.isDefinedAt(k) => Multimap(toMap.updated(k, toMap(k) + v))
    case (k, v) => Multimap(toMap + (k -> Set(v)))
  
  override def toString = toMap.mkString("Multimap(",", ",")")

object Multimap:
  def apply[K, V](kvs: (K,V)*): Multimap[K, V] = 
    new Multimap(kvs.groupBy(_._1).map((k,xs) => k -> xs.map(_._2).toSet))
\end{CodeSmall}
\end{Slide}


\Subsection{Tips inför veckans uppgifter}

\begin{Slide}{Speciella metoder på förändringsbara samlingar}\SlideFontSmall
Både \code{Set} och \code{Map} finns i \Alert{förändringsbara} varianter med extra metoder för uppdatering av innehållet ''på plats'' utan att nya samlingar skapas.
\begin{REPLsmall}
scala> import scala.collection.mutable

scala> val ms = mutable.Set.empty[Int]
val ms: scala.collection.mutable.Set[Int] = HashSet()

scala> ms += 42
val res0: scala.collection.mutable.Set[Int] = HashSet(42)

scala> ms ++= Seq(1, 2, 3, 1, 2, 3)  // metoden ++= gör samma som addAll 
val res1: scala.collection.mutable.Set[Int] = HashSet(1, 2, 3, 42)

scala> val ms2 = ms diff Set(1, 2)

scala> ms2.mkString("Mängd: ", ", ", " Antal: " + ms2.size)
val res2: String =Mängd: 3, 42 Antal: 2

scala> val ordpar = mutable.Map.empty[String, String]

scala> ordpar ++= Seq("hej" -> "svejs", "abra" -> "kadabra", "ada" -> "lovelace")

scala> ordpar("abra")
val res3: String = kadabra\end{REPLsmall}
\end{Slide}


\begin{Slide}{Övning: Förändringsbar lokalt, returnera oföränderlig}
\SlideFontSmall
Om du vill implementera en imperativ algoritm med en föränderlig samling:\\
Gör gärna detta \Alert{lokalt} i en \Alert{förändringsbar} samling och returnera sedan en \Emph{oföränderlig} samling, genom att köra t.ex. \code{toSet} på en mängd, eller \code{toMap} på en hashtabell, eller \code{toVector} på en \code{ArrayBuffer} eller \code{Array}.\\~\\
Exempel där lösningen har nytta av lokal förändring på plats:
\begin{Code}
def kastaTärningTillsAllaUtfallUtomEtt(sidor: Int = 6): (Int, Set[Int]) = ???
\end{Code}
\begin{REPL}
scala> kastaTärningTillsAllaUtfallUtomEtt()
val res0: (Int, Set[Int]) = (13,HashSet(5, 1, 6, 2, 3))
\end{REPL}
\end{Slide}


\begin{Slide}{Övning: Förändringsbar lokalt, returnera oföränderlig}
\SlideFontSmall
Om du vill implementera en imperativ algoritm med en föränderlig samling:\\
Gör gärna detta \Alert{lokalt} i en \Alert{förändringsbar} samling och returnera sedan en \Emph{oföränderlig} samling, genom att köra t.ex. \code{toSet} på en mängd, eller \code{toMap} på en hashtabell, eller \code{toVector} på en \code{ArrayBuffer} eller \code{Array}.
\begin{Code}
def kastaTärningTillsAllaUtfallUtomEtt(sidor: Int = 6): (Int, Set[Int]) = 
  /* 
    låt s vara en tom förändringsbar heltalsmängd
    låt n vara noll
    så länge mängden s är mindre än sidor - 1 gör:
      lägg till ett nytt tärningskast i s
      uppdatera n så att vi räknar hur många slumptal som dragits
  */
  (n, s.toSet)   // notera toSet som ger oföränderlig mängd
\end{Code}
\begin{REPL}
scala> kastaTärningTillsAllaUtfallUtomEtt()
val res0: (Int, Set[Int]) = (13,HashSet(5, 1, 6, 2, 3))
\end{REPL}
\end{Slide}


\begin{Slide}{Lösning: Förändringsbar lokalt, returnera oföränderlig}
\SlideFontSmall
Om du vill implementera en imperativ algoritm med en föränderlig samling:\\
Gör gärna detta \Alert{lokalt} i en \Alert{förändringsbar} samling och returnera sedan en \Emph{oföränderlig} samling, genom att köra t.ex. \code{toSet} på en mängd, eller \code{toMap} på en hashtabell, eller \code{toVector} på en \code{ArrayBuffer} eller \code{Array}.
\begin{Code}
def kastaTärningTillsAllaUtfallUtomEtt(sidor: Int = 6): (Int, Set[Int]) = 
  val s = scala.collection.mutable.Set.empty[Int] //förändringsbar lokalt
  var n = 0
  while s.size < sidor - 1 do
    s += util.Random.nextInt(sidor) + 1 
    n += 1
  (n, s.toSet)   // notera toSet som ger oföränderlig mängd
\end{Code}
\begin{REPL}
scala> kastaTärningTillsAllaUtfallUtomEtt()
val res0: (Int, Set[Int]) = (13,HashSet(5, 1, 6, 2, 3))
\end{REPL}
I veckans uppgifter används detta i en s.k. \Emph{builder}: Först bygga upp en förändringsbar struktur i \code{FreqMapBuilder} steg för steg,  
och sedan, då alla tillägg är gjorda, övergå till oföränderlig struktur \code{Map[String, Int]}. 
\end{Slide}


\begin{Slide}{Metoden \texttt{sliding}}\SlideFontSmall
Metoden \code{sliding(n)} skapar med ett ''glidande fönster'' en sekvens av
delsekvenser av längd \code{n} genom att ''svepa fönstret'' från början till slut:
\begin{REPL}
scala> val xs = "fem myror är fler än fyra elefanter".split(' ').toVector
val xs: Vector[String] = Vector(fem, myror, är, fler, än, fyra, elefanter)

scala> xs.sliding(2).toVector
val res0: Vector[Vector[String]] =
  Vector(Vector(fem, myror), Vector(myror, är), Vector(är, fler),
      Vector(fler, än), Vector(än, fyra), Vector(fyra, elefanter))

scala> xs.sliding(3).toVector
val res1: Vector[Vector[String]] =
  Vector(Vector(fem, myror, är), Vector(myror, är, fler),
    Vector(är, fler, än), Vector(fler, än, fyra),
      Vector(än, fyra, elefanter))
\end{REPL}
Denna metod har du nytta av på veckans laboration!
\\(se fler exempel på övning)
\end{Slide}
  
  

\begin{Slide}{Metoderna zipWithIndex, groupBy}
\vspace{-0.5em}
\begin{REPL}
scala> val kort = Vector("Knekt", "Dam", "Kung", "Äss")

scala> val kortIndex = kort.zipWithIndex.toMap
kortIndex: Map[String,Int] = Map(Knekt -> 0, Dam -> 1, Kung -> 2, Äss -> 3)

scala> kortIndex("Kung") > kortIndex("Knekt")
res0: Boolean = true

scala> kortIndex.map(p => p._1 -> (p._2 + 11))

scala> val tärningskast = Vector(1,2,3,4,5,6,2,4,6)

scala> val grupperaStörreÄnFyra = tärningskast.groupBy(_ > 4)
grupperaStörreÄnFyra: Map[Boolean,Vector[Int]] =
  Map(false -> Vector(1, 2, 3, 4, 2, 4), true -> Vector(5, 6, 6))

scala> val grupperaLika = tärningskast.groupBy(x => x)
grupperaLika: Map[Int,Vector[Int]] = Map(5 -> Vector(5), 1 -> Vector(1),
  6 -> Vector(6, 6), 2 -> Vector(2, 2), 3 -> Vector(3), 4 -> Vector(4, 4))

scala> val frekvens = tärningskast.groupBy(x => x).map((k,v) => k -> v.size)
frekvens: Map[Int,Int] = Map(5 -> 1, 1 -> 1, 6 -> 2, 2 -> 2, 3 -> 1, 4 -> 2)

\end{REPL}
\end{Slide}

\begin{Slide}{Fler användbara samlingsmetoder}
Exempel att öva på: räkna bokstäver i ord.  \\
Undersök vad som händer i REPL:
\begin{Code}[basicstyle=\SlideFontSize{9}{13}\ttfamily]
val ord = "sex laxar i en laxask sju sjösjuka sjömän"
val uppdelad = ord.split(' ').toVector
val ordlängd = uppdelad.map(_.length)
val ordlängdMap = uppdelad.map(s => (s, s.size)).toMap
val grupperaEfterFörstaBokstav = uppdelad.groupBy(s => s(0))
val bokstäver = ord.toVector.filter(_ != ' ')
val antalX = bokstäver.count(_ == 'x')
val grupperade = bokstäver.groupBy(ch => ch)
val antal = grupperade.map(p => p._1 -> p._2.size)
//samma som ovan men utnyttjar "parameter untupling":
val antal2 = grupperade.map((k,v) => k -> v.size) 
val sorterat = antal.toVector.sortBy(_._2)
val vanligast = antal.maxBy(_._2)
\end{Code}
%https://dotty.epfl.ch/docs/reference/other-new-features/parameter-untupling.html
\end{Slide}
    
    

%!TEX encoding = UTF-8 Unicode
%!TEX root = ../lect-w09.tex


\Subsection{Serialisering och deserialisering}

\begin{Slide}{Serialisering och deserialisering}
\begin{itemize}
  \item Att \Emph{serialisera} innebär att \Alert{koda objekt} i minnet till en avkodningsbar \Alert{sekvens av symboler}, som kan lagras t.ex. i en fil på din hårddisk.
  \item Att \Emph{de-serialisera} innebär att \Alert{avkoda en sekvens av symboler}, t.ex. från en fil, och \Alert{återskapa objekt} i minnet.
\end{itemize}
\end{Slide}


\begin{Slide}{Läsa text från fil och webbsida}\SlideFontSmall
I paketet \code{scala.io} finns singelobjektet \code{Source} med metoderna:
\begin{itemize}\SlideFontTiny
  \item  \code{fromFile} för läsning från fil
  \item \code{fromUrl} för läsning från URL (Universal Resource Locator) \texttt{http://some.site.domain/page.html}

\end{itemize}
\begin{CodeSmall}
def läsFil(filnamn: String, kodning: String = "UTF-8"): String = 
  val s = scala.io.Source.fromFile(filnamn, kodning)
  try s.mkString finally s.close // säkerställ stängning även vid krasch

def läsRaderFrånFil(filnamn: String, kodning: String = "UTF-8"): Vector[String] =
  val s = scala.io.Source.fromFile(filnamn, kodning)
  try s.getLines.toVector finally s.close 

def läsWebbsida(url: String, kodning: String = "UTF-8"): String = 
  val s = scala.io.Source.fromURL(url, kodning)
  try s.mkString finally s.close

def läsRaderWebbsida(url: String, kodning: String = "UTF-8"): Vector[String] =
  val s = scala.io.Source.fromURL(url, kodning) // läs med given teckenkodning
  try s.getLines.toVector finally s.close 

\end{CodeSmall}
{\SlideFontTiny Se vidare veckans övning. Exempel på annan teckenkodning: \code{"ISO-8859-1"} }
\end{Slide}


\begin{Slide}{Serialisering i modulen \texttt{introprog.IO}}\SlideFontSmall
\begin{itemize}
\item I kursens kodbibliotek \code{introprog} finns ett singelobjekt \code{IO} som samlar smidiga funktioner för serialisering och de-serialisering. 
\item Se api-dokumentation här: \\ \url{https://fileadmin.cs.lth.se/pgk/api} \\ Sök på IO och klicka på singelobjektet.
\item Se koden här:\\
\url{https://github.com/lunduniversity/introprog-scalalib/blob/master/src/main/scala/introprog/IO.scala}
\item Om du vill får du gärna använda \code{introprog.IO} istället för \code{scala.io.Source} på labben.  
\end{itemize}
\end{Slide}


%%!TEX encoding = UTF-8 Unicode
\chapter{Arv}\label{chapter:W09}
Begrepp som ingår i denna veckas studier:
\begin{itemize}[noitemsep,label={$\square$},leftmargin=*]
\item arv
\item polymorfism
\item trait
\item extends
\item asInstanceOf
\item with
\item inmixning
\item supertyp
\item subtyp
\item bastyp
\item override
\item klasshierarkin i Scala: Any AnyRef Object AnyVal Null Nothing
\item referenstyper vs värdetyper
\item klasshierarkin i scala.collection
\item Shape som bastyp till Rectangle och Circle
\item accessregler vid arv
\item protected
\item final
\item klass vs trait
\item abstract class
\item case-object
\item typer med uppräknade värden
\item gränssnitt
\item trait vs interface
\item programmeringsgränssnitt (api)\end{itemize}

%!TEX encoding = UTF-8 Unicode
%!TEX root = ../exercises.tex

\ifPreSolution


\Exercise{\ExeWeekNINE}\label{exe:W09}

\begin{Goals}
%!TEX encoding = UTF-8 Unicode
%!TEX root = ../compendium2.tex

%\item Kunna skapa och använda tupler, som variabelvärden, parametrar och returvärden.

%\item Förstå skillnaden mellan ett objekt och en klass och kunna förklara betydelsen av begreppet instans.

%\item Kunna skapa och använda attribut som medlemmar i objekt och klasser och som som klassparametrar.

%\item Kunna beskriva den praktiska nyttan med att ett attribut är privat.

%\item Kunna byta ut implementationen av metoden \code{toString}.

%\item Kunna skapa och använda en objektfabrik med metoden \code{apply}.

%\item Kunna skapa och använda en enkel case-klass.

%\item Kunna använda operatornotation och förklara relationen till punktnotation.

%\item Förstå konsekvensen av uppdatering av föränderlig data i samband med multipla referenser.

%\item Kunna förklara den principiella skillnaderna mellan olika typer av samlingar.
\item Kunna skapa och använda tupler som parametrar och returvärden.
\item Känna till och kunna använda grundläggande metoder på samlingar.
\item Kunna skapa och använda både oföränderliga och föränderliga mängder.
\item Förstå skillnader och likheter mellan en mängd och en sekvens.
\item Kunna beskriva hur algoritmen linjärsökning fungerar.
\item Kunna skapa och använda både oföränderliga och föränderliga nyckel-värde-tabeller.
\item Kunna använda nyckel-värde-tabeller för att implementera registrering.
\item Förstå likheter och skillnader mellan en nyckel-värde-tabell och en sekvens.
\item Kunna spara och läsa data till/från textfiler på disk.
 
\end{Goals}

\begin{Preparations}
\item \StudyTheory{09}
\end{Preparations}

\else

\ExerciseSolution{\ExeWeekNINE}

\fi



\BasicTasks %%%%%%%%%%%%%%%%




\WHAT{Para ihop begrepp med beskrivning.}

\QUESTBEGIN

\Task \what

\vspace{1em}\noindent Koppla varje begrepp med den (förenklade) beskrivning som passar bäst:

\begin{ConceptConnections}
  mängd & 1 & & A & leta i sekvens tills sökkriteriet är uppfyllt \\ 
  nyckel-värde-tabell & 2 & & B & avkoda symbolsekvens och återskapa objekt i minnet \\ 
  mappning & 3 & & C & en unik identifierare \\ 
  nyckel & 4 & & D & egenskapen att finnas kvar efter programmets avslut \\ 
  persistens & 5 & & E & koda objekt till avkodningsbar sekvens av symboler \\ 
  serialisera & 6 & & F & oordnad samling av mappningar med unika nycklar \\ 
  de-serialisera & 7 & & G & \code|nyckel -> värde| \\ 
  linjärsöka & 8 & & H & oordnad samling med unika element \\ 
\end{ConceptConnections}

\SOLUTION

\TaskSolved \what

\begin{ConceptConnections}
  mängd & 1 & ~~\Large$\leadsto$~~ &  H & oordnad samling med unika element \\ 
  nyckel-värde-tabell & 2 & ~~\Large$\leadsto$~~ &  F & oordnad samling av mappningar med unika nycklar \\ 
  mappning & 3 & ~~\Large$\leadsto$~~ &  G & \code|nyckel -> värde| \\ 
  nyckel & 4 & ~~\Large$\leadsto$~~ &  C & en unik identifierare \\ 
  persistens & 5 & ~~\Large$\leadsto$~~ &  D & egenskapen att finnas kvar efter programmets avslut \\ 
  serialisera & 6 & ~~\Large$\leadsto$~~ &  E & koda objekt till avkodningsbar sekvens av symboler \\ 
  de-serialisera & 7 & ~~\Large$\leadsto$~~ &  B & avkoda symbolsekvens och återskapa objekt i minnet \\ 
  linjärsöka & 8 & ~~\Large$\leadsto$~~ &  A & leta i sekvens tills sökkriteriet är uppfyllt \\ 
\end{ConceptConnections}

\QUESTEND



\WHAT{Vad är en mängd?}
\QUESTBEGIN

\Task \what~ Förklara vad som händer nedan. Varför hamnar elementen i en ''konstig'' ordning? Varför ''försvinner'' det element?

\begin{REPL}
scala> val xs = Vector(1,2,3,1,2,3,4,5,7).toSet
xs: scala.collection.immutable.Set[Int] = Set(5, 1, 2, 7, 3, 4)
scala> xs.foreach(print)
512734
\end{REPL}

\SOLUTION

\TaskSolved \what~En mängd är en samling som snabbt kan ge svaret på frågan om ett visst element ingår i samlingen eller ej. Elementen i en mängd är unika. Tilläg av redan existerande element ignoreras. En mängd är inte en  sekvens, eftersom traversering med t.ex. \code{map} eller \code{foreach} inte (nödvändigtvis) sker i den ordning som elementen gavs när mängden konstruerades eller uppdaterades.

\QUESTEND


\WHAT{Använda mängder.}

\QUESTBEGIN

\Task \what

\vspace{1em}\noindent Para ihop varje uttryck till vänster med ett uttryck till höger som har samma värde:

\begin{ConceptConnections}
\input{generated/quiz-w09-setops-taskrows-generated.tex}
\end{ConceptConnections}

\SOLUTION

\TaskSolved \what

\begin{ConceptConnections}
  \code|Set(1, 2) ++ Set(1, 2)          | & 1 & ~~\Large$\leadsto$~~ &  I & \code|Set(1, 2)     | \\ 
  \code|(1 to 3).toSet                  | & 2 & ~~\Large$\leadsto$~~ &  G & \code|Set(1) + 2 + 3| \\ 
  \code|Vector.fill(3)(1).toSet         | & 3 & ~~\Large$\leadsto$~~ &  F & \code|Set(1, 2) - 2 | \\ 
  \code|Set(1, 2, 3) diff Set(1, 2)     | & 4 & ~~\Large$\leadsto$~~ &  B & \code|Set(3)        | \\ 
  \code|(1 to 7).toSet.apply(8)         | & 5 & ~~\Large$\leadsto$~~ &  H & \code|false         | \\ 
  \code|Set(1, 2, 3).sorted             | & 6 & ~~\Large$\leadsto$~~ &  D & \code|error: ...    | \\ 
  \code|Set(2,4) subsetOf (1 to 7).toSet| & 7 & ~~\Large$\leadsto$~~ &  E & \code|true          | \\ 
  \code|Set(1, -1, 2, -2).map(_.abs).sum| & 8 & ~~\Large$\leadsto$~~ &  A & \code|3             | \\ 
  \code|Set(1, 1, 1, 1, 1, 5).sum       | & 9 & ~~\Large$\leadsto$~~ &  C & \code|6             | \\ 
\end{ConceptConnections}

\QUESTEND


\WHAT{Räkna unika ord med hjälp av en mängd.}

\QUESTBEGIN

\Task \what~På veckans laboration ska vi göra automatisk språkbehandling av långa texter som vi delar upp i ord. Med metoden \code{s.split(' ').toVector} kan du dela upp en sträng \code{s} i en sekvens av ord, där \code{s} blivit uppdelad i många strängar vid varje blanktecken och alla blanktecken är borttagna.

\Subtask Använd metoderna \code{split} och \code{toSet} för skapa ett uttryck som beräknar hur många unika ord det finns i strängen \code{hej} nedan:
\begin{REPLnonum}
scala> val hej = "hej hej hemskt mycket hej"
\end{REPLnonum}

\Subtask Mängder är snabba på att kolla om ett element finns i mängden men du kan inte förvänta dig att elementen finns i någon viss ordning. Det finns en sekvenssamlingsmetod som skapar en sekvens med unika element ur en sekvens och behåller den ursprungliga ordningen. Vad heter metoden? \\\emph{Tips:} Leta i snabbreferensen eller sök på nätet. Metoden fungerar på alla samlingar som är av typen \code{Seq} och har ett namn som börjar med bokstäverna \code{di}.

\SOLUTION

\TaskSolved \what~

\SubtaskSolved
\begin{REPL}
scala> val hej = "hej hej hemskt mycket hej"
scala> val n = hej.split(' ').toSet.size
n: Int = 3
\end{REPL}

\SubtaskSolved Metoden \code{distinct} returnerar en sekvens med unika element och bibehållen ursprunglig ordning.

\QUESTEND




\WHAT{Skapa 2-tupler med metoden \code{->} som kan uttalas ''mappas till''.}

\QUESTBEGIN

\Task \what~Vi har tidigare sett hur två olika värden kan samlas i en 2-tupel, till exempel \code{(0, true)}. Par kan även skapas med hjälp av metoden \code{->} enligt nedan. Testa detta i REPL:
\begin{REPL}
scala> ("Skåne", "Lund")          // ett strängpar med vanlig 2-tupel
scala> "Skåne" -> "Lund"           // operatornotation med ->
scala> "Skåne".->("Lund")         // punktnotation med -> (inte alls vanligt)
\end{REPL}
Metoden \code{->} fungerar med alla typer och är en fabriksmetod för par. Metodnamnet liknar en högerpil och illustrerar en mappning från första till andra värdet.

\Subtask Fungerar det på par skapade med \code{->} att använda metoderna \code{_1} och \code{_2}?


\Subtask Deklarera en variabel \code{val huvudstad: Vector[(String, String)]} som innehåller mappningar mellan geografiska områden och deras huvudstäder enligt tabellen nedan.

\begin{table}[H]
  \renewcommand{\arraystretch}{1.2}
  \begin{tabular}{|l|l|}\hline
  Sverige & Stockholm \\\hline
  Danmark & Köpenhamn \\\hline
  Grönland & Nuuk \\\hline
  Skåne & Lund \\\hline
  \end{tabular}
\end{table}

\Subtask Skriv ett uttryck som plockar fram \code{"Lund"} ur \code{huvudstad}.

\SOLUTION


\TaskSolved \what

\SubtaskSolved Ja, fabriksmetoden returnerar ett helt vanligt par:
\begin{REPLnonum}
scala> val härBorJag = "Skåne" -> "Lund"
val härBorJag: (String, String) = (Skåne,Lund)

scala> härBorJag._1
val res0: String = Skåne

scala> härBorJag._2
val res1: String = Lund
\end{REPLnonum}


\SubtaskSolved

\begin{Code}
val huvudstad = Vector(
  "Sverige"  -> "Stockholm",
  "Danmark"  -> "Köpenhamn",
  "Grönland" -> "Nuuk",
  "Skåne"    -> "Lund"
)
\end{Code}

\SubtaskSolved
\begin{REPL}
scala> huvudstad(3)._2
val res2: String = Lund
\end{REPL}

\QUESTEND



\WHAT{Linjärsöka efter nyckel i sekvens av mappningar.}

\QUESTBEGIN

\Task \what~

\Subtask Implementera funktionen \code{lookupIndex} nedan med hjälp av samlingsmetoden \code{indexWhere} så att linjärsökning sker efter index för ett par i sekvensen där \code{key} finns på första platsen i paret.

\begin{Code}
def lookupIndex(xs: Vector[(String, String)])(key: String): Int = ???
\end{Code}

\Subtask Testa din funktion i REPL genom att slå upp index för Skånes huvudstad i sekvensen \code{huvudstad} från föregående uppgift.

\SOLUTION

\TaskSolved \what~

\SubtaskSolved
\begin{Code}
def lookupIndex(xs: Vector[(String, String)])(key: String): Int =
  xs.indexWhere(_._1 == key)
\end{Code}

\SubtaskSolved
\begin{REPL}
scala> val i = lookupIndex(huvudstad)("Skåne")
val i: Int = 3

scala> huvudstad(i)._2
val res2: String = Lund
\end{REPL}

\noindent Eller med funktioner som återanvändbara dellösningar:
\begin{REPL}
scala> val indexOf = lookupIndex(huvudstad) _

scala> def capital(key: String) = huvudstad(indexOf(key))._2

scala> capital("Skåne")
val res3: String = Lund

scala> capital("Sverige")
val res4: String = Stockholm
\end{REPL}

\QUESTEND



\WHAT{Nyckel-värde-tabell.}

\QUESTBEGIN

\Task \what~En nyckel-värde-tabell är en smart datastruktur som gör att du kan slå upp det värde som en nyckel mappar till \emph{utan} att linjärsökning behöver ske. Värdet plockas fram direkt på en konstant tid, d.v.s. tiden att slå upp ett värde beror \emph{inte} på antalet element i samlingen, utan sker med mycket liten fördröjning.

I Scala heter nyckelvärdetabeller \code{Map} med stort M och är praktiska att använda i många olika sammanhang. \code{Map} finns i både en oföränderlig och en förändringsbar variant. Det går med metoder på formen \code{toXXX} lätt att omvandla mellan en \code{Map} och en sekvens av par av typen \code{XXX[(Nyckeltyp, Värdetyp)]}.

\Subtask Deklarera mappen \code{telnr} nedan i REPL och använd \code{apply} för att ta reda på telefonnumret till Fröken Ur.

\Subtask Vad har \code{telnr} för typ?

\Subtask Vad har \code{telnr.toVector} för typ?

\begin{Code}
val telnr = Map(
  "Anna"     -> 46462229812L,
  "Björn"     -> 46462229009L,
  "Sandra"    -> 46462220368L,
  "Fröken Ur" -> 4690510L,
)
\end{Code}
En uppsättning \code{Map}-instanser, vid behov nästlade, kan med fördel användas för att bygga upp en i-minnet-databas där inbyggda samlingsmetoder, t.ex. \code{map}, \code{filter}, och \code{for}-\code{yield}-uttryck, ger flexibla och effektiva sökmöjligheter. På veckans laboration ska du göra detta.

Samlingen \code{Map} är en generalisering av en sekvens, där man kan ''indexera'', inte bara med ett heltal, utan med vilken typ av värde som helst, t.ex. en sträng. Datastrukturen \code{Map} kallas också \emph{associativ array}\footnote{\href{https://en.wikipedia.org/wiki/Associative_array}{https://en.wikipedia.org/wiki/Associative\_array}} och är implementerad som en s.k. \emph{hashtabell}\footnote{\href{https://en.wikipedia.org/wiki/Hash_table}{https://en.wikipedia.org/wiki/Hash\_table}}, men du får vänta till fördjupningskursen innan vi går igenom hur en sådan datastruktur implementeras.

\SOLUTION

\TaskSolved \what~

\begin{REPL}
scala> telnr("Fröken Ur")
val res0: Long = 464690510

scala> :type telnr
Map[String,Long]

scala> :type telnr.toVector
Vector[(String, Long)]
\end{REPL}

\QUESTEND



\WHAT{Använda nyckel-värdetabell.}

\QUESTBEGIN

\Task \what~

\Subtask Skapa nedan variabler i REPL.
\begin{Code}
val follow = for i <- 2 to 16 by 2 yield (i, i + 1)
val xs = follow.toMap
val ys = xs.toVector
\end{Code}
Hamnar mappningarna i \code{ys} i samma ordning som \code{follow}? Varför?

\Subtask Med \code{xs} och \code{ys} deklarerade i REPL enligt ovan, para ihop yttryck till vänster med rätt resultat till höger. Om du är osäker på de sammansatta uttrycken, prova enklare uttryck i REPL och undersök värde och typ hos delresultat.

\begin{ConceptConnections}
  \code|xs(2) + xs(4)                 | & 1 & & A & \code|1                     | \\ 
  \code|ys(2) + ys(4)                 | & 2 & & B & \code|-9                    | \\ 
  \code|ys(0)                         | & 3 & & C & \code|8                     | \\ 
  \code|xs(0)                         | & 4 & & D & \code|7                     | \\ 
  \code|(xs + (0 -> 1)).apply(0)      | & 5 & & E & \code|NoSuchElementException| \\ 
  \code|xs.keySet.apply(2)            | & 6 & & F & \code|(10, 11)              | \\ 
  \code|xs.mapValues(v => -v).apply(8)| & 7 & & G & \code|false                 | \\ 
  \code|xs isDefinedAt 0              | & 8 & & H & \verb|error: type mismatch  | \\ 
  \code|xs.getOrElse(0, 7)            | & 9 & & I & \code|(16, 17)              | \\ 
  \code|xs.maxBy(_._2)                | & 10 & & J & \code|true                  | \\ 
\end{ConceptConnections}

\SOLUTION

\TaskSolved \what


\SubtaskSolved Nej nyckel-värde-paren lagras i någon speciell ordning som bestäms av en intern, smart lagringsprincip enligt en s.k. hashfunktion\footnote{\url{https://sv.wikipedia.org/wiki/Hashfunktion}}, för att åstadkomma snabba uppslagningar av värden från nycklar och vilket normalt inte sammanfaller med ordningen i den sekvens som de skapades ur.

\SubtaskSolved

\begin{ConceptConnections}
    \code|xs(2) + xs(4)                 | & 1 & ~~\Large$\leadsto$~~ &  A & \code|8                     | \\ 
  \code|ys(0)                         | & 2 & ~~\Large$\leadsto$~~ &  C & \code|(10, 11)              | \\ 
  \code|xs(0)                         | & 3 & ~~\Large$\leadsto$~~ &  I & \code|NoSuchElementException| \\ 
  \code|(xs + (0 -> 1)).apply(0)      | & 4 & ~~\Large$\leadsto$~~ &  D & \code|1                     | \\ 
  \code|xs.keySet.apply(2)            | & 5 & ~~\Large$\leadsto$~~ &  G & \code|true                  | \\ 
  \code|xs isDefinedAt 0              | & 6 & ~~\Large$\leadsto$~~ &  H & \code|false                 | \\ 
  \code|xs.getOrElse(0, 7)            | & 7 & ~~\Large$\leadsto$~~ &  B & \code|7                     | \\ 
  \code|xs.maxBy(_._2)                | & 8 & ~~\Large$\leadsto$~~ &  E & \code|(16, 17)              | \\ 
  \code|xs.map(p => p._1 -> -p._2)(8) | & 9 & ~~\Large$\leadsto$~~ &  F & \code|-9                    | \\ 
\end{ConceptConnections}

%%% BELOW IS SOLVED IN SCALA 3 AND the err msg is better! :)
% \noindent \emph{Fördjupning}:  Felmeddelandet som rad 2 ovan orsakar är lurigt:

% \begin{REPL}
% scala> ys(2)
% val res22: (Int, Int) = (6,7)

% scala> ys(4)
% val res23: (Int, Int) = (12,13)

% scala> ys(2) + ys(4)
% <console>:13: error: type mismatch;
%  found   : (Int, Int)
%  required: String
%        ys(2) + ys(4)

% \end{REPL}
% Det går som förväntat inte att addera två tupler, men varför säger kompilatorn att en sträng krävs?!? Detta beror på att, i enlighet med hur det fungerar i Java, valde Scala-språkets konstruktörer att låta strängsammanfogning fungera med alla möjliga typer vilket gör att kompilatorn inte ger upp när metoden \code{+} inte finns för tupler, utan i stället gör ett misslyckat försök med strängsammanfogning.

% Det mest olyckliga med detta är inte att felmeddelanden ibland blir missvisande, utan att det i vissa situationer inte ens \emph{blir} något felmeddelande, trots att man av rent misstag råkat strängkonkatenera i stället för t.ex. lägga till ett element i en mängd eller en mappning i en tabell. Detta typosäkra beteendet av strängsammanfogning har kritiserats, men det är inte okontroversiellt att ändra detta nu när så många utvecklare skrivit så mycket Scala-kod som bygger på strängars förmåga att kunna lägga till vad som helst på slutet. Situationen i Scala är dock inte hopplös efter introduktionen av stränginterpolering i Scala 2.10, som möjliggör infogande av värden i strängar på ett typsäkert sätt.
\QUESTEND





\WHAT{Registrering i förändringsbar nyckel-värde-tabell.}

\QUESTBEGIN

\Task \what~I denna uppgift ska du implementera en hjälpklass för registrering i en frekvenstabell som du sedan ska använda på veckans laboration. Klassen ska heta  \code{FreqMapBuilder} som efter upprepade anrop av metoden \code{add(s: String): Unit} kan skapa frekvenstabeller av typen \code{Map[String, Int]}, där nyckel-värde-paren i tabellen anger antalet förekomster av en viss sträng. Du ska utgå från koden nedan.

Klassen använder en förändringsbar tabell internt. Efter att man har lagt till många strängar kan man med metoden \code{toMap} få en oföränderlig tabell för  uppslagning av frekvenser för specifika strängar. Läs i snabbreferensen om vilka extra metoder för uppdatering som erbjuds av \code{mutable.Map[K, V]}.

\begin{Code}
class FreqMapBuilder:
  private val register = collection.mutable.Map.empty[String, Int]
  def toMap: Map[String, Int] = register.toMap
  def add(s: String): Unit = ???

object FreqMapBuilder:
  def apply(xs: String*): FreqMapBuilder = ???
\end{Code}

\noindent Implementera och testa \code{FreqMapBuilder}. \emph{Tips:} Du kan t.ex. använda metoderna \code{+=} och \code{getOrElse}.

\SOLUTION

\TaskSolved \what~
\begin{Code}
class FreqMapBuilder:
  private val register = scala.collection.mutable.Map.empty[String,Int]
  def toMap: Map[String, Int] = register.toMap
  def add(s: String): Unit =
    register += (s -> (register.getOrElse(s, 0) + 1))

object FreqMapBuilder:
  def apply(xs: String*): FreqMapBuilder = 
    val result = new FreqMapBuilder
    xs.foreach(result.add)
    result
\end{Code}

\QUESTEND



\WHAT{Metoden \code{sliding}.}

\QUESTBEGIN

\Task  \what~  I veckans laboration kommer du att ha nytta av metoden \code{sliding}, som ger en iterator för speciella delsekvenser av en sekvens, vilka kan liknas vid ''utsikten'' i ett ''glidande fönster''.

\Subtask Kör nedan i REPL och beskriv vad som händer.

\begin{REPL}
scala> val xs = Vector("fem", "gurkor", "är", "fler", "än", "fyra", "tomater")
scala> xs.sliding(2).toVector
scala> xs.sliding(3).toVector
scala> xs.sliding(10).toVector
\end{REPL}

\Subtask Använd \code{xs.sliding(2)} och omvandla varje element i resultatet till ett par. Gör sedan om sekvensen av par till en nyckel-värde-tabell. Vad kan tabellen användas till?

\SOLUTION

\TaskSolved \what

\SubtaskSolved
\begin{REPL}
scala> val xs = Vector("fem", "gurkor", "är", "fler", "än", "fyra", "tomater")
val xs: Vector[String] =
  Vector(fem, gurkor, är, fler, än, fyra, tomater)

scala> xs.sliding(2).toVector
val res9: Vector[Vector[String]] =
  Vector(Vector(fem, gurkor), Vector(gurkor, är), Vector(är, fler), Vector(fler, än), Vector(än, fyra), Vector(fyra, tomater))

scala> xs.sliding(3).toVector
val res10: Vector[Vector[String]] =
  Vector(Vector(fem, gurkor, är), Vector(gurkor, är, fler), Vector(är, fler, än), Vector(fler, än, fyra), Vector(än, fyra, tomater))

scala> xs.sliding(10).toVector
val res11: Vector[Vector[String]] =
  Vector(Vector(fem, gurkor, är, fler, än, fyra, tomater))

\end{REPL}
\code{xs.sliding(n).toVector} skapar en sekvens som innehåller sekvenser av längden \code{n} som bildas genom att ta varje element och dess \code{n - 1} efterföljande element.

\SubtaskSolved
\begin{REPL}
scala> xs.sliding(2).map(ys => ys(0) -> ys(1)).toMap
val res0: Map[String,String] =
  Map(är -> fler,
      än -> fyra,
      fyra -> tomater,
      gurkor -> är,
      fem -> gurkor,
      fler -> än
  )
\end{REPL}
Man kan använda tabellen till att slå upp vilket som är efterföljande ord. Det fungerar eftersom alla ord är unika. Om det funnits flera likadana ord med olika efterföljande ord så hade vi behövt skapa en tabell med nycklar som mappar till en samling som registrerar efterföljande ord. Detta ska vi göra på veckans laboration.

\QUESTEND




\WHAT{Läsa text från fil och webbservrar.}

\QUESTBEGIN

\Task \what~På laborationen ska du bygga upp tabeller från data i textformat. Då har du nytta av att kunna läsa text från filer och från webben. Testa detta i REPL:
\begin{REPL}
scala> val url = "https://fileadmin.cs.lth.se/pgk/europa.txt"
scala> val xs = io.Source.fromURL(url, "UTF-8").getLines.toVector
scala> val data = xs.map(_.split(';').toVector)
scala> data.head
scala> data.foreach(println)
\end{REPL}

\Subtask Skapa dessa tabeller ur sekvensen \code{data}:
\begin{Code}
val populationOf: Map[String, Int]    = ???  // länders invånarantal
val sizeOf:       Map[String, Int]    = ???  // länders yta i km^2
val capitalOf:    Map[String, String] = ???  // länders huvudstäder
\end{Code}
Testa tabellerna i REPL.

\Subtask Spara ner data i en textfil \code{europa.txt}. Läsa in data från filen med metoden \code{Source.fromFile(filnamn, teckenkodning)} på liknande sätt som med  \code{fromURL} ovan. Om du kör i en Linux-terminal kan du enkelt ladda ner en fil så här:
\begin{REPLnonum}
> wget https://fileadmin.cs.lth.se/pgk/europa.txt
\end{REPLnonum}
Skriv ut alla raderna i \code{europa.txt} med hjälp av \code{Source.fromFile} i REPL.

\SOLUTION

\TaskSolved \what~

\SubtaskSolved
\begin{CodeSmall}
val populationOf = data.tail.map(v => v(0) -> v(1).toInt).toMap
val sizeOf       = data.tail.map(v => v(0) -> v(2).toInt).toMap
val capitalOf    = data.tail.map(v => v(0) -> v(3)).toMap
\end{CodeSmall}

\begin{REPL}
scala> capitalOf("Sverige")
res2: String = Stockholm

scala> populationOf("Sverige")
res3: Int = 9223766

scala> sizeOf("Sverige")
res4: Int = 449964
\end{REPL}

\begin{REPL}
scala> val filename = "europa.txt"
scala> val xs = io.Source.fromFile(filename, "UTF-8").getLines.toVector
scala> val data = xs.map(_.split(';').toVector)
scala> data.map(_.map(_.take(15).padTo(15,' ')).mkString(" ")).foreach(println)
\end{REPL}
\QUESTEND





\ExtraTasks %%%%%%%%%%%%%%%%%%%%%%%%%%%%%%%%%%%%%%%%%%%%%%%%%%%%%%%%%%%%%%%%%%%%

\WHAT{Skapa ett textspel med hjälp av tabeller.}

\QUESTBEGIN

\Task \what~Gör ett enkelt spel för att träna på olika fakta om Europas länder och huvudstäder genom att läsa data från URL:en:\\ \url{https://fileadmin.cs.lth.se/pgk/europa.txt}
\\Där finns text kodad i UTF-8 med följande innehåll (endast de första raderna visas):
\begin{Code}
Land;Invånarantal;Storlek(km^2);Huvudstad
Albanien;3581655;28748;Tirana
Andorra;71201;468;Andorra la Vella
Belgien;10584534;30528;Bryssel
Bosnien-Hercegovina;4590310;51129;Sarajevo
Bulgarien;7385367;110910;Sofia
Cypern;854000;9250;Nicosia
Danmark;5475791;43094;Köpenhamn
Estland;1324333;45226;Tallinn
Finland;5315280;338145;Helsingfors
Frankrike;61538322;551695;Paris
Färöarna;48344;139574;Torshamn
Grekland;10964021;131940;Aten
// ... etcetera för alla Europas länder.
\end{Code}
Låt till exempel användaren svara på slumpvisa frågor av typen:
\begin{itemize}[noitemsep]
  \item Har Andorra fler invånare än Cypern?
  \item Vad heter huvudstaden i Bulgarien?
  \item Har Danmark större yta än Finland?
\end{itemize}
Använd oföränderliga tabeller med lämpliga nycklar och värden. Du kan använda en mängd med länder/huvudstäder som användaren hittills svarat rätt på för att kunna förhindra att dessa återkommer igen.
\SOLUTION

\TaskSolved --

\QUESTEND



\AdvancedTasks %%%%%%%%%%%%%%%%%%%%%%%%%%%%%%%%%%%%%%%%%%%%%%%%%%%%%%%%%%%%%%%%%


\WHAT{Registrering med \code{groupBy}.}

\QUESTBEGIN

\Task \what~Vi ska nu utnyttja ett riktigt listigt trick för att via en enda kodrad implementera registrering med hjälp av samlingsmetoderna \code{groupBy} och \code{map}.

\Subtask Läs om metoden \code{groupBy} i snabbreferensen. Du hittar den under rubriken \emph{''Methods in trait \code{Iterable[A]}''} eftersom \code{groupBy} fungerar på alla samlingar. Testa \code{groupBy} enligt nedan och beskriv vad som händer.

\begin{REPL}
scala> val xs = Vector(1, 1, 2, 2, 4, 4, 4).groupBy(x => x > 2)
scala> val ys = Vector(1, 1, 2, 2, 4, 4, 4).groupBy(x => x)
\end{REPL}

\Subtask Skapa en funktion \code{freq} med nedan funktionshuvud som returnerar en tabell med antalet förekomster av olika heltal i \code{xs}. Testa \code{freq} på en sekvens av 1000 slumpvisa tärningskast och förklara hur funktionen \code{freq} fungerar. \emph{Tips:} Gör först \code{groupBy(???)} och sedan \code{map(???)}.

\begin{Code}
def freq(xs: Vector[Int]): Map[Int, Int] = ???

def kasta(n: Int): Vector[Int] =
  Vector.fill(n)(scala.util.Random.nextInt(6) + 1)
\end{Code}

\SOLUTION

\TaskSolved \what~

\SubtaskSolved Metoden \code{groupBy} skapar en nyckel-värde-tabell där värdena i tabellen är en sekvens med elementen grupperade på ett speciellt sett.
Mer precist:

Resultatet av \code{xs.groupBy(f: K => V)} för en sekvens \code{xs} av typen \code{Vector[K]} blir en tabell av typen \code{Map[V,Vector[K]]} där varje element \code{e} i \code{xs} är grupperade i samma tabellvärde om de lika är enligt \code{f(e)}. Varje grupp får tabellnyckeln \code{f(e)}.

\emph{Listigt trick:} Om man låter funktionen \code{f} vara enhetsfunktionen som avbildar varje element på sig själv, alltså \code{x => x}, så grupperas värdena i samma sekvens om de är lika.

\begin{REPL}
scala> val xs = Vector(1, 1, 2, 2, 4, 4, 4).groupBy(x => x > 2)
val xs: Map[Boolean,Vector[Int]] =
  Map(false -> Vector(1, 1, 2, 2), true -> Vector(4, 4, 4))

scala> val ys = Vector(1, 1, 2, 2, 4, 4, 4).groupBy(x => x)
val ys: Map[Int,Vector[Int]] =
  Map(2 -> Vector(2, 2), 4 -> Vector(4, 4, 4), 1 -> Vector(1, 1))
\end{REPL}


\SubtaskSolved

\begin{Code}
def freq(xs: Vector[Int]): Map[Int, Int] =
  xs.groupBy(x => x).map(p => p._1 -> p._2.size)
\end{Code}
Förklaring: metoden \code{groupBy} skapar en tabell med par \code{k, v} där \code{v} är en sekvens med så många \code{k} som antalet gånger \code{k} förekommer i \code{xs}. Genom att omvandla alla värden \code{p._2} till storleken \code{p._2.size} får vi en frekvenstabell.

\begin{REPL}
scala> freq(kasta(1000))
val res0: Map[Int,Int] = 
  Map(5 -> 163, 1 -> 174, 6 -> 161, 2 -> 169, 3 -> 167, 4 -> 166)

scala> freq(kasta(1000)).toVector.sortBy(_._1).foreach(println)
(1,183)
(2,167)
(3,169)
(4,179)
(5,154)
(6,148)
\end{REPL}

\QUESTEND





\WHAT{Skriva till fil.}

\QUESTBEGIN

\Task \what~Som hjälp när du skapar egna intressanta applikationer eller bygger vidare på kursens laborationer och övningar med frivilliga extrauppgifter, kan du använda funktionerna i singelobjektet \code{IO} nedan, som finns i kursens scala-bibliotek \href{http://cs.lth.se/pgk/api}{introprog}.\footnote{Källkoden finns här och även på sidan \pageref{disk-access-code}:\\ \href{https://github.com/lunduniversity/introprog/blob/master/compendium/workspace/introprog/src/main/scala/introprog/IO.scala}{https://github.com/lunduniversity/introprog-scalalib/blob/master/src/main/scala/introprog/IO.scala}}

IO-modulen använder \code{scala.io.Source} för att serialisera och de-serialisera strängar till och från vanliga textfiler. IO-modulen använder även paketet \code{java.io} för att erbjuda funktioner som gör det enkelt att serialisera/de-serialisera godtyckliga objekt skapade med hjälp av serialserbara klasser till/från binärfiler. Case-klasser i Scala blir automatiskt serialiserbara.

I implementationen av \code{IO} används \code{try ... finally} för att säkerställa att filer inte lämnas öppnade även om något går fel under den läs/skriv-process som sköts av det underliggande operativsystemet.

\Subtask
Kompilera och resta nedan med \code{introprog} på classpath, t.ex. med hjälp av \code{sbt}.
\begin{Code}
import introprog.IO

case class Player(name: String)

@main def run(): Unit = 
  println("Test of output/input objects to/from disk:")
  val highscores = Map(Player("Sandra") -> 42, Player("Björn") -> 5)
  IO.saveObject(highscores,"highscores.ser")
  val highscores2 = IO.loadObject[Map[Player, Int]]("highscores.ser")
  val isSameContents = highscores2 == highscores
  val testResult = if (isSameContents) "SUCCESS :)" else "FAILURE :("
  println(testResult)
\end{Code}

\Subtask
Använd \code{IO}-modulen för att spara användarens poängresultat i ditt spel om Europas länder och städer, i extrauppgiften ovan. Implementationen av \code{introprog.IO} finns här: \url{https://github.com/lunduniversity/introprog-scalalib/blob/master/src/main/scala/introprog/IO.scala} 

% \begin{figure}
% %  \scalainputlisting[basicstyle=\ttfamily\fontsize{9.2}{11}\selectfont]{examples/IO.scala}
%   \scalainputlisting[basicstyle=\ttfamily\fontsize{9.2}{11}\selectfont]{../workspace/introprog/src/main/scala/introprog/IO.scala}
%   \label{disk-access-code}
% \end{figure}
\SOLUTION

\TaskSolved --

\QUESTEND



%
%
% \subsection{\TODO Värdera nedan gamla uppgifter}
%
%
%
% \WHAT{Objekt med attribut (fält).}
%
% \QUESTBEGIN
%
% \Task  \what~  Ett objekt kan samla data som hör ihop och på så sätt skapa en datastruktur. Data i ett objekt kallas \emph{attribut} eller \emph{fält}, \Eng{field}. Objekt som samlar enbart data kallas även \emph{post} \Eng{record}.
% \begin{REPLnonum}
% scala> object mittKonto { var saldo = 0; val nummer = 12345L }
% \end{REPLnonum}
% \Subtask Skriv en sats som sätter in ett slumpmässigt belopp mellan 0 och en miljon på \code{mittKonto} ovan med hjälp av punktnotation och tilldelning.
%
% \Subtask Vad händer om du försöker ändra attributet \code{nummer}?
%
% \SOLUTION
%
%
% \TaskSolved \what
%
%
% \SubtaskSolved   \code{mittKonto.saldo = (math.random() * 1000000).toInt}
%
% \SubtaskSolved   Går ej eftersom val är oföränderlig, man får alltså ett Error.
%
%
% \QUESTEND
%
%
%
%
% %%<AUTOEXTRACTED by mergesolu>%%      %Uppgift 2
%
%
%
%
% \WHAT{Klass med attribut.}
%
% \QUESTBEGIN
%
% \Task  \what~  Om du vill ha många objekt av samma typ, kan du använda en \textbf{klass}. På så sätt kan man skapa många datastrukturer av samma typ men med olika innehåll. Man skapar nya objekt med nyckelordet \code{new} följt av klassens namn. Klassen utgör en ''mall'' för objektet som skapas. Ett objekt som skapas med \code{new Klassnamn} kallas även en \textbf{instans} av klassen \code{Klassnamn}. Nedan skapas en datastruktur \code{Konto} som samlar data om ett bankonto. Instanser av typen \code{Konto} håller reda på hur mycket pengar det finns på kontot och vilket kontonumret är. Datavärden som sparas i varje objektinstans, så som \code{saldo} och \code{nummer}, kallas \textbf{attribut} \Eng{attribute} eller \textbf{fält} \Eng{field}.
%
% \begin{REPL}
% scala> class Konto {
%          var saldo = 0
%          var nummer = 0L
%        }
% scala> val k1 = new Konto
% scala> val k2 = new Konto
% scala> k1.saldo = 1000
% scala> k1.nummer = 12345L
% scala> k2.saldo = 2000
% scala> k2.nummer = 67890L
% scala> println("Konto: " + k1.nummer + " Saldo:" + k1.saldo)
% scala> println("Konto: " + k2.nummer + " Saldo:" + k2.saldo)
% \end{REPL}
%
% \Subtask\Pen Rita hur minnessituationen ser ut efter att ovan rader har exekverats.
%
% \Subtask\Pen Vad hade det fått för konsekvenser om attributet \code{nummer} vore oföränderligt i klassen ovan? (Jämför med objektet \code{mittKonto}.)
%
%
% \SOLUTION
%
%
% \TaskSolved \what
%
%
% \SubtaskSolved   \includegraphics[scale=0.5]{../img/w04-solutions/uppgift-3a}
%
% \SubtaskSolved
% Tilldelningen på rad 8 \code{k1.nummer = 12345L} ger felmeddelande eftersom variablen är oföränderlig.
%
%
% \QUESTEND
%
%
%
%
% %%<AUTOEXTRACTED by mergesolu>%%      %Uppgift 3
%
%
%
%
% \WHAT{Klass med attribut som parametrar.}
%
% \QUESTBEGIN
%
% \Task  \what~  Om man vill ge attributen initialvärden när objektet skapas med \code{new}, kan man placera attributen i en parameterlista till klassen. Koden som körs när objektet skapas och attributen tilldelas sina initialvärden, kallas \textbf{konstruktor} \Eng{constructor}.
%
% \begin{REPL}
% scala> class Konto(var saldo: Int, val nummer: Long)
% scala> val k = new Konto(0, 12345L)
% scala> println("Konto: " + k.nummer + " Saldo:" + k.saldo)
% scala> println(k)
% scala> k.toString
% \end{REPL}
%
% \Subtask Den två sista raderna ovan skriver ut den identifierare som JVM använder för att hålla reda på objektet i sina interna datastrukturer. Vad skrivs ut?
%
% \Subtask Skapa ännu en instans av klassen Konto  med samma saldo och nummer som \code{k} ovan och spara den i \code{val k2} och undersök dess objektidentifierare. Får objekten \code{k} och \code{k2} olika objektidentifierare?
%
% \Subtask Sätt in olika belopp på respektive konto.
%
% \Subtask Vad händer om du försöker ändra attributet \code{nummer}?
%
% \Subtask\Pen Ibland räcker det fint med en tupel, men ofta vill man ha en klass istället. Beskriv några fördelar med en Konto-klassen ovan jämfört med en tupel av typen \code{(Int, Long)}.
%
% \begin{REPLnonum}
% scala> var k3 = (0, 12345L)
% scala> k3 = (k3._1 + 100, k3._2)
% \end{REPLnonum}
%
% \SOLUTION
%
%
% \TaskSolved \what
%
%
% \SubtaskSolved   \code{String = Konto@cd576}, där \code{Konto@cd576} är ett unikt namn som identifierar instansen.
%
% \SubtaskSolved   Ja.
%
% \SubtaskSolved
% \begin{REPLnonum}
% scala> k.saldo = 42
% scala> k2.saldo = 67
% \end{REPLnonum}
%
% \SubtaskSolved   Eftersom variablen är oföränderlig ges ett felmeddelande.
%
% \SubtaskSolved   En fördel med klass är att man kan specificera att variablen ska kunna vara föränderlig. En till är att man kan inkludera metoder i klassen som man vill kunna använda på värdena.
%
%
% \QUESTEND
%
%
%
%
% %%<AUTOEXTRACTED by mergesolu>%%      %Uppgift 4
%
%
%
%
% \WHAT{Publikt eller privat attribut?}
%
% \QUESTBEGIN
%
% \Task  \what~  Man kan förhindra att ett attribut syns utanför klassen med hjälp av nyckelordet \code{private}.
%
% \begin{REPL}
% scala> class Konto1(val nummer: Long){ var saldo = 0 }
% scala> val k1 = new Konto1(12345678901L)
% scala> k1.nummer
% scala> k1.saldo += 1000
% scala> class Konto2(val nummer: Long){ private var saldo = 0 }
% scala> val k2 = new Konto2(12345678901L)
% scala> k2.nummer
% scala> k2.saldo += 1000
% \end{REPL}
%
% \Subtask Vad händer ovan?
%
% \Subtask Gör en ny version av klassen \code{Konto} enligt nedan:
%
% \begin{Code}
% class Konto(val nummer: Long){
%   private var saldo = 0
%   def in(belopp: Int): Unit = {saldo += belopp}
%   def ut(belopp: Int): Unit = {saldo -= belopp}
%   def show: Unit =
%     println("Konto Nr: " + nummer + " saldo: " + saldo)
% }
%
% object Main {
%   def main(args: Array[String]): Unit = {
%     val k = new Konto(1234L)
%     k.show
%     k.in(1000)
%     println("Uttag: " + k.ut(500))
%     println("Uttag: " + k.ut(1000))
%     k.show
%   }
% }
% \end{Code}
%
% \Subtask Spara koden i en fil, kompilera med \code{scalac} och kör. Testa även vad som händer om du försöker komma åt attributet \code{saldo} i main-metoden med t.ex. \code{println(k.saldo)} eller \code{k.saldo += 1000}.
%
% \Subtask Vi ska nu förhindra överuttag. Ändra i metoden \code{ut} så att den får signaturen \code{ut(belopp: Int): (Int, Int) = ???} och implementera \code{ut} så att den returnerar både beloppet man verkligen kan ta ut och kvarvarande saldo. Om man försöker ta ut mer än det finns på kontot så ska saldot bli 0 och man får bara ut det som finns kvar. Spara, kompilera, kör.
%
% \Subtask Förbättra metoderna \code{in} och \code{ut} så att man inte kan sätta in eller ta ut negativa belopp.
%
% \Subtask Vad är fördelen med att göra föränderliga attribut privata och bara påverka deras värden indirekt via metoder?
%
% \SOLUTION
%
%
% \TaskSolved \what
%
%
% \SubtaskSolved
% Det går bra att ändra på variablen saldo i instansen av Konto1 men inte av Konto2 där man får ett error på raden ''k2.saldo += 1000''
%
% \SubtaskSolved  -
%
% \SubtaskSolved
% ''println(k.saldo)'' och ''k.saldo += 1000'' ger båda error, pga privat attribut.
%
% \SubtaskSolved
% \begin{Code}
% def ut(belopp: Int): (Int, Int) = {
% 	if(saldo >= belopp) {
% 		saldo -= belopp
% 		(belopp, saldo)
% 	} else {
% 		val temp = saldo
% 		saldo = 0
% 		(temp, 0)
% 	}
% }
% \end{Code}
%
% \SubtaskSolved
% Lägg till en if-sats i båda funktionerna som omsluter den gamla koden.
% \begin{Code}
% def ut(belopp: Int): (Int, Int) = {
%   if(belopp >= 0) {
%     if(saldo >= belopp) {
%       saldo -= belopp
%       (belopp, saldo)
%     } else {
%       val temp = saldo
%       saldo = 0
%       (temp, 0)
%     }
%   }
% }
%
% def in(belopp: Int): Unit = {
%   if(belopp >= 0) {
%     saldo += belopp
%   }
% }
% \end{Code}
%
% \SubtaskSolved
% Genom att göra attributet privat och gör egna metoder kan man se till att attriuten endast ändras på säkra sätt. Så inte fel uppstår.
%
%
% \QUESTEND
%
%
%
%
% %%<AUTOEXTRACTED by mergesolu>%%      %Uppgift 5
%
%
%
%
% \WHAT{Vilken typ har ett objekt?}
%
% \QUESTBEGIN
%
% \Task  \what~  Objektets typ bestäms av klassen. Vid tilldelning måste typerna passa ihop.
%
% \Subtask Vilka rader nedan ger felmeddelande? Hur lyder felmeddelandet?
% \begin{REPL}
% scala> class Punkt(val x: Double, val y: Double)
% scala> val pt: Punkt = new Punkt(10.0, 10.0)
% scala> val i: Int = pt.x
% scala> val (x: Double, y: Double) = (pt.x, pt.y)
% scala> val p: Double = new Punkt(5.0, 5.0)
% scala> val p = new Punkt(5.0, 5.0): Double
% scala> val p = new Punkt(5.0, 5.0): Punkt
% scala> pt: Punkt
% \end{REPL}
%
%
% \Subtask Man kan undersöka om ett objekt är av en viss typ med metoden \\ \code{isInstanceOf[Typnamn]}. Vad ger nedan anrop av metoden \code{isInstanceOf} för värde?
% \begin{REPL}
% scala> class Punkt(val x: Double, val y: Double)
% scala> val pt: Punkt = new Punkt(1.0, 2.0)
% scala> pt.isInstanceOf[Punkt]
% scala> pt.isInstanceOf[Double]
% scala> pt.x.isInstanceOf[Punkt]
% scala> pt.x.isInstanceOf[Double]
% scala> pt.x.isInstanceOf[Int]
% \end{REPL}
%
% \SOLUTION
%
%
% \TaskSolved \what
%
%
% \SubtaskSolved
% ''val i: Int = pt.x'' error: type mismatch;
% Eftersom typen Int ej är kompatibel med ett värde av typen Double.
%
% ''val p: Double = new Punkt(5.0, 5.0)'' error: type mismatch;
% Eftersom typen Double ej är kompatibel med ett värde av typen Punkt.
%
% ''val p = new Punkt(5.0, 5.0): Double'' error: type mismatch;
% Eftersom typen Double ej är kompatibel med ett värde av typen Punkt.
%
% \SubtaskSolved
% Rad 3 till 7 i respektive ordning: true, false, false, true och false.
%
%
% \QUESTEND
%
%
%
%
% %%<AUTOEXTRACTED by mergesolu>%%      %Uppgift 6
%
%
%
%
% \WHAT{Topptypen \code{Any}.}
%
% \QUESTBEGIN
%
% \Task  \what~ Alla klasser är också av typen \code{Any}. Alla klasser får därmed med sig några gemensamma metoder som finns i den fördefinierade klassen \code{Any}, däribland metoderna  \code{isInstanceOf} och \code{toString}.  Vad blir resultatet av respektive rad nedan? Vilken rad ger ett felmeddelande?
%
%
% \begin{REPL}
% scala> class Punkt(val x: Double, val y: Double)
% scala> val pt: Punkt = new Punkt(1.0, 2.0)
% scala> pt.isInstanceOf[Punkt]
% scala> pt.isInstanceOf[Any]
% scala> pt.x.toString
% scala> println(pt.x)
% scala> val a: Any = pt
% scala> println(a.x)
% scala> a.toString
% scala> pt.y.toString
% scala> a.y.toString
% \end{REPL}
%
% \SOLUTION
%
%
% \TaskSolved \what
%
% \begin{enumerate}
% \item Definierar klassen Punkt.
% \item En variabel pt: Punkt skapas.
% \item true
% \item true
% \item String = 1.0
% \item skriver ut: 1.0
% \item En variabel med namnet a skapas med typen Any.
% \item error: value x is not a member of Any
% \item a ges nu typen String
% \item String = 2.0
% \item error: value y is not a member of Any
% \end{enumerate}
%
%
% \QUESTEND
%
%
%
%
% %%<AUTOEXTRACTED by mergesolu>%%      %Uppgift 7
%
%
%
%
% \WHAT{Byta ut metoden \code{toString}}.
%
% \QUESTBEGIN
%
% \Task  \what~ I klassen \code{Any} finns metoden \code{toString} som skapar en strängrepresentation av objektet. Du kan byta ut metoden \code{toString} i klassen \code{Any} mot din egen implementation. Man använder nyckelordet \code{override} när man vill byta ut en metodimplementation.
%
% \begin{REPL}
% scala> class Punkt(val x: Double, val y: Double) {
%          override def toString: String = "[x=" + x + ",y=" + y + "]"
%        }
% scala> val pt = new Punkt(1.0, 42.0)
% scala> pt.toString
% scala> println(pt)
% \end{REPL}
%
% \Subtask Vad händer egentligen på sista raden ovan?
%
% \Subtask Omdefiniera toString så att den ger en sträng på formen \code{Punkt(1.0, 42.0)}.
%
% \Subtask Vad händer om du utelämnar nyckelordet \code{override} vid omdefiniering?
%
% \SOLUTION
%
%
% \TaskSolved \what
%
%
% \SubtaskSolved
% ''println(pt)'' kallar på pt.toString, och eftersom metoden är överskriven kallas den nya version.
%
% \SubtaskSolved   \code{override def toString: String = ''Punkt('' + x + '', '' + y + '').''}
%
% \SubtaskSolved
% error: overriding method toString in class Object of type ()String;
%
%
% \QUESTEND
%
%
%
%
% %%<AUTOEXTRACTED by mergesolu>%%      %Uppgift 8
%
%
%
%
% \WHAT{Objektfabrik med \code{apply}-metod.}
%
% \QUESTBEGIN
%
% \Task  \what~  Man kan ordna så att man slipper skriva \code{new} med ett s.k. \emph{fabriksobjekt} \Eng{factory object}.
% \begin{Code}
% class Pt(val x: Double, y: Double) {
%   override def toString: String = "Pt(x=" + x + ",y=" + y + ")"
% }
% object Pt {
%   def apply(x: Double, y: Double): Pt = new Pt(x, y)
% }
% \end{Code}
%
% \Subtask Skriv satser som använder metoden \code{apply} i fabriksobjektet \code{object Pt} för att skapa flera olika punkter.
%
% \Subtask Ge applymetoden default-argument 0.0 för både x och y så att \code{Pt()} skapar en punkt i origo.
%
% \Subtask Skapa en klass \code{Rational} som representerar rationellt tal som en kvot mellan två heltal. Ge klassen två oföränderliga, publika klassparameterattribut med namnen \code{nom} för täljaren och \code{denom} för nämnaren.
%
% \Subtask Skapa ett fabriksobjekt med en \code{apply}-metod som tar två heltalsparametrar och skapar en instans av klassen \code{Rational}.
%
% \Subtask Skapa olika instanser av din klass \code{Rational} ovan med hjälp av fabriksobjektet.
%
%
% \SOLUTION
%
%
% \TaskSolved \what
%
%
% \SubtaskSolved
% \begin{REPL}
% scala> val pt = Pt(1.0, 2.0)
% pt: Pt = Pt(x=1.0,y=2.0)
%
% scala> Pt(4.0, 2.0)
% res0: Pt = Pt(x=4.0,y=2.0)
%
% scala> Pt(6.0, 3.0)
% res1: Pt = Pt(x=6.0,y=3.0)
%
% scala> Pt(666.0, 1337.0)
% res2: Pt = Pt(x=666.0,y=1337.0)
% \end{REPL}
%
% \SubtaskSolved  \code{def apply(): Pt = new Pt(0, 0)}
%
% \SubtaskSolved  \code{class Rational(val nom: Int, val denom: Int)}
%
% \SubtaskSolved
% \begin{REPLnonum}
% object Rational {
% def apply(nom: Int, denom: Int): Rational = new Rational(nom, denom)
% }
% \end{REPLnonum}
%
% \SubtaskSolved
% \begin{REPL}
% scala> Rational(2, 5)
% scala> Rational(2, 7)
% scala> Rational(7, 4)
% scala> Rational(666, 1337)
% \end{REPL}
%
%
% \QUESTEND
%
%
%
%
% %%<AUTOEXTRACTED by mergesolu>%%      %Uppgift 9
%
%
%
%
% \WHAT{Skapa en case-klass.}
%
% \QUESTBEGIN
%
% \Task  \what~  Med en case-klass får man \code{toString} och fabriksobjekt på köpet. Man behöver inte skriva \code{val} framför klassparametrar i case-klasser; klassparametrar blir publika, oföränderliga attribut automatiskt när man deklarerar en case-klass.
%
% \begin{REPL}
% scala> case class Pt(x: Double, y: Double)
% scala> val p = Pt(1.0, 42.0)
% scala> p.toString
% scala> println(p)
% scala> println(Pt(5,6))
% \end{REPL}
%
% \Subtask Implementera din klass \code{Rational} från föregående uppgift, men nu som en case-klass.
%
% \SOLUTION
%
%
% \TaskSolved \what
%
% \SubtaskSolved  \code{case class Rational(nom: Int, denom: Int)}
%
%
% \QUESTEND
%
%
%
%
% %%<AUTOEXTRACTED by mergesolu>%%      %Uppgift 10
%
%
%
%
% \WHAT{Metoder på datastrukturer.}
%
% \QUESTBEGIN
%
% \Task \label{task:point} \what~   En datastruktur blir mer användbar om det finns metoder som kan användas på datastrukturen. Metoder i Scala kan även ha (vissa) specialtecken som namn, t.ex. \code{+} enligt nedan.
% \begin{REPL}
% scala> case class Point(x: Double, y: Double) {
%          def distToOrigin: Double = math.hypot(x, y)
%          def add(p: Point): Point = Point(x + p.x, y + p.y)
%          def +(p: Point): Point = add(p)
%        }
% \end{REPL}
%
% \Subtask Använd metoden \code{distToOrigin} för att ta reda på vad punkten med koordinaterna (3, 4) har för avstånd till origo?
%
% \Subtask Skriv satser som skapar två punkter (3,4) och (5, 6) och låt variablerna p1 och p2 referera till respektive punkt. Låt variabeln p3 bli summan av p1 och p2 med hjälp av metoden \code{add}. Vad får uttrycken \code{p3.x} resp. \code{p3.y} för värden?
%
%
%
% \SOLUTION
%
%
% \TaskSolved \what
%
%
% \SubtaskSolved
% \begin{REPLnonum}
% scala> Point(3, 4).distToOrigin
% res0: Double = 5.0
% \end{REPLnonum}
%
% \SubtaskSolved
% p3.x = 8
% p3.y = 10
%
%
% \QUESTEND
%
%
%
%
% %%<AUTOEXTRACTED by mergesolu>%%      %Uppgift 11
%
%
%
%
% \WHAT{Operatornotation.}
%
% \QUESTBEGIN
%
% \Task  \what~  Vid punktnotation på formen: \\ \code{objekt.metod(argument)} \\ kan man skippa punkten och parenteserna och skriva:\\ \code{objekt metod argument}  \\
% Detta förenklade skrivsätt kallas \textbf{operatornotation}.
%
% \Subtask Använd klassen \code{Point} från uppgift \ref{task:point} och prova nedan satser. Vilka rader använder operatortnotation och vilka rader använder punktnotation? Vilka rader ger felmeddelande?
% \begin{REPL}
% scala> val p1 = Point(3,4)
% scala> val p2 = Point(3,4)
% scala> p1.add(p2)
% scala> p1 add p2
% scala> p1.+(p2)
% scala> p1 + p2
% scala> 42 + 1
% scala> 42.+(1)
% scala> 42.+ 1
% scala> 42 +(1)
% scala> 1.to(42)
% scala> 1 to 42
% scala> 1.to(42)
% \end{REPL}
%
% \Subtask Implementera metoderna \code{sub} och \code{-} i klassen \code{Point} och skriv uttryck som kombinerar add och sub, samt + och - i både punktnotation och operatornotation.
%
% \Subtask Operatornotation fungerar även med flera argument. Man använder då parenteser om listan med argumenten:
% \code{ objekt metod (arg1, arg2)}  \\
% Definiera en metod \\
% \code{def scale(a: Double, b: Double) = Point(x * a, y * b)} \\
% i klassen \code{Point} och skriv satser som använder metoden med punktnotation och operatornotation.
%
%
%
%
%
% \SOLUTION
%
%
% \TaskSolved \what
%
%
% \SubtaskSolved
% \\Operatornotation:	4, 6, 10, 12
% \\Punktnotation:		3, 5, 8, 9, 11, 13
% \\Felmeddelande:		9
%
% \SubtaskSolved
% \begin{Code}
% case class Point(x: Double, y: Double) {
%   def distToOrigin: Double = math.hypot(x, y)
%   def add(p: Point): Point = Point(x + p.x, y + p.y)
%   def +(p: Point): Point = add(p)
%   def sub(p: Point): Point = Point(x - p.x, y - p.y)
%   def -(p: Point): Point = sub(p)
% }
% \end{Code}
% \begin{REPL}
% scala> val p1: Point = Point(1, 9)
% scala> val p2: Point = Point(9, 6)
% scala> p1.sub(p2)
% scala> p1.-(p2)
% scala> p2 sub p1
% scala> p2 - p2
% scala> p1.add(p2.sub(p1))
% scala> p1 + (p2 - p1)
% \end{REPL}
%
% \SubtaskSolved
% \begin{Code}
% case class Point(x: Double, y: Double) {
%   def distToOrigin: Double = math.hypot(x, y)
%   def add(p: Point): Point = Point(x + p.x, y + p.y)
%   def +(p: Point): Point = add(p)
%   def sub(p: Point): Point = Point(x - p.x, y - p.y)
%   def -(p: Point): Point = sub(p)
%   def scale(a: Double, b: Double) = Point(x * a, y * b)
% }
% \end{Code}
% \begin{REPL}
% scala> val p: Point(13,  37)
% scala> p.scale(4, 2)
% scala> p scale (3, 7)
% \end{REPL}
%
%
% \QUESTEND
%
%
%
%
% %%<AUTOEXTRACTED by mergesolu>%%      %Uppgift 12
%
%
%
%
% \WHAT{Föränderlighet och oföränderlighet.}
%
% \QUESTBEGIN
%
% \Task  \what~  Oföränderliga och föränderliga objekt beter sig olika vid tilldelning.
%
% \Subtask\Pen Innan du kör nedan kod: Försök lista ut vad som kommer att skrivas ut. Rita minnessituationen efter varje tilldelning.
%
% \begin{Code}
% println("\n--- Example 1: mutable value assigmnent")
% var x1 = 42
% var y1 = x1
% x1 = x1 + 42
% println(x1)
% println(y1)
% \end{Code}
%
% \Subtask\Pen Innan du kör nedan kod: Försök lista ut vad som kommer att skrivas ut. Rita minnessituationen efter varje tilldelning.
%
% \begin{Code}
% println("\n--- Example 2: mutable object reference assignment")
% class MutableInt(private var i: Int) {
%   def +(a: Int): MutableInt = { i = i + a; this }
%   override def toString: String = i.toString
% }
% var x2 = new MutableInt(42)
% var y2 = x2
% x2 = x2 + 42
% println(x2)
% println(y2)
% \end{Code}
%
% \Subtask\Pen Innan du kör nedan kod: Försök lista ut vad som kommer att skrivas ut. Rita minnessituationen efter varje tilldelning.
%
% \begin{Code}
% println("\n--- Example 3: immutable object reference assignment")
% class ImmutableInt(val i: Int) {
%   def +(a: Int): ImmutableInt = new ImmutableInt(i + a)
%   override def toString: String = i.toString
% }
% var x3 = new ImmutableInt(42)
% var y3 = x3
% x3 = x3 + 42
% println(x3)
% println(y3)
% \end{Code}
%
% \Subtask\Pen Vad finns det för fördelar med oföränderliga datastrukturer?
%
%
% \SOLUTION
%
%
% \TaskSolved \what
%
%
% \SubtaskSolved   \includegraphics[scale=0.5]{../img/w04-solutions/uppgift-13a}
%
% \SubtaskSolved
% \begin{enumerate}
% \item \includegraphics[scale=0.5]{../img/w04-solutions/uppgift-13b-1}
% \item \includegraphics[scale=0.5]{../img/w04-solutions/uppgift-13b-2}
% \item \includegraphics[scale=0.5]{../img/w04-solutions/uppgift-13b-3}
% \end{enumerate}
%
% \SubtaskSolved
% \begin{enumerate}
% \item \includegraphics[scale=0.5]{../img/w04-solutions/uppgift-13c-1}
% \item \includegraphics[scale=0.5]{../img/w04-solutions/uppgift-13c-2}
% \item \includegraphics[scale=0.5]{../img/w04-solutions/uppgift-13c-3}
% \end{enumerate}
%
% \SubtaskSolved   En stor fördel är att vi till exempel kan skicka med en immutable som argument till en metod och vara säkra på att metoden inte ändrar på värdet.
%
%
% \QUESTEND
%
%
%
%
% %%<AUTOEXTRACTED by mergesolu>%%      %Uppgift 13
%
%
%
%
% \WHAT{Några användbara samlingar.}
%
% \QUESTBEGIN
%
% \Task  \what~  En \textbf{samling} \Eng{collection} är en datastruktur som samlar många objekt av samma typ. I \code{scala.collection} och \code{java.util} finns många olika samlingar med en uppsjö användbara metoder. De olika samlingarna i \code{scala.collection} är ordnade i en gemensam hierarki med många gemensamma metoder; därför har man nytta av det man lär sig om metoderna i en Scala-samling när man använder en annan samling. Vi har redan tidigare sett samlingen \code{Vector}:
%
% \begin{REPL}
% scala> val tärningskast = Vector.fill(10000)((math.random() * 6 + 1).toInt)
% scala> tä   // tryck TAB
% scala> tärningskast.  // tryck TAB
% \end{REPL}
%
% \Subtask Ungefär hur många metoder finns det som man kan göra på objekt av typen \code{Vector}? Det är svårt att lära sig alla dessa på en gång, så vi väljer ut några få i kommande uppgifter.
%
% \Subtask Jämför överlappet mellan metoderna i \code{Vector} och \code{List} och uppskatta hur stor andel av metoderna som är gemensamma:
% \begin{REPL}
% scala> val myntkast =
%          List.fill(10000)(if (math.random() < 0.5) "krona" else "klave")
% scala> my   // tryck TAB
% scala> myntkast.  // tryck TAB
% \end{REPL}
%
% \SOLUTION
%
%
% \TaskSolved \what
%
%
% \SubtaskSolved   Ungefär 150 metoder.
%
% \SubtaskSolved   Ungefär lika många.
%
%
% \QUESTEND
%
%
%
%
% %%<AUTOEXTRACTED by mergesolu>%%      %Uppgift 14
%
%
%
%
% \WHAT{Typparameter.}
%
% \QUESTBEGIN
%
% \Task  \what~  Vissa funktioner är generella för många typer och tar en så kallad \textbf{typparameter} inom hakparenteser. Ofta slipper man skriva typparametrar, då kompilatorn kan härleda typen utifrån argumenten. Om man anger typparametrar explicit så hjälper kompilatorn dig med att kolla att det verkligen är rätt typ i samlingen.
%
% \Subtask Vad händer nedan?
% \begin{REPL}
% scala> var xs = Vector.empty[Int]
% scala> xs = xs :+ "42"
% scala> xs = xs :+ 43 :+ 64 :+ 46
% scala> xs
% scala> xs :+= "42".toInt
% scala> var ys = Vector[Int]("ett", "två", "tre")
% scala> var ingenting = Vector.empty
% scala> ingenting = Vector(1,2,3)
% \end{REPL}
%
% \Subtask Samlingar är mer användbara om de är \emph{generiska}, vilket innebär att elementens typ avgörs av en typparameter och därför kan vara av vilken typ som helst. Man kan definiera egna funktioner som tar generiska samlingar som parametrar. Förklara vad som händer här:
% \begin{REPL}
% scala> val vego = Vector("gurka", "tomat", "apelsin", "banan")
% scala> val prim = Vector(2, 3, 5, 7, 11, 13)
% scala> def först[T](xs: Vector[T]): T = xs.head
% scala> def sist[T](xs: Vector[T]) = xs.last
% scala> def förstOchSist[T](xs: Vector[T]): (T, T) = (xs.head, xs.last)
% scala> först(vego)
% scala> sist(prim)
% scala> förstOchSist(vego)
% scala> förstOchSist(prim)
% scala> def wrap[T](pair: (T, T))(xs: Vector[T]) = pair._1 +: xs :+ pair._2
% scala> wrap("Odla", "och ät!")(vego)
% scala> wrap("Odla", "och ät!")(vego).mkString(" ")
% \end{REPL}
%
%
%
%
%
% \SOLUTION
%
%
% \TaskSolved \what
%
%
% \SubtaskSolved
% \\1. Instansierar en tom vektor med element av typen int och tilldelar värdet till en variabel xs.
% \\2. Error eftersom \code{xs :+ ''42''} ger en Vector[Any] när Vector[Int] krävs.
% \\3. xs tilldelas ett nytt värde av Vector(43, 64, 46)
% \\4. xs skrivs ut.
% \\5. Lägger till talet 42 i xs.
% \\6. Error: type mismatch
% \\7. Skapar en tom Vector i variablen ingenting
% \\8. error: type mismatch; found: Int(3), required: Nothing
%
% \SubtaskSolved
% Tre metoder skapas: den första för att få första elementet i en lista, och eftersom den definieras med specialtypen T går den att använda med alla vektorer oavsett typen av variabeln i vektorn. Den andra får fram sista elementet och den sista hämtar båda två.
%
% En till function definieras längre ner med  namnet ''wrap'', som tar en lista och lägger till ett element längst fram och ett längst bak.
%
%
% \QUESTEND
%
%
%
%
% %%<AUTOEXTRACTED by mergesolu>%%      %Uppgift 15
%
%
%
%
% \WHAT{Några viktiga samlingsmetoder.}
%
% \QUESTBEGIN
%
% \Task  \what~  Deklarera följande vektorer i REPL.
% \begin{REPL}
% scala> val xs = (1 to 10).toVector
% scala> val a = Vector("abra", "ka", "dabra")
% scala> val b = Vector( "sim", "sala", "bim", "sala", "bim")
% scala> val stor = Vector.fill(100000)(math.random())
% \end{REPL}
% Undersök i REPL vad som händer nedan. Alla dessa metoder fungerar på alla samlingar som är indexerbara sekvenser. Givet deklarationerna ovan: vad har uttrycken nedan för värde och typ? Förklara vad som händer hälp av denna  översikt: \href{http://docs.scala-lang.org/overviews/collections/seqs}{docs.scala-lang.org/overviews/collections/seqs}
%
% \Subtask \code{a(1) + xs(1)}
%
% \Subtask \code{a apply 0}
%
% \Subtask \code{a.isDefinedAt(3)}
%
% \Subtask \code{a.isDefinedAt(100)}
%
% \Subtask \code{stor.length}
%
% \Subtask \code{stor.size}
%
% \Subtask \code{stor.min}
%
% \Subtask \code{stor.max}
%
% \Subtask \code{a indexOf "ka"}
%
% \Subtask \code{b.lastIndexOf("sala")}
%
% \Subtask \code{"först" +: b   //minnesregel: colon on the collection side}
%
% \Subtask \code{a :+ "sist"    //minnesregel: colon on the collection side}
%
% \Subtask \code{xs.updated(2,42)}
%
% \Subtask \code{a.padTo(10, "!")}
%
% \Subtask \code{b.sorted}
%
% \Subtask \code{b.reverse}
%
% \Subtask \code{a.startsWith(Vector("abra", "ka"))}
%
% \Subtask \code{"hejsan".endsWith("san")}
%
% \Subtask \code{b.distinct}
%
%
%
% \SOLUTION
%
%
% \TaskSolved \what
%
%
% \SubtaskSolved   String = ''ka2''
%
% \SubtaskSolved   String = ''abra''
%
% \SubtaskSolved   false
%
% \SubtaskSolved   false
%
% \SubtaskSolved   100000
%
% \SubtaskSolved   100000
%
% \SubtaskSolved   minsta talet i listan
%
% \SubtaskSolved   största talet i listan
%
% \SubtaskSolved   1
%
% \SubtaskSolved   3
%
% \SubtaskSolved   Vektor b fast med ''först'' som första element
%
% \SubtaskSolved   Vektor a fast med ''sist'' som sista element.
%
% \SubtaskSolved   plats 3 i vektorn xs får värdet 42
%
% \SubtaskSolved   En ny vektor fylld med ''!'' från och med plats 4 till 10. Men de andra värdena samma som i a.
%
% \SubtaskSolved   b sorterad i bokstavsordning
%
% \SubtaskSolved   b baklänges
%
% \SubtaskSolved   true
%
% \SubtaskSolved   true
%
% \SubtaskSolved   en vektor med alla unika element i b.
%
%
% \QUESTEND
%
%
%
%
% %%<AUTOEXTRACTED by mergesolu>%%      %Uppgift 16
%
%
%
%
% \WHAT{Några generella samlingsmetoder.}
%
% \QUESTBEGIN
%
% \Task  \what~  Det finns metoder som går att köra på \emph{alla} samlingar även om de inte är indexerbara. Givet deklarationerna i föregående uppgift: vad har uttrycken nedan för värde och typ? Förklara vad som händer med hjälp av dessa översikter: \\ \href{http://docs.scala-lang.org/overviews/collections/trait-traversable}{docs.scala-lang.org/overviews/collections/trait-traversable} \\ \href{http://docs.scala-lang.org/overviews/collections/trait-iterable}{docs.scala-lang.org/overviews/collections/trait-iterable}
%
% \Subtask \code{a ++ b}
%
% \Subtask \code{a ++ stor}
%
% \Subtask \code{val ys = xs.map(_ * 5)}
%
% \Subtask \code{b.toSet     // En mängd har inga dubletter}
%
% \Subtask \code{a.head + b.last}
%
% \Subtask \code{a.tail}
%
% \Subtask \code{a.head +: a.tail == a}
%
% \Subtask \code{Vector(a.head) ++ Vector(b.last)}
%
% \Subtask \code{a.take(1) ++ b.takeRight(1)}
%
% \Subtask \code{a.drop(2) ++ b.drop(1).dropRight(2)}
%
% \Subtask \code{a.drop(100)}
%
% \Subtask \code{val e = Vector.empty[String]; e.take(100)}
%
% \Subtask \code{Vector(e.isEmpty, e.nonEmpty)}
%
% \Subtask \code{a.contains("ka")}
%
% \Subtask \code{"ka" contains "a"}
%
% \Subtask \code{a.filter(s => s.contains("k")) }
%
% \Subtask \code{a.filter(_.contains("k")) }
%
% \Subtask \code{a.map(_.toUpperCase).filterNot(_.contains("K")) }
%
% \Subtask \code{xs.filter(x => x % 2 == 0)}
%
% \Subtask \code{xs.filter(_ % 2 == 0)}
%
%
% \SOLUTION
%
%
% \TaskSolved \what
%
%
% \SubtaskSolved
% Metoden ger tillbaka en ny Vector[String] som nu består av alla element i a plus alla element i b. I samma ordning med elementen i a först.
%
% \SubtaskSolved
% Samma som i uppgift a fast vektorn som returnas är av typen Vector[Any]. Det är eftersom Any är den närmsta typen som String och Double delar. Elementen från vektor a är fortfarande först och uppföljt av elementen i stor.
%
% \SubtaskSolved
% Variablen ys får värdet av en Vector[Int] som innehåller alla talen från xs fast multiplicerade med 5. Alltså ys = 5, 10, 15..., osv.
%
% \SubtaskSolved
% Functionen tar alla värden från en Vektor och sätter in i ett Set (mängd). Eftersom en mängd ej har dubletter så försvinner ett ''sala'' och ett ''bim'', Vector[String] som returneras blir därför (''sim'', ''sala'', ''bim'').
%
% \SubtaskSolved
% Metoden head ger första elementet i en samling, och last sista. Därför blir kombinationen av a.head och b.last en ny Vector[String] som består av a:s första element, och b:s första element.
%
% \SubtaskSolved
% Ger en Vector[String] som innehåller alla element efter det första. Alltså i detta fallet ''ka'' och ''dabra''.
%
% \SubtaskSolved
% True, eftersom head ger första elementet och tail ger resten, sedan sätter metoden +: ihop dem till en vektor med samma värden som a.
%
% \SubtaskSolved
% Eftersom ++ sätter ihop alla värden från två vektorer måste vi först omvandla från en sträng till vektor. Resultatet blir en ny vektor av samma typ som innan med a:s första element och b:S sista.
%
% \SubtaskSolved
% Samma resultat som i h, metoden take börjar från vänster och tar så många element som man skickar med som parameter och gör till en ny lista. Med 1 som parameter motsvarar det att göra Vector(a.head). Metoden takeRight gör samma sak fast från höger.
%
% \SubtaskSolved
% Metoden drop är motsvarigheten till take fast exkluderar de specifierade elementen istället för att inkludera dem i vektorn.
%
% \SubtaskSolved
% Eftersom a endast innehåller 3 element returnerar drop(100) en tom vektor.
%
% \SubtaskSolved
% Returnerar en tom vektor med element typen String
%
% \SubtaskSolved
% returnerar Vector(true, false)
%
% \SubtaskSolved
% True, metoden contains kollar om en samling innehåller ett specifikt element.
%
% \SubtaskSolved
% True. Eftersom en sträng även kan ses som Vector[Char].
%
% \SubtaskSolved
% Filtrerar vektorn a till att endast innehålla strängar som innehåller k.
%
% \SubtaskSolved
% Exakt samma som i p
%
% \SubtaskSolved
% map(\_.toUpperCase) omvandlar alla strängar i a till stora bokstäver
% filterNot(\_.contains(''K'')) tar resultatet vi precis fick och tar bort alla strängar som innehåller stora K.
%
% \SubtaskSolved
% filtrerar så att endast jämna tal finns kvar.
%
% \SubtaskSolved
% Exakt samma som i s
%
%
%
%
% \QUESTEND
%
%
%
%
% %%<AUTOEXTRACTED by mergesolu>%%      %Uppgift 17
%
%
%
%
% \WHAT{NEEDS A TOPIC DESCRIPTION}
%
% \QUESTBEGIN
%
% \Task  \what~ De olika samlingarna i \code{scala.collection} används flitigt i andra paket, exempelvis \code{scala.util} och \code{scala.io}.
%
% \Subtask Vad händer här? (Metoden \code{shuffle} skapar en ny samling med elementen i slumpvis ordning.)
% \begin{REPL}
% val xs = Vector(1,2,3)
% def blandat = scala.util.Random.shuffle(xs)
% def test = if (xs == blandat) "lika" else "olika"
% (for(i <- 1 to 100) yield test).count(_ == "lika")
% \end{REPL}
%
%
% \Subtask Skapa en textfil med namnet \code{fil.txt} som innehåller lite text och läs in den med: \\\code{scala.io.Source.fromFile("fil.txt", "UTF-8").getLines.toVector}
% \begin{REPL}
% > cat > fil.txt
% hejsan
% svejsan
% > scala
% scala> val xs = scala.io.Source.fromFile("fil.txt", "UTF-8").getLines.toVector
% scala> xs.foreach(println)
% \end{REPL}
%
%
% \Subtask Vad händer här? (Metoden \code{trim} på värden av typen \code{String} ger en ny sträng med blanktecken i början och slutet borttagna.)
% \begin{REPL}
% scala> val pgk =
%   scala.io.Source.fromURL("http://cs.lth.se/pgk/","UTF-8").getLines.toVector
% scala> pgk.foreach(println)
% scala> pgk.map(_.trim).
%          filterNot(_.startsWith("<")).
%          filterNot(_.isEmpty).
%          foreach(println)
% \end{REPL}
%
%
%
% \SOLUTION
%
%
% \TaskSolved \what
%
%
% \SubtaskSolved
% Vi instansierar en vektor xs med talen 1, 2 och 3.
% sedan definierar vi en metod blandat som ger oss en randomiserad version av xs.
% sedan definierar vi en till metod som testar om xs är lika med resultatet från blandat. Om det är så returnerar den strängen ''lika'' annars ''olika''.
% Sist kör vi en for-loop där vi 100 gånger kör testet, samtidigt räknas hur många gånger ''lika'' returneras.
%
% Vårt resultat är en siffra på hur många gånger xs var samma som en blandad version av sig själv, eftersom det finns 6 permutationer med 3 variabler så borde det vara ungefär 1/6 chans.
%
% \SubtaskSolved  -
%
% \SubtaskSolved
% \\ \code{map(\_.trim)} tar bort alla onödiga mellanrum i början och slutet på varje rad
% \\ \code{filterNot(\_.startsWith(''<''))} filtrerar bort alla rader som börjar med strängen ''<''
% \\ \code{filterNot(\_.isEmpty)} filtrerar bort alla tomma rader.
% \\ \code{foreach(println)} skriver ut alla rader.
%
%
% \QUESTEND
%
%
%
%
% %%<AUTOEXTRACTED by mergesolu>%%      %Uppgift 18
%
%
%
%
% \WHAT{Jämföra List och Vector.}
%
% \QUESTBEGIN
%
% \Task  \what~  En indexerbar sekvens av värden kallas vektor eller lista. I Scala finns flera klasser som kan kan indexeras, däribland klasserna \code{Vector} och \code{List}.
%
% \Subtask \emph{Likheter mellan \code{Vector} och \code{List}.} Kör nedan rader i REPL. Prova indexera i båda och studera hur stor andel av metoderna som är gemensamma.
% \begin{REPL}
% scala> val sv = Vector("en", "två", "tre", "fyra")
% scala> val en = List("one", "two", "three", "four")
% scala> sv(0) + sv(3)
% scala> en(0) + en(3)
% scala> sv. //tryck TAB
% scala> en. //tryck TAB
% \end{REPL}
%
% \Subtask \emph{Skillnader mellan \code{Vector} och \code{List}.} Klassen \code{Vector} i Scala har ''under huven'' en avancerad datastruktur i form av ett s.k. självbalanserande träd, vilket gör att \code{Vector} är snabbare än \code{List} på nästan allt, \emph{utom} att bearbeta elementen i \emph{början} av sekvensen; vill man lägga till och ta bort i början av en \code{List} så kan det ibland gå ungefär dubbelt så fort jämfört med \code{Vector}, medan alla andra operationer är lika snabba eller snabbare med \code{Vector}. Det finns ett fåtal speciella metoder, som bara finns i \code{List}, för att skapa en lista och lägga till i början av en lista. Vad händer nedan?
%
% \begin{REPL}
% scala> var xs = "one" :: "two" :: "three" :: "four" :: Nil
% scala> xs = "zero" :: xs
% scala> val ys = xs.reverse ::: xs
% \end{REPL}
%
%
% \SOLUTION
%
%
% \TaskSolved \what
%
%
% \SubtaskSolved
% I princip alla metoder delas, en lista har några fler t. ex. ''::'', '':::'', ''mapConserve'' osv.
%
% \SubtaskSolved
% Först skapas en lista med 4 sträng värden och instansierar variablen xs med det värdet.
% sedan skapar vi en ny lista, som består av ''zero'' + den gamla listan och ger värdet till xs.
% Sist instansierar vi en ny variabel ys, som får värdet av xs omvänd plus xs.
%
%
% \QUESTEND
%
%
%
%
% %%<AUTOEXTRACTED by mergesolu>%%      %Uppgift 19
%
%
%
%
% \WHAT{Mängd.}
%
% \QUESTBEGIN
%
% \Task  \what~  En mängd är en samling som garanterar att det inte finns några dubbletter. Det går dessutom väldigt snabbt, även i stora mängder, att kolla om ett element finns eller inte i mängden. Elementen i samlingen \code{Set} hamnar ibland, av effektivitetsskäl, i en förvånande ordning.
% \begin{REPL}
% scala> val s = Set("Malmö", "Stockholm", "Göteborg", "Köpenhamn", "Oslo")
% s: scala.collection.immutable.Set[String] =
%      Set(Oslo, Malmö, Köpenhamn, Stockholm, Göteborg)
%
% scala> val t = Set("Sverige", "Sverige", "Sverige", "Danmark", "Norge")
% t: scala.collection.immutable.Set[String] = Set(Sverige, Danmark, Norge)
% \end{REPL}
% Givet ovan deklarationer: vad blir värde och typ av nedan uttryck?
%
% \Subtask \code{s + "Malmö" == s}
%
% \Subtask \code{s ++ t}
%
% \Subtask \code{Set("Malmö", "Oslo").subsetOf(s)}
%
% \Subtask \code{s subsetOf Set("Malmö", "Oslo")}
%
% \Subtask \code{s contains "Lund"}
%
% \Subtask \code{s apply "Lund"}
%
% \Subtask \code{s("Malmö")}
%
% \Subtask \code{s - "Stockholm"}
%
% \Subtask \code{t - ("Norge", "Danmark", "Tyskland")}
%
% \Subtask \code{s -- t}
%
% \Subtask \code{s -- Set("Malmö", "Oslo")}
%
% \Subtask \code{Set(1,2,3) intersect Set(2,3,4)}
%
% \Subtask \code{Set(1,2,3) & Set(2,3,4)}
%
% \Subtask \code{Set(1,2,3) union Set(2,3,4)}
%
% \Subtask \code{Set(1,2,3) | Set(2,3,4)}
%
%
% \SOLUTION
%
%
% \TaskSolved \what
%
%
% \SubtaskSolved
% true, Boolean
%
% \SubtaskSolved
% En samling av alla värden i s och t, Set[String]
%
% \SubtaskSolved
% true, Boolean
%
% \SubtaskSolved
% false, Boolean
%
% \SubtaskSolved
% false, Boolean
%
% \SubtaskSolved
% false, Boolean
%
% \SubtaskSolved
% true, Boolean
%
% \SubtaskSolved
% Samlingen s utan elementet ''Stockholm'', Set[String]
%
% \SubtaskSolved
% Samlingen t utan elementen ''Norge'' och ''Danmark'', Set[String]
%
% \SubtaskSolved
% returnerar s, Set[String]
%
% \SubtaskSolved
% Samlingen s utan ''Malmö'' och ''Oslo'', Set[String]
%
% \SubtaskSolved
% Set(2, 3), Set[Int]
%
% \SubtaskSolved
% se deluppgift l
%
% \SubtaskSolved
% Set(1, 2, 3 ,4), Set[Int]
%
% \SubtaskSolved
% se deluppgift n
%
%
% \QUESTEND
%
%
%
%
% %%<AUTOEXTRACTED by mergesolu>%%      %Uppgift 20
%
%
%
%
% \WHAT{Slå upp värden från nycklar med \code{Map}.}
%
% \QUESTBEGIN
%
% \Task  \what~  Samlingen \code{Map} är mycket användbar. Med den kan man snabbt leta upp ett värde om man har en nyckel. Samlingen \code{Map} är en generalisering av en vektor, där man kan ''indexera'', inte bara med ett heltal, utan med vilken typ av värde som helst, t.ex. en sträng. Datastrukturen \code{Map} är en s.k. \emph{associativ array}\footnote{\href{https://en.wikipedia.org/wiki/Associative_array}{https://en.wikipedia.org/wiki/Associative\_array}}, implementerad som en s.k. \emph{hashtabell}\footnote{\href{https://en.wikipedia.org/wiki/Hash_table}{https://en.wikipedia.org/wiki/Hash\_table}}.
% \begin{REPL}
% scala> var huvudstad =
%   Map("Sverige" -> "Stockholm", "Norge" -> "Oslo", "Skåne" -> "Malmö")
% \end{REPL}
% Givet ovan variabel \code{huvudstad}, förklara vad som händer nedan?
%
% \Subtask \code{huvudstad apply "Skåne"}
%
% \Subtask \code{huvudstad("Sverige")}
%
% \Subtask \code{huvudstad.contains("Skåne")}
%
% \Subtask \code{huvudstad.contains("Malmö")}
%
% \Subtask \code{huvudstad += "Danmark" -> "Köpenhamn"}
%
% \Subtask \code{huvudstad.foreach(println)}
%
% \Subtask \code{huvudstad getOrElse ("Norge", "???") }
%
% \Subtask \code{huvudstad getOrElse ("Finland", "???") }
%
% \Subtask \code{huvudstad.keys.toVector.sorted}
%
% \Subtask \code{huvudstad.values.toVector.sorted}
%
% \Subtask \code{huvudstad - "Skåne"}
%
% \Subtask \code{huvudstad - "Jylland"}
%
% \Subtask \code{huvudstad = huvudstad.updated("Skåne","Lund") }
%
%
%
% \SOLUTION
%
%
% \TaskSolved \what
%
%
% \SubtaskSolved
% Returnerar strängen ''Malmö'' eftersom det värdet är indexerat på platsen ''Skåne''.
%
% \SubtaskSolved
% Returnerar strängen ''Stockholm'' eftersom det värdet är indexerat på platsen ''Sverige''.
%
% \SubtaskSolved
% true, eftersom huvudstad innehåller indexet ''Skåne''
%
% \SubtaskSolved
% false, eftersom huvudstad ej innehåller indexet ''Malmö''. Notera att det är index och inte värden vi
% kollar om det finns.
%
% \SubtaskSolved
% Lägger till indexet ''Danmark'' med värdet ''Köpenhamn'' i samlingen.
%
% \SubtaskSolved
% Skriver ut alla 2-tupler.
%
% \SubtaskSolved
% Returnerar ''Oslo'', Note: Om indexet ''Norge'' inte hade funnits hade ''???'' returnerats istället.
%
% \SubtaskSolved
% Returnerar ''???''
%
% \SubtaskSolved
% Returnerar en sorterar vektor med alla index.
%
% \SubtaskSolved
% Returnerar en sorterar vektor med alla värden.
%
% \SubtaskSolved
% Returnerar en ny mängd men med ''Skåne'' -> ''Malmö'' borttaget.
%
% \SubtaskSolved
% Returnerar huvudstad mängden eftersom det inte finns ett ''Jylland'' index att ta bort.
%
% \SubtaskSolved
% Uppdaterar indexet ''Skåne'' till att istället leda till värdet ''Lund''
%
%
% \QUESTEND
%
%
%
%
% %%<AUTOEXTRACTED by mergesolu>%%      %Uppgift 21
%
%
%
%
% \WHAT{Skapa Map från en samling.}
%
% \QUESTBEGIN
%
% \Task  \what~
%
% \Subtask Definiera denna vektor och undersök dess typ:
% \begin{Code}
% val pairs = Vector(
%   ("Björn", 46462229009L),
%   ("Maj", 46462221667L),
%   ("Gustav", 46462224906L))
% \end{Code}
%
% \Subtask Vad har variablen \code{telnr} nedan för typ: \\ \code{var telnr = pairs.toMap}
%
% \Subtask Använd \code{telnr} för att slå upp telefonnummer för Maj och Kim med hjälp av metoderna \code{apply} och \code{get}.
%
% \Subtask Använd metoden \code{getOrElse} vid upplagningar av \code{telnr} och ge \code{-1L} som telefonnummer i händelse av att ett nummer inte finns.
%
% \Subtask Lägg till \code{("Fröken Ur", 464690510L)} i \code{telnr}-mappen.
%
% \Subtask Skapa en \code{Vector[(String, String)]} enligt nedan, så att telefonnumret blir en sträng utan inledande landsnummer men med en nolla i riktnumret. Byt ut \code{???} mot lämpligt uttryck.
% \begin{REPL}
% scala> telnr.toVector.map(p => ???)
% res85: Vector[(String, String)] = Vector(("Björn", "0462229009"), ("Maj",
% "0462221667"), ("Gustav", "0462224906"), ("Fröken Ur", 04690510"))
%
% \end{REPL}
%
% \Subtask Använd vektorn i resultatet ovan för att skapa en ny \code{Map[String, String]} med nationella telefonnumer. Slå upp numret till Fröken Ur.
%
% \SOLUTION
%
%
% \TaskSolved \what
%
%
% \SubtaskSolved
% \begin{REPLnonum}
% pairs: scala.collection.immutable.Vector[(String, Long)] =
% 					Vector((Björn,444), (Maj,441), (Lucy,666))
% \end{REPLnonum}
%
% \SubtaskSolved
% Map[String, Long]
%
% \SubtaskSolved
% \begin{REPLnonum}
% scala> telnr(''Maj'')
% res0: Long = 441
%
% scala> telnr.get(''Maj'')
% res1: Option[Long] = Some(441)
%
% scala> telnr(''Kim'')
% java.util.NoSuchElementException: key not found: 'Kim
%   at scala.collection.MapLike$class.default(MapLike.scala:228)
%   at scala.collection.AbstractMap.default(Map.scala:59)
%   at scala.collection.MapLike$class.apply(MapLike.scala:141)
%   at scala.collection.AbstractMap.apply(Map.scala:59)
%   ... 32 elided
%
% scala> telnr.get(''Kim'')
% res2: Option[Long] = None
% \end{REPLnonum}
%
% \SubtaskSolved
% \begin{REPLnonum}
% scala> telnr.getOrElse(''Maj'', -1L)
% res0: Long = 441
%
% scala> telnr.getOrElse(''Kim'', -1L)
% res1: Long = -1
% \end{REPLnonum}
%
% \SubtaskSolved
% telnr += ''Fröken Ur'' -> 464690510L
%
% \SubtaskSolved
% telnr.toVector.map(p => p.\_1 -> (''0'' + p.\_2.toString.substring(2)))
%
% \SubtaskSolved
% Använd metoden toMap och apply.
%
%
%
%
% \QUESTEND
%
%
%
%
% %%<AUTOEXTRACTED by mergesolu>%%      %Uppgift 22
%
%
%
%
% \WHAT{Samlingsmetoden \code{maxBy}.}
%
% \QUESTBEGIN
%
% \Task  \what~  Med samlingsmetoden \code{maxBy} kan man själv definiera vad som ska maximeras. (Denna metod kommer du att behöva i veckans laboration.)
%
% \Subtask Förklara vad som händer nedan.
% \begin{REPL}
% scala> val xs = Vector((2,3), (1,5), (-1, 1), (7, 2))
% scala> xs.maxBy(x => x._1)
% scala> xs.maxBy(x => x._2)
% \end{REPL}
%
% \Subtask Om man bara använder en parameter i en anonym funktion, till exempel parametern \code{x} i lambdauttrycket \code{x => x + 1} \emph{en enda} gång, och kompilatorn kan gissa alla typer, kan man använda understreck som ''platshållare'' för att förkorta lambdauttrycket så här: \code{ _ + 1}
%
% Skriv uttrycken på raderna 2 och 3 i föregående deluppgift på ett kortare sätt med hjälp platshållarsyntax \Eng{place holder syntax}.
%
% \Subtask På motsvarande sätt kan man använda \code{minBy} för att välja vilken funktion som definierar minimum. Prova \code{minBy} på motsvarande sätt som i föregående deluppgifter.
%
% \SOLUTION
%
%
% \TaskSolved \what
%
%
% \SubtaskSolved   Metoden maxBy hämtar det element som är ''störst'', på rad två gör \code{x => x._1} att första värdet i tuplerna används för att bestämma vilken som är störst. Likt gör \code{x => x._2} på rad tre att istället det andra värdet används.
%
% \SubtaskSolved
% \begin{REPLnonum}
% scala> xs.maxBy(_._1)
% scala> xs.maxBy(_._2)
% \end{REPLnonum}
%
% \SubtaskSolved
% \begin{REPLnonum}
% scala> xs.minBy(_._1)
% scala> xs.minBy(_._2)
% \end{REPLnonum}
%
%
%
% \QUESTEND
%
%
%
%
%
%
%
%
% \WHAT{NEEDS A TOPIC DESCRIPTION}
%
% \QUESTBEGIN
%
% \Task  \what~ Skriv nedan program med en editor och kompilera från terminalen. Lägg till kod i huvudprogrammet som testar klassen \code{Account} och kompilera och kör. Utvidga sedan klassen \code{Account} med fler attribut och funktioner som du väljer själv.
%
% \begin{Code}
% class Account(val number: Long, val maxCredit: Int){
%   private var balance = 0
%
%   def deposit(amount: Int): Int = {
%     if (amount > 0) {balance += amount}
%     balance
%   }
%
%   def withdraw(amount: Int): (Int, Int) = if (amount > 0) {
%     val allowedWithdrawal =
%       if (amount < balance + maxCredit) amount
%       else balance + maxCredit
%     balance = balance - allowedWithdrawal
%     (allowedWithdrawal, balance)
%   } else (0, balance)
%
%   def show: Unit =
%     println("Account Nbr: " + number + " balance: " + balance)
% }
%
% object Main {
%   def main(args: Array[String]): Unit = {
%     ???
%   }
% }
% \end{Code}
%
%
%
% \SOLUTION
%
%
% \QUESTEND
%
%
%
%
%
%
% \WHAT{NEEDS A TOPIC DESCRIPTION}
%
% \QUESTBEGIN
%
% \Task \label{task:keno-set} \what~  Läs om reglerna för spelet Keno här: \\ \url{https://sv.wikipedia.org/wiki/Keno} och gör deluppgifterna nedan.
%
% \Subtask Skapa en klass \code{Keno} som kan användas för att genomföra en Kenodragning. Låt klassen ha ett privat attribut \code{balls} som är en föränderlig mängd med heltal och som från början är tom. Implementera lämpliga metoder i klassen för att användaren av klassen ska kunna dra nya slumpmässiga bollar som inte redan är dragna.
%
% \Subtask Skapa en \code{case class KenoBet(bet: Set[Int])} för att hålla reda vilka 11 bollar en viss person satsar på. Definiera en metod \\ \code{def numberOfHits(keno: Keno): Int = ???}\\ i case-klassen \code{KenoBet} som givet en kenodragning räknar ut hur många bollar som satsats rätt.
%
% \Subtask Skriv ett huvudprogram som simulerar en enkel Kenodragning. Låt två personer satsa på 11 slumpmässiga bollar, genomför en dragning av 20 bollar ur 70 möjliga och kontrollera sedan hur många bollar som personerna har prickat rätt.
%
%
%
%
%
% \SOLUTION
%
%
% \QUESTEND
%
%
%
%
%
%
% \WHAT{Dokumentationen för \code{Any}.}
%
% \QUESTBEGIN
%
% \Task  \what~  Undersök vilka metoder som finns i klassen Any här: \href{http://www.scala-lang.org/api/current/scala/Any.html}{http://www.scala-lang.org/api/current/scala/Any.html}. Prova några av metoderna i REPL.
%
% \SOLUTION
%
%
% \QUESTEND
%
%
%
%
%
%
% \WHAT{Dokumentationen för samlingar.}
%
% \QUESTBEGIN
%
% \Task  \what~  Leta upp metoden \code{tabulate} i dokumentationen för objektet \code{Vector} nästan längst ner i listan här: \\ \href{http://www.scala-lang.org/api/current/scala/collection/immutable/Vector.html}{http://www.scala-lang.org/api/current/scala/collection/immutable/Vector.html} \\Leta upp den variant av \code{tabulate} som har signaturen:\\ \code{def tabulate[A](n: Int)(f: (Int) => A): Vector[A] }\\ Klicka på den gråfyllda trekanten till vänster om signaturen som fäller ut beskrivningen
%
% \Subtask Förklara vad som händer här:
% \begin{REPLnonum}
% scala> Vector.tabulate(10)(i => i % 3)
% \end{REPLnonum}
%
% \Subtask Klicka på det blåa stora o-et överst på sidan, för att växla till klass-vyn och studera listan med alla metoder  i klassen \code{Vector}.
%
%
% \SOLUTION
%
%
% \QUESTEND
%
%
%
%
%
%
% \WHAT{Fler metoder på indexerbara sekvenser.}
%
% \QUESTBEGIN
%
% \Task  \what~  Deklarera följande vektorer i REPL.
% \begin{REPL}
% scala> val xs = (1 to 10).toVector
% scala> val a = Vector("abra", "ka", "dabra")
% scala> val b = Vector( "sim", "sala", "bim", "sala", "bim")
% \end{REPL}
% Undersök i REPL vad som händer nedan. Alla dessa metoder fungerar på alla samlingar som är indexerbara sekvenser. Vad har uttrycken för värde och typ? Förklara vad metoden gör. Studera även denna  översikt: \href{http://docs.scala-lang.org/overviews/collections/seqs}{docs.scala-lang.org/overviews/collections/seqs}
%
% \Subtask \code{b.indexWhere(s => s.startsWith("b"))}  % advanced
%
% \Subtask \code{a.indices}  % advanced
%
% \Subtask \code{xs.patch(1, Vector(42,43,44), 7)} % advanced
%
% \Subtask \code{xs.segmentLength(_ < 8, 2)} % advanced
%
% \Subtask \code{b.sortBy(_.reverse)}  % advanced
%
% \Subtask \code{b.sortWith((s1, s2) => s1.size < s2.size)} % advanced
%
% \Subtask \code{a.reverseMap(_.size)}	% advanced
%
% \Subtask \code{a intersect Vector("ka", "boom", "pow")} % advanced
%
% \Subtask \code{a diff Vector("ka")} % advanced
%
% \Subtask \code{a union Vector("ka", "boom", "pow")} % advanced
%
%
%
% \SOLUTION
%
%
% \QUESTEND
%
%
%
%
% \WHAT{NEEDS A TOPIC DESCRIPTION}
%
% \QUESTBEGIN
%
% \Task  \what~ För samlingen \code{List} finns en alternativ metod till \code{+:} som heter \code{::} och kallas ''cons'' och som i kombination med objektet \code{Nil} kan användas för att med alternativ syntax bygga listor. Läs om detta här: \\ \href{http://alvinalexander.com/scala/how-create-scala-list-range-fill-tabulate-constructors}{alvinalexander.com/scala/how-create-scala-list-range-fill-tabulate-constructors} \\ och hitta på några egna övningar för att undersöka hur cons och Nil fungerar. Metoder som slutar med kolon är högerassociativa. Läs mer om detta här: \href{http://www.artima.com/pins1ed/basic-types-and-operations.html#5.8}{http://www.artima.com/pins1ed/basic-types-and-operations.html\#5.8}\SOLUTION
%
%
% \QUESTEND

%!TEX encoding = UTF-8 Unicode

%!TEX root = ../compendium2.tex


\Lab{\LabWeekNINE}
\begin{Goals}
%!TEX encoding = UTF-8 Unicode
%!TEX root = ../compendium2.tex

%\item Kunna använda en integrerad utvecklingsmiljö (IDE).
%\item Kunna använda färdiga funktioner för att läsa till, och skriva från, textfil.
%\item Kunna använda enkla case-klasser.
%\item Kunna skapa och använda enkla klasser med föränderlig data.
\item Kunna skapa och använda nyckel-värde-tabeller med samlingstypen \code{Map}.
\item Kunna skapa och använda mängder med samlingstypen \code{Set}.
\item Förstå skillnaden mellan en ordnad sekvens och en mängd.
\item Förstå likheter och skillnader mellan en sekvens av par och en nyckel-värde-tabell. 
\item Kunna implementera algoritmer som använder nästlade strukturer. 
%\item Kunna skapa en ny samling från en befintlig samling.
%\item Förstå skillnaden mellan kompileringsfel och exekveringsfel.
%\item Kunna felsöka i små program med hjälp av utskrifter.
%\item Kunna felsöka i små program med hjälp av en debugger i en IDE.

\end{Goals}

\begin{Preparations}
\item \DoExercise{\ExeWeekNINE}{09}
%\item Läs om integrerade utvecklingsmiljöer i appendix \ref{appendix:ide}.
%\item Välj vilken IDE du vill använda på denna lab. %Om du inte vet vilken, välj \textbf{Eclipse} med ScalaIDE, som flest handledare känner väl till.
%\item Bekanta dig med utvecklingsmiljön genom att skapa ett nytt projekt och gör ett ''Hello World''-program.
%\item Ladda hem kursens \emph{workspace} enligt instruktioner i appendix \ref{subsubsection:download--import-workspace} och kontrollera så att du med \emph{Run} kan köra igång de båda ofärdiga \code{main}-metoderna i projektet \code{w04_pirates} inifrån din IDE. Om du inte får rätt på \emph{Run Configuration...} etc. så fråga någon om hjälp.
\item Läs igenom hela laborationen.
\item Hämta och läs given kod via \href{https://github.com/lunduniversity/introprog/tree/master/workspace/w09_words}{kursen github-plats} eller via \href{https://cs.lth.se/pgk/download/}{cs.lth.se/pgk/download}
%\item {\"O}ppna Scala IDE i Eclipse enligt intruktionerna XX.
%\item Skapa ett projekt och skapa ett \code{object Hello} med en \code{main}-metod enligt XY.
%\item Skriv ut en h{\"a}lsning till terminalen med \code{println("...")} och testk{\"o}r programmet genom att markera filnamnet i projektmenyn och trycka p{\aa} den gr{\"o}na pilen. Kontrollera att h{\"a}lsningen skrivs ut!
\end{Preparations}


\subsection{Bakgrund}

Denna uppgift handlar om analys av naurligt språk \Eng{Natural Language Processing, NLP}. Språkanalys bygger ofta på statistik över förekomsten av olika ord i långa texter. Du ska skriva kod, som utifrån en lång text, till exempel en bok, kan hjälpa dig att svara på denna typ av frågor:
\begin{itemize}[noitemsep]
\item Hur vanligt är ett visst ord i en given text?
\item Vilket är det vanligaste ordet som följer efter ett visst ord?
\item Hur kan man generera ordsekvenser som liknar ordföljden i en given text?
\end{itemize}

\noindent För att kunna svara på sådana frågor ska du skapa frekvenstabeller och även så kallade \emph{n-gram}; sekvenser av $n$ ord som förekommer i följd i en text. Exempel på några 2-gram (kallas även \emph{bigram}) som finns i föregående mening: (för, att), (att, kunna), (kunna, svara), (svara, på), (på, sådana), och så vidare.\footnote{Du kan undersöka olika n-gram i en stor mängd böcker med hjälp av Googles n-gram-viewer: \url{https://books.google.com/ngrams/}}

\subsection{Obligatoriska uppgifter}

Du ska bygga ditt program med en editor, t.ex. VS \texttt{code}, och kompilera och köra din kod i terminalen med hjälp av \code{scala-cli} i \textit{watch mode} med det arbetssätt som beskrivs i appendix \ref{appendix:build} avsnitt \ref{appendix:build-scala-cli-watch-mode}. Medan du steg för steg utvecklar ditt program, ska du parallellt göra experiment i REPL för att undersöka hur du kan använda samlingsmetoder för att lösa uppgifterna.
Kod att utgå ifrån finns här: \url{https://github.com/lunduniversity/introprog/tree/master/workspace/w09_words}

Dessa ofärdiga kodfiler ligger i paketet \code{nlp}:
\begin{itemize}
  \item \href{https://github.com/lunduniversity/introprog/blob/master/workspace/w09_words/FreqMapBuilder.scala}{\texttt{FreqMapBuilder.scala}} innehåller ett skelett till en klass för att, ord för ord, bygga en nyckel-värde-tabell som registrerar antalet förekomster av olika ord. Att implementera denna ingick i övningen du gjorde tidigare i veckan.

  \item \href{https://github.com/lunduniversity/introprog/blob/master/workspace/w09_words/Text.scala}{\texttt{Text.scala}} innehåller ett skelett till en klass som kan göra textbehandling genom att analysera ord i en text.

  \item \href{https://github.com/lunduniversity/introprog/blob/master/workspace/w09_words/Main.scala}{\texttt{Main.scala}} innehåller ett ofärdigt huvudprogram som du kan använda i laborationens senare del.
\end{itemize}

\Task \emph{Skapa frekvenstabeller}. Du ska använda \code{FreqMapBuilder} från veckans övning för att skapa frekvenstabeller av typen \code{Map[String, Int]}, där nyckel-värde-paren i tabellen anger antalet förekomster av en viss sträng.

\Subtask Lägg klassen \code{FreqMapBuilder} i ett paket som heter \code{nlp} och kompilera.

\begin{figure}[H]
\scalainputlisting[numbers=left,basicstyle=\ttfamily\fontsize{10.5}{12.5}\selectfont]{../workspace/w09_words/FreqMapBuilder.scala}
%\caption{Den ofärdiga klassen \code{FreqMapBuilder}.}
%\label{data:fig-freqmap}
\end{figure}

\Subtask Testa noga så att din \code{FreqMapBuilder} fungerar korrekt. Exempel på test i REPL:
\begin{REPL}
scala> import nlp._

scala> val fmb = FreqMapBuilder("hej", "på", "dej")
fmb: nlp.FreqMapBuilder = nlp.FreqMapBuilder@458f85ef

scala> fmb.add("hej")

scala> fmb.toMap
res0: Map[String,Int] = Map(på -> 1, hej -> 2, dej -> 1)

scala> (1 to Short.MaxValue).foreach(i => fmb.add(i.toString))

scala> fmb.toMap.size
res1: Int = 32770

scala> fmb.toMap
res2: Map[String,Int] = 
  Map(10292 -> 1, 19125 -> 1, 26985 -> 1, 29301 -> 1, 5451 -> 1, 4018 -> 1, 31211 -> 1, ...
\end{REPL}

\noindent I kommande uppgifter ska du steg för steg skapa och testa case-klassen \code{Text}. %figur \ref{data:fig-text}.

\begin{figure}[H]
\scalainputlisting[numbers=left,basicstyle=\ttfamily\fontsize{10.4}{12.5}\selectfont]{../workspace/w09_words/Text.scala}
%\caption{Den ofärdiga klassen \code{Text}.}
%\label{data:fig-text}
\end{figure}





\Task \emph{Dela upp en sträng i ord}. Du ska implementera medlemmen \code{words}. Den ska innehålla en vektor med alla ord i \code{source}, utan andra tecken än bokstäver.
Detta åstadkommer du genom att utgå ifrån strängen \code{source} och i tur och ordning göra följande:
\begin{enumerate}%[nolistsep, noitemsep]
\item byta ut alla tecken i \code{source} för vilka \code{isLetter} är falskt mot \code{' '}
\item dela upp strängen från föregående steg i en array av strängar med \code{split(' ')}
\item filtrera bort alla tomma strängar
\item gör om alla bokstäver i alla strängar till små bokstäver
\item gör om arrayen till en sekvens av typen \code{Vector[String]}.
\end{enumerate}

\noindent Testa så att \code{words}, och de värden som använder \code{words}, fungerar i REPL:
\begin{REPL}
scala> val t = Text("Gurka är ingen tomat, men gurka är en grönsak.")

scala> t.words
res1: Vector[String] =
  Vector(gurka, är, ingen, tomat, men, gurka, är, en, grönsak)

scala> t.distinct
res2: Vector[String] =
  Vector(gurka, är, ingen, tomat, men, en, grönsak)

scala> t.wordSet
res3: Set[String] = Set(grönsak, är, gurka, men, ingen, tomat, en)

scala> t.wordsOfLength(5)
res4: Set[String] = Set(gurka, ingen, tomat)

\end{REPL}



\Task Du ska nu skapa ordfrekvenstabellen \code{wordFreq} genom att registrera ordförekomster med hjälp av \code{FreqMapBuilder}. Tabellen \code{wordFreq} ska bestå av nyckelvärdepar \code{w -> f} där \code{f} är antalet gånger ordet \code{w} förekommer i \code{words}. Testa \code{wordFreq} genom att ladda ner boken ''Skattkammarön'' skriven av Robert Louis Stevenson\footnote{Copyright för denna bok har gått ut, så du gör dig inte skyldig till piratkopiering (i juridisk mening).} och undersök frekvensen för olika vanliga ord. Vilket ord är vanligast? Näst vanligast?

\begin{REPL}[basicstyle=\color{white}\ttfamily\fontsize{9}{11}\selectfont]
scala> val piratbok = Text.fromURL("https://fileadmin.cs.lth.se/pgk/skattkammaron.txt")
val piratbok: nlp.Text = Text(Herr Trelawney, doktor Livesey och de övriga herrarna har bett mig att skriva ner alla omständigheter kring Skattkammarön, ...

scala> piratbok.words.size
val res0: Int = 69438

scala> piratbok.wordFreq("pirat")
val res1: Int = 7
\end{REPL}
Länkar till böcker i UTF-8-format som du kan använda i dina tester:
\begin{itemize}%[nolistsep,noitemsep]
\item ''Skattkammarön'' av R. L. Stevenson: \\\url{https://fileadmin.cs.lth.se/pgk/skattkammaron.txt}
\item ''Inferno'' av August Strindberg: \\\url{https://fileadmin.cs.lth.se/pgk/inferno.txt}
\item ''Pride and Prejudice'' av Jane Austen: \\\url{https://fileadmin.cs.lth.se/pgk/prideandprejudice.txt}
\item Projekt Gutenberg med många fritt tillgängliga böcker i textformat: \\\url{https://www.gutenberg.org/}
\end{itemize}






\Task Implementera metoden \code{ngrams} som ger en sekvens med alla ordföljder i $n$ steg. \emph{Tips:} På veckans övning ingick att undersöka hur metoden \code{sliding} fungerar, med vilken du kan skapa $n$-gram. Gör \code{toVector} på resultatet från \code{sliding}. Testa noga så att \code{ngrams} och \code{bigrams} fungerar korrekt innan du går vidare.
\begin{REPL}
scala> piratbok.ngrams(3).take(2)
val res1: Vector[Vector[String]] =
  Vector(Vector(herr, trelawney, doktor), Vector(trelawney, doktor, livesey))

scala> piratbok.bigrams.take(2)
val res2: Vector[(String, String)] =
  Vector((herr,trelawney), (trelawney,doktor))
\end{REPL}

\Task Implementera \code{followFreq}, som ska innehålla en nyckel-värde-tabell där värdet i sin tur är en frekvenstabell över de ord som kommer efter nyckeln. \label{task-follow-freq}

Genom att analysera alla ordpar kan vi få fram vilket som är det vanligaste ordet som följer efter ett givet ord. Metoden \code{bigrams} ger oss alla ordpar \code{(w1, w2)} där \code{w2} följer efter \code{w1}. Vi kan spara statistiken över efterföljande ord i en nyckelvärdetabell med mappningarna \code{w -> f} där nyckeln \code{w} är ett ord  och värdet \code{f} är en frekvenstabell av typen \code{Map[String, Int]}. I frekvenstabellen lagrar vi frekvensen för alla de ord som följer efter \code{w}. Du ska alltså bygga en nästlad tabell av typen \code{Map[String, Map[String, Int]]}. Rita en bild av den nästlade strukturen.\Pen

Implementera metoden followFreq genom att utgå från nedan pseudokod:
\begin{Code}
val result = collection.mutable.Map.empty[String, FreqMapBuilder]
for (key, next) <- bigrams do
  if /* key finns redan definierad i result */ then
    /* på "platsen" result(key): lägg till next i frekvenstabellen */
  else
    /* lägg till (key -> ny frekvenstabell med next) i result*/
end for
result.map(p => p._1 -> p._2.toMap).toMap // toMap ger oföränderlig Map
\end{Code}
Gör utskrifter för att ta reda på följande frågor. Skriv ner svaren och var redo att redovisa dem i samband med kontrollfrågorna (se avsnitt \ref{words-check}).\Pen

\Subtask Vilka ord brukar följa efter \emph{han} respektive \emph{hon} i Stevensons ''Skattkammarön''?

\Subtask Vilka ord brukar följa efter \emph{han} respektive \emph{hon} i Strindbergs ''Inferno''?

\Subtask Vilka ord brukar följa efter \emph{he} respektive \emph{she} i Austens ''Pride and Prejudice''?


\Task Skapa ett huvudprogram som rapporterar valfria, intressanta mått om orden i en text. Programmet ska ta textens källa som argument, givet som en URL eller ett filnamn. Skriv huvudprogrammet i filen \code{Main.scala} i ett singelobjekt med namnet \code{Main}. Exempel på en rapport som ditt huvudprogram kan generera finns nedan. Här ges även ett heltal som argument som styr topplistornas längd.
\begin{REPL}
> scala run . -- https://fileadmin.cs.lth.se/pgk/skattkammaron.txt 13

Källa: https://fileadmin.cs.lth.se/pgk/skattkammaron.txt

*** Antal ord: 69438

*** De 13 vanligaste orden och deras frekvens:
(och,3089), (jag,2007), (att,1594), (det,1382), (en,1262),
(i,1244), (som,1132), (på,1068), (han,1063), (var,990),
(med,854), (den,774), (av,740)

*** De 13 längsta orden och deras längd:
(besättningsmedlemmarnas,23), (befästningsanordningar,22),
(temperamentsuppvisning,22), (undsättningsexpedition,22),
(besättningsmedlemmarna,22), (försiktighetsåtgärder,21),
(undsättningsfartyget,20), (sjukdomsframkallande,20),
(husföreståndarinnans,20), (sjömansterminologin,19),
(parlamentärsflaggan,19), (bregravningsplatsen,19),
(tidvattenströmmarna,19)
\end{REPL}

\noindent Exempel på huvudprogram som kan skapa ovan utskrift:
\scalainputlisting[numbers=left,basicstyle=\ttfamily\fontsize{10.4}{12.5}\selectfont]{../workspace/w09_words/Main.scala}

\Task Para ihop dig med en annan student och planera hur ni tillsammans kan med hjälp av \url{https://cs.lth.se/pgk/muntabot} kan träna inför det muntliga provet där ni ömsesidigt agerar ''låtsasexaminator''. Gör en plan för när ni ska testa varandra på vilka veckor. Visa er plan för handledare och diskutera vad det innebär att vara en bra ''låtsasexaminator''.


\subsection{Kontrollfrågor}\label{words-check}

\begin{enumerate}[noitemsep, nolistsep]

\item Vilket är dina svar på uppgift \ref{task-follow-freq} a) b) c) på sidan \pageref{task-follow-freq}?

\item I vilken ordning hamnar elementen om man anropar \code{distinct} på en sekvens?

\item Om man itererar över en mängd, i vilken ordning behandlas elementen?

\item Ge exempel på när är det lämpligt att använda en mängd i stället för en sekvens av distinkta värden?

\item Är alla nycklar i en nyckel-värde-tabell garanterat unika?

\item Är alla värden i en nyckel-värde-tabell garanterat unika?

\item LTH-teknologen Oddput Clementin vill summera längden på varje sträng i en mängd och skriver:
\begin{REPL}
scala> Set("hej", "på", "dej").map(_.length).sum
res0: Int = 5
\end{REPL}
Varför blir det fel? Hur kan Oddput åtgärda problemet?
\end{enumerate}

\subsection{Frivilliga uppgifter}

\Task Bygg vidare på klassen \code{Text} och implementera nedan metod som ska ge ett slumpmässigt ord ur \code{wordSet}. Varje ord ska förekomma med lika stor sannolikhet.
\begin{Code}
def randomWord: String = ???
\end{Code}

\Task \label{task:words:randomSeq} Med NLP kan man generera slumpmässiga meningar som statistiskt sett liknar ''riktiga'', människoskapade meningar.

Implementera metoden \code{randomSeq(firstWord, n)} nedan i klassen \code{Text}. Den ska ge en sekvens $w_{1}, w_{2}, ..., w_{n}$  där $w_{1}$ är \code{firstWord} och $w_{i+1}$ är något slumpmässigt ord som är draget bland de ord som följer efter $w_{i}$. Detta kan du åstadkomma genom att varje efterföljande ord $w_{i+1}$ väljs ur \code{keys.toVector} för den \code{followFreq}-tabell som hör till $w_{i}$. Orden ska dras ur efterföljandemängden, med lika stor sannolikhet.
\begin{Code}
def randomSeq(firstWord: String, n: Int): Vector[String] = ???
\end{Code}
%\emph{Tips:} Ett sätt att garanterat välja slumpmässigt element med rektangelfördelning ur en sekvens är att använda metoden \code{scala.util.Random.shuffle} som tar en sekvens som argument och genererar en ny sekvens av samma typ, men med elementen ordnade i slumpmässig ordning på ett välblandat sätt, där varje möjlig ordning är lika sannolik.

\Task \label{task:words:mostCommonSeq} För att dina datorgenererade meningar verkligen ska likna mänskligt språk kan vi skapa de mest sannolika meningarna av olika längder ur vår analys av ordfrekvenser.

Lägg till metoden \code{mostCommonSeq} i klassen \code{Text} enligt nedan:
\begin{Code}
def mostCommonSeq(firstWord: String, n: Int): Vector[String] = ???
\end{Code}
\Subtask Implementera metoden så att resultatet blir en sekvens med \code{n} ord. Sekvensen ska börja med \code{firstWord} och därefter följas av det ord som är det \emph{vanligaste} efterföljande ordet efter \code{firstWord}, och därpå det vanligaste efterföljande ordet efter det, etc. \emph{Tips:} Använd en lokal variabel \code{val result} som är en ArrayBuffer till vilken du i en \code{while}-loop lägger de efterföljande orden.

\Subtask Jämför de slumpmässiga sekvenserna med sekvenser genererade med \code{randomSeq} i uppgift \ref{task:words:randomSeq}. Vilka sekvenser liknar mest ''riktiga'' meningar?


\Task Använd befintliga samlingsmetoder i stället för \code{FreqMapBuilder} för att registrera efterföljande ord.

\Subtask Undersök i REPL hur metoden \code{groupBy(x => x)} fungerar då den appliceras på en samling med strängar. Sök efter och studera dokumentationen för \code{groupBy}.

\Subtask Inför värdet \code{lazy val wordFreq2}. Den ska ge samma resultat som \code{wordFreq} men men implementeras med hjälp av \code{groupBy} och \code{map} i stället för \code{FreqMapBuilder}.

\Subtask\Uberkurs Jämför prestanda mellan \code{wordFreq2} och \code{wordFreq}. Vilken är snabbast för stora texter? Är skillnaden stor?

\Subtask Inför värdet \code{lazy val followsFreq2}. Den ska ge samma resultat som \code{followsFreq} men implementeras med hjälp av \code{groupBy} och \code{map} i stället för \code{FreqMapBuilder}.
Denna uppgift är ganska knepig. Experimentera dig fram i REPL, och bygg upp en lösning steg för steg. \emph{Tips:}
\begin{Code}
bigrams
  .groupBy(???)
  .map(p => p._1 -> p._2.map(???).groupBy(???).map(???))
\end{Code}

\Subtask\Uberkurs Jämför prestanda mellan \code{followsFreq2} och \code{followsFreq}. Vilken är snabbast för stora texter? Är skillnaden stor?

\Task \emph{Gör \code{FreqMapBuilder} generisk.} Generiska strukturer, alltså sådana som har typparametrar, är ofta väsentligt mycket mer användbara. Om du gör \code{FreqMapBuilder} generisk genom att införa en typparameter i stället för att hårdkoda typen till \code{String} så kan du använda \code{FreqMapBuilder} med godtycklig elementtyp. 

\Subtask Studera \code{FreqMapBuilder} och identifiera allt i den klassen som är specifikt för typen \code{String}.

\Subtask Inför en typparameter \code{A} inom hakparenteser efter klassnamnet och använd sedan \code{A} i stället för \code{String} i alla metoder.

\Subtask Testa så att din generiska frekvenstabellbyggare fungerar på sekvenser som innehåller annat än strängar.

Detta funkar eftersom inget i \code{FreqMapBuilder} egentligen förutsätter att elementen som ska räknas är av sträng-typ (det räcker att det finns en vettig \code{equals} och \code{hashcode}).


%!TEX encoding = UTF-8 Unicode

%!TEX root = ../compendium1.tex

%!TEX encoding = UTF-8 Unicode
\chapter{Sökning, Sortering}\label{chapter:W10}
Begrepp du ska lära dig denna vecka:
\begin{itemize}[noitemsep,label={$\square$},leftmargin=*]
\item compareTo på strängar
\item trait Ordered[T]
\item algoritm: LINEAR-SEARCH
\item algortim: BINARY-SEARCH
\item algoritmisk komplexitet
\item sortering till ny vektor
\item sortering på plats
\item algoritm: INSERTION-SORT
\item algoritm: SELECTION-SORT\end{itemize}

\clearpage\section{Teori}
\input{../slides/body/lect-w09-extends.tex}
\input{../slides/body/lect-w09-override.tex}
%%!TEX encoding = UTF-8 Unicode
\chapter{Sökning, Sortering}\label{chapter:W10}
Begrepp du ska lära dig denna vecka:
\begin{itemize}[noitemsep,label={$\square$},leftmargin=*]
\item compareTo på strängar
\item trait Ordered[T]
\item algoritm: LINEAR-SEARCH
\item algortim: BINARY-SEARCH
\item algoritmisk komplexitet
\item sortering till ny vektor
\item sortering på plats
\item algoritm: INSERTION-SORT
\item algoritm: SELECTION-SORT\end{itemize}


%!TEX encoding = UTF-8 Unicode
%!TEX root = ../exercises.tex

\ifPreSolution

\Exercise{\ExeWeekTEN}\label{exe:W10}

\begin{Goals}
\input{modules/w10-inheritance-exercise-goals.tex}
\end{Goals}

\begin{Preparations}
\item \StudyTheory{10}
\end{Preparations}

\BasicTasks

\else

\ExerciseSolution{\ExeWeekTEN}

\BasicTasks

\fi



\WHAT{Para ihop begrepp med beskrivning.}

\QUESTBEGIN

\Task \what

\vspace{1em}\noindent Koppla varje begrepp med den (förenklade) beskrivning som passar bäst:

\begin{ConceptConnections}
  bastyp & 1 & & A & har supertypen \code|AnyRef|, allokeras i heapen via referens \\ 
  supertyp & 2 & & B & kan ha många former, t.ex. en av flera subtyper \\ 
  subtyp & 3 & & C & klass utan namn, utvidgad med extra implementation \\ 
  körtidstyp & 4 & & D & en typ som är mer specifik \\ 
  dynamisk bindning & 5 & & E & kan ha parametrar, kan ej instansieras, kan ej mixas in \\ 
  polymorfism & 6 & & F & saknar implementation \\ 
  trait & 7 & & G & har supertypen \code|AnyVal|, lagras direkt på stacken \\ 
  inmixning & 8 & & H & tillföra egenskaper med \code|with| och en trait \\ 
  överskuggad medlem & 9 & & I & är abstrakt, kan mixas in, kan ha parametrar \\ 
  anonym klass & 10 & & J & kan vara mer specifik än den statiska typen \\ 
  skyddad medlem & 11 & & K & är endast synlig i subtyper \\ 
  abstrakt medlem & 12 & & L & körtidstypen avgör vilken metod som körs \\ 
  abstrakt klass & 13 & & M & medlem i subtyp ersätter medlem i supertyp \\ 
  förseglad typ & 14 & & N & den mest generella typen i en arvshierarki \\ 
  referenstyp & 15 & & O & en typ som är mer generell \\ 
  värdetyp & 16 & & P & subtypning utanför denna kodfil är förhindrad \\ 
\end{ConceptConnections}

\SOLUTION

\TaskSolved \what

\begin{ConceptConnections}
  bastyp & 1 & ~~\Large$\leadsto$~~ &  N & den mest generella typen i en arvshierarki \\ 
  supertyp & 2 & ~~\Large$\leadsto$~~ &  O & en typ som är mer generell \\ 
  subtyp & 3 & ~~\Large$\leadsto$~~ &  D & en typ som är mer specifik \\ 
  körtidstyp & 4 & ~~\Large$\leadsto$~~ &  J & kan vara mer specifik än den statiska typen \\ 
  dynamisk bindning & 5 & ~~\Large$\leadsto$~~ &  L & körtidstypen avgör vilken metod som körs \\ 
  polymorfism & 6 & ~~\Large$\leadsto$~~ &  B & kan ha många former, t.ex. en av flera subtyper \\ 
  trait & 7 & ~~\Large$\leadsto$~~ &  I & är abstrakt, kan mixas in, kan ha parametrar \\ 
  inmixning & 8 & ~~\Large$\leadsto$~~ &  H & tillföra egenskaper med \code|with| och en trait \\ 
  överskuggad medlem & 9 & ~~\Large$\leadsto$~~ &  M & medlem i subtyp ersätter medlem i supertyp \\ 
  anonym klass & 10 & ~~\Large$\leadsto$~~ &  C & klass utan namn, utvidgad med extra implementation \\ 
  skyddad medlem & 11 & ~~\Large$\leadsto$~~ &  K & är endast synlig i subtyper \\ 
  abstrakt medlem & 12 & ~~\Large$\leadsto$~~ &  F & saknar implementation \\ 
  abstrakt klass & 13 & ~~\Large$\leadsto$~~ &  E & kan ha parametrar, kan ej instansieras, kan ej mixas in \\ 
  förseglad typ & 14 & ~~\Large$\leadsto$~~ &  P & subtypning utanför denna kodfil är förhindrad \\ 
  referenstyp & 15 & ~~\Large$\leadsto$~~ &  A & har supertypen \code|AnyRef|, allokeras i heapen via referens \\ 
  värdetyp & 16 & ~~\Large$\leadsto$~~ &  G & har supertypen \code|AnyVal|, lagras direkt på stacken \\ 
\end{ConceptConnections}

\QUESTEND





\WHAT{Gemensam bastyp.}

\QUESTBEGIN

\Task  \what~  Man vill ofta lägga in objekt av olika typ i samma samling.
\begin{REPL}
scala> class Gurka(val vikt: Int)
scala> class Tomat(val vikt: Int)
scala> val gurkor = Vector(Gurka(100), Gurka(200))
scala> val grönsaker = Vector(Gurka(300), Tomat(42))
\end{REPL}

\Subtask Om en samling innehåller objekt av flera olika typer försöker kompilatorn härleda den mest specifika typen som objekten har gemensamt. Vad blir det för typ på värdet \code{grönsaker} ovan?

\Subtask Försök ta reda på summan av vikterna enligt nedan. Vad ger andra raden för felmeddelande? Varför?

\begin{REPL}
scala> gurkor.map(_.vikt).sum     // fungerar
scala> grönsaker.map(_.vikt).sum  // fungerar inte
\end{REPL}

\Subtask Du ska nu göra så att du kan komma åt vikten på alla grönsaker genom att ge gurkor och tomater en gemensam bastyp som de olika konkreta grönsakstyperna utvidgar med nyckelordet \code{extends}. Det heter att subtyperna \code{Gurka} och \code{Tomat} \textbf{ärver} egenskaperna hos supertypen \code{Grönsak}.

Skapa en bastyp \code{Grönsak} med ett abstrakt attribut \code{vikt}. Låt sedan de konkreta grönsakerna ärva bastypen:

\begin{REPL}
scala> trait Grönsak { val vikt: Int }
scala> class Gurka(val vikt: Int) extends Grönsak
scala> class Tomat(val vikt: Int) extends Grönsak
scala> val gurkor = Vector(Gurka(100), Gurka(200))
scala> val grönsaker = Vector(Gurka(300), Tomat(42))
\end{REPL}
När sker initialisering av attributet \code{vikt}?

\Subtask Vad blir det nu för typ på variabeln \code{grönsaker} ovan?

\Subtask Går det nu att summera av vikterna i \code{grönsaker} med uttrycket nedan? Varför?\\ \code{grönsaker.map(_.vikt).sum}


\Subtask En trait liknar en klass, men man kan inte instansiera den direkt. Vad blir det för felmeddelande om du försöker skapa en instans av en trait enligt nedan?
\begin{REPL}
scala> trait Grönsak { val vikt: Int }
scala> new Grönsak
\end{REPL}


\Subtask Traiten \code{Grönsak} har en abstrakt medlem \code{vikt}. Den sägs vara abstrakt eftersom den saknar definition -- medlemmen har bara ett namn och en typ men inget värde. Du kan instansiera den abstrakta traiten \code{Grönsak} om du fyller i det som ''fattas'', nämligen ett värde på \code{vikt}. Man kan fylla på det som fattas i genom att ''hänga på'' ett block efter typens namn vid instansiering. Man får då vad som kallas en \textbf{anonym klass}, i detta fall en ganska konstig grönsak som inte är någon speciell sorts grönsak med som ändå har en vikt.

Vad får \code{anonymGrönsak} nedan för typ och strängrepresenation?
\begin{REPL}
scala> val anonymGrönsak = new Grönsak { val vikt = 42 }
\end{REPL}

\Subtask Vad blir felmeddelandet om du skapar en anonym klass \code{Grönsak} med en kropp som saknar definition av vikt?

\SOLUTION


\TaskSolved \what


\SubtaskSolved  \code{Vector[Object]}. Typen \code{Object} i JVM är motsvarar typen \code{AnyRef} som är bastyp för alla referenstyper.

\SubtaskSolved  Felmeddelande:
\begin{REPLnonum}
scala> grönsaker.map(_.vikt).sum  
-- Error:                                                                                 
1 |grönsaker.map(_.vikt).sum
  |              ^^^^^^
  |             value vikt is not a member of Object - did you mean wait?
-- Error:
1 |grönsaker.map(_.vikt).sum
  |                         ^
  |ambiguous implicit arguments: both object DoubleIsFractional in object Numeric and object ShortIsIntegral in object Numeric match type Numeric[B] of parameter num of method sum in trait IterableOnceOps
\end{REPLnonum}
Det första felmeddelandet beror på att vektorns element är av typen \code{Object} och medlemmen \code{vikt} är inte definierat för denna typ. Det andra felmeddelandet är ett följdfel som beror på att en sekvens med element av typen \code{Object} inte kan summeras eftersom kompilatorn inte kan härleda att elementtypen är numerisk.

\SubtaskSolved  Attributet \code{vikt} initialiseras vid konstruktion av \code{Gurka} resp. \code{Tomat}. Värdet ges av resp. klassparameter.

\SubtaskSolved  \code{Vector[Grönsak]}.

\SubtaskSolved  Ja. Eftersom den statiska typen för elementen i sekvensen är \code{Grönsak} (den dynamiska typen kan vara godtycklig subtyp av \code{Grönsak}) och alla instanser av denna typ garanterat har attributet \code{vikt} som är av typen \code{Int} så kan kompilatorn vid \emph{kompileringstid} dra slutsatsen att summeringen är giltig och därmed kan kompilatorn kompilera koden till körbar maskinkod.

\SubtaskSolved  
\begin{REPLnonum}
scala> new Grönsak
-- Error:
1 |new Grönsak
  |    ^^^^^^^
  |    Grönsak is a trait; it cannot be instantiated
\end{REPLnonum}

\SubtaskSolved  
\begin{REPLnonum}
scala> val anonymGrönsak = new Grönsak { val vikt = 42 }
val anonymGrönsak: Grönsak = anon$1@1edde8b6
scala> anonymGrönsak.toString                                                                                      
val res0: String = anon$1@1edde8b6
\end{REPLnonum}
Typen är \code{Grönsak} och blir här en s.k. \emph{anonym klass}, eftersom vi inte har använt en namngiven klass med \code{extends}, utan bara ''hängt på'' en klasskropp inom klammerparenteser direkt vid konstruktion. När du skapar anonyma klasser måste du använda nyckelordet \code{new}.

Kompilatorn hittar på ett unikt klassnamn, här anon\$1, för att hålla reda på den anonyma klassen under kompilering till maskinkod. Strängrepresentationen innehåller ett hexadecimalt heltal som är unikt för instansen, här \code{1edde8b6}.

\SubtaskSolved  

\begin{REPLsmall}
scala> new Grönsak { }
-- Error:
1 |new Grönsak { }
  |^
  |object creation impossible, since val vikt: Int in trait Grönsak is not defined 

\end{REPLsmall}


\QUESTEND






\WHAT{Polymorfism vid arv, s.k. subtypspolymorfism.}

\QUESTBEGIN

\Task  \what~  Polymorfism betyder ''många skepnader''. I samband med arv  innebär det att flera subtyper, till exempel \code{Ko} och \code{Gris}, kan hanteras gemensamt som om de vore instanser av samma supertyp, så som \code{Djur}. Subklasser kan implementera en metod med samma namn på olika sätt. Vilken metod som exekveras bestäms vid körtid beroende på vilken subtyp som instansieras. På så sätt kan djur komma i många skepnader.

\Subtask Implementera funktionen \code{skapaDjur} nedan så att den returnerar antingen en ny \code{Ko} eller en ny \code{Gris} med lika sannolikhet.

\begin{REPL}
scala> trait Djur { def väsnas: Unit }
scala> class Ko   extends Djur { def väsnas = println("Muuuuuuu") }
scala> class Gris extends Djur { def väsnas = println("Nöffnöff") }
scala> def skapaDjur(): Djur = ???
scala> val bondgård = Vector.fill(42)(skapaDjur())
scala> bondgård.foreach(_.väsnas)
\end{REPL}

\Subtask Lägg till ett djur av typen Häst som väsnas på lämpligt sätt och modifiera \code{skapaDjur} så att det skapas kor, grisar och hästar med lika sannolikhet.


\SOLUTION


\TaskSolved \what


\SubtaskSolved
\begin{Code}
def skapaDjur(): Djur = 
  if math.random() > 0.5 then Ko() else Gris()
\end{Code}

\SubtaskSolved
\begin{Code}
class Häst extends Djur: 
  def väsnas = println("Gnääääägg") 

def skapaDjur(): Djur = 
   math.random() match
    case r if r < 0.33 => Ko() 
    case r if r < 0.67 => Gris() 
    case _             => Häst()
\end{Code}


\QUESTEND





\WHAT{Olika typer av heltalspar till laborationen \hyperref[section:lab:\LabWeekTEN]{\texttt{\LabWeekTEN}}.}


\QUESTBEGIN


\Task\label{exe:inheritance:labprep-pair}  \what~Under veckans laboration ska du använda olika typer av par som representerar riktning och position på en tvådimensionell spelplan, samt spelplanens storlek. I stället för att använda en vanlig 2-tupel till dessa tre olika typer av par ska du skapa egna, specifika  typer som alla ärver bastypen \code{Pair[T]}. Dessa typer ska alla ligga i filen \code{pairs.scala} i \code{package snake}.
\begin{Code}
// detta är en skiss på filen pairs.scala
package snake

trait Pair[T]:
  def x: T
  def y: T
  // uppgift a) lägg till den konkreta metoden tuple

// efterföljande deluppgifterna implementerar dessa subtyper till Pair:
//   case klass Dim beskriver en 2-dimensionell ytas storlek
//   case klass Pos beskriver en position på en yta av Dim storlek
//   enum Dir beskriver förflyttning mot North, South, East, West
\end{Code}
Skillnaden mellan \code{Pair[T]} och en vanlig 2-tupel är att medlemmarna \code{x} och \code{y} garanterat är av \emph{samma} typ, medan en 2-tupel kan innehålla element av olika typ.

I fig. \ref{snake:fig:pairs-uml} visas en bild av klasshierarkin som du steg-för-steg ska utveckla i efterföljande  uppgifter. Fördelen med att ha olika typer av par är att det är mer typsäkert \Eng{type safe}: vi får hjälp av kompilatorn att upptäcka om vi av misstag förväxlar t.ex. en position med en riktning.

\begin{figure}[H]
\begin{center}
\newcommand{\TextBox}[1]{\raisebox{0pt}[1em][0.5em]{#1}}
\tikzstyle{umlclass}=[rectangle, draw=black,  thick, anchor=north, text width=2cm, rectangle split, rectangle split parts = 3]
\begin{tikzpicture}[inner sep=0.5em,scale=1.2, every node/.style={transform shape}]

  \node [umlclass, rectangle split parts = 1, xshift=0cm, yshift=4.5cm] (BaseType1)  {
              \textit{\textbf{\centerline{\TextBox{\code{Pair[T]}}}}}
%              \nodepart[align=left]{second}\code{def x: T} \newline \code{def y: T}
          };


  \node [umlclass, rectangle split parts = 1, xshift=-3cm, yshift=2.5cm] (SubType1)  {
              \textit{\textbf{\centerline{\TextBox{\code{Dim}}}}}
%              \nodepart[align=left]{second}\code{val x: Int} \newline \code{val y: Int}
          };

\node [umlclass, rectangle split parts = 1, xshift=0cm, yshift=2.5cm] (SubType2)  {
            \textit{\textbf{\centerline{\TextBox{\code{Pos}}}}}
%            \nodepart[]{second}\TextBox{\code{val dim: Int}}
        };

\node [umlclass, rectangle split parts = 1, xshift=3cm, yshift=2.5cm] (SubType3)  {
            \textit{\textbf{\centerline{\TextBox{\code{Dir}}}}}
%            \nodepart[]{second}\TextBox{\code{val dim: Int}}
        };


\draw[umlarrow] (SubType1.north) -- ++(0,0.5) -| (BaseType1.south);
\draw[umlarrow] (SubType2.north) -- ++(0,0.5) -| (BaseType1.south);
\draw[umlarrow] (SubType3.north) -- ++(0,0.5) -| (BaseType1.south);

\end{tikzpicture}
\end{center}
\caption{Arvshierarki med \code{Pair[T]} som bastyp.}
\label{snake:fig:pairs-uml}
\end{figure}

\Subtask Öppna en editor och koda \code{trait Pair[T]} i en fil \code{pairs.scala}. Lägg dessutom till en konkret metod \code{tuple} i \code{Pair[T]} som returnerar en 2-tupel med de båda elementen i paret, så att det vid behov går att omvandla \code{Pair}-instanser till 2-tupler. Använd REPL via \code{sbt console} för att testa att detta fungerar:
\begin{REPLnonum}
scala> val p = new Pair[Int] { override val x = 10; override val y = 20 }
p: Pair[Int]{val x: Int; val y: Int} = $anon$1@784223e9

scala> p.tuple
val res0: (Int, Int) = (10,20)
\end{REPLnonum}

\Subtask Skapa en case-klass \code{Dim} som ärver \code{Pair[Int]}. Instanser av denna klass kommer du att använda under veckans laboration för att representera en spelplans storlek genom att låta \code{x} ange antalet horisontella positioner och \code{y} antalet vertikala positioner.

Lägg även till ett kompanjonsobjekt \code{Dim} med en \code{apply}-metod som kan skapa \code{Dim}-instanser givet en 2-tupel.
Testa i REPL enligt nedan.
\begin{REPLnonum}
scala> Dim(50, 60)
val res1: Dim = Dim(50,60)

scala> Dim((60, 50))
val res2: Dim = Dim(60,50)

scala> res2.tuple
val res3: (Int, Int) = (60,50)
\end{REPLnonum}

\Subtask Lägg till en case-klass \code{Pos} som ärver \code{Pair[Int]} som representerar en position med en \code{x}-koordinat och en \code{y}-koordinat, båda klassparametrar. Kordinaterna ska hållas inom en spelplansstorlek som ges av klassparametern \code{dim} av typen \code{Dim}. Kordinatpositionerna är heltal och räknas från \code{0} till (men inte med) \code{dim.x} resp. \code{dim.y}.

Gör primärkonstruktorn i case-klassen \code{Pos} \textbf{privat}, genom att skriva nyckelordet \code{private} efter klassnamnet men före klassparameterlistan, så att det inte går att skapa instanser via primärkonstruktorn utanför klasskroppen och kompanjonsobjektet. 

Implementera metoderna \code{+} och \code{-} i case-klassen \code{Pos}. Båda metoderna ska ta en parameter \code{p} av typen \code{Pair[Int]} och returnera en ny \code{Pos}, där \code{p.x} resp. \code{p.y} är adderat resp. subtraherat från aktuell position. Observera att du inte ska skriva \code{new} när du skapar en ny instans, eftersom dessa alltid ska skapas via kompanjonsobjektets \code{apply}-metod, som är en ''smart'' fabriksmetod som garanterar håller koordinaterna inom spelplanen. 

Lägg till ett kompanjonsobjekt \code{Pos} med en \code{apply}-metod som skapar en ny \code{Pos}-instans som ser till att koordinaterna alltid är inom \code{dim}. Aritmetiken ska ske modulo storleken \code{dim}, d.v.s en position ska aldrig kunna hamna utanför spelplanen; i stället så börjar man om på andra sidan (se exempel i REPL nedan). \\ \emph{Tips:} Använd  \code|java.lang.Math.floorMod| som hanterar negativa argument så att resultatet blir positivt (till skillnad från modulo-operatorn \%).

Lägg även till fabriksmetoden \code{random} som kan skapa nya slumpmässiga positioner inom \code{dim}. \emph{Tips:} Använd \code{scala.util.Random.nextInt}.

Testa att det fungerar enligt nedan:
\begin{REPLnonum}
scala> Pos(-1,20,Dim(10,20))
val res4: Pos = Pos(9,0,Dim(10,20))

scala> new Pos(-1,20,Dim(10,20))  // förbjuds med privat primärkonstruktor
-- Error:
1 |new Pos(-1,20,Dim(10,20))
  |    ^^^
  |constructor Pos cannot be accessed as a member of Pos

scala> Pos(0,0,Dim(5,5)) + Pos(6,12, Dim(5,5))                                                                     
val res5: Pos = Pos(1,2,Dim(5,5))

scala> Pos(0,0,Dim(5,5)) - Pos(1,2, Dim(5,5))                                                                     
val res6: Pos = Pos(4,3,Dim(5,5))

scala> for (_ <- 1 to 3) yield Pos.random(Dim(10,10))
val res7: IndexedSeq[Pos] = 
  Vector(Pos(8,8,Dim(10,10)), Pos(2,6,Dim(10,10)), Pos(3,7,Dim(10,10)))
\end{REPLnonum}

\Subtask Vad händer om du glömmer skriva \code{new} när du anropar den privata konstruktorn i din \code{apply}-metod? Varför finns inte detta problem i \code{apply}-metoden för \code{Dim}?

\Subtask Lägg till en \code{enum Dir} som ärver \code{Pair[Int]} och har två \code{val}-parametrar \code{x} och \code{y}. Lägg också till fyra fall med \code{case} som alla ärver \code{Dir} och som representerar en enstegsförflyttning i de fyra väderstrecken, genom att ge parametrarna \code{x} resp. \code{y} något av värden $1$, $-1$ eller $0$. Norrut ska anges med x-koordinaten $-1$ och y-koordinaten $0$, etc. Verifiera i REPL att enumerationen fungerar.

Lägg till en \code{export} som gör så att det räcker att importera \code{snake.*} för att få alla fyra riktningar synliga direkt (annars behövs även import av \code{Dir.*} på alla ställen där riktning används i och utanför paketet \code{snake})


\SOLUTION


\TaskSolved \what

\SubtaskSolved
\begin{CodeSmall}
trait Pair[T]:
  def x: T
  def y: T
  def tuple: (T, T) = (x, y)

\end{CodeSmall}

\SubtaskSolved
\begin{CodeSmall}
case class Dim(x: Int, y: Int) extends Pair[Int]
object Dim:
  def apply(dim: (Int, Int)): Dim = Dim(dim._1, dim._2)  
\end{CodeSmall}

\SubtaskSolved
\begin{CodeSmall}
case class Pos private (x: Int, y: Int, dim: Dim) extends Pair[Int]:
  def +(p: Pair[Int]): Pos = Pos(x + p.x, y + p.y, dim)
  def -(p: Pair[Int]): Pos = Pos(x - p.x, y - p.y, dim)

object Pos:
  def apply(x: Int, y: Int, dim: Dim): Pos = 
    import java.lang.Math.floorMod as mod
    new Pos(mod(x, dim.x), mod(y, dim.y), dim) //OBS: new nödvändig här!

  def random(dim: Dim): Pos = 
    import scala.util.Random.nextInt as rni
    Pos(rni(dim.x), rni(dim.y), dim)
\end{CodeSmall}

\SubtaskSolved Om du glömmer skriva \code{new} explicit i kompanjonsobjektets \code{apply}-metod så blir det ett rekursivt anrop som resulterar i en oändlig loop vid körtid. Med \code{new} så är det garanterat den privata primärkonstruktorn för \code{Pos} som anropas. 

I \code{Dim.apply} så skiljer sig parametertyperna åt mellan fabriksmetoden och primärkonstruktorn och kompilatorn väljer då primärkonstruktorn eftersom den passar med de givna två separata heltalen och inte med en 2-tupel.

\SubtaskSolved
\begin{CodeSmall}
enum Dir(val x: Int, val y: Int) extends Pair[Int]:
  case North extends Dir( 0, -1)
  case South extends Dir( 0,  1)
  case East  extends Dir( 1,  0)
  case West  extends Dir(-1,  0)
export Dir.*  // gör så att North etc blir synliga i paketet snake
\end{CodeSmall}

\QUESTEND






\WHAT{Supertyp med parameter.}

\QUESTBEGIN

\Task  \what~  Utbildningsdepartementet vill med sitt nya datasystem hålla koll på vissa personer och skapar därför en klasshierarki enligt nedan. Skriv in koden i en editor och testa i REPL med \code{sbt}.
\begin{Code}
class Person(val namn: String)

class Akademiker(
  namn: String,
  val universitet: String) extends Person(namn)

class Student(
  namn: String,
  universitet: String,
  program: String) extends Akademiker(namn, universitet)

class Forskare(
  namn: String,
  universitet: String,
  titel: String) extends Akademiker(namn, universitet)
\end{Code}


\Subtask Deklarera fyra olika \code{val}-variabler med lämpliga namn som refererar till olika instanser av alla olika klasser ovan och ge attributen valfria initialvärden genom olika parametrar till konstruktorerna.

\Subtask Skriv två satser: en som först stoppar in instanserna i en \code{Vector} och en som sedan loopar igenom vektorn och skriv ut alla instansers \code{toString} och \code{namn}.

\Subtask Utbildningsdepartementet vill att det inte ska gå att instansiera objekt av typerna \code{Person} och \code{Akademiker}. Det kan åstadkommas genom att placera nyckelordet \code{abstract} före \code{class}. Uppdatera koden i enlighet med detta. Vilket blir felmeddelande om man försöker instansiera en \code{abstract class}? Går det lika bra med en \code{trait}?

\Subtask Utbildningsdeparetementet vill slippa implementera \code{toString}. Gör därför om typerna \code{Student} och \code{Forskare} till case-klasser. \emph{Tips:} För att undkomma ett kompileringsfel (vilket?) behöver du använda \code{override val} på lämpligt ställe.
Skapa instanser av de nya case-klasserna \code{Student} och \code{Forskare} och skriv ut deras \code{toString}. 

\Subtask 
%Eftersom \code{Person} och \code{Akademiker} nu är abstrakta, vill utbildningsdepartementet att du gör om dessa typer till traits med abstrakta attribut istället för klasser. 
Använd abstrakta attribut i stället för parametrar för typerna som är abstrakta, så att du inte behöver skriva \code{override val} i klassparametrarna till de konkreta case-klasserna.
Du ska också införa en case-klass \code{IckeAkademiker} som ska användas i olika statistiska medborgarundersökningar.
Dessutom förser man alla personer med ett personnummer representerat som en \code{Long}.
Hur ser utbildningsdepartementets kod ut nu, efter alla ändringar? Skriv ett testprogram som skapar några instanser och skriver ut deras attribut.

\SOLUTION


\TaskSolved \what


\SubtaskSolved
\begin{Code}
val person = new Person("Person1")
val akademiker = new Akademiker("Person2", "LTH")
val student = new Student("Person3", "LTH", "D")
val forskare = new Forskare("Person4", "LTH", "Doktorand")
\end{Code}

\SubtaskSolved
\begin{Code}
val vec = Vector(person, akademiker, student, forskare)
for(i <- vec){ print(i.toString + i.namn) }
\end{Code}

\SubtaskSolved  
Felmeddelande vid instansiering av \code{abstract class Akademiker}:\\
\texttt{Akademiker is abstract; it cannot be instantiated}

Det går \emph{inte} lika bra med en \code{trait} i det speciella fallet \code{Akademiker}, eftersom en trait inte får skicka vidare parametrar till en supertyp. Felmeddelande:\\
\texttt{trait Akademiker may not call constructor of trait Person}
\begin{Code}
trait Person(val namn: String)

abstract class Akademiker(
  namn: String,
  val universitet: String) extends Person(namn)

class Student(
  namn: String,
  universitet: String,
  program: String) extends Akademiker(namn, universitet)

class Forskare(
  namn: String,
  universitet: String,
  titel: String) extends Akademiker(namn, universitet)
\end{Code}



\SubtaskSolved  
\begin{REPLnonum}
scala>  
     |trait Person(val namn: String)                                                                              
     | 
     | abstract class Akademiker(
     |   namn: String,
     |   val universitet: String) extends Person(namn)
     | 
     | case class Student(
     |   namn: String,
     |   universitet: String,
     |   program: String) extends Akademiker(namn, universitet)
     | 
     | case class Forskare(
     |   namn: String,
     |   universitet: String,
     |   titel: String) extends Akademiker(namn, universitet)
-- Error:     
8 |  namn: String,
  |  ^
  |  error overriding value namn in trait Person of type String;
  |    value namn of type String needs `override` modifier
-- Error:
9 |  universitet: String,
  |  ^
  |  error overriding value universitet in class Akademiker of type String;
  |    value universitet of type String needs `override` modifier
-- Error:
13 |  namn: String,
   |  ^
   |  error overriding value namn in trait Person of type String;
   |    value namn of type String needs `override` modifier
-- Error:
14 |  universitet: String,
   |  ^
   |  error overriding value universitet in class Akademiker of type String;
   |    value universitet of type String needs `override` modifier
\end{REPLnonum}

\begin{Code}
trait Person(val namn: String)

abstract class Akademiker(
  namn: String,
  val universitet: String) extends Person(namn)

case class Student(
  override val namn: String,
  override val universitet: String,
  program: String) extends Akademiker(namn, universitet)

case class Forskare(
  override val namn: String,
  override val universitet: String,
  titel: String) extends Akademiker(namn, universitet)
\end{Code}

\begin{REPLsmall}
scala> val ps = Vector(Student("Kim", "Lund", "D"), Forskare("Herz", "Lund", "Dr"))
val ps: Vector[Akademiker] = Vector(Student(Kim,Lund,D), Forskare(Herz,Lund,Dr))
scala> ps :+ new Person("Abstrakt") {}
val res0: Vector[Person] = 
  Vector(Student(Kim,Lund,D), Forskare(Herz,Lund,Dr), anon1@1941bbf3)
\end{REPLsmall}

\SubtaskSolved
\begin{Code}
trait Person: 
  val namn: String 
  val nbr: Long

trait Akademiker extends Person:
  val universitet: String

case class Student(
  namn: String,
  nbr: Long,
  universitet: String,
  program: String) extends Akademiker

case class Forskare(
  namn: String,
  nbr: Long,
  universitet: String,
  titel: String) extends Akademiker

case class IckeAkademiker(
    namn: String,
    nbr: Long) extends Person
\end{Code}



\QUESTEND




%\clearpage




\ExtraTasks %%%%%%%%%%%%%%%%%





%\WHAT{Uppräknade värden.}

%\QUESTBEGIN

% \Task  \what~  Ett sätt att hålla reda på uppräknade värden, t.ex. färgen på olika kort i en kortlek, är att använda olika heltal som får representera de olika värdena, till exempel så här:\footnote{Om namnkonventioner för konstanter i Scala: läs under rubriken ''Constants, Values, Variable and Methods'' här \href{http://docs.scala-lang.org/style/naming-conventions.html}{docs.scala-lang.org/style/naming-conventions.html}}
% \begin{Code}
% object Färg {
%   val Spader = 1
%   val Hjärter = 2
%   val Ruter = 3
%   val Klöver = 4
% }
% \end{Code}
% Dessa kan sedan användas som parametrar till denna case-klass vid skapande av kortobjekt:
% \begin{lstlisting}[language=,keywords={case,class}]
% case class Kort(färg: Int, valör: Int)
% \end{lstlisting}
% Man kan hålla reda på färgen med en variabel av typen \code{Int} och tilldela den en viss färg med ovan konstanter. Och när du skapar ett kort kan du använda färgnamnet och du slipper därmed att behöva komma ihåg vilket heltal som representerar färgen.
% \begin{REPL}
% scala> val f = Färg.Spader
% scala> import Färg._
% scala> Kort(Ruter, 7)
% \end{REPL}
% En annan fördelen med detta är att man lätt kan iterera över alla färger:
% \begin{REPL}
% scala> val kortlek = for (f <- 1 to 4; v <- 1 to 13) yield Kort(f, v)
% \end{REPL}
% Men den stora nackdelen med detta är att kompilatorn vid kompileringstid inte kollar om variablerna av misstag råkar ges något värde utanför det giltiga intervallet, eftersom alla heltal är möjliga. Detta får vi själv hålla koll på vid körtid, till exempel med funktionen \code{require} eller \code{if}-satser, etc.

% Istället kan man använda uppräknade värden med hjälp av case-objekt enligt nedan deluppgifter och därmed få hjälp av kompilatorn att hitta eventuella fel vid kompileringstid.  Ett case-objekt är som ett vanligt singelton-objekt men det får bl.a. automatiskt en \code{toString} som är samma som namnet. Case-objekt kan dessutom användas som värden i mönstermatchningar (mer om detta i kapitel \ref{chapter:W10}).

% \Subtask Deklarera följande uppräknade värden som singelton-objekt med gemensam bastyp. Med nyckelordet \code{sealed} så ''förseglas'' klassen och inga andra direkta subtyper tillåts förutom de som finns i samma kod-fil eller block. I detta exempel  med kortfärger vet vi ju att det inte finns fler än dessa fyra färger.
% \begin{Code}
% sealed trait Färg
% case object Spader extends Färg
% case object Hjärter extends Färg
% case object Ruter extends Färg
% case object Klöver extends Färg
% \end{Code}
% Dessa kan sedan användas som parametrar till denna case-klass vid skapande av kortobjekt:
% \begin{lstlisting}[language=,keywords={case,class}]
% case class Kort(färg: Färg, valör: Int)
% \end{lstlisting}
% Skapa därefter några exempelinstanser av klassen \code{Kort}. Vad är fördelen med ovan angreppssätt jämfört med att använda heltalskonstanter?

% \Subtask Om man vill kunna iterera över alla värden är det bra om de finns i en samling med alla värden. Vi kan lägga en sådan i ett kompanjonsobjekt till bastypen enligt nedan. Skriv ut alla färgvärden med en \code{for}-sats.

% \begin{Code}
% sealed trait Färg
% object Färg {
%   val values = Vector(Spader, Hjärter, Ruter, Klöver)
% }
% case object Spader extends Färg
% case object Hjärter extends Färg
% case object Ruter extends Färg
% case object Klöver extends Färg
% \end{Code}
% Skapa en kortlek med 52 olika kort och blanda den sedan med \code{Random.shuffle} enligt nedan. Använd en dubbel \code{for}-sats och \code{yield}.
% \begin{REPL}
% scala> val kortlek: Vector[Kort] = ???
% scala> val blandad = scala.util.Random.shuffle(kortlek)
% \end{REPL}

% \Subtask Skriv en funktion \code{ def blandadKortlek: Vector[Kort] = ???} som ger en ny blandad kortlek varje gång den anropas med metoden i föregående uppgift.

% \Subtask Om man även vill ha en heltalsrepresentation med en medlem \code{toInt} för alla värden, kan man ge bastypen en parameter och i stället för en trait (som inte kan ha några parametrar) använda en abstrakt klass.

% \begin{Code}
% sealed abstract class Färg(final val toInt: Int)
% object Färg {
%   val values = Vector(Spader, Hjärter, Ruter, Klöver)
% }
% case object Spader  extends Färg(0)
% case object Hjärter extends Färg(1)
% case object Ruter   extends Färg(2)
% case object Klöver  extends Färg(3)
% \end{Code}
% Skapa en funktion \code{färgPoäng} som räknar ut summan av heltalsrepresentationen av alla färger hos en vektor med kort, och använd den sedan för att räkna ut \code{färgPoäng} för de första fem korten.
% \begin{REPL}
% scala> def blandadKortlek: Vector[Kort] = ???
% scala> def färgPoäng(xs: Vector[Kort]): Int = ???
% scala> färgPoäng(blandadKortlek.take(5))
% \end{REPL}


% \SOLUTION

% \TaskSolved \what

% \SubtaskSolved  Sättet är säkrare då man inte kan tilldela korten en färg som inte finns. Med heltalskonstanterna kan man till exempel ge ett kort färgen 5, vilken inte korresponderar till någon riktig färg.

% \SubtaskSolved  \code{for (f <- Färg.values; v <- 1 to 13) yield Kort(f,v)}

% \SubtaskSolved
% \begin{Code}
% def blandadKortlek: Vector[Kort] = {
%   val kortlek =
%     for (f <- Färg.values; v <- 1 to 13) yield Kort(f,v)
%   scala.util.Random.shuffle(kortlek)
% }
% \end{Code}

% \SubtaskSolved  \code{def färgPoäng(xs: Vector[Kort]): Int = xs.map(_.färg.toInt).sum}

% \QUESTEND







\WHAT{Bastypen \code{Shape} och subtyperna \code{Rectangle} och \code{Circle}.}

\QUESTBEGIN

\Task  \what~  Du ska i denna uppgift skapa ett litet bibliotek för geometriska former med oföränderliga objekt implementerade med hjälp av case-klasser. De geometriska formerna har en gemensam bastyp kallad \code{Shape}. Utgå från koden nedan.

\begin{CodeSmall}
case class Point(x: Double, y: Double):
  def move(dx: Double, dy: Double): Point = Point(x + dx, y + dy)

trait Shape:
  def pos: Point
  def move(dx: Double, dy: Double): Shape

case class Rectangle(pos: Point, width: Double, height: Double) extends Shape:
  def move(dx: Double, dy: Double): Rectangle = copy(pos = pos.move(dx, dy))

case class Circle(pos: Point, radius: Double) extends Shape:
  def move(dx: Double, dy: Double): Circle = copy(pos = pos.move(dx, dy))

\end{CodeSmall}

\Subtask Instansiera några cirklar och rektanglar och gör några relativa förflyttningar av dina instanser genom att anropa \code{move}.

\Subtask Lägg till en konkret metod \code{moveTo} i \code{Point} som gör en absolut förflyttning till koordinaterna \code{x} och \code{y}. Lägg till en abstrakt metod \code{moveTo} \code{Shape} som implementeras i subklasserna. Testa med REPL på några instanser av \code{Rectangle} och \code{Circle}.

\Subtask Lägg till metoden \code{distanceTo(that: Point): Double } i case-klassen \code{Point} som räknar ut avståndet till en annan punkt med hjälp av \code{math.hypot}. Klistra in i REPL och testa på några instanser av \code{Point}.

\Subtask Lägg till en konkret metod \code{distanceTo(that: Shape): Double } i traiten \code{Shape} som räknar ut avståndet till positionen för en annan Shape. Testa i REPL på några instanser av \code{Rectangle} och \code{Circle}.

\Subtask Gör så att \code{distanceTo} kan anropas med operatornotation.

\SOLUTION


\TaskSolved \what


\SubtaskSolved
\begin{CodeSmall}
val c1 = Circle(Point(1, 1), 42)
val r1 = Rectangle(Point(3, 3), 20, 30)
c1.move(2, 3)
r1.move(3, 2)
\end{CodeSmall}

\SubtaskSolved  
\begin{CodeSmall}
case class Point(x: Double, y: Double):
  def move(dx: Double, dy: Double): Point = Point(x + dx, y + dy)
  def moveTo(x: Double, y: Double): Point = Point(x, y)

trait Shape:
  def pos: Point
  def move(dx: Double, dy: Double): Shape
  def moveTo(x: Double, y: Double): Shape

case class Rectangle(pos: Point, width: Double, height: Double) extends Shape:
  def move(dx: Double, dy: Double): Shape = copy(pos = pos.move(dx, dy))
  def moveTo(x: Double, y: Double): Shape = copy(pos.moveTo(x, y))

case class Circle(pos: Point, radius: Double) extends Shape:
  def move(dx: Double, dy: Double): Shape = copy(pos = pos.move(dx, dy))
  def moveTo(x: Double, y: Double): Shape = copy(pos.moveTo(x, y))
\end{CodeSmall}


\SubtaskSolved \code{def distanceTo(that: Point): Double = math.hypot(that.x - x, that.y - y)}

\SubtaskSolved \code{def distanceTo(that: Shape): Double = pos.distanceTo(that.pos)}.

\SubtaskSolved  
\begin{CodeSmall}
case class Point(x: Double, y: Double):
  def move(dx: Double, dy: Double): Point = Point(x + dx, y + dy)
  def moveTo(x: Double, y: Double): Point = Point(x, y)
  infix def distanceTo(that: Point): Double = math.hypot(that.x - x, that.y - y)

trait Shape:
  def pos: Point
  def move(dx: Double, dy: Double): Shape
  def moveTo(x: Double, y: Double): Shape
  infix def distanceTo(that: Shape): Double = pos.distanceTo(that.pos)

case class Rectangle(pos: Point, width: Double, height: Double) extends Shape:
  def move(dx: Double, dy: Double): Shape = copy(pos = pos.move(dx, dy))
  def moveTo(x: Double, y: Double): Shape = copy(pos.moveTo(x, y))

case class Circle(pos: Point, radius: Double) extends Shape:
  def move(dx: Double, dy: Double): Shape = copy(pos = pos.move(dx, dy))
  def moveTo(x: Double, y: Double): Shape = copy(pos.moveTo(x, y))
\end{CodeSmall}

\QUESTEND






% \WHAT{Regler för \code{override}, \code{private} och \code{final}.}

% \QUESTBEGIN

% \Task  \what~

% \Subtask \label{subtask:overriderules} Undersök överskuggningning av abstrakta, konkreta, privata och finala medlemmar genom att skriva in raderna nedan en i taget i REPL. Vilka rader ger felmeddelande? Varför? Vid felmeddelande: notera hur felmeddelandet lyder och ändra deklarationen av den felande medlemmen så att koden blir kompilerbar (eller om det är enda rimliga lösningen: ta bort den felaktiga medlemmen), innan du provar efterkommande rad.

% \begin{REPL}
% trait Super1 { def a: Int; def b = 42; private def c = "hemlis" }
% class Sub2 extends Super1 { def a = 43; def b = 43; def c = 43 }
% class Sub3 extends Super1 { def a = 43; override def b = 43 }
% class Sub4 extends Super1 { def a = 43; override def c = "43" }
% trait Super5 { final def a: Int; final def b = 42 }
% class Sub6 extends Super5 { override def a = 43; def b = 43 }
% class Sub7 extends Super5 { def a = 43; override def b = 43 }
% class Sub8 extends Super5 { def a = 43; override def c = "43" }
% trait Super9 { val a: Int; val b = 42; lazy val c: String = "lazy" }
% class Sub10 extends Super9 { override def a = 43; override val b = 43 }
% class Sub11 extends Super9 { val a = 43; override lazy val b = 43 }
% class Sub12 extends Super9 { val a = 43; override var b = 43 }
% class Sub13 extends Super9 { val a = 43; override lazy val c = "still lazy" }
% class SubSub extends Sub13 { override val a = 44}
% trait Super14 { var a: Int; var b = 42; var c: String }
% class Sub15 extends Super14 { def a = 43; override var b = 43; val c = "?" }
% \end{REPL}

% \Subtask Skapa instanser av klasserna \code{Sub3}, \code{Sub13} och \code{SubSub} från ovan deluppgift och undersök alla medlemmarnas värden för respektive instans. Förklara varför de har dessa värden.

% %\Subtask Läs igenom reglerna i kapitel  \ref{slideW07:overriderules} om vad som gäller vid arv och överskuggning av medlemmar. Gör några egna undersökningar i REPL som försöker bryta mot någon regel som inte testades i deluppgift \ref{subtask:overriderules}.

% \SOLUTION


% \TaskSolved \what


% \SubtaskSolved  2. Måste ha \code{override} framför \code{b} för att kunna ändra på metoden. \\
% 4. \code{c} är \code{private}, vilket betyder att den är gömd för subklasserna. Därför kan den inte överskuggas. Genom att ta bort \code{override} fungerar klassen. \\
% 5. En \code{final}-medlem måste ha ett bestämt värde. Kan lösas genom att tilldela \code{final a} ett värde eller ta bort \code{final}. \\
% 6. En \code{final}-medlem kan inte överskuggas, varken med eller utan \code{override}. Här får konflikterna tas bort.  \\
% 7. Se 6. \\
% 8. Eftersom \code{c} inte finns i \code{Super5} kan den inte överskuggas. Genom att ta bort \code{override} fungerar klassen. \\
% 10. Överskuggningen av \code{val} måste vara oföränderlig (immutable); detta är inte nödvändigtvis \code{def}. Löses genom att byta ut \code{def a} mot \code{val a} hos \code{Sub10}.  \\
% 11. Samma problem som i 10.; \code{lazy val} kan vara föränderlig. Löses genom att ta bort \code{lazy}. \\
% 12. Samma problem igen! \code{var} är föränderlig, vilket bryter mot typsäkerheten när man försöker överskugga en \code{val}. Löses genom att ändra \code{var} till \code{val}. \\
% 15.\code{def a = 43} och \code{val c = "?"} täcker inte allt som \code{var} kräver. Det behövs en setter för att kunna uppfylla kraven för överskuggning för en \code{var}. Dessutom finns det ingen anledning för en \code{val} att överskuggas; man kan ju ändra på den lite hur man vill!

% \SubtaskSolved  Sub3: a = 43, b = 43 eftersom medlemmen är överskuggad. c hittas inte eftersom den är \code{private}.

% Sub13: a = 43, b = 42, c = "still lazy" eftersom medlemmen överskuggas.

% SubSub: a = 44 eftersom medlemmen överskuggas, b = 42, c = "still lazy".

% \SubtaskSolved  -.


% \QUESTEND





%\clearpage





\AdvancedTasks %%%%%%%%%%%%%%%%%

% \WHAT{Använda \code{trait} eller \code{class}?}

% \QUESTBEGIN

% \Task \what~ I vilka sammanhang är det nödvändigt att använda en \code{trait} respektive en \code{class}? Läs här för fördjupning:\\  \href{http://www.artima.com/pins1ed/traits.html\#12.7}{http://www.artima.com/pins1ed/traits.html\#12.7}.


% \SOLUTION


% \TaskSolved \what~Man måste använda en klass om man behöver klassparametrar. Man måste använda en trait om man vill göra in-mixning med \code{with}. \\

%  \QUESTEND



\WHAT{Inmixning.}

\QUESTBEGIN

\Task \label{task:fyle} \what~   Man kan utvidga en klass med multipla traits med en kommaseparerad lista. På så sätt kan man fördela medlemmar i olika traits och återanvända gemensamma koddelar genom så kallad \textbf{inmixning}, så som nedan exempel visar.

En alternativ fågeltaxonomi, speciellt populär i Skåne, delar in alla fåglar i två specifika kategorier: Kråga respektive Ånka. Krågor kan flyga men inte simma, medan Ånkor kan simma och oftast även flyga. Fågel i generell, kollektiv bemärkelse kallas på gammal skånska för Fyle.%
\footnote{\href{http://www.klangfix.se/ordlista.htm}{www.klangfix.se/ordlista.htm}}

\begin{CodeSmall}
trait Fyle:
  val läte: String
  def väsnas: Unit = print(läte * 2)
  val ärSimkunnig: Boolean
  val ärFlygkunnig: Boolean

trait KanSimma       { val ärSimkunnig = true }
trait KanInteSimma   { val ärSimkunnig = false }
trait KanFlyga       { val ärFlygkunnig = true }
trait KanKanskeFlyga { val ärFlygkunnig = math.random() < 0.8 }

class Kråga extends Fyle, KanFlyga, KanInteSimma:
  val läte = "krax"

class Ånka extends Fyle, KanSimma, KanKanskeFlyga: 
  val läte = "kvack"
  override def väsnas = print(läte * 4)
\end{CodeSmall}

\Subtask En flitig ornitolog hittar 42 fåglar i en perfekt skog där alla fågelsorter är lika sannolika, representerat av vektorn \code{fyle} nedan. Skriv i REPL ett uttryck som undersöker hur många av dessa som är flygkunniga Ånkor, genom att använda metoderna \code{filter}, \code{isInstanceOf}, \code{ärFlygkunnig} och \code{size}.

\begin{REPL}
scala> val fyle =
         Vector.fill(42)(if math.random() > 0.5 then new Kråga else new Ånka)
scala> fyle.foreach(_.väsnas)
scala> val antalFlygånkor: Int = ???
\end{REPL}

\Subtask \label{subtask:fyle:sound} Om alla de fåglar som ornitologen hittade skulle väsnas exakt en gång var, hur många krax och hur många kvack skulle då höras? Använd metoderna \code{filter} och \code{size}, samt predikatet \code{ärSimkunnig} för att beräkna antalet krax respektive kvack.
\begin{REPL}
scala> val antalKrax: Int = ???
scala> val antalKvack: Int = ???
\end{REPL}

\SOLUTION


\TaskSolved \what


\SubtaskSolved
Det finns många olika sätt, några exempellösningar:
\begin{Code}
val antalFlygånkor: Int = 
  fyle.count(f => f.isInstanceOf[Ånka] && f.ärFlygkunnig)
\end{Code}

\begin{Code}
val antalFlygånkor: Int = 
  fyle.filter(f => f.isInstanceOf[Ånka] && f.ärFlygkunnig).size
\end{Code}

\begin{Code}
val antalFlygånkor: Int = 
  fyle.collect{case f: Ånka if f.ärFlygkunnig}.size
\end{Code}

\begin{Code}
val antalFlygånkor: Int = fyle.map(_ match
  case f: Ånka if f.ärFlygkunnig => 1
  case _ => 0
).sum
\end{Code}

\SubtaskSolved
\begin{Code}
val antalKrax: Int = fyle.filter(f => !f.ärSimkunnig).size * 2
val antalKvack: Int = fyle.filter(f => f.ärSimkunnig).size * 4
\end{Code}


\QUESTEND











\WHAT{Finala klasser.}

\QUESTBEGIN

\Task  \what~  Om man vill förhindra att man kan göra \code{extends} på en klass kan man göra den final genom att placera nyckelordet \code{final} före nyckelordet \code{class}.

\Subtask Eftersom klassificeringen av fåglar i uppgiften ovan i antingen Ånkor eller Krågor är fullständig och det inte finns några subtyper till varken Ånkor eller Krågor är det lämpligt att göra dessa finala. Ändra detta i din kod.

\Subtask Testa att ändå försöka göra en subklass \code{Simkråga extends Kråga}. Vad ger kompilatorn för felmeddelande om man försöker utvidga en final klass?


\SOLUTION


\TaskSolved \what


\SubtaskSolved  Sätt \code{final} framför \code{class} i klasserna.

\SubtaskSolved  error: illegal inheritance from final class Kråga.


\QUESTEND






\WHAT{Accessregler vid arv och nyckelordet \code{protected}.}

\QUESTBEGIN

\Task  \what~  Om en medlem i en supertyp är privat så kan man inte komma åt den i en subklass. Ibland vill man att subklassen ska kunna komma åt en medlem även om den ska vara otillgänglig i annan kod.

\begin{Code}
trait Super:
  private val minHemlis = 42
  protected val vårHemlis = 42

class Sub extends Super:
  def avslöja = minHemlis
  def kryptisk = vårHemlis * math.Pi

\end{Code}

\Subtask Vad blir felmeddelandet när klassen \code{Sub} försöker komma åt \code{minHemlis}?

\Subtask Deklarera \code{Sub} på nytt, men nu utan den förbjudna metoden \code{avslöja}. Vad blir felmeddelandet om du försöker komma åt \code{vårHemlis} via en instans av klassen \code{Sub}? Prova till exempel med \code{(new Sub).vårHemlis}

\Subtask Fungerar det att anropa metoden \code{kryptisk} på instanser av klassen \code{Sub}?

\SOLUTION


\TaskSolved \what


\SubtaskSolved  
\begin{REPL}
2 |  def avslöja = minHemlis
  |                ^^^^^^^^^
  |                Not found: minHemlis
\end{REPL}

\SubtaskSolved  
\begin{REPL}
scala> class Sub extends Super:
         def kryptisk = vårHemlis * math.Pi
scala> (new Sub).vårHemlis
-- Error:
1 |(new Sub).vårHemlis
  |^^^^^^^^^^^^^^^^^^^
  |value vårHemlis in trait Super cannot be accessed as a member of Sub.
  | Access to protected value vårHemlis not permitted because enclosing object 
  | is not a subclass of trait Super where target is defined
\end{REPL}

\SubtaskSolved  Ja.


\QUESTEND






\WHAT{Använding av \code{protected}.}

\QUESTBEGIN

\Task  \what~  Den flitige ornitologen från uppgift \ref{task:fyle} ska ringmärka alla 42 fåglar hen hittat i skogen. När hen ändå håller på bestämmer hen att även försöka ta reda på hur mycket oväsen som skapas av respektive fågelsort. För detta ändamål apterar den flitige ornitologen en Linuxdator på allt infångat fyle. Du ska hjälpa ornitologen att skriva programmet.

\Subtask Inför en \code{protected var räknaLäte} i traiten \code{Fyle} och skriv kod på lämpliga ställen för att räkna hur många läten som respektive fågelinstans yttrar.

\Subtask Inför en metod \code{antalLäten} som returnerar antalet krax respektive kvack som en viss fågel yttrat sedan dess skapelse.

\Subtask Varför inte använda \code{private} i stället for \code{protected}?

\Subtask Varför är det bra att göra räknar-variabeln oåtkomlig från ''utsidan''?



\SOLUTION


\TaskSolved \what


\SubtaskSolved  I Fyle:
\begin{Code}
protected var räknaLäte: Int = 0
def väsnas: Unit = { print(läte * 2); räknaLäte += 2 }
\end{Code}

I Ånka: \code| override def väsnas = { print(läte * 4); räknaLäte += 4 }|

\SubtaskSolved  \code{ def antalLäten: Int = räknaLäte }

\SubtaskSolved  Om en klass som representerar en fågel som skulle ge ifrån sig fler/färre läten än en vanlig \code{Fyle}, behöver \code{väsnas} ändras. Denna metod behöver tillgång till \code{räknaLäte}, vilken inte får vara \code{private}.

\SubtaskSolved  Räknar-variabeln ska inte kunna påverkas i någon annan del av programmet.


\QUESTEND





\WHAT{Inmixning av egenskaper.}

\QUESTBEGIN

\Task  \what~ Det visar sig att vår flitige ornitolog från uppgift \ref{task:fyle} på sidan \pageref{task:fyle} sov på en av föreläsningarna i zoologi när hen var nolla på Natfak, och därför helt missat fylekategorin \code{Pjodd}. Hjälp vår stackars ornitolog så att fylehierarkin nu även omfattar Pjoddar. En Pjodd kan flyga som en Kråga men den \code{ÄrLiten} medan en Kråga \code{ÄrStor}. En Pjodd kvittrar dubbelt så många gånger som en Ånka kvackar. En Pjodd \code{KanKanskeSimma} där simkunnighetssannolikheten är $0.2$. Låt ornitologen ånyo finna 42 slumpmässiga fåglar i skogen och filtrera fram lämpliga arter. Undersök sedan hur dessa väsnas, i likhet med deluppgift \ref{task:fyle}\ref{subtask:fyle:sound}.


\SOLUTION

\TaskSolved \what


\begin{Code}
trait Fyle:
  val läte: String
  def väsnas: Unit = { print(läte * 2); räknaLäte += 2 }
  protected var räknaLäte: Int = 0
  val ärSimkunnig: Boolean
  val ärFlygkunnig: Boolean
  val ärStor : Boolean
  def antalLäten: Int = räknaLäte

trait KanSimma { val ärSimkunnig = true }
trait KanInteSimma { val ärSimkunnig = false }
trait KanFlyga { val ärFlygkunnig = true }
trait KanKanskeFlyga { val ärFlygkunnig = math.random() < 0.8 }
trait KanKanskeSimma { val ärSimkunnig = math.random() < 0.2 }
trait ÄrStor { val ärStor = true }
trait ÄrLiten { val ärStor = false }

final class Kråga extends Fyle, KanFlyga, KanInteSimma, ÄrStor:
  val läte = "krax"

final class Ånka extends Fyle, KanSimma, KanKanskeFlyga, ÄrStor:
  val läte = "kvack"
  override def väsnas = { print(läte * 4); räknaLäte += 4 }

final class Pjodd extends Fyle, KanFlyga, KanKanskeSimma, ÄrLiten:
  val läte = "kvitter"
  override def väsnas = { print(läte * 8); räknaLäte += 8 }
\end{Code}

I REPL:
\begin{REPL}
val fyle = Vector.fill(42)(
  if math.random() < 0.33 then Kråga()
  else if math.random() < 0.5 then Ånka()
  else Pjodd()
)
fyle.filter(f => f.isInstanceOf[Kråga]).size * 2
fyle.filter(f => f.isInstanceOf[Ånka]).size * 4
fyle.filter(f => f.isInstanceOf[Pjodd]).size * 8
\end{REPL}

\QUESTEND





% \WHAT{Typtest och typkonvertering.}

% \QUESTBEGIN

% \Task  \what~I Scala kan man testa körtidstyp och samtidigt konvertera till en mer specifik typ på ett typsäkert sätt med hjälp av \emph{mönstermatchning} i \code{match}-uttryck som vi ska se i kommande övning \texttt{\ExeWeekTEN}. För att underlätta interoperabilitet med Java finns  Scala-metoderna \code{isInstanceOf} och \code{asInstanceOf}, som motsvarar hur typtest och typkonvertering görs i Java.\footnote{\code{isInstanceOf} och \code{asInstanceOf} används sällan i Scala eftersom \code{match} är kraftfullare och säkrare.}

% Gör nedan deklarationer.
% \begin{REPL}
% scala> trait A; trait B extends A; class C extends B; class D extends B
% scala> val (c, d) = (new C, new D)
% scala> val a: A = c
% scala> val b: B = d
% \end{REPL}

% \Subtask Rita en bild över vilka typer som ärver vilka.

% \Subtask Vilket resultat ger dessa typtester? Varför?
% \begin{REPL}
% scala> c.isInstanceOf[C]
% scala> c.isInstanceOf[D]
% scala> d.isInstanceOf[B]
% scala> c.isInstanceOf[A]
% scala> b.isInstanceOf[A]
% scala> b.isInstanceOf[D]
% scala> a.isInstanceOf[B]
% scala> c.isInstanceOf[AnyRef]
% scala> c.isInstanceOf[Any]
% scala> c.isInstanceOf[AnyVal]
% scala> c.isInstanceOf[Object]
% scala> 42.isInstanceOf[Object]
% scala> 42.isInstanceOf[Any]
% \end{REPL}

% \Subtask Vilka av dessa typkonverteringar ger felmeddelande? Vilket och varför?
% \begin{REPL}
% scala> c.asInstanceOf[B]
% scala> c.asInstanceOf[A]
% scala> d.asInstanceOf[C]
% scala> a.asInstanceOf[B]
% scala> a.asInstanceOf[C]
% scala> a.asInstanceOf[D]
% scala> a.asInstanceOf[E]
% scala> b.asInstanceOf[A]
% \end{REPL}



% \SOLUTION


% \TaskSolved \what


% \SubtaskSolved  B ärver A. C och D ärver B.

% \SubtaskSolved  1. True eftersom c är av typen C. \\
% 2. False eftersom c inte är av typen D. \\
% 3. True eftersom d är av typen D som är en subtyp av B. \\
% 4. True eftersom c är av typen C som är en subtyp av B, som i sin tur är en subtyp av A. \\
% 5. True eftersom b är av typen D, som är en subtyp av B, som i sin tur är en subtyp av A. \\
% 6. True eftersom b är av typen D. \\
% 7. True eftersom a är av typen C som är en subtyp av B. \\
% 8. True eftersom c är av typen C som är en subtyp av AnyRef. \\
% 9. True eftersom c är av typen C som är en subtyp av Any. \\
% 10. Error eftersom \code{isInstanceOf} inte kan använda sig av \code{AnyVal}.  \\
% 11. True eftersom c är av typen C som är en subtyp av Object (Object är java-representationen av AnyRef). \\
% 12. Error eftersom \code{isInstanceOf} inte kan testa om värdetyper (i detta fallet \code{42}) är referenstyper. \\
% 13. True eftersom \code{42} är av typen \code{Int} som är en subtyp av Any. \\

% \SubtaskSolved  3. Går inte eftersom c inte är av typen D, utan typen C. \\
% 6. Går inte eftersom a inte är av typen D, utan typen C. \\
% 7. Går inte eftersom typen E inte finns. \\


% \QUESTEND













% \WHAT{Saknad referens med \texttt{null} och bottentypen \texttt{Nothing}.}

% \QUESTBEGIN

% \Task  \what~ Hitta på en egen fördjupningsuppgift inspirerat av denna artikel på Stackoverflow: \url{http://stackoverflow.com/questions/16173477/usages-of-null-nothing-unit-in-scala}

% \SOLUTION


% \QUESTEND






\WHAT{Arvshierarki med matematiska tal.}

\QUESTBEGIN

\Task  \what~ Studera den djupa arvshierarkin i paketet \code{numbers} i koden på efterföljande sidor. Paketet  \code{numbers} modellerar olika sorters tal i matematiken, med syftet att erbjuda ett s.k. DSL \footnote{\url{https://en.wikipedia.org/wiki/Domain-specific_language}}, alltså ett specialspråk för en viss applikationsdomän\footnote{\url{https://stackoverflow.com/questions/49216312/what-is-dsl-in-scala}}, här: domänen matematiska tal.

Du kan ladda ner koden för \code{numbers} här: \\
\href{https://github.com/lunduniversity/introprog/blob/master/compendium/examples/numbers.scala}{github.com/lunduniversity/introprog/blob/master/compendium/examples/numbers.scala}
\\ Notera speciellt metoden \code{reduce} som reducerar ett tal till sin enklaste form. Metoden \code{reduce} överskuggas på lämpliga ställen med relevant reduktion.

\Subtask Rita en bild över typhierarkin, t.ex. som ett upp-och-nedvänt träd med bastypen  \code{Number} som rot.

\Subtask Skriv kod som använder de olika konkreta klasserna i \code{package numbers}. 
\begin{REPL}
scala> numbers.  // Tryck Tab
AbstractComplex   AbstractNatural    AbstractReal   Frac    Nat      Polar
AbstractInteger   AbstractRational   Complex        Integ   Number   Real

scala> numbers.Integ(12)
res0: numbers.Integ = Integ(12)

scala> import numbers.Syntax._
import numbers.Syntax._

scala> 42.j
res1: numbers.Complex = Complex(Real(0),Real(42))

scala> 42.42.j
res2: numbers.Complex = Complex(Real(0),Real(42.42))

\end{REPL}

\Subtask Ändra på metoden \code{+} i \code{trait Number} så att den blir abstrakt och implementera den i alla konkreta klasser.

\Subtask Implementera fler räknesätt och bygg vidare på koden så som du finner intressant.

\Subtask Gör så att metoden \code{reduce} i klassen \code{AbstractRational} använder algoritmen Greatest Common Divisor (GCD)\footnote{\url{https://sv.wikipedia.org/wiki/St\%C3\%B6rsta\_gemensamma\_delare}} så som beskrivs här: \\ \href{http://www.artima.com/pins1ed/functional-objects.html#6.8}{www.artima.com/pins1ed/functional-objects.html\#6.8} \\ så att täljare och nämnare blir så små som möjligt.

%\clearpage

\scalainputlisting[numbers=left, basicstyle=\ttfamily\fontsize{9.1}{12.2}\selectfont]{examples/numbers.scala}\SOLUTION


\QUESTEND

%!TEX encoding = UTF-8 Unicode
%!TEX root = ../compendium2.tex

\Teamlab{\LabWeekTEN}

\begin{Goals}
%!TEX encoding = UTF-8 Unicode
%!TEX root = ../compendium2.tex

\item Kunna använda arv.
\item Kunna göra överskuggning av medlemmar i en supertyp.
%\item Kunna referera till medlemmar i superklassen med \code{super} vid överskuggning.
\item Kunna förklara begreppet dynamisk bindning.
\item Kunna använda abstrakta klasser och skapa en klasshierarki.

\end{Goals}

\begin{Preparations}
\item Gör övning {\tt \ExeWeekNINE} i kapitel \ref{exe:W09}, speciellt uppgift \ref{exe:inheritance:labprep-pair}.
\item Läs dokumentationen för \code{introprog.BlockGame}.
\item Läs igenom hela laborationen och förbered dig inför första gruppmötet.
%!TEX encoding = UTF-8 Unicode
%!TEX root = compendium.tex
\item
Diskutera i din samarbetsgrupp hur ni ska dela upp koden mellan er i flera olika delar, som ni kan arbeta med var för sig. En sådan del kan vara en klass, en trait, ett objekt, ett paket, eller en funktion.
\item
Varje del ska ha en \textbf{huvudansvarig} individ.
\item
Arbetsfördelningen ska vara någorlunda jämnt fördelad mellan gruppmedlemmarna.
\item
Den som är huvudansvarig för en viss del redovisar den delen.
\item 
Ni ska ta fram en gruppgemensam checklista för kodgranskning. Alla ska granska minst en annan gruppmedlems kod enligt checklistan. 
\item
Grupplaborationen görs över \textbf{två veckor} uppdelat på två delredovisningar. Vid första redovisningen ska arbetsupplägget och pågående utveckling redovisas. Vid andra tillfället ska de färdig lösningarna presenteras av respektive huvudansvarig individ.
\item
Vid första redovisningen ska du redogöra för handledaren hur ni delat upp koden och vem som är huvudansvarig för vad och vad ditt ansvar omfattar, samt hur ni jobbar praktiskt med att synkronisera er utveckling.
\item Grupplaborationen är en \textbf{extra stor uppgift} och grupparbetet behöver ledtid för att ni ska hinna koordinera er sinsemellan. Du behöver därför planera för att arbeta med något i grupplabben i stort sett varje dag under de tillgängliga veckorna, och vara redo att bidra i diskussioner.

\item Träffas i din samarbetsgrupp och diskutera ert arbetssätt utifrån följande frågeställningar:
\begin{itemize}[nolistsep]
  \item Vilka krav ska ni implementera?
  \item Hur ska ni jobba med gemensamma koddelar?
  \item Hur ska ni dela med er av de koddelar som ni utvecklar var för sig?
\end{itemize}

\end{Preparations}

\subsection{Bakgrund}

Spelet \emph{Snake}\footnote{Även kallat ''masken''. \url{https://sv.wikipedia.org/wiki/Snake}} blev mäkta populärt i Sverige redan på 1980-talet, ofta spelat på den legendariska datorn ABC80. Spelet finns i flera varianter, både för en spelare och som duell mellan två spelare. Varje spelare styr en mask med huvud och svans som hela tiden rör sig framåt. Det gäller att undvika att köra in i en masksvans och att samla poäng t.ex. genom att äta äpple.

Figur~\ref{fig:snake-game} visar en startdialog där man kan välja antal spelare, samt ett exempel på spel med en och två spelare. I varianten för en spelare närmar sig maskens huvud äpplet och lyckas kanske äta det om spelaren styr rätt.  I varianten för två spelare vinner grön mask eftersom den blåa masken råkade köra in i den gröna maskens svans.
\begin{figure}[H]
\begin{minipage}{0.45\textwidth}
\includegraphics[width=1.0\textwidth]{../img/snake-start}

\includegraphics[width=1.0\textwidth]{../img/snake-oneplayer}
\end{minipage}
\begin{minipage}{0.5\textwidth}
\includegraphics[width=1.0\textwidth]{../img/snake-twoplayer}
\end{minipage}
\caption{Spelet snake för en spelare med äpple och för två spelare utan äpple. \label{fig:snake-game}}
\end{figure}

\subsection{Obligatoriska funktionella krav}

Följande funktionella krav ska uppfyllas av ert program om ni är sex personer i gruppen. Om ni är färre ingår de obligatoriska krav som visas i tabell \ref{lab:snak:table-reqt}.
%\footnote{Om någon student, p.g.a. långvarig sjukdom eller annat giltigt skäl, genomför laborationen själv i efterhand som en individuell laboration ska följande krav implementeras på egen hand: \code{Player}, \code{OnePlayerGame}, \code{Snake}, \code{Apple}.}
\begin{itemize}[nosep, label={$\square$},]
\item \textbf{\texttt{Player}}. Det ska finnas spelare som motsvarar mänskliga användare och som har ett namn och fyra tangenter som den kan spela med. Varje spelare har en egen orm som den kan styra med sina tangenter.

\item \textbf{\texttt{Snake}}. Det ska finnas ormar. En orm består av ett antal block, där det främsta blocket kallas huvud och resten av blocken kallas svans. Huvudet har en ljusare färg än kroppen. Svansens längd ökar under spelets gång. En orm rör sig i en viss riktning och varje spelare kan ändra riktningen på sin orm med sina tangenter, i en av fyra riktningar \code{North}, \code{South}, \code{East} eller \code{West}.

\item \textbf{\texttt{Apple}}. Det ska finnas (minst ett) äpple. Ett äpple består av ett rött block och finns på en slumpvis position. Ett äpple kan ätas av en orm om ormens huvud träffar äpplet. Varje gång ett äpple äts upp av en orm så teleporteras äpplet till en ny position och kan ätas igen.

\item \textbf{\texttt{Banana}}. Det ska finnas (minst en) banan. En banan består av tre vertikala gula block och finns på en slumpvis position. En banan äts upp av en orm om ormens huvud träffar bananen. Varje gång en banan äts upp av en orm så teleporteras bananen till en ny slumpvis position och kan ätas igen.

\item \textbf{\texttt{Monster}}. Det ska finnas (minst ett) monster. Ett monster består av fem rosa block i kryssform.  Ett monster föds på en slumpvis position och rör sig i en riktning som bestäms vid monstrets födelse. Ett orm blir uppäten och dör om ormens huvud nuddar ett monsterblock .

\item \textbf{\texttt{OneplayerGame}}. Det ska gå att spela ensam. I varianten med en spelare finns en orm och minst ett äpple (och ev. även bananer och monster). Varje gång användarens orm lyckas äta en frukt får användaren poäng. När ormen ätit ett visst antal äpplen, eller om ormen blivit uppäten av ett monster, är spelet slut och poängen visas. En ormsvans ska bli längre vid jämna tidsintervall eller om den äter frukt.

\item \textbf{\texttt{TwoplayerGame}}. Det ska gå att spela två och två. I varianten med två spelare finns två ormar. Det finns också äpplen, bananer och monster. Om en orm äter en banan blir dess svans längre. När ormen ätit ett visst antal äpplen, eller om ormen blivit uppäten av ett monster, är spelet slut och poängen visas. En ormsvans ska bli längre vid jämna tidsintervall eller om den äter frukt.

\item \textbf{\texttt{Settings}}. Inställningar för spelet ska vara konfigurerbara genom en textfil som laddas i början av spelet. Inställningar ska vara en kontextparameter.  

\end{itemize}
\begin{table}[H]
  \centering
  \caption{Krav som minst ska implementeras vid respektive gruppstorlek. Om du har särskilda skäl kan du efter godkännande från kursansvarig göra labben enskilt.  \label{lab:snak:table-reqt}}

\begin{tabular}{r | c c c c c c}
  Krav / Antal personer & 1       & 2       & 3       & 4       & 5       & 6 \\ \hline
  \texttt{Player}       & $\surd$ & $\surd$ & $\surd$ & $\surd$ & $\surd$ & $\surd$ \\
  \texttt{OnePlayerGame}& $\surd$ &         &         &         & $\surd$ & $\surd$ \\
  \texttt{TwoPlayerGame}&         & $\surd$ & $\surd$ & $\surd$ & $\surd$ & $\surd$ \\
  \texttt{Snake}        & $\surd$ & $\surd$ & $\surd$ & $\surd$ & $\surd$ & $\surd$ \\
  \texttt{Apple}        & $\surd$ &         & $\surd$ & $\surd$ & $\surd$ & $\surd$ \\
  \texttt{Banana}       &         &         &         & $\surd$ &         & $\surd$ \\
  \texttt{Monster}      &         &         & $\surd$ &         &         & $\surd$ \\
  \texttt{Settings}     & $\surd$ & $\surd$ & $\surd$ & $\surd$ & $\surd$ & $\surd$ \\
\end{tabular}
\end{table}

\subsection{Obligatoriska design-krav}

\begin{enumerate}[label={$\square$}, leftmargin=*]

\item Snake-spel ska gå att starta med huvudprogrammet nedan. Huvudprogrammet får ändras vid behov i enlighet med minimikrav vad gäller gruppstorlek i tabell \ref{lab:snak:table-reqt}, samt valbara extrakrav i avsnitt \ref{lab:snake:extra-reqts}, och era egna ideer.
\scalainputlisting{../workspace/w10_snake/src/main/scala/snake/Main.scala}

\item Spelet ska bygga vidare på \code{introprog.BlockGame} enligt typhierarkin i fig.~\ref{snake:fig:game-hierarchy}.

\begin{figure}[H]
\begin{center}
\newcommand{\TextBox}[1]{\raisebox{0pt}[1em][0.5em]{#1}}
\tikzstyle{umlclass}=[rectangle, draw=black,  thick, anchor=north, text width=3cm, rectangle split, rectangle split parts = 3]
\begin{tikzpicture}[inner sep=0.5em,scale=1.0, every node/.style={transform shape}]

  \node [umlclass, rectangle split parts = 1, xshift=0cm, yshift=4.5cm] (BaseType)  {
              \textit{\textbf{\centerline{\TextBox{\code{BlockGame}}}}}
%              \nodepart[align=left]{second}\code{def x: T} \newline \code{def y: T}
          };


  \node [umlclass, rectangle split parts = 1, xshift=0cm, yshift=3.0cm] (SubType)  {
              \textit{\textbf{\centerline{\TextBox{\code{SnakeGame}}}}}
%              \nodepart[align=left]{second}\code{val x: Int} \newline \code{val y: Int}
          };

\node [umlclass, rectangle split parts = 1, xshift=-3cm, yshift=1.0cm] (SubSubType1)  {
            {\textbf{\centerline{\TextBox{\code{OnePlayerGame}}}}}
%            \nodepart[]{second}\TextBox{\code{val dim: Int}}
        };

\node [umlclass, rectangle split parts = 1, xshift=3cm, yshift=1.0cm] (SubSubType2)  {
            {\textbf{\centerline{\TextBox{\code{TwoPlayerGame}}}}}
%            \nodepart[]{second}\TextBox{\code{val dim: Int}}
        };

\draw[umlarrow] (SubType.north) -- ++(0,0.5) -| (BaseType.south);
\draw[umlarrow] (SubSubType1.north) -- ++(0,0.5) -| (SubType.south);
\draw[umlarrow] (SubSubType2.north) -- ++(0,0.5) -| (SubType.south);

\end{tikzpicture}
\end{center}
\caption{Arvshierarki med klassen \code{introprog.BlockGame} som bastyp.}
\label{snake:fig:game-hierarchy}
\end{figure}


\item Ormar och frukt ska utgå från bastypen \code{Entity} enligt typhierarkin i ~\ref{snake:fig:entity-hierarchy}.

\begin{figure}[H]
\begin{center}
\newcommand{\TextBox}[1]{\raisebox{0pt}[1em][0.5em]{#1}}
\tikzstyle{umlclass}=[rectangle, draw=black,  thick, anchor=north, text width=2.5cm, rectangle split, rectangle split parts = 3]
\begin{tikzpicture}[inner sep=0.5em,scale=1.0, every node/.style={transform shape}]

  \node [umlclass, rectangle split parts = 1, xshift=0.0cm, yshift=4.5cm] (BaseType)  {
              \textit{\textbf{\centerline{\TextBox{\code{Entity}}}}}
%              \nodepart[align=left]{second}\code{def x: T} \newline \code{def y: T}
          };


  \node [umlclass, rectangle split parts = 1, xshift=-3.0cm, yshift=2.5cm] (SubType1)  {
              \textit{\textbf{\centerline{\TextBox{\code{CanMove}}}}}
%              \nodepart[align=left]{second}\code{val x: Int} \newline \code{val y: Int}
          };

\node [umlclass, rectangle split parts = 1, xshift=-4.75cm, yshift=0.5cm] (SubSubType01)  {
            {\textbf{\centerline{\TextBox{\code{Snake}}}}}
%            \nodepart[]{second}\TextBox{\code{val dim: Int}}
};

\node [umlclass, rectangle split parts = 1, xshift=-1.5cm, yshift=0.5cm] (SubSubType02)  {
            {\textbf{\centerline{\TextBox{\code{Monster}}}}}
%            \nodepart[]{second}\TextBox{\code{val dim: Int}}
};


\node [umlclass, rectangle split parts = 1, xshift=3.0cm, yshift=2.5cm] (SubType2)  {
            \textit{\textbf{\centerline{\TextBox{\code{CanTeleport}}}}}
%            \nodepart[]{second}\TextBox{\code{val dim: Int}}
        };

\node [umlclass, rectangle split parts = 1, xshift=1.75cm, yshift=0.5cm] (SubSubType1)  {
            {\textbf{\centerline{\TextBox{\code{Apple}}}}}
%            \nodepart[]{second}\TextBox{\code{val dim: Int}}
        };

\node [umlclass, rectangle split parts = 1, xshift=5.0cm, yshift=0.5cm] (SubSubType2)  {
            {\textbf{\centerline{\TextBox{\code{Banana}}}}}
%            \nodepart[]{second}\TextBox{\code{val dim: Int}}
        };


\draw[umlarrow] (SubType1.north) -- ++(0,0.5) -| (BaseType.south);
\draw[umlarrow] (SubType2.north) -- ++(0,0.5) -| (BaseType.south);
\draw[umlarrow] (SubSubType1.north) -- ++(0,0.5) -| (SubType2.south);
\draw[umlarrow] (SubSubType2.north) -- ++(0,0.5) -| (SubType2.south);
\draw[umlarrow] (SubSubType01.north) -- ++(0,0.5) -| (SubType1.south);
\draw[umlarrow] (SubSubType02.north) -- ++(0,0.5) -| (SubType1.south);

\end{tikzpicture}
\end{center}
\caption{Arvshierarki med klassen \code{Entity} som bastyp.}
\label{snake:fig:entity-hierarchy}
\end{figure}


\item \code{Entity} representerar en varelse i ett spel och ska se ut så här:
\scalainputlisting{../workspace/w10_snake/src/main/scala/snake/Entity.scala}
% \begin{Code}
% trait Entity {
%   def draw():   Unit
%   def erase():  Unit
%   def update(): Unit
%   def reset():  Unit
% }
% \end{Code}
Metoderna \code{draw} resp. \code{erase} anropas vid ritning resp. radering. Metoden \code{reset} återställer ursprungstillståndet. Metoden \code{update} anropas en gång i varje runda i spel-loopen. Predikatet \code{isOccupyingBlockAt} ger sant om positionen \code{p} finns bland de block som varelsen ockuperar på skärmen.

\item \code{CanMove} representerar en entitet som kan röra sig i en viss hastighet, enligt:
\scalainputlisting{../workspace/w10_snake/src/main/scala/snake/CanMove.scala}

% \begin{Code}
% trait MovingEntity extends Entity {
%   def move(): Unit
%
%   var movesPerSecond: Double = 20.0
%
%   final def millisBetweenMoves(): Int =
%     (1000 / movesPerSecond).round.toInt max 1
%
%   private var _timestampLastMove: Long = System.currentTimeMillis
%   final def timestampLastMove = _timestampLastMove
%
%   override final def update(): Unit =
%     if (System.currentTimeMillis > _
%           timestampLastMove + millisBetweenMoves) {
%       _timestampLastMove = System.currentTimeMillis
%       move()
%     }
% }
% \end{Code}

\item \code{CanTeleport} representerar en entitet som finns på en viss plats men som efter ett visst antal uppdateringar utan förvarning teleporterar sig till en ny position:
\scalainputlisting{../workspace/w10_snake/src/main/scala/snake/CanTeleport.scala}

\item Det ska finnas en enumeration \code{State} i singelobjektet \code{SnakeGame} som representerar spelets övergripande tillstånd enligt följande:
\begin{Code}
package snake 

object SnakeGame:
  enum State:
    case Starting, Playing, GameOver, Quitting
  export State.* // gör alla tillstånd synliga i SnakeGame
\end{Code}

\item Vid varje runda i spelloopen ska följande logik exekveras. Denna kod placeras förslagsvis i \code{gameLoopAction}, se vidare \code{SnakeGame} i avsnitt \ref{lab:snake:tips}.
\begin{Code}
    if state == Playing && !isPaused then
      _iterationsSinceStart += 1
      entities.foreach(_.erase())
      entities.foreach(_.update())
      entities.foreach(_.draw())
      onIteration()
      if isGameOver then enterGameOverState()
\end{Code}

\item Det ska finnas ett singelobjekt \code{Colors} där alla färger som används i spelet samlas.

\item Filen \code{pairs.scala} ska enligt laborationsförberedelser i övningsuppgift   \ref{exe:inheritance:labprep-pair} på sidan \pageref{exe:inheritance:labprep-pair} innehålla
\code{Pair[T]}, \code{Dim}, \code{Pos}, \code{Dir}, \code{North}, \code{South}, \code{East}, \code{West}. Se workspace här:\\
\url{https://github.com/lunduniversity/introprog/tree/master/workspace/}

\item Klassen \code{Player} ska se ut som följer:

\end{enumerate}

\scalainputlisting[basicstyle=\ttfamily\fontsize{10.5}{13}\selectfont]
{../workspace/w10_snake/src/main/scala/snake/Player.scala}




\subsection{Valbara krav -- varje person ska välja minst ett}\label{lab:snake:extra-reqts}

Varje person i gruppen ska implementera \emph{minst ett} (gärna flera) av kraven nedan. Vid implementation av flera av dessa krav blir spelet väsentligt roligare.
\begin{itemize}[nosep, label={$\square$}]

\item \textbf{\code{Points}}. Inför ett poängsystem, där poängen beror på t.ex. längden på svansen, antalet steg, antalet svängar, antal uppätna äpplen, etc.

\item \textbf{\code{Highscore}}. Spelet ska visa en lista med de spelare som fått flest poäng.

\item \texttt{\textbf{Äpple}}. Om inte redan ingår bland obl. krav enl.~ \ref{lab:snak:table-reqt}.

\item \textbf{\code{Monster}}. Om inte redan ingår bland obl. krav enl. 
\ref{lab:snak:table-reqt}.

\item \textbf{\code{Banan}}. Om inte redan ingår bland obl. krav enl. 
\ref{lab:snak:table-reqt}.

\item \textbf{\code{SelfTailCrash}}. Om en spelare kör in i sin egen orms svans så är spelet förlorat. (Om detta krav ej implementeras så \emph{får} man köra igenom sin egen svans utan att något händer.)

\item \textbf{\code{BoundaryCrash}}. Om en spelare kör utanför spelplanen så är spelet förlorat. (Om detta krav ej implementeras så ska ormen fortsätta på andra sidan spelplanen när man når kanten.)

\item \textbf{\code{EnterPlayerName}}. Spelare kan ange sitt namn, t.ex. via en dialog eller genom argument till \code{main}. Namnet används i meddelandefältet vid poängräkning och i meddelanden om vem som vunnit.

\item \textbf{\code{TwoPlayerComp extends Competition}}. Två spelare ska kunna tävla i en bäst-av-$n$-matcher-tävling i en sekvens av \code{TwoPlayerGame.play}, där den som vinner flest matcher blir blir totalvinnare.

\item \textbf{\code{SinglePlayerComp extends Competition}}. Flera spelare ska kunna tävla i en-persons-Snake, där den som får flest poäng av $n$ \code{OnePlayerGame}-spel blir totalvinnare.

\item \textbf{\code{Tournament extends Competition}}. Många spelare ska kunna spela en turnering.\footnote{\url{https://en.wikipedia.org/wiki/Tournament}} Namnen på spelarna läses in från en textfil. Valbara varianter:

\begin{itemize}[nosep, label={$\square$}]
\item \textbf{\code{KnockOut extends Tournament}}. Det ska gå att spela en utslagsturnering, som avslutas med final efter semi-final, etc., beroende på antal spelare.
\item \textbf{\code{RoundRobin extends Tournament}}. Det ska gå att spela en alla-möter-alla-turnering, där alla möjliga par av spelare möts i slumpvis ordning.
\end{itemize}

\end{itemize}


\subsection{Tips och förslag}\label{lab:snake:tips}

I detta stycke presenteras skisser till några av de klasser som behövs i enlighet med designkraven. Det är tillåtet att ändra, ta bort och lägga till, så länge de obligatoriska designkraven uppfylls. Koden finns här: \\
\url{https://github.com/lunduniversity/introprog/tree/master/workspace/}

% Här följer en skiss på klassen \code{Apple}:
% \scalainputlisting%[basicstyle=\ttfamily\fontsize{9.1}{12.2}\selectfont]
% {../workspace/w10_snake/src/main/scala/snake/Apple.scala}
% %
% Här följer en skiss på klassen \code{Banana}:
% \scalainputlisting%[basicstyle=\ttfamily\fontsize{9.1}{12.2}\selectfont]
% {../workspace/w10_snake/src/main/scala/snake/Banana.scala}
% Bananens ''kropp'' består av tre vertikalt ordnade blockpositioner i stället för en. Låt \code{pos}-attributet t.ex. betyda det översta av de tre bananblocken.

% Här följer en skiss på klassen \code{Banana}:
% \scalainputlisting%[basicstyle=\ttfamily\fontsize{9.1}{12.2}\selectfont]
% {../workspace/w10_snake/src/main/scala/snake/Monster.scala}
% Monsterkroppen består av fem blockpositioner ordnade som ett kryss. Låt \code{pos}-attributet t.ex. betyda det mittersta av de fem monsterblocken.


Här följer en skiss på klassen \code{Snake}:
\scalainputlisting[basicstyle=\ttfamily\fontsize{9}{12}\selectfont]
{../workspace/w10_snake/src/main/scala/snake/Snake.scala}


Här följer en skiss på den abstrakta klassen \code{SnakeGame} med de abstrakta metoderna \code{isGameOver} och \code{play} som överskuggas i de efterföljande underklasserna \code{OnePlayerGame} och \code{TwoPlayerGame}:
\scalainputlisting[basicstyle=\ttfamily\fontsize{9}{11.9}\selectfont]
{../workspace/w10_snake/src/main/scala/snake/SnakeGame.scala}


%!TEX encoding = UTF-8 Unicode

%!TEX root = ../compendium2.tex

%!TEX encoding = UTF-8 Unicode
\chapter{TODO: Kontextuella abstraktioner}\label{chapter:W11}
Begrepp som ingår i denna veckas studier:
\begin{itemize}[noitemsep,label={$\square$},leftmargin=*]
\item syntaxskillnader mellan Scala och Java
\item klasser i Scala och Java
\item referensvariabler i Java
\item enkla värden i Java
\item primitiva typer i Java
\item referenstilldelning och värdetilldelning i Java
\item alternativ konstruktor i Scala och Java
\item for-sats i Java
\item for-each-sats i Java
\item java.util.ArrayList
\item autoboxing i Java
\item wrapperklasser i Java
\item samlingar i Java
\item scala.jdk.CollectionConverters
\item namnkonventioner för konstanter i Scala och Java
\item kodläsbarhet
\item idiom
\item kodningsstandard\end{itemize}

\clearpage\section{Teori}
%!TEX encoding = UTF-8 Unicode
%!TEX root = ../lect-w11.tex

%%%

\Subsection{\TODO Kontextuella abstraktioner}

 

\begin{Slide}{\TODO Behovet av att abstrahera över kontext}\SlideFontSmall
\begin{itemize}\SlideFontTiny
\item \TODO
\end{itemize}
\end{Slide}




\Subsection{\TODO kolla om finns annan plats: Array}

\begin{Slide}{Repetition: Den primitiva typen Array i JVM}
\begin{itemize}
\item Primitiva arrayer (\code{Array} i Scala, \code{[]} i Java) har \Emph{fördelar}:%
\footnote{\href{http://stackoverflow.com/questions/2843928/benefits-of-arrays}{stackoverflow.com/questions/2843928/benefits-of-arrays}}
\begin{itemize}\SlideFontSmall
\item Det är den snabbaste indexerbara datastrukturen i JVM: att läsa och uppdatera ett element på en viss plats är mycket effektivt om man vet platsens index.
\item Fungerar lika bra med både primitiva värden och objektreferenser
\end{itemize}
\item ... men också \Alert{nackdelar}:
\begin{itemize}\SlideFontSmall
\item Man måste bestämma sig för antalet element som ska allokeras när man gör \code{new}.
\item Man kan ta i lite extra när man allokerar om man behöver plats för fler senare, men då måste man hålla reda på hur många platser man använder och veta var nästa lediga plats finns.
\item Det är krångligt att stoppa in \Eng{insert} och ta bort \Eng{delete} element.
\item Vill man ha fler platser måste man allokera en helt ny, större array och kopiera över alla befintliga element.
\end{itemize}

\end{itemize}
\end{Slide}



\begin{Slide}{Exempel: Polygon med primitiv array i Java}
\begin{Code}[numberstyle=,numbers=left,language=Java]
public class Polygon {
    private Point[] vertices; // array med hörnpunkter
    private int n;            // antalet hörnpunkter

    /** Skapar en polygon */
    public Polygon() {
        vertices = new Point[1];
        n = 0;
    }

    ...
\end{Code}
\end{Slide}

\begin{Slide}{Polygon med primitiv array i Java: stoppa in sist och vid behov skapa mer plats}\SlideFontSmall
Implementera:\\
\jcode{private void extend()                // dubbla storleken}\\
\jcode{public void addVertex(int x, int y)  // lägg till hörnpunkt}
\pause
\begin{Code}[numberstyle=,numbers=left,language=Java]
    private void extend(){
        Point[] oldVertices = vertices;
        vertices = new Point[2 * vertices.length]; // skapa dubbel plats
        for (int i = 0; i < oldVertices.length; i++) {  // kopiera
            vertices[i] = oldVertices[i];
        }
    }

    public void addVertex(int x, int y) {
        if (n == vertices.length) extend();
        vertices[n] = new Point(x, y);
        n++;
    }
\end{Code}
\end{Slide}


\begin{Slide}{Polygon med primitiv array i Java: stoppa in mitt i på angiven plats }\SlideFontSmall
Implementera:\\
\jcode{/** Sätt in hörnpunkt på plats pos */}\\
\jcode{public void insertVertex(int pos, int x, int y)}
\pause
\begin{Code}[numberstyle=,numbers=left,language=Java]
    public void insertVertex(int pos, int x, int y) {
        if (n == vertices.length) extend();   // utöka vid behov
        for (int i = n; i > pos; i--) {       // flytta element bakifrån
            vertices[i] = vertices[i - 1];
        }
        vertices[pos] = new Point(x, y);
        n++;
    }
\end{Code}
\end{Slide}


\Subsection{Scanner}

\begin{Slide}{Scanna filer och strängar med \texttt{java.util.Scanner}}\SlideFontTiny
\setlength{\leftmargini}{0pt}
\begin{itemize}
\item I Scala kan man läsa från fil så här (se quickref sid 3 längst ner):

\begin{Code}
val names = scala.io.Source.fromFile("src/names.txt").getLines.toVector
\end{Code}

\item Klassen \code{java.util.Scanner} kan också läsa från fil (se Java Snabbref sid 4):


\begin{Code}
def readFromFile(fileName: String): Vector[String] = {
  val file = new java.io.File(fileName)
  val scan = new java.util.Scanner(file)
  val buffer = scala.collection.mutable.ArrayBuffer.empty[String]
  while (scan.hasNext) {
    buffer += scan.next
  }
  scan.close
  buffer.toVector
}
\end{Code}

\item Med \code{new java.util.Scanner(System.in)} kan man även scanna tangentbordet.

\item Med \code{new java.util.Scanner("hej 42")} kan man även scanna en sträng.

\item Scanna \code{Int} och \code{Double} med metoderna \code{nextInt} och \code{nextDouble}. Se doc: \href{https://docs.oracle.com/javase/8/docs/api/java/util/Scanner.html}{\SlideFontTiny docs.oracle.com/javase/8/docs/api/java/util/Scanner.html}
\end{itemize}
\end{Slide}


\begin{Slide}{Exempel: Scanner}
\begin{REPL}
scala> val scan = new java.util.Scanner("hej 42 42.0   42 slut")

scala> scan.hasNext
res0: Boolean = true

scala> scan.hasNextInt
res1: Boolean = false

scala> scan.next
res2: String = hej

scala> scan.hasNextInt
res3: Boolean = true

scala> scan.nextInt
res4: Int = 42

scala> while (scan.hasNext) println(scan.next)
42.0
42
slut
\end{REPL}
\end{Slide}


\Subsection{\TODO flytta till fördjupning eller bara i java-appendix?: scala.jdk.CollectionConverters}

\begin{Slide}{Hjälp att använda Java-samlingar i Scala med \texttt{CollectionConverters}}\SlideFontSmall
Med hjälp av \code{import scala.jdk.CollectionConverters._} \\
får man smidig \Emph{interoperabilitet} med Java och dess standardbibliotek, \\
speciellt metoderna \Alert{\code{asJava}} och \Alert{\code{asScala}}:
\begin{REPL}
scala> import scala.jdk.CollectionConverters._

scala> Vector(1,2,3).asJava
res0: java.util.List[Int] = [1, 2, 3]

scala> val xs = new java.util.ArrayList[String]()
xs: java.util.ArrayList[String] = []

scala> xs.add("hej")
res1: Boolean = true

scala> xs.asScala
res2: scala.collection.mutable.Buffer[String] = Buffer(hej)
\end{REPL}

\noindent Läs mer här: %
\ifkompendium\\\fi%
\scriptsize%
\url{https://docs.scala-lang.org/overviews/collections-2.13/conversions-between-java-and-scala-collections.html}

\end{Slide}




\Subsection{ArrayList}

\begin{Slide}{Generiska samlingar i Java}
\begin{itemize}
\item Från och med version 5 av Java (2004) så introducerades \Emph{generics} vilket möjliggör skapandet av klasser som kan erbjuda generell behandling av olika typer av objekt.

\item Generiska klasser i Java känns igen med syntaxen \code{Klassnamn<Typ>}, till exempel  \code{ArrayList<Point>}

\item Fördjupning: \href{https://docs.oracle.com/javase/tutorial/extra/generics/intro.html}{docs.oracle.com/javase/tutorial/extra/generics/intro.html}, mer om detta i fördjupningskursen.

\end{itemize}
\end{Slide}

\begin{Slide}{Om ArrayList i Java}\SlideFontSmall
\code{java.util.ArrayList} liknar \code{scala.collection.mutable.ArrayBuffer} som båda har dessa fördelar:
\begin{itemize}
\item Lagrar sina element internt i snabbindexerade primitiva arrayer.
\item Fungerar för alla typer av objekt.
\item Utökar samlingens storlek av sig själv vid behov.
\end{itemize}
Det finns dock vissa nackdelar med \code{ArrayList} i Java\\(som inte gäller för \code{ArrayBuffer} i Scala):
\begin{itemize}
\item Fungerar \Alert{inte} rakt av med primitiva typer \code{int}, \code{double}, \code{char}, ... \\ (men det finns sätt komma runt detta, tack vare s.k. wrapper-klasser och autoboxing; mer om detta snart)

\item Namnet \code{ArrayList} är inte helt lyckat, eftersom ordet ''lista'' normalt används för länkade snarare än array-liknande strukturer.
\end{itemize}
\end{Slide}

\begin{Slide}{Polygon med ArrayList i Java}\SlideFontSmall
Klassen \code{Polygon}, nu med ett attribut av typen \code{ArrayList<Point>}:
\begin{Code}[numberstyle=,language=Java]
public class Polygon {
    private ArrayList<Point> vertices; // lista med hörnpunkter

    /** Skapar en polygon */
    public Polygon() {
        vertices = new ArrayList<Point>();
    }

    ...
\end{Code}
Det behövs inget attribut \code{n} eftersom vi inte själva behöver hålla reda på antalet allokerade platser: allokering, insättning, och utökning av antalet platser sköts helt automatiskt av \code{ArrayList}-klassen vid behov.
\end{Slide}

\begin{Slide}{Viktiga operationer på ArrayList (Urval)}
\begin{JavaSpec}{class ArrayList}
/** Skapar en ny lista */
ArrayList<E>();

/** Tar reda på elementet på plats pos */
E get(int pos);

/** Lägger in objektet obj sist */
void add(E obj);

/** Lägger in obj på plats pos; efterföljande flyttas */
void add(int pos, E obj);

/** Tar bort elementet på plats pos och returnerar det */
E remove(int pos);

/** Tar reda på antalet element i listan */
int size();
\end{JavaSpec}
Lär dig vad som finns om ArrayList i snabbreferensen för Java\\
\SlideFontSmall Överkurs för den nyfikne: kolla implementation av ArrayList här: \\ {\SlideFontTiny\url{http://www.docjar.com/html/api/java/util/ArrayList.java.html}}
\end{Slide}


\begin{Slide}{Övning ArrayList: new och add}
Skriv Java-kod som skapar en lista med element av typen \code{Point} och lägger in tre punkter i listan med koordinaterna:\\ (50, 50), (50,10) och (30, 40).
\pause
~\\~\\ Lösning: \ifkompendium\else\\~\\\fi
\begin{Code}[numberstyle=,language=Java]
ArrayList<Point> vertices = new ArrayList<Point>();
vertices.add(new Point(50, 50));
vertices.add(new Point(50, 10));
vertices.add(new Point(30, 40));
\end{Code}
\end{Slide}


\begin{Slide}{For-each-sats i Java:}\SlideFontSmall
\begin{itemize}
\item  Antag att vi vill gå igenom alla element i en lista.
\begin{Code}[numberstyle=,language=Java]
        ArrayList<String> words = new ArrayList<String>();
\end{Code}
\item Det finns två olika typer av \jcode{for}-satser i Java som kan göra detta:
\begin{itemize}\SlideFontSmall
\item  Vanlig \jcode{for}-sats:
\begin{Code}[numberstyle=,language=Java]
for (int i = 0; i < words.size(); i++) {
    System.out.println(i + ": " + words.get(i));
}
\end{Code}

\item  Så kallad \Emph{for-each-sats} med denna syntax:\\
\jcode+for (Elementtyp element: samling) { ... }+ \\
\vspace{1em}Exempel:
\begin{Code}[numberstyle=,language=Java]
for (String s: words) {
    System.out.println(s);
}
\end{Code}
Men vi får ingen indexvariabel då...
\end{itemize}
\end{itemize}
\end{Slide}


\begin{Slide}{Polygon med ArrayList: metoderna blir enklare}
\begin{Code}[numberstyle=,language=Java]
    public void addVertex(int x, int y) {
        vertices.add(new Point(x, y));
    }

    public void move(int dx, int dy) {
        for (Point p: vertices){
            p.move(dx, dy);
        }
    }

    public void insertVertex(int pos, int x, int y) {
        vertices.add(pos, new Point(x, y));
    }

    public void removeVertex(int pos) {
        vertices.remove(pos);
    }
\end{Code}

Se hela lösningen här:
\href{https://github.com/lunduniversity/introprog/tree/master/compendium/examples/scalajava/list/Polygon.java}{compendium/examples/scalajava/list/Polygon.java}
\end{Slide}

\begin{Slide}{Polygon med ArrayList: iterera över alla hörnpunkter i draw med indexering}
\begin{Code}[numberstyle=,language=Java]
    public void draw(SimpleWindow w) {
        if (vertices.size() == 0) {
            return;
        }
        Point start = vertices.get(0);
        w.moveTo(start.getX(), start.getY());
        for (int i = 1; i < vertices.size(); i++) {
            w.lineTo(vertices.get(i).getX(),
                     vertices.get(i).getY());
        }
        w.lineTo(start.getX(), start.getY());
    }
\end{Code}

Övning: Skriv om med for-each-sats.
\end{Slide}

\begin{Slide}{Polygon med ArrayList: iterera över alla hörnpunkter i draw med foreach-sats}
\begin{Code}[numberstyle=,language=Java]
    public void draw(SimpleWindow w) {
        if (vertices.size() == 0) {
            return;
        }
        Point start = vertices.get(0);
        w.moveTo(start.getX(), start.getY());
        for (Point p: vertices){
            w.lineTo(p.getX(), p.getY());
        }
        w.lineTo(start.getX(), start.getY());
    }
\end{Code}

Se hela lösningen här:
\href{https://github.com/lunduniversity/introprog/tree/master/compendium/examples/scalajava/list/Polygon.java}{compendium/examples/scalajava/list/Polygon.java}
\end{Slide}




\begin{Slide}{Övning ArrayList: implementera metoden hasVertex}
Skriv kod som implementerar denna metod i klassen \code{Polygon}:
\begin{Code}[numberstyle=,language=Java]
/** Undersöker om polygonen har någon hörnpunkt med koordinaterna x, y. */
public boolean hasVertex(int x, int y) {
    ???
}
\end{Code}
\end{Slide}

\begin{Slide}{Lösning ArrayList: implementera metoden hasVertex}
\begin{Code}[numberstyle=,language=Java]
    public boolean hasVertex(int x, int y) {
        for (Point p: vertices) {
            if (p.getX() == x && p.getY() == y) {
                return true;
            }
        }
        return false;
    }
\end{Code}
\end{Slide}


\begin{Slide}{For-each-sats med array}
For-each-sats fungerar även med primitiv array:
\begin{Code}[numberstyle=,language=Java]
        String[] stringArray = {"hej", "på", "dej"};
        for (String s: stringArray) {
            System.out.println(s);
        }
\end{Code}
\end{Slide}





\Subsection{Autoboxing}



\begin{Slide}{Generiska klasser (t.ex. ArrayList) med primitiva typer}
Detta går tyvärr \Alert{INTE} i Java: \\
  \sout{\texttt{ArrayList<int> list = new ArrayList<int>();}}
  
\pause
\begin{itemize}\SlideFontSmall
\item Hur gör man om man vill ha heltalselement (eller andra primitiva värden) i en generisk samling?

\item Javas lösning på problemet består av två delar:
\begin{itemize}\SlideFontSmall
\item Klasser som packar in primitiva typer, \Eng{wrapper classes}
\item Speciella regler för implicita konverteringar, s.k. ''auto-boxing'' \Eng{Boxing / Unboxing conversions}
\end{itemize}
\end{itemize}
\SlideFontTiny\vspace{1em}
Ofta fungerar det fint, men det finns fallgropar.\\
(Om du är nyfiken på alla intrikata detaljer, se
\href{https://docs.oracle.com/javase/tutorial/java/data/autoboxing.html}{Java tutorial} och   \href{https://docs.oracle.com/javase/specs/jls/se8/html/jls-5.html#jls-5.1.7}{Javaspecifikationen}.)
\end{Slide}

\begin{Slide}{Wrapper-klassen \code{Integer}}\SlideFontSmall
En skiss av klassen \code{Integer} \\ (ligger i paketet \href{http://docs.oracle.com/javase/8/docs/api/java/lang/package-summary.html}{\code{java.lang}} och importeras därmed implicit):

\ifkompendium\vspace{1em}\fi%
\begin{minipage}{0.65\textwidth}
\begin{Code}[numberstyle=,language=Java,backgroundcolor=\color{white},
  frame=none]
public class Integer {
    private int value;

    public static final MIN_VALUE = -2147483648;
    public static final MAX_VALUE = 2147483647;

    public Integer(int value) {
        this.value = value;
    }

    public int intValue() {
        return value;
    }
    ...
}
\end{Code}
\end{minipage}
\begin{minipage}{0.33\textwidth}
\centering\includegraphics[width=0.95\textwidth]{../img/box}
\end{minipage}
Javadoc för klasen \code{Integer} finns här: \\
\SlideFontTiny\url{http://docs.oracle.com/javase/8/docs/api/java/lang/Integer.html}
\end{Slide}





\begin{Slide}{Wrapper-klasser i \code{java.lang}}\SlideFontSmall
\begin{tabular}{l | l}
\Emph{Primitiv typ}                  & \Emph{Inpackad typ}                 \\ \hline

 boolean & Boolean\\
 byte & Byte\\
 short& Short\\
 char & Character\\
 int & Integer\\
 long & Long\\
 float & Float\\
 double & Double\\
\end{tabular}
\end{Slide}


\begin{Slide}{Övning: primitiva versus inpackade typer}
Med papper och penna:
\begin{itemize}
\item Deklarera en variabel med namnet \code{gurka} av den primitiva heltalstypen och initiera den till värdet 42.
\item Deklarera en referensvariabel med namnet  \code{tomat} av den inpackade (''wrappade'') heltalstypen och initiera den till värdet 43.
\item Rita hur det ser ut i minnet.
\end{itemize}
\end{Slide}

\begin{Slide}{Exempel: Lista med heltal utan autoboxning}
\lstinputlisting[language=Java, basicstyle=\small\ttfamily\SlideFontSize{6.7}{8.5},backgroundcolor=\color{white},frame=none
]{../compendium/examples/scalajava/generics/TestIntegerList.java}
\SlideFontTiny Koden finns här: \href{https://github.com/lunduniversity/introprog/tree/master/compendium/examples/scalajava/generics/TestIntegerList.java}{compendium/examples/scalajava/TestIntegerList.java}
\end{Slide}




\begin{Slide}{Specialregler för wrapper-klasser}\SlideFontSmall

\begin{itemize}
\item Om ett \code{int}-värde förekommer där det behövs ett \code{Integer}-objekt, så lägger kompilatorn \Alert{automatiskt} ut kod som skapar ett \code{Integer}-objekt som packar in värdet.
\item Om ett \code{Integer}-objekt förekommer där det behövs ett \code{int}-värde, lägger kompilatorn \Alert{automatiskt} ut kod som anropar metoden \code{intValue()}.
\end{itemize}
Samma gäller mellan alla primitiva typer och dess wrapper-klasser:

\begin{tabular}{r c l}
 {\lstinline!boolean!} &$\Leftrightarrow$& {\lstinline!Boolean!} \\
 {\lstinline!byte!} &$\Leftrightarrow$& {\lstinline!Byte!}\\
 {\lstinline!short!}&$\Leftrightarrow$& {\lstinline!Short!}\\
 {\lstinline!char!} &$\Leftrightarrow$& {\lstinline!Character!}\\
 {\lstinline!int!} &$\Leftrightarrow$& {\lstinline!Integer!}\\
 {\lstinline!long!} &$\Leftrightarrow$& {\lstinline!Long!}\\
 {\lstinline!float!} &$\Leftrightarrow$& {\lstinline!Float!}\\
 {\lstinline!double!} &$\Leftrightarrow$&{\lstinline!Double!}\\
\end{tabular}

\end{Slide}






\begin{Slide}{Exempel: Lista med heltal och autoboxing}
\lstinputlisting[language=Java, basicstyle=\small\ttfamily\SlideFontSize{6}{8}
,backgroundcolor=\color{white},
  frame=none]{../compendium/examples/scalajava/generics/TestIntegerListAutoboxing.java}
\SlideFontTiny Koden finns här: \href{https://github.com/lunduniversity/introprog/tree/master/compendium/examples/scalajava/generics/TestIntegerList.java}{scalajava/generics/TestIntegerListAutoboxing.java}
\end{Slide}

\begin{Slide}{Fallgropar vid autoboxing}
\begin{itemize}
\item Jämförelser med \code{==} och \code{!=} \\
\href{https://github.com/lunduniversity/introprog/blob/master/compendium/examples/scalajava/generics/TestPitfall1.java}
{\SlideFontSmall  compendium/examples/scalajava/generics/TestPitfall1.java}
\item[]
\item Kompilatorn hittar inte förväxlad parameterordning, t.ex. \code{add(pos, item)} i fel ordning: \sout{\code{add(item, pos)}}\\
\href{https://github.com/lunduniversity/introprog/blob/master/compendium/examples/scalajava/generics/TestPitfall2.java}
{\SlideFontSmall compendium/examples/scalajava/generics/TestPitfall2.java}
\end{itemize}
\end{Slide}

\Subsection{Equals}
\begin{Slide}{Fallgrop med samlingar: metoden contains kräver implementation av equals}\SlideFontSmall
Antag att vi vill implementera \code{hasVertex()} i klassen \code{Polygon} genom att använda metoden \code{contains} på en lista. Hur gör vi då?
\pause
\begin{Code}[numberstyle=,language=Java]
public boolean hasVertex(int x, int y) {
    return vertices.contains(new Point(x, y)); // FUNKAR INTE om ...
    // ... inte Point har en equals som kollar innehållslikhet
}
\end{Code}
Vi behöver implementera metoden \code{equals(Object obj)} i klassen \code{Point} som kollar innehållslikhet och ersätter den \code{equals} som finns i \code{Object} som kollar referenslikhet, eftersom metoden \code{contains} i klassen \code{ArrayList} anropar \code{equals} när den letar igenom listan efter lika objekt. \\
Se exempel här: \href{https://github.com/lunduniversity/introprog/tree/master/compendium/examples/scalajava/generics/TestPitfall3.java}{compendium/examples/scalajava/generics/TestPitfall3.java} \\


\vspace{1em}{\SlideFontTiny\noindent Det krävs ofta även att man även ersätter  \href{http://stackoverflow.com/questions/27581/what-issues-should-be-considered-when-overriding-equals-and-hashcode-in-java}{\code{hashCode}}, mer om det i fortsättningskursen.}
\end{Slide}


\begin{Slide}{Fördjupning: Fullständigt recept för \texttt{equals}}
För den nyfikne inför fortsättningskursen efter jul: 

\vspace{1em}\noindent
Läs om fallgropar för att implementera equals i \Emph{Java} här: \\
\href{http://www.artima.com/lejava/articles/equality.html}{www.artima.com/lejava/articles/equality.html}


\vspace{1em}\noindent
Läs receptet för att implementera equals i \Emph{Scala} här: \\
\href{http://www.artima.com/pins1ed/object-equality.html#28.4}{www.artima.com/pins1ed/object-equality.html\#28.4}
\end{Slide}



\Subsection{Fördjupning diverse}


\begin{Slide}{Fördjupning: Villkorsuttryck i Java}\SlideFontSmall
Det går att använda villkorsuttryck i Java, men med syntax från språket C:
\begin{multicols}{2}
  \noindent\Emph{Scala}
\begin{CodeSmall}[basicstyle=\ttfamily\SlideFontSize{6}{8},backgroundcolor=\color{white},
  frame=none]
var r = math.random()
var answer = if (r > 0.5) 42 else 0
\end{CodeSmall}

\columnbreak

\noindent\Emph{Java}
\begin{CodeSmall}[language=Java,basicstyle=\ttfamily\SlideFontSize{6}{8},backgroundcolor=\color{white},
  frame=none]
double r = Math.random();
int answer = (r > 0.5) ? 42 : 0;
\end{CodeSmall}
\end{multicols}

\end{Slide}




\begin{Slide}{Fördjupning: Typtest och typkonvertering}

\begin{multicols}{2}
  \noindent\Emph{Scala}
\begin{CodeSmall}[basicstyle=\small\ttfamily\SlideFontSize{6}{8},backgroundcolor=\color{white},
  frame=none]
var x = "hej"

var isString = x.isInstanceOf[String]

var y = 42

var z = y.asInstanceOf[Double]

\end{CodeSmall}

\columnbreak

\noindent\Emph{Java}
\begin{CodeSmall}[language=Java,basicstyle=\small\ttfamily\SlideFontSize{6}{8},backgroundcolor=\color{white},
  frame=none]
String x = "hej";

boolean isString = x instanceof String;

int y = 42;

double z = (double) y;
\end{CodeSmall}
\end{multicols}


\end{Slide}


\begin{Slide}{Fördjupning: Fånga undantag i Scala och Java}
Typisk skillnad mellan Scala och Java:\\konstruktioner som är \Emph{uttryck} i Scala är ofta \Alert{satser} i Java.
\begin{multicols}{2}
  \noindent\Emph{Scala}
\begin{CodeSmall}[basicstyle=\ttfamily\SlideFontSize{6}{8},backgroundcolor=\color{white},
  frame=none]
val a = try { 2 / 0 } catch {
  case e: ArithmeticException => 0
}

val b = try { 4 / 2 } catch {
  case e: ArithmeticException => 0
}
\end{CodeSmall}

\columnbreak

\noindent\Emph{Java}
\begin{CodeSmall}[language=Java,basicstyle=\ttfamily\SlideFontSize{6}{8},backgroundcolor=\color{white},
  frame=none]
int a;
try {
    a = 2 / 0;
} catch (ArithmeticException e) {
    a = 0;
}

int b;
try {
    b = 4 / 2;
} catch (ArithmeticException e) {
    b = 0;
}

\end{CodeSmall}
\end{multicols}

Mer om undantag \Eng{exceptions} i fortsättningskursen.
\end{Slide}




\begin{Slide}{Fördjupning: Gränssnittet \texttt{List} i Java}\SlideFontSmall
\begin{itemize}
\item I Java finns inte \code{trait} och inmixning.

\item I stället finns \jcode{interface} som liknar \code{trait} men är mer begränsad vad gäller vilka medlemmar som får finnas.

\item Man kan bara göra \code{extends} på exakt en annan klass, men man kan i Java göra \jcode{implements} på flera \jcode{interface}.\\(Jämför Scalas \code{with} på \code{trait}s)

\item Exempel:
\begin{Code}[language=Java,backgroundcolor=\color{white},
  frame=none]
public class ArrayList<E> extends AbstractList<E>
    implements List<E>, RandomAccess, Cloneable, java.io.Serializable
\end{Code}

\item Att implementera ett gränssnitt innebär att uppfylla ett kontrakt som utlovar att vissa speciella metoder finns tillgängliga.

\item Gränssninttet \code{List} uppfylls av en av dess implementationer \code{ArrayList} \\

på liknande sätt i Scala där gränssnittet \code{Seq} uppfylls av \code{Vector} etc.

\item[] \jcode{List<String> xs = new ArrayList<String>();}

\item I Hangman-övningen:

\item[]\jcode{Set<Character> found = new HashSet<Character>();}

\item Mer om gränssnitt i fördjupningskursen.

\end{itemize}
\end{Slide}

\begin{Slide}{Fördjupning: Skapa generisk Array}\SlideFontTiny
\begin{itemize}
\item I Java kan man \Alert{inte} skapa en primitiv array av godtycklig typ enligt generisk typparameter: \sout{\code{T[] xs = new T[42]}}

\item Man måste istället skapa en array av den mest generella referenstypen: \\
\code{Object[] xs = new Object[42]} \\
och sedan typtesta och typkonvertera under körtid; se t.ex. implementationen av \code{ArrayList} på rad 119: \href{http://developer.classpath.org/doc/java/util/ArrayList-source.html}{http://developer.classpath.org/doc/java/util/ArrayList-source.html}

\item[]
\pause
\item Detta går faktiskt att göra i Scala med hjälp av \code{reflect.ClassTag} \pause så här: \\
\begin{REPLnonum}[basicstyle=\ttfamily\SlideFontSize{6}{8}\color{white}]
scala> def fyll[T](n: Int, x: T): Array[T] = Array.fill(n)(x)
<console>:11: error: No ClassTag available for T

scala> def fyll[T: reflect.ClassTag](n: Int, x: T): Array[T] = Array.fill(n)(x)
fyll: [T](x: T)(implicit evidence: scala.reflect.ClassTag[T])Array[T]

scala> fyll(42, "hej")
res2: Array[String] = Array(hej, hej, hej, hej, hej, hej, hej, hej, hej, hej, hej, hej, hej, hej, hej, hej, hej, hej, hej, hej, hej, hej, hej, hej, hej, hej, hej, hej, hej, hej, hej, hej, hej, hej, hej, hej, hej, hej, hej, hej, hej, hej)

scala> fyll(42, 1)
res3: Array[Int] = Array(1, 1, 1, 1, 1, 1, 1, 1, 1, 1, 1, 1, 1, 1, 1, 1, 1, 1, 1, 1, 1, 1, 1, 1, 1, 1, 1, 1, 1, 1, 1, 1, 1, 1, 1, 1, 1, 1, 1, 1, 1, 1)

\end{REPLnonum}


\end{itemize}


\end{Slide}


%%!TEX encoding = UTF-8 Unicode
\chapter{TODO: Kontextuella abstraktioner}\label{chapter:W11}
Begrepp som ingår i denna veckas studier:
\begin{itemize}[noitemsep,label={$\square$},leftmargin=*]
\item syntaxskillnader mellan Scala och Java
\item klasser i Scala och Java
\item referensvariabler i Java
\item enkla värden i Java
\item primitiva typer i Java
\item referenstilldelning och värdetilldelning i Java
\item alternativ konstruktor i Scala och Java
\item for-sats i Java
\item for-each-sats i Java
\item java.util.ArrayList
\item autoboxing i Java
\item wrapperklasser i Java
\item samlingar i Java
\item scala.jdk.CollectionConverters
\item namnkonventioner för konstanter i Scala och Java
\item kodläsbarhet
\item idiom
\item kodningsstandard\end{itemize}


%!TEX encoding = UTF-8 Unicode
%!TEX root = ../exercises.tex

\ifPreSolution


\Exercise{\ExeWeekELEVEN}\label{exe:W11}

\TODO övningar på given using, extensionsmetoder, typklasser, Ordering etc.
\TODO flytta ordering till hit

\begin{Goals}
\item \TODO
\end{Goals}

\begin{Preparations}
\item \StudyTheory{11}
\end{Preparations}

\BasicTasks %%%%%%%%%%%%%%%%

\else

\ExerciseSolution{\ExeWeekELEVEN}

\BasicTasks %%%%%%%%%%%

\fi


\WHAT{Användning av givna värden.}

\QUESTBEGIN

\Task  \what~  \TODO

\Subtask \TODO



\SOLUTION


\TaskSolved \what

\SubtaskSolved  \TODO


\QUESTEND







%!TEX encoding = UTF-8 Unicode
%!TEX root = ../compendium2.tex

\Lab{\LabWeekELEVEN}

\begin{Goals}
\item \TODO
\end{Goals}

\begin{Preparations}
\item \DoExercise{\ExeWeekELEVEN}{11}
\end{Preparations}

\subsection{Redovisning av grupplabb}

\begin{enumerate}
  \item \TODO
  \begin{enumerate}
    \item en kort förklaring av kodens struktur,
    \item en kort förklaring av koncept som du tränat på,
    \item en kort redogörelse för vad du lärt dig om svårigheterna med systemutveckling i grupp,
    \item en kort redogörelse för den återkoppling du fått från granskningar och hur du arbetat med att förbättra läsbarheten under dina stegvisa utvidgningar av din kod,
    \item en kort redogörelse för hur du givit andra feedback när du granskat,
  \end{enumerate}
\end{enumerate}



\input{modules/w12-sorting-chapter.tex}
%%!TEX encoding = UTF-8 Unicode
\chapter{Trådar}\label{chapter:W12}
Begrepp du ska lära dig denna vecka:
\begin{multicols}{2}\begin{itemize}[nosep,label={$\square$},leftmargin=*]
\item tråd
\item jämlöpande exekvering
\item icke-blockerande anrop
\item callback
\item java.lang.Thread
\item java.util.concurrent.atomic.AtomicInteger
\item scala.concurrent.Future\end{itemize}\end{multicols}

\input{modules/w12-sorting-exercise.tex}
\input{modules/w12-sorting-lab.tex}

%!TEX encoding = UTF-8 Unicode

%!TEX root = ../compendium2.tex

%!TEX encoding = UTF-8 Unicode
\chapter{Design}\label{chapter:W13}
Koncept du ska lära dig denna vecka:
\begin{multicols}{2}\begin{itemize}[nosep,label={$\square$},leftmargin=*]
\item\end{itemize}\end{multicols}

\clearpage\section{Tips}
%!TEX encoding = UTF-8 Unicode
%!TEX root = ../lect-w13.tex
%%%



\Subsection{Repetition på begäran}

\newcommand{\Vecka}[1]{\hfill\href{https://fileadmin.cs.lth.se/pgk/lect-w#1.pdf}{w#1}}

% \begin{Slide}{Repetitionsämnen 2020}
% Gör en lista på saker du behöver repetera.\\Exempel på önskade repetitionsämnen från tidigare år:
% \begin{itemize}\SlideFontSmall
%   \item closure (''fångad variabelrymd'') \Vecka{03}
%   \item Skillnad på objekt och singelobjekt? \Vecka{04}
%   \item Mönstermatchning. \Vecka{06}
%   \item \code{Option}  \Vecka{06}
%   \item \code{Try} med stort T  \Vecka{06}
%   \item \code{enum}: när och hur? eller case-klass? \Vecka{07}
%   \item När använda vilken sekvenstyp? \Vecka{07}
%   \item Typhärledning. \Vecka{08}
%   \item komposition eller arv?  \Vecka{10}
% \end{itemize}  
% \end{Slide}

% \begin{Slide}{Repetitionsämnen på begäran från tidigare år}
% \begin{enumerate}\SlideFontSmall
%    \item namnanrop, värdeanrop \Vecka{03}
%    \item funktionsvärde, funktionstyp och thunk \Vecka{03}
%    \item \code{--classpath} \Vecka{04}
%    \item \code{import} \Vecka{04}
%    \item \code{Option} \Vecka{06}
%    \item \code{Try} \Vecka{06}
%    \item \code{enum} \Vecka{07}
%    \item avlusning, läsa felmeddelande \Vecka{08}
%    \item \code{given using} \Vecka{11}
% \end{enumerate}  
% \end{Slide}

% \begin{Slide}{Några extra önskemål från tidigare år (i mån av tid)}
% \begin{enumerate}\SlideFontSmall
%   \item Closure (''fångad variabelrymd'') \Vecka{03}
%   \item Skillnad på objekt och singelobjekt? \Vecka{04}
%   \item Mönstermatchning. \Vecka{06}
%   \item När använda vilken sekvenstyp? \Vecka{07}
%   \item Typhärledning. \Vecka{08}
%   \item Komposition eller arv?  \Vecka{10}
% \end{enumerate}  
% \end{Slide}



\begin{Slide}{På begäran 2025}
\Emph{Grumligt}
\begin{enumerate}\SlideFontSmall
  \item namnanrop och värdeanrop \Vecka{03}
  \item konstruktor \Vecka{05}
  \item mönstermatchning med \code{match} ... \code{case} \Vecka{06}
  \item enumerationer \Vecka{07}
  \item synlighet, import/export, private/protected \Vecka{10}

\end{enumerate}  
\vspace{1em}\Alert{Nyfiken-på}
\begin{enumerate}\SlideFontSmall
  \item säker kod och felhantering
\end{enumerate}  
\end{Slide}


\begin{Slide}{På begäran 2024}
\Emph{Grumligt}
\begin{enumerate}\SlideFontSmall
  \item När är det bra/dåligt att använda anonyma funktioner? \Vecka{03}
  \item Klasser och kompanjonsobjekt: vad passar bäst var? \Vecka{05}
  \item Hur göra felhantering med \code{Option} och \code{Try}? \Vecka{06}
  \item Skillnaden mellan sats \& uttryck, tex \code{if}, \code{for}? \Vecka{01}

\end{enumerate}  
\vspace{1em}\Alert{Nyfiken-på}
\begin{enumerate}\SlideFontSmall
  \item Flertrådad programmering
  \item Fönsterhantering i introprog under huven 
  \item Generiska typgränser \code{<:} \code{>:}
\end{enumerate}  
\end{Slide}
  


% \begin{Slide}{På begäran 2023}
% \Emph{Grumligt}
% \begin{enumerate}\SlideFontSmall
%   \item Jämför: \code{class}, \code{trait}, \code{enum} \Vecka{05}
%   \item Hur fungerar kompanjonsobjekt? \Vecka{05}
%   \item Jämför: \code{try catch finally} och \code{Try Success Failure} \Vecka{06}
%   \item Hur fungerar \code{enum}? \Vecka{07} 
%   \item \code{match} \code{case} \Vecka{06}
%   \item Vad händer i minnet? Aktiveringspost, stacken, heapen \Vecka{03}
% \end{enumerate}  
% \vspace{1em}\Alert{Nyfiken-på}
% \begin{enumerate}\SlideFontSmall
%   \item Auto-formatera kod \hfill \url{https://scalameta.org/scalafmt/}
%   \item Flertrådad programmering: Övning Extra Vecka 12 \\ Kap 12.2.2 Uppgifter om trådar och jämlöpande exekvering
%   \item Opaka typer \hfill \url{https://docs.scala-lang.org/scala3/reference/other-new-features/opaques.html} 
% \end{enumerate}  
% \end{Slide}


%!TEX encoding = UTF-8 Unicode
%!TEX root = ../lect-w12.tex

%%%


\begin{Slide}{Repetition: Vad är en algoritm? }\SlideFontTiny
En \href{https://sv.wikipedia.org/wiki/Algoritm}{algoritm} är en stegvis beskrivning av hur man löser ett problem. \\ 
Exempel: SWAP, MIN, Registrering, Sökning, Sortering \\
\pause\vspace{0.5em}
Problemlösningsprocessens olika steg (inte nödvändigtvis i denna ordning): 
\begin{itemize}
\item Dela upp problemet i enklare delproblem och sätt samman.
\item Finns redan färdig lösning på (del)problem?
\item Formulera (del)\Emph{problemet} och ange tydligt indata och utdata: \\ exempel MIN: indata: sekvens av heltal; utdata: minsta talet
\item Kom på en \Emph{lösningsidé}: (kan  vara mycket klurigt och svårt) \\ exempel MIN: iterera över talen och håll reda på ''minst hittills''
\item Formulera en \Emph{stegvis beskrivning} som löser problemet: \\ exempel: pseudo-kod med sekvens av instruktioner
\item Implementera en \Emph{körbar lösning} i ''riktig'' kod: \\ exempel: en Scala-metod i en klass eller i ett singelobjekt
\item Har algoritmen acceptabla tids- och minneskrav?
\end{itemize}
\pause\vspace{0.5em} Det krävs ofta \Emph{kreativitiet} i stegen ovan  -- även i att \Emph{känna igen} problemet!\\
Simpelt exempel: Du stöter på problemet MAX och ser likheten med MIN.\\
\pause\vspace{0.5em}\Emph{Övning}: Diskutera hur du löser detta problem i relation till stegen ovan: \\
\emph{Att räkna antalet förekomster av olika unika ord i en textsträng.} 
\end{Slide}















\begin{Slide}{Repetition: Tumregler/tips vid val av abstraktion}\SlideFontSmall
Extensionsmetod, singelobjekt, case-klass, klass, trait, eller enum?
\begin{itemize}\SlideFontTiny
\item Om du vill lägga till en metod på befintlig typ utan behov av nya attribut etc., använd \code{extension}.
\item Använd \code{object} om du behöver samla metoder (och variabler) i en modul som bara finns i en upplaga. Du får lokal namnrymd och punktnotation på köpet.
\item Behöver du modellera \Emph{oföränderlig data}, använd en \code{case class} eller \code{enum}.  
\item Om du vill ha uppräknade värden som du vill kunna iterera över och matcha på i förseglad struktur, med värden i egen namnrymd, använd \code{enum}.
\item Med \code{case class} och \code{enum} får du även innehållslikhet och en massa annat godis på köpet!
\item Behöver du \Alert{förändringsbart tillstånd} \Eng{mutable state} använd en vanlig \code{class}. Det normala är att det föränderliga tillståndet (de attribut som är föränderliga) är \code{private} eller \code{protected} och att all uppdatering och avläsning av tillståndet sker indirekt genom metoder (getters/setters/...).
\item Behöver du en abstrakt bastyp använd en \code{trait}, speciellt om du vill ha möjlighet till inmixning.  Om du vill förhindra inmixning eller underlätta användning från Java, använd \code{abstract class}. 
\end{itemize}
\end{Slide}


% \begin{Slide}{Tips om hur man läser en specifikation}\SlideFontSmall
% När du läser en specifikation av en klass, en trait, eller ett singelobjekt:
% \begin{itemize}
% \item Tänk igenom vilket ansvar olika delar av koden har
% \item Vad håller klassen reda på? \\$\rightarrow$ Ledtrådar till attribut
% \item Vad ska klassen göra/räkna ut? \\$\rightarrow$ Ledtrådar till metoder och deras algoritm
% \item Vilka andra klasser har nytta av denna metod? \\$\rightarrow$ Ledtrådar till hur klasserna samverkar för att lösa uppgiften
% \end{itemize}
% Rita gärna en bild med ett specifikt exempel på vilken data som olika instanser håller reda på och fundera på hur data skapas/beräknas/förändras
% \end{Slide}


\begin{Slide}{Repetition: Tips om val av samling}\SlideFontSmall

Det är ofta enklare med oföränderliga samlingar med oföränderliga element och skapa nya samlingar vid förändring. Men för vissa algoritmer blir det enklare eller effektivare om du ändrar på plats i förändringsbar samling.

\begin{itemize}
\item Behöver du hantera värden i sekvens?
\begin{itemize}\SlideFontTiny
\item Om du klarar dig utan förändring av innehållet efter konstruktion:\\
\code{val}-referens till \code{Vector}
\item Om du behöver ändra innehåll men \Alert{inte} antal element:\\
\code{val}-referens till \code{Array}
\item Om du behöver ändra innehåll \Alert{och} antal element:
\\ \code{var}-referens till \code{Vector} och t.ex. metoden \code{patch}, eller \\
\code{val}-referens till \code{ArrayBuffer} och t.ex. metoden \code{insert}
\end{itemize}

\item Behöver du hantera värden \code{x} som ska vara unika?
\begin{itemize}\SlideFontTiny
\item Oföränderlig: \code{  Set}
\item Förändringsbar: \code{val}-referens till \code{scala.collection.mutable.Set}
\end{itemize}

\item Behöver du leta upp värden \code{x:Int} utifrån en nyckel av t.ex. String?
\begin{itemize}\SlideFontTiny
\item Oföränderlig: \code{Map[String, Int] }
\item Förändringsbar: \code{val}-referens till \code{scala.collection.mutable.Map[String, Int]}
\end{itemize}


\end{itemize}
\end{Slide}

% \begin{Slide}{ArrayBuffer}
% Ändra på plats: update, insert, remove, append
% {\SlideFontTiny

% \vspace{2.5em}\begin{tabular}{@{}p{4.2cm}  p{6.5cm}}
% \texttt{xs(i) = x \newline xs.update(i, x)} & Replace element at index i with x. \newline Return type Unit.\\   \cline{1-2}

% \texttt{xs.insert(i, x)\newline xs.remove(i)} & Insert x at index \texttt{i}. Remove element at i. \newline Return type Unit.\\   \cline{1-2}

% \texttt{xs.append(x)~~~xs~+=~x} & Insert x at end.  Return type Unit.\\   \cline{1-2}

% \texttt{xs.prepend(x)~~x~+=:~xs} & Insert x in front.  Return type Unit.\\   \cline{1-2}

% \texttt{xs -= x} & Remove first occurance of x (if exists). \newline Returns xs itself. \\\cline{1-2}

% \texttt{xs ++= ys} & Appends all elements in ys to xs and returns xs itself. \\

% \end{tabular}
% }

% \end{Slide}


\Subsection{Tentatips}

\begin{Slide}{Före tentan:}\SlideFontSmall
\begin{enumerate}
\item Repetera övningar och labbar i kompendiet.
\item Läs igenom föreläsningsanteckningar.
\item Studera \Emph{snabbref} \Alert{mycket noga} så att du vet vad som är givet och var det står, så att du kan hitta det du behöver snabbt.
\item Skapa och \Emph{memorera} en personlig \Emph{checklista} med programmeringsfel du brukar göra, som även inkluderar småfel, så som glömda parenteser och semikolon, och annat som en kompilator/IDE normalt hittar.
\item Tänk igenom hur du ska disponera dina 5 timmar på tentan.
\item Gör minst en extenta som om det vore \Alert{skarpt läge}:
\begin{enumerate}\SlideFontTiny
\item Avsätt 5 ostörda timmar (stäng av telefon, dator etc).
\item Inga hjälpmedel. Bara snabbref.
\item Förbered dryck och tilltugg.
\end{enumerate}
\end{enumerate}
\end{Slide}

\begin{Slide}{På tentan:} \SlideFontTiny
\begin{enumerate}
\item Läs igenom \Alert{hela} tentan först. \\ \Emph{Varför?} Förstå helheten. Delarna hänger ihop.
\item Notera och begrunda specifika begrepp och definitioner. \\ \Emph{Varför?} Begreppen är avgörande för förståelsen av uppgiften.
\item Notera förenklingar, antaganden och specialfall. \\ \Emph{Varför?} Uppgiften blir mkt enklare om du inte behöver hantera dessa.
\item \Alert{Fråga} tentamensansvarig om du inte förstår uppgiften -- speciellt om det finns misstänkta felaktigheter eller förmodat oavsiktliga oklarheter. \\ \Emph{Varför?} Det är inte lätt att konstruera en ''perfekt'' tenta. \\ Du får fråga vad du vill, men det är inte säkert du får svar...
\item Läs specifikationskommentarerna och metodsignaturerna i alla givna klass-specifikationer \Alert{mycket noga}. \\ \Emph{Varför?} Det är ett vanligt misstag att förbise de ledtrådar som ges där.
\item Återskapa din memorerade personliga checklista för vanliga fel som du brukar göra och avsätt tid till att gå igenom den på tentan. Varje fix plockar poäng!
\item Lämna in ett försök även om du vet att lösningen inte är fullständig. Det gäller att plocka så många poäng det går. En ofullständig lösning kan ändå ge poäng.

\item Om du har svårigheter kan det bli kamp mot klockan. Försök hålla huvudet kallt och prioritera utifrån var du kan plocka flest poäng. Ge inte upp! Ta en kort äta-dricka-paus för att få mer energi!

\end{enumerate}
\end{Slide}

\ifkompendium\else

\begin{Slide}{Planeringstips}\SlideFontTiny
Exempel på saker som du kan lägga in tid för i din julpluggkalender:
\begin{enumerate}
\item Ta reda på vad just \Alert{du} behöver träna på!
\item Välja ut övningar att repetera.
\item Repetera övning X, Y, Z, ... Både läsa och skriva kod. Fundera på typ och värde.
\item Välja ut labbar att repetera.
\item Repetera labb X, Y, Z, ... Lär dig ''trick'' och ''mönster''.
\item Träna på att skriva program med papper och penna.
\item Gör så många extentor du orkar, simulera ''skarp läge''.
\item Gör en checklista med vanliga fel och misstag som du brukar göra.
%\item Det finns inte så många Scala-extentor, men du kan också göra Java-extentor och lösa vissa delar i Scala och vissa delar i Java beroende på vad du behöver träna på.

\item Läsa igenom alla de extentor som du väljer att inte göra ''i fiktivt skarpt läge'' och studera generella mönster och typiska trick.
\end{enumerate}
\end{Slide}

\begin{Slide}{Tentans struktur}
\begin{itemize}\SlideFontSmall
\item Del A 20\%:\\\Emph{Evaluera uttryck} där du ska \Alert{ange typ och värde}
\begin{itemize}\SlideFontTiny
\item Testar förståelse av variabler, uttryck, samlingar, algoritmer, arv, etc.
\item Det är bra/nödvändigt att anteckna delsteg och variablers värden, då det är mycket svårt att tänka ut svaren direkt i huvudet.
%\item Ev. ''rättningströskel'': \textit{Om du på del A erhåller färre poäng än vad som krävs för att nå upp till en bestämd ''rättningströskel'', kan din tentamen komma att underkännas utan att del B bedöms.}
\end{itemize}


\item Del B 80\%:\\\Emph{Skriva kod} som uppfyller \Alert{krav och design}
\begin{itemize}\SlideFontTiny
\item Testar att du själv kan skapa kod med delar som samverkar
\item Testar förmåga att gå från indata-utdata till algoritm \\
 givet: ledtrådar, design, ev. skiss på lösning, ev. pseudokod etc.
\end{itemize}
\item Blanka inlämningar ger 0 poäng; det är alltid bättre att försöka än att lämna in blankt. Lämna inte in kladdpapper eller dubbla lösningar.
\end{itemize}
\end{Slide}


\begin{Slide}{Vad kommer på tentan?}
\begin{itemize}
\item Grundläggande begrepp och det som tränas på grundövningar och labbar är basen för att bli godkänd.
\item Begrepp, föreläsningsbilder och övningar som är markerade \Emph{''fördjupning''} krävs ej för att klara tentan men ökar förståelsen och hjälper dig att nå högre betyg.
\item Det är helt ok på tentan om du väljer en \Emph{enkel lösning med basala begrepp} \Alert{som fungerar bra}, i stället för en kortare/elegantare/mer avancerad lösning.
\item Extra-övningarna i läsvecka 12 ingår ej på tentan.
\end{itemize}
\end{Slide}


\Subsection{Avslutning}

\begin{Slide}{När du om några år tänker tillbaka på pgk...}
...hoppas jag du uppskattar den allmänbildning du fick om:
\begin{itemize}
  \item sekvens -- alternativ -- repetition -- abstraktion 
  \item viktiga datavetenskapliga idéer\\namnrymd, datatyp, samling, uppdelning i delproblem, ...
  \item grundläggande algoritmer\\registrering, linjärsökning, insättningssortering, ...
  \item träning i att självständigt skapa lättläst kod
  \item färdighet i användning av programmeringsverktyg  
\end{itemize}

\vspace{1em} Bonus: ett värdefullt socialt sammanhang med framtida kollegor.
\end{Slide}

\begin{Slide}{Scala då, nu och i framtiden}\SlideFontSmall

% {\SlideFontSize{7}{10}\url{
% https://en.wikipedia.org/wiki/Scala_(programming_language)#Versions
% }}
{\SlideFontSize{12}{13} Scalas övergripande målsättning: \Emph{smidigt} OCH \Alert{säkert}}
\begin{itemize}
\item Scala 1.0 (2003) första pre-release
\item Scala 2.0-2.9 (2006-2011) pionjärer: Twitter, LinkedIn, The Guardian, ...
\item Scala 2.10 (2013) brett genombrott, viktiga språkutvidgningar
\item Scala 2.11 (2014) allmän industriell spridning, stabilitet, prestanda, \\
% {\SlideFontSize{7}{10}\url{
% https://en.wikipedia.org/wiki/Scala_(programming_language)#Companies
% }}
\item Scala 2.12 (2016) fokus på prestanda, snabbare bytekod: lambda i JVM
\item Scala 2.13 (2019) fokus på standardbiblioteket och \code{scala.collection}, migreringsverktyg för Scala 3
\item Scala 3.0 (2021): \Alert{stort} \Emph{tekniksprång} med många nya språkdelar %:\\enum, top-level defs, @main, trait params, given, export, creator applications, ...,\\ "uppstädning" + förenklingar baserat på lärdomar från Scala 2.
\item Scala 3.8 (2026): första experimentella steget mot ännu säkrare system (fångstkontroll, nullkontroll, strikt likhet, ...)  
\end{itemize}

% Historiker på wikipedia är tyvärr inte uppdaterad...
%Läs mer om historik här: \url{https://en.wikipedia.org/wiki/Scala_(programming_language)}

Läs mer om nya experimentella delar i Scala här: \url{https://docs.scala-lang.org/scala3/reference/experimental}
\end{Slide}

% \begin{Slide}{Scala 3}\SlideFontSmall
% \begin{itemize}
%   \item Scala 3 släpptes i början av 2021.
%   \item Nya språkkonstruktioner, t.ex.  optional braces, if then else, while do, enum, top-level defs, @main, trait params, extension methods, given, using, export, creator applications, union types, intersection types,  ...,
%   \item \Emph{Tasty}: nytt format för kodträd som kompletterar bytekod och möjliggör omkompilering i efterhand och korsvis användning Scala 2 \& 3. \\
%   \item Formell bas för Scala: DOT (en algebra för dependent object types)
% \item Några viktiga Scala-ramverk för stordata, massiv parallellism, AI:
% \begin{itemize}\SlideFontTiny
%   \item \href{https://akka.io/}{Akka} ramverk för skalbara parallella arkitekturer
%   \item \href{https://spark.apache.org/}{Apache Spark} för parallell behandling av stordata i molnet, för AI, ML ...
%   \item \href{https://en.wikipedia.org/wiki/Apache_Kafka}{Apache Kafka} för hantering av strömmande data (initierad av LinkedIn)
%   \item \href{https://www.playframework.com/}{Play framework} för moderna, skalbara webbappar
% \end{itemize}
% \item Flera ''backends'' som breddar Scalas användningsområde:
% \begin{itemize}\SlideFontTiny
%   \item \href{http://www.scala-js.org/}{scala-js.org}: dela kod+kompetens mellan backend och frontend
%   \item \href{http://scala-native.org}{scala-native.org}: kör Scala kompilerat direkt ''på metallen''
% \end{itemize}
% \end{itemize}
% \end{Slide}


% \begin{Slide}{Scala på JVM, Scala JS, Scala Native}
% \begin{multicols}{3}

% \begin{tikzpicture}[node distance=1.4cm]
% \node (input) [startstop] {Scala-kod};
% \node (compile) [process, below of=input] {\texttt{scalac}};
% \node (output) [startstop, below of=compile] {byte-kod};
% \node (interp) [process, below of=output] {JVM};
% %\node (output2) [startstop, below of=interp] {maskinkod};
% \draw [arrow] (input) -- (compile);
% \draw [arrow] (compile) -- (output);
% \draw [arrow] (output) -- (interp);
% %\draw [arrow] (interp) -- (output2);
% \end{tikzpicture}


% \columnbreak %---------

% %https://www.sharelatex.com/blog/2013/08/29/tikz-series-pt3.html
% \begin{tikzpicture}[node distance=1.4cm]
% \node (input) [startstop] {Scala-kod};
% \node (compile) [process, below of=input] {\texttt{Scala JS}};
% \node (output) [startstop, below of=compile] {Javascript};
% \node (interp) [process, below of=output] {Browser | NodeJS};
% %\node (output2) [process, right of=interp, minimum size=6mm] {NodeJS};
% \draw [arrow] (input) -- (compile);
% \draw [arrow] (compile) -- (output);
% \draw [arrow] (output) -- (interp);
% %\draw [arrow] (interp) -- (output2);
% \end{tikzpicture}

% \columnbreak

% \begin{tikzpicture}[node distance=1.4cm]
% \node (input) [startstop] {Scala-kod};
% \node (compile) [process, below of=input] {\texttt{Scala Native}};
% \node (output) [startstop, below of=compile] {Mellankod (IR)};
% \node (interp) [process, below of=output] {LLVM};
% \node (output2) [startstop, below of=interp] {maskinkod};
% \draw [arrow] (input) -- (compile);
% \draw [arrow] (compile) -- (output);
% \draw [arrow] (output) -- (interp);
% \draw [arrow] (interp) -- (output2);
% \end{tikzpicture}

% \end{multicols} 
% \end{Slide}



\begin{Slide}{Hur håller jag mig uppdaterad om Scalas utveckling?}
\begin{itemize}%\SlideFontTiny
  \item Officiell blog: \url{https://www.scala-lang.org/blog/}
  \item Scala-nyheter: \url{http://scalatimes.com/}
%  \item Open online courses: \\\url{https://www.coursera.org/courses?query=scala}
  \item User-forum: \url{https://users.scala-lang.org/}
  \item Tjatt: \url{https://discord.gg/h9452YPJ}
  \item Scala Center: \url{https://scala.epfl.ch/}
  \item Scala-bibliotek: \url{https://index.scala-lang.org/}
  \item Contributors: \url{https://contributors.scala-lang.org/}
  \item Scala-språkets pågående förbättring: \url{https://docs.scala-lang.org/sips}
  % \item Scala Improvement Process: \\
  % \url{http://docs.scala-lang.org/sips/all.html}
\end{itemize}
\end{Slide}



\begin{Slide}{CEQ -- Course Experience Questionnaire}\SlideFontSmall
\begin{itemize}
\item Görs på hela LTH på samma sätt. Alla får länkar via mejl.
\item Snälla fyll i CEQ! Jag är \Alert{mycket tacksam} för all konstruktiv feedback! \\ Hög svarsfrekvens är viktigt för att kunna dra slutsatser om variationen i svaren och signifikansen i sammanställningen.
\item Del 1: Generella påståenden, alla med 5-gradig skala: \\ tar helt avstånd ... instämmer helt
\item Del 2: \Emph{Fritextfrågor}: \\
''Vad  tycker  du  var  det  bästa  med  den här  kursen?'' \\
''Vad  tycker  du  främst  behöver  förbättras?''
\item Mer om CEQ här: \url{https://www.ceq.lth.se/}
\item \Emph{Fördel} med CEQ: Samma alla kurser alla år medger jämförelse över tid.
\item \Alert{Begränsning}: Saknar frågor kopplat till specifika kursmoment.
\end{itemize}
\end{Slide}

\begin{Slide}{Kursspecifik utvärdering om specifika kursmoment}\SlideFontSmall
\begin{itemize}
\item Jag vill gärna att \Alert{alla} gör den LTH-gemensamma, anonyma kursutvärderingsenkäten \href{https://www.ceq.lth.se/}{CEQ}. Dina fritext-kommentarer om vad som är det bästa med kursen och vad som främst behöver förbättra emottages mycket tacksamt i CEQ-utvärderingen!
\item Jag kommer att komplettera CEQ med en \Emph{kursspecifik utvärdering} av specifika kursmoment i denna kurs och jag är därför \Alert{mycket tacksam} om alla fyller enkäten när länk kommer via anslag i Canvas.
\item Jag behandlar dina svar \Alert{konfidentiellt}, men sparar din email så att jag kan återkomma om jag mot förmodan undrar något mer.
\item Din input är \Emph{mycket värdefull} vid framtida kursutveckling!
\end{itemize}
\end{Slide}

\begin{Slide}{Intresserad av att arbeta som handledare?}
\begin{itemize}
\item Vi har ständigt behov av nya handledare i våra kurser
\item Det är lärorikt att jobba som handledare
\item Information om ansökningsprocessen: 
\begin{itemize}
  \item \url{https://www.cs.lth.se/utbildning/amanuenser}
  \item Sista ansökningsdatum för pgk är senare delen av \Alert{maj} (se exakt datum via länk från ovan sida)
  \item Skriv i ansökan om du helst jobbar i pgk.
  \item Ange övriga relevanta meriter, tex. ledarskap i föreningar, pedagogiska uppdrag, etc.
  \item Prata också gärna med handledare om hur det är att jobba i pgk eller mejla bjorn.regnell@cs.lth.se om du har frågor.
\end{itemize}
\end{itemize}
\end{Slide}


\begin{Slide}{Ett stort TACK...}
\begin{itemize}
  \item
... till alla \Emph{handledare} som jobbat hårt för att ni ska lära er så mycket som möjligt!
\item ... till alla \Alert{studenter} som gått kursen för:
\begin{itemize}
\item ... att ni kämpat så hårt!
\item ... att ni ställt massor med frågor!
\item ... att det har varit så hög närvaro på föreläsningarna!
\item ... att ni hjälp till med värdefull återkoppling!
\item ... att ni är så konstruktiva och verkligen vill lära er!
\end{itemize}
\vspace{2em} \pause

\end{itemize}
\Alert{Ett stort LYCKA TILL på vägen till att bli en \\ kompetent och innovativ systemutvecklare!}
\end{Slide}

\begin{Slide}{Hoppas att pgk-kursen varit givande!}
\includegraphics[width=5cm]{../img/gurka.jpg}\includegraphics[width=5cm]{../img/ukulele.jpg}
\end{Slide}


% subjekt och predikat -> public static void: https://www.youtube.com/watch?v=1ZPaR_wH-R8


% \begin{Slide}{Koda i Scala}

%   {\footnotesize\it Melodi: McDonalds-låten}
% % https://youtu.be/cTVhZqNwn3Y

% \begin{verbatim}

%           E         A             E          B 
% Det finns stunder i livet som man alltid har kvar

%           E           A               B 
% Det finns villkor och uttryck som man spar 

%         F#           B             F#           C#
% Och när koden är öppen finns gemenskap för fler 

% F#      B      C#       F#
% Koda i Scala; det ger meeeeeeer! 
% \end{verbatim}

% \end{Slide}


% \begin{Slide}{Ljuvliga språk}
% \fontsize{7}{8}\selectfont
%   \begin{verbatim}  
% F        F9   Fmaj7 F9        Fmaj7
% Å detta språk detta ljuvliga språk

% F             Gm    Gm7     C7
% som vi kallar bella Scala

% Gm                 Gm7       C7
% se vilken syn alla uttryck i skyn

%       Gm    C7    F
% detta ljuva bella Scala

% Cm                  Cm7  F7     Bbmaj7        Bb
% fjärran från det du älskar blir koden ödsligt tom

%     Dm7   G7     Dm7      G7        Gm7         C
% men i din närhet sluts du in i dess trolska rikedom

% C7#5   F        Am           Eb        D
% åh åh detta språk det är ungdomens språk

%        Gm     C     F      C7       
% som vi kallar bella Scala
% \end{verbatim}
% \end{Slide}




\fi

%%!TEX encoding = UTF-8 Unicode
\chapter{Design}\label{chapter:W13}
Koncept du ska lära dig denna vecka:
\begin{multicols}{2}\begin{itemize}[nosep,label={$\square$},leftmargin=*]
\item\end{itemize}\end{multicols}

%!TEX encoding = UTF-8 Unicode
%!TEX root = ../exercises.tex

\ifPreSolution

\Exercise{\ExeWeekTHIRTEEN}\label{exe:W13}
\begin{Goals}
\item Kunna skriva tentamenslika program med papper, penna och snabbreferens som enda hjälpmedel.
\item Förbereda projektredovisningen.
\item Kunna skapa dokumentation med \code{scaladoc} och \code{javadoc}.
\item Kunna skapa jar-filer.
\end{Goals}

% \begin{Preparations}
% \item \StudyTheory{13}
% \end{Preparations}

\else

\ExerciseSolution{\ExeWeekTHIRTEEN}

\fi


\subsection{Förberedelse inför examination}




\WHAT{Gör en extenta.} %%%%%%%%%%%%%%%%%%%%%%%%%%%%%%%%%%%%%%%%%%%%%%%%%%%%%%%%

\QUESTBEGIN

\Task \what~\TODO

\SOLUTION

\TaskSolved \what~\TODO

\QUESTEND




\WHAT{Förbered din projektredovisning.} %%%%%%%%%%%%%%%%%%%%%%%%%%%%%%%%%%%%%%%

\QUESTBEGIN

\Task \what~\TODO

\SOLUTION

\TaskSolved \what~\TODO

\QUESTEND



\WHAT{Skapa dokumentation.} %%%%%%%%%%%%%%%%%%%%%%%%%%%%%%%%%%%%%%%%%%%%%%%%%%%

\QUESTBEGIN

\Task  \what~

\Subtask \TODO kör nedan kommando i terminalen:

\begin{REPL}
> scaladoc paket.scala
> ls
> firefox index.html   # eller öppna index.html i valfri webbläsare
\end{REPL}

Vad händer?

\Subtask Lägg till några fler metoder i något av objekten i filen \code{paket.scala} och lägg även till några dokumentationskommentarer. Kompilera om och kör. Generera om dokumentationen.

\begin{verbatim}
//... ändra i filen paket.scala

/** min paketdokumentationskommentar p2 */
package p2 {
  /** min paketdokumentationskommentar p21 */
  package p21 {
    /** ett hälsningsobjekt */
    object hello {
      /** en hälsningsmetod i p2.p21 */
      def hello = println("Hej paket p2.p21!")

      /** en metod som skriver ut tiden */
      def date = println(new java.util.Date)
    }
  }
}

\end{verbatim}

\begin{REPL}
> gedit paket.scala
> scalac paket.scala
> jar cvf mittpaket.jar gurka
> scala -cp mittpaket.jar
scala> gurka.tomat.banan.p2.p21.hello.date
scala> :q
> scaladoc paket.scala
> firefox index.html
\end{REPL}

\SOLUTION


\TaskSolved \what

\SubtaskSolved  -

\SubtaskSolved  -

\QUESTEND



\WHAT{Repetera övningar och laborationer.} %%%%%%%%%%%%%%%%%%%%%%%%%%%%%%%%%%%%

\QUESTBEGIN

\Task \what~\TODO

\SOLUTION

\TaskSolved \what~\TODO

\QUESTEND

%!TEX encoding = UTF-8 Unicode
%!TEX root = ../compendium.tex

\Assignment{bank}

\subsection{Obligatoriska uppgifter}

\Task En uppgift.

\Subtask En underuppgift.

\Subtask En underuppgift.

\subsection{Frivilliga extrauppgifter}

\Task En uppgift.

\Subtask En underuppgift.

\Subtask En underuppgift.


\input{modules/w13-assignment-greta.tex}
\input{modules/w13-assignment-music.tex}
\input{modules/w13-assignment-photo.tex}

%!TEX encoding = UTF-8 Unicode

%!TEX root = ../compendium2.tex

%!TEX encoding = UTF-8 Unicode
\chapter{Muntlig examen}\label{chapter:W14}


\TODO Beskrivning av hur muntligt prov går till. Vad händer om du behöver visa mer av dina kunskaper?

%%!TEX encoding = UTF-8 Unicode
\chapter{Muntlig examen}\label{chapter:W14}

%!TEX encoding = UTF-8 Unicode
%!TEX root = ../exercises.tex

\ifPreSolution

\Exercise{\ExeWeekFOURTEEN}\label{exe:W14}

\begin{Goals}
\item Känna till vad en tråd är och kunna förklara begreppet jämlöpande exekvering.
\item Känna till vad metoderna \code{run} och \code{start} gör i klassen \code{Thread}.
\item Kunna skapa och starta en tråd med överskuggad \code{run}-metod.
\item Kunna skapa ett enkelt program som från två trådar tävlar om att uppdatera en variabel och förklara varför beteendet kan bli oförutsägbart.
\item Kunna använda en \code{Future} för att köra igång flera parallella beräkningar.
\item Kunna registrera en callback på en \code{Future} med metoden \code{onComplete}.
%\item Känna till att webbsidor beskrivs av HTML-kod och kunna skapa en minimal webbsida.
%\item Kunna ladda ner en webbsida med \code{scala.io.Source.fromURL}.
\end{Goals}

% \begin{Preparations}
% \item \StudyTheory{14}
% \end{Preparations}

\else

\ExerciseSolution{\ExeWeekFOURTEEN}

\fi


\subsection{Frivilliga extrauppgifter}



\WHAT{Trådar.}

\QUESTBEGIN

\Task  \what~   Klassen \code{java.lang.Thread} används för att skapa  \textbf{trådar} med jämlöpande exekvering \Eng{concurrent execution}. På så sätt kan man få olika koddelar att köra samtidigt.

Klassen \code{Thread} definierar en tom \code{run}-metod. Vill man att tråden ska göra något vettigt får man överskugga \code{run} med det man vill ska göras.

En tråd körs igång med metoden \code{start} och då anropas automatiskt \code{run}-metoden och tråden exekverar koden i \code{run} jämlöpande med övriga trådar. Om man anropar \code{run} direkt blir det \emph{inte} jämlöpande exekvering.

\Subtask Skapa en tråd som gör något som tar lite tid och kör med \code{run} resp. \code{start}.
\begin{REPL}
def zzz = { print("zzzzzz"); Thread.sleep(5000); println(" VAKEN!")}
zzz
val t2 = new Thread{ override def run = zzz }
t2.run
t2.run; println("Gomorron!")
t2.start; println("Gomorron!")
t2.start
\end{REPL}

\Subtask Vad händer om man anropar \code{start} mer än en gång på samma tråd?

\Subtask Skapa två trådar med överskuggade \code{run}-metoder och kör igång dem samtidigt enligt nedan. Vilken ordning skrivs hälsningarna ut efter rad 3 resp. rad 4 nedan? Förklara vad som händer.
\begin{REPL}
val g = new Thread{ override def run = for (i <- 1 to 100) print("Gurka ") }
val t = new Thread{ override def run = for (i <- 1 to 100) print("Tomat ") }
g.run; t.run
g.start; t.start
\end{REPL}

\Subtask Använd \code{Thread.sleep} enligt nedan. Är beteendet helt förutsägbart (deterministiskt)? Förklara vad som händer. Du kan (om du kör Linux) avbryta REPL med Ctrl+C%
\footnote{\href{http://stackoverflow.com/questions/6248884/can-i-stop-the-execution-of-an-infinite-loop-in-scala-repl}{stackoverflow.com/questions/6248884/can-i-stop-the-execution-of-an-infinite-loop-in-scala-repl}}.
\begin{REPL}
def ibland(block: => Unit) = new Thread {
  override def run = while(true) { block; Thread.sleep(600) }
}.start
ibland(print("zzz ")); ibland(print("snark ")); ibland(println("hej!"))
\end{REPL}


\SOLUTION


\TaskSolved \what
     %%%TODO number  1 %%%starts with: \emph{Trådar.}  %%%

\SubtaskSolved   -

\SubtaskSolved  \code {java.lang.IllegalThreadStateException}. Det går inte att starta en tråd mer än en gång. Tråden kan därför inte startas om när den redan har exekverats.

\SubtaskSolved   När \code {start} anropas exekveras koden i \code{run} parallellt. Därför skrivs \code{Gurka} och \code{Tomat} ut omlöpande. Om istället \code{run} anropas direkt blir det inte jämnlöpande exekvering och \code{Gurka} skrivs ut 100 gånger, sedan skrivs \code{Tomat} ut 100 gånger.

\SubtaskSolved   \code{Thread.sleep} pausar inte tråden i exakt den tiden som angets. Alltså kommer det skrivas ut \code{zzz snark hej!} i de flesta fall, men det är inte garanterat.



\QUESTEND






\WHAT{Jämlöpande variabeluppdatering.}

\QUESTBEGIN

\Task \label{task:racecondition} \what~   Skriv klasserna \code{Bank} och \code{Kund} i en editor och klistra sedan in koden i REPL.

\begin{Code}
class Bank {
  private var saldo = 0;
  def visaSaldo: Unit = println("saldo: " + saldo)
  def sättIn: Unit = { saldo += 1 }
  def taUt: Unit   = { saldo -= 1 }
}

class Kund(bank: Bank) {
  def slösaSpara = {bank.taUt; Thread.sleep(1); bank.sättIn}
}
\end{Code}

\Subtask Använd funktionen \code{ibland} från föregående uppgift och kör nedan rader i REPL. Resultatet av jämlöpande variabeluppdatering blir här heltokigt och leder till mycket upprörda bankkunder och -ägare. Förklara vad som händer.

\begin{REPL}
val bank = new Bank
bank.visaSaldo
bank.sättIn
bank.visaSaldo
bank.taUt
bank.visaSaldo

val bamse = new Kund(bank)
val skutt = new Kund(bank)

bamse.slösaSpara
skutt.slösaSpara
bank.visaSaldo

def ofta(block: => Unit) = new Thread {
  override def run = while(true) { block; Thread.sleep(1) }
}.start

ofta(bamse.slösaSpara); ofta(skutt.slösaSpara)

ibland(bank.visaSaldo)
\end{REPL}


\SOLUTION


\TaskSolved \what
     %%%TODO number  2 %%%starts with: \emph{Jämlöpande variabeluppdat%%%

\SubtaskSolved  I \code{slösaSpara} hämtas saldot, ändras och placeras tillbaka i minnet -  fördröjs -  upprepas. Om \code{bamse} blir klar med att ladda, ändra och lagra innan skutt gör detsamma med den muterbara variablen hade det inte varit perfekt. Problemet ligger i  när en tråd laddar och innan den kan lagra det förändrade värdet laddar den andra tråden samma värde. Bara en av dessa trådar vinner racet och får lagra sitt ändrade tal. \code{skutt} och \code{bamse} blir alltså upprörda för att inte alla dess uttag och insättningar registreras.


\QUESTEND






\WHAT{Trådsäkra \code{AtomicInteger}.}

\QUESTBEGIN

\Task  \what~  Det finns stöd i JVM för att åstadkomma uppdateringar som inte kan avbrytas av andra trådar under pågånde minnesskrivning. En operation som inte kan avbrytas kallas \textbf{atomär} \Eng{atomic}. Studera dokumentationen för \code{AtomicInteger}\footnote{\href{https://docs.oracle.com/javase/8/docs/api/java/util/concurrent/atomic/AtomicInteger.html}{docs.oracle.com/javase/8/docs/api/java/util/concurrent/atomic/AtomicInteger.html}} och prova nedan kod. Förklara vad som händer.

Använd funktionerna \code{ofta} och \code{ibland} från tidigare uppgifter.
\begin{Code}
class SäkerBank {
  import java.util.concurrent.atomic.AtomicInteger
  private var saldo = new AtomicInteger
  def visaSaldo: Unit = println(s"saldo: ${saldo.get}")
  def sättIn: Unit = { saldo.incrementAndGet }
  def taUt: Unit   = { saldo.decrementAndGet }
}

class SäkerKund(bank: SäkerBank) {
  def slösaSpara = {bank.taUt; Thread.sleep(1); bank.sättIn}
}
\end{Code}
\begin{REPL}
val säkerBank = new SäkerBank
val farmor = new SäkerKund(säkerBank)
val vargen = new SäkerKund(säkerBank)

ofta(farmor.slösaSpara); ofta(vargen.slösaSpara)

ibland(säkerBank.visaSaldo)
\end{REPL}





\SOLUTION


\TaskSolved \what
     %%%TODO number  3 %%%starts with: \emph{Jämlöpande exekvering med%%%

Nu är \code{farmor} garanterad att kunna ladda saldot, ta ut pengar/ändra och lagra innan \code{vargen} kan överskriva resultatet. I \code{slösaSpara} pausas tråden i en millisekund så \code{vargen} kan fortfarande ta ut pengar innan \code{farmor} hinner sätta in pengar igen. Dock kommer alla uttag och insättningar registreras eftersom operationerna är atomära.


\QUESTEND






\WHAT{Jämlöpande exekvering med \code{scala.concurrent.Future}.}

\QUESTBEGIN

\Task \label{task:future} \what~   Att skapa och hålla reda på trådar kan bli ganska omständligt och knepigt att få rätt på.
Med hjälp av \code{scala.concurrent.Future} kan man på ett enklare sätta skapa jämlöpande exekvering.

\begin{Background}
Med en \code{Future} skapas jämlöpande exekvering som ''under huven'' använder ett ramverk som heter Akka\footnote{\url{http://akka.io/}}, skrivet i Scala och Java. Akka erbjuder automatisk  multitrådning med s.k. trådpooler och möjliggör avancerad parallellprogrammering på en hög  abstraktionsnivå, där man själv slipper skapa instanser av klassen \code{Thread}. I stället kan man helt enkelt placera sin kod inramad med \code|Future{ "körs parallellt" }| efter att man importerat det som behövs.
\end{Background}

\Subtask För att skapa jämlöpande exekvering med \code{Future} behöver man först göra import enligt nedan; då skapas ett exekveringssammanhang med trådpooler redo för användning. Starta om REPL och studera felmeddelandet efter rad 1 nedan. Importera därefter enligt nedan. Vad har \code{f} för typ?
\begin{REPL}
scala> concurrent.Future { Thread.sleep(1000); println("En sekund senare!") }
scala> import scala.concurrent._
scala> import ExecutionContext.Implicits.global
scala> val f = Future { Thread.sleep(1000); println("En sekund senare!") }
\end{REPL}

\Subtask Skapa en procedur \code{printLater} enligt nedan som skriver ut argumentet efter slumpmässig tid. Förklara vad som händer nedan.
\begin{REPL}
scala> def printLater(a: Any): Unit =
         Future { Thread.sleep((math.random * 10000).toInt); print(a + " ") }
scala> (1 to 42).foreach(i => printLater(i)); println("alla är igång!")
\end{REPL}

\Subtask Skapa enligt nedan en \code{Future} som räknar ut hur många siffror det är i ett väldigt stort tal. Med \code{onComplete} kan man ange vad som ska göras när den tunga beräkningen är färdig; detta kallas att ''registrera en callback''. Vilken returtyp har \code{big}? Hur många siffror har det stora talet? Vad har \code{r} för typ? Justera argumentet till \code{big} om du inte orkar vänta på resultatet...

\begin{REPL}
scala> BigInt(10).pow(100)
scala> BigInt(10).pow(100).toString.size
scala> def big(n: Int) = Future { BigInt(n).pow(n).toString.size }
scala> big(1234567).onComplete{r => println(r + " siffror") }
\end{REPL}

\Subtask Den stora vinsten med \code{Future} är att man kan köra vidare under tiden, varför anropet av \code{Future} kallas \textbf{icke-blockerande} \Eng{non-blocking}. Det händer ibland att man ändå vill blockera exekveringen i väntan på ett resultat. Man kan då använda objektet \code{scala.concurrent.Await} och dess metod \code{result} enligt nedan. Använd \code{big} från föregående uppgift och gör en blockerande väntan på resultatet enligt nedan. Vad händer? Vad händer om du väntar för kort tid?

\begin{REPL}
scala> import scala.concurrent.duration._
scala> Await.result(big(1234567), 20.seconds)
\end{REPL}



\SOLUTION


\TaskSolved \what
     %%%TODO number  4 %%%starts with: TODO  %%%%%%%%%%%%%%%%%%%\Advan%%%

\SubtaskSolved  error: Cannot find an implicit ExecutionContext. Future behöver en ExecutionContext för att kunna köras. \code{f} är av typen Future[Unit].

\SubtaskSolved  Funktionen \code{printLater} har en Future, vilket innebär att när både \code{printLater} och \code{println} anropas i foreach-loopen exekveras de jämnlöpande. Eftersom det tar längre tid att starta upp en Future för datorn är \code{println} snabbare och skriver ut att alla är igång först. Sedan skrivs siffrorna från 1 - 42 ut med oregelbundna mellanrum eftersom tråden pausas olika länge.

\SubtaskSolved  \code{big} är en Future[Int]. Det stora talet har 7 520 383 siffror. \code{r} är av typen Try[Int] (se dokumentationen för Future om du är osäker)

\SubtaskSolved  Eftersom exekveringen blockas tills den har fått ett resultat går det inte att fortsätta skriva i REPL medan uträkningen pågår. Väntar man för kort tid får man ett TimeOutException och uträkningen avbryts.


\QUESTEND






\WHAT{Använda \code{Future} för att göra flera saker samtidigt.}

\QUESTBEGIN

\Task  \what~
I denna uppgift ska du ladda ner webbsidor parallellt med hjälp av \code{Future}, så att en nedladdning kan avslutas under tiden en annan dröjer.

\Subtask Koden för en minimal webbsida ser ut som nedan. Du kan beskåda sidan här: \url{http://fileadmin.cs.lth.se/pgk/mini.html} eller skriva in nedan kod i en fil som heter något som slutar på \texttt{.html} och öppna filen i din webbläsare.

\begin{verbatim}
<!DOCTYPE html>
<html>
<body>
HELLO WORLD!
</body>
</html>
\end{verbatim}

\Subtask För att simulera slöa webbservrar kan man ladda ner en sida via sajten \texttt{http://deelay.me/}. Ladda ner ovan sida med 2 sekunders fördröjning:\\
\url{http://deelay.me/2000/http://fileadmin.cs.lth.se/pgk/mini.html}

\Subtask Man kan ladda ner webbsidor med \code{scala.io.Source}. Vad händer nedan? Försök, med ledning av hur \code{delay} beräknas, uppskatta hur lång tid du måste vänta i medeltal, i bästa fall, respektive värsta fall, innan du kan se första webbsidan i vektorn \code{laddningar} nedan?

\begin{REPL}
scala> def ladda(url: String) = scala.io.Source.fromURL(url).getLines.toVector
scala> def slöladda(url: String) = {
         val delay = (math.random * 1000 + 2000).toInt
         val delaySite = s"http://deelay.me/$delay/"
         ladda(delaySite+url)
      }
scala> ladda("http://fileadmin.cs.lth.se/pgk/mini.html")
scala> def seg = slöladda("http://fileadmin.cs.lth.se/pgk/mini.html")
scala> val laddningar = Vector.fill(10)(seg)
scala> laddningar(0)
\end{REPL}

\Subtask Innan vi kan köra igång en \code{Future} så måste vi, som visats i uppgift \ref{task:future} importera den underliggande exekveringsmiljön som är redo att parallelisera ditt program i trådar utan att du själv måste skapa dem. Vad händer nedan?
\begin{REPL}
scala> import scala.concurrent._
scala> import ExecutionContext.Implicits.global
scala> val f = Future{ seg }
scala> f   // kolla om den är klar annars prova igen senare
scala> f
\end{REPL}

\Subtask Ladda indata utan att blockera \Eng{non-blocking input}. Förklara vad som händer nedan.
\begin{REPL}
scala> val nonblock = Future{ Vector.fill(10)(seg) }
scala> nonblock   // kolla igen senare om ej klar
scala> nonblock
\end{REPL}

\Subtask Ladda indata separat i olika parallella trådar. Förklara vad som händer nedan. Kör uttrycket på rad 3 nedan upprepade gånger i snabb följd efter varandra med pil-upp+Enter i REPL.
\begin{REPL}
scala> val para = Vector.fill(10)(Future{ seg })
scala> para
scala> para.map(_.isCompleted)
scala> para.map(_.isCompleted) // studera hur de blir färdiga en efter en
scala> para(0)
\end{REPL}

\Subtask Registrera en callback med metoden \code{onComplete}. Förklara vad som händer nedan.

\begin{REPL}
scala> val action = Vector.fill(10)(Future{ seg })
scala> action(0).onComplete(xs => println(s"ready:$xs"))
scala> // vänta tills laddning på plats 0 är klar
\end{REPL}

\Subtask Registrera en callback för felhantering i händelse av undantag med metoden \code{onFailure}. Förklara vad som händer nedan.
\begin{REPL}
scala> def lycka  = { Thread.sleep(3000); println(":)") }
scala> def olycka = { Thread.sleep(3000); 42 / 0; lycka }
scala> Future{ lycka  }.onFailure{ case e => println(s":( $e") }
scala> Future{ olycka }.onFailure{ case e => println(s":( $e") }
\end{REPL}



\SOLUTION


\TaskSolved \what
     %%%TODO number  5 %%%starts with: Sök upp och studera dokumentati%%%

\SubtaskSolved  -

\SubtaskSolved  -

\SubtaskSolved  Varje sida fördröjs med mellan 2 upp till 3 sekunder (2000-3000 millisekunder). Så i medeltal tar det 2.5 sekunder för varje sida att laddas. Vektorn måste fyllas innan exekveringen kan fortsätta. Därför laddas alla 10 stycken sidor in innan man kan se första websidan. Det tar därför i medeltal 2.5 x 10 = 25 sekunder.

\SubtaskSolved  \code{f} ger en Vektor fylld med strängar där varje element ges av en rad på hemsidan. Då \code{f} körs i bakgrunden kan programmet fortlöpa medan innehållet räknas ut. Du kan därför skriva \code{f} i REPL:n men det är inte säkert att proccessen är klar och det slutgilltiga resultatet visas.

\SubtaskSolved  Samma som ovan, förutom att det blir en vektor där varje element är i sig en vektor med strängar.

\SubtaskSolved  Laddar in datan parallelt så nedladdingen sker samtidigt, men det går olika snabbt pga metoden seg.

\SubtaskSolved  Eftersom datan laddas i parallella trådar utan blockering blir de inte klara i ordning, utan i den ordningen tråden körs klart. Till slut blir alla klara och resultatet visar en vektor med \code{true} värden.

\SubtaskSolved  Metoden \code{lycka} är väldefinerad och kastar därför inga undantag. Den skriver alltid ut \code{:)}. Metoden \code{olycka} är inte väldefinerad då division med 0 ger \code{java.lang.ArithmeticException}. Detta fångas upp vid callbacken och det skrivs ut \code{:(} samt det specifierade undantaget.

\ExtraTasks %%%%%%%%%%%%


\QUESTEND






\WHAT{}

\QUESTBEGIN

\Task  \what~ Räkna ut stora primtal parallellt genom att använda nedan funktioner. Implementera \code{isPrime} enligt pseudokod från den engelska wikipediasidan om primtalstest\footnote{\href{https://en.wikipedia.org/wiki/Primality_test}{en.wikipedia.org/wiki/Primality\_test}} med den s.k. ''naiva algoritmen''.  Räkna ut 10 st slumpvisa primtal med 16 siffror vardera. Gör beräkningarna parallellt med hjälp av \code{Future}.

\begin{Code}
def isPrime(n: BigInt): Boolean = ???

def nextPrime(start: BigInt): BigInt = {
  var i = start
  while (!isPrime(i)) { i += 1 }
  i
}

def randomBigInt(nDigits: Int): BigInt = {
   def rndChar = ('0' + (math.random * 10).toInt).toChar
   val str = Array.fill(nDigits)(rndChar).mkString
   BigInt(str)
}
\end{Code}

\SOLUTION


\TaskSolved \what
  %%%TODO number  6 %%%

\begin{Code}
def isPrime(n: BigInt): Boolean = n match {
  case _ if (n <= 1) => false
  case _ if (n <= 3) => true
  case _ if n % 2 == 0 || n % 3 == 0 => false
  case _ =>
    var i = BigInt(5)
    while (i * i < n) {
      if (n % i == 0 || n % (i + 2) == 0) false
      i += 6
    }
    true
}

import scala.concurrent._
import ExecutionContext.Implicits.global

val primes = Vector.fill(10)(Future{nextPrime(randomBigInt(16))})
primes.foreach(_.onSuccess{case i => println(i)})
\end{Code}


\QUESTEND






\WHAT{Svara på teorifrågor.}

\QUESTBEGIN

\Task  \what~\Pen

\Subtask Vad är en tråd?

\Subtask Hur skapar man en tråd med klassen \code{Thread}?

\Subtask Hur startar man en tråd?

\Subtask Vilka problem kan man råka ut för om man uppdaterar samma resurs i flera olika trådar?

\Subtask Vad innbär det att kod är \emph{trådsäker}?

\Subtask Nämn några fördelar med att använda Future jämfört med att använda trådar direkt.


\SOLUTION


\TaskSolved \what
 %%%TODO number  7 %%%

\SubtaskSolved  Stackoverflow ger följande förklaring:

A thread is an independent set of values for the processor registers (for a single core). Since this includes the Instruction Pointer (aka Program Counter), it controls what executes in what order. It also includes the Stack Pointer, which had better point to a unique area of memory for each thread or else they will interfere with each other.

\SubtaskSolved

\begin{Code}
val thread = new Thread(new Runnable{
	def run(){println(''Det här är en tråd'')}
})
\end{Code}

\SubtaskSolved  \code{thread.start}

\SubtaskSolved  Det kan bli kapplöpning(race conditions) om vilken tråds resurser blir sparade. Vilket leder till att de andra trådarnas ändringar blir ignorerade.

\SubtaskSolved  Trådsäkerhet innebär att flera trådar kan köras parallellt utan felaktigheter i resultatet. Exempelvis får man vara väldigt försiktig om man vill ha en muterbar variabel som alla trådar ska ändra samtidigt.

\SubtaskSolved  Till exempel slipper man skapa instanser av klassen Thread eftersom man kan placera koden i en Future istället. Den löser även mycket under huven för kodaren.


\QUESTEND






\WHAT{Klasser med atomär uppdatering.}

\QUESTBEGIN

\Task  \what~ Läs om och testa klasserna AtomicBoolean, AtomicDouble och AtomicReference för atomär uppdatering i paketet \\ \code{java.util.concurrent.atomic}.

Använd några av dessa tillsammans med \code{scala.concurrent.Future}.


\SOLUTION

\TaskSolved --

\QUESTEND





\WHAT{Skapa din egen multitrådade webbserver.}

\QUESTBEGIN

\Task  \what~

\Subtask Skriv in\footnote{Eller ladda ner här: \href{https://github.com/lunduniversity/introprog/blob/master/compendium/examples/simple-web-server/webserver.scala}{github.com/lunduniversity/introprog/blob/master/compendium/examples/simple-web-server/webserver.scala}} nedan kod i en editor och spara i en fil med namn \texttt{webserver.scala} och kompilera och kör med \texttt{scala webserver.start} och beskriv vad som händer när du med din webbläsare surfar till adressen: \\ \url{http://localhost:8089/abbasillen}

\scalainputlisting[numbers=left,basicstyle=\ttfamily\fontsize{11}{12}\selectfont]{examples/simple-web-server/webserver.scala}

\Subtask Du ska nu skapa en webbserver som gör något lite mer intressant. Den ska svara med det 13:e Fibonacci-talet\footnote{\href{https://sv.wikipedia.org/wiki/Fibonaccital}{https://sv.wikipedia.org/wiki/Fibonaccital}} om du surfar till \url{http://localhost:8089/fib/13}.
Spara din webbserver från föregående deluppgift under det nya namnet \texttt{fibserver.scala} och använd koden nedan och lägg till och ändra så att din server kan svara med Fibonaccital. Vi börjar med att räkna ut Fibonaccital i funktionen \code{compute.fib} nedan på ett onödigt processorkrävande sätt med exponentiell tidskomplexitet så att webbservern verkligen får jobba, för att i senare deluppgifter implementera \code{compute.fib} med linjär tidskomplexitet och därmed undvika onödig planetuppvärmning.
\begin{CodeSmall}
  object compute {
    def fib(n: BigInt): BigInt = {
      if (n < 0) 0 else
      if (n == 1 || n == 2) 1
      else fib(n - 1) + fib(n -2)
    }
  }

  def fibResponse(num: String) = Try { num.toInt } match {
    case Success(n) => html.page(s"fib($n) == " + compute.fib(n))
    case Failure(e) => html.page(s"FEL $e: skriv heltal, inte $num")
  }

  def errorResponse(uri:String) = html.page("FATTAR NOLL: " + uri)

  def handleRequest(cmd: String, uri: String, socket: Socket): Unit = {
    val os = socket.getOutputStream
    val parts = uri.split('/').drop(1) // skip initial slash
    val response: String = (parts.head, parts.tail) match {
      case (head, Array(num)) => fibResponse(num)
      case _                  => errorResponse(uri)
    }
    os.write(html.header(response.size).getBytes("UTF-8"))
    os.write(response.getBytes("UTF-8"))
    os.close
    socket.close
  }
\end{CodeSmall}
Kör i terminalen med \texttt{scala fibserver.start} och beskriv vad som händer i din webbläsare när du surfar till servern.


%%%\textbf{KOD TILL FACIT:}
%%%\scalainputlisting[numbers=left,basicstyle=\ttfamily\fontsize{11}{12}\selectfont]{examples/simple-web-server/fibserver.scala}


\Subtask Surfa efter flera stora Fibonacci-tal samtidigt i olika flikar i din browser. Hur märks det att servern bara kör i en enda tråd?

\Subtask Gör din server multitrådad med hjälp av den nya server-loopen nedan.

\begin{CodeSmall}
import scala.concurrent._
import ExecutionContext.Implicits.global

  def serverLoop(server: ServerSocket): Unit = {
    println(s"http://localhost:${server.getLocalPort}/hej")
		while (true) {
  		Try {
  		  var socket = server.accept  // blocks thread until connect
	  	  val scan = new Scanner(socket.getInputStream, "UTF-8")
		    val (cmd, uri) = (scan.next, scan.next)
			  println(s"Request: $cmd $uri")
		    Future { handleRequest(cmd, uri, socket) }.onFailure {
		      case e => println(s"Reqest failed: $e")
		    }
		  }.recover{ case e: Throwable => s"Connection failed: $e" }
		}
  }
\end{CodeSmall}

\Subtask Surfa efter flera stora Fibonacci-tal samtidigt i olika flikar i din browser. Hur märks det att servern är multitrådad?


\Subtask Det är onödigt att räkna ut samma Fibonacci-tal flera gånger. Med hjälp av en cache i form av en föränderlig \code{Map} kan du spara undan redan uträknade värden. Det funkar dock inte med en vanlig \code{scala.collection.mutable.Map} i vår multitrådade webbserver, eftersom den inte är \textbf{trådsäker} \Eng{thread-safe}. Med trådosäkra föränderliga datastrukturer blir det samma besvärliga beteende som i uppgift \ref{task:racecondition}.

Du ska i stället använda \code{java.util.concurrent.ConcurrentHashMap}. Sök upp  dokumentationen för \code{ConcurrentHashMap} och försök förstå koden nedan. Hur fungerar metoderna \code{containsKey}, \code{put} och \code{get}?
\begin{Code}
object compute {
  import java.util.concurrent.ConcurrentHashMap
  val memcache = new ConcurrentHashMap[BigInt, BigInt]

  def fib(n: BigInt): BigInt =
    if (memcache.containsKey(n)) {
      println("CACHE HIT!!! no need to compute: " + n)
      memcache.get(n)
    } else {
      println("cache miss :( must compute fib:  " + n)
      val f = fastFib(n)
      memcache.put(n, f)
      f
    }

  private def fastFib(n: BigInt): BigInt = {
    if (n < 0) 0 else
    if (n == 1 || n == 2) 1
    else fib(n - 1) + fib(n -2)
  }
}
\end{Code}

\Subtask Använd ovan \code{fib}-objekt i en ny version av din webserver. Spara den i en ny kodfil med namnet \texttt{fibserver-memcached.scala}. Undersök hur snabbt det går med stora Fibonaccital med den nya varianten. Hur stora tal kan du räkna ut? Kan servern fortsätta efter överflödad stack? Förklara varför.

\Subtask Nu när vi kan få väldigt stora Fibonacci-tal kan det vara användbart att stoppa in radbrytningar på webbsidan. Html-taggen \texttt{</br>} ger en radbrytning.
\begin{Code}
  def insertBreak(s: String, n: Int = 80): String = {
    if (s.size < n) s
    else s.take(n) + "</br>" + insertBreak(s.drop(n),n)
  }
\end{Code}
Använd den rekursiva funktionen ovan för att pilla in radbrytningstaggar på var $n$:te position i långa strängar. Testa hur det ser ut på webbsidan med ovan funktion när din server svarar med väldigt stora tal.

\Subtask Vi ska nu använda det större heap-minnet i stället för stack-minnet och därmed inte begränsas av stackens max-storlek. Skriv om \code{fastFib} så att den använder en \code{while}-sats i stället för ett rekursivt anrop. Denna uppgift är ganska klurig, men om du kör fast kan du snegla i lösningarna i Appendix för inspiration.

Hur stora tal klarar din server nu? Vad händer med servern när minnet tar slut? Hur kan du skydda servern så att den inte kan hänga sig?

\SOLUTION


\TaskSolved \what
 %%%TODO number  9 %%%

\SubtaskSolved  \code{abbasillen} skrivs ut baklänges till \code{nellisabba}.

\SubtaskSolved

\SubtaskSolved

\SubtaskSolved

\SubtaskSolved

\SubtaskSolved

\SubtaskSolved

\SubtaskSolved

\SubtaskSolved

Lösningsförslag:
\scalainputlisting[numbers=left,basicstyle=\ttfamily\fontsize{11}{12}\selectfont]{examples/simple-web-server/fibserver-threaded-memcached-while.scala}


\QUESTEND






\WHAT{}

\QUESTBEGIN

\Task  \what~ Utöka din server med fler beräkningsintensiva funktioner. Exempelvis primtalsberäkningar eller beräkningar av valfritt antal decimaler av $\pi$ eller $e$. Utnyttja gärna det du lärt dig i  matematiken om summor och serieutvecklingar.

\SOLUTION


\TaskSolved \what
 %%%TODO number  10 %%%

---


\QUESTEND






\WHAT{}

\QUESTBEGIN

\Task  \what~ Läs mer om \code{Future} och jämlöpande exekvering i Scala här:\\
\href{http://alvinalexander.com/scala/future-example-scala-cookbook-oncomplete-callback}{alvinalexander.com/scala/future-example-scala-cookbook-oncomplete-callback}

\SOLUTION


\TaskSolved \what
 %%%TODO number  11 %%%

---


\QUESTEND






\WHAT{}

\QUESTBEGIN

\Task  \what~ Läs mer om jämlöpande exekvering och multitrådade program i Java här: \href{http://www.tutorialspoint.com/java/java_multithreading.htm}{www.tutorialspoint.com/java/java\_multithreading.htm}  \\
\noindent När man skriver program med jämlöpande exekvering finns det många fallgropar; det kan bli kapplöpning \Eng{race conditions} om gemensamma resurser och dödläge \Eng{deadlock} där inget händer för att trådar väntar på varandra. Mer om detta i senare kurser.


\SOLUTION


\TaskSolved \what
 %%%TODO number  12 %%%

---


\QUESTEND






\WHAT{Studera dokumentationen i \code{scala.concurrent}.}

\QUESTBEGIN

\Task  \what~\Pen

\Subtask Studera dokumentationen för \code{scala.concurrent.Future}\footnote{\href{http://www.scala-lang.org/files/archive/api/current/\#scala.concurrent.Future}{http://www.scala-lang.org/files/archive/api/current/\#scala.concurrent.Future}}. Hur samverkar \code{Future} med \code{Try} och \code{Option}? Vilka vanliga samlingsmetoder känner du igen?

\Subtask Studera dokumentationen för \code{scala.concurrent.duration.Duration}\footnote{\href{http://www.scala-lang.org/api/current/\#scala.concurrent.duration.Duration}{www.scala-lang.org/api/current/\#scala.concurrent.duration.Duration}}. Vilka tidsenheter kan användas?

\Subtask Vid import av \code{scala.concurrent.duration._ } dekoreras de numeriska klasserna med metoder för att skapa instanser av klassen \code{Duration}. Detta möjligörs med hjälp av klassen \code{scala.concurrent.duration.DurationConversions}. Studera dess dokumentation och testa att i REPL skapa några tidsperioder med metoderna på \code{Int}.



\SOLUTION


\TaskSolved \what
 %%%TODO number  13 %%%

\SubtaskSolved

\SubtaskSolved

\SubtaskSolved


\QUESTEND






\WHAT{}

\QUESTBEGIN

\Task  \what~ Fördjupa dig inom webbteknologi.

\Subtask Lär dig om HTML, CSS och JavaScript här: \url{https://developer.mozilla.org/en-US/docs/Learn}

\Subtask Lär dig om Scala.JS här: \url{http://www.scala-js.org/}\SOLUTION


\TaskSolved \what
 %%%TODO number  14 %%%

\SubtaskSolved  ---

\SubtaskSolved  ---

\SubtaskSolved  ---

\SubtaskSolved  ---
\QUESTEND

\input{modules/w14-extra-lab.tex}


\part{Appendix}
\appendix

%\setcounter{chapter}{3} %next after 3 is D in \Alph
%!TEX encoding = UTF-8 Unicode
%!TEX root = ../compendium2.tex

\chapter{Kojo}\label{appendix:kojo}

\section{Vad är Kojo?}

Kojo%
\footnote{\href{https://en.wikipedia.org/wiki/Kojo_(programming_language)}{en.wikipedia.org/wiki/Kojo\_(programming\_language)}}
 är en integrerad utvecklingsmiljö för Scala som är speciellt anpassad för programmeringsundervisning i grundskolan. Kojo används i LTH:s Science Center Vattenhallen för utbildning av grundskolelärare i programmering och vid skolbesök och annan besöksverksamhet, i vilken lärare och studenter vid LTH arbetar som handledare. 
 
 Kojo är öppen källkod och utvecklingsgemenskapen leds av Lalit Pant från Indien. I Kojo finns även lättillgängliga bibliotek som gör tröskeln lägre att programmera rörlig grafik och enkla spel.

Under kursens första laboration använder vi grafikbiblioteket i Kojo för att illustrera grundläggande begrepp, så som sekvens, alternativ, repetition och abstraktion.  


\begin{figure}[H]
\centering
\includegraphics[width=0.8\textwidth]{../img/kojo/kojo.png}
\caption{Den nybörjarvänliga utvecklingsmiljön Kojo för Scala på svenska.}
\label{fig:appendix:ide:kojo}
\end{figure}

\section{Använda grafikbiblioteket i Kojo}\label{appendix:ide:kojo:install}

Kojo bygger på den beprövade pedagogiska idén med sköldpaddsgrafik \Eng{turtle graphics}\footnote{\url{https://en.wikipedia.org/wiki/Turtle_graphics}}, där du skriver program som styr en sköldpadda med en penna under magen. När sköldpaddan rör sig bildas ett streck av valfri färg på skärmen. Beroende på hur du bestämmer att sköldpaddan ska röra sig och vilken färg du bestämmer att pennan ska ha, kan du skapa olika intressanta bilder och samtidigt lära dig om programmeringens grunder.

Under kursens första laboration ska du använda grafikbiblioteket i Kojo tillsammans med editorn VS \code{code} och \code{scala-cli} i terminalen (se appendix \ref{appendix:terminal} och \ref{appendix:compile}). Ladda ner filen \texttt{kojo.scala} från \url{https://cs.lth.se/pgk/kojolib} och spara i en ny katalog med hjälp av din webbläsare, eller via dessa kommandon:

\begin{REPLnonum}
> mkdir w01-kojo
> cd w01-kojo
> curl -o kojolib.scala -sL https://cs.lth.se/pgk/kojolib
\end{REPLnonum}

Nu kan du starta Scala REPL och rita med Kojo så här:

\begin{REPLnonum}
> scala-cli repl .
Welcome to Scala 3.1.2 (17.0.2, Java OpenJDK 64-Bit Server VM).
Type in expressions for evaluation. Or try :help.
                                                                                                                               
scala> fram; höger; fram; vänster

\end{REPLnonum}

Du kan starta VS \code{code} i aktuellt bibliotek så här:
\begin{REPLnonum}
> code .
\end{REPLnonum}

Skriv nedan progam i VS \code{code} och spara det i samma katalog som den tidigare nedladdade filen, under ett nytt valfritt filnamn, t.ex. \code{rita.scala}:

\begin{Code}
@main def rita = fram; höger; fram; vänster
\end{Code}

Kör ditt fristående program med:
\begin{REPLnonum}
> scala-cli run .
\end{REPLnonum}

Du ska nu få upp ett fönster som heter Kojo Canvas med en sköldpadda som ritat två streck. När du stänger fönstret så avslutas programmet. Prova fler sköldpaddsfunktioner enligt tabell \ref{table:kojo:functions}.

I stället för att ladda ned filen \code{kojolib.scala} så kan du placera dess innehåll på lämpligt ställe i ditt program enligt nedan. Observera att raden som börjar med \code{//> using lib} ska vara en enda lång rad utan radbrytningar.%\code{export} gör Kojos kommandon tillgängliga utan prefix:
\lstinputlisting[breaklines=true,basicstyle=\ttfamily\fontsize{9}{11}\selectfont]{../workspace/w01_kojo/kojo.scala}

\noindent Scala-koden för den svenska paddans api finns här: \\
%\href{https://github.com/litan/kojo/blob/master/src/main/scala/net/kogics/kojo/lite/i18n/svInit.scala}{github.com/litan/kojo/blob/master/src/main/scala/net/kogics/kojo/lite/i18n/svInit.scala} \\
\href{https://github.com/litan/kojo-lib/blob/main/src/main/scala/net/kogics/kojo/i18n/Swedish.scala}{github.com/litan/kojo-lib/blob/main/src/main/scala/net/kogics/kojo/i18n/Swedish.scala}


%Kojo kräver (numera) \emph{inte} att \texttt{java} finns på din dator utan kommer med en egen JVM. 
%Eftersom du behöver tillgång till JDK i kursen, är det lika bra att installera hela JDK direkt (och inte bara JRE, så som beskrivs å länken ovan); se vidare hur du gör detta i avsnitt \ref{appendix:compile:install-jdk}.
%\href{http://www.kogics.net/kojo-download}{www.kogics.net/kojo-download}



\section{Kojo Desktop}

Kojo finns som fristående skrivbordsapplikation, kallad Kojo Desktop. Kojo Desktop innehåller en egen editor med syntaxfärgning för Scala, men fungerar ännu så länge bara för Scala 2. En av de synligaste skillnaderna mellan Scala 2 och Scala 3 är att klammerparenteser vid flerradiga funktioner är nödvändiga i Scala 2, medan Scala 3 har valfria klammerparenteser. Så om du använder Kojo Desktop behöver du komma ihåg att omgärda sekvenser av rader som hör ihop med \code|{| och \code|}|. 

Kojo Desktop är förinstallerad på LTH:s datorer och körs igång med terminalkommandot \texttt{kojo} eller via applikationsmenyn.  För instruktioner om hur du installerar Kojo Desktop på din egen dator se här: \href{http://www.lth.se/programmera/installera/}{lth.se/programmera/installera}

När du startar Kojo första gången, välj ''Svenska'' i språkmenyn och starta om Kojo. Därefter fungerar grafikfunktionerna på svenska enligt tabell \ref{table:kojo:functions} på sidan \pageref{table:kojo:functions}. När du startat om Kojo inställt på svenska ser programmet ut ungefär som i figur \ref{fig:appendix:ide:kojo} på sidan \pageref{fig:appendix:ide:kojo}.

Det finns ett antal användbara kortkommando som du hittar i menyerna i Kojo Desktop. Undersök speciellt Ctrl+Alt+Mellanslag som ger autokomplettering baserat på det du börjat skriva.

\section{Kojo i Webbläsaren}

En begränsad variant av Kojo finns tillgänglig för programmering direkt i din webbläsare här: \url{http://kojo.lu.se/}

När du trycker på play-knappen så kompileras din kod på en server till Javascript via ScalaJS och därefter körs Javascript-koden i din webbläsare. 
Kojo på webben är också ännu så länge begränsad till Scala 2 och kräver att du omgärdar sekvenser av rader som hör ihop med \code|{| och \code|}|.


\section{Mer om Kojo}

I detta dokument finns en enkel introduktion till Kojo: \\ ''Introduction to Kojo'' \url{http://www.kogics.net/kojo-ebooks#intro}

\noindent I tabell \ref{table:kojo:functions}, som fortsätter på efterföljande sidor, finns ett urval av kommando i Kojo på svenska och engelska.

{\small\renewcommand{\arraystretch}{1.4}
\begin{longtable}{@{}p{0.42\textwidth} p{0.55\textwidth}}

\caption{Ett urval av funktioner i Kojo. Se även \href{http://lth.se/programmera}{lth.se/programmera}}\label{table:kojo:functions}\\

\emph{Svenska/Engelska} & \emph{Vad händer?}  \\ \hline
\code|sudda| \newline \code|clear| & Ritfönstret suddas \\
\code|fram| \newline \code|forward()| & Paddan går framåt 25 steg. \\
\code|fram(100)| \newline \code|forward(100)| & Paddan går framåt 100 steg. \\
\code|höger| \newline \code|right(90)| & Paddan vrider sig 90 grader åt höger. \\
\code|höger(45)| \newline \code|right(45)| & Paddan vrider sig 45 grader åt höger. \\
\code|vänster| \newline \code|left(90)| & Paddan vrider sig 90 grader åt vänster. \\
\code|vänster(45)| \newline \code|left(45)| & Paddan vrider sig 45 grader åt vänster. \\
\code|hoppa| \newline \code|hop| & Paddan hoppar 25 steg utan att rita. \\
\code|hoppa(100)| \newline \code|hop(100)| & Paddan hoppar 100 steg utan att rita. \\
\code|hoppaTill(100, 200)| \newline \code|jumpTo(100, 200)| & Paddan hoppar till läget (100, 200) utan att rita. \\
\code|gåTill(100, 200)| \newline \code|moveTo(100, 200)| & Paddan vrider sig och går till läget (100, 200). \\
\code|hem| \newline \code|home| & Paddan går tillbaka till utgångsläget (0, 0). \\
\code|öster| \newline \code|setHeading(0)| & Paddan vrider sig så att nosen pekar åt höger. \\
\code|väster| \newline \code|setHeading(180)| & Paddan vrider sig så att nosen pekar åt vänster. \\
\code|norr| \newline \code|setHeading(90)| & Paddan vrider sig så att nosen pekar uppåt. \\
\code|söder| \newline \code|setHeading(-90)  | & Paddan vrider sig så att nosen pekar neråt. \\
\code|mot(100,200)| \newline \code|towards(100, 200)| & Paddan vrider sig så att nosen pekar mot läget (100, 200) \\
\code|sättVinkel(90)| \newline \code|setHeading(90)| & Paddan vrider nosen till vinkeln 90 grader. \\
\code|vinkel| \newline \code|heading| & Ger vinkelvärdet dit paddans nos pekar. \\
\code|sakta(5000)| \newline \code|setAnimationDelay(5000) | & Gör så att paddan ritar jättesakta. \\
\code|suddaUtdata| \newline \code|clearOutput| & Utdatafönstret suddas. \\
\code|utdata("hej")| \newline \code|println("hej")| & Skriver texten \texttt{hej} i utdatafönstret. \\
\code|val t = indata("Skriv")| \newline \code|val t = readln("Skriv:")| & Väntar på inmatning efter ledtexten \texttt{Skriv} och sparar den inmatade texten i t.  \\
\code|textstorlek(100)| \newline \code|setPenFontSize(100)| & Paddan skriver med jättestor text nästa gång du gör skriv. \\
\code|båge(100, 90)| \newline \code|arc(100, 90)| & Paddan ritar en båge med radie 100 och vinkel 90. \\
\code|cirkel(100)| \newline \code|circle(radie)| & Paddan ritar en cirkel med radie 100. \\
\code|synlig| \newline \code|visible| & Paddan blir synlig. \\
\code|osynlig| \newline \code|invisible| & Paddan blir osynlig. \\
\code|läge.x| \newline \code|position.x| & Ger paddans x-läge \\
\code|läge.y| \newline \code|position.y| & Ger paddans y-läge \\
\code|pennaNer| \newline \code|penDown| & Sätter ner paddans penna så att den ritar när den går. \\
\code|pennaUpp| \newline \code|penUp| & Lyfter upp paddans penna så att den INTE ritar när den går. \\
\code|pennanÄrNere| \newline \code|penIsDown| & Kollar om pennan är nere eller inte. \\
\code|färg(rosa)| \newline \code|setPenColor(pink)| & Sätter pennans färg till rosa. \\
\code|fyll(lila)| \newline \code|setFillColor(purple)| & Sätter ifyllnadsfärgen till lila. \\
\code|fyll(genomskinlig)| \newline \code|setFillColor(noColor)| & Gör så att paddan inte fyller i något när den ritar. \\
\code|bredd(20)| \newline \code|setPenThickness(20)| & Gör så att pennan får bredden 20. \\
\code|sparaStil| \newline \code|saveStyle| & Sparar pennans färg, bredd och fyllfärg. \\
\code|laddaStil| \newline \code|restoreStyle| & Laddar tidigare sparad färg, bredd och fyllfärg. \\
\code|sparaLägeRiktning| \newline \code|savePosHe| & Sparar pennans läge och riktning \\
\code|laddaLägeRiktning| \newline \code|restorePosHe| & Laddar tidigare sparad riktning och läge \\
\code|siktePå| \newline \code|beamsOn| & Sätter på siktet. \\
\code|sikteAv| \newline \code|beamsOff| & Stänger av siktet. \\
\code|bakgrund(svart)| \newline \code|setBackground(black)| & Bakgrundsfärgen blir svart. \\
\code|bakgrund2(grön,gul)| \newline \code|setBackgroundV(green, yellow)| & Bakgrund med övergång från grönt till gult. \\
\code|upprepa(4){fram; höger}| \newline \code|repeat(4){forward; right}| & Paddan går fram och svänger höger 4 gånger. \\
\code|avrunda(3.99)| & Avrundar 3.99 till 4.0 \\
\code|slumptal(100)| & Ger ett slumptal mellan 0 och 99. \\
\code|slumptalMedDecimaler(100)| & Ger ett slumptal mellan 0 och 99.99999999 \\
\code|systemtid| & Ger nuvarande systemklocka i sekunder. \\
\code|räknaTill(5000)| & Kollar hur lång tid det tar för din dator att räkna till 5000. \\


\end{longtable}
}%end small

%!TEX root = ../compendium.tex

\chapter{Terminalfönster och kommandoskal}

\section{Vad är ett terminalfönster?}

I ett terminalfönster kan man skriva kommandon som till exempel kör program och hanterar filer på din dator. När man programmerar använder man ofta terminalkommando för att kompilera och exekvera sina program.   
 
\subsubsection{Terminal i Linux}

\subsubsection{PowerShell i Microsoft Windows}
Microsoft Windows är inte Unix-baserat, men i kommandotolken PowerShell finns alias definierat för en del vanliga unix-kommandon. Du startar Powershell t.ex. genom att genom att trycka på Windows-knappen och skriva \texttt{powershell}.

\subsubsection{Terminal i Apple OS X}
Apple OS X är ett Unix-baserat operativsystem. Många kommandon som fungerar under Linux fungerar också under Apple OS X.

\section{Några viktiga terminalkommando}
%!TEX encoding = UTF-8 Unicode
%!TEX root = ../compendium.tex

\chapter{Kompilera och exekvera}\label{appendix:compile}

\section{Vad är en kompilator?}

\section{Java JDK}

\subsection{Installera Java JDK}

\section{Scala}

\subsection{Installera Scala-kompilatorn}

\subsection{Scala Read-Evaluate-Print-Loop (REPL)}\label{appendix:compile:REPL}

För många språk, t.ex. Scala och Python, finns det en interaktiv tolk som gör det möjligt att exekvera enstaka programrader och direkt se effekten. En sådan tolk kallas Read-Evaluate-Print-Loop eftersom den läser en rad i taget och översätter till maskinkod som körs direkt.    

\TODO Kortkommandon: Ctrl+K etc.

\TODO :paste


%!TEX encoding = UTF-8 Unicode
%!TEX root = ../compendium.tex

\chapter{Fixa fel}\label{appendix:debug}



\section{Vad är en bugg?}

En bugg är en oönskad egenskap hos ett program. 

\textbf{Varför heter det bugg?}


\textbf{Olika sorters fel?}

Kompileringsfel och exekveringsfel. 

Oändliga loopar eller bara långsamt... 

Det är viktigt att skilja på felorsak och felyttring.

\textbf{Bugg eller feature?} Är det verkligen ett fel eller är det egentligen ett avsett beteende?

\textbf{Felhanteringsverktyg} exempel Jira.
s
\section{Förebygga fel}

\begin{itemize}
\item \textbf{Begriplig kod}.
\item \textbf{Bra namn}.
\item \textbf{Typannoteringar}.
\item \textbf{Kontrollera villkor}.
\item \textbf{Hantera saknade värden}.
\item \textbf{Slänga undantag}.
\item \textbf{Granska kod}.
\item \textbf{Testa kod}.
\end{itemize}


\section{Vad är debugging?}

När fel identifierats, vid testning eller under användning av slutanvändare ''i produktion'' ska felorsaken hittas och felet åtgärdas. Detta kallas avlusning \Eng{debugging}.

Identifiering går vi inte vidare in på här. Testning är ett stort område....



\section{Hitta felorsaken}

När det blir fel som är svåra att hitta beror det ofta på att man har en felaktig hypotes om vad koden egentligen gör. Du stirrar dig blind på ett kodstycke och är övertygad om att en viss sak händer, men \emph{egentligen} är det \emph{inte} det du \emph{tror} händer som \emph{verkligen} händer. Exempelvis kanske du antar att en räknare räknas upp i en loop, men i själva verket saknas uppräkningen. 

En grundläggande princip i felsökning är att explicit formulera dina hypoteser och seda verifiera att de verkligen stämmer. Du ska alltså tydligt beskriva hur du tror att koden fungerar och sedan med olika former av instrumentering, t.ex. genom utskrifter i terminalen av variablers värden, kontrollera att så verkligen är fallet.

\subsection{Återskapa buggen med ett minimalt testfall}

\subsection{Instrumentering med utskrifter, ''print-debugging''}

\section{Använda en debugger}

\begin{itemize}
\item \textbf{Sätta brytpunkter}.
\item \textbf{Stegad exekvering}.
\item \textbf{Inspektera variabler}.
\end{itemize}

\subsection{Debuggern i Eclipse med ScalaIDE}
\subsubsection{Sätta brytpunkter i Eclipse}
\subsubsection{Stegad exekvering i Eclipse}
\subsubsection{Inspektera variabler i Eclipse}

\subsection{Debuggern i IntelliJ IDEA med Scala-plugin}
\subsubsection{Sätta brytpunkter i IntelliJ}
\subsubsection{Stegad exekvering i IntelliJ}
\subsubsection{Inspektera variabler i IntelliJ}



\section{Åtgärda fel}

Ibland är det det svåraste att \emph{hitta} buggen medan själva buggrättningen visar sig trivial. Har du, till exemple, väl hittat den saknade uppräkningen av din loop-variabel är det uppenbart vad du ska göra.

Men ibland är det riktigt knepigt att åtgärda felet. Nedan sammanfattas några av de situationer som kan uppkomma, som gör att felrättningen blir långt ifrån trivial. 

\begin{itemize}
\item Kanske är själva algoritmen i grunden feltänkt och en helt ny algoritm behöver konstrueras. Att skapa nya algoritmer från grunden kan visa sig mycket svårt i en del fall. I fortsättningskurser får du lära dig mer om algoritmkonstruktionens svåra konst.

\item Kanske algoritmen fungerar för olika normalfall, medan undantagsfallen inte hanteras korrekt. Att på ett bra sätt hantera alla upptänkliga fall kan visa sig väldigt knepigt. Tyvärr är det ofta undantagsfallen som öppnar för säkerhetsluckor som elaka hackare kan utnyttja för att få systemet att krascha eller smittas av virus.

\item Kanske är problemet i sig väldigt svårt att lösa på ett korrekt sätt. Algoritmen kan vara riktigt knepig med många villkor, loopar och nästlade datastrukturer. Blir det fel i en sådan algoritm kan det ta lång tid att få ändringar att fungera och alla villkor, loopar och nästlade datastrukter att passa ihop. 

\item Medan man rättar en bug kan man råka att, av misstag, skapa nya buggar. Risken för detta är speciellt stor om koden är komplex. Ibland låter man till och med bli att åtgärda ett fel om systemet ändå fungerar hjälpligt i andra avseenden och risken är för stor att ändra innan systemet strukturerats om så att det blir lättare att ändra i.

\item Kanske växer exekveringstiden exponetiellt med datamängden och det kan vara i praktiken omöjligt att skriva ett program som i alla lägen blir färdigt inom rimlig tid för tillräckigt stora datamängder. Då får man försöka tänka ut kluriga genvägar och det kan vara riktigt svårt och ibland kräva mycket avancerad programmeringsteknik.
 
\end{itemize}
%!TEX root = ../compendium.tex

\chapter{Dokumentation}

\section{Vad gör ett dokumentationsverktyg?}

\section{scaladoc}

\section{javadoc}
%!TEX root = ../compendium.tex

\chapter{Byggverktyg}

\section{Vad gör ett byggverktyg?}

\section{Byggverktyget sbt}

\subsection{Installera sbt}

\subsection{Använda sbt}
%!TEX encoding = UTF-8 Unicode
%!TEX root = ../compendium2.tex


\chapter{Versionshantering och kodlagring}

\section{Vad är versionshantering?}

\textbf{Versionshantering}\footnote{\href{https://en.wikipedia.org/wiki/Version_control}{en.wikipedia.org/wiki/Version\_control}} \Eng{version control \textup{eller} revision control} av mjukvara innebär att hålla koll på olika versioner av koden i ett utvecklingsprojekt allteftersom koden ändras. Versionshantering är en deldisciplin inom \textbf{konfigurationshantering} \Eng{software configuration management} som inbegriper allt i processen för att identifiera, besluta, genomföra och följa upp ändringar.

En viktig del av versionshantering är att \textit{lagra} olika versioner av koden allt eftersom den utvecklas, så att tidigare versioner kan \textit{återskapas} vid behov. Ett bra verktygsstöd och en väldefinierad arbetsprocess för versionshanteringen, som alla i utvecklingsprojektet följer, möjliggör att flera utvecklare kan \textit{arbeta parallellt} med att sammanfoga \Eng{merge} varandras tillägg och ändringar i den gemensamma kodbasen utan att det blir kaos och förvirring.

God versionshantering är helt avgörande för utvecklarnas produktivitet, speciellt för stora projekt med många utvecklare som jobbar parallellt mot en omfattande kodbas med många olika interna och externa komponenter. 
Men även ett litet projekt med en enda utvecklare kan ha god nytta av ett versionshanteringsverktyg och ett disciplinerat förfarande för att namnge versioner, t.ex. för att kunna återskapa tidigare versioner av projektets olika kodfiler när en ändring visar sig mindre lyckad.   

Det finns flera olika modeller för hur kodlagringen sker:
\begin{itemize}
\item \textbf{lokal}; alla utvecklare jobbar i samma, lokala filsystem där alla olika versioner lagras.
\item \textbf{centraliserad}; ett repositorium (förk. repo), alltså en databas med koden, finns centralt på en server som alla jobbar mot med hjälp av en versionshanteringsklient.
\item \textbf{distribuerad}; alla utvecklare har sitt eget lokala repo och varje utvecklare initierar enskilt delning av ändringar mellan olika repo. 
\end{itemize}


\section{Versionshanteringsverktyget Git}

Det finns många olika versionshanteringsverktyg\footnote{\href{https://en.wikipedia.org/wiki/List_of_version_control_software}{https://en.wikipedia.org/wiki/List\_of\_version\_control\_software}}
 som använder olika modeller för kodlagring; lokal, centraliserad, distribuerad eller kombinationer därav. 
På senare tid har verktyget \textbf{Git}\footnote{\href{https://en.wikipedia.org/wiki/Git_(software)}{https://en.wikipedia.org/wiki/Git\_(software)}} fått en stark ställning, speciellt i öppenkällkodsvärlden. Git utvecklades ursprungligen av Linus Torvalds för att versionshantera Linuxkärnan, men har växt till ett omfattande öppenkällkodsprojekt med stor spridning och många användare och bidragsgivare. 

Git är skapad för \textbf{distribuerad} versionshantering där var och en kan jobba snabbt och smidigt i sitt eget lokala repo, utan att behöva vänta på att en klient ska synkronisera koden med ett centralt repo på en server över nätverket. Ändringar delas mellan repo på begäran av enskilda utvecklare. 

Varje ny version av koden lagras som en avgränsad mängd ändringar sedan förra versionen, en s.k. \textbf{commit}%
\footnote{På svenska kan t.ex. ''inlämning'' användas, men låneordet commit är redan etablerat.}%
, och hanteras internt av Git i en lokal databas i katalogen \code{.git} som ligger överst i din projektkatalog. Genom olika kommandon i terminalen, eller via en klient med ett grafiskt användargränssnitt, kan din kod överföras till och från den lokala koddatabasen, alternativt delas med andra repon via nätet. 

Det finns en välskriven bok kallad \textit{''Pro Git''} som förklarar Git på djupet och är tillgänglig fritt här: 
\url{https://git-scm.com/book/en/v2}.
Läs kapitel 1 och 2 så får du en bra grund att stå på. 

Dessa termer är bra att kunna utantill innan du kör igång med Git:
\newcommand{\TermItem}[3]{\item \textbf{#1} (\textit{substantiv}: #2, \textit{verb}: #3).}
\begin{itemize}

\item \textbf{repo} (\textit{substantiv}: ett repositorium, \textit{eng. a repository}) En koddatabas med ändringshistorik. 

\TermItem{commit}{en inlämning}{att lämna in} 
  En avgränsad mängd nya ändringar lämnas in i det lokala repot. Repots ändringshistorik utgörs av sekvensen av alla inlämningar.

\TermItem{push}{en leverans}{att leverera, att trycka upp} En eller flera inlämningar trycks upp till ett annat repo.

\TermItem{pull}{en hämtning}{att hämta, att dra ner} En eller flera inlämningar dras ner från ett annat repo.

\TermItem{merge}{en sammanfogning}{att sammanfoga} En eller flera inlämningar slås samman till en ny inlämning. 

\item \textbf{merge conflict} (\textit{substantiv}: en sammanfogningskonflikt, \textit{eng. a merge conflict}) Problem vid sammanfogning; ändringar kan inte enkelt sammanfogas på ett entydigt sätt.

\item \textbf{pull request} (förk. PR, \textit{substantiv}: en hämtningsbegäran, \textit{verb}: att begära en hämtning). Utvecklare A ber en annan utvecklare B att hämta en eller flera inlämningar från A:s repo och sammanfoga med B:s repo.

\end{itemize}

\subsection{Installera git}\label{subsection:install-git}

Git finns förinstallerat på LTH:s Linuxdatorer. Du kan kolla om Git redan finns på din maskin genom att skriva \code{git help} i terminalen. 

Det finns bra instruktioner om hur du installerar Git på din egen maskin här: \url{https://git-scm.com/book/en/v2/Getting-Started-Installing-Git}

VS Code har speciellt stöd för git och du kan inne ifrån VS Code göra t.ex. add, commit, push och pull via editorns grafiska gränssnitt. Läs mer här: \url{https://code.visualstudio.com/docs/editor/versioncontrol}

Det finns även många andra grafiska användargränssnitt till git, t.ex. \href{https://desktop.github.com/}{GitHub Desktop (Windows/Mac) eller \href{https://www.gitkraken.com/}{GitKraken (Linux/Windows/Mac)}}. Se fler exempel här: \url{https://git-scm.com/downloads/guis} 

%Om du inte vet vilken du ska välja, prova GitKraken som är gratis (men stängd) och finns för alla plattformar: \url{https://www.gitkraken.com/}.


\subsection{Anpassa Git}

Innan du börjar använda git, konfigurera ditt namn och din email med nedan terminalkommando, där du anger ditt namn i stället för \code{Förnamn Efternamn} och din mejladress i stället för \code{mejladr@plats.se}. Namnet och mejladressen kommer lagras i varje commit som du gör så att det går att se vem som har gjort en given ändring.
\begin{REPLnonum}
> git config --global user.name "Förnamn Efternamn"
> git config --global user.email mejladr@plats.se
\end{REPLnonum}

Läs mer om hur du gör andra inställningar här, t.ex. hur du anger vilken editor som git startar när du ska skriva commit-beskrivningar: \\ \url{https://git-scm.com/book/en/v2/Getting-Started-First-Time-Git-Setup}


\subsection{Använda git}

Nedan listas några vanliga terminalkommandon i Git.

\begin{itemize}[leftmargin=*]

\item Skapa ett repo i en katalog:
\begin{REPLnonum}
> cd myproject
> git init
\end{REPLnonum} 

\item Se vilka filer som ändrats och ännu ej lämnats in:
\begin{REPLnonum}
> git status
> git status -s
\end{REPLnonum} 

\item Se vilka ändringar som gjorts i filer som ännu ej lämnats in:
\begin{REPLnonum}
> git diff 
\end{REPLnonum} 

\item Se vilka inlämningar som finns i ändringshistoriken:
\begin{REPLnonum}
> git log 
> git log --oneline -5
\end{REPLnonum} 

\item Lägg till filer som ska ingå i nästa inlämning och gör sedan inlämningen; ge inlämningen en bra beskrivning som förklarar vad inlämningen omfattar:
\begin{REPLnonum}
> git add *.scala
> git commit -m 'initial project version'
\end{REPLnonum} 

\item Ångra alla tillägg inför inlämning (ändringarna finns kvar och kan läggas till igen om du vill):
\begin{REPLnonum}
> git reset 
\end{REPLnonum} 

\item Du kan skippa de senaste, ännu ej commitade, ändringar i filen \code{filename}, och göra ''\textit{undo}'', med kommandot \code{git checkout} på filen enligt nedan. Gör bara detta om du är helt säker på att du vill ångra dina senaste ändringar.
\\ \mbox{\colorbox{red!30}{VARNING!} Dina senaste ändringar i filen förloras för alltid; kan ej ångras!}   
\begin{REPLnonum}
> git checkout filename 
\end{REPLnonum} 

\item Man vill förhindra versionshantering av vissa filer, t.ex. binärkodsfiler så som \code{.class}-filer och andra genererade filer. Detta gör du genom att skapa en fil med namnet \code{.gitignore} och lägga in filändelser enligt nedan syntax, där \code{**/} avser alla kataloger och underkataloger och \code{*} kan vara vilken början på ett filnamn som helst. Symbolen \code{#} föregår en kommentarsrad.
\begin{Code}[language=]
# this is my .gitignore

# Java / Scala
**/*.class

# Sbt
**/target

\end{Code} 


\end{itemize}
 

\clearpage 
  
\section{Kodlagringsplatser på nätet}\label{section:code-hosting}

Många utvecklare använder kodlagringsplatser på nätet (''i molnet'') \Eng{code hosting} för att underlätta samarbete kring kod och för att dela med sig av öppen källkod. Det finns många olika kodlagringsplatser som kan användas gratis under vissa förutsättningar eller mot betalning med tillhörande extratjänster. 

\begin{oframed}
  \noindent \textbf{OBS!} Du får \emph{inte} lagra dina lösningar på kursens laborationer i ett öppet repo. Om du vill använda en kodlagringsplats måste du säkerställa att dina lösningar förblir i ett stängt repo utan att någon annan kan komma åt det.
\end{oframed}

Nedan beskrivs några vanliga nätplatser för öppen och sluten kodlagring, som alla är Git-baserade:

\begin{itemize}
\item  \textbf{GitHub}, \url{https://github.com}, är en av de mest populära kodlagringsplatserna för öppen källkod, men har även blivit en populär plats för jobbsökande utvecklare att visa upp sina  kodarbetsprover för framtida arbetsgivare. GitHub är gratis att använda för dig som privatperson. Många företag betalar GitHub för att lagra sin stängda kod med tilläggstjänster för att testa, bygga och driftsätta kod etc. Koden som styr själva kodlagringsplatsen GitHub är stängd, till skillnad från GitLab. GitHub köptes \href{https://computersweden.idg.se/2.2683/1.703485/microsoft-kop-github}{2018} av Microsoft för 65 miljarder kronor.

\item \textbf{GitLab}, \url{https://gitlab.com}, erbjuder gratis kodlagring för öppen källkod, men det är även gratis för privatpersoner och gemenskapsprojekt att ha stängda repo. Företag kan betala för stängd kodlagring med extratjänster för att testa, bygga och driftsätta kod etc. GitLab är i sig ett öppenkällkodsprojekt och koden som styr kodlagringsplatsen är öppen och fri. Detta innebär att du själv kan ladda ner koden och starta en kodlagringsplats. LTH har en GitLab-baserad kodlagringsplats här: \url{https://git.cs.lth.se}

\item \textbf{BitBucket}, \url{https://bitbucket.org}, är en populär kodlagringsplats både för öppen och stängd källkod och drivs av det australiensiska företaget Atlassian. Det är gratis för privatpersoner och små team att ha både öppna och slutna repon, men bara om det är få bidragsgivare. Kostnader tillkommer om antalet bidragsgivare kommer över en viss nivå. Universitetsanställda och studenter kan få mer gynnsamma villkor efter ansökan. Atlassian erbjuder en hel verktygssvit för att hantera buggar och samarbeta över nätet. BitBucket stödjer, förutom Git, även andra versionshanteringsverktyg.

\end{itemize}

\subsubsection{Använda kodlagringsplatser}

Om du inte redan gjort det är det bra om du registrerar ditt användarnamn, förslagsvis \code{fornamnefternamn} som ett ord utan svenska tecken med små bokstäver, på någon eller alla av ovan sajter, dels för att paxa ditt namn och dels för börja samarbeta med utvecklare världen över. Det är bra att välja \textit{ett} användarnamn för \textit{alla} kodlagringsplatser på nätet, speciellt om du jobbar med öppen källkod där ditt namn kommer associeras med alla de kodbidrag du gör under ditt yrkesliv.

Om du inte vet vilken sajt du ska välja, börja med \url{https://github.com}. Om du vill att även kodlagringssajten ska drivas av öppen källkod, testa \url{https://gitlab.com}.

Med en Git-baserad kodlagringsplats får du möjlighet att synka ditt lokala repo mot en server på nätet med hjälp av \code{git}-kommandon i terminalen eller via en Git-klient med grafiskt användargränssnitt, se avsnitt \ref{subsection:install-git}. 

Innan du börjar använda en kodlagringsplats är det bra att sätta sig in i begreppen nedan.

\begin{itemize}
\TermItem{clone}{en klon är kopia av ett (nätlagrat) repo}{att klona, att skapa en kopia} Genom att klona ett repo som ligger på en nätlagringsplats kan du bygga, undersöka och vidareutveckla koden lokalt på din dator. Om du har rättigheter att lämna in kod till det centrala originalet kan du pusha dina commits direkt via terminalkommando eller Git-klient.

\TermItem{fork}{en förgrening av ett helt repo}{att förgrena ett repo, att ''forka''} Genom att förgrena ett repo skapar du en kopia, normalt även den nätlagrad på en kodlagringsplats, som du kan utveckla separat från originalet. Det blir då möjligt för dig att lämna in ändringar och trycka upp dem, även om du inte har rättigheter att leverera (''pusha'') till originalet. Gör en ändringsbegäran (Pull Request, PR) om du vill bidra med dina ändringar, så kan ägaren av originalet sedan välja att sammanfoga (''merga'') dina ändringar med originalet. Många nätlagringsplatser, så som GitHub, har en speciell knapp som du trycker på för att enkelt skapa en fork av ett repo under din användare. 

\item \textbf{upstream} (\textit{preposition}: uppströms, \textit{substantiv}: uppströmsrepo) Ett uppströmsrepo utgör original till ett förgrenat repo (en ''fork''). 
\begin{itemize}[noitemsep,nolistsep]

\item Här beskrivs hur du länkar en förgrening uppströms: \\ 
{\small\url{https://help.github.com/articles/configuring-a-remote-for-a-fork/}}

\item Här beskrivs hur du synkar en förgrening uppströms:\\
{\small\url{https://help.github.com/articles/syncing-a-fork/}}

\end{itemize}

\end{itemize}

Om du vill bidra till ett öppenkällkodsprojekt, börja med att forka repot på kodlagringsplatsen och sedan klona repot till din lokala dator. Därefter kan du commita ändringar och pusha till din fork och slutligen göra en pull request från din fork till upstream. Läs om hur ett bidrag kan gå till i avsnitt \ref{section:OSS-contribution-example}.

Här följer några användbara kommandon:

\begin{itemize}
\item Skapa en lokal kopia av ett fjärran \Eng{remote} repo; här visas hur du klonar kursens repo från GitHub:
\begin{REPLnonum}
$ git clone --depth 1 https://github.com/lunduniversity/introprog
\end{REPLnonum} 

\item Dra ner nya inlämningar från ett fjärran repo:
\begin{REPLnonum}
$ git pull 
\end{REPLnonum} 

\item Trycka upp nya lokala inlämning till ett fjärran repo:
\begin{REPLnonum}
$ git push 
\end{REPLnonum} 

\end{itemize}



%!TEX root = ../compendium.tex

\chapter{Integrerad utvecklingsmiljö}

\section{Vad är en IDE?}

\section{ScalaIDE och Eclipse}

\subsection{Installera ScalaIDE}

\section{Handledning ScalaIDE/Eclipse}
%%!TEX encoding = UTF-8 Unicode
%!TEX root = ../compendium2.tex

\chapter{Skapa webb-appar med ScalaJS}\label{appendix:scalajs}

\TODO

\noindent Läs först appendix \ref{appendix:build}

\begin{itemize}
\item \url{http://scala-js.org/}
\end{itemize} %TODO!!
\setcounter{chapter}{9} %next after 9 is J in \Alph
%!TEX encoding = UTF-8 Unicode
%!TEX root = ../compendium2.tex

%!TEX encoding = UTF-8 Unicode

%!TEX root = ../compendium2.tex

\chapter{Introduktion till Java}\label{appendix:java}
\clearpage\section{Teori}
%!TEX encoding = UTF-8 Unicode
%!TEX root = ../lect-w12.tex

%%%



\Subsection{Jämförelse Scala och Java}
\begin{Slide}{Övning \texttt{scalajava} och labb \texttt{javatext}}\SlideFontSmall
%\Emph{Labbförberedelse:}
\begin{itemize}
\item Övning \texttt{scalajava}:
\begin{itemize}\SlideFontTiny
\item Översätta Java till Scala och från Scala till Java
\item Undersöka autoboxning \Eng{autoboxing}
\item Använda \code{import scala.jdk.CollectionConverters.*}
%\item[] Utfasad \Eng{deprecated} sedan Scala 2.13.0: \code{import scala.collection.JavaConverters._}
\end{itemize}
\item Laboration \code{javatext}:
\begin{itemize}\SlideFontTiny
  \item Gör ett textspel för terminalen huvudsakligen i Java men vissa delar i Scala, enligt krav, tips och inspiration i labb-instruktionerna.

\end{itemize}
\item Tips: Scala CLI och sbt kan blanda \code{.scala} och \code{.java} i samma projekt.
\end{itemize}
\end{Slide}


\begin{Slide}{''Hello world!'' i Java.}

\noindent Ett minimalt huvudprogram i Java:
\begin{Code}[language=Java]
public class Hello {
    public static void main(String[] args) {
        System.out.println("Hello world!");
    }
}
\end{Code}

% Motsvarande i Scala:
% \begin{Code}
% object Hello:
%   def main(args: Array[String]): Unit = println("Hello world!")
% \end{Code}

% I Scala (men inte Java) går det även att skriva typsäkra main-funktioner på toppnivå med annoteringen \code{@main}.
% \begin{Code}
% @main def run(n: Int): Unit = println("Hello world!" * n)
% \end{Code}
\end{Slide}


\begin{Slide}{Testa Java i \texttt{jshell}}
\begin{itemize}\SlideFontSmall
  \item Java har en motsvarighet till Scalas REPL: kommandot \code{jshell}
\end{itemize}
\begin{REPLsmall}
> jshell
|  Welcome to JShell -- Version 11.0.11
|  For an introduction type: /help intro

jshell> /help intro
|  
|                                   intro
|                                   =====
|  
|  The jshell tool allows you to execute Java code, getting immediate results.
|  You can enter a Java definition (variable, method, class, etc), like:  int x = 8
|  or a Java expression, like:  x + x
|  or a Java statement or import.
|  These little chunks of Java code are called 'snippets'.
|  
|  There are also the jshell tool commands that allow you to understand and
|  control what you are doing, like:  /list
|  
|  For a list of commands: /help

jshell> 
\end{REPLsmall}
\end{Slide}

\begin{Slide}{Grundläggande likheter och skillnader Java--Scala}\SlideFontSmall
\Emph{Några likheter:}
\begin{itemize}\SlideFontTiny
\item Kompilerar till bytekod som kör på JVM på många olika plattformar.
\item Statisk typning: ger snabb maskinkod, kompilatorn kan ge stöd vid förändring av kod (s.k. refactoring) och hittar många buggar redan vid kompilering.
\end{itemize}

\noindent \Emph{Liknande men} \Alert{viss skillnad:}
\ifkompendium
\\~\\
\else
\vspace{-1em}\begin{multicols}{2}
\fi
\Emph{Java}
\begin{itemize}\SlideFontTiny
\item \Emph{Objektorientering}, men inte ''äkta'' \Eng{pure} eftersom inte alla värden är objekt

\item Primitiva typer är inte objekt; representeras effektivt, normalt \Emph{utan boxning}

\item Visst stöd för \Emph{funktionsprogrammering}

%\item Typer måste anges, ibland två gånger (variabeldeklaration + instansiering)
\end{itemize}

\ifkompendium\else
\columnbreak
\fi

\noindent\Emph{Scala}
\begin{itemize}\SlideFontTiny
\item \Emph{Äkta objektorienterat} eftersom alla värden är objekt, även funktioner

\item \code{AnyVal}-instanser är äkta objekt men representeras ändå effektivt, normalt \Emph{utan boxning}

\item Omfattande stöd för \Emph{funktionsprogrammering}

%\item Typer kan för det mesta härledas av kompilatorn.
\end{itemize}
\ifkompendium\else
\end{multicols}
\fi
\end{Slide}



\begin{Slide}{Huvudprogram i Scala och Java}
\begin{multicols}{2}
  \noindent\Emph{Scala 2}
\begin{CodeSmall}[basicstyle=\footnotesize\ttfamily\SlideFontSize{6}{8},backgroundcolor=\color{white},
  frame=none]
object Main {
  def main(args: Array[String]): Unit = {
    println("Hello!")
  }
}
\end{CodeSmall}

\columnbreak

\noindent\Emph{Java}
\begin{CodeSmall}[language=Java,basicstyle=\footnotesize\ttfamily\SlideFontSize{6}{8},backgroundcolor=\color{white},
  frame=none]
public class JMain {
  public static void main(String[] args){
    System.out.println("Hello!");
  }
}
\end{CodeSmall}
\end{multicols}

\pause\vspace{1em}\noindent\Emph{Scala 3}
\begin{CodeSmall}[basicstyle=\footnotesize\ttfamily\SlideFontSize{6}{8},backgroundcolor=\color{white},
  frame=none]
@main 
def sumFirst(n: Int, xs: Int*): Unit = println(xs.take(n).sum)
  
\end{CodeSmall}

\end{Slide}


\begin{Slide}{Loopa genom argumenten i ett Java-huvudprogram}
\begin{REPLnonum}
> code HelloJavaArgs.java
\end{REPLnonum}
\begin{Code}[language=Java]
public class HelloJavaArgs {
    public static void main(String[] args) {
    int i = 0;
    while (i < args.length) {
      System.out.println(args[i]);
      i = i + 1;
    }
  }
}
\end{Code}
Kompilera och kör:
\begin{REPL}
> javac HelloJavaArgs.java
> java HelloJavaArgs hej gurka tomat
hej
gurka
tomat
\end{REPL}
\end{Slide}


\begin{Slide}{HIGHSCORE implementerad i Java}
\begin{Code}[language=Java]
import java.util.Scanner;

public class HighScore {
    public static void main(String[] args){
        Scanner scan = new Scanner(System.in);
        System.out.println("Hur många poäng fick du?");
        int points =  scan.nextInt();
        System.out.println("Vad var highscore före senaste spelet?");
        int highscore = scan.nextInt();
        if (points > highscore) {
            System.out.println("GRATTIS!");
        } else {
            System.out.println("Försök igen!");
        }
    }
}
\end{Code}
\end{Slide}



\begin{Slide}{Några saker som finns i Scala men inte i Java}\SlideFontTiny
\ifkompendium\else
\vspace{-1em}\begin{multicols}{2}
\fi
\begin{itemize}\SlideFontSize{6.8}{7.2}
\item \code{case}-klasser

\item Lokala funktioner

\item Metoder som operatorer

\item Infix operatornotation

\item Defaultargument

\item Namngivna argument

\item Engångsinitialisering: \code{val}

\item Fördröjd initialisering: \code{lazy val}

\item Enhetlig access för \code{def}, \code{val}, \code{var}

\item Egna setters med \code{def namn_=}

\item Namnanrop, fördröjd evaluering

\item Matchning, mönster och garder

\item Klassparametrar, primärkonstruktor

\item Singelobjekt: \code{object}

\item Kompanjonsobjekt

\item Inmixning: \code{trait}

\item \code{for}-\code{yield}-uttryck

\item Block är uttryck; slipper \code{return}

\item Tomma värdet () av typen \code{Unit}

\item \code{Option}, \code{Some}, \code{None}  (Java har \code{Optional} som ger en del, men inte allt...)

\item \code{Try}, \code{Success}, \code{Failure}

\item Samlingarna i Scalas standardbibliotek, speciellt de \Emph{oföränderliga} samlingarna \code{Vector}, \code{Map}, \code{Set}, \code{List}, etc.

\item Innehållslikhet med \code{==} för oföränderliga strukturer, inkl. \code{< <= > >= } på strängar

\item \Alert{Enhetlig} användning av samlingar \Emph{inkl. Array} (förutom innehållslikhet för Array)

\item Kontextuella abstraktioner \code{given} \code{using}

\item Mer precis synlighet \code{private[mypack]}

\item Namnändring vid \code{import}

\item Flexibel filstruktur och filnamngivning

\item Flexibel nästling av klasser, objekt, traits

\item Typ-alias och abstrakta typer med \code{type}

\item Extensionsmetoder \code{extension}

\item ...
\end{itemize}
\ifkompendium\else
\end{multicols}
\fi
\end{Slide}


\begin{Slide}{Några saker som finns i Java men inte i Scala}
\ifkompendium\else
\vspace{-0.7em}\begin{multicols}{2}\SlideFontTiny
\fi
\begin{itemize}
\item[\textbf{\texttt{+}}] Snabbare kompilering

\item[\textbf{\texttt{+}}] Mognare verktygsstöd

\item Variabeldeklaration utan initialisering

\item Förändringsbara parametrar

\item C-liknande prefix- och postfix-inkrementering och -dekrementering: \code{i++ ++i i-- --i}

\item C-liknande \code{for}-sats

\item Semikolon krävs efter alla satser

\item \jcode{return} krävs i alla metoder som har returvärde

\item Nyckelordet \jcode{void}

\item Parenteser efter alla metoder

\item Specialsyntax för indexering av array \code{[]} ej som i andra samlingar

\item Hoppa ut ur loop med \jcode{break} \\ \href{https://docs.oracle.com/javase/tutorial/java/nutsandbolts/branch.html}{\SlideFontSize{6}{8.5}docs.oracle.com/javase/tutorial/}

\item \jcode{switch} ''faller igenom'' om du glömmer skriva \jcode{break}

\item Kontrollerade undantag \Eng{checked exceptions} och \jcode{throws}

\item ...
\end{itemize}
\ifkompendium\else
\end{multicols}
\fi
\end{Slide}


\begin{Slide}{Begränsningar med funktionsprogrammering i Java}
\begin{itemize}\SlideFontTiny
\item[] Av alla dessa funktionsprogrammeringskoncept i Scala...
\begin{itemize}\SlideFontSize{6}{8}
\item \Emph{överlagring}
\item \Emph{anonyma funktioner}
\item mönstermatchning
\item funktioner etc. på toppnivå
\item utelämna tom parameterlista (enhetlig access)
\item defaultargument
\item namngivna argument
\item lokala funktioner
\item funktioner som äkta värden
\item klammerparentes vid ensam paramenter
\item multipla parameterlistor
\item egendefinierade kontrollstrukturer
\item namnanrop (fördröjd evaluering)
\item stegade funktioner (''Curry-funktioner'')
\item fångad variablelrymd i funktionsobjekt (''closure'')
\item ad hoc polymorfism (''typklasser'')
\item kontextuella abstraktioner
\end{itemize}
\item[] ...kan man i Java %\footnote{\SlideFontTiny\href{https://en.wikipedia.org/wiki/Java_version_history}{en.wikipedia.org/wiki/Java\_version\_history}}  
endast göra: \Emph{överlagring} (''overloading'') och \Emph{anonyma funktioner} (''lambda'') där det senare har starka begränsningar. %\footnote{\SlideFontTiny\href{https://en.wikipedia.org/wiki/Anonymous_function\#Java_Limitations}{en.wikipedia.org/wiki/Anonymous\_function\#Java\_Limitations}}

%\item \vspace{0.5em} En av de saker jag saknar mest i Java: \Alert{lokala funktioner}!
%\item[] Det är \Alert{kombinationen} av alla koncept som \Alert{skapar uttryckskraften} i Scala.
\end{itemize}
\ifkompendium\else
\vfill\SlideFontSize{6}{8} 
\fi Läs mer om Java här:\\
\url{{https://en.wikipedia.org/wiki/Java_version_history}}
\url{https://en.wikipedia.org/wiki/Anonymous_function\#Java_Limitations}

\end{Slide}

\begin{Slide}{Grundtyper i Scala och primitiva typer Java}\SlideFontSmall
\begin{table}[H]
\renewcommand{\arraystretch}{1.4}
\begin{tabular}{l|l|l|l}
\Alert{Grundtyp} i &  Antal                &      Omfång&\Alert{primitiv typ} i\\
  \Emph{Scala} & bitar & minsta/största värde &\Emph{Java} \\ \hline
\texttt{Byte}   &  8  & $-2^7$ ... $2^7-1$   & \texttt{byte} \\
\texttt{Short}  &  16 & $-2^{15}$ ... $2^{15}-1$ & \texttt{short} \\
\texttt{Char}   &  16 & $0$ ... $2^{16}-1$ & \texttt{char} \\
\texttt{Int}    &  32 & $-2^{31}$ ... $2^{31}-1$ & \texttt{int} \\
\texttt{Long}   &  64 & $-2^{63}$ ... $2^{63}-1$ & \texttt{long} \\
\texttt{Float}  &  32 & ± $3.4028235 \cdot 10^{38}$  & \texttt{float} \\
\texttt{Double} &  64 & ± $1.7976931348623157 \cdot 10^{308}$ & \texttt{double} \\
\end{tabular}
\end{table}
\end{Slide}
  


\begin{Slide}{Javas switch-sats}\SlideFontSmall
De flesta C-liknande språk (men inte Scala) har en \jcode{switch}-sats som man kan använda istället för (vissa) nästlade if-else-satser:
\javainputlisting[basicstyle=\ttfamily\SlideFontSize{5}{6}\selectfont]{../compendium/examples/match/Switch.java}
{\SlideFontTiny
\code{switch} i Java har stora begränsningar (fungerar t.ex. bara på primitiva typer och några till, tex String); i framtiden planerar man att anamma en del av det \code{match} i Scala kan.
}
\end{Slide}


\begin{Slide}{Javas switch-sats utan break}\SlideFontSmall
Saknad \jcode{break}-sats ''faller igenom'' till efterföljande gren:

\javainputlisting[basicstyle=\ttfamily\SlideFontSize{6}{7}\selectfont]{../compendium/examples/match/SwitchNoBreak.java}
En glömd \jcode{break} kan ge svårhittad bugg...
\end{Slide}

\begin{Slide}{Javas switch-sats med glömd break}\SlideFontSmall

\vspace{-0.5em}\javainputlisting[basicstyle=\ttfamily\SlideFontSize{5.5}{6.8}\selectfont]{../compendium/examples/match/SwitchForgotBreak.java}

\vspace{-0.7em}\pause
\begin{REPLsmall}
> java SwitchForgotBreak
Skriv grönsak:
gurka
gott!
gott!
\end{REPLsmall}

\end{Slide}


% \begin{Slide}{Regler för identifierare i Java}\footnotesize
% När kompilatorn ''läser''  \footnote{man säger ofta ''parsa'' i stället för ''läsa'' när kompilatorn tolkar koden} koden och och försöker hitta variabelnamn, antar den att du följer de entydiga syntaktiska reglerna för språket.  \\ \vskip1em För namn i Java gäller följande regler: %https://docs.oracle.com/javase/tutorial/java/nutsandbolts/variables.html
% \begin{itemize}
% \item Namn får inte vara \href{https://docs.oracle.com/javase/tutorial/java/nutsandbolts/_keywords.html}{reserverade ord}
% \item Stora och små bokstäver spelar roll \Eng{case sensistive} \\ \lstinline{int highScore;} och \lstinline{int highscore;} ger alltså två \textit{olika} variabler
% \item Namnet måste börja med en bokstav, ett understreck \_ eller ett dollartecken \$
% \item Namn får \textit{inte} innehålla blanktecken
% \item Namn får innehålla bokstäver, siffror, understreck \_ och dollartecken \$, men \textit{inte} andra specialtecken (alltså inte \lstinline~%&@!{(})/+-*~ etc.)
% \end{itemize}
% \end{Slide}

\ifkompendium  % bad hardcoded hack to fix column alignment problems
\clearpage
\fi

\begin{Slide}{Syntax för variabeldeklaration i Scala och Java}\SlideFontSmall
Exempel på variabeldeklarationer i
\begin{multicols}{2}
\noindent\Emph{Scala}
\begin{CodeSmall}[basicstyle=\ttfamily\SlideFontSize{7}{10},backgroundcolor=\color{white},
  frame=none]
  var i1: Int = 0
  var i2 = 0
  var i3 = 0: Int
  var p1: Point = new Point(0, 0)
  var p2 = new Point(0, 0)
  var (x, y) = (0, 0)
  val a = 0
  final val Constant = 42
\end{CodeSmall}
\begin{itemize}\SlideFontTiny
\item i2 härledd typ; går ej i Java 8,9 men finns med \code{var} i Java 10

\item i3 typ varhelst i uttryck; går ej i Java

\item (x, y) mönster i init; går ej i Java

\item \code{val} ger ''engångsinit''; ingen exakt motsvarighet i Java men \code{final} kan ofta användas i stället
\end{itemize}

\columnbreak

\noindent\Emph{Java}
\begin{CodeSmall}[language=Java,morekeywords={var},basicstyle=\ttfamily\SlideFontSize{7}{10},backgroundcolor=\color{white},
  frame=none]
  int i1 = 0;
  var i2 = 0; // från Java 10
  int i4;
  Point p1 = new Point(0, 0);
  var p2 = new Point(0, 0); 
  final int CONSTANT = 42;
\end{CodeSmall}
\begin{itemize}\SlideFontTiny
\item i4 ej explicit init; går ej i Scala
\end{itemize}
\end{multicols}

\end{Slide}


\begin{Slide}{For-sats i Scala och Java}
\begin{multicols}{2}
\noindent\Emph{Scala}
\begin{CodeSmall}[basicstyle=\ttfamily\SlideFontSize{6}{8},backgroundcolor=\color{white},
  frame=none]
val s = "Abbasillen"

// Loopa över index framlänges:

for i <- 0 until s.length do
  println(s(i))


// Loopa över index baklänges:

for i <- s.length-1 to 0 by -1 do
  println(s(i))

\end{CodeSmall}

\columnbreak

\noindent\Emph{Java}
\begin{CodeSmall}[language=Java,basicstyle=\ttfamily\SlideFontSize{6}{8},backgroundcolor=\color{white},
  frame=none]
String s = "Abbasillen";

// Loopa över index framlänges:

for (int i = 0; i < s.length(); i++) {
    System.out.println(s.charAt(i));
}

// Loopa över index baklänges:

for (int i = s.length()-1; i >= 0; i--) {
    System.out.println(s.charAt(i));
}
\end{CodeSmall}
\end{multicols}
I Scala är \code{s.indices} att föredra!
\end{Slide}

\ifkompendium  % bad hardcoded hack to fix column alignment problems
\clearpage
\fi


\begin{Slide}{For-sats i Scala med indices}
\begin{multicols}{2}
\noindent\Emph{Scala}
\begin{CodeSmall}[basicstyle=\ttfamily\SlideFontSize{6}{8},backgroundcolor=\color{white},
  frame=none]
val s = "Abbasillen"

// Loopa över index framlänges:

for i <- s.indices do
  println(s(i))


// Loopa över index baklänges:

for i <- s.indices.reverse do
  println(s(i))

\end{CodeSmall}

\columnbreak

\noindent\Emph{Java}
\begin{CodeSmall}[language=Java,basicstyle=\footnotesize\ttfamily\SlideFontSize{6}{8},backgroundcolor=\color{white},
  frame=none]
String s = "Abbasillen";

// Loopa över index framlänges:

for (int i = 0; i < s.length(); i++) {
    System.out.println(s.charAt(i));
}

// Loopa över index baklänges:

for (int i = s.length()-1; i >= 0; i--) {
    System.out.println(s.charAt(i));
}
\end{CodeSmall}
\end{multicols}
\end{Slide}


\begin{Slide}{For-satser och arrayer i Java}\SlideFontSmall
En for-sats i Java har följande struktur:
\begin{Code}[language=Java, basicstyle=\fontsize{10}{12}\ttfamily\selectfont]
for (initialisering; slutvillkor; inkrementering) {
    sats1;
    sats2;
    ...
}
\end{Code}
En primitiv heltals-array deklareras så här i Java:
\begin{Code}[language=Java, basicstyle=\fontsize{9}{11}\ttfamily\selectfont]
int[] xs = new int[42];  // rymmer 42 st heltal, init 0:or
int[] ys = {10, 42, -1}; // initera med 3 st heltal
\end{Code}
Exempel på for-sats: fyll en array med 1:or
\begin{Code}[language=Java, basicstyle=\fontsize{9}{11}\ttfamily\selectfont]
for (int i = 0; i < xs.length; i++){ 
  xs[i] = 1;    // indexera med [i]
}
\end{Code}

\end{Slide}
  
\begin{Slide}{Implementation av SEQ-COPY i Java med \texttt{for}-sats}
\begin{minipage}{0.55\textwidth}
\vspace{-0.5em}
\javainputlisting[numbers=left,numberstyle=,basicstyle=\fontsize{6.5}{8}\ttfamily\selectfont]{../compendium/examples/workspace/w05-seqalg/src/SeqCopyForJava.java}
\end{minipage}
\begin{minipage}{0.44\textwidth}\SlideFontTiny\vspace{-1.5em}\ifkompendium\small\fi
~~~Lite syntax och semantik för Java:
\begin{itemize}
\item En Java-klass med enbart statiska medlemmar motsvarar ett singelobjekt i Scala.

\item Typen kommer \Alert{före} namnet.

\item Man \Alert{måste} skriva \code{return}.

\item Man \Alert{måste} ha semikolon efter varje sats.

\item Metodnamn \Alert{måste} följas av parenteser; om inga parametrar finns används \code{()}

\item En array i Java är inget vanligt objekt, men har ett ''attribut'' \code{length} som ger antal element.

\item \Emph{Övning}: skriv om med \code{while}-sats i stället; har samma syntax i Scala \& Java.

\end{itemize}
\end{minipage}

\end{Slide}
  


\begin{Slide}{Element för element med speciell for-each-sats i Java}
\begin{multicols}{2}
\noindent\Emph{Scala}
\begin{CodeSmall}[basicstyle=\ttfamily\SlideFontSize{6}{8},backgroundcolor=\color{white},
  frame=none]
val s = "Abbasillen"

// Loopa över alla tecken:

for ch <- s do println(ch)


\end{CodeSmall}

\columnbreak

\noindent\Emph{Java}
\begin{CodeSmall}[language=Java,basicstyle=\ttfamily\SlideFontSize{6}{8},backgroundcolor=\color{white},
  frame=none]
String s = "Abbasillen";

// Loopa över alla tecken:

for (char ch: s.toCharArray()) {
  System.out.println(ch);
}
\end{CodeSmall}
\end{multicols}

\pause
{\noindent\SlideFontSmall
\code{s.foreach(println)  } går ej i Java men
från Java 8 finns metoden \code{chars} som ger en \code{IntStream} och då kan man: \\
 \jcode{str.chars().forEachOrdered(i -> System.out.println((char) i));}
 }
\end{Slide}





%\Subsection{Klasser i Java}

\begin{Slide}{Typisk utformning av Java-klass}
Typisk ''anatomi'' hos en Java-klass:
\begin{Code}[language=Java]
public class Klassnamn {
    attribut, normalt privata
    konstruktorer, normalt publika
    metoder: publika getters, och vid förändringsbara objekt även setters
    metoder: privata abstraktioner för internt bruk
    metoder: publika abstraktioner tänkta att användas av klientkoden
}
\end{Code}
\href{http://www.oracle.com/technetwork/java/codeconventions-141855.html#1852}{\SlideFontSize{9}{8}www.oracle.com/technetwork/java/codeconventions-141855.html\#1852}
\end{Slide}


\begin{Slide}{Statiska medlemmar i Java}
\begin{itemize}
\item Man kan \Alert{inte} deklarera explicita singelobjekt i Java och det finns inget nyckelord \code{object}.

\item I stället kan man deklarera \Emph{statiska medlemmar} i en klass med Java-nyckelordet \jcode{static}.

\item Exempel på hur vi ska göra detta inuti en klassen \code{JPerson}:

\begin{Code}[language=Java,basicstyle=\SlideFontSize{10}{12}\ttfamily\selectfont]
public static final int ADULT_AGE = 18;
\end{Code}

\item Effekten blir den samma som ett singelobjekt i Scala:
\begin{itemize}
\item Alla statiska medlemmar i en Java-klass allokeras automatiskt och hamnar i en egen singulär ''klassinstans'' som existerar oberoende av de dynamiska instanserna.
\item De statiska medlemmarna accessas med punktnotation genom klassnamnet, \Alert{utan} \code{new}:
\begin{Code}[language=Java,basicstyle=\SlideFontSize{11}{13}\ttfamily\selectfont]
System.out.println(JPerson.ADULT_AGE);
\end{Code}

\end{itemize}


\end{itemize}
\end{Slide}
  




\begin{Slide}{Exempel: oföränderlig klass i Scala och Java}
\ifkompendium
\noindent\Emph{Scala:}
\else
\SlideFontTiny\vspace{-1em}\begin{multicols}{2}
\fi
\begin{CodeSmall}[basicstyle=\ttfamily\SlideFontSize{5.7}{6.7}]
class Person(val name: String, val age: Int):
  def isAdult = age >= Person.AdultAge

object Person:
  val AdultAge = 18
\end{CodeSmall}

\ifkompendium
\noindent\Emph{Java:}
\else
\columnbreak
\fi

\pause
\begin{CodeSmall}[language=Java,basicstyle=\ttfamily\SlideFontSize{5.7}{6.7}]
public class JPerson {
    private String name;
    private int age;
    public static final int ADULT_AGE = 18;

    public JPerson(String name, int age) {
      this.name = name;
      this.age = age;
    }

    public String getName() {
        return name;
    }

    public int getAge() {
        return age;
    }

    public boolean isAdult() {
        return age >= ADULT_AGE;
    }
}
\end{CodeSmall}
Lär dig detta mönster för en typisk Java-klass utantill så du snabbt får grejerna på plats!
\ifkompendium\\\else
\end{multicols}
\pause\vspace{-13em}%
\fi%
~\\\Alert{Övning:}\\
\ifkompendium
Gör \code{Person} + \code{JPerson} \Alert{förändringsbara} så att namnet och åldern går att uppdatera och följande krav uppfylls:
\else
Gör \code{Person} + \code{JPerson} \Alert{förändringsbara}\\så att namnet och åldern går att uppdatera\\och följande krav uppfylls:
\fi
\begin{itemize}
\item namnet ska ges vid konstruktion,
\item åldern ska initieras till 0 vid konstr.,
\item åldern ska aldrig kunna bli negativ.
\end{itemize}
\end{Slide}


\begin{Slide}{Exempel: Scala-klassen Complex}
\scalainputlisting[basicstyle=\ttfamily\SlideFontSize{7}{9},numbers=left,numberstyle=\SlideFontSize{7}{7}\ttfamily\selectfont]{../compendium/examples/complex7.scala}
{\SlideFontTiny Scala:~~\url{https://github.com/lunduniversity/introprog/blob/master/compendium/examples/complex7.scala}}

{\SlideFontTiny Java:~~~\url{https://github.com/lunduniversity/introprog/blob/master/compendium/examples/JComplex.scala}}

\end{Slide}
  

\begin{Slide}{Exempel: Motsvarande Java-klassen JComplex}\SlideFontTiny\SlideOnly{\vspace{-1em}}
\javainputlisting[basicstyle=\SlideFontSize{5}{5.7}\ttfamily\selectfont,numbers=left,numberstyle=\SlideFontSize{5}{5}\ttfamily\selectfont]{../compendium/examples/JComplex.java}
\end{Slide}

  
  


\begin{Slide}{Exempel: Använda JComplex i Scala-kod}
\begin{REPL}
$ javac JComplex.java
$ scala
Welcome to Scala 2.12.9 (OpenJDK 64-Bit Server VM, Java 1.8.0_222).
Type in expressions for evaluation. Or try :help.

scala> val jc1 = JComplex(3, 4)
jc1: JComplex = 3.0 + 4.0i

scala> val polarForm = (jc1.getR, jc1.getFi)
polarForm: (Double, Double) = (5.0,0.6435011087932844)

scala> val jc2 = JComplex(1, 2)
jc2: JComplex = 1.0 + 2.0i

scala> jc1 add jc2
res0: JComplex = 4.0 + 6.0i
\end{REPL}
\end{Slide}




\begin{Slide}{Exempel: Använda JComplex i Java-kod}\SlideFontSmall
\javainputlisting[basicstyle=\SlideFontSize{8}{10}\ttfamily\selectfont]{../compendium/examples/JComplexTest.java}
\begin{itemize}
\item I Java måste man skriva \code{new}.  
\item I Java måste man skriva \Alert{tomma parentes-par} efter metodnamnet vid \Alert{anrop av parameterlösa metoder}.

\item \Alert{Tupler finns inte} i Java, så det går inte på ett enkelt sätt att skapa par av värden som i Scala; ovan görs polär form till en sträng för utskrift.

\item \Alert{Operatornotation för metoder finns inte} i Java, så man måste i Java använda punktnotation och skriva: \code{jc1.add(jc2)}
\end{itemize}
\end{Slide}
  
  
  







\begin{Slide}{Exempel: Förändringsbar klass i Scala och Java}
\ifkompendium
\noindent\Emph{Scala:}
\else  
\SlideFontTiny\vspace{-1.75em}\begin{multicols}{2}
\fi 
\begin{CodeSmall}[basicstyle=\ttfamily\SlideFontSize{5}{6}]
class MutablePerson(var name: String):
  private var _age = 0

  def age: Int = _age

  def age_=(a: Int): Unit =
    if (a >= 0) _age = a else _age = 0  
      // eller hellre kasta undantag?

  def isAdult: Boolean =
    age >= MutablePerson.AdultAge

object MutablePerson:
  val AdultAge = 18
\end{CodeSmall}

\ifkompendium
~\\\noindent\Emph{Java:}
\else
\columnbreak
\fi 

\pause

\begin{CodeSmall}[language=Java,basicstyle=\ttfamily\SlideFontSize{5}{6}]
public class JMutablePerson {
    private String name;
    private int age = 0;
    public static final int ADULT_AGE = 18;

    public JMutablePerson(String name) {
      this.name = name;
    }

    public String getName() {
        return name;
    }

    public void setName(String name) {
        this.name = name;
    }

    public int getAge() {
        return age;
    }

    public void setAge(int age) {
        if (age >= 0) {
          this.age = age;
        } else {
          this.age = 0;
        }
    }

    public boolean isAdult() {
        return age >= ADULT_AGE;
    }
}
\end{CodeSmall}
\ifkompendium\else
\end{multicols}
\fi
\end{Slide}


\ifkompendium\pagebreak\fi

% \begin{Slide}{Övning: Implementera dessa specifikationer}
% \ifkompendium\else  
% \begin{multicols}{2}
% \fi
% {\hskip-0.31em\colorbox{black!70}{\parbox{\dimexpr0.44\textwidth-20\fboxsep-1.9\fboxrule\relax}{\fontsize{7}{8}\selectfont\color{white}{\textit{Specification} \textbf{Vegetable}}}}}

% \ifkompendium\else
% \vspace{-2em}
% \fi

% \begin{CodeSmall}
% /** Representerar en grönsak. */
% class Vegetable(val name: String):
%   /** Returnerar nuvarande vikt i gram. */
%   def weight: Int = ???

%   /** Ändrar vikten till w gram.
%    *  w ska vara positiv, blir annars 0 */
%   def weight_=(w: Int): Unit = ???
% \end{CodeSmall}

% \ifkompendium\else
% \columnbreak
% \fi

% \begin{JavaSpec}{class JVegetable}
% /** Skapar en grönsak. */
% JVegetable(String name);

% /** Returnerar namnet. */
% String getName();

% /** Returnerar nuvarande vikt i gram. */
% int getWeight();

% /** Ändrar vikten till weight gram.
%  *  w ska vara positiv, blir annars 0 */
% void setWeight(int weight);
% \end{JavaSpec}
% \ifkompendium\else
% \end{multicols}
% \pause\SlideFontTiny
% \fi
% \noindent Fördjupning:\\ Kasta undantaget \code{IllegalArgumentException} vid försök till negativ vikt.\\
% Läs om undantag i Java här: \href{https://docs.oracle.com/javase/tutorial/essential/exceptions/index.html}{docs.oracle.com/javase/tutorial/essential/exceptions/}

% \end{Slide}




\begin{Slide}{Scalas ''case-klass-godis'' finns inte i Java}
\ifkompendium\else
\SlideFontTiny\vspace{-0.5em}\begin{multicols}{2}
\fi 

En oföränderlig datatyp implementeras i \Emph{Scala} helst som en \pause\code{case}-klass:

\begin{CodeSmall}[basicstyle=\ttfamily\SlideFontSize{5.7}{6.7}]
case class Person(name: String, age: Int):
  def isAdult = age >= Person.AdultAge

object Person:
  val AdultAge = 18
\end{CodeSmall}

\pause
\ifkompendium\else\columnbreak\fi

\noindent En oföränderlig datatyp i \Emph{Java} med \Alert{motsvarande} funktionalitet kräver egen implementation av dessa metoder:
\ifkompendium\else\vspace{-0.25em}\fi
\begin{itemize}
\item en getter för varje attribut
\item \code{equals}
\item \code{hashCode} (förklaras i forts.kurs)
\item \code{apply} \\ (men man kallar nog den \code{create} el. likn.; namnet måste ju skrivas)
\item \code{toString}
\item \code{copy} \\ (men det finns ju inte namngivna parametrar och default-argument så denna blir osmidig)
\item \code{unapply} \\ (men det finns ju inte mönstermatchning så denna struntar man nog i)
\end{itemize}

\ifkompendium\else
\end{multicols}
\fi
\end{Slide}




\Subsection{Array i Java}

\begin{Slide}{Repetition: Den primitiva typen Array i JVM}
\begin{itemize}
\item Primitiva arrayer (\code{Array} i Scala, \code{[]} i Java) har \Emph{fördelar}:%
\footnote{\href{http://stackoverflow.com/questions/2843928/benefits-of-arrays}{stackoverflow.com/questions/2843928/benefits-of-arrays}}
\begin{itemize}\SlideFontSmall
\item Det är den snabbaste indexerbara datastrukturen i JVM: att läsa och uppdatera ett element på en viss plats är mycket effektivt om man vet platsens index.
\item Fungerar lika bra med både primitiva värden och objektreferenser
\end{itemize}
\item ... men också \Alert{nackdelar}:
\begin{itemize}\SlideFontSmall
\item Man måste bestämma sig för antalet element som ska allokeras när man gör \code{new}.
\item Man kan ta i lite extra när man allokerar om man behöver plats för fler senare, men då måste man hålla reda på hur många platser man använder och veta var nästa lediga plats finns.
\item Det är krångligt att stoppa in \Eng{insert} och ta bort \Eng{delete} element.
\item Vill man ha fler platser måste man allokera en helt ny, större array och kopiera över alla befintliga element.
\end{itemize}

\end{itemize}
\end{Slide}




\begin{Slide}{Syntax för Array i Scala och Java}
\begin{multicols}{2}
\Emph{Scala}

\begin{CodeSmall}[basicstyle=\ttfamily\SlideFontSize{6}{8}\ifkompendium 
,backgroundcolor=\color{white},frame=none 
\fi
]
var xs = Array(42, 43, 44)




val n = xs.length

var strings = new Array[String](42)
// eller      Array.ofDim[String](42)
// eller      Array.fill(42)(null: String)

strings(0) = "first"

strings(1) = "second"
\end{CodeSmall}

\columnbreak

\Emph{Java}

\begin{CodeSmall}[language=Java,basicstyle=\ttfamily\SlideFontSize{6}{8}\ifkompendium 
,backgroundcolor=\color{white},frame=none 
\fi
]
int[] xs = new int[]{42, 43, 44};

// samma som ovan, men kortare:
int[] xs2 = {42, 43, 44};

int n = xs.length;  // EJ length()

String[] strings = new String[42];



strings[0] = "first";

strings[1] = "second";
\end{CodeSmall}

\end{multicols}
\end{Slide}






\begin{Slide}{Exempel: Polygon med primitiv array i Java}
\begin{Code}[numberstyle=,numbers=left,language=Java]
public class Polygon {
    private Point[] vertices; // array med hörnpunkter
    private int n;            // antalet hörnpunkter

    /** Skapar en polygon */
    public Polygon() {
        vertices = new Point[1];
        n = 0;
    }

    ...
\end{Code}
\end{Slide}

\begin{Slide}{Polygon med primitiv array i Java: stoppa in sist och vid behov skapa mer plats}\SlideFontSmall
Implementera:\\
\jcode{private void extend()                // dubbla storleken}\\
\jcode{public void addVertex(int x, int y)  // lägg till hörnpunkt}
\pause
\begin{Code}[numberstyle=,numbers=left,language=Java]
    private void extend(){
        Point[] oldVertices = vertices;
        vertices = new Point[2 * vertices.length]; // skapa dubbel plats
        for (int i = 0; i < oldVertices.length; i++) {  // kopiera
            vertices[i] = oldVertices[i];
        }
    }

    public void addVertex(int x, int y) {
        if (n == vertices.length) extend();
        vertices[n] = new Point(x, y);
        n++;
    }
\end{Code}
\end{Slide}


\begin{Slide}{Polygon med primitiv array i Java: stoppa in mitt i på angiven plats }\SlideFontSmall
Implementera:\\
\jcode{/** Sätt in hörnpunkt på plats pos */}\\
\jcode{public void insertVertex(int pos, int x, int y)}
\pause
\begin{Code}[numberstyle=,numbers=left,language=Java]
    public void insertVertex(int pos, int x, int y) {
        if (n == vertices.length) extend();   // utöka vid behov
        for (int i = n; i > pos; i--) {       // flytta element bakifrån
            vertices[i] = vertices[i - 1];
        }
        vertices[pos] = new Point(x, y);
        n++;
    }
\end{Code}
\end{Slide}


%\Subsection{Scanner}

\begin{Slide}{Scanna filer och strängar med \texttt{java.util.Scanner}}\SlideFontTiny
\setlength{\leftmargini}{0pt}
\begin{itemize}
\item I Scala kan man läsa från fil så här (se quickref sid 3 längst ner):

\begin{Code}
val names = scala.io.Source.fromFile("src/names.txt").getLines.toVector
\end{Code}

\item Klassen \code{java.util.Scanner} kan också läsa från fil (se Java Snabbref sid 4):


\begin{Code}
def readFromFile(fileName: String): Vector[String] = {
  val file = new java.io.File(fileName)
  val scan = new java.util.Scanner(file)
  val buffer = scala.collection.mutable.ArrayBuffer.empty[String]
  while (scan.hasNext) {
    buffer += scan.next
  }
  scan.close
  buffer.toVector
}
\end{Code}

\item Med \code{new java.util.Scanner(System.in)} kan man även scanna tangentbordet.

\item Med \code{new java.util.Scanner("hej 42")} kan man även scanna en sträng.

\item Scanna \code{Int} och \code{Double} med metoderna \code{nextInt} och \code{nextDouble}.\\Se doc: \href{https://docs.oracle.com/javase/8/docs/api/java/util/Scanner.html}{\SlideFontTiny docs.oracle.com/javase/8/docs/api/java/util/Scanner.html}
\end{itemize}
\end{Slide}


\begin{Slide}{Exempel: Scanner}
\begin{REPL}
scala> val scan = new java.util.Scanner("hej 42 42.0   42 slut")

scala> scan.hasNext
res0: Boolean = true

scala> scan.hasNextInt
res1: Boolean = false

scala> scan.next
res2: String = hej

scala> scan.hasNextInt
res3: Boolean = true

scala> scan.nextInt
res4: Int = 42

scala> while (scan.hasNext) println(scan.next)
42.0
42
slut
\end{REPL}
\end{Slide}



%\Subsection{scala.jdk.CollectionConverters}

\begin{Slide}{Använda Java-samlingar i Scala med \texttt{CollectionConverters}}\SlideFontSmall
Med hjälp av \code{import scala.jdk.CollectionConverters.*} \\
får du smidig \Emph{interoperabilitet} med Java och dess standardbibliotek, \\
speciellt metoderna \Alert{\code{asJava}} och \Alert{\code{asScala}}:
\begin{REPL}
scala> import scala.jdk.CollectionConverters.*

scala> Vector(1,2,3).asJava
res0: java.util.List[Int] = [1, 2, 3]

scala> val xs = new java.util.ArrayList[String]()
xs: java.util.ArrayList[String] = []

scala> xs.add("hej")
res1: Boolean = true

scala> xs.asScala
res2: scala.collection.mutable.Buffer[String] = Buffer(hej)
\end{REPL}

\noindent Läs mer här: %
\ifkompendium\\\fi%
\scriptsize%
\url{https://docs.scala-lang.org/overviews/collections-2.13/conversions-between-java-and-scala-collections.html}

\end{Slide}




%\Subsection{ArrayList}

\begin{Slide}{Generiska samlingar i Java}
\begin{itemize}
\item Från och med version 5 av Java (2004) så introducerades \Emph{generics} vilket möjliggör skapandet av klasser som kan erbjuda generell behandling av olika typer av objekt.

\item Generiska klasser i Java känns igen med syntaxen \code{Klassnamn<Typ>}, till exempel  \code{ArrayList<Point>}

\item Fördjupning: \href{https://docs.oracle.com/javase/tutorial/extra/generics/intro.html}{docs.oracle.com/javase/tutorial/extra/generics/intro.html}, mer om detta i fördjupningskursen.

\end{itemize}
\end{Slide}

\begin{Slide}{Om ArrayList i Java}\SlideFontSmall
\code{java.util.ArrayList} liknar \code{scala.collection.mutable.ArrayBuffer} som båda har dessa fördelar:
\begin{itemize}
\item Lagrar sina element internt i snabbindexerade primitiva arrayer.
\item Fungerar för alla typer av objekt.
\item Utökar samlingens storlek av sig själv vid behov.
\end{itemize}
Det finns dock vissa nackdelar med \code{ArrayList} i Java\\(som inte gäller för \code{ArrayBuffer} i Scala):
\begin{itemize}
\item Fungerar \Alert{inte} rakt av med primitiva typer \code{int}, \code{double}, \code{char}, ... \\ (men det finns sätt komma runt detta, tack vare s.k. wrapper-klasser och autoboxning; mer om detta snart)

\item Namnet \code{ArrayList} är inte helt lyckat, eftersom ordet ''lista'' normalt används för länkade snarare än array-liknande strukturer.
\end{itemize}
\end{Slide}

\begin{Slide}{Polygon med ArrayList i Java}\SlideFontSmall
Klassen \code{Polygon}, nu med ett attribut av typen \code{ArrayList<Point>}:
\begin{Code}[numberstyle=,language=Java]
public class Polygon {
    private ArrayList<Point> vertices; // lista med hörnpunkter

    /** Skapar en polygon */
    public Polygon() {
        vertices = new ArrayList<Point>();
    }

    ...
\end{Code}
Det behövs inget attribut \code{n} eftersom vi inte själva behöver hålla reda på antalet allokerade platser: allokering, insättning, och utökning av antalet platser sköts helt automatiskt av \code{ArrayList}-klassen vid behov.
\end{Slide}

\begin{Slide}{Några viktiga operationer på ArrayList<E>}
\SlideFontTiny\url{https://docs.oracle.com/en/java/javase/17/docs/api/java.base/java/util/ArrayList.html}
\begin{Code}[numberstyle=,language=Java]
/** Tar reda på elementet på plats pos */
E get(int pos);

/** Lägger in objektet obj sist */
void add(E obj);

/** Lägger in obj på plats pos; efterföljande flyttas */
void add(int pos, E obj);

/** Tar bort elementet på plats pos och returnerar det */
E remove(int pos);

/** Tar reda på antalet element i listan */
int size();
\end{Code}
Lär dig vad som finns om ArrayList i snabbreferensen för Java\\
Överkurs för den nyfikne: kolla implementation av ArrayList här: \\ \url{http://www.docjar.com/html/api/java/util/ArrayList.java.html}
\end{Slide}


\begin{Slide}{Övning ArrayList: new och add}
Skriv Java-kod som skapar en lista med element av typen \code{Point} och lägger in tre punkter i listan med koordinaterna:\\ (50, 50), (50,10) och (30, 40).
\pause
~\\~\\ Lösning: \ifkompendium\else\\~\\\fi
\begin{Code}[numberstyle=,language=Java]
ArrayList<Point> vertices = new ArrayList<Point>();
vertices.add(new Point(50, 50));
vertices.add(new Point(50, 10));
vertices.add(new Point(30, 40));
\end{Code}
\end{Slide}


\begin{Slide}{For-each-sats i Java:}\SlideFontSmall
\begin{itemize}
\item  Antag att vi vill gå igenom alla element i en lista.
\begin{Code}[numberstyle=,language=Java]
        ArrayList<String> words = new ArrayList<String>();
\end{Code}
\item Det finns två olika typer av \jcode{for}-satser i Java som kan göra detta:
\begin{itemize}\SlideFontSmall
\item  Vanlig \jcode{for}-sats:
\begin{Code}[numberstyle=,language=Java]
for (int i = 0; i < words.size(); i++) {
    System.out.println(i + ": " + words.get(i));
}
\end{Code}

\item  Så kallad \Emph{for-each-sats} med denna syntax:\\
\jcode+for (Elementtyp element: samling) { ... }+ \\
\vspace{1em}Exempel:
\begin{Code}[numberstyle=,language=Java]
for (String s: words) {
    System.out.println(s);
}
\end{Code}
Men vi får ingen indexvariabel då...
\end{itemize}
\end{itemize}
\end{Slide}


\begin{Slide}{Polygon med ArrayList: metoderna blir enklare}
\begin{Code}[numberstyle=,language=Java]
    public void addVertex(int x, int y) {
        vertices.add(new Point(x, y));
    }

    public void move(int dx, int dy) {
        for (Point p: vertices){
            p.move(dx, dy);
        }
    }

    public void insertVertex(int pos, int x, int y) {
        vertices.add(pos, new Point(x, y));
    }

    public void removeVertex(int pos) {
        vertices.remove(pos);
    }
\end{Code}

Se hela lösningen här:
\href{https://github.com/lunduniversity/introprog/tree/master/compendium/examples/scalajava/list/Polygon.java}{compendium/examples/scalajava/list/Polygon.java}
\end{Slide}

\begin{Slide}{Polygon med ArrayList: iterera över alla hörnpunkter i draw med indexering}
\begin{Code}[numberstyle=,language=Java]
    public void draw(SimpleWindow w) {
        if (vertices.size() == 0) {
            return;
        }
        Point start = vertices.get(0);
        w.moveTo(start.getX(), start.getY());
        for (int i = 1; i < vertices.size(); i++) {
            w.lineTo(vertices.get(i).getX(),
                     vertices.get(i).getY());
        }
        w.lineTo(start.getX(), start.getY());
    }
\end{Code}

Övning: Skriv om med for-each-sats.
\end{Slide}

\begin{Slide}{Polygon med ArrayList: iterera över alla hörnpunkter i draw med foreach-sats}
\begin{Code}[numberstyle=,language=Java]
    public void draw(SimpleWindow w) {
        if (vertices.size() == 0) {
            return;
        }
        Point start = vertices.get(0);
        w.moveTo(start.getX(), start.getY());
        for (Point p: vertices){
            w.lineTo(p.getX(), p.getY());
        }
        w.lineTo(start.getX(), start.getY());
    }
\end{Code}

Se hela lösningen här:
\href{https://github.com/lunduniversity/introprog/tree/master/compendium/examples/scalajava/list/Polygon.java}{compendium/examples/scalajava/list/Polygon.java}
\end{Slide}




\begin{Slide}{Övning ArrayList: implementera metoden hasVertex}
Skriv kod som implementerar denna metod i klassen \code{Polygon}:
\begin{Code}[numberstyle=,language=Java]
/** Undersöker om polygonen har någon hörnpunkt med koordinaterna x, y. */
public boolean hasVertex(int x, int y) {
    ???
}
\end{Code}
\end{Slide}

\begin{Slide}{Lösning ArrayList: implementera metoden hasVertex}
\begin{Code}[numberstyle=,language=Java]
    public boolean hasVertex(int x, int y) {
        for (Point p: vertices) {
            if (p.getX() == x && p.getY() == y) {
                return true;
            }
        }
        return false;
    }
\end{Code}
\end{Slide}


\begin{Slide}{For-each-sats med array}
For-each-sats fungerar även med primitiv array:
\begin{Code}[numberstyle=,language=Java]
        String[] stringArray = {"hej", "på", "dej"};
        for (String s: stringArray) {
            System.out.println(s);
        }
\end{Code}
\end{Slide}





\Subsection{Autoboxning i Java}



\begin{Slide}{Generiska klasser (t.ex. ArrayList) med primitiva typer}
Detta går tyvärr \Alert{INTE} i Java: \\
  \sout{\texttt{ArrayList<int> list = new ArrayList<int>();}}
  
\pause
\begin{itemize}\SlideFontSmall
\item Hur gör man om man vill ha heltalselement (eller andra primitiva värden) i en generisk samling?

\item Javas lösning på problemet består av två delar:
\begin{itemize}\SlideFontSmall
\item Klasser som packar in primitiva typer, \Eng{wrapper classes}
\item Speciella regler för implicita konverteringar, s.k. ''auto-boxing'' \Eng{Boxing / Unboxing conversions}
\end{itemize}
\end{itemize}
\SlideFontTiny\vspace{1em}
Ofta fungerar det fint, men det finns fallgropar.\\
(Om du är nyfiken på alla intrikata detaljer, se
\href{https://docs.oracle.com/javase/tutorial/java/data/autoboxing.html}{Java tutorial} och   \href{https://docs.oracle.com/javase/specs/jls/se8/html/jls-5.html#jls-5.1.7}{Javaspecifikationen}.)
\end{Slide}

\begin{Slide}{Wrapper-klassen \code{Integer}}\SlideFontSmall
En skiss av klassen \code{Integer} \\ (ligger i paketet \href{http://docs.oracle.com/javase/8/docs/api/java/lang/package-summary.html}{\code{java.lang}} och importeras därmed implicit):

\ifkompendium\vspace{1em}\fi%
\begin{minipage}{0.65\textwidth}
\begin{Code}[numberstyle=,language=Java,backgroundcolor=\color{white},
  frame=none]
public class Integer {
    private int value;

    public static final MIN_VALUE = -2147483648;
    public static final MAX_VALUE = 2147483647;

    public Integer(int value) {
        this.value = value;
    }

    public int intValue() {
        return value;
    }
    ...
}
\end{Code}
\end{minipage}
\begin{minipage}{0.33\textwidth}
\centering\includegraphics[width=0.95\textwidth]{../img/box}
\end{minipage}
Javadoc för klasen \code{Integer} finns här: \\
\SlideFontTiny\url{http://docs.oracle.com/javase/8/docs/api/java/lang/Integer.html}
\end{Slide}





\begin{Slide}{Wrapper-klasser i \code{java.lang}}\SlideFontSmall
\begin{tabular}{l | l}
\Emph{Primitiv typ}                  & \Emph{Inpackad typ}                 \\ \hline

 boolean & Boolean\\
 byte & Byte\\
 short& Short\\
 char & Character\\
 int & Integer\\
 long & Long\\
 float & Float\\
 double & Double\\
\end{tabular}
\end{Slide}


\begin{Slide}{Övning: primitiva versus inpackade typer}
Med papper och penna:
\begin{itemize}
\item Deklarera en variabel med namnet \code{gurka} av den primitiva heltalstypen och initiera den till värdet 42.
\item Deklarera en referensvariabel med namnet  \code{tomat} av den inpackade (''wrappade'') heltalstypen och initiera den till värdet 43.
\item Rita hur det ser ut i minnet.
\end{itemize}
\end{Slide}

\begin{Slide}{Exempel: Lista med heltal utan autoboxning}
\lstinputlisting[language=Java, basicstyle=\small\ttfamily\SlideFontSize{6.7}{8.5},backgroundcolor=\color{white},frame=none
]{../compendium/examples/scalajava/generics/TestIntegerList.java}
\SlideFontTiny Koden finns här: \href{https://github.com/lunduniversity/introprog/tree/master/compendium/examples/scalajava/generics/TestIntegerList.java}{compendium/examples/scalajava/TestIntegerList.java}
\end{Slide}




\begin{Slide}{Specialregler för wrapper-klasser}\SlideFontSmall

\begin{itemize}
\item Om ett \code{int}-värde förekommer där det behövs ett \code{Integer}-objekt, så lägger kompilatorn \Alert{automatiskt} ut kod som skapar ett \code{Integer}-objekt som packar in värdet.
\item Om ett \code{Integer}-objekt förekommer där det behövs ett \code{int}-värde, lägger kompilatorn \Alert{automatiskt} ut kod som anropar metoden \code{intValue()}.
\end{itemize}
Samma gäller mellan alla primitiva typer och dess wrapper-klasser:

\begin{tabular}{r c l}
 {\lstinline!boolean!} &$\Leftrightarrow$& {\lstinline!Boolean!} \\
 {\lstinline!byte!} &$\Leftrightarrow$& {\lstinline!Byte!}\\
 {\lstinline!short!}&$\Leftrightarrow$& {\lstinline!Short!}\\
 {\lstinline!char!} &$\Leftrightarrow$& {\lstinline!Character!}\\
 {\lstinline!int!} &$\Leftrightarrow$& {\lstinline!Integer!}\\
 {\lstinline!long!} &$\Leftrightarrow$& {\lstinline!Long!}\\
 {\lstinline!float!} &$\Leftrightarrow$& {\lstinline!Float!}\\
 {\lstinline!double!} &$\Leftrightarrow$&{\lstinline!Double!}\\
\end{tabular}

\end{Slide}






\begin{Slide}{Exempel: Lista med heltal och autoboxning}
\lstinputlisting[language=Java, basicstyle=\small\ttfamily\SlideFontSize{6}{8}
,backgroundcolor=\color{white},
  frame=none]{../compendium/examples/scalajava/generics/TestIntegerListAutoboxing.java}
\SlideFontTiny Koden finns här: \href{https://github.com/lunduniversity/introprog/tree/master/compendium/examples/scalajava/generics/TestIntegerList.java}{scalajava/generics/TestIntegerListAutoboxing.java}
\end{Slide}

\begin{Slide}{Fallgropar vid autoboxning}
\begin{itemize}
\item Jämförelser med \code{==} och \code{!=} \\
\href{https://github.com/lunduniversity/introprog/blob/master/compendium/examples/scalajava/generics/TestPitfall1.java}
{\SlideFontSmall  compendium/examples/scalajava/generics/TestPitfall1.java}
\item[]
\item Kompilatorn hittar inte förväxlad parameterordning, t.ex. \code{add(pos, item)} i fel ordning: \sout{\code{add(item, pos)}}\\
\href{https://github.com/lunduniversity/introprog/blob/master/compendium/examples/scalajava/generics/TestPitfall2.java}
{\SlideFontSmall compendium/examples/scalajava/generics/TestPitfall2.java}
\end{itemize}
\end{Slide}

\Subsection{Equals i Java}

\begin{Slide}{Referenslikhet eller innehållslikhet i Scala och Java}\SlideFontSmall
Det finns två \Alert{principiellt olika} sorters \Emph{likhet}:
\begin{itemize}
\item \Emph{Referenslikhet} \Eng{reference equality}: två referenser anses lika om de refererar till \Emph{samma instans} i minnet.
\item \Emph{Innehållslikhet}, ä.k. strukturlikhet \Eng{structural equality}: två referenser anses lika om de refererar till objekt med \Emph{samma innehåll}.

\pause

\item I Scala finns flera metoder som testar likhet:
\begin{itemize}\SlideFontSmall
\item metoden \code{eq} testar \Alert{referenslikhet} och \code{r1.eq(r2)} ger \code{true} om \code{r1} och \code{r2} refererar till \Emph{samma} instans.

\item metoden \code{ne} testar referens\textbf{o}likhet och \code{r1.ne(r2)} ger \code{true} om \code{r1} och \code{r2} refererar till \Alert{olika} instanser.

\item metoden \code{==} som anropar metoden \code{equals} som default testar referenslikhet men som \Alert{kan överskuggas} om man \Emph{själv vill bestämma} om det ska vara referenslikhet eller strukturlikhet.
\end{itemize}

\pause

\item Scalas \Emph{standardbibliotek} och \Emph{grundtyperna} \code{Int}, \code{String} etc. testar \Emph{innehållslikhet} genom metoden \code{==}
\pause
\item I \Alert{Java} är det \Alert{annorlunda}: symbolen \code{==} är ingen metod i Java utan \Alert{specialsyntax} som vid instansjämförelse alltid testar \Alert{referenslikhet}, medan metoden \code{equals} kan överskuggas med valfri likhetstest.
\end{itemize}
\end{Slide}

\begin{Slide}{Fallgrop med samlingar: metoden contains kräver implementation av equals}\SlideFontSmall
Antag att vi vill implementera \code{hasVertex()} i klassen \code{Polygon} genom att använda metoden \code{contains} på en lista. Hur gör vi då?
\pause
\begin{Code}[numberstyle=,language=Java]
public boolean hasVertex(int x, int y) {
    return vertices.contains(new Point(x, y)); // FUNKAR INTE om ...
    // ... inte Point har en equals som kollar innehållslikhet
}
\end{Code}
Vi behöver implementera metoden \code{equals(Object obj)} i klassen \code{Point} som kollar innehållslikhet och ersätter den \code{equals} som finns i \code{Object} som kollar referenslikhet, eftersom metoden \code{contains} i klassen \code{ArrayList} anropar \code{equals} när den letar igenom listan efter lika objekt. \\
Se exempel här: \href{https://github.com/lunduniversity/introprog/tree/master/compendium/examples/scalajava/generics/TestPitfall3.java}{compendium/examples/scalajava/generics/TestPitfall3.java} \\


\vspace{1em}{\SlideFontTiny\noindent Det krävs ofta även att man även ersätter  \href{http://stackoverflow.com/questions/27581/what-issues-should-be-considered-when-overriding-equals-and-hashcode-in-java}{\code{hashCode}}, mer om det i fortsättningskursen.}
\end{Slide}


\begin{Slide}{Fullständigt recept för \texttt{equals}}
För den nyfikne inför fortsättningskursen efter jul: 

\vspace{1em}\noindent
Läs om fallgropar för att implementera equals i \Emph{Java} här: \\
\href{http://www.artima.com/lejava/articles/equality.html}{www.artima.com/lejava/articles/equality.html}


\vspace{1em}\noindent
Läs receptet för att implementera equals i \Emph{Scala} här: \\
\href{http://www.artima.com/pins1ed/object-equality.html#28.4}{www.artima.com/pins1ed/object-equality.html\#28.4}
\end{Slide}




\begin{Slide}{Villkorsuttryck i Java}\SlideFontSmall
Det går att använda villkorsuttryck i Java, men med syntax från språket C:
\begin{multicols}{2}
  \noindent\Emph{Scala}
\begin{CodeSmall}[basicstyle=\ttfamily\SlideFontSize{6}{8},backgroundcolor=\color{white},
  frame=none]
var r = math.random()
var answer = if r > 0.5 then 42 else 0
\end{CodeSmall}

\columnbreak

\noindent\Emph{Java}
\begin{CodeSmall}[language=Java,basicstyle=\ttfamily\SlideFontSize{6}{8},backgroundcolor=\color{white},
  frame=none]
double r = Math.random();
int answer = (r > 0.5) ? 42 : 0;
\end{CodeSmall}
\end{multicols}

\end{Slide}




\begin{Slide}{Typtest och typkonvertering}

\begin{multicols}{2}
  \noindent\Emph{Scala}
\begin{CodeSmall}[basicstyle=\small\ttfamily\SlideFontSize{6}{8},backgroundcolor=\color{white},
  frame=none]
var x = "hej"

var isString = x.isInstanceOf[String]

var y = 42

var z = y.asInstanceOf[Double]

\end{CodeSmall}

\columnbreak

\noindent\Emph{Java}
\begin{CodeSmall}[language=Java,basicstyle=\small\ttfamily\SlideFontSize{6}{8},backgroundcolor=\color{white},
  frame=none]
String x = "hej";

boolean isString = x instanceof String;

int y = 42;

double z = (double) y;
\end{CodeSmall}
\end{multicols}

\pause {\SlideFontTiny Detta görs ju i Scala bäst med \code{match} och typmönster!}

\end{Slide}

\begin{Slide}{Regler för överskuggning i Java}
\url{http://docs.oracle.com/javase/tutorial/java/IandI/override.html}
\end{Slide}
  


\begin{Slide}{Fånga undantag i Scala och Java}
Typisk skillnad mellan Scala och Java:\\konstruktioner som är \Emph{uttryck} i Scala är ofta \Alert{satser} i Java.
\begin{multicols}{2}
  \noindent\Emph{Scala}
\begin{CodeSmall}[basicstyle=\ttfamily\SlideFontSize{6}{8},backgroundcolor=\color{white},
  frame=none]
val a = try 2 / 0 catch
  case e: ArithmeticException => 0


val b = try 4 / 2 catch 
  case e: ArithmeticException => 0
\end{CodeSmall}

\columnbreak

\noindent\Emph{Java}
\begin{CodeSmall}[language=Java,basicstyle=\ttfamily\SlideFontSize{6}{8},backgroundcolor=\color{white},
  frame=none]
int a;
try {
    a = 2 / 0;
} catch (ArithmeticException e) {
    a = 0;
}

int b;
try {
    b = 4 / 2;
} catch (ArithmeticException e) {
    b = 0;
}

\end{CodeSmall}
\end{multicols}
\end{Slide}




\begin{Slide}{Gränssnittet \texttt{List} i Java}\SlideFontSmall
\begin{itemize}
\item I Java finns inte \code{trait} och inmixning.

\item I stället finns \jcode{interface} som liknar \code{trait} men är mer begränsad vad gäller vilka medlemmar som får finnas.

\item Man kan bara göra \code{extends} på exakt en annan klass, men man kan i Java göra \jcode{implements} på flera \jcode{interface}.%(Jämför Scalas \code{with} på \code{trait}s)

\item Exempel:
\begin{Code}[language=Java,backgroundcolor=\color{white},
  frame=none]
public class ArrayList<E> extends AbstractList<E>
    implements List<E>, RandomAccess, Cloneable, java.io.Serializable
\end{Code}

\item Att implementera ett gränssnitt innebär att uppfylla ett kontrakt som utlovar att vissa speciella metoder finns tillgängliga.

\item Gränssninttet \code{List} uppfylls av en av dess implementationer \code{ArrayList} \\

på liknande sätt i Scala där gränssnittet \code{Seq} uppfylls av \code{Vector} etc.

\item[] \jcode{List<String> xs = new ArrayList<String>();}

\item Liknande exempel från övninegn Hangman: \\\jcode{Set<Character> found = new HashSet<Character>();}

\item En Scala-trait med enbart abstrakta medlemmar kompileras till ett Java-interface i JVM bytekode.

\item Mer om gränssnitt i Java i fördjupningskursen.

\end{itemize}
\end{Slide}

\begin{Slide}{Det går inte att skapa generisk Array i Java}\SlideFontTiny
\begin{itemize}
\item I Java kan man \Alert{inte} skapa en primitiv array av godtycklig typ enligt generisk typparameter: \sout{\code{T[] xs = new T[42]}}

\item Man måste istället skapa en array av den mest generella referenstypen: \\
\code{Object[] xs = new Object[42]} \\
och sedan typtesta och typkonvertera under körtid; se t.ex. implementationen av \code{ArrayList} på rad 119: \href{http://developer.classpath.org/doc/java/util/ArrayList-source.html}{http://developer.classpath.org/doc/java/util/ArrayList-source.html}

\item[]
\pause
\item Detta går faktiskt att göra i Scala med hjälp av \code{reflect.ClassTag} \pause så här: \\
\begin{REPLsmall}
scala> def fyll[T](n: Int, x: T): Array[T] = Array.fill(n)(x)
-- Error:
1 |def fyll[T](n: Int, x: T): Array[T] = Array.fill(n)(x)
  |                                                      ^
  |  No ClassTag available for T

scala> def fyll[T: reflect.ClassTag](n: Int, x: T): Array[T] = Array.fill(n)(x)

scala> fyll(42, "hej")
res2: Array[String] = Array(hej, hej, hej, hej, hej, hej, hej, hej, hej, hej, hej, hej, hej, hej, hej, hej, hej, hej, hej, hej, hej, hej, hej, hej, hej, hej, hej, hej, hej, hej, hej, hej, hej, hej, hej, hej, hej, hej, hej, hej, hej, hej)

\end{REPLsmall}


\end{itemize}


\end{Slide}

\begin{Slide}{Jämföra strängar i Java}\SlideFontTiny
\begin{itemize}
\item I Java kan man \Alert{inte} jämföra strängar med operatorerna \code{<}, \code{<=}, \code{>}, och \code{>=}

\item Dessutom ger operatorerna \code{==} och \code{!=} \emph{inte} innehålls(o)likhet utan \Alert{referens(o)likhet} \code{:(}

\item Istället \Alert{måste} man i Java använda metoderna \code{equals} och \code{compareTo}
\\Dessa fungerar även i Scala eftersom strängar i Scala och Java är av samma typ, nämligen \code{java.lang.String}.
\pause
\item \code{s1.compareTo(s2)} ger ett heltal som är:
\begin{itemize}\SlideFontTiny
\item \code{0} om s1 och s2 har samma innehåll
\item \Alert{negativt} om s1 < s2 i lexikografisk mening, alltså s1 ska sorteras \Alert{före}
\item \Emph{positivt} om s1 > s2 i lexikografisk mening, alltså s1 ska sorteras \Emph{efter}
\end{itemize}

\pause
\item Undersök följande:
\begin{REPL}
scala> new String("hej") eq new String("hej") // motsvarar == i Java
scala> "hej".equals("hej")                    // samma som == i Scala
scala> "hej".compareTo("hej")
scala> "hej".compareTo("HEJ")         // alla stora är 'före' alla små
scala> "HEJ".compareTo("hej")
\end{REPL}
\end{itemize}

\href{http://docs.oracle.com/javase/8/docs/api/java/lang/String.html#compareTo-java.lang.String-}{docs.oracle.com/javase/8/docs/api/java/lang/String.html\#compareTo-java.lang.String-}
\end{Slide}


\begin{Slide}{Jämföra strängar i Java: exempel}\SlideFontSmall
Vad skriver detta Java-program ut?
\javainputlisting{../compendium/examples/StringEqTest.java}
\pause
\begin{REPL}
$ javac StringEqTest.java
$ java StringEqTest
false
true
0
\end{REPL}
\end{Slide}


% \Subsection{StringBuilder}

% \begin{Slide}{Förändringsbar eller oföränderlig?}
% \begin{itemize}
% \item Om den underliggande \Emph{oföränderliga} datastrukturen är \Alert{smart} implementerad så att den \Emph{återanvänder redan allokerade objekt} -- vilket ju är ofarligt eftersom de aldrig kommer att ändras -- så är oföränderlighet \Emph{minst lika snabbt} som förändring på plats.

% \item Det är först när man gör \Alert{väldigt många} upprepade ändringar på, för datastrukturen ogynnsam plats, som det blir långsamt.

% \item Hur många är ''väldigt många''?  \\ \pause $\rightarrow$ Det ska vi undersöka nu.

% \end{itemize}
% \end{Slide}

% \begin{Slide}{String eller StringBuilder?}
% \begin{itemize}
% \item Strängar i JVM är \Emph{oföränderliga}.

% \item Implementationen av sekvensdatastrukturen \code{java.lang.String} är \Alert{mycket effektivt} implementerad, där \Emph{redan allokerade objekt ofta kan återanvänds}.

% \item \Alert{MEN} väldigt många tillägg på slutet blir långsamt. Därför finns den föränderliga \code{StringBuilder} metoden \code{append} som är snabbare än \code{+} på \code{String} vid mycket stora strängar. (Hur stora?)

% \pause
% \item Undersök dokumentationen för \code{StringBuilder} här:
% {\SlideFontTiny\url{https://docs.oracle.com/javase/8/docs/api/java/lang/StringBuilder.html}}

% \pause
% \item För vilka teckensekvensalgoritmer är det lönt att använda \code{StringBuilder}? \\
% \pause $\rightarrow$ Det ska vi undersöka nu.

% \end{itemize}
% \end{Slide}

% \ifkompendium\else
% \begin{SlideExtra}{Timer}\SlideFontSmall
% \setlength{\leftmargini}{0pt}
% \begin{itemize}
% \item \href{https://docs.oracle.com/javase/8/docs/api/java/lang/System.html#currentTimeMillis--}{\code{System.currentTimeMillis}} ger tiden i millisekunder sedan januari 1970.

% \item Med \code|Timer.measure{ xxx }| nedan kan man mäta tiden det tar för xxx.

% \item Ett par \code{(elapsedMillis, result)} returneras som innehåller tiden det tar att köra blocket, samt resultatet av blocket.
% \end{itemize}
% \vspace{0em}\scalainputlisting[numbers=left,numberstyle=,basicstyle=\fontsize{6.5}{8}\ttfamily\selectfont]{../compendium/examples/workspace/w05-seqalg/src/Timer.scala}
% \end{SlideExtra}


% \begin{SlideExtra}{NanananananananaNanananananananaBatman}\SlideFontTiny
% Prova denna kod:
% \href{https://github.com/lunduniversity/introprog/blob/master/compendium/examples/workspace/w05-seqalg/src/NanananananananaNanananananananaBatman.scala}{compendium/examples/workspace/w05-seqalg/src/\\NanananananananaNanananananananaBatman.scala} \\

% medan du lyssnar till: \href{https://www.youtube.com/watch?v=oDc-1zfffMw}{www.youtube.com/watch?v=oDc-1zfffMw}

% \vspace{-0.7em}\scalainputlisting[numbers=left,numberstyle=,basicstyle=\ttfamily\SlideFontSize{4}{5}]{../compendium/examples/workspace/w05-seqalg/src/NanananananananaNanananananananaBatman.scala}
% \end{SlideExtra}
% \fi






%!TEX encoding = UTF-8 Unicode
%!TEX root = ../compendium.tex

\ifPreSolution

\Exercise{java}\label{exe:java}

\begin{Goals}
\item Kunna förklara och beskriva viktiga skillnader mellan Scala och Java.
\item Kunna översätta enkla algoritmer, klasser och singeltonobjekt från Scala till Java och vice versa.
\item Känna till vad en case-klass innehåller i termer av en Javaklass.
%\item Förstå hur autoboxing fungerar.
\item Kunna använda Javatyperna \code{List}, \code{ArrayList}, \code{Set}, \code{HashSet} och översätta till deras Scalamotsvarigheter med \code{CollectionConverters}.
\item Kunna förklara hur autoboxning fungerar i Java, samt beskriva fördelar och fallgropar.
\end{Goals}

\begin{Preparations}
\item Studera teori i början av detta Appendix.
\end{Preparations}

\BasicTasks %%%%%%%%%%%%%%%%

\else

\ExerciseSolution{java}

\BasicTasks %%%%%%%%%%%

\fi





\WHAT{Översätta metoder från Java till Scala.}

\QUESTBEGIN

\Task  \what~  I denna uppgift ska du översätta en Java-klass som används som en modul\footnote{\href{https://en.wikipedia.org/wiki/Modular_programming}{en.wikipedia.org/wiki/Modular\_programming}} och bara innehåller statiska metoder och inget förändringsbart tillstånd som kan ändras utifrån. (I nästa uppgift ska du sedan översätta klasser med förändringsbara  tillstånd.)

Vi börjar med att göra översättningen från Java till Scala rad för rad och du ska behålla så mycket som möjligt av syntax och semantik så att Scala-koden blir så Java-lik som möjligt. I efterföljande deluppgift ska du sedan omforma översättningen så att Scala-koden blir mer idiomatisk\footnote{\href{https://sv.wikipedia.org/wiki/Idiom_\%28programmering\%29}{sv.wikipedia.org/wiki/Idiom\_\%28programmering\%29}}.

\Subtask Studera klassen \code{Hangman} nedan. Du ska översätta den från Java till Scala enlig de riktlinjer och tips som följer efter koden. Läs igenom alla riktlinjer och tips innan du börjar.

\javainputlisting[numbers=left]{examples/scalajava/Hangman.java}

\noindent\emph{Riktlinjer och tips för översättningen:}

\begin{enumerate}[noitemsep]

\item Skriv Scala-koden med en texteditor i en fil som heter \texttt{hangman1.scala} och kompilera med \code{scalac hangman1.scala} i terminalen; använd alltså \emph{inte} en IDE, så som Eclipse eller IntelliJ, utan en ''vanlig'' texteditor, t.ex. VS \code{code}.

\item Översätt i denna första deluppgift rad för rad så likt den ursprungliga Java-kodens utseende (syntax)  som möjligt, med så få ändringar som möjligt. Du ska alltså ha kvar dessa Scalaovanligheter, även om det inte alls blir som man brukar skriva i Scala:
\begin{enumerate}[nolistsep, noitemsep]
\item långa indrag, \item onödiga semikolon, \item onödiga \code{()}, \item onödiga \code|{}|, \item onödiga \code{System.out}, och \item onödiga \code{return}.
\end{enumerate}

\item Försök också i denna deluppgift göra så att betydelsen (semantiken) så långt som möjligt motsvarar den i Java, t.ex. genom att använda \code{var} överallt, även där man i Scala normalt använder \code{val}.

\item En Javaklass med bara statiska medlemmar motsvarar ett singeltonobjekt i Scala, alltså en \code{object}-deklaration innehållande ''vanliga'' medlemmar.

\item För att tydliggöra att du använder Javas \code{Set} och \code{HashSet} i din Scala-kod, använd följande import-satser i \code{hangman1.scala}, som därmed döper om dina importerade namn och gör så att de inte krockar med Scalas inbyggda \code{Set}. Denna form av import går inte att göra i Java.
\begin{Code}
import java.util.{Set => JSet};
import java.util.{HashSet => JHashSet};
\end{Code}

\item Javas \code{i++} fungerar inte i Scala; man får istället skriva \code{i += 1} eller mindre vanliga \code{i = i + 1}.

\item Typparametrar i Java skrivs inom \code{<>} medan Scalas syntax för typparametrar använder \code{[]}.

\item Till skillnad från Java så har Scalas metoddeklarationer ett tilldelningstecken \code{=} efter returtypen, före kroppen.

\item Du kan ladda ner Java-koden till \code{Hangman}-klassen nedan från kursens repo%
\footnote{\href{https://github.com/lunduniversity/introprog/blob/master/compendium/examples/scalajava/Hangman.java}{github.com/lunduniversity/introprog/blob/master/compendium/examples/scalajava/Hangman.java}}. I samma bibliotek ligger även lösningarna till översättningen i Scala, men kolla \emph{inte} på dessa förrän du gjort klart översättningarna och fått dem att kompilera och köra felfritt! Tanken är att du ska träna på att läsa felmeddelande från kompilatorn och åtgärda dem i en upprepad kompilera-testa-rätta-cykel.

\end{enumerate}







\Subtask Skapa en ny fil \code{hangman2.scala} som till att börja med innehåller en kopia av din direkt-översatta Java-kod från föregående deluppgift. Omforma koden så att den blir mer som man brukar skriva i Scala, alltså mer Scala-idiomatisk. Försök förenkla och förkorta så mycket du kan utan att göra avkall på läsbarheten.

\emph{Tips och riktlinjer:}

\begin{enumerate}[nolistsep, noitemsep]

\item Kalla Scala-objektet för \code{hangman}. När man använder ett Scalaobjekt som en modul (alltså en samling funktioner i en gemensam, avgränsad namnrymd) har man gärna liten begynnelsebokstav, i likhet med konventionen för paketnamn. Ett paket är ju också en slags modul och med en namngivningskonvention som är gemensam kan man senare, utan att behöva ändra koden som använder modulen, ändra från ett singelobjekt till ett paket och vice versa om man så önskar.

\item Gör alla metoder publikt tillgängliga och låt även strängvektorn \code{hangman} vara publikt tillgänglig. Deklarera \code{hangman} som en \code{val} och konstruera den med \code{Vector}. Eftersom \code{Vector} är oföränderlig och man inte kan ärva från singelobjekt och \code{hangman} är deklarerad med \code{val} finns inga speciella risker med att göra den konstanta vektorn publik om  vi inte har något emot att annan kod kan läsa (och eventuellt göra sig beroende av) vår hänggubbetext.

\item I metoden \code{renderHangman}, använd \code{take} och \code{mkString}.

\item I metoden \code{hideSecret}, använd \code{map} i stället för en \code{for}-sats.

\item Det går att ersätta metoden \code{foundAll} med det kärnfulla uttrycket \\ \code{(secret forall found)} där \code{secret} är en sträng och \code{found} är en mängd av tecken (undersök gärna i REPL hur detta fungerar). Skippa därför den metoden helt och använd det kortare uttrycket direkt.

\item I metoden \code{makeGuess}, i stället för \code{Scanner}, använd \code{scala.io.StdIn.readLine}.

\item Om du vill träna på att använda rekursion i stället för imperativa loopar: Gör metoden \code{makeGuess} rekursiv i stället för att använda \code{do}-\code{while}.

\item I metoden \code{download}, i stället för \code{java.net.URL} och \code{java.util.ArrayList}, använd \code{scala.io.Source.fromURL(address, coding).getLines.toVector} och gör en lokal import av \code{scala.io.Source.fromURL} överst i det block där den används. Det går inte att ha lokala \code{import}-satser i Java.

\item Låt metoden \code{download} returnera en \code{Option[String]} som i fallet att nedladdningen misslyckas returnerar \code{None}.

\item I metoden \code{download}, i stället för \code{try}-\code{catch} använd \code{scala.util.Try} och dess smidiga metod \code{toOption}.

\item Om du vill träna på att använda rekursion i stället för imperativa loopar: Använd, i stället för \code{while}-satsen i metoden \code{play}, en lokal rekursiv funktion med denna signatur:
\begin{Code}
  def loop(found: Set[Char], bad: Int): (Int, Boolean)
\end{Code}
Funktionen \code{loop} returnerar en 2-tupel med antalet felgissningar och \code{true} om man hittat alla bokstäver eller \code{false} om man blev hängd.

\end{enumerate}





\SOLUTION


\TaskSolved \what
     %%%TODO number  1 %%%starts with: \emph{Översätta algoritmer och %%%

\SubtaskSolved  \scalainputlisting[numbers=left,basicstyle=\ttfamily\fontsize{10.3}{12}\selectfont]{examples/scalajava/hangman1.scala}

\SubtaskSolved  \scalainputlisting[numbers=left,basicstyle=\ttfamily\fontsize{11.2}{13}\selectfont]{examples/scalajava/hangman2.scala}



\QUESTEND






\WHAT{Översätta mellan klasser i Scala och klasser i Java.}

\QUESTBEGIN

\Task  \what~
Klassen \code{Point} nedan är en modell av en punkt som kan sparas på begäran i en lista. Listan är privat för kompanjonsobjektet och kan skrivas ut med en metod \code{showSaved}. I koden används en \code{ArrayBuffer}, men i framtiden vill man, vid behov, kunna ändra från \code{ArrayBuffer} till en annan sekvenssamlingsimplementation, t.ex. \code{ListBuffer}, som uppfyller egenskaperna hos supertypen \code{Buffer}, men har andra prestandaegenskaper för olika operationer. Därför är attributet \code{saved} i kompanjonsobjektet deklarerat med den mer generella typen.

\scalainputlisting[numbers=left]{examples/scalajava/Point.scala}

\Subtask Översätt klassen \code{Point} ovan från Scala till Java. Vi ska i nästa deluppgift kompilera både Scala-programmet ovan och ditt motsvarande Java-program i terminalen och testa i REPL att klasserna har motsvarande funktionalitet.

\emph{Tips och riktlinjer:}
\begin{enumerate}[nolistsep, noitemsep]
\item För att namnen inte ska krocka i våra kommande tester, kalla Javatypen för \code{JPoint}.
\item  I stället för Scalas \code{ArrayBuffer} och \code{Buffer}, använd Javas \code{ArrayList} och \code{List} som båda ligger i paketet \code{java.util}.
\item Undersök dokumentationen för \code{java.util.List} för att hitta en motsvarighet till \code{prepend} för att lägga till i början av listan.
\item I stället för default-argumentet i Scalas primärkonstruktor, använd en extra Java-konstruktor.
\item Det finns inga singelobjekt och inga kompanjonsobjekt i Java; istället kan man använda statiska klassmedlemmar. Placera kompanjonsobjektets medlemmars motsvarigheter \emph{inuti} Java-klassen och gör dem till \jcode{static}-medlemmar.
\item Kod i klasskroppen i Scalaklassen, så som if-satsen på rad 4, placeras i lämplig konstruktor i Javaklassen.
\item Utskrifter med \code{print} och \code{println} behöver i Java föregås av \code{System.out}.
\item Det finns inget nyckelord \code{override} i Java, men en s.k. annotering som ger samma kompilatorhjälp. Den skrivs med ett snabel-a och stor begynnelsebokstav, så här: \jcode{ @Override }  före metoddeklarationen.
\item I Java används konventionen att börja getter-metoder med ordet \code{get}, t.ex. \code{getX()}.
\item Det finns ingen motsvarighet till \code{mkString} för \code{List} så du behöver själv gå igenom listan och hämta elementreferenser för utskrift med en \jcode{for}-loop. Notera att efter sista elementet ska radbrytning göras i utskriften och att inget komma ska skrivas ut efter sista elementet.
\item I Java behövs en ny \jcode{import}-deklaration om man vill importera ännu en typ från samma paket. Man kan även i Java använda asterisk \code{*}, (motsvarande \code{_} i Scala), för att importera allt i ett paket, men då får man med alla möjliga namn och det vill man kanske inte.
\item Metoder i Java slutar med \code{()} om de saknar parametrar.
\item Alla satser i Java slutar med lättglömda semikolon. (Efter att man i skrivit mycket Javakod och växlar till Scalakod är det svårt att vänja sig av med att skriva semikolon...)
\end{enumerate}


\Subtask Starta REPL i samma bibliotek som du kompilerat kodfilerna. Testa så att klasserna \code{Point} och \code{JPoint} beter sig på samma vis enligt nedan. Skriv även testkod i REPL för att avläsa de attributvärden som har getters och undersök att allt funkar som det ska.
\begin{REPLnonum}
> scalac Point.scala
> javac JPoint.java
> scala
scala> val (p, jp) = (new Point, new JPoint)
scala> p.distanceTo(new Point(3, 4))
scala> Point.showSaved
scala> jp.distanceTo(new JPoint(3, 4))
scala> JPoint.showSaved
scala> for (i <- 1 to 10) { new Point(i, i, true) }
scala> Point.showSaved
scala> for (i <- 1 to 10) { new JPoint(i, i, true) }
scala> JPoint.showSaved
\end{REPLnonum}


\Subtask Översätt nedan Javaklass \code{JPerson} till en \code{case class Person} i Scala med  motsvarande funktionalitet.


\javainputlisting[numbers=left]{examples/scalajava/JPerson.java}


\Subtask\Pen Undersök i REPL vilken funktionalitet i Scala-case-klassen \code{Person} som \emph{inte} är implementerad i Java-klassen \code{JPerson} ovan. Skriv upp namnen på några av case-klassens extra metoder samt deras signatur genom att för en \code{Person}-instans, och för kompanjonsobjektet \code{Person}, trycka på TAB-tangenten. Prova några av de extra metoderna i REPL och förklara vad de gör.

\begin{REPL}
scala> val p = Person("Björn", 49)
scala> p.      // tryck TAB en gång
scala> Person. // tryck TAB en gång
scala> p.copy  // tryck TAB en gång
scala> p.copy()
scala> p.copy(age = p.age + 1)
scala> Person.unapply(p)
\end{REPL}


\SOLUTION


\TaskSolved \what
     %%%TODO number  2 %%%starts with: \emph{Översätta mellan klasser %%%

\SubtaskSolved   \javainputlisting[numbers=left]{examples/scalajava/JPoint.java}

\SubtaskSolved   -

\SubtaskSolved   \begin{Code}
case class Person(name: String, age: Int = 0)
\end{Code}

\SubtaskSolved  p.*TAB* - copy, producArity, ProductIterator, productElement, productPrefix

Person.*TAB* - apply, curried, tupled, unapply

\begin{REPLnonum}
scala> p.copy
   def copy(name: String,age: Int): Person

scala> p.copy()
res0: Person = Person(Björn,49)

scala> p.copy(age = p.age + 1)
res1: Person = Person(Björn,50)

scala> Person.unapply(p)
res2: Option[(String, Int)] = Some((Björn,49))
\end{REPLnonum}



\QUESTEND






\WHAT{Auto(un)boxing.}

\QUESTBEGIN

\Task  \what~  I JVM måste typparametern för generiska klasser vara av referenstyp. I Scala löser kompilatorn detta åt oss så att vi ändå kan ha t.ex. \code{Int} som argument till en typparameter i Scala, medan man i Java \emph{inte} direkt kan ha den primitiva typen \jcode{int} som typparameter till t.ex. \code{ArrayList}.

I Java och i den underliggande plattformen JVM används s.k. wrapper-klasser för att lösa detta, t.ex. genom wrapper-klassen \code{Integer} som boxar den primitiva typen \jcode{int}. Java-kompilatorn har stöd för att automatiskt packa in värden av primitiv typ i sådana wrapper-klasser för att skapa referenstyper och kan även automatiskt packa upp dem.

\Subtask Studera hur Scala-kompilatorn låter oss arbeta med en \code{Cell[Int]} även om det underliggande JVM:ens körtidstyp \Eng{runtime type} är en wrapper-klass. Man kan se JVM-körtidstypen med metoderna \code{getClass} och \code{getTypeName} enligt nedan.
\begin{REPL}
scala> class Cell[T](var value: T){
         val typeName: String = value.getClass.getTypeName
         override def toString = "Cell[" + typeName + "](" + value + ")"
       }
scala> val c = new Cell[Int](42)
scala> c.value.getClass.getTypeName
\end{REPL}


\Subtask Vad är körtidstypen för \code{c.value} ovan? Förklara hur det kan komma sig trots att vi deklarerade med typargumentet \code{Int}?

\Subtask Studera dokumentationen för \code{java.lang.Integer}\footnote{\href{https://docs.oracle.com/javase/8/docs/api/java/lang/Integer.html}{docs.oracle.com/javase/8/docs/api/java/lang/Integer.html}} och testa i REPL några av \emph{klassmetoderna} (de som är \jcode{static} och därmed kan anropas med punktnotation direkt på klassens namn utan \code{new}) och några av \emph{instansmetoderna} (de som inte är \jcode{static}).
\begin{REPL}
scala> Integer.  //tryck TAB
scala> Integer.
scala> Integer.toBinaryString(42)
scala> Integer.valueOf(42)
scala> val i = new Integer(42)
scala> i.  // tryck TAB
scala> i.toString
scala> i.compareTo  // tryck TAB 2 gånger
scala> i.compareTo(Integer.valueOf(42))
scala> i.compareTo(42)  // varför fungerar detta?
\end{REPL}

\Subtask\Pen Enligt dokumentationen\footnote{\href{https://docs.oracle.com/javase/8/docs/api/java/lang/Integer.html\#compareTo-java.lang.Integer-}{docs.oracle.com/javase/8/docs/api/java/lang/Integer.html\#compareTo-java.lang.Integer-}} tar instansmetoden \code{compareTo} i klassen \code{Integer} en \code{Integer} som parameter. Hur kan det då komma sig att sista raden ovan fungerar med en \code{Int}?

\Subtask Studera nedan Java-program och beskriv vad som kommer att skrivas ut \emph{innan} du kompilerar och testkör.

\javainputlisting[numbers=left]{examples/scalajava/Autoboxing.java}

\Subtask Ändra i programmet ovan så att autoboxing och autounboxing utnyttjas på alla ställen där så är möjligt. Utnyttja även att \code{toString}-metoden på \code{Integer} ger samma stränrepresentation som \jcode{int} vid utskrift. Fixa också så att du undviker \emph{fallgropen} att i Java jämföra med referenslikhet i stället för att använda \code{equals}. Testa så att allt fungerar som det borde efter dina ändringar.


\Subtask\Pen Antag att du råkar skriva \jcode{xs.add(0, pos)} på rad 14 i ditt program från föregående uppgift. Förklara hur autoboxingen stjälper dig i en \emph{fallgrop} då.

\Subtask\Pen Med ledning av de båda tidigare deluppgifterna: sammanfatta de två nämnda fallgropar med autoboxing i Java i två generella punkter, så att du har nytta av att memorera dem inför din framtida Javakodning.


\SOLUTION


\TaskSolved \what
     %%%TODO number  3 %%%starts with: \emph{Auto(un)boxing.} I JVM må%%%

\SubtaskSolved   -

\SubtaskSolved   Cell har typen java.lang.Integer. När man hämtar ut värdet med \code{c.value} hämtas den primitiva typ \code{int} ut.

\SubtaskSolved   Med hjälp av autoboxing förvandlas 42 till typen \code{Integer} och kan därför jämföras med en annan \code{Integer}.

\SubtaskSolved   i.compareTo(42) fungerar på grund av autoboxing. Då JVM packar in den primitiva typ int i en Integer-objekt automatiskt.

\SubtaskSolved
\begin{REPLnonum}
0 10 20 30 40 50 60 ... 390 400 410

[0]: 0
[42]: 0
NOT EQUAL
\end{REPLnonum}

\SubtaskSolved   \javainputlisting[numbers=left]{examples/scalajava/Autoboxing2.java}

\SubtaskSolved   42 kommer läggas längst fram i listan istället för längst bak, då autounboxing kommer göra Integer(0) till 0 och tvärtom med variablen \code{pos}.

\SubtaskSolved   Om man ska undersöka om två int-variabler är lika ska man använda ==, men om variablerna är av typen Integer måste man använda \code{equals}.

JVM kommer inte varna om man vänder på \code{Integer} och \code{int}, som i \code{xs.add(0, pos)}.



\QUESTEND






\WHAT{CollectionConverters.}

\QUESTBEGIN

\Task  \what~  Med \code{import scala.jdk.CollectionConverters._} får man i sina Scalaprogram tillgång till de smidiga metoderna \code{asJava} och \code{asScala} som översätter mellan motsvarande samlingar i resp språks standardbibliotek. Kör nedan i REPL och gör efterföljande deluppgifter.

\begin{REPL}
scala> val sv = Vector(1,2,3)
scala> val ss = Set('a','b','c')
scala> val sm = Map("gurka" -> 42, "tomat" -> 0)
scala> val ja = new java.util.ArrayList[Int]
scala> ja.add(42)
scala> val js = new java.util.HashSet[Char]
scala> js.add('a')
scala> import scala.jdk.CollectionConverters._
\end{REPL}

\Subtask Till vilka typer konverteras Scalasamlingarna
\code{Vector[Int]}, \code{Set[Char]} och \\ \code{Map[String, Int]} om du anropar metoden \code{asJava} på dessa?

\Subtask Till vilka typer konverteras Javasamlingarna \code{ArrayList[Int]} och \code{HashSet[Char]}  om du anropar metoden \code{asScala} på dessa? Blir det föränderliga eller oföränderliga motsvarigheter?

\Subtask Vad får resultatet för typ om du kör \code{toSet} på en samling av typen \code{mutable.Set}?

\Subtask Undersök hur du kan efter att du gjort \code{sm.asJava.asScala} anropa ytterligare en metod för att få tillbaka en oföränderlig \code{immutable.Map}.

\Subtask Läs mer i dokumentationen om CollectionConverters\footnote{\href{https://docs.scala-lang.org/overviews/collections-2.13/conversions-between-java-and-scala-collections.html}{docs.scala-lang.org/overviews/collections-2.13/conversions-between-java-and-scala-collections.html}}
och prova några fler konverteringar.



\SOLUTION


\TaskSolved \what
     %%%TODO number  4 %%%starts with: \emph{CollectionConverters.} Med \cod%%%

\SubtaskSolved

Vector[Int] -> java.util.List[Int]

Set[Char] -> java.util.Set[Char]

Map[String, Int] -> java.util.Map[String, Int]

\SubtaskSolved

ArrayList[Int] -> scala.collection.mutable.Buffer[Int]

HashSet[Char] -> scala.collection.mutable.Set[Char]

Båda blir föränderliga motsvarigheter. Det visas genom att de till hör \code{scaka.collection.mutable} och både \code{ArrayList} och \code{HashSet} är förändrliga i Java.

\SubtaskSolved   \code{scala.collection.immutable.Set}

\SubtaskSolved   \code{sm.asJava.asScala} ger typen \code{scala.collection.mutable.Map[String,Int]}

\code{sm.asJava.asScala.toMap} ger typen \code{scala.collection.immutable.Map[String,Int]}

\SubtaskSolved   -

\QUESTEND


\WHAT{Hur fungerar en \jcode{switch}-sats i Java (och flera andra språk)?}

\QUESTBEGIN

\Task \label{task:switch} \what~   Det händer ofta att man vill testa om ett värde är ett av många olika alternativ. Då kan man använda en sekvens av många \code{if}-\code{else}, ett för varje alternativ. Men det finns ett annat sätt i Java och många andra språk: man kan använda \jcode{switch} som kollar flera alternativ i en och samma sats, se t.ex. \href{https://en.wikipedia.org/wiki/Switch_statement}{en.wikipedia.org/wiki/Switch\_statement}.

\Subtask Skriv in nedan kod i en kodeditor. Spara med namnet \texttt{Switch.java} och kompilera filen med kommandot \texttt{javac Switch.java}. Kör den med \texttt{java Switch} och ange din favoritgrönsak som argument till programmet. Vad händer? Förklara hur \jcode{switch}-satsen fungerar.

\javainputlisting[numbers=left,basicstyle=\ttfamily\fontsize{9}{11}\selectfont]{examples/Switch.java}

\Subtask \label{subtask:break} Vad händer om du tar bort \jcode{break}-satsen på rad 16?




\SOLUTION


\TaskSolved \what


\SubtaskSolved  Beroende på första bokstaven i din favoritgrönsak får du olika svar såsom \textit{gurka är gott!} vid första bokstaven $g$.\\
Javas \jcode{switch}-sats testar den första bokstaven på favoritgrönsaken genom att stegvis jämföra den med \jcode{case}-uttrycken. Om första bokstaven \jcode{firstChar} matchar bokstaven efter ett \jcode{case} körs koden efter kolonet till \jcode{switch}-satsens slut eller tills ett \jcode{break} avbryter \jcode{switch}-satsen.\\
Matchar inte \jcode{firstChar} något \jcode{case} så finns även \jcode{default}, som körs oavsett vilken första bokstaven är, ett generellt fall.

\SubtaskSolved  Om \jcode{case 't'} körs kommer både  \textit{tomat är gott!} och \textit{broccoli är gott!} skrivas ut, man säger att koden $"$faller igenom$"$. Utan \jcode{break}-satsen i Java körs koden i efterkommande \jcode{case} tills ett \jcode{break} avbryter exekveringen eller \jcode{switch}-satsen tar slut.



\QUESTEND




\WHAT{Fånga undantantag i Java med en \jcode{try}-\jcode{catch}-sats.}

\QUESTBEGIN

\Task \label{task:javatry} \what~   Det finns som vi såg i förra uppgiften inbyggt stöd i JVM för att hantera när program avbryts på oväntade sätt, t.ex. på grund av division med noll eller ej förväntade indata från användaren. Spara koden nedan\footnote{\url{https://github.com/lunduniversity/introprog/blob/master/compendium/examples/TryCatch.java}} i en fil med namnet \texttt{TryCatch.java} och kompilera med \texttt{javac TryCatch.java} i terminalen.

\javainputlisting[numbers=left,basicstyle=\ttfamily\fontsize{11}{12}\selectfont]{examples/TryCatch.java}

\Subtask Förklara vad som händer när du kör programmet med olika indata:
\begin{REPL}
> java TryCatch 42
> java TryCatch 0
> java TryCatch safe 42
> java TryCatch safe 0
> java TryCatch
\end{REPL}

\Subtask Vad händer om du ''glömmer bort'' raden 15 och därmed missar att initialisera input? Hur lyder felmeddelandet? Är det ett körtidsfel eller kompileringsfel?

%\Subtask Beskriv några skillnader och likheter i syntax och semantik mellan \code{try}-\code{catch} i Java respektive Scala.



\SOLUTION


\TaskSolved \what


\SubtaskSolved  \begin{enumerate}
\item Eftersom första argumentet inte är strängen \textit{safe} görs en oskyddad division av 42 med 42 där slutsvaret 1 visas.
\item Eftersom första argumentet inte är strängen \textit{safe} görs en oskyddad division av 42 med 0 som ger \code{ArithmeticException} eftersom ett tal inte kan delas med noll.
\item Eftersom första argumentet är strängen \textit{safe} görs en skyddad division av 42 med 42 där slutsvaret 1 visas.
\item Eftersom första argumentet är strängen \textit{safe} görs en skyddad division av 42 med 0. Denna gång fångas \code{ArithmeticException} av \code{try-catch}-satsen vilket ersätter den gamla division med en säker division med 1 där slutsvaret 42 visas.
\item Eftersom inga argument givits kastas ett \code{ArrayIndexOutOfBoundsException} när programmet försöker anropa \code{equals} metoden hos en sträng som inte finns. Detta kunde också kontrollerats av en \code{try-catch}-sats.
\end{enumerate}

\SubtaskSolved  \begin{REPL}
TryCatch.java:16: error: variable input might not have been initialized
\end{REPL}
Ett kompileringsfel uppstår på grund av risken att \code{input} inte blivit definierad vid division.

% \SubtaskSolved  Den mest markanta skillnaden mellan språken är att Scala varken kräver att ett undantag fångas av en \code{catch} eller att ett undantag behöver deklareras innan det kastas med en \code{@throws}. Dessutom saknar \code{catch}-metoden hos Java de \code{match}-egenskaper Scala har. Inte heller returnerar \code{catch} hos Java något värde vilket gör det nödvändigt att definiera variabler för detta innan. I övrigt är semantiken och syntaxen väldigt lika mellan båda språken. De använder samma struktur och samma ord, dessutom har de en hel del \code{Exception} gemensamt.



\QUESTEND




\WHAT{Matriser med array i Java.}

\QUESTBEGIN

\Task \label{task:arraymatrix-java} \what~   Om man redan vid allokering vet hur många element en matris ska ha, använder man i Java gärna en array av arrayer. En heltalsmatris (en array av array av heltal) skrivs i Java med dubbla hakparentespar \jcode{int[][]} direkt efter typen. Vid allokering använder man nyckelordet \code{new} och antalet element i respektive dimension anges inom hakparenteserna; t.ex. så ger \jcode{new int[42][21]} en matris med 42 rader och 21 kolumner, vilket motsvarar att man i Scala skriver \code{Array.ofDim[Int](42,21)}%
\footnote{
Ett annat sätt att skriva detta i Scala där initialvärdet framgår explicit: \code{Array.fill(42,21)(0)}
}. Alla element får defaultvärdet för typen, här \code{0} för heltal.

\Subtask Skriv nedan program i en editor och spara koden i filen \texttt{JavaArrayTest.java} och kompilera med \texttt{javac JavaArrayTest.java} och kör i terminalen med \texttt{java JavaArrayTest} och undersök utskriften. Förklara vad som händer. Notera några skillnader i hur matriser används i Scala och Java.


\begin{Code}[language=Java]
public class JavaArrayTest {

    public static void showMatrix(int[][] m){
        System.out.println("\n--- showMatrix ---");
        for (int row = 0; row < m.length; row++){
            for (int col = 0; col < m[row].length; col++) {
                System.out.print("[" + row + "]");
                System.out.print("[" + col + "] = ");
                System.out.print(m[row][col] + "; ");
            }
            System.out.println();
        }
    }

    public static void main(String[] args) {
        System.out.println("Hello JavaArrayTest!");
        int[][] xss = new int[10][5];
        showMatrix(xss);
    }
}
\end{Code}

\Subtask Implementera nedan metod \code{fillRnd} inuti klassen \code{JavaArrayTest}. Skriv kod som fyller matrisen \code{m} med slumptal mellan \code{1} och \code{n}.
\begin{Code}[language=Java]
    public static void fillRnd(int[][] m, int n){
        /* ??? */
    }
\end{Code}
\noindent \emph{Tips:} med detta uttryck skapas ett slumptal mellan 1 och 42 i Java:\\
\jcode{(int) (Math.random() * 42 + 1);} \\
där typkonverteringen \jcode{(int)} ger samma effekt som ett anrop av metoden \code{toInt} i Scala; alltså att dubbelprecisionsflyttal omvandlas till heltal genom avkortning av alla decimaler.


Ändra huvudprogrammet så det anropar \jcode{fillRnd(xss, 6)}. %
% \begin{Code}[language=Java]
%     public static void main(String[] args) {
%         System.out.println("Hello JavaArrayTest!");
%         int[][] xss = new int[10][5];
%         fillRnd(xss, 6);
%         showMatrix(xss);
%     }
% \end{Code}
Programmet ska ge en utskrift som liknar följande:
\begin{REPL}
Hello JavaArrayTest!

--- showMatrix ---
[0][0] = 6; [0][1] = 2; [0][2] = 6; [0][3] = 3; [0][4] = 5;
[1][0] = 2; [1][1] = 4; [1][2] = 6; [1][3] = 1; [1][4] = 1;
[2][0] = 5; [2][1] = 4; [2][2] = 4; [2][3] = 1; [2][4] = 5;
[3][0] = 4; [3][1] = 6; [3][2] = 6; [3][3] = 1; [3][4] = 3;
[4][0] = 4; [4][1] = 6; [4][2] = 2; [4][3] = 3; [4][4] = 2;
[5][0] = 2; [5][1] = 4; [5][2] = 5; [5][3] = 5; [5][4] = 3;
[6][0] = 6; [6][1] = 5; [6][2] = 2; [6][3] = 4; [6][4] = 3;
[7][0] = 1; [7][1] = 6; [7][2] = 1; [7][3] = 6; [7][4] = 2;
[8][0] = 1; [8][1] = 1; [8][2] = 5; [8][3] = 3; [8][4] = 2;
[9][0] = 1; [9][1] = 1; [9][2] = 1; [9][3] = 5; [9][4] = 4;

\end{REPL}

\SOLUTION

\TaskSolved \what
     %starts with: \label{task:arraymatrix-java} \%%%

%6.a)
\SubtaskSolved  Vid initialisering fylls alla element i \code{xss} med standardvärdet för typen, \code{0} i fallet med \code{int}. Den yttre \code{for}-loopen i \code{showMatrix()} itererar över raderna i \code{xss}. Den inre \code{for}-loopen itererar i sin tur längs med elementen på den aktuella raden och skriver ut rad, kolumn och innehåll. Efter varje rad sker en radbrytning, så att en rad i utskriften även motsvarar en rad i matrisen.\\
Exempel på skillnader mellan användning av matriser i scala och java:
\begin{itemize}
\item åtkomst: \code{minArray(rad)(kolumn)} respektive \code{minArray[rad][kolumn]}
\item typnamn: \code{Array[Array[elementTyp]]} respektive  \code{elementTyp[][]}
\item allokering: \code{Array.ofDim[typ](xDim,yDim)} respektive \code{new typ[xDim][yDim]}
\end{itemize}

%6.b)
\SubtaskSolved  \begin{Code}[language=Java]
public class JavaArrayTest {

	public static void showMatrix(int[][] m){
		System.out.println("\n--- showMatrix ---");
		for (int row = 0; row < m.length; row++){
			for (int col = 0; col < m[row].length; col++) {
				System.out.print("[" + row + "]");
				System.out.print("[" + col + "] = ");
				System.out.print(m[row][col] + ";");
			} System.out.println();
		}
	}

	public static void fillRnd(int[][] m, int n){
		for (int row = 0; row < m.length; row++){
			for (int col = 0; col < m[row].length; col++) {
				m[row][col] = (int) (Math.random() * n + 1);
			}
		}
	}

	public static void main(String[] args) {
    System.out.println("Hello JavaArrayTest!");
		int[][] xss = new int[10][5];
		fillRnd(xss, 6);
		showMatrix(xss);
	}
}
\end{Code}

\QUESTEND


%\ExtraTasks %%%%%%%%%%%%%%%%%%%


\WHAT{Översätta från Java till Scala.}

\QUESTBEGIN

\Task  \what~ Översätt nedan kod från Java till Scala. Skriv koden i en fil som heter \texttt{showInt.scala} och kalla Scala-objektet med \code{main}-metoden för \code{showInt}. Läs tipsen som följer efter koden innan du börjar.

\javainputlisting[numbers=left]{examples/scalajava/JShowInt.java}

\emph{Tips:}
\begin{itemize}[nolistsep, noitemsep]
\item En Javaklass med bara statiska medlemmar motsvaras av ett singeltonobjekt i Scala, alltså en \code{object}-deklaration. Scala har därför inte nyckelordet \jcode{static}.
\item Typen \jcode{Object} i Java motsvaras av Scalas \code{Any}.
\item Du kan använda Scalas möjlighet med default-argument (som saknas i Java) för att bara definiera en enda \code{show}-metod med en tom sträng som default \code{msg}-argument.
\item I Scala har objekt av typen \code{Char} en metod \code{def *(n: Int): String} som skapar en sträng med tecknet repeterat \code{n} gånger. Men du kan ju välja att ändå implementera metoden \code{repeatChar} med \code{StringBuilder} som nedan om du vill träna på att översätta en \code{for}-loop från Java till Scala.
\item I stället för \code{Scanner.nextLine} kan du använda \code{scala.io.StdIn.readLine} som tar en prompt som parameter, men du kan också använda \code{Scanner} i Scala om du vill träna på det.
\item I Java \emph{måste} man använda nyckelordet \jcode{return} om metoden inte är en \jcode{void}-metod, medan man i Scala faktiskt \emph{får} använda \code{return} även om man brukar undvika det och i stället utnyttja att satser i Scala också är uttryck.
\end{itemize}
Kompilera din Scala-kod och kör i terminalen och testa så att allt funkar. Vill du även kompilera Java-koden så finns den i kursens repo i filen\\ \texttt{compendium/examples/scalajava/JShowInt.java}


\SOLUTION


\TaskSolved \what


\begin{Code}[numbers=left]
object showInt {
  def show(obj: Any, msg: String = ""): Unit = println(msg + obj)

  def repeatChar(ch: Char, n: Int): String = ch.toString * n

  def showInt(i: Int): Unit = {
    val leading = Integer.numberOfLeadingZeros(i)
    val binaryString = repeatChar('0', leading) + i.toBinaryString
    show(i,               "Heltal : ")
    show(i.asInstanceOf[Char],         "Tecken : ")
    show(binaryString,    "Binärt : ")
    show(i.toHexString,   "Hex    : ")
    show(i.toOctalString, "Oktal  : ")
  }


  import scala.io.StdIn.readLine
  import scala.util.{Try,Success,Failure}

  def loop: Unit =
    Try { readLine("Heltal annars pang: ").toInt } match {
      case Failure(e) => show(e); show("PANG!")
      case Success(i) => showInt(i); loop
    }

  def main(args: Array[String]): Unit =
    if(args.length > 0) args.foreach(i => showInt(i.toInt))
    else loop
}
\end{Code}



\QUESTEND






\WHAT{Innehållslikhet och referenslikhet i Java.}

\QUESTBEGIN

\Task  \what~ Studera och prova denna fallgrop med innehållslikhet: \href{https://github.com/bjornregnell/lth-eda016-2015/blob/master/lectures/examples/eclipse-ws/lecture-examples/src/week10/generics/TestPitfall3.java}{TestPitfall3.java}







\SOLUTION


\TaskSolved \what
     %%%TODO number  6 %%%starts with: \TODO Fallgrop med Point som in%%%



\QUESTEND




%\AdvancedTasks %%%%%%%%%%%%%%%%%


\WHAT{Implementera innehållslikhet i Java.}

\QUESTBEGIN

\Task  \what~\Pen Studera fallgropar för hur man skriver en \code{equals}-metod i Java här:
\href{http://www.artima.com/lejava/articles/equality.html}{www.artima.com/lejava/articles/equality.html} och jämför med  det fullständiga receptet för hur man skriver en välfungerande \code{equals} och \code{hashcode} i Scala här: \href{http://www.artima.com/pins1ed/object-equality.html}{www.artima.com/pins1ed/object-equality.html}

\Subtask Vilka skillnader och likheter finns vid överskuggning av equals i Java respektive Scala, som ska ge en fungerande innehållstest för en hierarki med bastyper och subtyper?

\Subtask Vilka fallgropar är gemensamma för Java och Scala?\SOLUTION


\TaskSolved \what
     %%%TODO number  7 %%%starts with: \TODO \emph{Gränssnitt i Scala %%%



\QUESTEND

%!TEX encoding = UTF-8 Unicode
%!TEX root = ../compendium2.tex

\Lab{java}

\begin{Goals}
\item Förstå skillnaden mellan primitiva typer och objekt i Java.
\item Kunna förklara hur autoboxing fungerar i Java.
\item Kunna förklara vad statiska metoder och attribut i Java innebär.
\item Kunna använda \code{ArrayList} och arrayer i Java.
\item Kunna använda Java-klassen \code{Scanner}.
\item Kunna skapa en for-sats i Java.
\item Känna till hur man kan förenkla användandet av Java och Scala i samma program med hjälp av \code{scala.jdk.CollectionConverters}.
\end{Goals}

\begin{Preparations}
\item Gör övningarna tidigare i detta Appendix.
\item Studera given kod här: \\ \href{https://github.com/lunduniversity/introprog/tree/master/workspace/javatext/}{github.com/lunduniversity/introprog/tree/master/workspace/javatext/}
\end{Preparations}

\subsection{Krav}

Du ska skapa ett textspel för terminalen som är (lagom) intressant/roligt att spela och sparar poäng per spelomgång för olika spelare. Till din hjälp har du den färdiga filen \code{Main.java} (som går bra att förändra om det behövs) samt de två kodskeletten \code{Game.java} och \code{UserInterface.scala}. Ditt textspel ska köras i terminalen och uppfylla följande krav och riktlinjer:

\begin{enumerate}
  \item När ditt program kör ska man ska kunna starta flera spelomgångar efter varandra utan att behöva avsluta programmet.
  \item För varje spelomgång ska programmet komma ihåg spelarens namn\footnote{eller spelar\emph{nas} namn om det är ett spel för två eller flera personer} med tillhörande resultat.
  \item Efter begäran ska programmet kunna visa en topplista med bästa poäng, både för alla spelare och för ett specifikt spelarnamn.
%  \item Speltiden för varje spelomgång ska mätas och sparas tillsammans med poängresultatet för respektive spelare.
  \item Koden för själva spelet ska vara skriven i Java, men Scala ska användas för att implementera funktionerna i singelobjektet \code{UserInterface}.
  \item I Scala-koden ska du för träningens skull använda Java-klassen \code{java.util.Scanner} när du läser in data från terminalen.
  \item Koden i singelobjektet \code{UserInterface} ska använda omvandlingsmetoden \code{asScala} efter \code{import scala.jdk.CollectionConverters._} för att omvandla argument av typen \code{java.util.ArrayList}.
  \item Ditt spel ska i Java-kod använda minst en av datastrukturerna
  \code{ArrayList},
  \code{HashSet},
  \code{HashMap} ur paketet \code{java.util}, samt minst en array. (Den givna koden i \code{Main.java} räknas inte till detta krav.)
  \item Du ska spela någon annans halvfärdiga spel och, efter att du studerat koden, ge återkoppling på kodens läsbarhet.
  \item Du ska låta någon annan spela ditt halvfärdiga spel och visa din kod och fråga om återkoppling på läsbarheten. Du ska anteckna den återkoppling du får.
  \item Du ska inför redovisningen förbereda följande:
  \begin{enumerate}
    \item en kort genomgång av spelets idé,
    \item en kort förklaring av kodens struktur och de olika Java-klassernas ansvar,
    \item en kort redogörelse för den återkoppling du fått på din kods läsbarhet och hur du arbetat med att förbättra läsbarheten under dina stegvisa utvidgningar av din kod,
    \item en lista med koncept som du tränat på när du skapat ditt textspel.
  \end{enumerate}
\end{enumerate}

\subsection{Frivilliga extrauppgifter}

\begin{enumerate}
	\item Spara resultat i en fil efter varje spelomgång, och läs in resultat från filen antingen när programmet startas eller när användaren vill se poänglistan, så att det går att se spelresultat från tidigare körningar av programmet. Den kod du behöver lägga till för att åstadkomma detta kan vara skriven antingen i Java eller Scala. Tänk på att du kan behöva göra ändringar även i \code{Main}-klassen.
	\item   Mät speltiden för varje spelomgång och spara tiden tillsammans med poängresultatet för respektive spelare.

\end{enumerate}

\subsection{Inspiration och tips}

\begin{enumerate}
  \item Utgå från Hangman i veckans övning eller,
  \item Yatzy från tidigare övningar, eller
  \item skapa ett kortspel inspirerat av \code{shuffle}-labben, eller
  \item inspireras av listan med sällskapsspel på wikipedia:\\ \href{https://sv.wikipedia.org/wiki/Kategori:Sällskapsspel}{sv.wikipedia.org/wiki/Kategori:Sällskapsspel}
  \item eller hitta på ett eget textspel.
  \item Börja med en starkt förenklad variant som du sedan bygger vidare på.
  \item Kompilera och testa efter varje ändring, så att du hela tiden har ett fungerande program.
  \item Dela upp din spelkod i flera metoder, och även flera klasser om det är lämpligt.
  \item Det finns mycket information på nätet om hur man skriver Java-kod och använder JDK, t.ex. på \url{https://stackoverflow.com/}
  \item Träna på att använda JDK-dokumentationen här:\\ \url{https://docs.oracle.com/javase/8/docs/api/}
\end{enumerate}
 %TODO!!
%%!TEX root = ../compendium.tex

\chapter{Virtuell maskin}\label{appendix:vbox}

\section{Vad är en virtuell maskin?}

Du kan köra alla kursens verktyg i en så kallad virtuell maskin (vm). Det är ett enkelt och säkert sätt att installera ett nytt operativsystem i en "sandlåda" som inte påverkar din dators ursprungliga operativsystem. 

\section{Installera kursens vm}
Det finns en virtuell maskin förberedd med alla verktyg som du behöver förinstallerade. Gör så här:
\begin{enumerate}
\item     Installera VirtualBox v5 här: \\ \url{https://www.virtualbox.org/wiki/Downloads}
\item     Ladda ner filen vbox.zip här: \\ \url{http://fileadmin.cs.lth.se/pgk/vbox.zip} \\ OBS! Då filen är på nästan 4GB kan nedladdningen ta mycket lång tid.
\item     Packa upp filen vbox.zip i biblioteket "VirtualBox VMs" som du fick i din hemkatalog när du installerade VirtualBox. Du får då 3 filer som heter något med "introprog-ubuntu-64bit".
\item     Kolla med hjälp av denna sida: \\ \url{https://md5file.com/calculator} \\ så att filen "introprog-ubuntu-64bit.vdi" har denna sha256-cheksumma: \\ --- ska-stå-checksumma-här-sen ---
\item     Öppna VirtualBox och lägg till maskinen introprog-ubuntu-64bit genom menyn ''add''.
\item     Starta maskinen.
\item     Öppna ett terminalfönster och skriv scala och du är igång och kan göra första övningen!
\end{enumerate}

\section{Vad innehåller kursens vm?}

Den virtuella maskinen kör Xubuntu 14.04 med fönstermiljön XFCE, vilket är samma miljö som E-husets linuxdatorer kör. 

I den virtuella maskinen finns detta förinstallerat:

\begin{itemize}
\item Java JDK 8
\item Scala 2.11.8
\item Kojo 2.4.08
\item Eclipse Mars.2 med ScalaIDE 4.3
\item gedit med syntaxfärgning för Scala och Java
\item git
\item sbt
\item Ammonite REPL
\end{itemize}  %TODO!!

\part{Lösningar}

\setcounter{chapter}{11} %next is L in \Alph
\chapter{Lösningar till övningarna}\label{chapter:solutions}
\setcounter{section}{7}

\PreSolutionfalse

\let\QUESTBEGIN\ifPreSolution  % to mark formatting and numbering of exercises
\let\SOLUTION\else  % to mark solutions in the same file as questions
\let\QUESTEND\fi    % to mark end of exercise


%!TEX encoding = UTF-8 Unicode
%!TEX root = ../exercises.tex

\ifPreSolution

\Exercise{\ExeWeekEIGHT}\label{exe:W08}

\begin{Goals}
\item Kunna skapa och använda matriser med nästlade strukturer av \code{Vector}.
\item Kunna iterera över elementen i en matris med nästlade \code{for}-satser och \code{for}-\code{yield}-uttryck, samt nästlad applicering av \code{map} respektive \code{foreach}.
\item Kunna skapa och använda funktioner som tar matriser som parametrar.
\item Kunna skapa en enkel generisk klass och enkla generiska funktioner med hjälp av en typparameter.
\item Kunna beskriva skillnader och likheter mellan Scala och Java vad gäller indexering och iterering i matriser implementerade med nästlade arrayer.
%\item Kunna skapa och använda matriser med hjälp inbyggda arrayer i Java.
%\item Kunna använda nästlade \code{for}-satser i Java för att iterera över elementen i en matris.
\end{Goals}

\begin{Preparations}
\item \StudyTheory{08}
\end{Preparations}

\BasicTasks

\else

\ExerciseSolution{\ExeWeekEIGHT}

\BasicTasks

\fi



\WHAT{Para ihop begrepp med beskrivning.}

\QUESTBEGIN

\Task \what

\vspace{1em}\noindent Koppla varje begrepp med den (förenklade) beskrivning som passar bäst:

\begin{ConceptConnections}
  matris & 1 & & A & konkret typ, binds till typparameter vid kompilering \\ 
  generisk & 2 & & B & indexerbar datastruktur i två dimensioner \\ 
  typargument & 3 & & C & har abstrakt typparameter, typen är generell \\ 
  typhärledning & 4 & & D & kompilatorn beräknar typ ur sammanhanget \\ 
\end{ConceptConnections}

\SOLUTION

\TaskSolved \what

\begin{ConceptConnections}
  matris & 1 & ~~\Large$\leadsto$~~ &  A & indexerbar datastruktur i två dimensioner \\ 
  radvektor & 2 & ~~\Large$\leadsto$~~ &  F & matris av dimension $1\times{}m$ med $m$ horisontella värden \\ 
  kolumnvektor & 3 & ~~\Large$\leadsto$~~ &  G & matris av dimension $m\times{}1$ med $m$ vertikala värden \\ 
  kolonn & 4 & ~~\Large$\leadsto$~~ &  C & annat ord för kolumn \\ 
  generisk & 5 & ~~\Large$\leadsto$~~ &  B & har abstrakt typparameter, typen är generell \\ 
  typargument & 6 & ~~\Large$\leadsto$~~ &  D & konkret typ, binds till typparameter vid kompilering \\ 
  typhärledning & 7 & ~~\Large$\leadsto$~~ &  E & kompilatorn beräknar typ ur sammanhanget \\ 
\end{ConceptConnections}

\QUESTEND




\WHAT{Skapa matriser med hjälp av nästlade samlingar.}

\QUESTBEGIN

\Task  \what~  Man kan i ett datorprogram, med hjälp av samlingar som innehåller samlingar, skapa nästlade strukturer som kan indexeras i två dimensioner och på så sätt representera en  \textbf{matris}.\footnote{\href{https://sv.wikipedia.org/wiki/Matris}{sv.wikipedia.org/wiki/Matris}}

\Subtask Rita minnessituationen efter tilldelningen på rad 1 nedan. Vad har \code{m} för typ och värde? Vad har \code{m} för dimensioner? Hur sker indexeringen i ett datorprogram jämfört med i matematiken?

\begin{REPL}
scala> val m = Vector((1 to 5).toVector, (3 to 7).toVector)
scala> m.apply(0).apply(1)
scala> m(1)
scala> m(1)(4)
\end{REPL}

\Subtask Vad ger uttrycken på raderna 2, 3 och 4 ovan för värden och typ?

\Subtask Man kan i ett datorprogram mycket väl skapa tvådimensionella, nästlade strukturer där raderna \emph{inte} innehåller samma antal element. Det blir då ingen äkta matris i strikt matematisk mening, men man kallar ofta ändå en sådan struktur för en ''matris''. Vilken typ har variablerna \code{m2}, \code{m3}, \code{m4} och \code{m5} nedan?

\begin{REPL}
scala> val m2 = Vector(Vector(1,2,3),Vector(4,5),Vector(42))
scala> val m3 = Vector(Vector(1,2), Vector(1.0, 2.0, 3.0))
scala> val m4 = m3(1) +: Vector("a") +: m3
scala> val m5 = Vector.fill(42){ m2(1).map(e => (e * math.random()).toInt) }
\end{REPL}

\Subtask Vilken av variablerna \code{m2}, \code{m3}, \code{m4} och \code{m5} ovan representerar en äkta matris i matematisk mening? Vilken är dess dimensioner?

\SOLUTION

\TaskSolved \what

\SubtaskSolved   \includegraphics{../img/w09-solutions/1a} \\
Typ: \code{Vector[Vector[Int]]}\\
Värde: \code{Vector(Vector(1, 2, 3, 4, 5), Vector(3, 4, 5, 6, 7))} \\
Dimensioner: $2 \times 5$\\
Inom matematiken sker indexering enligt konvention med 1 som lägsta index. I scala är lägsta index 0, man använder s.k. 0-indexering. \footnote{Detta är inte fallet i alla programmeringsspråk, vilket du kan läsa mer om på \url{https://en.wikipedia.org/wiki/Array\_data\_type\#Index\_origin}}

\SubtaskSolved
\begin{REPL}
scala> val m = Vector((1 to 5).toVector, (3 to 7).toVector)
m: Vector[Vector[Int]] = Vector(Vector(1, 2, 3, 4, 5), Vector(3, 4, 5, 6, 7))

scala> m.apply(0).apply(1)
res4: Int = 2

scala> m(1)
res5: Vector[Int] = Vector(3, 4, 5, 6, 7)

scala> m(1)(4)
res6: Int = 7
\end{REPL}

\SubtaskSolved  \\
m2: \code{Vector[Vector[Int]]}\\
m3: \code{Vector[Vector[Int | Double]]}\\
m4: \code{Vector[Vector[Int | Double | String]]}\\
m5: \code{Vector[Vector[Int]]}

\SubtaskSolved  m5, $42 \times 2$

\QUESTEND





\WHAT{Skapa och iterera över matriser.}

\QUESTBEGIN

\Task  \label{matrices:task:yatzy} \what~  Du ska skapa matriser där varje rad representerar 5 kast med en tärning i spelet Yatzy.\footnote{\href{https://sv.wikipedia.org/wiki/Yatzy}{sv.wikipedia.org/wiki/Yatzy}}


\Subtask Definiera i REPL en funktion \code{def throwDie: Int = ???} som returnerar ett slumptal mellan 1 och 6.

\Subtask Skapa nedan heltalsmatris i REPL. Vilken dimension får matrisen?
\begin{REPL}
scala> val ds1 = for (i <- 1 to 1000) yield 
            for (j <- 1 to 5) yield throwDie
          
\end{REPL}

\Subtask Man kan också använda nedan varianter för att skapa en heltalsmatris. Vilken av varianterna \code{ds1} ... \code{ds6} tycker du är lättast att läsa och förstå? Prova respektive variant i REPL och ange vilken typ på \code{ds1} ... \code{ds6} som härleds av kompilatorn.
\begin{REPL}
val ds2 = (1 to 1000).map(i => (1 to 5).map(j => throwDie))
val ds3 = (1 to 1000).map(i => Vector.fill(5)(throwDie))
val ds4 = for (i <- 1 to 1000) yield Vector.fill(5)(throwDie)
val ds5 = Vector.fill(1000)(Vector.fill(5)(throwDie))
val ds6 = Vector.fill(1000, 5)(throwDie)
\end{REPL}


\Subtask Definiera en funktion \\ \code{def roll(n: Int): Vector[Int] = ???}\\ som ger en heltalsvektor med $n$ stycken slumpvisa tärningskast. Kasten ska vara sorterade i växande ordning; använd för detta ändamål samlingsmetoden \code{sorted}.


\Subtask \label{matrices:subtask:isyatzyforall} Definera i REPL en funktion \code{isYatzy(xs: Vector[Int]): Boolean = ???} som testar om alla elementen i en heltalsvektor är samma. Använd samlingsmetoden \code{forall}.


\Subtask Skapa en funktion  \\ \code{def diceMatrix(m: Int, n: Int): Vector[Vector[Int]] = ???} \\ som med hjälp av funktionen \code{roll} skapar en matris med \code{m} st vektorer med vardera \code{n} slumpvisa tärningskast.


\Subtask \label{matrices:subtask:diceMatrixToString} Skapa en funktion som returnerar en utskriftsvänlig sträng \\ \code{def diceMatrixToString(xss: Vector[Vector[Int]]): String = ???} \\med hjälp av \code{map} och \code{mkString}, som fungerar enligt nedan.
\begin{REPL}
scala> val dm2s = diceMatrixToString(diceMatrix(4, 5))
val dm2s: String = 1 4 4 6 6
1 1 2 6 6
2 4 4 5 6
1 1 5 6 6

scala> println(dm2s)
1 4 4 6 6
1 1 2 6 6
2 4 4 5 6
1 1 5 6 6
\end{REPL}



\Subtask Implementera funktionen \\ \code{def filterYatzy(xss: Vector[Vector[Int]]): Vector[Vector[Int]]} \\ som filtrerar fram alla yatzy-rader i matrisen \code{xss} enligt nedan. Använd din funktion \code{isYatzy} och samlingsmetoden \code{filter}.
\begin{REPL}
scala> println(diceMatrixToString(filterYatzy(diceMatrix(10000, 5))))
4 4 4 4 4
6 6 6 6 6
4 4 4 4 4
6 6 6 6 6
4 4 4 4 4
4 4 4 4 4
2 2 2 2 2
\end{REPL}



\Subtask Implementera funktionen \\
\code{def yatzyPips(xss: Vector[Vector[Int]]): Vector[Int] = ???}\\
som ska ge en vektor med de tärningsvärden som gav yatzy, för kasten i matrisen \code{xss} enligt nedan. Använd din funktion \code{filterYatzy}.
\begin{REPL}
scala> val dm = Vector(Vector(1,2,3,4,5),Vector(4,4,4,4,4),Vector(3,3,3,3,3))
scala> yatzyPips(dm)
val res42: Vector[Int] = Vector(4, 3)
\end{REPL}

\SOLUTION

\TaskSolved \what

\SubtaskSolved
\begin{Code}
def throwDie: Int = (math.random() * 6).toInt + 1
\end{Code}
Eller:
\begin{Code}
def throwDie: Int = scala.util.Random.nextInt(6) + 1
\end{Code}

\SubtaskSolved  Matrisdimension i matematisk notation: $1000 \times 5$, vilket motsvarar en matris med 1000 rader och 5 kolumner.

\SubtaskSolved
\begin{Code}
ds1: IndexedSeq[IndexedSeq[Int]]
ds2: IndexedSeq[IndexedSeq[Int]]
ds3: IndexedSeq[Vector[Int]]
ds4: IndexedSeq[Vector[Int]]
ds5: Vector[Vector[Int]]
ds6: Vector[Vector[Int]]
\end{Code}
\code{IndexedSeq} och \code{Vector} ovan finns i paketet \code{scala.collection.immutable}

\SubtaskSolved  \begin{Code}
def roll(n: Int) = Vector.fill(n)(throwDie).sorted
\end{Code}

\SubtaskSolved  \begin{Code}
def isYatzy(xs: Vector[Int]): Boolean = xs.forall(_ == xs(0))
\end{Code}



%2.g)
\SubtaskSolved  \begin{Code}
def diceMatrix(m: Int, n: Int): Vector[Vector[Int]] =
  Vector.fill(m)(roll(n))
\end{Code}

\SubtaskSolved  \begin{Code}
def diceMatrixToString(xss: Vector[Vector[Int]]): String =
  xss.map(_.mkString(" ")).mkString("\n")
\end{Code}


%2.j)
\SubtaskSolved
\begin{Code}
def filterYatzy(xss: Vector[Vector[Int]]): Vector[Vector[Int]] =
  xss.filter(isYatzy)
\end{Code}



%2.m)
\SubtaskSolved  \begin{Code}
def yatzyPips(xss: Vector[Vector[Int]]): Vector[Int] =
  filterYatzy(xss).map(_.head)
\end{Code}

\QUESTEND








\WHAT{En oföränderlig, generisk matris-klass till veckans laboration \hyperref[section:lab:\LabWeekEIGHT]{\texttt{\LabWeekEIGHT}}.}

\QUESTBEGIN

\Task\label{exe:matrices:labprep}  \what~Under veckans laboration ska du simulera en enkel form av ''liv'' som består av celler i ett rutnät. För detta ändamål har vi nytta av en matris-klass som du ska implementera steg för steg i denna övning.
Skapa case-klassen nedan med en editor i filen \code{Matrix.scala}. Testa din lösning med hjälp av valfri \hyperref[appendix:ide]{IDE}, t.ex. \code{scalaide} eller \code{idea}.
\begin{Code}
case class Matrix(data: Vector[Vector[String]]){
  def apply(row: Int, col: Int): String = data(row)(col)
}
object Matrix {
  def fill(dim: (Int, Int))(value: String): Matrix =
    Matrix(Vector.fill(dim._1, dim._2)(value))
}
\end{Code}

\begin{REPLnonum}
scala> val m = Matrix.fill(3,4)("hej")
scala> val e = m(2, 2)
\end{REPLnonum}

\Subtask Vad får \code{m} ovan för typ?

\Subtask Vad får \code{e} ovan för typ?

\Subtask På hur många ställen måste du ändra i \code{Matrix} ovan för att den i stället ska representera en matris av heltal?

\Subtask Du ska nu med hjälp av en \textbf{typparameter} göra \code{Matrix} \textbf{generisk} \Eng{generic}, så att den blir en mer användbar matrisklass som kan innehålla element av vilken typ som helst. Genomför följande ändringar i \code{Matrix.scala}:

\begin{itemize}[noitemsep, nolistsep]
  \item Lägg till en typparameter \code{T} inom klammerparenteser efter namnet \code{Matrix} på alla ställen där det förekommer \emph{utom} efter namnet på kompanjonsobjektet\footnote{Singelobjekt kan inte ha typparametrar, men deras medlemmar kan.}.
  \item Byt ut \code{String} mot \code{T} på alla ställen där \code{String} förekommer.
  \item Lägg till en typparameter \code{T} inom klammerparenteser efter \code{def fill}.
\end{itemize}
Testa din generiska klass i REPL genom att skapa en boolesk matris:
\begin{REPLnonum}
scala> val bm = Matrix.fill(3,4)(false)
scala> val be = bm(0, 0)
\end{REPLnonum}

\Subtask Vad får \code{bm} ovan för typ?

\Subtask Vad får \code{be} ovan för typ?

\Subtask Lägg en kodrad i början av klasskroppen som med hjälp av \code{require} garanterar att alla rader i matrisen är lika långa.

\Subtask Lägg till en medlem \code{val dim: (Int, Int)} i klasskroppen efter \code{require}-satsen som ger ett par (alltså en 2-tupel) med antalet rader resp. kolumner i matrisen.

\Subtask Lägg till en metod \code{def updated(row: Int, col: Int)(value: T): Matrix[T]} som ger en ny matris där element på platsen \code{(row, col)} har uppdaterats till \code{value}.

\Subtask Lägg till en metod \code{def foreachIndex(f: (Int, Int) => Unit): Unit} som för varje index i \code{data} applicerar funktionen \code{f}.

\Subtask Lägg till en metod \code{override def toString} som så att en instans av \code{Matrix} visas enligt följande:
\begin{REPLnonum}
scala> val dm = Matrix.fill(3,4)(42.0)
val dm: Matrix[Double] =
Matrix of dim (3,4):
42.0 42.0 42.0 42.0
42.0 42.0 42.0 42.0
42.0 42.0 42.0 42.0
\end{REPLnonum}


\SOLUTION


\TaskSolved \what

\SubtaskSolved Typen på \code{m} blir \code{Matrix}.

\SubtaskSolved Typen på \code{e} blir \code{String}.

\SubtaskSolved Man behöver ändra på 3 ställen från \code{String} till \code{Int}.

\SubtaskSolved Generisk matris \code{Matrix[T]} för element av godtycklig typ \code{T}:

\begin{CodeSmall}
case class Matrix[T](data: Vector[Vector[T]]):
  def apply(row: Int, col: Int): T = data(row)(col)

object Matrix:
  def fill[T](dim: (Int, Int))(value: T): Matrix[T] =
    Matrix[T](Vector.fill(dim._1, dim._2)(value))
\end{CodeSmall}

\SubtaskSolved Tack vare kompilatorns typinferens så får \code{bm} typen \code{Matrix[Boolean]}.

\SubtaskSolved Typen på \code{be} blir \code{Boolean}.

\noindent \SubtaskSolved \SubtaskSolved \SubtaskSolved \SubtaskSolved \SubtaskSolved är alla implementerade i koden nedan: \vspace{-0.5em}
\begin{CodeSmall}
case class Matrix[T](data: Vector[Vector[T]]):
  require(data.forall(row => row.length == data(0).length))

  val dim: (Int, Int) = (data.length, data(0).length)

  def apply(row: Int, col: Int): T = data(row)(col)

  def updated(row: Int, col: Int)(value: T): Matrix[T] =
    Matrix(data.updated(row, data(row).updated(col, value)))

  def foreachIndex(f: (Int, Int) => Unit): Unit =
    for r <- data.indices; c <- data(r).indices do f(r, c)

  override def toString =
    s"""Matrix of dim $dim:\n${ data.map(_.mkString(" ")).mkString("\n") }"""

object Matrix:
  def fill[T](dim: (Int, Int))(value: T): Matrix[T] =
    Matrix[T](Vector.fill(dim._1, dim._2)(value))

\end{CodeSmall}

\QUESTEND


\clearpage

\ExtraTasks %%%%%%%%%%%%%%%%%%%%%%%%%%%%%%%%%%%%%%%%%%%%%%%%%


\WHAT{Imperativa matrisalgoritmer.}

\QUESTBEGIN

\Task  \what~Imperativa angreppssätt är nödvändiga att kunna när du stöter på samlingar och/eller språk som saknar funktionella metoder och/eller funktionsprogrammeringsmöjligheter. Genom att studera imperativa lösningar till de ofta mer koncisa funktionella lösningarna, får du träning i att skapa algoritmer som använder förändring genom tilldelning vid iterering.

\Subtask Implementera \code{isYatzy} från uppgift \ref{matrices:task:yatzy}\ref{matrices:subtask:isyatzyforall} igen, men nu med ett imperativt angreppssätt som använder en \code{while}-sats i stället för funktionella \code{forall}. Ta hjälp av en variabel \code{i} som håller reda på index och en variabel \code{foundDiff} som håller reda på om ett avvikande värde upptäcks. Funktionen kräver ca 9 rader, så det kan vara lämpligt att öppna en editor att skriva i medan du klurar ut lösningen. Börja med att skriva pseudokod, gärna med penna på papper. Prova genom att klistra in i REPL.

\Subtask En imperativ implementation av \code{diceMatrixToString} från uppgift \ref{matrices:task:yatzy}\ref{matrices:subtask:diceMatrixToString} med hjälp av förändringsbara  \code{StringBuilder}\footnote{\url{https://www.scala-lang.org/api/2.12.9/scala/collection/mutable/StringBuilder.html}} visas nedan. Förklara hur nedan kod fungerar. Vad händer om \code{xss} är tom? Vad händer om \code{xss} bara innehåller tomma vektorer? Nämn en fördel och en nackdel med att använda \code{val sb: StringBuilder} och \code{append}, jämfört med en vanlig, oföränderlig \code{var s: String} och \code{+} för tillägg i slutet.
\begin{Code}
def diceMatrixToString(xss: Vector[Vector[Int]]): String = 
  val sb = new StringBuilder()
  for(m <- xss.indices) do
    for(n <- xss(m).indices) do
      sb.append(xss(m)(n).toString)
      if n < xss(m).size - 1 then sb.append(" ")
      else if m < xss.size - 1 then sb.append("\n")
    end for
  end for
  sb.toString
\end{Code}

\Subtask Gör som träning en imperativ implementation av \code{filterYatzy} med en \code{for}-\code{do}-sats (alltså utan att använda \code{filter}, och utan att använda \code{yield}).


\Subtask Förklara hur nedan funktionella implementation av \code{filterYatzy} med \code{for}-\code{yield}-uttryck fungerar. Tycker du din imperativa lösning är lättare eller svårare att läsa och förstå jämfört nedan funktionella lösning?
\begin{CodeSmall}
def filterYatzy(xss: Vector[Vector[Int]]): Vector[Vector[Int]] = 
  (for i <- xss.indices if isYatzy(xss(i)) yield xss(i)).toVector
\end{CodeSmall}


\SOLUTION

\TaskSolved \what

\SubtaskSolved  \begin{Code}
def isYatzy(xs: Vector[Int]): Boolean = 
  var foundDiff = false
  var i = 0
  while (i < xs.size && !foundDiff) do
    foundDiff = xs(i) != xs(0)
    i += 1
  end while
  !foundDiff
\end{Code}


\SubtaskSolved  Funktionen går igenom varje matrisrad, där den i sin tur går igenom
varje element på raden och lägger till i \code{StringBuilder}-objektet. Om det inte är
det sista elementet på raden läggs även ett blanktecken till, annars läggs ett
nyradstecken till. Undantaget är sista raden, där inget nyradstecken läggs till.
Slutligen konverteras \code{StringBuilder}-objektet till en \code{String} som
returneras.


Är \code{xss} tom blir \code{xss.indices} en tom \code{Range} och den yttre \code{for}-loopen hoppas över och en tom sträng returneras.
Är alla rader tomma hoppas i stället de inre \code{for}-looparna över, med samma resultat.

\emph{Fördel:} \code{StringBuilder} är snabbare vid tillägg på slutet vid stora strängar (men här kommer det inte märkas eftersom strängen är så liten).

\emph{Nackdel:} StringBuilder-koden uppfattas av många som svårare att läsa.

\SubtaskSolved
\begin{Code}
def filterYatzy(xss: Vector[Vector[Int]]): Vector[Vector[Int]] = 
  var result: Vector[Vector[Int]] = Vector()
  for i <- xss.indices if isYatzy(xss(i)) do result = result :+ xss(i)
  result
\end{Code}

\SubtaskSolved  Varje looprunda ger en vektor \code{xss(i)} om filtervillkoret är uppfyllt och resultatet av \code{for}-uttrycket blir en vektor med vektorer som är yatzyslag.

\QUESTEND



\WHAT{Strängtabell med kolumnrubriker.}

\QUESTBEGIN

\Task  \what~  %Denna övning utgör en början på laboration \hyperref[section:lab:survey]{\texttt{survey}} i avsnitt \ref{section:lab:survey} på sidan \pageref{section:lab:survey}.

\Subtask Implementera case-klassen \code{Table} enligt specifikationen nedan. Du kan förutsätta att alla rader har lika många kolumner som antalet element i \code{headings}, samt att alla rubrikerna i \code{headings} är unika. Parametern \code{sep} anger det tecken som används för att separera kolumner. Detta förutsätts också gälla för indatafiler som läses in med \code{fromFile}.

\emph{Tips:}
\begin{itemize}%[nolistsep,noitemsep]
\item Värdet \code{indexOfHeading} kan skapas med hjälp av metoden \code{zipWithIndex} som fungerar på alla sekvenssamlingar, samt metoden \code{toMap} som fungerar på sekvenser av 2-tupler. Undersök först hur metoderna fungerar i REPL och sök upp deras dokumentation.
\item Skapa en indatafil som du kan använda för att testa att \code{Table} fungerar.
\end{itemize}


\begin{CodeSmall}
case class Table(
  data: Vector[Vector[String]],
  headings: Vector[String],
  sep: Char
):
  /** A 2-tuple with (number of rows, number of columns) in data */
  val dim: (Int, Int) = ???

  /** The element in row r and column c of data, counting from 0 */
  def apply(r: Int, c: Int): String = ???

  /** The row-vector r in data, counting from 0 */
  def row(r: Int): Vector[String]= ???

  /** The column-vector c in data, counting from 0 */
  def col(c: Int): Vector[String] = ???

  /** A map from heading to index counting from 0 */
  lazy val indexOfHeading: Map[String, Int] = ???

  /** The column-vector with heading h in data */
  def col(h: String): Vector[String] = ???

  /** A vector with the distinct, sorted values of col with heading h */
  def values(h: String): Vector[String] = ???

  /** Headings and data with columns separated by sep */
  override lazy val toString: String = ???

object Table:
  /** Creates a new Table from fileName with columns split by sep */
  def fromFile(fileName: String, sep: Char = ';'): Table = ???
\end{CodeSmall}

\Subtask Skapa med hjälp av \code{Table} ett program som kan köras från terminalen med \texttt{scala run infile.csv ';'} som ger en utskrift av antalet förekomster av olika värden i respektive kolumn (alltså en variant av registrering).



\SOLUTION

\TaskSolved \what

\SubtaskSolved  \begin{CodeSmall}
case class Table(
  data: Vector[Vector[String]],
  headings: Vector[String],
  sep: Char
):

  val dim: (Int, Int) = (data.size, headings.size)

  def apply(r: Int, c: Int): String = data(r)(c)

  def row(r: Int): Vector[String]= data(r)

  def col(c: Int): Vector[String] = data.map(r => r(c))

  lazy val indexOfHeading: Map[String, Int] = headings.zipWithIndex.toMap

  def col(h: String): Vector[String] = col(indexOfHeading(h))

  def values(h: String): Vector[String] = col(h).distinct.sorted

  override def toString: String =
    val s = sep.toString
    headings.mkString(s) + "\n" +data.map(_.mkString(s)).mkString("\n")

object Table:
  def fromFile(fileName: String, sep: Char = ';'): Table = 
    val lines = scala.io.Source.fromFile(fileName).getLines.toVector
    val matrix= lines.map(_.split(sep).toVector)
    new Table(matrix.tail, matrix.head, sep)
\end{CodeSmall}

\SubtaskSolved  \begin{CodeSmall}
@main 
def run(fileName: String, separator: String): Unit = 
  require(separator.length == 1, "separator ska vara exakt ett tecken")
  val t = Table.fromFile(fileName, separator.head)
  val counts: Vector[Vector[String]] =
    (0 until t.dim._2)
      .map(i => t.values(t.headings(i))
      .map(x => s"$x: ${t.col(i).count(_ == x)}"))
      .toVector
  for (i <- 0 until t.dim._2) do
    println(s"\nColumn: ${i + 1}, ${t.headings(i)}:")
    for (j <- 0 until counts(i).length) do
      println(counts(i)(j))
\end{CodeSmall}

\QUESTEND




\WHAT{Skapa ett yatzy-spel för användning i terminalen.}

\QUESTBEGIN

\Task  \what~%
% \Subtask Skapa en yatzy-matris enligt nedan specifikation. Läs om hur de olika predikaten för att kolla olika giltiga kombinationer i Yatzy ska fungera här: \href{https://en.wikipedia.org/wiki/Yahtzee}{en.wikipedia.org/wiki/Yahtzee}. Bygg ett huvudprogram som testar dina funktioner. Kompilera och testa i terminalen allteftersom du lägger till nya funktioner.
%
% \begin{CodeSmall}
% /** En skiss på en klass som kan användas till ett förenklat yatzy-spel */
% case class YatzyRows(val rows: Vector[Vector[Int]]) {
%   /** A new YatzyRows with a new row of 5 dice rolls appended to rows  */
%   def roll: YatzyRows = ???
%
%   /** A new YatzyRows with some indices of the last row re-rolled  */
%   def reroll(indices: Vector[Int]): YatzyRows = ???
% }
%
% object YatzyRows {
%   def isYatzy(xs: Vector[Int]): Boolean = ???
%   def isThreeOfAKind(xs: Vector[Int]): Boolean = ???
%   def isFourOfAKind(xs: Vector[Int]): Boolean = ???
%   def isFullHouse(xs: Vector[Int]): Boolean = ???
%   def isSmallStraight(xs: Vector[Int]): Boolean = ???
%   def isLargeStraight(xs: Vector[Int]): Boolean = ???
% }
% \end{CodeSmall}
%
%
% \Subtask Använd \code{YatzyRows} för att med hjälp av många tärningskast beräkna sannolikheter för några olika giltiga kombinationer. Använd, om du vill, möjligheten som reglerna ger att slå om tärningar i två ytterliggare kast, där de tärningar som slås om väljs slumpmässigt.
%
%\Subtask
Bygg ett förenklat yatzy-spel i terminalen där användaren kan bestämma vilka tärningar som ska slås om. Börja med något riktigt enkelt och bygg sedan vidare på ditt spel genom att införa fler och fler funktioner.

\SOLUTION


\TaskSolved \what
     %starts with: \emph{Skapa ett yatzy-spel för %%%

 --

% \SubtaskSolved   \begin{CodeSmall}
% /** En skiss på en klass som kan användas till ett förenklat yatzy-spel */
% case class YatzyRows(val rows: Vector[Vector[Int]]) {
%
%   private def throwDie: Int = (math.random() * 6).toInt + 1
%
%   /** A new YatzyRows with a new row of 5 dice rolls appended to rows */
%   def roll: YatzyRows = new YatzyRows(rows :+ Vector.fill(5)(throwDie))
%
%   /** A new YatzyRow with some indices of the last row re-rolled */
%   def reroll(indices: Vector[Int]): YatzyRows =
%     new YatzyRows(rows :+ rows(rows.length - 1).zipWithIndex.map {
%       case (x, i) => if (indices.contains(i)) throwDie else x
%     })
% }
% object YatzyRows {
%
%   def isYatzy(xs: Vector[Int]): Boolean = xs.forall(_ == xs(0))
%
%   def isThreeOfAKind(xs: Vector[Int]): Boolean =
%     xs.exists(x => xs.count(_ == x) >= 3)
%
%   def isFourOfAKind(xs: Vector[Int]): Boolean =
%     xs.exists(x => xs.count(_ == x) >= 4)
%
%   def isFullHouse(xs: Vector[Int]): Boolean =
%     xs.exists(x => xs.count(_ == x) == 3) &&
%     xs.exists(x => xs.count(_ == x) == 2)
%
%   def isSmallStraight(xs: Vector[Int]): Boolean =
%     xs.forall(x => xs.count(_ == x) == 1) && !xs.exists(_ == 6)
%
%   def isLargeStraight(xs: Vector[Int]): Boolean =
%     xs.forall(x => xs.count(_ == x) == 1) && !xs.exists(_ == 1)
% }
%
% \end{CodeSmall}
% Observera att fem stycken 2:or uppfyller kraven för Yatzy, men även för triss och fyrtal.
%
% \SubtaskSolved   Slumpen gör att utfallet inte kommer stämma exakt överens med teorin, men för ett stort antal kast bör resultaten hamna ganska nära. De teoretiska sannolikheterna (utan omkast) finns i \ref{yatzyProb}.
% \begin{table}[h]
% \centering
% \caption{Sannolikhet för olika Yatzy-resultat}
% \label{yatzyProb}
% \begin{tabular}{ll}
% Yatzy&  $0,077\%$  \\
% $\geq3$ av samma& $21\%$\\
% $\geq4$ av samma& $2,0\%$\\
% Kåk& $3,9\%$\\
% Liten stege& $1,5\%$\\
% Stor stege& $1,5\%$
% \end{tabular}
% \end{table}
%
% Kodexempel:
% \begin{CodeSmall}
% import YatzyRows._
%
% object YatzyStats extends App {
%   val n = 1000000.0
%   var yr = YatzyRows(Vector(Vector[Int]()))
%   for (i <- 1 to n.toInt) yr = yr.roll
%   println(s"Yatzy: ${yr.rows.count(isYatzy(_)) / n * 100}%")
%   println(s"Three of a kind: ${yr.rows.count(isThreeOfAKind(_)) / n * 100}%")
%   println(s"Four of a kind: ${yr.rows.count(isFourOfAKind(_)) / n * 100}%")
%   println(s"Full house: ${yr.rows.count(isFullHouse(_)) / n * 100}%")
%   println(s"Small straight: ${yr.rows.count(isSmallStraight(_)) / n * 100}%")
%   println(s"Large straight: ${yr.rows.count(isLargeStraight(_)) / n * 100}%")
% }
% \end{CodeSmall}
%
% \SubtaskSolved  --

\QUESTEND






\clearpage

\AdvancedTasks %%%%%%%%%%%%%%%%%


\WHAT{Generiska funktioner.}

\QUESTBEGIN

\Task  \what~  En generisk funktion har (minst) en typparameter inom klammerparenteser efter namnet, till exempel \code{[T]}. Denna typ förekommer sedan som typ på (någon av) parametrarna i parameterlistan. Kompilatorn härleder en konkret typ vid kompileringstid och ersätter typparametern med denna konkreta typ. På så sätt kan en funktion fungera för många olika typer.

\Subtask Förklara för varje rad nedan vad som händer.

\begin{REPL}
scala> def tnirp[T](x: T): Unit = println(x.toString.reverse)
scala> tnirp(42)
scala> tnirp("hej")
scala> case class Gurka(vikt: Int)
scala> tnirp(Gurka(42))
scala> tnirp[String](42)
scala> tnirp[Double](42)
\end{REPL}

\Subtask Man kan kombinera generiska funktioner med funktioner som tar funktioner som parametrar. Det är så \code{map} och \code{foreach} är implementerade. Förklara för varje rad nedan vad som händer.

\begin{REPL}
scala> def compose[A, B, C](f: A => B, g: B => C)(x: A): C = g(f(x))
scala> def inc(x: Int): Int = x + 1
scala> def half(x: Int): Double = x / 2.0
scala> compose(inc, half)(42)
scala> compose(half, inc)(42)
\end{REPL}

\Subtask Hur lyder felmeddelandet på sista raden ovan? Ändra \code{inc} och/eller \code{half} så att typerna passar.

\SOLUTION

\TaskSolved \what
     %starts with: \emph{Generiska funkioner.} En %%%

%4.a)
\SubtaskSolved   \begin{enumerate}
\item --
\item Strängrepresentationen av \code{42} spegelvänds
\item \code{"hej"} spegelvänds - \code{toString} av en sträng ger en likadan sträng
\item --
\item Gurk-objektets strängrepresentation spegelvänds
\item Funktionens typparameter matchar inte parameterns typ: \code{42} är ingen sträng
\item Implicit typkonvertering till \code{Double} sker för att stämma överens med typparametern, vilket ger en strängrepresentation med decimal
\end{enumerate}

%4.b)
\SubtaskSolved   \begin{enumerate}
\item En funktion definieras så att den tar emot två andra funktioner som argument, sätter ihop dem, och matar in ett tredje argument till den den sammansatta funktionen.
\item En funktion som inkrementerar ett heltal med 1 definieras.
\item En funktion som halverar ett flyttal definieras.
\item \code{42} matas in i \code{inc()} och resultatet (\code{43}) matas vidare till \code{half()}. Inuti \code{half()} sker implicit typkonvertering till \code{Double} då talet divideras med ett flyttal (\code{2.0}) och resultatet blir \code{43.0 / 2.0}, alltså \code{21.5}.
\item Resultatet från \code{half()} är av typ \code{Double}, medan \code{inc()} tar emot ett argument av typ \code{Int}. Då flyttal generellt inte kan konverteras till heltal utan informationsförlust sker ingen implicit konvertering, istället sker ett kompileringsfel.
\end{enumerate}

%4.c)
\SubtaskSolved  \begin{Code}
def inc(x: Double): Double = x + 1.0
\end{Code}
Nu ges kompileringsfel på rad 4 istället, vilket kan lösas med följande ändring:
\begin{Code}
def half(x: Double): Double = x / 2.0
\end{Code}

\QUESTEND




\WHAT{Generiska klasser.}

\QUESTBEGIN

\Task  \what~  Även klasser kan vara generiska. En generisk klass har (minst) en typparameter inom klammerparenteser efter klassens namn.

\Subtask Testa nedan generiska klass \code{Cell[T]} i REPL. Skapa instanser av klassen \code{Cell[T]} där typparametern \code{T} binds till olika konkreta typer och förklara vad som händer.

\begin{REPL}
scala> class Cell[T](var value: T):
         override def toString = "Cell(" + value + ")"
       
scala> new Cell(42)
scala> new Cell("hej")
scala> new Cell(new Cell(math.Pi))
scala> new Cell[String](42)
scala> new Cell[Double](42)
\end{REPL}

\Subtask Lägg till metoden \code{def concat[U](that: Cell[U]):Cell[String]} i klassen \code{Cell} som konkatenerar strängrepresentationerna av de båda cellvärdena.

\begin{REPL}
scala> val a = new Cell("hej")
scala> val b = new Cell(42)
scala> a concat b
\end{REPL}

\Subtask Vilken sorts celler kan du konkatenera om du tar bort typparameternamnet \code{U} i \code{concat} samtidigt som du använder \code{Cell[T]} som typ på värdeparametern \code{that}? Vad ger det för konsekvenser för celler av annan typ än \code{Cell[String]}?

\SOLUTION

\TaskSolved \what

%5.a)
\SubtaskSolved  --

%5.b)
\SubtaskSolved  \begin{Code}
class Cell[T](var value: T):
  override def toString = "Cell(" + value + ")"
  def concat[U](that: Cell[U]): Cell[String] = 
    Cell(s"$value${that.value}")
\end{Code}

%5.c)
\SubtaskSolved   Endast celler med samma typparameter kan nu konkateneras. Eftersom \code{concat()} returnerar ett objekt av typ \code{Cell[String]} kan ett ojämnt antal celler med någon annan typparameter än \code{String} alltså inte längre konkateneras. Är antalet jämnt går det att konkatenera dem parvis och sedan konkatenera de returnerade \code{Cell[String]}-objekten, men det är något omständigt.

\QUESTEND

\WHAT{Implementera fler generiska metoder i \code{Matrix[T]}.}

\QUESTBEGIN

\Task \what~ Bygg vidare på uppgift \ref{exe:matrices:labprep} och implementera nedan specifikation. Skapa egna tester som kontrollerar att alla metoder fungerar som förväntat.

\begin{ScalaSpec}{Matrix[T]}
/** En oföränderlig, generisk Matris-klass. */
case class Matrix[T](data: Vector[Vector[T]]):
  require(???)  // garantera att alla rader har lika många kolumner

  /** Ger ett par med antal rader och kolumner. */
  val dim: (Int, Int) = ???

  /** Ger elementet på plats (row, col). */
  def apply(row: Int, col: Int): T = ???

  /** Ger en ny matris där elementet på plats (row, col) har värdet value. */
  def updated(row: Int, col: Int)(value: T): Matrix[T] =  ???

  /** Applicerar f på alla element. */
  def foreach(f: T => Unit): Unit = ???

  /** Applicerar f på alla index. */
  def foreachIndex(f: (Int, Int) => Unit): Unit = ???

  /** Ger en ny matris med resultaten av elementvis applicering av f. */
  def map[U](f: T => U): Matrix[U] = ???

  /** Ger en ny matris med resultaten av applicering av f på varje index. */
  def mapIndex[U](f: (Int, Int) => U): Matrix[U] = ???

  /** Ger en utskriftsvänlig strängrepresentation av matrisen. */
  override def toString = ???

object Matrix:
  /** Ger en matris med dimension dim där alla element har värdet value. */
  def fill[T](dim: (Int, Int))(value: T): Matrix[T] = ???
\end{ScalaSpec}

\SOLUTION


\TaskSolved \what

\begin{CodeSmall}
case class Matrix[T](data: Vector[Vector[T]]):
  require(data.forall(row => row.size == data(0).size))

  val dim: (Int, Int) = (data.length, data(0).length)

  def apply(row: Int, col: Int): T = data(row)(col)

  def updated(row: Int, col: Int)(value: T): Matrix[T] =
    Matrix(data.updated(row, data(row).updated(col, value)))

  def foreach(f: T => Unit): Unit = data.foreach(_.foreach(f))

  def foreachIndex(f: (Int, Int) => Unit): Unit =
    for r <- data.indices; c <- data(r).indices do f(r, c)

  def map[U](f: T => U): Matrix[U] = Matrix(data.map(_.map(f)))

  def mapIndex[U](f: (Int, Int) => U): Matrix[U] =
    var result = Matrix.fill(dim)(f(0,0))
    for 
      r <- data.indices
      c <- data(r).indices 
    do
      result = result.updated(r, c)(f(r, c))
    end for
    result

  override def toString =
    s"""Matrix of dim $dim:\n${ data.map(_.mkString(" ")).mkString("\n") }"""

object Matrix:
  def fill[T](dim: (Int, Int))(value: T): Matrix[T] =
    Matrix[T](Vector.fill(dim._1, dim._2)(value))
\end{CodeSmall}


\QUESTEND





% \WHAT{Skapa en generisk, oföränderlig matrisklass.}
%
% \QUESTBEGIN
%
% \Task \label{task:generic-matrix} \what~   Med hjälp av en typparameter kan vi skapa en matrisklass som kan innehålla vilka element som helst. Implementera nedan specifikation. Testa din matrisklass i REPL för olika typer av element.
%
% \begin{ScalaSpec}{Matrix[T]}
% case class Matrix[T](data: Vector[Vector[T]]){
%
%   def foreachRowCol(f: (Int, Int, T) => Unit): Unit =
%     for (r <- 0 until data.size) {
%       for (c <- 0 until data(r).size) {
%         f(r, c, data(r)(c))
%       }
%     }
%
%   def map[U](f: T => U): Matrix[U] = Matrix(data.map(_.map(f)))
%
%   /** The element at row r and column c */
%   def apply(r: Int, c: Int): T = ???
%
%   /** Gives Some[T](element) at row r and column c
%    *  if r and c are within index bounds, else None */
%   def get(r: Int, c: Int): Option[T] = ???
%
%   /** The row vector of row r */
%   def row(r: Int): Vector[T] = ???
%
%   /** The column vector of column c */
%   def col(c: Int): Vector[T] = ???
%
%   /** A new Matrix with element at row r and col c updated */
%   def updated(r: Int, c: Int, value: T): Matrix[T] = ???
% }
% object Matrix {
%   def fill[T](rowSize: Int, colSize: Int)(init: T): Matrix[T] =
%     new Matrix(Vector.fill(rowSize)(Vector.fill(colSize)(init)))
% }
% \end{ScalaSpec}
%
% \SOLUTION
%
%
% \TaskSolved \what
%      %%%TODO number  8 %%%starts with: \label{task:generic-matrix} \em%%%
%
% \SubtaskSolved  -- %%%TODO in task 8 %%%
%
%
%
% \QUESTEND
%

% \clearpage
%
% \WHAT{Skapa en Sprite-editor.}
%
% \QUESTBEGIN
%
% \Task  \what~ Använd matrisklassen från uppgift \ref{task:generic-matrix} för att göra en SpriteEditor med JColorChoser enligt nedan skiss.
%
% \begin{Code}
% object ColorChooser {
%   import java.awt.Color
%   import javax.swing.JColorChooser
%
%   var title = "Pick Color"
%   private val chooser = new JColorChooser(Color.BLACK)
%   private val dialog = JColorChooser.
%     createDialog(null, title, true, jcs, null, null)
%
%   def getColor(initColor: Color = Color.BLACK): Color = {
%     chooser.setColor(initColor)
%     dialog.setVisible(true)
%     chooser.getColor
%   }
% }
%
% class Sprite(// en bild med många lager av pixlar i olika färger
%   val id: String,
%   val size: (Int, Int),
%   val pixels: Matrix[Int],   // färg i colors, -1 betyder genomskinlig
%   var scale: Int,            // uppskalning av storlek i pixlar
%   var colors: Vector[Color], // tillgängliga färger
%   var pos: (Int, Int, Int)   // (row, col, layer)
% ){
%   def row = pos._1
%   def col = pos._2
%   def layer = pos._3
% }
%
% class SpriteEditor(
%     rows: Int = 64, cols: Int = 64,
%     scale: Int = 16, nColors: Int = 16) {
%   private val w = new SimpleWindow(???)
%   def edit: Unit = ???
% }
%
% \end{Code}
%
%
%
% \SOLUTION
%
%
% \TaskSolved \what
%      %%%TODO number  9 %%%starts with: \TODO \emph{Klasser för täta oc%%%
%
% \SubtaskSolved  -- %%%TODO in task 9 %%%
%
% \SubtaskSolved  -- %%%TODO in task 9 %%%
%
% \SubtaskSolved  -- %%%TODO in task 9 %%%
%
% \SubtaskSolved  -- %%%TODO in task 9 %%%
%
% \SubtaskSolved  -- %%%TODO in task 9 %%%
%
% \SubtaskSolved  -- %%%TODO in task 9 %%%
%
%
%
% \QUESTEND




% \WHAT{Klasser för täta och glesa matematiska matriser med flyttal.}
%
% \QUESTBEGIN
%
% \Task  \what~   Läs om matrisräkning här: \href{https://sv.wikipedia.org/wiki/Matris}{sv.wikipedia.org/wiki/Matris}
%
% \Subtask Skapa en oföränderlig klass \code{DenseMatrix} för matematiska matriser med dubbelprecisionsflyttal. \code{DenseMatrix} ska internt lagra elementen i en privat \emph{endimensionell} array av flyttal av typen \code{Array[Double]}.
%
% Klassen ska inte vara en case-klass. Det ska gå att skapa matriser med uttryck så som  \code{DenseMatrix.ofDim(3,7)(1.0,42,3.2,1.0,2.2,3)} tack vare ett kompanjonsobjekt med lämplig fabriksmetod som anropar den privata konstruktorn.  Om antalet element är för litet i förhållande till den angivna dimensionen så fyll på med nollor.
%
% \Subtask Överskugga metoderna equals och hashcode och ge \code{DenseMatrix} innehållslikhet i stället för referenslikhet.
%
% \Subtask Implementera egna innehålllikhetsmetoder med namnet \code{===} på \code{DenseMatrix} som är typsäker, d.v.s. bara tillåter innehållsjämförelse mellan täta matriser.
%
% \Subtask Läs om glesa matriser här: \href{https://sv.wikipedia.org/wiki/Gles_matris}{https://sv.wikipedia.org/wiki/Gles\_matris} och implementera \code{SparseMatrix} med ett privat attribut av typen \\ \code{mutable.Map[(Int, Int), Double]} som bara lagrar index som inte är noll.
%
% \Subtask Skapa ett \code{trait Matrix} som både \code{DenseMatrix} och \code{SparseMatrix} ärver, med lämpliga abstrakta och konkreta medlemmar. Implementera addition, subtraktion och multiplikation av täta och glesa matriser.
%
% %\Task \emph{Matriser med \jcode{ArrayList} i Java.} Om man i Java inte vet antalet element i matrisen från början kan man använda en lista av typen \jcode{ArrayList}, där varje element i sin tur innehåller en lista av typen\jcode{ArrayList}. Javas \jcode{ArrayList} är en generisk samling som motsvaras av Scalas \code{ArrayBuffer}. Generiska samlingar i Java kan endast innehålla referenstyper; vill man ha en primitiv typ, t.ex. \jcode{int}, behöver man packa in denna i en s.k. wrapper-klass, t.ex.  klassen \jcode{Integer}. Det finns en wrapper-klass för varje primitiv typ i Java. Matristypen för en heltalstyp i Java skrivs \jcode{ArrayList<ArrayList<Integer>>} där alltså \code{<T>} motsvarar Scalas hakparenteser \code{[T]} för typparametern T.
% %
% %
%
% \SOLUTION
%
% \TaskSolved \what
%      %%%TODO number  10 %%%starts with: \emph{Matriser med \jcode{Array%%%
%
% \SubtaskSolved  -- %%%TODO in task 10 %%%
% \QUESTEND

%!TEX encoding = UTF-8 Unicode
%!TEX root = ../exercises.tex

\ifPreSolution


\Exercise{\ExeWeekNINE}\label{exe:W09}

\begin{Goals}
%!TEX encoding = UTF-8 Unicode
%!TEX root = ../compendium2.tex

%\item Kunna skapa och använda tupler, som variabelvärden, parametrar och returvärden.

%\item Förstå skillnaden mellan ett objekt och en klass och kunna förklara betydelsen av begreppet instans.

%\item Kunna skapa och använda attribut som medlemmar i objekt och klasser och som som klassparametrar.

%\item Kunna beskriva den praktiska nyttan med att ett attribut är privat.

%\item Kunna byta ut implementationen av metoden \code{toString}.

%\item Kunna skapa och använda en objektfabrik med metoden \code{apply}.

%\item Kunna skapa och använda en enkel case-klass.

%\item Kunna använda operatornotation och förklara relationen till punktnotation.

%\item Förstå konsekvensen av uppdatering av föränderlig data i samband med multipla referenser.

%\item Kunna förklara den principiella skillnaderna mellan olika typer av samlingar.
\item Kunna skapa och använda tupler som parametrar och returvärden.
\item Känna till och kunna använda grundläggande metoder på samlingar.
\item Kunna skapa och använda både oföränderliga och föränderliga mängder.
\item Förstå skillnader och likheter mellan en mängd och en sekvens.
\item Kunna beskriva hur algoritmen linjärsökning fungerar.
\item Kunna skapa och använda både oföränderliga och föränderliga nyckel-värde-tabeller.
\item Kunna använda nyckel-värde-tabeller för att implementera registrering.
\item Förstå likheter och skillnader mellan en nyckel-värde-tabell och en sekvens.
\item Kunna spara och läsa data till/från textfiler på disk.
 
\end{Goals}

\begin{Preparations}
\item \StudyTheory{09}
\end{Preparations}

\else

\ExerciseSolution{\ExeWeekNINE}

\fi



\BasicTasks %%%%%%%%%%%%%%%%




\WHAT{Para ihop begrepp med beskrivning.}

\QUESTBEGIN

\Task \what

\vspace{1em}\noindent Koppla varje begrepp med den (förenklade) beskrivning som passar bäst:

\begin{ConceptConnections}
  mängd & 1 & & A & leta i sekvens tills sökkriteriet är uppfyllt \\ 
  nyckel-värde-tabell & 2 & & B & avkoda symbolsekvens och återskapa objekt i minnet \\ 
  mappning & 3 & & C & en unik identifierare \\ 
  nyckel & 4 & & D & egenskapen att finnas kvar efter programmets avslut \\ 
  persistens & 5 & & E & koda objekt till avkodningsbar sekvens av symboler \\ 
  serialisera & 6 & & F & oordnad samling av mappningar med unika nycklar \\ 
  de-serialisera & 7 & & G & \code|nyckel -> värde| \\ 
  linjärsöka & 8 & & H & oordnad samling med unika element \\ 
\end{ConceptConnections}

\SOLUTION

\TaskSolved \what

\begin{ConceptConnections}
  mängd & 1 & ~~\Large$\leadsto$~~ &  H & oordnad samling med unika element \\ 
  nyckel-värde-tabell & 2 & ~~\Large$\leadsto$~~ &  F & oordnad samling av mappningar med unika nycklar \\ 
  mappning & 3 & ~~\Large$\leadsto$~~ &  G & \code|nyckel -> värde| \\ 
  nyckel & 4 & ~~\Large$\leadsto$~~ &  C & en unik identifierare \\ 
  persistens & 5 & ~~\Large$\leadsto$~~ &  D & egenskapen att finnas kvar efter programmets avslut \\ 
  serialisera & 6 & ~~\Large$\leadsto$~~ &  E & koda objekt till avkodningsbar sekvens av symboler \\ 
  de-serialisera & 7 & ~~\Large$\leadsto$~~ &  B & avkoda symbolsekvens och återskapa objekt i minnet \\ 
  linjärsöka & 8 & ~~\Large$\leadsto$~~ &  A & leta i sekvens tills sökkriteriet är uppfyllt \\ 
\end{ConceptConnections}

\QUESTEND



\WHAT{Vad är en mängd?}
\QUESTBEGIN

\Task \what~ Förklara vad som händer nedan. Varför hamnar elementen i en ''konstig'' ordning? Varför ''försvinner'' det element?

\begin{REPL}
scala> val xs = Vector(1,2,3,1,2,3,4,5,7).toSet
xs: scala.collection.immutable.Set[Int] = Set(5, 1, 2, 7, 3, 4)
scala> xs.foreach(print)
512734
\end{REPL}

\SOLUTION

\TaskSolved \what~En mängd är en samling som snabbt kan ge svaret på frågan om ett visst element ingår i samlingen eller ej. Elementen i en mängd är unika. Tilläg av redan existerande element ignoreras. En mängd är inte en  sekvens, eftersom traversering med t.ex. \code{map} eller \code{foreach} inte (nödvändigtvis) sker i den ordning som elementen gavs när mängden konstruerades eller uppdaterades.

\QUESTEND


\WHAT{Använda mängder.}

\QUESTBEGIN

\Task \what

\vspace{1em}\noindent Para ihop varje uttryck till vänster med ett uttryck till höger som har samma värde:

\begin{ConceptConnections}
\input{generated/quiz-w09-setops-taskrows-generated.tex}
\end{ConceptConnections}

\SOLUTION

\TaskSolved \what

\begin{ConceptConnections}
  \code|Set(1, 2) ++ Set(1, 2)          | & 1 & ~~\Large$\leadsto$~~ &  I & \code|Set(1, 2)     | \\ 
  \code|(1 to 3).toSet                  | & 2 & ~~\Large$\leadsto$~~ &  G & \code|Set(1) + 2 + 3| \\ 
  \code|Vector.fill(3)(1).toSet         | & 3 & ~~\Large$\leadsto$~~ &  F & \code|Set(1, 2) - 2 | \\ 
  \code|Set(1, 2, 3) diff Set(1, 2)     | & 4 & ~~\Large$\leadsto$~~ &  B & \code|Set(3)        | \\ 
  \code|(1 to 7).toSet.apply(8)         | & 5 & ~~\Large$\leadsto$~~ &  H & \code|false         | \\ 
  \code|Set(1, 2, 3).sorted             | & 6 & ~~\Large$\leadsto$~~ &  D & \code|error: ...    | \\ 
  \code|Set(2,4) subsetOf (1 to 7).toSet| & 7 & ~~\Large$\leadsto$~~ &  E & \code|true          | \\ 
  \code|Set(1, -1, 2, -2).map(_.abs).sum| & 8 & ~~\Large$\leadsto$~~ &  A & \code|3             | \\ 
  \code|Set(1, 1, 1, 1, 1, 5).sum       | & 9 & ~~\Large$\leadsto$~~ &  C & \code|6             | \\ 
\end{ConceptConnections}

\QUESTEND


\WHAT{Räkna unika ord med hjälp av en mängd.}

\QUESTBEGIN

\Task \what~På veckans laboration ska vi göra automatisk språkbehandling av långa texter som vi delar upp i ord. Med metoden \code{s.split(' ').toVector} kan du dela upp en sträng \code{s} i en sekvens av ord, där \code{s} blivit uppdelad i många strängar vid varje blanktecken och alla blanktecken är borttagna.

\Subtask Använd metoderna \code{split} och \code{toSet} för skapa ett uttryck som beräknar hur många unika ord det finns i strängen \code{hej} nedan:
\begin{REPLnonum}
scala> val hej = "hej hej hemskt mycket hej"
\end{REPLnonum}

\Subtask Mängder är snabba på att kolla om ett element finns i mängden men du kan inte förvänta dig att elementen finns i någon viss ordning. Det finns en sekvenssamlingsmetod som skapar en sekvens med unika element ur en sekvens och behåller den ursprungliga ordningen. Vad heter metoden? \\\emph{Tips:} Leta i snabbreferensen eller sök på nätet. Metoden fungerar på alla samlingar som är av typen \code{Seq} och har ett namn som börjar med bokstäverna \code{di}.

\SOLUTION

\TaskSolved \what~

\SubtaskSolved
\begin{REPL}
scala> val hej = "hej hej hemskt mycket hej"
scala> val n = hej.split(' ').toSet.size
n: Int = 3
\end{REPL}

\SubtaskSolved Metoden \code{distinct} returnerar en sekvens med unika element och bibehållen ursprunglig ordning.

\QUESTEND




\WHAT{Skapa 2-tupler med metoden \code{->} som kan uttalas ''mappas till''.}

\QUESTBEGIN

\Task \what~Vi har tidigare sett hur två olika värden kan samlas i en 2-tupel, till exempel \code{(0, true)}. Par kan även skapas med hjälp av metoden \code{->} enligt nedan. Testa detta i REPL:
\begin{REPL}
scala> ("Skåne", "Lund")          // ett strängpar med vanlig 2-tupel
scala> "Skåne" -> "Lund"           // operatornotation med ->
scala> "Skåne".->("Lund")         // punktnotation med -> (inte alls vanligt)
\end{REPL}
Metoden \code{->} fungerar med alla typer och är en fabriksmetod för par. Metodnamnet liknar en högerpil och illustrerar en mappning från första till andra värdet.

\Subtask Fungerar det på par skapade med \code{->} att använda metoderna \code{_1} och \code{_2}?


\Subtask Deklarera en variabel \code{val huvudstad: Vector[(String, String)]} som innehåller mappningar mellan geografiska områden och deras huvudstäder enligt tabellen nedan.

\begin{table}[H]
  \renewcommand{\arraystretch}{1.2}
  \begin{tabular}{|l|l|}\hline
  Sverige & Stockholm \\\hline
  Danmark & Köpenhamn \\\hline
  Grönland & Nuuk \\\hline
  Skåne & Lund \\\hline
  \end{tabular}
\end{table}

\Subtask Skriv ett uttryck som plockar fram \code{"Lund"} ur \code{huvudstad}.

\SOLUTION


\TaskSolved \what

\SubtaskSolved Ja, fabriksmetoden returnerar ett helt vanligt par:
\begin{REPLnonum}
scala> val härBorJag = "Skåne" -> "Lund"
val härBorJag: (String, String) = (Skåne,Lund)

scala> härBorJag._1
val res0: String = Skåne

scala> härBorJag._2
val res1: String = Lund
\end{REPLnonum}


\SubtaskSolved

\begin{Code}
val huvudstad = Vector(
  "Sverige"  -> "Stockholm",
  "Danmark"  -> "Köpenhamn",
  "Grönland" -> "Nuuk",
  "Skåne"    -> "Lund"
)
\end{Code}

\SubtaskSolved
\begin{REPL}
scala> huvudstad(3)._2
val res2: String = Lund
\end{REPL}

\QUESTEND



\WHAT{Linjärsöka efter nyckel i sekvens av mappningar.}

\QUESTBEGIN

\Task \what~

\Subtask Implementera funktionen \code{lookupIndex} nedan med hjälp av samlingsmetoden \code{indexWhere} så att linjärsökning sker efter index för ett par i sekvensen där \code{key} finns på första platsen i paret.

\begin{Code}
def lookupIndex(xs: Vector[(String, String)])(key: String): Int = ???
\end{Code}

\Subtask Testa din funktion i REPL genom att slå upp index för Skånes huvudstad i sekvensen \code{huvudstad} från föregående uppgift.

\SOLUTION

\TaskSolved \what~

\SubtaskSolved
\begin{Code}
def lookupIndex(xs: Vector[(String, String)])(key: String): Int =
  xs.indexWhere(_._1 == key)
\end{Code}

\SubtaskSolved
\begin{REPL}
scala> val i = lookupIndex(huvudstad)("Skåne")
val i: Int = 3

scala> huvudstad(i)._2
val res2: String = Lund
\end{REPL}

\noindent Eller med funktioner som återanvändbara dellösningar:
\begin{REPL}
scala> val indexOf = lookupIndex(huvudstad) _

scala> def capital(key: String) = huvudstad(indexOf(key))._2

scala> capital("Skåne")
val res3: String = Lund

scala> capital("Sverige")
val res4: String = Stockholm
\end{REPL}

\QUESTEND



\WHAT{Nyckel-värde-tabell.}

\QUESTBEGIN

\Task \what~En nyckel-värde-tabell är en smart datastruktur som gör att du kan slå upp det värde som en nyckel mappar till \emph{utan} att linjärsökning behöver ske. Värdet plockas fram direkt på en konstant tid, d.v.s. tiden att slå upp ett värde beror \emph{inte} på antalet element i samlingen, utan sker med mycket liten fördröjning.

I Scala heter nyckelvärdetabeller \code{Map} med stort M och är praktiska att använda i många olika sammanhang. \code{Map} finns i både en oföränderlig och en förändringsbar variant. Det går med metoder på formen \code{toXXX} lätt att omvandla mellan en \code{Map} och en sekvens av par av typen \code{XXX[(Nyckeltyp, Värdetyp)]}.

\Subtask Deklarera mappen \code{telnr} nedan i REPL och använd \code{apply} för att ta reda på telefonnumret till Fröken Ur.

\Subtask Vad har \code{telnr} för typ?

\Subtask Vad har \code{telnr.toVector} för typ?

\begin{Code}
val telnr = Map(
  "Anna"     -> 46462229812L,
  "Björn"     -> 46462229009L,
  "Sandra"    -> 46462220368L,
  "Fröken Ur" -> 4690510L,
)
\end{Code}
En uppsättning \code{Map}-instanser, vid behov nästlade, kan med fördel användas för att bygga upp en i-minnet-databas där inbyggda samlingsmetoder, t.ex. \code{map}, \code{filter}, och \code{for}-\code{yield}-uttryck, ger flexibla och effektiva sökmöjligheter. På veckans laboration ska du göra detta.

Samlingen \code{Map} är en generalisering av en sekvens, där man kan ''indexera'', inte bara med ett heltal, utan med vilken typ av värde som helst, t.ex. en sträng. Datastrukturen \code{Map} kallas också \emph{associativ array}\footnote{\href{https://en.wikipedia.org/wiki/Associative_array}{https://en.wikipedia.org/wiki/Associative\_array}} och är implementerad som en s.k. \emph{hashtabell}\footnote{\href{https://en.wikipedia.org/wiki/Hash_table}{https://en.wikipedia.org/wiki/Hash\_table}}, men du får vänta till fördjupningskursen innan vi går igenom hur en sådan datastruktur implementeras.

\SOLUTION

\TaskSolved \what~

\begin{REPL}
scala> telnr("Fröken Ur")
val res0: Long = 464690510

scala> :type telnr
Map[String,Long]

scala> :type telnr.toVector
Vector[(String, Long)]
\end{REPL}

\QUESTEND



\WHAT{Använda nyckel-värdetabell.}

\QUESTBEGIN

\Task \what~

\Subtask Skapa nedan variabler i REPL.
\begin{Code}
val follow = for i <- 2 to 16 by 2 yield (i, i + 1)
val xs = follow.toMap
val ys = xs.toVector
\end{Code}
Hamnar mappningarna i \code{ys} i samma ordning som \code{follow}? Varför?

\Subtask Med \code{xs} och \code{ys} deklarerade i REPL enligt ovan, para ihop yttryck till vänster med rätt resultat till höger. Om du är osäker på de sammansatta uttrycken, prova enklare uttryck i REPL och undersök värde och typ hos delresultat.

\begin{ConceptConnections}
  \code|xs(2) + xs(4)                 | & 1 & & A & \code|1                     | \\ 
  \code|ys(2) + ys(4)                 | & 2 & & B & \code|-9                    | \\ 
  \code|ys(0)                         | & 3 & & C & \code|8                     | \\ 
  \code|xs(0)                         | & 4 & & D & \code|7                     | \\ 
  \code|(xs + (0 -> 1)).apply(0)      | & 5 & & E & \code|NoSuchElementException| \\ 
  \code|xs.keySet.apply(2)            | & 6 & & F & \code|(10, 11)              | \\ 
  \code|xs.mapValues(v => -v).apply(8)| & 7 & & G & \code|false                 | \\ 
  \code|xs isDefinedAt 0              | & 8 & & H & \verb|error: type mismatch  | \\ 
  \code|xs.getOrElse(0, 7)            | & 9 & & I & \code|(16, 17)              | \\ 
  \code|xs.maxBy(_._2)                | & 10 & & J & \code|true                  | \\ 
\end{ConceptConnections}

\SOLUTION

\TaskSolved \what


\SubtaskSolved Nej nyckel-värde-paren lagras i någon speciell ordning som bestäms av en intern, smart lagringsprincip enligt en s.k. hashfunktion\footnote{\url{https://sv.wikipedia.org/wiki/Hashfunktion}}, för att åstadkomma snabba uppslagningar av värden från nycklar och vilket normalt inte sammanfaller med ordningen i den sekvens som de skapades ur.

\SubtaskSolved

\begin{ConceptConnections}
    \code|xs(2) + xs(4)                 | & 1 & ~~\Large$\leadsto$~~ &  A & \code|8                     | \\ 
  \code|ys(0)                         | & 2 & ~~\Large$\leadsto$~~ &  C & \code|(10, 11)              | \\ 
  \code|xs(0)                         | & 3 & ~~\Large$\leadsto$~~ &  I & \code|NoSuchElementException| \\ 
  \code|(xs + (0 -> 1)).apply(0)      | & 4 & ~~\Large$\leadsto$~~ &  D & \code|1                     | \\ 
  \code|xs.keySet.apply(2)            | & 5 & ~~\Large$\leadsto$~~ &  G & \code|true                  | \\ 
  \code|xs isDefinedAt 0              | & 6 & ~~\Large$\leadsto$~~ &  H & \code|false                 | \\ 
  \code|xs.getOrElse(0, 7)            | & 7 & ~~\Large$\leadsto$~~ &  B & \code|7                     | \\ 
  \code|xs.maxBy(_._2)                | & 8 & ~~\Large$\leadsto$~~ &  E & \code|(16, 17)              | \\ 
  \code|xs.map(p => p._1 -> -p._2)(8) | & 9 & ~~\Large$\leadsto$~~ &  F & \code|-9                    | \\ 
\end{ConceptConnections}

%%% BELOW IS SOLVED IN SCALA 3 AND the err msg is better! :)
% \noindent \emph{Fördjupning}:  Felmeddelandet som rad 2 ovan orsakar är lurigt:

% \begin{REPL}
% scala> ys(2)
% val res22: (Int, Int) = (6,7)

% scala> ys(4)
% val res23: (Int, Int) = (12,13)

% scala> ys(2) + ys(4)
% <console>:13: error: type mismatch;
%  found   : (Int, Int)
%  required: String
%        ys(2) + ys(4)

% \end{REPL}
% Det går som förväntat inte att addera två tupler, men varför säger kompilatorn att en sträng krävs?!? Detta beror på att, i enlighet med hur det fungerar i Java, valde Scala-språkets konstruktörer att låta strängsammanfogning fungera med alla möjliga typer vilket gör att kompilatorn inte ger upp när metoden \code{+} inte finns för tupler, utan i stället gör ett misslyckat försök med strängsammanfogning.

% Det mest olyckliga med detta är inte att felmeddelanden ibland blir missvisande, utan att det i vissa situationer inte ens \emph{blir} något felmeddelande, trots att man av rent misstag råkat strängkonkatenera i stället för t.ex. lägga till ett element i en mängd eller en mappning i en tabell. Detta typosäkra beteendet av strängsammanfogning har kritiserats, men det är inte okontroversiellt att ändra detta nu när så många utvecklare skrivit så mycket Scala-kod som bygger på strängars förmåga att kunna lägga till vad som helst på slutet. Situationen i Scala är dock inte hopplös efter introduktionen av stränginterpolering i Scala 2.10, som möjliggör infogande av värden i strängar på ett typsäkert sätt.
\QUESTEND





\WHAT{Registrering i förändringsbar nyckel-värde-tabell.}

\QUESTBEGIN

\Task \what~I denna uppgift ska du implementera en hjälpklass för registrering i en frekvenstabell som du sedan ska använda på veckans laboration. Klassen ska heta  \code{FreqMapBuilder} som efter upprepade anrop av metoden \code{add(s: String): Unit} kan skapa frekvenstabeller av typen \code{Map[String, Int]}, där nyckel-värde-paren i tabellen anger antalet förekomster av en viss sträng. Du ska utgå från koden nedan.

Klassen använder en förändringsbar tabell internt. Efter att man har lagt till många strängar kan man med metoden \code{toMap} få en oföränderlig tabell för  uppslagning av frekvenser för specifika strängar. Läs i snabbreferensen om vilka extra metoder för uppdatering som erbjuds av \code{mutable.Map[K, V]}.

\begin{Code}
class FreqMapBuilder:
  private val register = collection.mutable.Map.empty[String, Int]
  def toMap: Map[String, Int] = register.toMap
  def add(s: String): Unit = ???

object FreqMapBuilder:
  def apply(xs: String*): FreqMapBuilder = ???
\end{Code}

\noindent Implementera och testa \code{FreqMapBuilder}. \emph{Tips:} Du kan t.ex. använda metoderna \code{+=} och \code{getOrElse}.

\SOLUTION

\TaskSolved \what~
\begin{Code}
class FreqMapBuilder:
  private val register = scala.collection.mutable.Map.empty[String,Int]
  def toMap: Map[String, Int] = register.toMap
  def add(s: String): Unit =
    register += (s -> (register.getOrElse(s, 0) + 1))

object FreqMapBuilder:
  def apply(xs: String*): FreqMapBuilder = 
    val result = new FreqMapBuilder
    xs.foreach(result.add)
    result
\end{Code}

\QUESTEND



\WHAT{Metoden \code{sliding}.}

\QUESTBEGIN

\Task  \what~  I veckans laboration kommer du att ha nytta av metoden \code{sliding}, som ger en iterator för speciella delsekvenser av en sekvens, vilka kan liknas vid ''utsikten'' i ett ''glidande fönster''.

\Subtask Kör nedan i REPL och beskriv vad som händer.

\begin{REPL}
scala> val xs = Vector("fem", "gurkor", "är", "fler", "än", "fyra", "tomater")
scala> xs.sliding(2).toVector
scala> xs.sliding(3).toVector
scala> xs.sliding(10).toVector
\end{REPL}

\Subtask Använd \code{xs.sliding(2)} och omvandla varje element i resultatet till ett par. Gör sedan om sekvensen av par till en nyckel-värde-tabell. Vad kan tabellen användas till?

\SOLUTION

\TaskSolved \what

\SubtaskSolved
\begin{REPL}
scala> val xs = Vector("fem", "gurkor", "är", "fler", "än", "fyra", "tomater")
val xs: Vector[String] =
  Vector(fem, gurkor, är, fler, än, fyra, tomater)

scala> xs.sliding(2).toVector
val res9: Vector[Vector[String]] =
  Vector(Vector(fem, gurkor), Vector(gurkor, är), Vector(är, fler), Vector(fler, än), Vector(än, fyra), Vector(fyra, tomater))

scala> xs.sliding(3).toVector
val res10: Vector[Vector[String]] =
  Vector(Vector(fem, gurkor, är), Vector(gurkor, är, fler), Vector(är, fler, än), Vector(fler, än, fyra), Vector(än, fyra, tomater))

scala> xs.sliding(10).toVector
val res11: Vector[Vector[String]] =
  Vector(Vector(fem, gurkor, är, fler, än, fyra, tomater))

\end{REPL}
\code{xs.sliding(n).toVector} skapar en sekvens som innehåller sekvenser av längden \code{n} som bildas genom att ta varje element och dess \code{n - 1} efterföljande element.

\SubtaskSolved
\begin{REPL}
scala> xs.sliding(2).map(ys => ys(0) -> ys(1)).toMap
val res0: Map[String,String] =
  Map(är -> fler,
      än -> fyra,
      fyra -> tomater,
      gurkor -> är,
      fem -> gurkor,
      fler -> än
  )
\end{REPL}
Man kan använda tabellen till att slå upp vilket som är efterföljande ord. Det fungerar eftersom alla ord är unika. Om det funnits flera likadana ord med olika efterföljande ord så hade vi behövt skapa en tabell med nycklar som mappar till en samling som registrerar efterföljande ord. Detta ska vi göra på veckans laboration.

\QUESTEND




\WHAT{Läsa text från fil och webbservrar.}

\QUESTBEGIN

\Task \what~På laborationen ska du bygga upp tabeller från data i textformat. Då har du nytta av att kunna läsa text från filer och från webben. Testa detta i REPL:
\begin{REPL}
scala> val url = "https://fileadmin.cs.lth.se/pgk/europa.txt"
scala> val xs = io.Source.fromURL(url, "UTF-8").getLines.toVector
scala> val data = xs.map(_.split(';').toVector)
scala> data.head
scala> data.foreach(println)
\end{REPL}

\Subtask Skapa dessa tabeller ur sekvensen \code{data}:
\begin{Code}
val populationOf: Map[String, Int]    = ???  // länders invånarantal
val sizeOf:       Map[String, Int]    = ???  // länders yta i km^2
val capitalOf:    Map[String, String] = ???  // länders huvudstäder
\end{Code}
Testa tabellerna i REPL.

\Subtask Spara ner data i en textfil \code{europa.txt}. Läsa in data från filen med metoden \code{Source.fromFile(filnamn, teckenkodning)} på liknande sätt som med  \code{fromURL} ovan. Om du kör i en Linux-terminal kan du enkelt ladda ner en fil så här:
\begin{REPLnonum}
> wget https://fileadmin.cs.lth.se/pgk/europa.txt
\end{REPLnonum}
Skriv ut alla raderna i \code{europa.txt} med hjälp av \code{Source.fromFile} i REPL.

\SOLUTION

\TaskSolved \what~

\SubtaskSolved
\begin{CodeSmall}
val populationOf = data.tail.map(v => v(0) -> v(1).toInt).toMap
val sizeOf       = data.tail.map(v => v(0) -> v(2).toInt).toMap
val capitalOf    = data.tail.map(v => v(0) -> v(3)).toMap
\end{CodeSmall}

\begin{REPL}
scala> capitalOf("Sverige")
res2: String = Stockholm

scala> populationOf("Sverige")
res3: Int = 9223766

scala> sizeOf("Sverige")
res4: Int = 449964
\end{REPL}

\begin{REPL}
scala> val filename = "europa.txt"
scala> val xs = io.Source.fromFile(filename, "UTF-8").getLines.toVector
scala> val data = xs.map(_.split(';').toVector)
scala> data.map(_.map(_.take(15).padTo(15,' ')).mkString(" ")).foreach(println)
\end{REPL}
\QUESTEND





\ExtraTasks %%%%%%%%%%%%%%%%%%%%%%%%%%%%%%%%%%%%%%%%%%%%%%%%%%%%%%%%%%%%%%%%%%%%

\WHAT{Skapa ett textspel med hjälp av tabeller.}

\QUESTBEGIN

\Task \what~Gör ett enkelt spel för att träna på olika fakta om Europas länder och huvudstäder genom att läsa data från URL:en:\\ \url{https://fileadmin.cs.lth.se/pgk/europa.txt}
\\Där finns text kodad i UTF-8 med följande innehåll (endast de första raderna visas):
\begin{Code}
Land;Invånarantal;Storlek(km^2);Huvudstad
Albanien;3581655;28748;Tirana
Andorra;71201;468;Andorra la Vella
Belgien;10584534;30528;Bryssel
Bosnien-Hercegovina;4590310;51129;Sarajevo
Bulgarien;7385367;110910;Sofia
Cypern;854000;9250;Nicosia
Danmark;5475791;43094;Köpenhamn
Estland;1324333;45226;Tallinn
Finland;5315280;338145;Helsingfors
Frankrike;61538322;551695;Paris
Färöarna;48344;139574;Torshamn
Grekland;10964021;131940;Aten
// ... etcetera för alla Europas länder.
\end{Code}
Låt till exempel användaren svara på slumpvisa frågor av typen:
\begin{itemize}[noitemsep]
  \item Har Andorra fler invånare än Cypern?
  \item Vad heter huvudstaden i Bulgarien?
  \item Har Danmark större yta än Finland?
\end{itemize}
Använd oföränderliga tabeller med lämpliga nycklar och värden. Du kan använda en mängd med länder/huvudstäder som användaren hittills svarat rätt på för att kunna förhindra att dessa återkommer igen.
\SOLUTION

\TaskSolved --

\QUESTEND



\AdvancedTasks %%%%%%%%%%%%%%%%%%%%%%%%%%%%%%%%%%%%%%%%%%%%%%%%%%%%%%%%%%%%%%%%%


\WHAT{Registrering med \code{groupBy}.}

\QUESTBEGIN

\Task \what~Vi ska nu utnyttja ett riktigt listigt trick för att via en enda kodrad implementera registrering med hjälp av samlingsmetoderna \code{groupBy} och \code{map}.

\Subtask Läs om metoden \code{groupBy} i snabbreferensen. Du hittar den under rubriken \emph{''Methods in trait \code{Iterable[A]}''} eftersom \code{groupBy} fungerar på alla samlingar. Testa \code{groupBy} enligt nedan och beskriv vad som händer.

\begin{REPL}
scala> val xs = Vector(1, 1, 2, 2, 4, 4, 4).groupBy(x => x > 2)
scala> val ys = Vector(1, 1, 2, 2, 4, 4, 4).groupBy(x => x)
\end{REPL}

\Subtask Skapa en funktion \code{freq} med nedan funktionshuvud som returnerar en tabell med antalet förekomster av olika heltal i \code{xs}. Testa \code{freq} på en sekvens av 1000 slumpvisa tärningskast och förklara hur funktionen \code{freq} fungerar. \emph{Tips:} Gör först \code{groupBy(???)} och sedan \code{map(???)}.

\begin{Code}
def freq(xs: Vector[Int]): Map[Int, Int] = ???

def kasta(n: Int): Vector[Int] =
  Vector.fill(n)(scala.util.Random.nextInt(6) + 1)
\end{Code}

\SOLUTION

\TaskSolved \what~

\SubtaskSolved Metoden \code{groupBy} skapar en nyckel-värde-tabell där värdena i tabellen är en sekvens med elementen grupperade på ett speciellt sett.
Mer precist:

Resultatet av \code{xs.groupBy(f: K => V)} för en sekvens \code{xs} av typen \code{Vector[K]} blir en tabell av typen \code{Map[V,Vector[K]]} där varje element \code{e} i \code{xs} är grupperade i samma tabellvärde om de lika är enligt \code{f(e)}. Varje grupp får tabellnyckeln \code{f(e)}.

\emph{Listigt trick:} Om man låter funktionen \code{f} vara enhetsfunktionen som avbildar varje element på sig själv, alltså \code{x => x}, så grupperas värdena i samma sekvens om de är lika.

\begin{REPL}
scala> val xs = Vector(1, 1, 2, 2, 4, 4, 4).groupBy(x => x > 2)
val xs: Map[Boolean,Vector[Int]] =
  Map(false -> Vector(1, 1, 2, 2), true -> Vector(4, 4, 4))

scala> val ys = Vector(1, 1, 2, 2, 4, 4, 4).groupBy(x => x)
val ys: Map[Int,Vector[Int]] =
  Map(2 -> Vector(2, 2), 4 -> Vector(4, 4, 4), 1 -> Vector(1, 1))
\end{REPL}


\SubtaskSolved

\begin{Code}
def freq(xs: Vector[Int]): Map[Int, Int] =
  xs.groupBy(x => x).map(p => p._1 -> p._2.size)
\end{Code}
Förklaring: metoden \code{groupBy} skapar en tabell med par \code{k, v} där \code{v} är en sekvens med så många \code{k} som antalet gånger \code{k} förekommer i \code{xs}. Genom att omvandla alla värden \code{p._2} till storleken \code{p._2.size} får vi en frekvenstabell.

\begin{REPL}
scala> freq(kasta(1000))
val res0: Map[Int,Int] = 
  Map(5 -> 163, 1 -> 174, 6 -> 161, 2 -> 169, 3 -> 167, 4 -> 166)

scala> freq(kasta(1000)).toVector.sortBy(_._1).foreach(println)
(1,183)
(2,167)
(3,169)
(4,179)
(5,154)
(6,148)
\end{REPL}

\QUESTEND





\WHAT{Skriva till fil.}

\QUESTBEGIN

\Task \what~Som hjälp när du skapar egna intressanta applikationer eller bygger vidare på kursens laborationer och övningar med frivilliga extrauppgifter, kan du använda funktionerna i singelobjektet \code{IO} nedan, som finns i kursens scala-bibliotek \href{http://cs.lth.se/pgk/api}{introprog}.\footnote{Källkoden finns här och även på sidan \pageref{disk-access-code}:\\ \href{https://github.com/lunduniversity/introprog/blob/master/compendium/workspace/introprog/src/main/scala/introprog/IO.scala}{https://github.com/lunduniversity/introprog-scalalib/blob/master/src/main/scala/introprog/IO.scala}}

IO-modulen använder \code{scala.io.Source} för att serialisera och de-serialisera strängar till och från vanliga textfiler. IO-modulen använder även paketet \code{java.io} för att erbjuda funktioner som gör det enkelt att serialisera/de-serialisera godtyckliga objekt skapade med hjälp av serialserbara klasser till/från binärfiler. Case-klasser i Scala blir automatiskt serialiserbara.

I implementationen av \code{IO} används \code{try ... finally} för att säkerställa att filer inte lämnas öppnade även om något går fel under den läs/skriv-process som sköts av det underliggande operativsystemet.

\Subtask
Kompilera och resta nedan med \code{introprog} på classpath, t.ex. med hjälp av \code{sbt}.
\begin{Code}
import introprog.IO

case class Player(name: String)

@main def run(): Unit = 
  println("Test of output/input objects to/from disk:")
  val highscores = Map(Player("Sandra") -> 42, Player("Björn") -> 5)
  IO.saveObject(highscores,"highscores.ser")
  val highscores2 = IO.loadObject[Map[Player, Int]]("highscores.ser")
  val isSameContents = highscores2 == highscores
  val testResult = if (isSameContents) "SUCCESS :)" else "FAILURE :("
  println(testResult)
\end{Code}

\Subtask
Använd \code{IO}-modulen för att spara användarens poängresultat i ditt spel om Europas länder och städer, i extrauppgiften ovan. Implementationen av \code{introprog.IO} finns här: \url{https://github.com/lunduniversity/introprog-scalalib/blob/master/src/main/scala/introprog/IO.scala} 

% \begin{figure}
% %  \scalainputlisting[basicstyle=\ttfamily\fontsize{9.2}{11}\selectfont]{examples/IO.scala}
%   \scalainputlisting[basicstyle=\ttfamily\fontsize{9.2}{11}\selectfont]{../workspace/introprog/src/main/scala/introprog/IO.scala}
%   \label{disk-access-code}
% \end{figure}
\SOLUTION

\TaskSolved --

\QUESTEND



%
%
% \subsection{\TODO Värdera nedan gamla uppgifter}
%
%
%
% \WHAT{Objekt med attribut (fält).}
%
% \QUESTBEGIN
%
% \Task  \what~  Ett objekt kan samla data som hör ihop och på så sätt skapa en datastruktur. Data i ett objekt kallas \emph{attribut} eller \emph{fält}, \Eng{field}. Objekt som samlar enbart data kallas även \emph{post} \Eng{record}.
% \begin{REPLnonum}
% scala> object mittKonto { var saldo = 0; val nummer = 12345L }
% \end{REPLnonum}
% \Subtask Skriv en sats som sätter in ett slumpmässigt belopp mellan 0 och en miljon på \code{mittKonto} ovan med hjälp av punktnotation och tilldelning.
%
% \Subtask Vad händer om du försöker ändra attributet \code{nummer}?
%
% \SOLUTION
%
%
% \TaskSolved \what
%
%
% \SubtaskSolved   \code{mittKonto.saldo = (math.random() * 1000000).toInt}
%
% \SubtaskSolved   Går ej eftersom val är oföränderlig, man får alltså ett Error.
%
%
% \QUESTEND
%
%
%
%
% %%<AUTOEXTRACTED by mergesolu>%%      %Uppgift 2
%
%
%
%
% \WHAT{Klass med attribut.}
%
% \QUESTBEGIN
%
% \Task  \what~  Om du vill ha många objekt av samma typ, kan du använda en \textbf{klass}. På så sätt kan man skapa många datastrukturer av samma typ men med olika innehåll. Man skapar nya objekt med nyckelordet \code{new} följt av klassens namn. Klassen utgör en ''mall'' för objektet som skapas. Ett objekt som skapas med \code{new Klassnamn} kallas även en \textbf{instans} av klassen \code{Klassnamn}. Nedan skapas en datastruktur \code{Konto} som samlar data om ett bankonto. Instanser av typen \code{Konto} håller reda på hur mycket pengar det finns på kontot och vilket kontonumret är. Datavärden som sparas i varje objektinstans, så som \code{saldo} och \code{nummer}, kallas \textbf{attribut} \Eng{attribute} eller \textbf{fält} \Eng{field}.
%
% \begin{REPL}
% scala> class Konto {
%          var saldo = 0
%          var nummer = 0L
%        }
% scala> val k1 = new Konto
% scala> val k2 = new Konto
% scala> k1.saldo = 1000
% scala> k1.nummer = 12345L
% scala> k2.saldo = 2000
% scala> k2.nummer = 67890L
% scala> println("Konto: " + k1.nummer + " Saldo:" + k1.saldo)
% scala> println("Konto: " + k2.nummer + " Saldo:" + k2.saldo)
% \end{REPL}
%
% \Subtask\Pen Rita hur minnessituationen ser ut efter att ovan rader har exekverats.
%
% \Subtask\Pen Vad hade det fått för konsekvenser om attributet \code{nummer} vore oföränderligt i klassen ovan? (Jämför med objektet \code{mittKonto}.)
%
%
% \SOLUTION
%
%
% \TaskSolved \what
%
%
% \SubtaskSolved   \includegraphics[scale=0.5]{../img/w04-solutions/uppgift-3a}
%
% \SubtaskSolved
% Tilldelningen på rad 8 \code{k1.nummer = 12345L} ger felmeddelande eftersom variablen är oföränderlig.
%
%
% \QUESTEND
%
%
%
%
% %%<AUTOEXTRACTED by mergesolu>%%      %Uppgift 3
%
%
%
%
% \WHAT{Klass med attribut som parametrar.}
%
% \QUESTBEGIN
%
% \Task  \what~  Om man vill ge attributen initialvärden när objektet skapas med \code{new}, kan man placera attributen i en parameterlista till klassen. Koden som körs när objektet skapas och attributen tilldelas sina initialvärden, kallas \textbf{konstruktor} \Eng{constructor}.
%
% \begin{REPL}
% scala> class Konto(var saldo: Int, val nummer: Long)
% scala> val k = new Konto(0, 12345L)
% scala> println("Konto: " + k.nummer + " Saldo:" + k.saldo)
% scala> println(k)
% scala> k.toString
% \end{REPL}
%
% \Subtask Den två sista raderna ovan skriver ut den identifierare som JVM använder för att hålla reda på objektet i sina interna datastrukturer. Vad skrivs ut?
%
% \Subtask Skapa ännu en instans av klassen Konto  med samma saldo och nummer som \code{k} ovan och spara den i \code{val k2} och undersök dess objektidentifierare. Får objekten \code{k} och \code{k2} olika objektidentifierare?
%
% \Subtask Sätt in olika belopp på respektive konto.
%
% \Subtask Vad händer om du försöker ändra attributet \code{nummer}?
%
% \Subtask\Pen Ibland räcker det fint med en tupel, men ofta vill man ha en klass istället. Beskriv några fördelar med en Konto-klassen ovan jämfört med en tupel av typen \code{(Int, Long)}.
%
% \begin{REPLnonum}
% scala> var k3 = (0, 12345L)
% scala> k3 = (k3._1 + 100, k3._2)
% \end{REPLnonum}
%
% \SOLUTION
%
%
% \TaskSolved \what
%
%
% \SubtaskSolved   \code{String = Konto@cd576}, där \code{Konto@cd576} är ett unikt namn som identifierar instansen.
%
% \SubtaskSolved   Ja.
%
% \SubtaskSolved
% \begin{REPLnonum}
% scala> k.saldo = 42
% scala> k2.saldo = 67
% \end{REPLnonum}
%
% \SubtaskSolved   Eftersom variablen är oföränderlig ges ett felmeddelande.
%
% \SubtaskSolved   En fördel med klass är att man kan specificera att variablen ska kunna vara föränderlig. En till är att man kan inkludera metoder i klassen som man vill kunna använda på värdena.
%
%
% \QUESTEND
%
%
%
%
% %%<AUTOEXTRACTED by mergesolu>%%      %Uppgift 4
%
%
%
%
% \WHAT{Publikt eller privat attribut?}
%
% \QUESTBEGIN
%
% \Task  \what~  Man kan förhindra att ett attribut syns utanför klassen med hjälp av nyckelordet \code{private}.
%
% \begin{REPL}
% scala> class Konto1(val nummer: Long){ var saldo = 0 }
% scala> val k1 = new Konto1(12345678901L)
% scala> k1.nummer
% scala> k1.saldo += 1000
% scala> class Konto2(val nummer: Long){ private var saldo = 0 }
% scala> val k2 = new Konto2(12345678901L)
% scala> k2.nummer
% scala> k2.saldo += 1000
% \end{REPL}
%
% \Subtask Vad händer ovan?
%
% \Subtask Gör en ny version av klassen \code{Konto} enligt nedan:
%
% \begin{Code}
% class Konto(val nummer: Long){
%   private var saldo = 0
%   def in(belopp: Int): Unit = {saldo += belopp}
%   def ut(belopp: Int): Unit = {saldo -= belopp}
%   def show: Unit =
%     println("Konto Nr: " + nummer + " saldo: " + saldo)
% }
%
% object Main {
%   def main(args: Array[String]): Unit = {
%     val k = new Konto(1234L)
%     k.show
%     k.in(1000)
%     println("Uttag: " + k.ut(500))
%     println("Uttag: " + k.ut(1000))
%     k.show
%   }
% }
% \end{Code}
%
% \Subtask Spara koden i en fil, kompilera med \code{scalac} och kör. Testa även vad som händer om du försöker komma åt attributet \code{saldo} i main-metoden med t.ex. \code{println(k.saldo)} eller \code{k.saldo += 1000}.
%
% \Subtask Vi ska nu förhindra överuttag. Ändra i metoden \code{ut} så att den får signaturen \code{ut(belopp: Int): (Int, Int) = ???} och implementera \code{ut} så att den returnerar både beloppet man verkligen kan ta ut och kvarvarande saldo. Om man försöker ta ut mer än det finns på kontot så ska saldot bli 0 och man får bara ut det som finns kvar. Spara, kompilera, kör.
%
% \Subtask Förbättra metoderna \code{in} och \code{ut} så att man inte kan sätta in eller ta ut negativa belopp.
%
% \Subtask Vad är fördelen med att göra föränderliga attribut privata och bara påverka deras värden indirekt via metoder?
%
% \SOLUTION
%
%
% \TaskSolved \what
%
%
% \SubtaskSolved
% Det går bra att ändra på variablen saldo i instansen av Konto1 men inte av Konto2 där man får ett error på raden ''k2.saldo += 1000''
%
% \SubtaskSolved  -
%
% \SubtaskSolved
% ''println(k.saldo)'' och ''k.saldo += 1000'' ger båda error, pga privat attribut.
%
% \SubtaskSolved
% \begin{Code}
% def ut(belopp: Int): (Int, Int) = {
% 	if(saldo >= belopp) {
% 		saldo -= belopp
% 		(belopp, saldo)
% 	} else {
% 		val temp = saldo
% 		saldo = 0
% 		(temp, 0)
% 	}
% }
% \end{Code}
%
% \SubtaskSolved
% Lägg till en if-sats i båda funktionerna som omsluter den gamla koden.
% \begin{Code}
% def ut(belopp: Int): (Int, Int) = {
%   if(belopp >= 0) {
%     if(saldo >= belopp) {
%       saldo -= belopp
%       (belopp, saldo)
%     } else {
%       val temp = saldo
%       saldo = 0
%       (temp, 0)
%     }
%   }
% }
%
% def in(belopp: Int): Unit = {
%   if(belopp >= 0) {
%     saldo += belopp
%   }
% }
% \end{Code}
%
% \SubtaskSolved
% Genom att göra attributet privat och gör egna metoder kan man se till att attriuten endast ändras på säkra sätt. Så inte fel uppstår.
%
%
% \QUESTEND
%
%
%
%
% %%<AUTOEXTRACTED by mergesolu>%%      %Uppgift 5
%
%
%
%
% \WHAT{Vilken typ har ett objekt?}
%
% \QUESTBEGIN
%
% \Task  \what~  Objektets typ bestäms av klassen. Vid tilldelning måste typerna passa ihop.
%
% \Subtask Vilka rader nedan ger felmeddelande? Hur lyder felmeddelandet?
% \begin{REPL}
% scala> class Punkt(val x: Double, val y: Double)
% scala> val pt: Punkt = new Punkt(10.0, 10.0)
% scala> val i: Int = pt.x
% scala> val (x: Double, y: Double) = (pt.x, pt.y)
% scala> val p: Double = new Punkt(5.0, 5.0)
% scala> val p = new Punkt(5.0, 5.0): Double
% scala> val p = new Punkt(5.0, 5.0): Punkt
% scala> pt: Punkt
% \end{REPL}
%
%
% \Subtask Man kan undersöka om ett objekt är av en viss typ med metoden \\ \code{isInstanceOf[Typnamn]}. Vad ger nedan anrop av metoden \code{isInstanceOf} för värde?
% \begin{REPL}
% scala> class Punkt(val x: Double, val y: Double)
% scala> val pt: Punkt = new Punkt(1.0, 2.0)
% scala> pt.isInstanceOf[Punkt]
% scala> pt.isInstanceOf[Double]
% scala> pt.x.isInstanceOf[Punkt]
% scala> pt.x.isInstanceOf[Double]
% scala> pt.x.isInstanceOf[Int]
% \end{REPL}
%
% \SOLUTION
%
%
% \TaskSolved \what
%
%
% \SubtaskSolved
% ''val i: Int = pt.x'' error: type mismatch;
% Eftersom typen Int ej är kompatibel med ett värde av typen Double.
%
% ''val p: Double = new Punkt(5.0, 5.0)'' error: type mismatch;
% Eftersom typen Double ej är kompatibel med ett värde av typen Punkt.
%
% ''val p = new Punkt(5.0, 5.0): Double'' error: type mismatch;
% Eftersom typen Double ej är kompatibel med ett värde av typen Punkt.
%
% \SubtaskSolved
% Rad 3 till 7 i respektive ordning: true, false, false, true och false.
%
%
% \QUESTEND
%
%
%
%
% %%<AUTOEXTRACTED by mergesolu>%%      %Uppgift 6
%
%
%
%
% \WHAT{Topptypen \code{Any}.}
%
% \QUESTBEGIN
%
% \Task  \what~ Alla klasser är också av typen \code{Any}. Alla klasser får därmed med sig några gemensamma metoder som finns i den fördefinierade klassen \code{Any}, däribland metoderna  \code{isInstanceOf} och \code{toString}.  Vad blir resultatet av respektive rad nedan? Vilken rad ger ett felmeddelande?
%
%
% \begin{REPL}
% scala> class Punkt(val x: Double, val y: Double)
% scala> val pt: Punkt = new Punkt(1.0, 2.0)
% scala> pt.isInstanceOf[Punkt]
% scala> pt.isInstanceOf[Any]
% scala> pt.x.toString
% scala> println(pt.x)
% scala> val a: Any = pt
% scala> println(a.x)
% scala> a.toString
% scala> pt.y.toString
% scala> a.y.toString
% \end{REPL}
%
% \SOLUTION
%
%
% \TaskSolved \what
%
% \begin{enumerate}
% \item Definierar klassen Punkt.
% \item En variabel pt: Punkt skapas.
% \item true
% \item true
% \item String = 1.0
% \item skriver ut: 1.0
% \item En variabel med namnet a skapas med typen Any.
% \item error: value x is not a member of Any
% \item a ges nu typen String
% \item String = 2.0
% \item error: value y is not a member of Any
% \end{enumerate}
%
%
% \QUESTEND
%
%
%
%
% %%<AUTOEXTRACTED by mergesolu>%%      %Uppgift 7
%
%
%
%
% \WHAT{Byta ut metoden \code{toString}}.
%
% \QUESTBEGIN
%
% \Task  \what~ I klassen \code{Any} finns metoden \code{toString} som skapar en strängrepresentation av objektet. Du kan byta ut metoden \code{toString} i klassen \code{Any} mot din egen implementation. Man använder nyckelordet \code{override} när man vill byta ut en metodimplementation.
%
% \begin{REPL}
% scala> class Punkt(val x: Double, val y: Double) {
%          override def toString: String = "[x=" + x + ",y=" + y + "]"
%        }
% scala> val pt = new Punkt(1.0, 42.0)
% scala> pt.toString
% scala> println(pt)
% \end{REPL}
%
% \Subtask Vad händer egentligen på sista raden ovan?
%
% \Subtask Omdefiniera toString så att den ger en sträng på formen \code{Punkt(1.0, 42.0)}.
%
% \Subtask Vad händer om du utelämnar nyckelordet \code{override} vid omdefiniering?
%
% \SOLUTION
%
%
% \TaskSolved \what
%
%
% \SubtaskSolved
% ''println(pt)'' kallar på pt.toString, och eftersom metoden är överskriven kallas den nya version.
%
% \SubtaskSolved   \code{override def toString: String = ''Punkt('' + x + '', '' + y + '').''}
%
% \SubtaskSolved
% error: overriding method toString in class Object of type ()String;
%
%
% \QUESTEND
%
%
%
%
% %%<AUTOEXTRACTED by mergesolu>%%      %Uppgift 8
%
%
%
%
% \WHAT{Objektfabrik med \code{apply}-metod.}
%
% \QUESTBEGIN
%
% \Task  \what~  Man kan ordna så att man slipper skriva \code{new} med ett s.k. \emph{fabriksobjekt} \Eng{factory object}.
% \begin{Code}
% class Pt(val x: Double, y: Double) {
%   override def toString: String = "Pt(x=" + x + ",y=" + y + ")"
% }
% object Pt {
%   def apply(x: Double, y: Double): Pt = new Pt(x, y)
% }
% \end{Code}
%
% \Subtask Skriv satser som använder metoden \code{apply} i fabriksobjektet \code{object Pt} för att skapa flera olika punkter.
%
% \Subtask Ge applymetoden default-argument 0.0 för både x och y så att \code{Pt()} skapar en punkt i origo.
%
% \Subtask Skapa en klass \code{Rational} som representerar rationellt tal som en kvot mellan två heltal. Ge klassen två oföränderliga, publika klassparameterattribut med namnen \code{nom} för täljaren och \code{denom} för nämnaren.
%
% \Subtask Skapa ett fabriksobjekt med en \code{apply}-metod som tar två heltalsparametrar och skapar en instans av klassen \code{Rational}.
%
% \Subtask Skapa olika instanser av din klass \code{Rational} ovan med hjälp av fabriksobjektet.
%
%
% \SOLUTION
%
%
% \TaskSolved \what
%
%
% \SubtaskSolved
% \begin{REPL}
% scala> val pt = Pt(1.0, 2.0)
% pt: Pt = Pt(x=1.0,y=2.0)
%
% scala> Pt(4.0, 2.0)
% res0: Pt = Pt(x=4.0,y=2.0)
%
% scala> Pt(6.0, 3.0)
% res1: Pt = Pt(x=6.0,y=3.0)
%
% scala> Pt(666.0, 1337.0)
% res2: Pt = Pt(x=666.0,y=1337.0)
% \end{REPL}
%
% \SubtaskSolved  \code{def apply(): Pt = new Pt(0, 0)}
%
% \SubtaskSolved  \code{class Rational(val nom: Int, val denom: Int)}
%
% \SubtaskSolved
% \begin{REPLnonum}
% object Rational {
% def apply(nom: Int, denom: Int): Rational = new Rational(nom, denom)
% }
% \end{REPLnonum}
%
% \SubtaskSolved
% \begin{REPL}
% scala> Rational(2, 5)
% scala> Rational(2, 7)
% scala> Rational(7, 4)
% scala> Rational(666, 1337)
% \end{REPL}
%
%
% \QUESTEND
%
%
%
%
% %%<AUTOEXTRACTED by mergesolu>%%      %Uppgift 9
%
%
%
%
% \WHAT{Skapa en case-klass.}
%
% \QUESTBEGIN
%
% \Task  \what~  Med en case-klass får man \code{toString} och fabriksobjekt på köpet. Man behöver inte skriva \code{val} framför klassparametrar i case-klasser; klassparametrar blir publika, oföränderliga attribut automatiskt när man deklarerar en case-klass.
%
% \begin{REPL}
% scala> case class Pt(x: Double, y: Double)
% scala> val p = Pt(1.0, 42.0)
% scala> p.toString
% scala> println(p)
% scala> println(Pt(5,6))
% \end{REPL}
%
% \Subtask Implementera din klass \code{Rational} från föregående uppgift, men nu som en case-klass.
%
% \SOLUTION
%
%
% \TaskSolved \what
%
% \SubtaskSolved  \code{case class Rational(nom: Int, denom: Int)}
%
%
% \QUESTEND
%
%
%
%
% %%<AUTOEXTRACTED by mergesolu>%%      %Uppgift 10
%
%
%
%
% \WHAT{Metoder på datastrukturer.}
%
% \QUESTBEGIN
%
% \Task \label{task:point} \what~   En datastruktur blir mer användbar om det finns metoder som kan användas på datastrukturen. Metoder i Scala kan även ha (vissa) specialtecken som namn, t.ex. \code{+} enligt nedan.
% \begin{REPL}
% scala> case class Point(x: Double, y: Double) {
%          def distToOrigin: Double = math.hypot(x, y)
%          def add(p: Point): Point = Point(x + p.x, y + p.y)
%          def +(p: Point): Point = add(p)
%        }
% \end{REPL}
%
% \Subtask Använd metoden \code{distToOrigin} för att ta reda på vad punkten med koordinaterna (3, 4) har för avstånd till origo?
%
% \Subtask Skriv satser som skapar två punkter (3,4) och (5, 6) och låt variablerna p1 och p2 referera till respektive punkt. Låt variabeln p3 bli summan av p1 och p2 med hjälp av metoden \code{add}. Vad får uttrycken \code{p3.x} resp. \code{p3.y} för värden?
%
%
%
% \SOLUTION
%
%
% \TaskSolved \what
%
%
% \SubtaskSolved
% \begin{REPLnonum}
% scala> Point(3, 4).distToOrigin
% res0: Double = 5.0
% \end{REPLnonum}
%
% \SubtaskSolved
% p3.x = 8
% p3.y = 10
%
%
% \QUESTEND
%
%
%
%
% %%<AUTOEXTRACTED by mergesolu>%%      %Uppgift 11
%
%
%
%
% \WHAT{Operatornotation.}
%
% \QUESTBEGIN
%
% \Task  \what~  Vid punktnotation på formen: \\ \code{objekt.metod(argument)} \\ kan man skippa punkten och parenteserna och skriva:\\ \code{objekt metod argument}  \\
% Detta förenklade skrivsätt kallas \textbf{operatornotation}.
%
% \Subtask Använd klassen \code{Point} från uppgift \ref{task:point} och prova nedan satser. Vilka rader använder operatortnotation och vilka rader använder punktnotation? Vilka rader ger felmeddelande?
% \begin{REPL}
% scala> val p1 = Point(3,4)
% scala> val p2 = Point(3,4)
% scala> p1.add(p2)
% scala> p1 add p2
% scala> p1.+(p2)
% scala> p1 + p2
% scala> 42 + 1
% scala> 42.+(1)
% scala> 42.+ 1
% scala> 42 +(1)
% scala> 1.to(42)
% scala> 1 to 42
% scala> 1.to(42)
% \end{REPL}
%
% \Subtask Implementera metoderna \code{sub} och \code{-} i klassen \code{Point} och skriv uttryck som kombinerar add och sub, samt + och - i både punktnotation och operatornotation.
%
% \Subtask Operatornotation fungerar även med flera argument. Man använder då parenteser om listan med argumenten:
% \code{ objekt metod (arg1, arg2)}  \\
% Definiera en metod \\
% \code{def scale(a: Double, b: Double) = Point(x * a, y * b)} \\
% i klassen \code{Point} och skriv satser som använder metoden med punktnotation och operatornotation.
%
%
%
%
%
% \SOLUTION
%
%
% \TaskSolved \what
%
%
% \SubtaskSolved
% \\Operatornotation:	4, 6, 10, 12
% \\Punktnotation:		3, 5, 8, 9, 11, 13
% \\Felmeddelande:		9
%
% \SubtaskSolved
% \begin{Code}
% case class Point(x: Double, y: Double) {
%   def distToOrigin: Double = math.hypot(x, y)
%   def add(p: Point): Point = Point(x + p.x, y + p.y)
%   def +(p: Point): Point = add(p)
%   def sub(p: Point): Point = Point(x - p.x, y - p.y)
%   def -(p: Point): Point = sub(p)
% }
% \end{Code}
% \begin{REPL}
% scala> val p1: Point = Point(1, 9)
% scala> val p2: Point = Point(9, 6)
% scala> p1.sub(p2)
% scala> p1.-(p2)
% scala> p2 sub p1
% scala> p2 - p2
% scala> p1.add(p2.sub(p1))
% scala> p1 + (p2 - p1)
% \end{REPL}
%
% \SubtaskSolved
% \begin{Code}
% case class Point(x: Double, y: Double) {
%   def distToOrigin: Double = math.hypot(x, y)
%   def add(p: Point): Point = Point(x + p.x, y + p.y)
%   def +(p: Point): Point = add(p)
%   def sub(p: Point): Point = Point(x - p.x, y - p.y)
%   def -(p: Point): Point = sub(p)
%   def scale(a: Double, b: Double) = Point(x * a, y * b)
% }
% \end{Code}
% \begin{REPL}
% scala> val p: Point(13,  37)
% scala> p.scale(4, 2)
% scala> p scale (3, 7)
% \end{REPL}
%
%
% \QUESTEND
%
%
%
%
% %%<AUTOEXTRACTED by mergesolu>%%      %Uppgift 12
%
%
%
%
% \WHAT{Föränderlighet och oföränderlighet.}
%
% \QUESTBEGIN
%
% \Task  \what~  Oföränderliga och föränderliga objekt beter sig olika vid tilldelning.
%
% \Subtask\Pen Innan du kör nedan kod: Försök lista ut vad som kommer att skrivas ut. Rita minnessituationen efter varje tilldelning.
%
% \begin{Code}
% println("\n--- Example 1: mutable value assigmnent")
% var x1 = 42
% var y1 = x1
% x1 = x1 + 42
% println(x1)
% println(y1)
% \end{Code}
%
% \Subtask\Pen Innan du kör nedan kod: Försök lista ut vad som kommer att skrivas ut. Rita minnessituationen efter varje tilldelning.
%
% \begin{Code}
% println("\n--- Example 2: mutable object reference assignment")
% class MutableInt(private var i: Int) {
%   def +(a: Int): MutableInt = { i = i + a; this }
%   override def toString: String = i.toString
% }
% var x2 = new MutableInt(42)
% var y2 = x2
% x2 = x2 + 42
% println(x2)
% println(y2)
% \end{Code}
%
% \Subtask\Pen Innan du kör nedan kod: Försök lista ut vad som kommer att skrivas ut. Rita minnessituationen efter varje tilldelning.
%
% \begin{Code}
% println("\n--- Example 3: immutable object reference assignment")
% class ImmutableInt(val i: Int) {
%   def +(a: Int): ImmutableInt = new ImmutableInt(i + a)
%   override def toString: String = i.toString
% }
% var x3 = new ImmutableInt(42)
% var y3 = x3
% x3 = x3 + 42
% println(x3)
% println(y3)
% \end{Code}
%
% \Subtask\Pen Vad finns det för fördelar med oföränderliga datastrukturer?
%
%
% \SOLUTION
%
%
% \TaskSolved \what
%
%
% \SubtaskSolved   \includegraphics[scale=0.5]{../img/w04-solutions/uppgift-13a}
%
% \SubtaskSolved
% \begin{enumerate}
% \item \includegraphics[scale=0.5]{../img/w04-solutions/uppgift-13b-1}
% \item \includegraphics[scale=0.5]{../img/w04-solutions/uppgift-13b-2}
% \item \includegraphics[scale=0.5]{../img/w04-solutions/uppgift-13b-3}
% \end{enumerate}
%
% \SubtaskSolved
% \begin{enumerate}
% \item \includegraphics[scale=0.5]{../img/w04-solutions/uppgift-13c-1}
% \item \includegraphics[scale=0.5]{../img/w04-solutions/uppgift-13c-2}
% \item \includegraphics[scale=0.5]{../img/w04-solutions/uppgift-13c-3}
% \end{enumerate}
%
% \SubtaskSolved   En stor fördel är att vi till exempel kan skicka med en immutable som argument till en metod och vara säkra på att metoden inte ändrar på värdet.
%
%
% \QUESTEND
%
%
%
%
% %%<AUTOEXTRACTED by mergesolu>%%      %Uppgift 13
%
%
%
%
% \WHAT{Några användbara samlingar.}
%
% \QUESTBEGIN
%
% \Task  \what~  En \textbf{samling} \Eng{collection} är en datastruktur som samlar många objekt av samma typ. I \code{scala.collection} och \code{java.util} finns många olika samlingar med en uppsjö användbara metoder. De olika samlingarna i \code{scala.collection} är ordnade i en gemensam hierarki med många gemensamma metoder; därför har man nytta av det man lär sig om metoderna i en Scala-samling när man använder en annan samling. Vi har redan tidigare sett samlingen \code{Vector}:
%
% \begin{REPL}
% scala> val tärningskast = Vector.fill(10000)((math.random() * 6 + 1).toInt)
% scala> tä   // tryck TAB
% scala> tärningskast.  // tryck TAB
% \end{REPL}
%
% \Subtask Ungefär hur många metoder finns det som man kan göra på objekt av typen \code{Vector}? Det är svårt att lära sig alla dessa på en gång, så vi väljer ut några få i kommande uppgifter.
%
% \Subtask Jämför överlappet mellan metoderna i \code{Vector} och \code{List} och uppskatta hur stor andel av metoderna som är gemensamma:
% \begin{REPL}
% scala> val myntkast =
%          List.fill(10000)(if (math.random() < 0.5) "krona" else "klave")
% scala> my   // tryck TAB
% scala> myntkast.  // tryck TAB
% \end{REPL}
%
% \SOLUTION
%
%
% \TaskSolved \what
%
%
% \SubtaskSolved   Ungefär 150 metoder.
%
% \SubtaskSolved   Ungefär lika många.
%
%
% \QUESTEND
%
%
%
%
% %%<AUTOEXTRACTED by mergesolu>%%      %Uppgift 14
%
%
%
%
% \WHAT{Typparameter.}
%
% \QUESTBEGIN
%
% \Task  \what~  Vissa funktioner är generella för många typer och tar en så kallad \textbf{typparameter} inom hakparenteser. Ofta slipper man skriva typparametrar, då kompilatorn kan härleda typen utifrån argumenten. Om man anger typparametrar explicit så hjälper kompilatorn dig med att kolla att det verkligen är rätt typ i samlingen.
%
% \Subtask Vad händer nedan?
% \begin{REPL}
% scala> var xs = Vector.empty[Int]
% scala> xs = xs :+ "42"
% scala> xs = xs :+ 43 :+ 64 :+ 46
% scala> xs
% scala> xs :+= "42".toInt
% scala> var ys = Vector[Int]("ett", "två", "tre")
% scala> var ingenting = Vector.empty
% scala> ingenting = Vector(1,2,3)
% \end{REPL}
%
% \Subtask Samlingar är mer användbara om de är \emph{generiska}, vilket innebär att elementens typ avgörs av en typparameter och därför kan vara av vilken typ som helst. Man kan definiera egna funktioner som tar generiska samlingar som parametrar. Förklara vad som händer här:
% \begin{REPL}
% scala> val vego = Vector("gurka", "tomat", "apelsin", "banan")
% scala> val prim = Vector(2, 3, 5, 7, 11, 13)
% scala> def först[T](xs: Vector[T]): T = xs.head
% scala> def sist[T](xs: Vector[T]) = xs.last
% scala> def förstOchSist[T](xs: Vector[T]): (T, T) = (xs.head, xs.last)
% scala> först(vego)
% scala> sist(prim)
% scala> förstOchSist(vego)
% scala> förstOchSist(prim)
% scala> def wrap[T](pair: (T, T))(xs: Vector[T]) = pair._1 +: xs :+ pair._2
% scala> wrap("Odla", "och ät!")(vego)
% scala> wrap("Odla", "och ät!")(vego).mkString(" ")
% \end{REPL}
%
%
%
%
%
% \SOLUTION
%
%
% \TaskSolved \what
%
%
% \SubtaskSolved
% \\1. Instansierar en tom vektor med element av typen int och tilldelar värdet till en variabel xs.
% \\2. Error eftersom \code{xs :+ ''42''} ger en Vector[Any] när Vector[Int] krävs.
% \\3. xs tilldelas ett nytt värde av Vector(43, 64, 46)
% \\4. xs skrivs ut.
% \\5. Lägger till talet 42 i xs.
% \\6. Error: type mismatch
% \\7. Skapar en tom Vector i variablen ingenting
% \\8. error: type mismatch; found: Int(3), required: Nothing
%
% \SubtaskSolved
% Tre metoder skapas: den första för att få första elementet i en lista, och eftersom den definieras med specialtypen T går den att använda med alla vektorer oavsett typen av variabeln i vektorn. Den andra får fram sista elementet och den sista hämtar båda två.
%
% En till function definieras längre ner med  namnet ''wrap'', som tar en lista och lägger till ett element längst fram och ett längst bak.
%
%
% \QUESTEND
%
%
%
%
% %%<AUTOEXTRACTED by mergesolu>%%      %Uppgift 15
%
%
%
%
% \WHAT{Några viktiga samlingsmetoder.}
%
% \QUESTBEGIN
%
% \Task  \what~  Deklarera följande vektorer i REPL.
% \begin{REPL}
% scala> val xs = (1 to 10).toVector
% scala> val a = Vector("abra", "ka", "dabra")
% scala> val b = Vector( "sim", "sala", "bim", "sala", "bim")
% scala> val stor = Vector.fill(100000)(math.random())
% \end{REPL}
% Undersök i REPL vad som händer nedan. Alla dessa metoder fungerar på alla samlingar som är indexerbara sekvenser. Givet deklarationerna ovan: vad har uttrycken nedan för värde och typ? Förklara vad som händer hälp av denna  översikt: \href{http://docs.scala-lang.org/overviews/collections/seqs}{docs.scala-lang.org/overviews/collections/seqs}
%
% \Subtask \code{a(1) + xs(1)}
%
% \Subtask \code{a apply 0}
%
% \Subtask \code{a.isDefinedAt(3)}
%
% \Subtask \code{a.isDefinedAt(100)}
%
% \Subtask \code{stor.length}
%
% \Subtask \code{stor.size}
%
% \Subtask \code{stor.min}
%
% \Subtask \code{stor.max}
%
% \Subtask \code{a indexOf "ka"}
%
% \Subtask \code{b.lastIndexOf("sala")}
%
% \Subtask \code{"först" +: b   //minnesregel: colon on the collection side}
%
% \Subtask \code{a :+ "sist"    //minnesregel: colon on the collection side}
%
% \Subtask \code{xs.updated(2,42)}
%
% \Subtask \code{a.padTo(10, "!")}
%
% \Subtask \code{b.sorted}
%
% \Subtask \code{b.reverse}
%
% \Subtask \code{a.startsWith(Vector("abra", "ka"))}
%
% \Subtask \code{"hejsan".endsWith("san")}
%
% \Subtask \code{b.distinct}
%
%
%
% \SOLUTION
%
%
% \TaskSolved \what
%
%
% \SubtaskSolved   String = ''ka2''
%
% \SubtaskSolved   String = ''abra''
%
% \SubtaskSolved   false
%
% \SubtaskSolved   false
%
% \SubtaskSolved   100000
%
% \SubtaskSolved   100000
%
% \SubtaskSolved   minsta talet i listan
%
% \SubtaskSolved   största talet i listan
%
% \SubtaskSolved   1
%
% \SubtaskSolved   3
%
% \SubtaskSolved   Vektor b fast med ''först'' som första element
%
% \SubtaskSolved   Vektor a fast med ''sist'' som sista element.
%
% \SubtaskSolved   plats 3 i vektorn xs får värdet 42
%
% \SubtaskSolved   En ny vektor fylld med ''!'' från och med plats 4 till 10. Men de andra värdena samma som i a.
%
% \SubtaskSolved   b sorterad i bokstavsordning
%
% \SubtaskSolved   b baklänges
%
% \SubtaskSolved   true
%
% \SubtaskSolved   true
%
% \SubtaskSolved   en vektor med alla unika element i b.
%
%
% \QUESTEND
%
%
%
%
% %%<AUTOEXTRACTED by mergesolu>%%      %Uppgift 16
%
%
%
%
% \WHAT{Några generella samlingsmetoder.}
%
% \QUESTBEGIN
%
% \Task  \what~  Det finns metoder som går att köra på \emph{alla} samlingar även om de inte är indexerbara. Givet deklarationerna i föregående uppgift: vad har uttrycken nedan för värde och typ? Förklara vad som händer med hjälp av dessa översikter: \\ \href{http://docs.scala-lang.org/overviews/collections/trait-traversable}{docs.scala-lang.org/overviews/collections/trait-traversable} \\ \href{http://docs.scala-lang.org/overviews/collections/trait-iterable}{docs.scala-lang.org/overviews/collections/trait-iterable}
%
% \Subtask \code{a ++ b}
%
% \Subtask \code{a ++ stor}
%
% \Subtask \code{val ys = xs.map(_ * 5)}
%
% \Subtask \code{b.toSet     // En mängd har inga dubletter}
%
% \Subtask \code{a.head + b.last}
%
% \Subtask \code{a.tail}
%
% \Subtask \code{a.head +: a.tail == a}
%
% \Subtask \code{Vector(a.head) ++ Vector(b.last)}
%
% \Subtask \code{a.take(1) ++ b.takeRight(1)}
%
% \Subtask \code{a.drop(2) ++ b.drop(1).dropRight(2)}
%
% \Subtask \code{a.drop(100)}
%
% \Subtask \code{val e = Vector.empty[String]; e.take(100)}
%
% \Subtask \code{Vector(e.isEmpty, e.nonEmpty)}
%
% \Subtask \code{a.contains("ka")}
%
% \Subtask \code{"ka" contains "a"}
%
% \Subtask \code{a.filter(s => s.contains("k")) }
%
% \Subtask \code{a.filter(_.contains("k")) }
%
% \Subtask \code{a.map(_.toUpperCase).filterNot(_.contains("K")) }
%
% \Subtask \code{xs.filter(x => x % 2 == 0)}
%
% \Subtask \code{xs.filter(_ % 2 == 0)}
%
%
% \SOLUTION
%
%
% \TaskSolved \what
%
%
% \SubtaskSolved
% Metoden ger tillbaka en ny Vector[String] som nu består av alla element i a plus alla element i b. I samma ordning med elementen i a först.
%
% \SubtaskSolved
% Samma som i uppgift a fast vektorn som returnas är av typen Vector[Any]. Det är eftersom Any är den närmsta typen som String och Double delar. Elementen från vektor a är fortfarande först och uppföljt av elementen i stor.
%
% \SubtaskSolved
% Variablen ys får värdet av en Vector[Int] som innehåller alla talen från xs fast multiplicerade med 5. Alltså ys = 5, 10, 15..., osv.
%
% \SubtaskSolved
% Functionen tar alla värden från en Vektor och sätter in i ett Set (mängd). Eftersom en mängd ej har dubletter så försvinner ett ''sala'' och ett ''bim'', Vector[String] som returneras blir därför (''sim'', ''sala'', ''bim'').
%
% \SubtaskSolved
% Metoden head ger första elementet i en samling, och last sista. Därför blir kombinationen av a.head och b.last en ny Vector[String] som består av a:s första element, och b:s första element.
%
% \SubtaskSolved
% Ger en Vector[String] som innehåller alla element efter det första. Alltså i detta fallet ''ka'' och ''dabra''.
%
% \SubtaskSolved
% True, eftersom head ger första elementet och tail ger resten, sedan sätter metoden +: ihop dem till en vektor med samma värden som a.
%
% \SubtaskSolved
% Eftersom ++ sätter ihop alla värden från två vektorer måste vi först omvandla från en sträng till vektor. Resultatet blir en ny vektor av samma typ som innan med a:s första element och b:S sista.
%
% \SubtaskSolved
% Samma resultat som i h, metoden take börjar från vänster och tar så många element som man skickar med som parameter och gör till en ny lista. Med 1 som parameter motsvarar det att göra Vector(a.head). Metoden takeRight gör samma sak fast från höger.
%
% \SubtaskSolved
% Metoden drop är motsvarigheten till take fast exkluderar de specifierade elementen istället för att inkludera dem i vektorn.
%
% \SubtaskSolved
% Eftersom a endast innehåller 3 element returnerar drop(100) en tom vektor.
%
% \SubtaskSolved
% Returnerar en tom vektor med element typen String
%
% \SubtaskSolved
% returnerar Vector(true, false)
%
% \SubtaskSolved
% True, metoden contains kollar om en samling innehåller ett specifikt element.
%
% \SubtaskSolved
% True. Eftersom en sträng även kan ses som Vector[Char].
%
% \SubtaskSolved
% Filtrerar vektorn a till att endast innehålla strängar som innehåller k.
%
% \SubtaskSolved
% Exakt samma som i p
%
% \SubtaskSolved
% map(\_.toUpperCase) omvandlar alla strängar i a till stora bokstäver
% filterNot(\_.contains(''K'')) tar resultatet vi precis fick och tar bort alla strängar som innehåller stora K.
%
% \SubtaskSolved
% filtrerar så att endast jämna tal finns kvar.
%
% \SubtaskSolved
% Exakt samma som i s
%
%
%
%
% \QUESTEND
%
%
%
%
% %%<AUTOEXTRACTED by mergesolu>%%      %Uppgift 17
%
%
%
%
% \WHAT{NEEDS A TOPIC DESCRIPTION}
%
% \QUESTBEGIN
%
% \Task  \what~ De olika samlingarna i \code{scala.collection} används flitigt i andra paket, exempelvis \code{scala.util} och \code{scala.io}.
%
% \Subtask Vad händer här? (Metoden \code{shuffle} skapar en ny samling med elementen i slumpvis ordning.)
% \begin{REPL}
% val xs = Vector(1,2,3)
% def blandat = scala.util.Random.shuffle(xs)
% def test = if (xs == blandat) "lika" else "olika"
% (for(i <- 1 to 100) yield test).count(_ == "lika")
% \end{REPL}
%
%
% \Subtask Skapa en textfil med namnet \code{fil.txt} som innehåller lite text och läs in den med: \\\code{scala.io.Source.fromFile("fil.txt", "UTF-8").getLines.toVector}
% \begin{REPL}
% > cat > fil.txt
% hejsan
% svejsan
% > scala
% scala> val xs = scala.io.Source.fromFile("fil.txt", "UTF-8").getLines.toVector
% scala> xs.foreach(println)
% \end{REPL}
%
%
% \Subtask Vad händer här? (Metoden \code{trim} på värden av typen \code{String} ger en ny sträng med blanktecken i början och slutet borttagna.)
% \begin{REPL}
% scala> val pgk =
%   scala.io.Source.fromURL("http://cs.lth.se/pgk/","UTF-8").getLines.toVector
% scala> pgk.foreach(println)
% scala> pgk.map(_.trim).
%          filterNot(_.startsWith("<")).
%          filterNot(_.isEmpty).
%          foreach(println)
% \end{REPL}
%
%
%
% \SOLUTION
%
%
% \TaskSolved \what
%
%
% \SubtaskSolved
% Vi instansierar en vektor xs med talen 1, 2 och 3.
% sedan definierar vi en metod blandat som ger oss en randomiserad version av xs.
% sedan definierar vi en till metod som testar om xs är lika med resultatet från blandat. Om det är så returnerar den strängen ''lika'' annars ''olika''.
% Sist kör vi en for-loop där vi 100 gånger kör testet, samtidigt räknas hur många gånger ''lika'' returneras.
%
% Vårt resultat är en siffra på hur många gånger xs var samma som en blandad version av sig själv, eftersom det finns 6 permutationer med 3 variabler så borde det vara ungefär 1/6 chans.
%
% \SubtaskSolved  -
%
% \SubtaskSolved
% \\ \code{map(\_.trim)} tar bort alla onödiga mellanrum i början och slutet på varje rad
% \\ \code{filterNot(\_.startsWith(''<''))} filtrerar bort alla rader som börjar med strängen ''<''
% \\ \code{filterNot(\_.isEmpty)} filtrerar bort alla tomma rader.
% \\ \code{foreach(println)} skriver ut alla rader.
%
%
% \QUESTEND
%
%
%
%
% %%<AUTOEXTRACTED by mergesolu>%%      %Uppgift 18
%
%
%
%
% \WHAT{Jämföra List och Vector.}
%
% \QUESTBEGIN
%
% \Task  \what~  En indexerbar sekvens av värden kallas vektor eller lista. I Scala finns flera klasser som kan kan indexeras, däribland klasserna \code{Vector} och \code{List}.
%
% \Subtask \emph{Likheter mellan \code{Vector} och \code{List}.} Kör nedan rader i REPL. Prova indexera i båda och studera hur stor andel av metoderna som är gemensamma.
% \begin{REPL}
% scala> val sv = Vector("en", "två", "tre", "fyra")
% scala> val en = List("one", "two", "three", "four")
% scala> sv(0) + sv(3)
% scala> en(0) + en(3)
% scala> sv. //tryck TAB
% scala> en. //tryck TAB
% \end{REPL}
%
% \Subtask \emph{Skillnader mellan \code{Vector} och \code{List}.} Klassen \code{Vector} i Scala har ''under huven'' en avancerad datastruktur i form av ett s.k. självbalanserande träd, vilket gör att \code{Vector} är snabbare än \code{List} på nästan allt, \emph{utom} att bearbeta elementen i \emph{början} av sekvensen; vill man lägga till och ta bort i början av en \code{List} så kan det ibland gå ungefär dubbelt så fort jämfört med \code{Vector}, medan alla andra operationer är lika snabba eller snabbare med \code{Vector}. Det finns ett fåtal speciella metoder, som bara finns i \code{List}, för att skapa en lista och lägga till i början av en lista. Vad händer nedan?
%
% \begin{REPL}
% scala> var xs = "one" :: "two" :: "three" :: "four" :: Nil
% scala> xs = "zero" :: xs
% scala> val ys = xs.reverse ::: xs
% \end{REPL}
%
%
% \SOLUTION
%
%
% \TaskSolved \what
%
%
% \SubtaskSolved
% I princip alla metoder delas, en lista har några fler t. ex. ''::'', '':::'', ''mapConserve'' osv.
%
% \SubtaskSolved
% Först skapas en lista med 4 sträng värden och instansierar variablen xs med det värdet.
% sedan skapar vi en ny lista, som består av ''zero'' + den gamla listan och ger värdet till xs.
% Sist instansierar vi en ny variabel ys, som får värdet av xs omvänd plus xs.
%
%
% \QUESTEND
%
%
%
%
% %%<AUTOEXTRACTED by mergesolu>%%      %Uppgift 19
%
%
%
%
% \WHAT{Mängd.}
%
% \QUESTBEGIN
%
% \Task  \what~  En mängd är en samling som garanterar att det inte finns några dubbletter. Det går dessutom väldigt snabbt, även i stora mängder, att kolla om ett element finns eller inte i mängden. Elementen i samlingen \code{Set} hamnar ibland, av effektivitetsskäl, i en förvånande ordning.
% \begin{REPL}
% scala> val s = Set("Malmö", "Stockholm", "Göteborg", "Köpenhamn", "Oslo")
% s: scala.collection.immutable.Set[String] =
%      Set(Oslo, Malmö, Köpenhamn, Stockholm, Göteborg)
%
% scala> val t = Set("Sverige", "Sverige", "Sverige", "Danmark", "Norge")
% t: scala.collection.immutable.Set[String] = Set(Sverige, Danmark, Norge)
% \end{REPL}
% Givet ovan deklarationer: vad blir värde och typ av nedan uttryck?
%
% \Subtask \code{s + "Malmö" == s}
%
% \Subtask \code{s ++ t}
%
% \Subtask \code{Set("Malmö", "Oslo").subsetOf(s)}
%
% \Subtask \code{s subsetOf Set("Malmö", "Oslo")}
%
% \Subtask \code{s contains "Lund"}
%
% \Subtask \code{s apply "Lund"}
%
% \Subtask \code{s("Malmö")}
%
% \Subtask \code{s - "Stockholm"}
%
% \Subtask \code{t - ("Norge", "Danmark", "Tyskland")}
%
% \Subtask \code{s -- t}
%
% \Subtask \code{s -- Set("Malmö", "Oslo")}
%
% \Subtask \code{Set(1,2,3) intersect Set(2,3,4)}
%
% \Subtask \code{Set(1,2,3) & Set(2,3,4)}
%
% \Subtask \code{Set(1,2,3) union Set(2,3,4)}
%
% \Subtask \code{Set(1,2,3) | Set(2,3,4)}
%
%
% \SOLUTION
%
%
% \TaskSolved \what
%
%
% \SubtaskSolved
% true, Boolean
%
% \SubtaskSolved
% En samling av alla värden i s och t, Set[String]
%
% \SubtaskSolved
% true, Boolean
%
% \SubtaskSolved
% false, Boolean
%
% \SubtaskSolved
% false, Boolean
%
% \SubtaskSolved
% false, Boolean
%
% \SubtaskSolved
% true, Boolean
%
% \SubtaskSolved
% Samlingen s utan elementet ''Stockholm'', Set[String]
%
% \SubtaskSolved
% Samlingen t utan elementen ''Norge'' och ''Danmark'', Set[String]
%
% \SubtaskSolved
% returnerar s, Set[String]
%
% \SubtaskSolved
% Samlingen s utan ''Malmö'' och ''Oslo'', Set[String]
%
% \SubtaskSolved
% Set(2, 3), Set[Int]
%
% \SubtaskSolved
% se deluppgift l
%
% \SubtaskSolved
% Set(1, 2, 3 ,4), Set[Int]
%
% \SubtaskSolved
% se deluppgift n
%
%
% \QUESTEND
%
%
%
%
% %%<AUTOEXTRACTED by mergesolu>%%      %Uppgift 20
%
%
%
%
% \WHAT{Slå upp värden från nycklar med \code{Map}.}
%
% \QUESTBEGIN
%
% \Task  \what~  Samlingen \code{Map} är mycket användbar. Med den kan man snabbt leta upp ett värde om man har en nyckel. Samlingen \code{Map} är en generalisering av en vektor, där man kan ''indexera'', inte bara med ett heltal, utan med vilken typ av värde som helst, t.ex. en sträng. Datastrukturen \code{Map} är en s.k. \emph{associativ array}\footnote{\href{https://en.wikipedia.org/wiki/Associative_array}{https://en.wikipedia.org/wiki/Associative\_array}}, implementerad som en s.k. \emph{hashtabell}\footnote{\href{https://en.wikipedia.org/wiki/Hash_table}{https://en.wikipedia.org/wiki/Hash\_table}}.
% \begin{REPL}
% scala> var huvudstad =
%   Map("Sverige" -> "Stockholm", "Norge" -> "Oslo", "Skåne" -> "Malmö")
% \end{REPL}
% Givet ovan variabel \code{huvudstad}, förklara vad som händer nedan?
%
% \Subtask \code{huvudstad apply "Skåne"}
%
% \Subtask \code{huvudstad("Sverige")}
%
% \Subtask \code{huvudstad.contains("Skåne")}
%
% \Subtask \code{huvudstad.contains("Malmö")}
%
% \Subtask \code{huvudstad += "Danmark" -> "Köpenhamn"}
%
% \Subtask \code{huvudstad.foreach(println)}
%
% \Subtask \code{huvudstad getOrElse ("Norge", "???") }
%
% \Subtask \code{huvudstad getOrElse ("Finland", "???") }
%
% \Subtask \code{huvudstad.keys.toVector.sorted}
%
% \Subtask \code{huvudstad.values.toVector.sorted}
%
% \Subtask \code{huvudstad - "Skåne"}
%
% \Subtask \code{huvudstad - "Jylland"}
%
% \Subtask \code{huvudstad = huvudstad.updated("Skåne","Lund") }
%
%
%
% \SOLUTION
%
%
% \TaskSolved \what
%
%
% \SubtaskSolved
% Returnerar strängen ''Malmö'' eftersom det värdet är indexerat på platsen ''Skåne''.
%
% \SubtaskSolved
% Returnerar strängen ''Stockholm'' eftersom det värdet är indexerat på platsen ''Sverige''.
%
% \SubtaskSolved
% true, eftersom huvudstad innehåller indexet ''Skåne''
%
% \SubtaskSolved
% false, eftersom huvudstad ej innehåller indexet ''Malmö''. Notera att det är index och inte värden vi
% kollar om det finns.
%
% \SubtaskSolved
% Lägger till indexet ''Danmark'' med värdet ''Köpenhamn'' i samlingen.
%
% \SubtaskSolved
% Skriver ut alla 2-tupler.
%
% \SubtaskSolved
% Returnerar ''Oslo'', Note: Om indexet ''Norge'' inte hade funnits hade ''???'' returnerats istället.
%
% \SubtaskSolved
% Returnerar ''???''
%
% \SubtaskSolved
% Returnerar en sorterar vektor med alla index.
%
% \SubtaskSolved
% Returnerar en sorterar vektor med alla värden.
%
% \SubtaskSolved
% Returnerar en ny mängd men med ''Skåne'' -> ''Malmö'' borttaget.
%
% \SubtaskSolved
% Returnerar huvudstad mängden eftersom det inte finns ett ''Jylland'' index att ta bort.
%
% \SubtaskSolved
% Uppdaterar indexet ''Skåne'' till att istället leda till värdet ''Lund''
%
%
% \QUESTEND
%
%
%
%
% %%<AUTOEXTRACTED by mergesolu>%%      %Uppgift 21
%
%
%
%
% \WHAT{Skapa Map från en samling.}
%
% \QUESTBEGIN
%
% \Task  \what~
%
% \Subtask Definiera denna vektor och undersök dess typ:
% \begin{Code}
% val pairs = Vector(
%   ("Björn", 46462229009L),
%   ("Maj", 46462221667L),
%   ("Gustav", 46462224906L))
% \end{Code}
%
% \Subtask Vad har variablen \code{telnr} nedan för typ: \\ \code{var telnr = pairs.toMap}
%
% \Subtask Använd \code{telnr} för att slå upp telefonnummer för Maj och Kim med hjälp av metoderna \code{apply} och \code{get}.
%
% \Subtask Använd metoden \code{getOrElse} vid upplagningar av \code{telnr} och ge \code{-1L} som telefonnummer i händelse av att ett nummer inte finns.
%
% \Subtask Lägg till \code{("Fröken Ur", 464690510L)} i \code{telnr}-mappen.
%
% \Subtask Skapa en \code{Vector[(String, String)]} enligt nedan, så att telefonnumret blir en sträng utan inledande landsnummer men med en nolla i riktnumret. Byt ut \code{???} mot lämpligt uttryck.
% \begin{REPL}
% scala> telnr.toVector.map(p => ???)
% res85: Vector[(String, String)] = Vector(("Björn", "0462229009"), ("Maj",
% "0462221667"), ("Gustav", "0462224906"), ("Fröken Ur", 04690510"))
%
% \end{REPL}
%
% \Subtask Använd vektorn i resultatet ovan för att skapa en ny \code{Map[String, String]} med nationella telefonnumer. Slå upp numret till Fröken Ur.
%
% \SOLUTION
%
%
% \TaskSolved \what
%
%
% \SubtaskSolved
% \begin{REPLnonum}
% pairs: scala.collection.immutable.Vector[(String, Long)] =
% 					Vector((Björn,444), (Maj,441), (Lucy,666))
% \end{REPLnonum}
%
% \SubtaskSolved
% Map[String, Long]
%
% \SubtaskSolved
% \begin{REPLnonum}
% scala> telnr(''Maj'')
% res0: Long = 441
%
% scala> telnr.get(''Maj'')
% res1: Option[Long] = Some(441)
%
% scala> telnr(''Kim'')
% java.util.NoSuchElementException: key not found: 'Kim
%   at scala.collection.MapLike$class.default(MapLike.scala:228)
%   at scala.collection.AbstractMap.default(Map.scala:59)
%   at scala.collection.MapLike$class.apply(MapLike.scala:141)
%   at scala.collection.AbstractMap.apply(Map.scala:59)
%   ... 32 elided
%
% scala> telnr.get(''Kim'')
% res2: Option[Long] = None
% \end{REPLnonum}
%
% \SubtaskSolved
% \begin{REPLnonum}
% scala> telnr.getOrElse(''Maj'', -1L)
% res0: Long = 441
%
% scala> telnr.getOrElse(''Kim'', -1L)
% res1: Long = -1
% \end{REPLnonum}
%
% \SubtaskSolved
% telnr += ''Fröken Ur'' -> 464690510L
%
% \SubtaskSolved
% telnr.toVector.map(p => p.\_1 -> (''0'' + p.\_2.toString.substring(2)))
%
% \SubtaskSolved
% Använd metoden toMap och apply.
%
%
%
%
% \QUESTEND
%
%
%
%
% %%<AUTOEXTRACTED by mergesolu>%%      %Uppgift 22
%
%
%
%
% \WHAT{Samlingsmetoden \code{maxBy}.}
%
% \QUESTBEGIN
%
% \Task  \what~  Med samlingsmetoden \code{maxBy} kan man själv definiera vad som ska maximeras. (Denna metod kommer du att behöva i veckans laboration.)
%
% \Subtask Förklara vad som händer nedan.
% \begin{REPL}
% scala> val xs = Vector((2,3), (1,5), (-1, 1), (7, 2))
% scala> xs.maxBy(x => x._1)
% scala> xs.maxBy(x => x._2)
% \end{REPL}
%
% \Subtask Om man bara använder en parameter i en anonym funktion, till exempel parametern \code{x} i lambdauttrycket \code{x => x + 1} \emph{en enda} gång, och kompilatorn kan gissa alla typer, kan man använda understreck som ''platshållare'' för att förkorta lambdauttrycket så här: \code{ _ + 1}
%
% Skriv uttrycken på raderna 2 och 3 i föregående deluppgift på ett kortare sätt med hjälp platshållarsyntax \Eng{place holder syntax}.
%
% \Subtask På motsvarande sätt kan man använda \code{minBy} för att välja vilken funktion som definierar minimum. Prova \code{minBy} på motsvarande sätt som i föregående deluppgifter.
%
% \SOLUTION
%
%
% \TaskSolved \what
%
%
% \SubtaskSolved   Metoden maxBy hämtar det element som är ''störst'', på rad två gör \code{x => x._1} att första värdet i tuplerna används för att bestämma vilken som är störst. Likt gör \code{x => x._2} på rad tre att istället det andra värdet används.
%
% \SubtaskSolved
% \begin{REPLnonum}
% scala> xs.maxBy(_._1)
% scala> xs.maxBy(_._2)
% \end{REPLnonum}
%
% \SubtaskSolved
% \begin{REPLnonum}
% scala> xs.minBy(_._1)
% scala> xs.minBy(_._2)
% \end{REPLnonum}
%
%
%
% \QUESTEND
%
%
%
%
%
%
%
%
% \WHAT{NEEDS A TOPIC DESCRIPTION}
%
% \QUESTBEGIN
%
% \Task  \what~ Skriv nedan program med en editor och kompilera från terminalen. Lägg till kod i huvudprogrammet som testar klassen \code{Account} och kompilera och kör. Utvidga sedan klassen \code{Account} med fler attribut och funktioner som du väljer själv.
%
% \begin{Code}
% class Account(val number: Long, val maxCredit: Int){
%   private var balance = 0
%
%   def deposit(amount: Int): Int = {
%     if (amount > 0) {balance += amount}
%     balance
%   }
%
%   def withdraw(amount: Int): (Int, Int) = if (amount > 0) {
%     val allowedWithdrawal =
%       if (amount < balance + maxCredit) amount
%       else balance + maxCredit
%     balance = balance - allowedWithdrawal
%     (allowedWithdrawal, balance)
%   } else (0, balance)
%
%   def show: Unit =
%     println("Account Nbr: " + number + " balance: " + balance)
% }
%
% object Main {
%   def main(args: Array[String]): Unit = {
%     ???
%   }
% }
% \end{Code}
%
%
%
% \SOLUTION
%
%
% \QUESTEND
%
%
%
%
%
%
% \WHAT{NEEDS A TOPIC DESCRIPTION}
%
% \QUESTBEGIN
%
% \Task \label{task:keno-set} \what~  Läs om reglerna för spelet Keno här: \\ \url{https://sv.wikipedia.org/wiki/Keno} och gör deluppgifterna nedan.
%
% \Subtask Skapa en klass \code{Keno} som kan användas för att genomföra en Kenodragning. Låt klassen ha ett privat attribut \code{balls} som är en föränderlig mängd med heltal och som från början är tom. Implementera lämpliga metoder i klassen för att användaren av klassen ska kunna dra nya slumpmässiga bollar som inte redan är dragna.
%
% \Subtask Skapa en \code{case class KenoBet(bet: Set[Int])} för att hålla reda vilka 11 bollar en viss person satsar på. Definiera en metod \\ \code{def numberOfHits(keno: Keno): Int = ???}\\ i case-klassen \code{KenoBet} som givet en kenodragning räknar ut hur många bollar som satsats rätt.
%
% \Subtask Skriv ett huvudprogram som simulerar en enkel Kenodragning. Låt två personer satsa på 11 slumpmässiga bollar, genomför en dragning av 20 bollar ur 70 möjliga och kontrollera sedan hur många bollar som personerna har prickat rätt.
%
%
%
%
%
% \SOLUTION
%
%
% \QUESTEND
%
%
%
%
%
%
% \WHAT{Dokumentationen för \code{Any}.}
%
% \QUESTBEGIN
%
% \Task  \what~  Undersök vilka metoder som finns i klassen Any här: \href{http://www.scala-lang.org/api/current/scala/Any.html}{http://www.scala-lang.org/api/current/scala/Any.html}. Prova några av metoderna i REPL.
%
% \SOLUTION
%
%
% \QUESTEND
%
%
%
%
%
%
% \WHAT{Dokumentationen för samlingar.}
%
% \QUESTBEGIN
%
% \Task  \what~  Leta upp metoden \code{tabulate} i dokumentationen för objektet \code{Vector} nästan längst ner i listan här: \\ \href{http://www.scala-lang.org/api/current/scala/collection/immutable/Vector.html}{http://www.scala-lang.org/api/current/scala/collection/immutable/Vector.html} \\Leta upp den variant av \code{tabulate} som har signaturen:\\ \code{def tabulate[A](n: Int)(f: (Int) => A): Vector[A] }\\ Klicka på den gråfyllda trekanten till vänster om signaturen som fäller ut beskrivningen
%
% \Subtask Förklara vad som händer här:
% \begin{REPLnonum}
% scala> Vector.tabulate(10)(i => i % 3)
% \end{REPLnonum}
%
% \Subtask Klicka på det blåa stora o-et överst på sidan, för att växla till klass-vyn och studera listan med alla metoder  i klassen \code{Vector}.
%
%
% \SOLUTION
%
%
% \QUESTEND
%
%
%
%
%
%
% \WHAT{Fler metoder på indexerbara sekvenser.}
%
% \QUESTBEGIN
%
% \Task  \what~  Deklarera följande vektorer i REPL.
% \begin{REPL}
% scala> val xs = (1 to 10).toVector
% scala> val a = Vector("abra", "ka", "dabra")
% scala> val b = Vector( "sim", "sala", "bim", "sala", "bim")
% \end{REPL}
% Undersök i REPL vad som händer nedan. Alla dessa metoder fungerar på alla samlingar som är indexerbara sekvenser. Vad har uttrycken för värde och typ? Förklara vad metoden gör. Studera även denna  översikt: \href{http://docs.scala-lang.org/overviews/collections/seqs}{docs.scala-lang.org/overviews/collections/seqs}
%
% \Subtask \code{b.indexWhere(s => s.startsWith("b"))}  % advanced
%
% \Subtask \code{a.indices}  % advanced
%
% \Subtask \code{xs.patch(1, Vector(42,43,44), 7)} % advanced
%
% \Subtask \code{xs.segmentLength(_ < 8, 2)} % advanced
%
% \Subtask \code{b.sortBy(_.reverse)}  % advanced
%
% \Subtask \code{b.sortWith((s1, s2) => s1.size < s2.size)} % advanced
%
% \Subtask \code{a.reverseMap(_.size)}	% advanced
%
% \Subtask \code{a intersect Vector("ka", "boom", "pow")} % advanced
%
% \Subtask \code{a diff Vector("ka")} % advanced
%
% \Subtask \code{a union Vector("ka", "boom", "pow")} % advanced
%
%
%
% \SOLUTION
%
%
% \QUESTEND
%
%
%
%
% \WHAT{NEEDS A TOPIC DESCRIPTION}
%
% \QUESTBEGIN
%
% \Task  \what~ För samlingen \code{List} finns en alternativ metod till \code{+:} som heter \code{::} och kallas ''cons'' och som i kombination med objektet \code{Nil} kan användas för att med alternativ syntax bygga listor. Läs om detta här: \\ \href{http://alvinalexander.com/scala/how-create-scala-list-range-fill-tabulate-constructors}{alvinalexander.com/scala/how-create-scala-list-range-fill-tabulate-constructors} \\ och hitta på några egna övningar för att undersöka hur cons och Nil fungerar. Metoder som slutar med kolon är högerassociativa. Läs mer om detta här: \href{http://www.artima.com/pins1ed/basic-types-and-operations.html#5.8}{http://www.artima.com/pins1ed/basic-types-and-operations.html\#5.8}\SOLUTION
%
%
% \QUESTEND


%!TEX encoding = UTF-8 Unicode
%!TEX root = ../exercises.tex

\ifPreSolution

\Exercise{\ExeWeekTEN}\label{exe:W10}

\begin{Goals}
\input{modules/w10-inheritance-exercise-goals.tex}
\end{Goals}

\begin{Preparations}
\item \StudyTheory{10}
\end{Preparations}

\BasicTasks

\else

\ExerciseSolution{\ExeWeekTEN}

\BasicTasks

\fi



\WHAT{Para ihop begrepp med beskrivning.}

\QUESTBEGIN

\Task \what

\vspace{1em}\noindent Koppla varje begrepp med den (förenklade) beskrivning som passar bäst:

\begin{ConceptConnections}
  bastyp & 1 & & A & har supertypen \code|AnyRef|, allokeras i heapen via referens \\ 
  supertyp & 2 & & B & kan ha många former, t.ex. en av flera subtyper \\ 
  subtyp & 3 & & C & klass utan namn, utvidgad med extra implementation \\ 
  körtidstyp & 4 & & D & en typ som är mer specifik \\ 
  dynamisk bindning & 5 & & E & kan ha parametrar, kan ej instansieras, kan ej mixas in \\ 
  polymorfism & 6 & & F & saknar implementation \\ 
  trait & 7 & & G & har supertypen \code|AnyVal|, lagras direkt på stacken \\ 
  inmixning & 8 & & H & tillföra egenskaper med \code|with| och en trait \\ 
  överskuggad medlem & 9 & & I & är abstrakt, kan mixas in, kan ha parametrar \\ 
  anonym klass & 10 & & J & kan vara mer specifik än den statiska typen \\ 
  skyddad medlem & 11 & & K & är endast synlig i subtyper \\ 
  abstrakt medlem & 12 & & L & körtidstypen avgör vilken metod som körs \\ 
  abstrakt klass & 13 & & M & medlem i subtyp ersätter medlem i supertyp \\ 
  förseglad typ & 14 & & N & den mest generella typen i en arvshierarki \\ 
  referenstyp & 15 & & O & en typ som är mer generell \\ 
  värdetyp & 16 & & P & subtypning utanför denna kodfil är förhindrad \\ 
\end{ConceptConnections}

\SOLUTION

\TaskSolved \what

\begin{ConceptConnections}
  bastyp & 1 & ~~\Large$\leadsto$~~ &  N & den mest generella typen i en arvshierarki \\ 
  supertyp & 2 & ~~\Large$\leadsto$~~ &  O & en typ som är mer generell \\ 
  subtyp & 3 & ~~\Large$\leadsto$~~ &  D & en typ som är mer specifik \\ 
  körtidstyp & 4 & ~~\Large$\leadsto$~~ &  J & kan vara mer specifik än den statiska typen \\ 
  dynamisk bindning & 5 & ~~\Large$\leadsto$~~ &  L & körtidstypen avgör vilken metod som körs \\ 
  polymorfism & 6 & ~~\Large$\leadsto$~~ &  B & kan ha många former, t.ex. en av flera subtyper \\ 
  trait & 7 & ~~\Large$\leadsto$~~ &  I & är abstrakt, kan mixas in, kan ha parametrar \\ 
  inmixning & 8 & ~~\Large$\leadsto$~~ &  H & tillföra egenskaper med \code|with| och en trait \\ 
  överskuggad medlem & 9 & ~~\Large$\leadsto$~~ &  M & medlem i subtyp ersätter medlem i supertyp \\ 
  anonym klass & 10 & ~~\Large$\leadsto$~~ &  C & klass utan namn, utvidgad med extra implementation \\ 
  skyddad medlem & 11 & ~~\Large$\leadsto$~~ &  K & är endast synlig i subtyper \\ 
  abstrakt medlem & 12 & ~~\Large$\leadsto$~~ &  F & saknar implementation \\ 
  abstrakt klass & 13 & ~~\Large$\leadsto$~~ &  E & kan ha parametrar, kan ej instansieras, kan ej mixas in \\ 
  förseglad typ & 14 & ~~\Large$\leadsto$~~ &  P & subtypning utanför denna kodfil är förhindrad \\ 
  referenstyp & 15 & ~~\Large$\leadsto$~~ &  A & har supertypen \code|AnyRef|, allokeras i heapen via referens \\ 
  värdetyp & 16 & ~~\Large$\leadsto$~~ &  G & har supertypen \code|AnyVal|, lagras direkt på stacken \\ 
\end{ConceptConnections}

\QUESTEND





\WHAT{Gemensam bastyp.}

\QUESTBEGIN

\Task  \what~  Man vill ofta lägga in objekt av olika typ i samma samling.
\begin{REPL}
scala> class Gurka(val vikt: Int)
scala> class Tomat(val vikt: Int)
scala> val gurkor = Vector(Gurka(100), Gurka(200))
scala> val grönsaker = Vector(Gurka(300), Tomat(42))
\end{REPL}

\Subtask Om en samling innehåller objekt av flera olika typer försöker kompilatorn härleda den mest specifika typen som objekten har gemensamt. Vad blir det för typ på värdet \code{grönsaker} ovan?

\Subtask Försök ta reda på summan av vikterna enligt nedan. Vad ger andra raden för felmeddelande? Varför?

\begin{REPL}
scala> gurkor.map(_.vikt).sum     // fungerar
scala> grönsaker.map(_.vikt).sum  // fungerar inte
\end{REPL}

\Subtask Du ska nu göra så att du kan komma åt vikten på alla grönsaker genom att ge gurkor och tomater en gemensam bastyp som de olika konkreta grönsakstyperna utvidgar med nyckelordet \code{extends}. Det heter att subtyperna \code{Gurka} och \code{Tomat} \textbf{ärver} egenskaperna hos supertypen \code{Grönsak}.

Skapa en bastyp \code{Grönsak} med ett abstrakt attribut \code{vikt}. Låt sedan de konkreta grönsakerna ärva bastypen:

\begin{REPL}
scala> trait Grönsak { val vikt: Int }
scala> class Gurka(val vikt: Int) extends Grönsak
scala> class Tomat(val vikt: Int) extends Grönsak
scala> val gurkor = Vector(Gurka(100), Gurka(200))
scala> val grönsaker = Vector(Gurka(300), Tomat(42))
\end{REPL}
När sker initialisering av attributet \code{vikt}?

\Subtask Vad blir det nu för typ på variabeln \code{grönsaker} ovan?

\Subtask Går det nu att summera av vikterna i \code{grönsaker} med uttrycket nedan? Varför?\\ \code{grönsaker.map(_.vikt).sum}


\Subtask En trait liknar en klass, men man kan inte instansiera den direkt. Vad blir det för felmeddelande om du försöker skapa en instans av en trait enligt nedan?
\begin{REPL}
scala> trait Grönsak { val vikt: Int }
scala> new Grönsak
\end{REPL}


\Subtask Traiten \code{Grönsak} har en abstrakt medlem \code{vikt}. Den sägs vara abstrakt eftersom den saknar definition -- medlemmen har bara ett namn och en typ men inget värde. Du kan instansiera den abstrakta traiten \code{Grönsak} om du fyller i det som ''fattas'', nämligen ett värde på \code{vikt}. Man kan fylla på det som fattas i genom att ''hänga på'' ett block efter typens namn vid instansiering. Man får då vad som kallas en \textbf{anonym klass}, i detta fall en ganska konstig grönsak som inte är någon speciell sorts grönsak med som ändå har en vikt.

Vad får \code{anonymGrönsak} nedan för typ och strängrepresenation?
\begin{REPL}
scala> val anonymGrönsak = new Grönsak { val vikt = 42 }
\end{REPL}

\Subtask Vad blir felmeddelandet om du skapar en anonym klass \code{Grönsak} med en kropp som saknar definition av vikt?

\SOLUTION


\TaskSolved \what


\SubtaskSolved  \code{Vector[Object]}. Typen \code{Object} i JVM är motsvarar typen \code{AnyRef} som är bastyp för alla referenstyper.

\SubtaskSolved  Felmeddelande:
\begin{REPLnonum}
scala> grönsaker.map(_.vikt).sum  
-- Error:                                                                                 
1 |grönsaker.map(_.vikt).sum
  |              ^^^^^^
  |             value vikt is not a member of Object - did you mean wait?
-- Error:
1 |grönsaker.map(_.vikt).sum
  |                         ^
  |ambiguous implicit arguments: both object DoubleIsFractional in object Numeric and object ShortIsIntegral in object Numeric match type Numeric[B] of parameter num of method sum in trait IterableOnceOps
\end{REPLnonum}
Det första felmeddelandet beror på att vektorns element är av typen \code{Object} och medlemmen \code{vikt} är inte definierat för denna typ. Det andra felmeddelandet är ett följdfel som beror på att en sekvens med element av typen \code{Object} inte kan summeras eftersom kompilatorn inte kan härleda att elementtypen är numerisk.

\SubtaskSolved  Attributet \code{vikt} initialiseras vid konstruktion av \code{Gurka} resp. \code{Tomat}. Värdet ges av resp. klassparameter.

\SubtaskSolved  \code{Vector[Grönsak]}.

\SubtaskSolved  Ja. Eftersom den statiska typen för elementen i sekvensen är \code{Grönsak} (den dynamiska typen kan vara godtycklig subtyp av \code{Grönsak}) och alla instanser av denna typ garanterat har attributet \code{vikt} som är av typen \code{Int} så kan kompilatorn vid \emph{kompileringstid} dra slutsatsen att summeringen är giltig och därmed kan kompilatorn kompilera koden till körbar maskinkod.

\SubtaskSolved  
\begin{REPLnonum}
scala> new Grönsak
-- Error:
1 |new Grönsak
  |    ^^^^^^^
  |    Grönsak is a trait; it cannot be instantiated
\end{REPLnonum}

\SubtaskSolved  
\begin{REPLnonum}
scala> val anonymGrönsak = new Grönsak { val vikt = 42 }
val anonymGrönsak: Grönsak = anon$1@1edde8b6
scala> anonymGrönsak.toString                                                                                      
val res0: String = anon$1@1edde8b6
\end{REPLnonum}
Typen är \code{Grönsak} och blir här en s.k. \emph{anonym klass}, eftersom vi inte har använt en namngiven klass med \code{extends}, utan bara ''hängt på'' en klasskropp inom klammerparenteser direkt vid konstruktion. När du skapar anonyma klasser måste du använda nyckelordet \code{new}.

Kompilatorn hittar på ett unikt klassnamn, här anon\$1, för att hålla reda på den anonyma klassen under kompilering till maskinkod. Strängrepresentationen innehåller ett hexadecimalt heltal som är unikt för instansen, här \code{1edde8b6}.

\SubtaskSolved  

\begin{REPLsmall}
scala> new Grönsak { }
-- Error:
1 |new Grönsak { }
  |^
  |object creation impossible, since val vikt: Int in trait Grönsak is not defined 

\end{REPLsmall}


\QUESTEND






\WHAT{Polymorfism vid arv, s.k. subtypspolymorfism.}

\QUESTBEGIN

\Task  \what~  Polymorfism betyder ''många skepnader''. I samband med arv  innebär det att flera subtyper, till exempel \code{Ko} och \code{Gris}, kan hanteras gemensamt som om de vore instanser av samma supertyp, så som \code{Djur}. Subklasser kan implementera en metod med samma namn på olika sätt. Vilken metod som exekveras bestäms vid körtid beroende på vilken subtyp som instansieras. På så sätt kan djur komma i många skepnader.

\Subtask Implementera funktionen \code{skapaDjur} nedan så att den returnerar antingen en ny \code{Ko} eller en ny \code{Gris} med lika sannolikhet.

\begin{REPL}
scala> trait Djur { def väsnas: Unit }
scala> class Ko   extends Djur { def väsnas = println("Muuuuuuu") }
scala> class Gris extends Djur { def väsnas = println("Nöffnöff") }
scala> def skapaDjur(): Djur = ???
scala> val bondgård = Vector.fill(42)(skapaDjur())
scala> bondgård.foreach(_.väsnas)
\end{REPL}

\Subtask Lägg till ett djur av typen Häst som väsnas på lämpligt sätt och modifiera \code{skapaDjur} så att det skapas kor, grisar och hästar med lika sannolikhet.


\SOLUTION


\TaskSolved \what


\SubtaskSolved
\begin{Code}
def skapaDjur(): Djur = 
  if math.random() > 0.5 then Ko() else Gris()
\end{Code}

\SubtaskSolved
\begin{Code}
class Häst extends Djur: 
  def väsnas = println("Gnääääägg") 

def skapaDjur(): Djur = 
   math.random() match
    case r if r < 0.33 => Ko() 
    case r if r < 0.67 => Gris() 
    case _             => Häst()
\end{Code}


\QUESTEND





\WHAT{Olika typer av heltalspar till laborationen \hyperref[section:lab:\LabWeekTEN]{\texttt{\LabWeekTEN}}.}


\QUESTBEGIN


\Task\label{exe:inheritance:labprep-pair}  \what~Under veckans laboration ska du använda olika typer av par som representerar riktning och position på en tvådimensionell spelplan, samt spelplanens storlek. I stället för att använda en vanlig 2-tupel till dessa tre olika typer av par ska du skapa egna, specifika  typer som alla ärver bastypen \code{Pair[T]}. Dessa typer ska alla ligga i filen \code{pairs.scala} i \code{package snake}.
\begin{Code}
// detta är en skiss på filen pairs.scala
package snake

trait Pair[T]:
  def x: T
  def y: T
  // uppgift a) lägg till den konkreta metoden tuple

// efterföljande deluppgifterna implementerar dessa subtyper till Pair:
//   case klass Dim beskriver en 2-dimensionell ytas storlek
//   case klass Pos beskriver en position på en yta av Dim storlek
//   enum Dir beskriver förflyttning mot North, South, East, West
\end{Code}
Skillnaden mellan \code{Pair[T]} och en vanlig 2-tupel är att medlemmarna \code{x} och \code{y} garanterat är av \emph{samma} typ, medan en 2-tupel kan innehålla element av olika typ.

I fig. \ref{snake:fig:pairs-uml} visas en bild av klasshierarkin som du steg-för-steg ska utveckla i efterföljande  uppgifter. Fördelen med att ha olika typer av par är att det är mer typsäkert \Eng{type safe}: vi får hjälp av kompilatorn att upptäcka om vi av misstag förväxlar t.ex. en position med en riktning.

\begin{figure}[H]
\begin{center}
\newcommand{\TextBox}[1]{\raisebox{0pt}[1em][0.5em]{#1}}
\tikzstyle{umlclass}=[rectangle, draw=black,  thick, anchor=north, text width=2cm, rectangle split, rectangle split parts = 3]
\begin{tikzpicture}[inner sep=0.5em,scale=1.2, every node/.style={transform shape}]

  \node [umlclass, rectangle split parts = 1, xshift=0cm, yshift=4.5cm] (BaseType1)  {
              \textit{\textbf{\centerline{\TextBox{\code{Pair[T]}}}}}
%              \nodepart[align=left]{second}\code{def x: T} \newline \code{def y: T}
          };


  \node [umlclass, rectangle split parts = 1, xshift=-3cm, yshift=2.5cm] (SubType1)  {
              \textit{\textbf{\centerline{\TextBox{\code{Dim}}}}}
%              \nodepart[align=left]{second}\code{val x: Int} \newline \code{val y: Int}
          };

\node [umlclass, rectangle split parts = 1, xshift=0cm, yshift=2.5cm] (SubType2)  {
            \textit{\textbf{\centerline{\TextBox{\code{Pos}}}}}
%            \nodepart[]{second}\TextBox{\code{val dim: Int}}
        };

\node [umlclass, rectangle split parts = 1, xshift=3cm, yshift=2.5cm] (SubType3)  {
            \textit{\textbf{\centerline{\TextBox{\code{Dir}}}}}
%            \nodepart[]{second}\TextBox{\code{val dim: Int}}
        };


\draw[umlarrow] (SubType1.north) -- ++(0,0.5) -| (BaseType1.south);
\draw[umlarrow] (SubType2.north) -- ++(0,0.5) -| (BaseType1.south);
\draw[umlarrow] (SubType3.north) -- ++(0,0.5) -| (BaseType1.south);

\end{tikzpicture}
\end{center}
\caption{Arvshierarki med \code{Pair[T]} som bastyp.}
\label{snake:fig:pairs-uml}
\end{figure}

\Subtask Öppna en editor och koda \code{trait Pair[T]} i en fil \code{pairs.scala}. Lägg dessutom till en konkret metod \code{tuple} i \code{Pair[T]} som returnerar en 2-tupel med de båda elementen i paret, så att det vid behov går att omvandla \code{Pair}-instanser till 2-tupler. Använd REPL via \code{sbt console} för att testa att detta fungerar:
\begin{REPLnonum}
scala> val p = new Pair[Int] { override val x = 10; override val y = 20 }
p: Pair[Int]{val x: Int; val y: Int} = $anon$1@784223e9

scala> p.tuple
val res0: (Int, Int) = (10,20)
\end{REPLnonum}

\Subtask Skapa en case-klass \code{Dim} som ärver \code{Pair[Int]}. Instanser av denna klass kommer du att använda under veckans laboration för att representera en spelplans storlek genom att låta \code{x} ange antalet horisontella positioner och \code{y} antalet vertikala positioner.

Lägg även till ett kompanjonsobjekt \code{Dim} med en \code{apply}-metod som kan skapa \code{Dim}-instanser givet en 2-tupel.
Testa i REPL enligt nedan.
\begin{REPLnonum}
scala> Dim(50, 60)
val res1: Dim = Dim(50,60)

scala> Dim((60, 50))
val res2: Dim = Dim(60,50)

scala> res2.tuple
val res3: (Int, Int) = (60,50)
\end{REPLnonum}

\Subtask Lägg till en case-klass \code{Pos} som ärver \code{Pair[Int]} som representerar en position med en \code{x}-koordinat och en \code{y}-koordinat, båda klassparametrar. Kordinaterna ska hållas inom en spelplansstorlek som ges av klassparametern \code{dim} av typen \code{Dim}. Kordinatpositionerna är heltal och räknas från \code{0} till (men inte med) \code{dim.x} resp. \code{dim.y}.

Gör primärkonstruktorn i case-klassen \code{Pos} \textbf{privat}, genom att skriva nyckelordet \code{private} efter klassnamnet men före klassparameterlistan, så att det inte går att skapa instanser via primärkonstruktorn utanför klasskroppen och kompanjonsobjektet. 

Implementera metoderna \code{+} och \code{-} i case-klassen \code{Pos}. Båda metoderna ska ta en parameter \code{p} av typen \code{Pair[Int]} och returnera en ny \code{Pos}, där \code{p.x} resp. \code{p.y} är adderat resp. subtraherat från aktuell position. Observera att du inte ska skriva \code{new} när du skapar en ny instans, eftersom dessa alltid ska skapas via kompanjonsobjektets \code{apply}-metod, som är en ''smart'' fabriksmetod som garanterar håller koordinaterna inom spelplanen. 

Lägg till ett kompanjonsobjekt \code{Pos} med en \code{apply}-metod som skapar en ny \code{Pos}-instans som ser till att koordinaterna alltid är inom \code{dim}. Aritmetiken ska ske modulo storleken \code{dim}, d.v.s en position ska aldrig kunna hamna utanför spelplanen; i stället så börjar man om på andra sidan (se exempel i REPL nedan). \\ \emph{Tips:} Använd  \code|java.lang.Math.floorMod| som hanterar negativa argument så att resultatet blir positivt (till skillnad från modulo-operatorn \%).

Lägg även till fabriksmetoden \code{random} som kan skapa nya slumpmässiga positioner inom \code{dim}. \emph{Tips:} Använd \code{scala.util.Random.nextInt}.

Testa att det fungerar enligt nedan:
\begin{REPLnonum}
scala> Pos(-1,20,Dim(10,20))
val res4: Pos = Pos(9,0,Dim(10,20))

scala> new Pos(-1,20,Dim(10,20))  // förbjuds med privat primärkonstruktor
-- Error:
1 |new Pos(-1,20,Dim(10,20))
  |    ^^^
  |constructor Pos cannot be accessed as a member of Pos

scala> Pos(0,0,Dim(5,5)) + Pos(6,12, Dim(5,5))                                                                     
val res5: Pos = Pos(1,2,Dim(5,5))

scala> Pos(0,0,Dim(5,5)) - Pos(1,2, Dim(5,5))                                                                     
val res6: Pos = Pos(4,3,Dim(5,5))

scala> for (_ <- 1 to 3) yield Pos.random(Dim(10,10))
val res7: IndexedSeq[Pos] = 
  Vector(Pos(8,8,Dim(10,10)), Pos(2,6,Dim(10,10)), Pos(3,7,Dim(10,10)))
\end{REPLnonum}

\Subtask Vad händer om du glömmer skriva \code{new} när du anropar den privata konstruktorn i din \code{apply}-metod? Varför finns inte detta problem i \code{apply}-metoden för \code{Dim}?

\Subtask Lägg till en \code{enum Dir} som ärver \code{Pair[Int]} och har två \code{val}-parametrar \code{x} och \code{y}. Lägg också till fyra fall med \code{case} som alla ärver \code{Dir} och som representerar en enstegsförflyttning i de fyra väderstrecken, genom att ge parametrarna \code{x} resp. \code{y} något av värden $1$, $-1$ eller $0$. Norrut ska anges med x-koordinaten $-1$ och y-koordinaten $0$, etc. Verifiera i REPL att enumerationen fungerar.

Lägg till en \code{export} som gör så att det räcker att importera \code{snake.*} för att få alla fyra riktningar synliga direkt (annars behövs även import av \code{Dir.*} på alla ställen där riktning används i och utanför paketet \code{snake})


\SOLUTION


\TaskSolved \what

\SubtaskSolved
\begin{CodeSmall}
trait Pair[T]:
  def x: T
  def y: T
  def tuple: (T, T) = (x, y)

\end{CodeSmall}

\SubtaskSolved
\begin{CodeSmall}
case class Dim(x: Int, y: Int) extends Pair[Int]
object Dim:
  def apply(dim: (Int, Int)): Dim = Dim(dim._1, dim._2)  
\end{CodeSmall}

\SubtaskSolved
\begin{CodeSmall}
case class Pos private (x: Int, y: Int, dim: Dim) extends Pair[Int]:
  def +(p: Pair[Int]): Pos = Pos(x + p.x, y + p.y, dim)
  def -(p: Pair[Int]): Pos = Pos(x - p.x, y - p.y, dim)

object Pos:
  def apply(x: Int, y: Int, dim: Dim): Pos = 
    import java.lang.Math.floorMod as mod
    new Pos(mod(x, dim.x), mod(y, dim.y), dim) //OBS: new nödvändig här!

  def random(dim: Dim): Pos = 
    import scala.util.Random.nextInt as rni
    Pos(rni(dim.x), rni(dim.y), dim)
\end{CodeSmall}

\SubtaskSolved Om du glömmer skriva \code{new} explicit i kompanjonsobjektets \code{apply}-metod så blir det ett rekursivt anrop som resulterar i en oändlig loop vid körtid. Med \code{new} så är det garanterat den privata primärkonstruktorn för \code{Pos} som anropas. 

I \code{Dim.apply} så skiljer sig parametertyperna åt mellan fabriksmetoden och primärkonstruktorn och kompilatorn väljer då primärkonstruktorn eftersom den passar med de givna två separata heltalen och inte med en 2-tupel.

\SubtaskSolved
\begin{CodeSmall}
enum Dir(val x: Int, val y: Int) extends Pair[Int]:
  case North extends Dir( 0, -1)
  case South extends Dir( 0,  1)
  case East  extends Dir( 1,  0)
  case West  extends Dir(-1,  0)
export Dir.*  // gör så att North etc blir synliga i paketet snake
\end{CodeSmall}

\QUESTEND






\WHAT{Supertyp med parameter.}

\QUESTBEGIN

\Task  \what~  Utbildningsdepartementet vill med sitt nya datasystem hålla koll på vissa personer och skapar därför en klasshierarki enligt nedan. Skriv in koden i en editor och testa i REPL med \code{sbt}.
\begin{Code}
class Person(val namn: String)

class Akademiker(
  namn: String,
  val universitet: String) extends Person(namn)

class Student(
  namn: String,
  universitet: String,
  program: String) extends Akademiker(namn, universitet)

class Forskare(
  namn: String,
  universitet: String,
  titel: String) extends Akademiker(namn, universitet)
\end{Code}


\Subtask Deklarera fyra olika \code{val}-variabler med lämpliga namn som refererar till olika instanser av alla olika klasser ovan och ge attributen valfria initialvärden genom olika parametrar till konstruktorerna.

\Subtask Skriv två satser: en som först stoppar in instanserna i en \code{Vector} och en som sedan loopar igenom vektorn och skriv ut alla instansers \code{toString} och \code{namn}.

\Subtask Utbildningsdepartementet vill att det inte ska gå att instansiera objekt av typerna \code{Person} och \code{Akademiker}. Det kan åstadkommas genom att placera nyckelordet \code{abstract} före \code{class}. Uppdatera koden i enlighet med detta. Vilket blir felmeddelande om man försöker instansiera en \code{abstract class}? Går det lika bra med en \code{trait}?

\Subtask Utbildningsdeparetementet vill slippa implementera \code{toString}. Gör därför om typerna \code{Student} och \code{Forskare} till case-klasser. \emph{Tips:} För att undkomma ett kompileringsfel (vilket?) behöver du använda \code{override val} på lämpligt ställe.
Skapa instanser av de nya case-klasserna \code{Student} och \code{Forskare} och skriv ut deras \code{toString}. 

\Subtask 
%Eftersom \code{Person} och \code{Akademiker} nu är abstrakta, vill utbildningsdepartementet att du gör om dessa typer till traits med abstrakta attribut istället för klasser. 
Använd abstrakta attribut i stället för parametrar för typerna som är abstrakta, så att du inte behöver skriva \code{override val} i klassparametrarna till de konkreta case-klasserna.
Du ska också införa en case-klass \code{IckeAkademiker} som ska användas i olika statistiska medborgarundersökningar.
Dessutom förser man alla personer med ett personnummer representerat som en \code{Long}.
Hur ser utbildningsdepartementets kod ut nu, efter alla ändringar? Skriv ett testprogram som skapar några instanser och skriver ut deras attribut.

\SOLUTION


\TaskSolved \what


\SubtaskSolved
\begin{Code}
val person = new Person("Person1")
val akademiker = new Akademiker("Person2", "LTH")
val student = new Student("Person3", "LTH", "D")
val forskare = new Forskare("Person4", "LTH", "Doktorand")
\end{Code}

\SubtaskSolved
\begin{Code}
val vec = Vector(person, akademiker, student, forskare)
for(i <- vec){ print(i.toString + i.namn) }
\end{Code}

\SubtaskSolved  
Felmeddelande vid instansiering av \code{abstract class Akademiker}:\\
\texttt{Akademiker is abstract; it cannot be instantiated}

Det går \emph{inte} lika bra med en \code{trait} i det speciella fallet \code{Akademiker}, eftersom en trait inte får skicka vidare parametrar till en supertyp. Felmeddelande:\\
\texttt{trait Akademiker may not call constructor of trait Person}
\begin{Code}
trait Person(val namn: String)

abstract class Akademiker(
  namn: String,
  val universitet: String) extends Person(namn)

class Student(
  namn: String,
  universitet: String,
  program: String) extends Akademiker(namn, universitet)

class Forskare(
  namn: String,
  universitet: String,
  titel: String) extends Akademiker(namn, universitet)
\end{Code}



\SubtaskSolved  
\begin{REPLnonum}
scala>  
     |trait Person(val namn: String)                                                                              
     | 
     | abstract class Akademiker(
     |   namn: String,
     |   val universitet: String) extends Person(namn)
     | 
     | case class Student(
     |   namn: String,
     |   universitet: String,
     |   program: String) extends Akademiker(namn, universitet)
     | 
     | case class Forskare(
     |   namn: String,
     |   universitet: String,
     |   titel: String) extends Akademiker(namn, universitet)
-- Error:     
8 |  namn: String,
  |  ^
  |  error overriding value namn in trait Person of type String;
  |    value namn of type String needs `override` modifier
-- Error:
9 |  universitet: String,
  |  ^
  |  error overriding value universitet in class Akademiker of type String;
  |    value universitet of type String needs `override` modifier
-- Error:
13 |  namn: String,
   |  ^
   |  error overriding value namn in trait Person of type String;
   |    value namn of type String needs `override` modifier
-- Error:
14 |  universitet: String,
   |  ^
   |  error overriding value universitet in class Akademiker of type String;
   |    value universitet of type String needs `override` modifier
\end{REPLnonum}

\begin{Code}
trait Person(val namn: String)

abstract class Akademiker(
  namn: String,
  val universitet: String) extends Person(namn)

case class Student(
  override val namn: String,
  override val universitet: String,
  program: String) extends Akademiker(namn, universitet)

case class Forskare(
  override val namn: String,
  override val universitet: String,
  titel: String) extends Akademiker(namn, universitet)
\end{Code}

\begin{REPLsmall}
scala> val ps = Vector(Student("Kim", "Lund", "D"), Forskare("Herz", "Lund", "Dr"))
val ps: Vector[Akademiker] = Vector(Student(Kim,Lund,D), Forskare(Herz,Lund,Dr))
scala> ps :+ new Person("Abstrakt") {}
val res0: Vector[Person] = 
  Vector(Student(Kim,Lund,D), Forskare(Herz,Lund,Dr), anon1@1941bbf3)
\end{REPLsmall}

\SubtaskSolved
\begin{Code}
trait Person: 
  val namn: String 
  val nbr: Long

trait Akademiker extends Person:
  val universitet: String

case class Student(
  namn: String,
  nbr: Long,
  universitet: String,
  program: String) extends Akademiker

case class Forskare(
  namn: String,
  nbr: Long,
  universitet: String,
  titel: String) extends Akademiker

case class IckeAkademiker(
    namn: String,
    nbr: Long) extends Person
\end{Code}



\QUESTEND




%\clearpage




\ExtraTasks %%%%%%%%%%%%%%%%%





%\WHAT{Uppräknade värden.}

%\QUESTBEGIN

% \Task  \what~  Ett sätt att hålla reda på uppräknade värden, t.ex. färgen på olika kort i en kortlek, är att använda olika heltal som får representera de olika värdena, till exempel så här:\footnote{Om namnkonventioner för konstanter i Scala: läs under rubriken ''Constants, Values, Variable and Methods'' här \href{http://docs.scala-lang.org/style/naming-conventions.html}{docs.scala-lang.org/style/naming-conventions.html}}
% \begin{Code}
% object Färg {
%   val Spader = 1
%   val Hjärter = 2
%   val Ruter = 3
%   val Klöver = 4
% }
% \end{Code}
% Dessa kan sedan användas som parametrar till denna case-klass vid skapande av kortobjekt:
% \begin{lstlisting}[language=,keywords={case,class}]
% case class Kort(färg: Int, valör: Int)
% \end{lstlisting}
% Man kan hålla reda på färgen med en variabel av typen \code{Int} och tilldela den en viss färg med ovan konstanter. Och när du skapar ett kort kan du använda färgnamnet och du slipper därmed att behöva komma ihåg vilket heltal som representerar färgen.
% \begin{REPL}
% scala> val f = Färg.Spader
% scala> import Färg._
% scala> Kort(Ruter, 7)
% \end{REPL}
% En annan fördelen med detta är att man lätt kan iterera över alla färger:
% \begin{REPL}
% scala> val kortlek = for (f <- 1 to 4; v <- 1 to 13) yield Kort(f, v)
% \end{REPL}
% Men den stora nackdelen med detta är att kompilatorn vid kompileringstid inte kollar om variablerna av misstag råkar ges något värde utanför det giltiga intervallet, eftersom alla heltal är möjliga. Detta får vi själv hålla koll på vid körtid, till exempel med funktionen \code{require} eller \code{if}-satser, etc.

% Istället kan man använda uppräknade värden med hjälp av case-objekt enligt nedan deluppgifter och därmed få hjälp av kompilatorn att hitta eventuella fel vid kompileringstid.  Ett case-objekt är som ett vanligt singelton-objekt men det får bl.a. automatiskt en \code{toString} som är samma som namnet. Case-objekt kan dessutom användas som värden i mönstermatchningar (mer om detta i kapitel \ref{chapter:W10}).

% \Subtask Deklarera följande uppräknade värden som singelton-objekt med gemensam bastyp. Med nyckelordet \code{sealed} så ''förseglas'' klassen och inga andra direkta subtyper tillåts förutom de som finns i samma kod-fil eller block. I detta exempel  med kortfärger vet vi ju att det inte finns fler än dessa fyra färger.
% \begin{Code}
% sealed trait Färg
% case object Spader extends Färg
% case object Hjärter extends Färg
% case object Ruter extends Färg
% case object Klöver extends Färg
% \end{Code}
% Dessa kan sedan användas som parametrar till denna case-klass vid skapande av kortobjekt:
% \begin{lstlisting}[language=,keywords={case,class}]
% case class Kort(färg: Färg, valör: Int)
% \end{lstlisting}
% Skapa därefter några exempelinstanser av klassen \code{Kort}. Vad är fördelen med ovan angreppssätt jämfört med att använda heltalskonstanter?

% \Subtask Om man vill kunna iterera över alla värden är det bra om de finns i en samling med alla värden. Vi kan lägga en sådan i ett kompanjonsobjekt till bastypen enligt nedan. Skriv ut alla färgvärden med en \code{for}-sats.

% \begin{Code}
% sealed trait Färg
% object Färg {
%   val values = Vector(Spader, Hjärter, Ruter, Klöver)
% }
% case object Spader extends Färg
% case object Hjärter extends Färg
% case object Ruter extends Färg
% case object Klöver extends Färg
% \end{Code}
% Skapa en kortlek med 52 olika kort och blanda den sedan med \code{Random.shuffle} enligt nedan. Använd en dubbel \code{for}-sats och \code{yield}.
% \begin{REPL}
% scala> val kortlek: Vector[Kort] = ???
% scala> val blandad = scala.util.Random.shuffle(kortlek)
% \end{REPL}

% \Subtask Skriv en funktion \code{ def blandadKortlek: Vector[Kort] = ???} som ger en ny blandad kortlek varje gång den anropas med metoden i föregående uppgift.

% \Subtask Om man även vill ha en heltalsrepresentation med en medlem \code{toInt} för alla värden, kan man ge bastypen en parameter och i stället för en trait (som inte kan ha några parametrar) använda en abstrakt klass.

% \begin{Code}
% sealed abstract class Färg(final val toInt: Int)
% object Färg {
%   val values = Vector(Spader, Hjärter, Ruter, Klöver)
% }
% case object Spader  extends Färg(0)
% case object Hjärter extends Färg(1)
% case object Ruter   extends Färg(2)
% case object Klöver  extends Färg(3)
% \end{Code}
% Skapa en funktion \code{färgPoäng} som räknar ut summan av heltalsrepresentationen av alla färger hos en vektor med kort, och använd den sedan för att räkna ut \code{färgPoäng} för de första fem korten.
% \begin{REPL}
% scala> def blandadKortlek: Vector[Kort] = ???
% scala> def färgPoäng(xs: Vector[Kort]): Int = ???
% scala> färgPoäng(blandadKortlek.take(5))
% \end{REPL}


% \SOLUTION

% \TaskSolved \what

% \SubtaskSolved  Sättet är säkrare då man inte kan tilldela korten en färg som inte finns. Med heltalskonstanterna kan man till exempel ge ett kort färgen 5, vilken inte korresponderar till någon riktig färg.

% \SubtaskSolved  \code{for (f <- Färg.values; v <- 1 to 13) yield Kort(f,v)}

% \SubtaskSolved
% \begin{Code}
% def blandadKortlek: Vector[Kort] = {
%   val kortlek =
%     for (f <- Färg.values; v <- 1 to 13) yield Kort(f,v)
%   scala.util.Random.shuffle(kortlek)
% }
% \end{Code}

% \SubtaskSolved  \code{def färgPoäng(xs: Vector[Kort]): Int = xs.map(_.färg.toInt).sum}

% \QUESTEND







\WHAT{Bastypen \code{Shape} och subtyperna \code{Rectangle} och \code{Circle}.}

\QUESTBEGIN

\Task  \what~  Du ska i denna uppgift skapa ett litet bibliotek för geometriska former med oföränderliga objekt implementerade med hjälp av case-klasser. De geometriska formerna har en gemensam bastyp kallad \code{Shape}. Utgå från koden nedan.

\begin{CodeSmall}
case class Point(x: Double, y: Double):
  def move(dx: Double, dy: Double): Point = Point(x + dx, y + dy)

trait Shape:
  def pos: Point
  def move(dx: Double, dy: Double): Shape

case class Rectangle(pos: Point, width: Double, height: Double) extends Shape:
  def move(dx: Double, dy: Double): Rectangle = copy(pos = pos.move(dx, dy))

case class Circle(pos: Point, radius: Double) extends Shape:
  def move(dx: Double, dy: Double): Circle = copy(pos = pos.move(dx, dy))

\end{CodeSmall}

\Subtask Instansiera några cirklar och rektanglar och gör några relativa förflyttningar av dina instanser genom att anropa \code{move}.

\Subtask Lägg till en konkret metod \code{moveTo} i \code{Point} som gör en absolut förflyttning till koordinaterna \code{x} och \code{y}. Lägg till en abstrakt metod \code{moveTo} \code{Shape} som implementeras i subklasserna. Testa med REPL på några instanser av \code{Rectangle} och \code{Circle}.

\Subtask Lägg till metoden \code{distanceTo(that: Point): Double } i case-klassen \code{Point} som räknar ut avståndet till en annan punkt med hjälp av \code{math.hypot}. Klistra in i REPL och testa på några instanser av \code{Point}.

\Subtask Lägg till en konkret metod \code{distanceTo(that: Shape): Double } i traiten \code{Shape} som räknar ut avståndet till positionen för en annan Shape. Testa i REPL på några instanser av \code{Rectangle} och \code{Circle}.

\Subtask Gör så att \code{distanceTo} kan anropas med operatornotation.

\SOLUTION


\TaskSolved \what


\SubtaskSolved
\begin{CodeSmall}
val c1 = Circle(Point(1, 1), 42)
val r1 = Rectangle(Point(3, 3), 20, 30)
c1.move(2, 3)
r1.move(3, 2)
\end{CodeSmall}

\SubtaskSolved  
\begin{CodeSmall}
case class Point(x: Double, y: Double):
  def move(dx: Double, dy: Double): Point = Point(x + dx, y + dy)
  def moveTo(x: Double, y: Double): Point = Point(x, y)

trait Shape:
  def pos: Point
  def move(dx: Double, dy: Double): Shape
  def moveTo(x: Double, y: Double): Shape

case class Rectangle(pos: Point, width: Double, height: Double) extends Shape:
  def move(dx: Double, dy: Double): Shape = copy(pos = pos.move(dx, dy))
  def moveTo(x: Double, y: Double): Shape = copy(pos.moveTo(x, y))

case class Circle(pos: Point, radius: Double) extends Shape:
  def move(dx: Double, dy: Double): Shape = copy(pos = pos.move(dx, dy))
  def moveTo(x: Double, y: Double): Shape = copy(pos.moveTo(x, y))
\end{CodeSmall}


\SubtaskSolved \code{def distanceTo(that: Point): Double = math.hypot(that.x - x, that.y - y)}

\SubtaskSolved \code{def distanceTo(that: Shape): Double = pos.distanceTo(that.pos)}.

\SubtaskSolved  
\begin{CodeSmall}
case class Point(x: Double, y: Double):
  def move(dx: Double, dy: Double): Point = Point(x + dx, y + dy)
  def moveTo(x: Double, y: Double): Point = Point(x, y)
  infix def distanceTo(that: Point): Double = math.hypot(that.x - x, that.y - y)

trait Shape:
  def pos: Point
  def move(dx: Double, dy: Double): Shape
  def moveTo(x: Double, y: Double): Shape
  infix def distanceTo(that: Shape): Double = pos.distanceTo(that.pos)

case class Rectangle(pos: Point, width: Double, height: Double) extends Shape:
  def move(dx: Double, dy: Double): Shape = copy(pos = pos.move(dx, dy))
  def moveTo(x: Double, y: Double): Shape = copy(pos.moveTo(x, y))

case class Circle(pos: Point, radius: Double) extends Shape:
  def move(dx: Double, dy: Double): Shape = copy(pos = pos.move(dx, dy))
  def moveTo(x: Double, y: Double): Shape = copy(pos.moveTo(x, y))
\end{CodeSmall}

\QUESTEND






% \WHAT{Regler för \code{override}, \code{private} och \code{final}.}

% \QUESTBEGIN

% \Task  \what~

% \Subtask \label{subtask:overriderules} Undersök överskuggningning av abstrakta, konkreta, privata och finala medlemmar genom att skriva in raderna nedan en i taget i REPL. Vilka rader ger felmeddelande? Varför? Vid felmeddelande: notera hur felmeddelandet lyder och ändra deklarationen av den felande medlemmen så att koden blir kompilerbar (eller om det är enda rimliga lösningen: ta bort den felaktiga medlemmen), innan du provar efterkommande rad.

% \begin{REPL}
% trait Super1 { def a: Int; def b = 42; private def c = "hemlis" }
% class Sub2 extends Super1 { def a = 43; def b = 43; def c = 43 }
% class Sub3 extends Super1 { def a = 43; override def b = 43 }
% class Sub4 extends Super1 { def a = 43; override def c = "43" }
% trait Super5 { final def a: Int; final def b = 42 }
% class Sub6 extends Super5 { override def a = 43; def b = 43 }
% class Sub7 extends Super5 { def a = 43; override def b = 43 }
% class Sub8 extends Super5 { def a = 43; override def c = "43" }
% trait Super9 { val a: Int; val b = 42; lazy val c: String = "lazy" }
% class Sub10 extends Super9 { override def a = 43; override val b = 43 }
% class Sub11 extends Super9 { val a = 43; override lazy val b = 43 }
% class Sub12 extends Super9 { val a = 43; override var b = 43 }
% class Sub13 extends Super9 { val a = 43; override lazy val c = "still lazy" }
% class SubSub extends Sub13 { override val a = 44}
% trait Super14 { var a: Int; var b = 42; var c: String }
% class Sub15 extends Super14 { def a = 43; override var b = 43; val c = "?" }
% \end{REPL}

% \Subtask Skapa instanser av klasserna \code{Sub3}, \code{Sub13} och \code{SubSub} från ovan deluppgift och undersök alla medlemmarnas värden för respektive instans. Förklara varför de har dessa värden.

% %\Subtask Läs igenom reglerna i kapitel  \ref{slideW07:overriderules} om vad som gäller vid arv och överskuggning av medlemmar. Gör några egna undersökningar i REPL som försöker bryta mot någon regel som inte testades i deluppgift \ref{subtask:overriderules}.

% \SOLUTION


% \TaskSolved \what


% \SubtaskSolved  2. Måste ha \code{override} framför \code{b} för att kunna ändra på metoden. \\
% 4. \code{c} är \code{private}, vilket betyder att den är gömd för subklasserna. Därför kan den inte överskuggas. Genom att ta bort \code{override} fungerar klassen. \\
% 5. En \code{final}-medlem måste ha ett bestämt värde. Kan lösas genom att tilldela \code{final a} ett värde eller ta bort \code{final}. \\
% 6. En \code{final}-medlem kan inte överskuggas, varken med eller utan \code{override}. Här får konflikterna tas bort.  \\
% 7. Se 6. \\
% 8. Eftersom \code{c} inte finns i \code{Super5} kan den inte överskuggas. Genom att ta bort \code{override} fungerar klassen. \\
% 10. Överskuggningen av \code{val} måste vara oföränderlig (immutable); detta är inte nödvändigtvis \code{def}. Löses genom att byta ut \code{def a} mot \code{val a} hos \code{Sub10}.  \\
% 11. Samma problem som i 10.; \code{lazy val} kan vara föränderlig. Löses genom att ta bort \code{lazy}. \\
% 12. Samma problem igen! \code{var} är föränderlig, vilket bryter mot typsäkerheten när man försöker överskugga en \code{val}. Löses genom att ändra \code{var} till \code{val}. \\
% 15.\code{def a = 43} och \code{val c = "?"} täcker inte allt som \code{var} kräver. Det behövs en setter för att kunna uppfylla kraven för överskuggning för en \code{var}. Dessutom finns det ingen anledning för en \code{val} att överskuggas; man kan ju ändra på den lite hur man vill!

% \SubtaskSolved  Sub3: a = 43, b = 43 eftersom medlemmen är överskuggad. c hittas inte eftersom den är \code{private}.

% Sub13: a = 43, b = 42, c = "still lazy" eftersom medlemmen överskuggas.

% SubSub: a = 44 eftersom medlemmen överskuggas, b = 42, c = "still lazy".

% \SubtaskSolved  -.


% \QUESTEND





%\clearpage





\AdvancedTasks %%%%%%%%%%%%%%%%%

% \WHAT{Använda \code{trait} eller \code{class}?}

% \QUESTBEGIN

% \Task \what~ I vilka sammanhang är det nödvändigt att använda en \code{trait} respektive en \code{class}? Läs här för fördjupning:\\  \href{http://www.artima.com/pins1ed/traits.html\#12.7}{http://www.artima.com/pins1ed/traits.html\#12.7}.


% \SOLUTION


% \TaskSolved \what~Man måste använda en klass om man behöver klassparametrar. Man måste använda en trait om man vill göra in-mixning med \code{with}. \\

%  \QUESTEND



\WHAT{Inmixning.}

\QUESTBEGIN

\Task \label{task:fyle} \what~   Man kan utvidga en klass med multipla traits med en kommaseparerad lista. På så sätt kan man fördela medlemmar i olika traits och återanvända gemensamma koddelar genom så kallad \textbf{inmixning}, så som nedan exempel visar.

En alternativ fågeltaxonomi, speciellt populär i Skåne, delar in alla fåglar i två specifika kategorier: Kråga respektive Ånka. Krågor kan flyga men inte simma, medan Ånkor kan simma och oftast även flyga. Fågel i generell, kollektiv bemärkelse kallas på gammal skånska för Fyle.%
\footnote{\href{http://www.klangfix.se/ordlista.htm}{www.klangfix.se/ordlista.htm}}

\begin{CodeSmall}
trait Fyle:
  val läte: String
  def väsnas: Unit = print(läte * 2)
  val ärSimkunnig: Boolean
  val ärFlygkunnig: Boolean

trait KanSimma       { val ärSimkunnig = true }
trait KanInteSimma   { val ärSimkunnig = false }
trait KanFlyga       { val ärFlygkunnig = true }
trait KanKanskeFlyga { val ärFlygkunnig = math.random() < 0.8 }

class Kråga extends Fyle, KanFlyga, KanInteSimma:
  val läte = "krax"

class Ånka extends Fyle, KanSimma, KanKanskeFlyga: 
  val läte = "kvack"
  override def väsnas = print(läte * 4)
\end{CodeSmall}

\Subtask En flitig ornitolog hittar 42 fåglar i en perfekt skog där alla fågelsorter är lika sannolika, representerat av vektorn \code{fyle} nedan. Skriv i REPL ett uttryck som undersöker hur många av dessa som är flygkunniga Ånkor, genom att använda metoderna \code{filter}, \code{isInstanceOf}, \code{ärFlygkunnig} och \code{size}.

\begin{REPL}
scala> val fyle =
         Vector.fill(42)(if math.random() > 0.5 then new Kråga else new Ånka)
scala> fyle.foreach(_.väsnas)
scala> val antalFlygånkor: Int = ???
\end{REPL}

\Subtask \label{subtask:fyle:sound} Om alla de fåglar som ornitologen hittade skulle väsnas exakt en gång var, hur många krax och hur många kvack skulle då höras? Använd metoderna \code{filter} och \code{size}, samt predikatet \code{ärSimkunnig} för att beräkna antalet krax respektive kvack.
\begin{REPL}
scala> val antalKrax: Int = ???
scala> val antalKvack: Int = ???
\end{REPL}

\SOLUTION


\TaskSolved \what


\SubtaskSolved
Det finns många olika sätt, några exempellösningar:
\begin{Code}
val antalFlygånkor: Int = 
  fyle.count(f => f.isInstanceOf[Ånka] && f.ärFlygkunnig)
\end{Code}

\begin{Code}
val antalFlygånkor: Int = 
  fyle.filter(f => f.isInstanceOf[Ånka] && f.ärFlygkunnig).size
\end{Code}

\begin{Code}
val antalFlygånkor: Int = 
  fyle.collect{case f: Ånka if f.ärFlygkunnig}.size
\end{Code}

\begin{Code}
val antalFlygånkor: Int = fyle.map(_ match
  case f: Ånka if f.ärFlygkunnig => 1
  case _ => 0
).sum
\end{Code}

\SubtaskSolved
\begin{Code}
val antalKrax: Int = fyle.filter(f => !f.ärSimkunnig).size * 2
val antalKvack: Int = fyle.filter(f => f.ärSimkunnig).size * 4
\end{Code}


\QUESTEND











\WHAT{Finala klasser.}

\QUESTBEGIN

\Task  \what~  Om man vill förhindra att man kan göra \code{extends} på en klass kan man göra den final genom att placera nyckelordet \code{final} före nyckelordet \code{class}.

\Subtask Eftersom klassificeringen av fåglar i uppgiften ovan i antingen Ånkor eller Krågor är fullständig och det inte finns några subtyper till varken Ånkor eller Krågor är det lämpligt att göra dessa finala. Ändra detta i din kod.

\Subtask Testa att ändå försöka göra en subklass \code{Simkråga extends Kråga}. Vad ger kompilatorn för felmeddelande om man försöker utvidga en final klass?


\SOLUTION


\TaskSolved \what


\SubtaskSolved  Sätt \code{final} framför \code{class} i klasserna.

\SubtaskSolved  error: illegal inheritance from final class Kråga.


\QUESTEND






\WHAT{Accessregler vid arv och nyckelordet \code{protected}.}

\QUESTBEGIN

\Task  \what~  Om en medlem i en supertyp är privat så kan man inte komma åt den i en subklass. Ibland vill man att subklassen ska kunna komma åt en medlem även om den ska vara otillgänglig i annan kod.

\begin{Code}
trait Super:
  private val minHemlis = 42
  protected val vårHemlis = 42

class Sub extends Super:
  def avslöja = minHemlis
  def kryptisk = vårHemlis * math.Pi

\end{Code}

\Subtask Vad blir felmeddelandet när klassen \code{Sub} försöker komma åt \code{minHemlis}?

\Subtask Deklarera \code{Sub} på nytt, men nu utan den förbjudna metoden \code{avslöja}. Vad blir felmeddelandet om du försöker komma åt \code{vårHemlis} via en instans av klassen \code{Sub}? Prova till exempel med \code{(new Sub).vårHemlis}

\Subtask Fungerar det att anropa metoden \code{kryptisk} på instanser av klassen \code{Sub}?

\SOLUTION


\TaskSolved \what


\SubtaskSolved  
\begin{REPL}
2 |  def avslöja = minHemlis
  |                ^^^^^^^^^
  |                Not found: minHemlis
\end{REPL}

\SubtaskSolved  
\begin{REPL}
scala> class Sub extends Super:
         def kryptisk = vårHemlis * math.Pi
scala> (new Sub).vårHemlis
-- Error:
1 |(new Sub).vårHemlis
  |^^^^^^^^^^^^^^^^^^^
  |value vårHemlis in trait Super cannot be accessed as a member of Sub.
  | Access to protected value vårHemlis not permitted because enclosing object 
  | is not a subclass of trait Super where target is defined
\end{REPL}

\SubtaskSolved  Ja.


\QUESTEND






\WHAT{Använding av \code{protected}.}

\QUESTBEGIN

\Task  \what~  Den flitige ornitologen från uppgift \ref{task:fyle} ska ringmärka alla 42 fåglar hen hittat i skogen. När hen ändå håller på bestämmer hen att även försöka ta reda på hur mycket oväsen som skapas av respektive fågelsort. För detta ändamål apterar den flitige ornitologen en Linuxdator på allt infångat fyle. Du ska hjälpa ornitologen att skriva programmet.

\Subtask Inför en \code{protected var räknaLäte} i traiten \code{Fyle} och skriv kod på lämpliga ställen för att räkna hur många läten som respektive fågelinstans yttrar.

\Subtask Inför en metod \code{antalLäten} som returnerar antalet krax respektive kvack som en viss fågel yttrat sedan dess skapelse.

\Subtask Varför inte använda \code{private} i stället for \code{protected}?

\Subtask Varför är det bra att göra räknar-variabeln oåtkomlig från ''utsidan''?



\SOLUTION


\TaskSolved \what


\SubtaskSolved  I Fyle:
\begin{Code}
protected var räknaLäte: Int = 0
def väsnas: Unit = { print(läte * 2); räknaLäte += 2 }
\end{Code}

I Ånka: \code| override def väsnas = { print(läte * 4); räknaLäte += 4 }|

\SubtaskSolved  \code{ def antalLäten: Int = räknaLäte }

\SubtaskSolved  Om en klass som representerar en fågel som skulle ge ifrån sig fler/färre läten än en vanlig \code{Fyle}, behöver \code{väsnas} ändras. Denna metod behöver tillgång till \code{räknaLäte}, vilken inte får vara \code{private}.

\SubtaskSolved  Räknar-variabeln ska inte kunna påverkas i någon annan del av programmet.


\QUESTEND





\WHAT{Inmixning av egenskaper.}

\QUESTBEGIN

\Task  \what~ Det visar sig att vår flitige ornitolog från uppgift \ref{task:fyle} på sidan \pageref{task:fyle} sov på en av föreläsningarna i zoologi när hen var nolla på Natfak, och därför helt missat fylekategorin \code{Pjodd}. Hjälp vår stackars ornitolog så att fylehierarkin nu även omfattar Pjoddar. En Pjodd kan flyga som en Kråga men den \code{ÄrLiten} medan en Kråga \code{ÄrStor}. En Pjodd kvittrar dubbelt så många gånger som en Ånka kvackar. En Pjodd \code{KanKanskeSimma} där simkunnighetssannolikheten är $0.2$. Låt ornitologen ånyo finna 42 slumpmässiga fåglar i skogen och filtrera fram lämpliga arter. Undersök sedan hur dessa väsnas, i likhet med deluppgift \ref{task:fyle}\ref{subtask:fyle:sound}.


\SOLUTION

\TaskSolved \what


\begin{Code}
trait Fyle:
  val läte: String
  def väsnas: Unit = { print(läte * 2); räknaLäte += 2 }
  protected var räknaLäte: Int = 0
  val ärSimkunnig: Boolean
  val ärFlygkunnig: Boolean
  val ärStor : Boolean
  def antalLäten: Int = räknaLäte

trait KanSimma { val ärSimkunnig = true }
trait KanInteSimma { val ärSimkunnig = false }
trait KanFlyga { val ärFlygkunnig = true }
trait KanKanskeFlyga { val ärFlygkunnig = math.random() < 0.8 }
trait KanKanskeSimma { val ärSimkunnig = math.random() < 0.2 }
trait ÄrStor { val ärStor = true }
trait ÄrLiten { val ärStor = false }

final class Kråga extends Fyle, KanFlyga, KanInteSimma, ÄrStor:
  val läte = "krax"

final class Ånka extends Fyle, KanSimma, KanKanskeFlyga, ÄrStor:
  val läte = "kvack"
  override def väsnas = { print(läte * 4); räknaLäte += 4 }

final class Pjodd extends Fyle, KanFlyga, KanKanskeSimma, ÄrLiten:
  val läte = "kvitter"
  override def väsnas = { print(läte * 8); räknaLäte += 8 }
\end{Code}

I REPL:
\begin{REPL}
val fyle = Vector.fill(42)(
  if math.random() < 0.33 then Kråga()
  else if math.random() < 0.5 then Ånka()
  else Pjodd()
)
fyle.filter(f => f.isInstanceOf[Kråga]).size * 2
fyle.filter(f => f.isInstanceOf[Ånka]).size * 4
fyle.filter(f => f.isInstanceOf[Pjodd]).size * 8
\end{REPL}

\QUESTEND





% \WHAT{Typtest och typkonvertering.}

% \QUESTBEGIN

% \Task  \what~I Scala kan man testa körtidstyp och samtidigt konvertera till en mer specifik typ på ett typsäkert sätt med hjälp av \emph{mönstermatchning} i \code{match}-uttryck som vi ska se i kommande övning \texttt{\ExeWeekTEN}. För att underlätta interoperabilitet med Java finns  Scala-metoderna \code{isInstanceOf} och \code{asInstanceOf}, som motsvarar hur typtest och typkonvertering görs i Java.\footnote{\code{isInstanceOf} och \code{asInstanceOf} används sällan i Scala eftersom \code{match} är kraftfullare och säkrare.}

% Gör nedan deklarationer.
% \begin{REPL}
% scala> trait A; trait B extends A; class C extends B; class D extends B
% scala> val (c, d) = (new C, new D)
% scala> val a: A = c
% scala> val b: B = d
% \end{REPL}

% \Subtask Rita en bild över vilka typer som ärver vilka.

% \Subtask Vilket resultat ger dessa typtester? Varför?
% \begin{REPL}
% scala> c.isInstanceOf[C]
% scala> c.isInstanceOf[D]
% scala> d.isInstanceOf[B]
% scala> c.isInstanceOf[A]
% scala> b.isInstanceOf[A]
% scala> b.isInstanceOf[D]
% scala> a.isInstanceOf[B]
% scala> c.isInstanceOf[AnyRef]
% scala> c.isInstanceOf[Any]
% scala> c.isInstanceOf[AnyVal]
% scala> c.isInstanceOf[Object]
% scala> 42.isInstanceOf[Object]
% scala> 42.isInstanceOf[Any]
% \end{REPL}

% \Subtask Vilka av dessa typkonverteringar ger felmeddelande? Vilket och varför?
% \begin{REPL}
% scala> c.asInstanceOf[B]
% scala> c.asInstanceOf[A]
% scala> d.asInstanceOf[C]
% scala> a.asInstanceOf[B]
% scala> a.asInstanceOf[C]
% scala> a.asInstanceOf[D]
% scala> a.asInstanceOf[E]
% scala> b.asInstanceOf[A]
% \end{REPL}



% \SOLUTION


% \TaskSolved \what


% \SubtaskSolved  B ärver A. C och D ärver B.

% \SubtaskSolved  1. True eftersom c är av typen C. \\
% 2. False eftersom c inte är av typen D. \\
% 3. True eftersom d är av typen D som är en subtyp av B. \\
% 4. True eftersom c är av typen C som är en subtyp av B, som i sin tur är en subtyp av A. \\
% 5. True eftersom b är av typen D, som är en subtyp av B, som i sin tur är en subtyp av A. \\
% 6. True eftersom b är av typen D. \\
% 7. True eftersom a är av typen C som är en subtyp av B. \\
% 8. True eftersom c är av typen C som är en subtyp av AnyRef. \\
% 9. True eftersom c är av typen C som är en subtyp av Any. \\
% 10. Error eftersom \code{isInstanceOf} inte kan använda sig av \code{AnyVal}.  \\
% 11. True eftersom c är av typen C som är en subtyp av Object (Object är java-representationen av AnyRef). \\
% 12. Error eftersom \code{isInstanceOf} inte kan testa om värdetyper (i detta fallet \code{42}) är referenstyper. \\
% 13. True eftersom \code{42} är av typen \code{Int} som är en subtyp av Any. \\

% \SubtaskSolved  3. Går inte eftersom c inte är av typen D, utan typen C. \\
% 6. Går inte eftersom a inte är av typen D, utan typen C. \\
% 7. Går inte eftersom typen E inte finns. \\


% \QUESTEND













% \WHAT{Saknad referens med \texttt{null} och bottentypen \texttt{Nothing}.}

% \QUESTBEGIN

% \Task  \what~ Hitta på en egen fördjupningsuppgift inspirerat av denna artikel på Stackoverflow: \url{http://stackoverflow.com/questions/16173477/usages-of-null-nothing-unit-in-scala}

% \SOLUTION


% \QUESTEND






\WHAT{Arvshierarki med matematiska tal.}

\QUESTBEGIN

\Task  \what~ Studera den djupa arvshierarkin i paketet \code{numbers} i koden på efterföljande sidor. Paketet  \code{numbers} modellerar olika sorters tal i matematiken, med syftet att erbjuda ett s.k. DSL \footnote{\url{https://en.wikipedia.org/wiki/Domain-specific_language}}, alltså ett specialspråk för en viss applikationsdomän\footnote{\url{https://stackoverflow.com/questions/49216312/what-is-dsl-in-scala}}, här: domänen matematiska tal.

Du kan ladda ner koden för \code{numbers} här: \\
\href{https://github.com/lunduniversity/introprog/blob/master/compendium/examples/numbers.scala}{github.com/lunduniversity/introprog/blob/master/compendium/examples/numbers.scala}
\\ Notera speciellt metoden \code{reduce} som reducerar ett tal till sin enklaste form. Metoden \code{reduce} överskuggas på lämpliga ställen med relevant reduktion.

\Subtask Rita en bild över typhierarkin, t.ex. som ett upp-och-nedvänt träd med bastypen  \code{Number} som rot.

\Subtask Skriv kod som använder de olika konkreta klasserna i \code{package numbers}. 
\begin{REPL}
scala> numbers.  // Tryck Tab
AbstractComplex   AbstractNatural    AbstractReal   Frac    Nat      Polar
AbstractInteger   AbstractRational   Complex        Integ   Number   Real

scala> numbers.Integ(12)
res0: numbers.Integ = Integ(12)

scala> import numbers.Syntax._
import numbers.Syntax._

scala> 42.j
res1: numbers.Complex = Complex(Real(0),Real(42))

scala> 42.42.j
res2: numbers.Complex = Complex(Real(0),Real(42.42))

\end{REPL}

\Subtask Ändra på metoden \code{+} i \code{trait Number} så att den blir abstrakt och implementera den i alla konkreta klasser.

\Subtask Implementera fler räknesätt och bygg vidare på koden så som du finner intressant.

\Subtask Gör så att metoden \code{reduce} i klassen \code{AbstractRational} använder algoritmen Greatest Common Divisor (GCD)\footnote{\url{https://sv.wikipedia.org/wiki/St\%C3\%B6rsta\_gemensamma\_delare}} så som beskrivs här: \\ \href{http://www.artima.com/pins1ed/functional-objects.html#6.8}{www.artima.com/pins1ed/functional-objects.html\#6.8} \\ så att täljare och nämnare blir så små som möjligt.

%\clearpage

\scalainputlisting[numbers=left, basicstyle=\ttfamily\fontsize{9.1}{12.2}\selectfont]{examples/numbers.scala}\SOLUTION


\QUESTEND


%!TEX encoding = UTF-8 Unicode
%!TEX root = ../exercises.tex

\ifPreSolution


\Exercise{\ExeWeekELEVEN}\label{exe:W11}

\TODO övningar på given using, extensionsmetoder, typklasser, Ordering etc.
\TODO flytta ordering till hit

\begin{Goals}
\item \TODO
\end{Goals}

\begin{Preparations}
\item \StudyTheory{11}
\end{Preparations}

\BasicTasks %%%%%%%%%%%%%%%%

\else

\ExerciseSolution{\ExeWeekELEVEN}

\BasicTasks %%%%%%%%%%%

\fi


\WHAT{Användning av givna värden.}

\QUESTBEGIN

\Task  \what~  \TODO

\Subtask \TODO



\SOLUTION


\TaskSolved \what

\SubtaskSolved  \TODO


\QUESTEND







\input{modules/w12-sorting-exercise.tex}
%!TEX encoding = UTF-8 Unicode
%!TEX root = ../exercises.tex

\ifPreSolution

\Exercise{\ExeWeekTHIRTEEN}\label{exe:W13}
\begin{Goals}
\item Kunna skriva tentamenslika program med papper, penna och snabbreferens som enda hjälpmedel.
\item Förbereda projektredovisningen.
\item Kunna skapa dokumentation med \code{scaladoc} och \code{javadoc}.
\item Kunna skapa jar-filer.
\end{Goals}

% \begin{Preparations}
% \item \StudyTheory{13}
% \end{Preparations}

\else

\ExerciseSolution{\ExeWeekTHIRTEEN}

\fi


\subsection{Förberedelse inför examination}




\WHAT{Gör en extenta.} %%%%%%%%%%%%%%%%%%%%%%%%%%%%%%%%%%%%%%%%%%%%%%%%%%%%%%%%

\QUESTBEGIN

\Task \what~\TODO

\SOLUTION

\TaskSolved \what~\TODO

\QUESTEND




\WHAT{Förbered din projektredovisning.} %%%%%%%%%%%%%%%%%%%%%%%%%%%%%%%%%%%%%%%

\QUESTBEGIN

\Task \what~\TODO

\SOLUTION

\TaskSolved \what~\TODO

\QUESTEND



\WHAT{Skapa dokumentation.} %%%%%%%%%%%%%%%%%%%%%%%%%%%%%%%%%%%%%%%%%%%%%%%%%%%

\QUESTBEGIN

\Task  \what~

\Subtask \TODO kör nedan kommando i terminalen:

\begin{REPL}
> scaladoc paket.scala
> ls
> firefox index.html   # eller öppna index.html i valfri webbläsare
\end{REPL}

Vad händer?

\Subtask Lägg till några fler metoder i något av objekten i filen \code{paket.scala} och lägg även till några dokumentationskommentarer. Kompilera om och kör. Generera om dokumentationen.

\begin{verbatim}
//... ändra i filen paket.scala

/** min paketdokumentationskommentar p2 */
package p2 {
  /** min paketdokumentationskommentar p21 */
  package p21 {
    /** ett hälsningsobjekt */
    object hello {
      /** en hälsningsmetod i p2.p21 */
      def hello = println("Hej paket p2.p21!")

      /** en metod som skriver ut tiden */
      def date = println(new java.util.Date)
    }
  }
}

\end{verbatim}

\begin{REPL}
> gedit paket.scala
> scalac paket.scala
> jar cvf mittpaket.jar gurka
> scala -cp mittpaket.jar
scala> gurka.tomat.banan.p2.p21.hello.date
scala> :q
> scaladoc paket.scala
> firefox index.html
\end{REPL}

\SOLUTION


\TaskSolved \what

\SubtaskSolved  -

\SubtaskSolved  -

\QUESTEND



\WHAT{Repetera övningar och laborationer.} %%%%%%%%%%%%%%%%%%%%%%%%%%%%%%%%%%%%

\QUESTBEGIN

\Task \what~\TODO

\SOLUTION

\TaskSolved \what~\TODO

\QUESTEND

%!TEX encoding = UTF-8 Unicode
%!TEX root = ../exercises.tex

\ifPreSolution

\Exercise{\ExeWeekFOURTEEN}\label{exe:W14}

\begin{Goals}
\item Känna till vad en tråd är och kunna förklara begreppet jämlöpande exekvering.
\item Känna till vad metoderna \code{run} och \code{start} gör i klassen \code{Thread}.
\item Kunna skapa och starta en tråd med överskuggad \code{run}-metod.
\item Kunna skapa ett enkelt program som från två trådar tävlar om att uppdatera en variabel och förklara varför beteendet kan bli oförutsägbart.
\item Kunna använda en \code{Future} för att köra igång flera parallella beräkningar.
\item Kunna registrera en callback på en \code{Future} med metoden \code{onComplete}.
%\item Känna till att webbsidor beskrivs av HTML-kod och kunna skapa en minimal webbsida.
%\item Kunna ladda ner en webbsida med \code{scala.io.Source.fromURL}.
\end{Goals}

% \begin{Preparations}
% \item \StudyTheory{14}
% \end{Preparations}

\else

\ExerciseSolution{\ExeWeekFOURTEEN}

\fi


\subsection{Frivilliga extrauppgifter}



\WHAT{Trådar.}

\QUESTBEGIN

\Task  \what~   Klassen \code{java.lang.Thread} används för att skapa  \textbf{trådar} med jämlöpande exekvering \Eng{concurrent execution}. På så sätt kan man få olika koddelar att köra samtidigt.

Klassen \code{Thread} definierar en tom \code{run}-metod. Vill man att tråden ska göra något vettigt får man överskugga \code{run} med det man vill ska göras.

En tråd körs igång med metoden \code{start} och då anropas automatiskt \code{run}-metoden och tråden exekverar koden i \code{run} jämlöpande med övriga trådar. Om man anropar \code{run} direkt blir det \emph{inte} jämlöpande exekvering.

\Subtask Skapa en tråd som gör något som tar lite tid och kör med \code{run} resp. \code{start}.
\begin{REPL}
def zzz = { print("zzzzzz"); Thread.sleep(5000); println(" VAKEN!")}
zzz
val t2 = new Thread{ override def run = zzz }
t2.run
t2.run; println("Gomorron!")
t2.start; println("Gomorron!")
t2.start
\end{REPL}

\Subtask Vad händer om man anropar \code{start} mer än en gång på samma tråd?

\Subtask Skapa två trådar med överskuggade \code{run}-metoder och kör igång dem samtidigt enligt nedan. Vilken ordning skrivs hälsningarna ut efter rad 3 resp. rad 4 nedan? Förklara vad som händer.
\begin{REPL}
val g = new Thread{ override def run = for (i <- 1 to 100) print("Gurka ") }
val t = new Thread{ override def run = for (i <- 1 to 100) print("Tomat ") }
g.run; t.run
g.start; t.start
\end{REPL}

\Subtask Använd \code{Thread.sleep} enligt nedan. Är beteendet helt förutsägbart (deterministiskt)? Förklara vad som händer. Du kan (om du kör Linux) avbryta REPL med Ctrl+C%
\footnote{\href{http://stackoverflow.com/questions/6248884/can-i-stop-the-execution-of-an-infinite-loop-in-scala-repl}{stackoverflow.com/questions/6248884/can-i-stop-the-execution-of-an-infinite-loop-in-scala-repl}}.
\begin{REPL}
def ibland(block: => Unit) = new Thread {
  override def run = while(true) { block; Thread.sleep(600) }
}.start
ibland(print("zzz ")); ibland(print("snark ")); ibland(println("hej!"))
\end{REPL}


\SOLUTION


\TaskSolved \what
     %%%TODO number  1 %%%starts with: \emph{Trådar.}  %%%

\SubtaskSolved   -

\SubtaskSolved  \code {java.lang.IllegalThreadStateException}. Det går inte att starta en tråd mer än en gång. Tråden kan därför inte startas om när den redan har exekverats.

\SubtaskSolved   När \code {start} anropas exekveras koden i \code{run} parallellt. Därför skrivs \code{Gurka} och \code{Tomat} ut omlöpande. Om istället \code{run} anropas direkt blir det inte jämnlöpande exekvering och \code{Gurka} skrivs ut 100 gånger, sedan skrivs \code{Tomat} ut 100 gånger.

\SubtaskSolved   \code{Thread.sleep} pausar inte tråden i exakt den tiden som angets. Alltså kommer det skrivas ut \code{zzz snark hej!} i de flesta fall, men det är inte garanterat.



\QUESTEND






\WHAT{Jämlöpande variabeluppdatering.}

\QUESTBEGIN

\Task \label{task:racecondition} \what~   Skriv klasserna \code{Bank} och \code{Kund} i en editor och klistra sedan in koden i REPL.

\begin{Code}
class Bank {
  private var saldo = 0;
  def visaSaldo: Unit = println("saldo: " + saldo)
  def sättIn: Unit = { saldo += 1 }
  def taUt: Unit   = { saldo -= 1 }
}

class Kund(bank: Bank) {
  def slösaSpara = {bank.taUt; Thread.sleep(1); bank.sättIn}
}
\end{Code}

\Subtask Använd funktionen \code{ibland} från föregående uppgift och kör nedan rader i REPL. Resultatet av jämlöpande variabeluppdatering blir här heltokigt och leder till mycket upprörda bankkunder och -ägare. Förklara vad som händer.

\begin{REPL}
val bank = new Bank
bank.visaSaldo
bank.sättIn
bank.visaSaldo
bank.taUt
bank.visaSaldo

val bamse = new Kund(bank)
val skutt = new Kund(bank)

bamse.slösaSpara
skutt.slösaSpara
bank.visaSaldo

def ofta(block: => Unit) = new Thread {
  override def run = while(true) { block; Thread.sleep(1) }
}.start

ofta(bamse.slösaSpara); ofta(skutt.slösaSpara)

ibland(bank.visaSaldo)
\end{REPL}


\SOLUTION


\TaskSolved \what
     %%%TODO number  2 %%%starts with: \emph{Jämlöpande variabeluppdat%%%

\SubtaskSolved  I \code{slösaSpara} hämtas saldot, ändras och placeras tillbaka i minnet -  fördröjs -  upprepas. Om \code{bamse} blir klar med att ladda, ändra och lagra innan skutt gör detsamma med den muterbara variablen hade det inte varit perfekt. Problemet ligger i  när en tråd laddar och innan den kan lagra det förändrade värdet laddar den andra tråden samma värde. Bara en av dessa trådar vinner racet och får lagra sitt ändrade tal. \code{skutt} och \code{bamse} blir alltså upprörda för att inte alla dess uttag och insättningar registreras.


\QUESTEND






\WHAT{Trådsäkra \code{AtomicInteger}.}

\QUESTBEGIN

\Task  \what~  Det finns stöd i JVM för att åstadkomma uppdateringar som inte kan avbrytas av andra trådar under pågånde minnesskrivning. En operation som inte kan avbrytas kallas \textbf{atomär} \Eng{atomic}. Studera dokumentationen för \code{AtomicInteger}\footnote{\href{https://docs.oracle.com/javase/8/docs/api/java/util/concurrent/atomic/AtomicInteger.html}{docs.oracle.com/javase/8/docs/api/java/util/concurrent/atomic/AtomicInteger.html}} och prova nedan kod. Förklara vad som händer.

Använd funktionerna \code{ofta} och \code{ibland} från tidigare uppgifter.
\begin{Code}
class SäkerBank {
  import java.util.concurrent.atomic.AtomicInteger
  private var saldo = new AtomicInteger
  def visaSaldo: Unit = println(s"saldo: ${saldo.get}")
  def sättIn: Unit = { saldo.incrementAndGet }
  def taUt: Unit   = { saldo.decrementAndGet }
}

class SäkerKund(bank: SäkerBank) {
  def slösaSpara = {bank.taUt; Thread.sleep(1); bank.sättIn}
}
\end{Code}
\begin{REPL}
val säkerBank = new SäkerBank
val farmor = new SäkerKund(säkerBank)
val vargen = new SäkerKund(säkerBank)

ofta(farmor.slösaSpara); ofta(vargen.slösaSpara)

ibland(säkerBank.visaSaldo)
\end{REPL}





\SOLUTION


\TaskSolved \what
     %%%TODO number  3 %%%starts with: \emph{Jämlöpande exekvering med%%%

Nu är \code{farmor} garanterad att kunna ladda saldot, ta ut pengar/ändra och lagra innan \code{vargen} kan överskriva resultatet. I \code{slösaSpara} pausas tråden i en millisekund så \code{vargen} kan fortfarande ta ut pengar innan \code{farmor} hinner sätta in pengar igen. Dock kommer alla uttag och insättningar registreras eftersom operationerna är atomära.


\QUESTEND






\WHAT{Jämlöpande exekvering med \code{scala.concurrent.Future}.}

\QUESTBEGIN

\Task \label{task:future} \what~   Att skapa och hålla reda på trådar kan bli ganska omständligt och knepigt att få rätt på.
Med hjälp av \code{scala.concurrent.Future} kan man på ett enklare sätta skapa jämlöpande exekvering.

\begin{Background}
Med en \code{Future} skapas jämlöpande exekvering som ''under huven'' använder ett ramverk som heter Akka\footnote{\url{http://akka.io/}}, skrivet i Scala och Java. Akka erbjuder automatisk  multitrådning med s.k. trådpooler och möjliggör avancerad parallellprogrammering på en hög  abstraktionsnivå, där man själv slipper skapa instanser av klassen \code{Thread}. I stället kan man helt enkelt placera sin kod inramad med \code|Future{ "körs parallellt" }| efter att man importerat det som behövs.
\end{Background}

\Subtask För att skapa jämlöpande exekvering med \code{Future} behöver man först göra import enligt nedan; då skapas ett exekveringssammanhang med trådpooler redo för användning. Starta om REPL och studera felmeddelandet efter rad 1 nedan. Importera därefter enligt nedan. Vad har \code{f} för typ?
\begin{REPL}
scala> concurrent.Future { Thread.sleep(1000); println("En sekund senare!") }
scala> import scala.concurrent._
scala> import ExecutionContext.Implicits.global
scala> val f = Future { Thread.sleep(1000); println("En sekund senare!") }
\end{REPL}

\Subtask Skapa en procedur \code{printLater} enligt nedan som skriver ut argumentet efter slumpmässig tid. Förklara vad som händer nedan.
\begin{REPL}
scala> def printLater(a: Any): Unit =
         Future { Thread.sleep((math.random * 10000).toInt); print(a + " ") }
scala> (1 to 42).foreach(i => printLater(i)); println("alla är igång!")
\end{REPL}

\Subtask Skapa enligt nedan en \code{Future} som räknar ut hur många siffror det är i ett väldigt stort tal. Med \code{onComplete} kan man ange vad som ska göras när den tunga beräkningen är färdig; detta kallas att ''registrera en callback''. Vilken returtyp har \code{big}? Hur många siffror har det stora talet? Vad har \code{r} för typ? Justera argumentet till \code{big} om du inte orkar vänta på resultatet...

\begin{REPL}
scala> BigInt(10).pow(100)
scala> BigInt(10).pow(100).toString.size
scala> def big(n: Int) = Future { BigInt(n).pow(n).toString.size }
scala> big(1234567).onComplete{r => println(r + " siffror") }
\end{REPL}

\Subtask Den stora vinsten med \code{Future} är att man kan köra vidare under tiden, varför anropet av \code{Future} kallas \textbf{icke-blockerande} \Eng{non-blocking}. Det händer ibland att man ändå vill blockera exekveringen i väntan på ett resultat. Man kan då använda objektet \code{scala.concurrent.Await} och dess metod \code{result} enligt nedan. Använd \code{big} från föregående uppgift och gör en blockerande väntan på resultatet enligt nedan. Vad händer? Vad händer om du väntar för kort tid?

\begin{REPL}
scala> import scala.concurrent.duration._
scala> Await.result(big(1234567), 20.seconds)
\end{REPL}



\SOLUTION


\TaskSolved \what
     %%%TODO number  4 %%%starts with: TODO  %%%%%%%%%%%%%%%%%%%\Advan%%%

\SubtaskSolved  error: Cannot find an implicit ExecutionContext. Future behöver en ExecutionContext för att kunna köras. \code{f} är av typen Future[Unit].

\SubtaskSolved  Funktionen \code{printLater} har en Future, vilket innebär att när både \code{printLater} och \code{println} anropas i foreach-loopen exekveras de jämnlöpande. Eftersom det tar längre tid att starta upp en Future för datorn är \code{println} snabbare och skriver ut att alla är igång först. Sedan skrivs siffrorna från 1 - 42 ut med oregelbundna mellanrum eftersom tråden pausas olika länge.

\SubtaskSolved  \code{big} är en Future[Int]. Det stora talet har 7 520 383 siffror. \code{r} är av typen Try[Int] (se dokumentationen för Future om du är osäker)

\SubtaskSolved  Eftersom exekveringen blockas tills den har fått ett resultat går det inte att fortsätta skriva i REPL medan uträkningen pågår. Väntar man för kort tid får man ett TimeOutException och uträkningen avbryts.


\QUESTEND






\WHAT{Använda \code{Future} för att göra flera saker samtidigt.}

\QUESTBEGIN

\Task  \what~
I denna uppgift ska du ladda ner webbsidor parallellt med hjälp av \code{Future}, så att en nedladdning kan avslutas under tiden en annan dröjer.

\Subtask Koden för en minimal webbsida ser ut som nedan. Du kan beskåda sidan här: \url{http://fileadmin.cs.lth.se/pgk/mini.html} eller skriva in nedan kod i en fil som heter något som slutar på \texttt{.html} och öppna filen i din webbläsare.

\begin{verbatim}
<!DOCTYPE html>
<html>
<body>
HELLO WORLD!
</body>
</html>
\end{verbatim}

\Subtask För att simulera slöa webbservrar kan man ladda ner en sida via sajten \texttt{http://deelay.me/}. Ladda ner ovan sida med 2 sekunders fördröjning:\\
\url{http://deelay.me/2000/http://fileadmin.cs.lth.se/pgk/mini.html}

\Subtask Man kan ladda ner webbsidor med \code{scala.io.Source}. Vad händer nedan? Försök, med ledning av hur \code{delay} beräknas, uppskatta hur lång tid du måste vänta i medeltal, i bästa fall, respektive värsta fall, innan du kan se första webbsidan i vektorn \code{laddningar} nedan?

\begin{REPL}
scala> def ladda(url: String) = scala.io.Source.fromURL(url).getLines.toVector
scala> def slöladda(url: String) = {
         val delay = (math.random * 1000 + 2000).toInt
         val delaySite = s"http://deelay.me/$delay/"
         ladda(delaySite+url)
      }
scala> ladda("http://fileadmin.cs.lth.se/pgk/mini.html")
scala> def seg = slöladda("http://fileadmin.cs.lth.se/pgk/mini.html")
scala> val laddningar = Vector.fill(10)(seg)
scala> laddningar(0)
\end{REPL}

\Subtask Innan vi kan köra igång en \code{Future} så måste vi, som visats i uppgift \ref{task:future} importera den underliggande exekveringsmiljön som är redo att parallelisera ditt program i trådar utan att du själv måste skapa dem. Vad händer nedan?
\begin{REPL}
scala> import scala.concurrent._
scala> import ExecutionContext.Implicits.global
scala> val f = Future{ seg }
scala> f   // kolla om den är klar annars prova igen senare
scala> f
\end{REPL}

\Subtask Ladda indata utan att blockera \Eng{non-blocking input}. Förklara vad som händer nedan.
\begin{REPL}
scala> val nonblock = Future{ Vector.fill(10)(seg) }
scala> nonblock   // kolla igen senare om ej klar
scala> nonblock
\end{REPL}

\Subtask Ladda indata separat i olika parallella trådar. Förklara vad som händer nedan. Kör uttrycket på rad 3 nedan upprepade gånger i snabb följd efter varandra med pil-upp+Enter i REPL.
\begin{REPL}
scala> val para = Vector.fill(10)(Future{ seg })
scala> para
scala> para.map(_.isCompleted)
scala> para.map(_.isCompleted) // studera hur de blir färdiga en efter en
scala> para(0)
\end{REPL}

\Subtask Registrera en callback med metoden \code{onComplete}. Förklara vad som händer nedan.

\begin{REPL}
scala> val action = Vector.fill(10)(Future{ seg })
scala> action(0).onComplete(xs => println(s"ready:$xs"))
scala> // vänta tills laddning på plats 0 är klar
\end{REPL}

\Subtask Registrera en callback för felhantering i händelse av undantag med metoden \code{onFailure}. Förklara vad som händer nedan.
\begin{REPL}
scala> def lycka  = { Thread.sleep(3000); println(":)") }
scala> def olycka = { Thread.sleep(3000); 42 / 0; lycka }
scala> Future{ lycka  }.onFailure{ case e => println(s":( $e") }
scala> Future{ olycka }.onFailure{ case e => println(s":( $e") }
\end{REPL}



\SOLUTION


\TaskSolved \what
     %%%TODO number  5 %%%starts with: Sök upp och studera dokumentati%%%

\SubtaskSolved  -

\SubtaskSolved  -

\SubtaskSolved  Varje sida fördröjs med mellan 2 upp till 3 sekunder (2000-3000 millisekunder). Så i medeltal tar det 2.5 sekunder för varje sida att laddas. Vektorn måste fyllas innan exekveringen kan fortsätta. Därför laddas alla 10 stycken sidor in innan man kan se första websidan. Det tar därför i medeltal 2.5 x 10 = 25 sekunder.

\SubtaskSolved  \code{f} ger en Vektor fylld med strängar där varje element ges av en rad på hemsidan. Då \code{f} körs i bakgrunden kan programmet fortlöpa medan innehållet räknas ut. Du kan därför skriva \code{f} i REPL:n men det är inte säkert att proccessen är klar och det slutgilltiga resultatet visas.

\SubtaskSolved  Samma som ovan, förutom att det blir en vektor där varje element är i sig en vektor med strängar.

\SubtaskSolved  Laddar in datan parallelt så nedladdingen sker samtidigt, men det går olika snabbt pga metoden seg.

\SubtaskSolved  Eftersom datan laddas i parallella trådar utan blockering blir de inte klara i ordning, utan i den ordningen tråden körs klart. Till slut blir alla klara och resultatet visar en vektor med \code{true} värden.

\SubtaskSolved  Metoden \code{lycka} är väldefinerad och kastar därför inga undantag. Den skriver alltid ut \code{:)}. Metoden \code{olycka} är inte väldefinerad då division med 0 ger \code{java.lang.ArithmeticException}. Detta fångas upp vid callbacken och det skrivs ut \code{:(} samt det specifierade undantaget.

\ExtraTasks %%%%%%%%%%%%


\QUESTEND






\WHAT{}

\QUESTBEGIN

\Task  \what~ Räkna ut stora primtal parallellt genom att använda nedan funktioner. Implementera \code{isPrime} enligt pseudokod från den engelska wikipediasidan om primtalstest\footnote{\href{https://en.wikipedia.org/wiki/Primality_test}{en.wikipedia.org/wiki/Primality\_test}} med den s.k. ''naiva algoritmen''.  Räkna ut 10 st slumpvisa primtal med 16 siffror vardera. Gör beräkningarna parallellt med hjälp av \code{Future}.

\begin{Code}
def isPrime(n: BigInt): Boolean = ???

def nextPrime(start: BigInt): BigInt = {
  var i = start
  while (!isPrime(i)) { i += 1 }
  i
}

def randomBigInt(nDigits: Int): BigInt = {
   def rndChar = ('0' + (math.random * 10).toInt).toChar
   val str = Array.fill(nDigits)(rndChar).mkString
   BigInt(str)
}
\end{Code}

\SOLUTION


\TaskSolved \what
  %%%TODO number  6 %%%

\begin{Code}
def isPrime(n: BigInt): Boolean = n match {
  case _ if (n <= 1) => false
  case _ if (n <= 3) => true
  case _ if n % 2 == 0 || n % 3 == 0 => false
  case _ =>
    var i = BigInt(5)
    while (i * i < n) {
      if (n % i == 0 || n % (i + 2) == 0) false
      i += 6
    }
    true
}

import scala.concurrent._
import ExecutionContext.Implicits.global

val primes = Vector.fill(10)(Future{nextPrime(randomBigInt(16))})
primes.foreach(_.onSuccess{case i => println(i)})
\end{Code}


\QUESTEND






\WHAT{Svara på teorifrågor.}

\QUESTBEGIN

\Task  \what~\Pen

\Subtask Vad är en tråd?

\Subtask Hur skapar man en tråd med klassen \code{Thread}?

\Subtask Hur startar man en tråd?

\Subtask Vilka problem kan man råka ut för om man uppdaterar samma resurs i flera olika trådar?

\Subtask Vad innbär det att kod är \emph{trådsäker}?

\Subtask Nämn några fördelar med att använda Future jämfört med att använda trådar direkt.


\SOLUTION


\TaskSolved \what
 %%%TODO number  7 %%%

\SubtaskSolved  Stackoverflow ger följande förklaring:

A thread is an independent set of values for the processor registers (for a single core). Since this includes the Instruction Pointer (aka Program Counter), it controls what executes in what order. It also includes the Stack Pointer, which had better point to a unique area of memory for each thread or else they will interfere with each other.

\SubtaskSolved

\begin{Code}
val thread = new Thread(new Runnable{
	def run(){println(''Det här är en tråd'')}
})
\end{Code}

\SubtaskSolved  \code{thread.start}

\SubtaskSolved  Det kan bli kapplöpning(race conditions) om vilken tråds resurser blir sparade. Vilket leder till att de andra trådarnas ändringar blir ignorerade.

\SubtaskSolved  Trådsäkerhet innebär att flera trådar kan köras parallellt utan felaktigheter i resultatet. Exempelvis får man vara väldigt försiktig om man vill ha en muterbar variabel som alla trådar ska ändra samtidigt.

\SubtaskSolved  Till exempel slipper man skapa instanser av klassen Thread eftersom man kan placera koden i en Future istället. Den löser även mycket under huven för kodaren.


\QUESTEND






\WHAT{Klasser med atomär uppdatering.}

\QUESTBEGIN

\Task  \what~ Läs om och testa klasserna AtomicBoolean, AtomicDouble och AtomicReference för atomär uppdatering i paketet \\ \code{java.util.concurrent.atomic}.

Använd några av dessa tillsammans med \code{scala.concurrent.Future}.


\SOLUTION

\TaskSolved --

\QUESTEND





\WHAT{Skapa din egen multitrådade webbserver.}

\QUESTBEGIN

\Task  \what~

\Subtask Skriv in\footnote{Eller ladda ner här: \href{https://github.com/lunduniversity/introprog/blob/master/compendium/examples/simple-web-server/webserver.scala}{github.com/lunduniversity/introprog/blob/master/compendium/examples/simple-web-server/webserver.scala}} nedan kod i en editor och spara i en fil med namn \texttt{webserver.scala} och kompilera och kör med \texttt{scala webserver.start} och beskriv vad som händer när du med din webbläsare surfar till adressen: \\ \url{http://localhost:8089/abbasillen}

\scalainputlisting[numbers=left,basicstyle=\ttfamily\fontsize{11}{12}\selectfont]{examples/simple-web-server/webserver.scala}

\Subtask Du ska nu skapa en webbserver som gör något lite mer intressant. Den ska svara med det 13:e Fibonacci-talet\footnote{\href{https://sv.wikipedia.org/wiki/Fibonaccital}{https://sv.wikipedia.org/wiki/Fibonaccital}} om du surfar till \url{http://localhost:8089/fib/13}.
Spara din webbserver från föregående deluppgift under det nya namnet \texttt{fibserver.scala} och använd koden nedan och lägg till och ändra så att din server kan svara med Fibonaccital. Vi börjar med att räkna ut Fibonaccital i funktionen \code{compute.fib} nedan på ett onödigt processorkrävande sätt med exponentiell tidskomplexitet så att webbservern verkligen får jobba, för att i senare deluppgifter implementera \code{compute.fib} med linjär tidskomplexitet och därmed undvika onödig planetuppvärmning.
\begin{CodeSmall}
  object compute {
    def fib(n: BigInt): BigInt = {
      if (n < 0) 0 else
      if (n == 1 || n == 2) 1
      else fib(n - 1) + fib(n -2)
    }
  }

  def fibResponse(num: String) = Try { num.toInt } match {
    case Success(n) => html.page(s"fib($n) == " + compute.fib(n))
    case Failure(e) => html.page(s"FEL $e: skriv heltal, inte $num")
  }

  def errorResponse(uri:String) = html.page("FATTAR NOLL: " + uri)

  def handleRequest(cmd: String, uri: String, socket: Socket): Unit = {
    val os = socket.getOutputStream
    val parts = uri.split('/').drop(1) // skip initial slash
    val response: String = (parts.head, parts.tail) match {
      case (head, Array(num)) => fibResponse(num)
      case _                  => errorResponse(uri)
    }
    os.write(html.header(response.size).getBytes("UTF-8"))
    os.write(response.getBytes("UTF-8"))
    os.close
    socket.close
  }
\end{CodeSmall}
Kör i terminalen med \texttt{scala fibserver.start} och beskriv vad som händer i din webbläsare när du surfar till servern.


%%%\textbf{KOD TILL FACIT:}
%%%\scalainputlisting[numbers=left,basicstyle=\ttfamily\fontsize{11}{12}\selectfont]{examples/simple-web-server/fibserver.scala}


\Subtask Surfa efter flera stora Fibonacci-tal samtidigt i olika flikar i din browser. Hur märks det att servern bara kör i en enda tråd?

\Subtask Gör din server multitrådad med hjälp av den nya server-loopen nedan.

\begin{CodeSmall}
import scala.concurrent._
import ExecutionContext.Implicits.global

  def serverLoop(server: ServerSocket): Unit = {
    println(s"http://localhost:${server.getLocalPort}/hej")
		while (true) {
  		Try {
  		  var socket = server.accept  // blocks thread until connect
	  	  val scan = new Scanner(socket.getInputStream, "UTF-8")
		    val (cmd, uri) = (scan.next, scan.next)
			  println(s"Request: $cmd $uri")
		    Future { handleRequest(cmd, uri, socket) }.onFailure {
		      case e => println(s"Reqest failed: $e")
		    }
		  }.recover{ case e: Throwable => s"Connection failed: $e" }
		}
  }
\end{CodeSmall}

\Subtask Surfa efter flera stora Fibonacci-tal samtidigt i olika flikar i din browser. Hur märks det att servern är multitrådad?


\Subtask Det är onödigt att räkna ut samma Fibonacci-tal flera gånger. Med hjälp av en cache i form av en föränderlig \code{Map} kan du spara undan redan uträknade värden. Det funkar dock inte med en vanlig \code{scala.collection.mutable.Map} i vår multitrådade webbserver, eftersom den inte är \textbf{trådsäker} \Eng{thread-safe}. Med trådosäkra föränderliga datastrukturer blir det samma besvärliga beteende som i uppgift \ref{task:racecondition}.

Du ska i stället använda \code{java.util.concurrent.ConcurrentHashMap}. Sök upp  dokumentationen för \code{ConcurrentHashMap} och försök förstå koden nedan. Hur fungerar metoderna \code{containsKey}, \code{put} och \code{get}?
\begin{Code}
object compute {
  import java.util.concurrent.ConcurrentHashMap
  val memcache = new ConcurrentHashMap[BigInt, BigInt]

  def fib(n: BigInt): BigInt =
    if (memcache.containsKey(n)) {
      println("CACHE HIT!!! no need to compute: " + n)
      memcache.get(n)
    } else {
      println("cache miss :( must compute fib:  " + n)
      val f = fastFib(n)
      memcache.put(n, f)
      f
    }

  private def fastFib(n: BigInt): BigInt = {
    if (n < 0) 0 else
    if (n == 1 || n == 2) 1
    else fib(n - 1) + fib(n -2)
  }
}
\end{Code}

\Subtask Använd ovan \code{fib}-objekt i en ny version av din webserver. Spara den i en ny kodfil med namnet \texttt{fibserver-memcached.scala}. Undersök hur snabbt det går med stora Fibonaccital med den nya varianten. Hur stora tal kan du räkna ut? Kan servern fortsätta efter överflödad stack? Förklara varför.

\Subtask Nu när vi kan få väldigt stora Fibonacci-tal kan det vara användbart att stoppa in radbrytningar på webbsidan. Html-taggen \texttt{</br>} ger en radbrytning.
\begin{Code}
  def insertBreak(s: String, n: Int = 80): String = {
    if (s.size < n) s
    else s.take(n) + "</br>" + insertBreak(s.drop(n),n)
  }
\end{Code}
Använd den rekursiva funktionen ovan för att pilla in radbrytningstaggar på var $n$:te position i långa strängar. Testa hur det ser ut på webbsidan med ovan funktion när din server svarar med väldigt stora tal.

\Subtask Vi ska nu använda det större heap-minnet i stället för stack-minnet och därmed inte begränsas av stackens max-storlek. Skriv om \code{fastFib} så att den använder en \code{while}-sats i stället för ett rekursivt anrop. Denna uppgift är ganska klurig, men om du kör fast kan du snegla i lösningarna i Appendix för inspiration.

Hur stora tal klarar din server nu? Vad händer med servern när minnet tar slut? Hur kan du skydda servern så att den inte kan hänga sig?

\SOLUTION


\TaskSolved \what
 %%%TODO number  9 %%%

\SubtaskSolved  \code{abbasillen} skrivs ut baklänges till \code{nellisabba}.

\SubtaskSolved

\SubtaskSolved

\SubtaskSolved

\SubtaskSolved

\SubtaskSolved

\SubtaskSolved

\SubtaskSolved

\SubtaskSolved

Lösningsförslag:
\scalainputlisting[numbers=left,basicstyle=\ttfamily\fontsize{11}{12}\selectfont]{examples/simple-web-server/fibserver-threaded-memcached-while.scala}


\QUESTEND






\WHAT{}

\QUESTBEGIN

\Task  \what~ Utöka din server med fler beräkningsintensiva funktioner. Exempelvis primtalsberäkningar eller beräkningar av valfritt antal decimaler av $\pi$ eller $e$. Utnyttja gärna det du lärt dig i  matematiken om summor och serieutvecklingar.

\SOLUTION


\TaskSolved \what
 %%%TODO number  10 %%%

---


\QUESTEND






\WHAT{}

\QUESTBEGIN

\Task  \what~ Läs mer om \code{Future} och jämlöpande exekvering i Scala här:\\
\href{http://alvinalexander.com/scala/future-example-scala-cookbook-oncomplete-callback}{alvinalexander.com/scala/future-example-scala-cookbook-oncomplete-callback}

\SOLUTION


\TaskSolved \what
 %%%TODO number  11 %%%

---


\QUESTEND






\WHAT{}

\QUESTBEGIN

\Task  \what~ Läs mer om jämlöpande exekvering och multitrådade program i Java här: \href{http://www.tutorialspoint.com/java/java_multithreading.htm}{www.tutorialspoint.com/java/java\_multithreading.htm}  \\
\noindent När man skriver program med jämlöpande exekvering finns det många fallgropar; det kan bli kapplöpning \Eng{race conditions} om gemensamma resurser och dödläge \Eng{deadlock} där inget händer för att trådar väntar på varandra. Mer om detta i senare kurser.


\SOLUTION


\TaskSolved \what
 %%%TODO number  12 %%%

---


\QUESTEND






\WHAT{Studera dokumentationen i \code{scala.concurrent}.}

\QUESTBEGIN

\Task  \what~\Pen

\Subtask Studera dokumentationen för \code{scala.concurrent.Future}\footnote{\href{http://www.scala-lang.org/files/archive/api/current/\#scala.concurrent.Future}{http://www.scala-lang.org/files/archive/api/current/\#scala.concurrent.Future}}. Hur samverkar \code{Future} med \code{Try} och \code{Option}? Vilka vanliga samlingsmetoder känner du igen?

\Subtask Studera dokumentationen för \code{scala.concurrent.duration.Duration}\footnote{\href{http://www.scala-lang.org/api/current/\#scala.concurrent.duration.Duration}{www.scala-lang.org/api/current/\#scala.concurrent.duration.Duration}}. Vilka tidsenheter kan användas?

\Subtask Vid import av \code{scala.concurrent.duration._ } dekoreras de numeriska klasserna med metoder för att skapa instanser av klassen \code{Duration}. Detta möjligörs med hjälp av klassen \code{scala.concurrent.duration.DurationConversions}. Studera dess dokumentation och testa att i REPL skapa några tidsperioder med metoderna på \code{Int}.



\SOLUTION


\TaskSolved \what
 %%%TODO number  13 %%%

\SubtaskSolved

\SubtaskSolved

\SubtaskSolved


\QUESTEND






\WHAT{}

\QUESTBEGIN

\Task  \what~ Fördjupa dig inom webbteknologi.

\Subtask Lär dig om HTML, CSS och JavaScript här: \url{https://developer.mozilla.org/en-US/docs/Learn}

\Subtask Lär dig om Scala.JS här: \url{http://www.scala-js.org/}\SOLUTION


\TaskSolved \what
 %%%TODO number  14 %%%

\SubtaskSolved  ---

\SubtaskSolved  ---

\SubtaskSolved  ---

\SubtaskSolved  ---
\QUESTEND


%!TEX encoding = UTF-8 Unicode
%!TEX root = ../compendium.tex

\ifPreSolution

\Exercise{java}\label{exe:java}

\begin{Goals}
\item Kunna förklara och beskriva viktiga skillnader mellan Scala och Java.
\item Kunna översätta enkla algoritmer, klasser och singeltonobjekt från Scala till Java och vice versa.
\item Känna till vad en case-klass innehåller i termer av en Javaklass.
%\item Förstå hur autoboxing fungerar.
\item Kunna använda Javatyperna \code{List}, \code{ArrayList}, \code{Set}, \code{HashSet} och översätta till deras Scalamotsvarigheter med \code{CollectionConverters}.
\item Kunna förklara hur autoboxning fungerar i Java, samt beskriva fördelar och fallgropar.
\end{Goals}

\begin{Preparations}
\item Studera teori i början av detta Appendix.
\end{Preparations}

\BasicTasks %%%%%%%%%%%%%%%%

\else

\ExerciseSolution{java}

\BasicTasks %%%%%%%%%%%

\fi





\WHAT{Översätta metoder från Java till Scala.}

\QUESTBEGIN

\Task  \what~  I denna uppgift ska du översätta en Java-klass som används som en modul\footnote{\href{https://en.wikipedia.org/wiki/Modular_programming}{en.wikipedia.org/wiki/Modular\_programming}} och bara innehåller statiska metoder och inget förändringsbart tillstånd som kan ändras utifrån. (I nästa uppgift ska du sedan översätta klasser med förändringsbara  tillstånd.)

Vi börjar med att göra översättningen från Java till Scala rad för rad och du ska behålla så mycket som möjligt av syntax och semantik så att Scala-koden blir så Java-lik som möjligt. I efterföljande deluppgift ska du sedan omforma översättningen så att Scala-koden blir mer idiomatisk\footnote{\href{https://sv.wikipedia.org/wiki/Idiom_\%28programmering\%29}{sv.wikipedia.org/wiki/Idiom\_\%28programmering\%29}}.

\Subtask Studera klassen \code{Hangman} nedan. Du ska översätta den från Java till Scala enlig de riktlinjer och tips som följer efter koden. Läs igenom alla riktlinjer och tips innan du börjar.

\javainputlisting[numbers=left]{examples/scalajava/Hangman.java}

\noindent\emph{Riktlinjer och tips för översättningen:}

\begin{enumerate}[noitemsep]

\item Skriv Scala-koden med en texteditor i en fil som heter \texttt{hangman1.scala} och kompilera med \code{scalac hangman1.scala} i terminalen; använd alltså \emph{inte} en IDE, så som Eclipse eller IntelliJ, utan en ''vanlig'' texteditor, t.ex. VS \code{code}.

\item Översätt i denna första deluppgift rad för rad så likt den ursprungliga Java-kodens utseende (syntax)  som möjligt, med så få ändringar som möjligt. Du ska alltså ha kvar dessa Scalaovanligheter, även om det inte alls blir som man brukar skriva i Scala:
\begin{enumerate}[nolistsep, noitemsep]
\item långa indrag, \item onödiga semikolon, \item onödiga \code{()}, \item onödiga \code|{}|, \item onödiga \code{System.out}, och \item onödiga \code{return}.
\end{enumerate}

\item Försök också i denna deluppgift göra så att betydelsen (semantiken) så långt som möjligt motsvarar den i Java, t.ex. genom att använda \code{var} överallt, även där man i Scala normalt använder \code{val}.

\item En Javaklass med bara statiska medlemmar motsvarar ett singeltonobjekt i Scala, alltså en \code{object}-deklaration innehållande ''vanliga'' medlemmar.

\item För att tydliggöra att du använder Javas \code{Set} och \code{HashSet} i din Scala-kod, använd följande import-satser i \code{hangman1.scala}, som därmed döper om dina importerade namn och gör så att de inte krockar med Scalas inbyggda \code{Set}. Denna form av import går inte att göra i Java.
\begin{Code}
import java.util.{Set => JSet};
import java.util.{HashSet => JHashSet};
\end{Code}

\item Javas \code{i++} fungerar inte i Scala; man får istället skriva \code{i += 1} eller mindre vanliga \code{i = i + 1}.

\item Typparametrar i Java skrivs inom \code{<>} medan Scalas syntax för typparametrar använder \code{[]}.

\item Till skillnad från Java så har Scalas metoddeklarationer ett tilldelningstecken \code{=} efter returtypen, före kroppen.

\item Du kan ladda ner Java-koden till \code{Hangman}-klassen nedan från kursens repo%
\footnote{\href{https://github.com/lunduniversity/introprog/blob/master/compendium/examples/scalajava/Hangman.java}{github.com/lunduniversity/introprog/blob/master/compendium/examples/scalajava/Hangman.java}}. I samma bibliotek ligger även lösningarna till översättningen i Scala, men kolla \emph{inte} på dessa förrän du gjort klart översättningarna och fått dem att kompilera och köra felfritt! Tanken är att du ska träna på att läsa felmeddelande från kompilatorn och åtgärda dem i en upprepad kompilera-testa-rätta-cykel.

\end{enumerate}







\Subtask Skapa en ny fil \code{hangman2.scala} som till att börja med innehåller en kopia av din direkt-översatta Java-kod från föregående deluppgift. Omforma koden så att den blir mer som man brukar skriva i Scala, alltså mer Scala-idiomatisk. Försök förenkla och förkorta så mycket du kan utan att göra avkall på läsbarheten.

\emph{Tips och riktlinjer:}

\begin{enumerate}[nolistsep, noitemsep]

\item Kalla Scala-objektet för \code{hangman}. När man använder ett Scalaobjekt som en modul (alltså en samling funktioner i en gemensam, avgränsad namnrymd) har man gärna liten begynnelsebokstav, i likhet med konventionen för paketnamn. Ett paket är ju också en slags modul och med en namngivningskonvention som är gemensam kan man senare, utan att behöva ändra koden som använder modulen, ändra från ett singelobjekt till ett paket och vice versa om man så önskar.

\item Gör alla metoder publikt tillgängliga och låt även strängvektorn \code{hangman} vara publikt tillgänglig. Deklarera \code{hangman} som en \code{val} och konstruera den med \code{Vector}. Eftersom \code{Vector} är oföränderlig och man inte kan ärva från singelobjekt och \code{hangman} är deklarerad med \code{val} finns inga speciella risker med att göra den konstanta vektorn publik om  vi inte har något emot att annan kod kan läsa (och eventuellt göra sig beroende av) vår hänggubbetext.

\item I metoden \code{renderHangman}, använd \code{take} och \code{mkString}.

\item I metoden \code{hideSecret}, använd \code{map} i stället för en \code{for}-sats.

\item Det går att ersätta metoden \code{foundAll} med det kärnfulla uttrycket \\ \code{(secret forall found)} där \code{secret} är en sträng och \code{found} är en mängd av tecken (undersök gärna i REPL hur detta fungerar). Skippa därför den metoden helt och använd det kortare uttrycket direkt.

\item I metoden \code{makeGuess}, i stället för \code{Scanner}, använd \code{scala.io.StdIn.readLine}.

\item Om du vill träna på att använda rekursion i stället för imperativa loopar: Gör metoden \code{makeGuess} rekursiv i stället för att använda \code{do}-\code{while}.

\item I metoden \code{download}, i stället för \code{java.net.URL} och \code{java.util.ArrayList}, använd \code{scala.io.Source.fromURL(address, coding).getLines.toVector} och gör en lokal import av \code{scala.io.Source.fromURL} överst i det block där den används. Det går inte att ha lokala \code{import}-satser i Java.

\item Låt metoden \code{download} returnera en \code{Option[String]} som i fallet att nedladdningen misslyckas returnerar \code{None}.

\item I metoden \code{download}, i stället för \code{try}-\code{catch} använd \code{scala.util.Try} och dess smidiga metod \code{toOption}.

\item Om du vill träna på att använda rekursion i stället för imperativa loopar: Använd, i stället för \code{while}-satsen i metoden \code{play}, en lokal rekursiv funktion med denna signatur:
\begin{Code}
  def loop(found: Set[Char], bad: Int): (Int, Boolean)
\end{Code}
Funktionen \code{loop} returnerar en 2-tupel med antalet felgissningar och \code{true} om man hittat alla bokstäver eller \code{false} om man blev hängd.

\end{enumerate}





\SOLUTION


\TaskSolved \what
     %%%TODO number  1 %%%starts with: \emph{Översätta algoritmer och %%%

\SubtaskSolved  \scalainputlisting[numbers=left,basicstyle=\ttfamily\fontsize{10.3}{12}\selectfont]{examples/scalajava/hangman1.scala}

\SubtaskSolved  \scalainputlisting[numbers=left,basicstyle=\ttfamily\fontsize{11.2}{13}\selectfont]{examples/scalajava/hangman2.scala}



\QUESTEND






\WHAT{Översätta mellan klasser i Scala och klasser i Java.}

\QUESTBEGIN

\Task  \what~
Klassen \code{Point} nedan är en modell av en punkt som kan sparas på begäran i en lista. Listan är privat för kompanjonsobjektet och kan skrivas ut med en metod \code{showSaved}. I koden används en \code{ArrayBuffer}, men i framtiden vill man, vid behov, kunna ändra från \code{ArrayBuffer} till en annan sekvenssamlingsimplementation, t.ex. \code{ListBuffer}, som uppfyller egenskaperna hos supertypen \code{Buffer}, men har andra prestandaegenskaper för olika operationer. Därför är attributet \code{saved} i kompanjonsobjektet deklarerat med den mer generella typen.

\scalainputlisting[numbers=left]{examples/scalajava/Point.scala}

\Subtask Översätt klassen \code{Point} ovan från Scala till Java. Vi ska i nästa deluppgift kompilera både Scala-programmet ovan och ditt motsvarande Java-program i terminalen och testa i REPL att klasserna har motsvarande funktionalitet.

\emph{Tips och riktlinjer:}
\begin{enumerate}[nolistsep, noitemsep]
\item För att namnen inte ska krocka i våra kommande tester, kalla Javatypen för \code{JPoint}.
\item  I stället för Scalas \code{ArrayBuffer} och \code{Buffer}, använd Javas \code{ArrayList} och \code{List} som båda ligger i paketet \code{java.util}.
\item Undersök dokumentationen för \code{java.util.List} för att hitta en motsvarighet till \code{prepend} för att lägga till i början av listan.
\item I stället för default-argumentet i Scalas primärkonstruktor, använd en extra Java-konstruktor.
\item Det finns inga singelobjekt och inga kompanjonsobjekt i Java; istället kan man använda statiska klassmedlemmar. Placera kompanjonsobjektets medlemmars motsvarigheter \emph{inuti} Java-klassen och gör dem till \jcode{static}-medlemmar.
\item Kod i klasskroppen i Scalaklassen, så som if-satsen på rad 4, placeras i lämplig konstruktor i Javaklassen.
\item Utskrifter med \code{print} och \code{println} behöver i Java föregås av \code{System.out}.
\item Det finns inget nyckelord \code{override} i Java, men en s.k. annotering som ger samma kompilatorhjälp. Den skrivs med ett snabel-a och stor begynnelsebokstav, så här: \jcode{ @Override }  före metoddeklarationen.
\item I Java används konventionen att börja getter-metoder med ordet \code{get}, t.ex. \code{getX()}.
\item Det finns ingen motsvarighet till \code{mkString} för \code{List} så du behöver själv gå igenom listan och hämta elementreferenser för utskrift med en \jcode{for}-loop. Notera att efter sista elementet ska radbrytning göras i utskriften och att inget komma ska skrivas ut efter sista elementet.
\item I Java behövs en ny \jcode{import}-deklaration om man vill importera ännu en typ från samma paket. Man kan även i Java använda asterisk \code{*}, (motsvarande \code{_} i Scala), för att importera allt i ett paket, men då får man med alla möjliga namn och det vill man kanske inte.
\item Metoder i Java slutar med \code{()} om de saknar parametrar.
\item Alla satser i Java slutar med lättglömda semikolon. (Efter att man i skrivit mycket Javakod och växlar till Scalakod är det svårt att vänja sig av med att skriva semikolon...)
\end{enumerate}


\Subtask Starta REPL i samma bibliotek som du kompilerat kodfilerna. Testa så att klasserna \code{Point} och \code{JPoint} beter sig på samma vis enligt nedan. Skriv även testkod i REPL för att avläsa de attributvärden som har getters och undersök att allt funkar som det ska.
\begin{REPLnonum}
> scalac Point.scala
> javac JPoint.java
> scala
scala> val (p, jp) = (new Point, new JPoint)
scala> p.distanceTo(new Point(3, 4))
scala> Point.showSaved
scala> jp.distanceTo(new JPoint(3, 4))
scala> JPoint.showSaved
scala> for (i <- 1 to 10) { new Point(i, i, true) }
scala> Point.showSaved
scala> for (i <- 1 to 10) { new JPoint(i, i, true) }
scala> JPoint.showSaved
\end{REPLnonum}


\Subtask Översätt nedan Javaklass \code{JPerson} till en \code{case class Person} i Scala med  motsvarande funktionalitet.


\javainputlisting[numbers=left]{examples/scalajava/JPerson.java}


\Subtask\Pen Undersök i REPL vilken funktionalitet i Scala-case-klassen \code{Person} som \emph{inte} är implementerad i Java-klassen \code{JPerson} ovan. Skriv upp namnen på några av case-klassens extra metoder samt deras signatur genom att för en \code{Person}-instans, och för kompanjonsobjektet \code{Person}, trycka på TAB-tangenten. Prova några av de extra metoderna i REPL och förklara vad de gör.

\begin{REPL}
scala> val p = Person("Björn", 49)
scala> p.      // tryck TAB en gång
scala> Person. // tryck TAB en gång
scala> p.copy  // tryck TAB en gång
scala> p.copy()
scala> p.copy(age = p.age + 1)
scala> Person.unapply(p)
\end{REPL}


\SOLUTION


\TaskSolved \what
     %%%TODO number  2 %%%starts with: \emph{Översätta mellan klasser %%%

\SubtaskSolved   \javainputlisting[numbers=left]{examples/scalajava/JPoint.java}

\SubtaskSolved   -

\SubtaskSolved   \begin{Code}
case class Person(name: String, age: Int = 0)
\end{Code}

\SubtaskSolved  p.*TAB* - copy, producArity, ProductIterator, productElement, productPrefix

Person.*TAB* - apply, curried, tupled, unapply

\begin{REPLnonum}
scala> p.copy
   def copy(name: String,age: Int): Person

scala> p.copy()
res0: Person = Person(Björn,49)

scala> p.copy(age = p.age + 1)
res1: Person = Person(Björn,50)

scala> Person.unapply(p)
res2: Option[(String, Int)] = Some((Björn,49))
\end{REPLnonum}



\QUESTEND






\WHAT{Auto(un)boxing.}

\QUESTBEGIN

\Task  \what~  I JVM måste typparametern för generiska klasser vara av referenstyp. I Scala löser kompilatorn detta åt oss så att vi ändå kan ha t.ex. \code{Int} som argument till en typparameter i Scala, medan man i Java \emph{inte} direkt kan ha den primitiva typen \jcode{int} som typparameter till t.ex. \code{ArrayList}.

I Java och i den underliggande plattformen JVM används s.k. wrapper-klasser för att lösa detta, t.ex. genom wrapper-klassen \code{Integer} som boxar den primitiva typen \jcode{int}. Java-kompilatorn har stöd för att automatiskt packa in värden av primitiv typ i sådana wrapper-klasser för att skapa referenstyper och kan även automatiskt packa upp dem.

\Subtask Studera hur Scala-kompilatorn låter oss arbeta med en \code{Cell[Int]} även om det underliggande JVM:ens körtidstyp \Eng{runtime type} är en wrapper-klass. Man kan se JVM-körtidstypen med metoderna \code{getClass} och \code{getTypeName} enligt nedan.
\begin{REPL}
scala> class Cell[T](var value: T){
         val typeName: String = value.getClass.getTypeName
         override def toString = "Cell[" + typeName + "](" + value + ")"
       }
scala> val c = new Cell[Int](42)
scala> c.value.getClass.getTypeName
\end{REPL}


\Subtask Vad är körtidstypen för \code{c.value} ovan? Förklara hur det kan komma sig trots att vi deklarerade med typargumentet \code{Int}?

\Subtask Studera dokumentationen för \code{java.lang.Integer}\footnote{\href{https://docs.oracle.com/javase/8/docs/api/java/lang/Integer.html}{docs.oracle.com/javase/8/docs/api/java/lang/Integer.html}} och testa i REPL några av \emph{klassmetoderna} (de som är \jcode{static} och därmed kan anropas med punktnotation direkt på klassens namn utan \code{new}) och några av \emph{instansmetoderna} (de som inte är \jcode{static}).
\begin{REPL}
scala> Integer.  //tryck TAB
scala> Integer.
scala> Integer.toBinaryString(42)
scala> Integer.valueOf(42)
scala> val i = new Integer(42)
scala> i.  // tryck TAB
scala> i.toString
scala> i.compareTo  // tryck TAB 2 gånger
scala> i.compareTo(Integer.valueOf(42))
scala> i.compareTo(42)  // varför fungerar detta?
\end{REPL}

\Subtask\Pen Enligt dokumentationen\footnote{\href{https://docs.oracle.com/javase/8/docs/api/java/lang/Integer.html\#compareTo-java.lang.Integer-}{docs.oracle.com/javase/8/docs/api/java/lang/Integer.html\#compareTo-java.lang.Integer-}} tar instansmetoden \code{compareTo} i klassen \code{Integer} en \code{Integer} som parameter. Hur kan det då komma sig att sista raden ovan fungerar med en \code{Int}?

\Subtask Studera nedan Java-program och beskriv vad som kommer att skrivas ut \emph{innan} du kompilerar och testkör.

\javainputlisting[numbers=left]{examples/scalajava/Autoboxing.java}

\Subtask Ändra i programmet ovan så att autoboxing och autounboxing utnyttjas på alla ställen där så är möjligt. Utnyttja även att \code{toString}-metoden på \code{Integer} ger samma stränrepresentation som \jcode{int} vid utskrift. Fixa också så att du undviker \emph{fallgropen} att i Java jämföra med referenslikhet i stället för att använda \code{equals}. Testa så att allt fungerar som det borde efter dina ändringar.


\Subtask\Pen Antag att du råkar skriva \jcode{xs.add(0, pos)} på rad 14 i ditt program från föregående uppgift. Förklara hur autoboxingen stjälper dig i en \emph{fallgrop} då.

\Subtask\Pen Med ledning av de båda tidigare deluppgifterna: sammanfatta de två nämnda fallgropar med autoboxing i Java i två generella punkter, så att du har nytta av att memorera dem inför din framtida Javakodning.


\SOLUTION


\TaskSolved \what
     %%%TODO number  3 %%%starts with: \emph{Auto(un)boxing.} I JVM må%%%

\SubtaskSolved   -

\SubtaskSolved   Cell har typen java.lang.Integer. När man hämtar ut värdet med \code{c.value} hämtas den primitiva typ \code{int} ut.

\SubtaskSolved   Med hjälp av autoboxing förvandlas 42 till typen \code{Integer} och kan därför jämföras med en annan \code{Integer}.

\SubtaskSolved   i.compareTo(42) fungerar på grund av autoboxing. Då JVM packar in den primitiva typ int i en Integer-objekt automatiskt.

\SubtaskSolved
\begin{REPLnonum}
0 10 20 30 40 50 60 ... 390 400 410

[0]: 0
[42]: 0
NOT EQUAL
\end{REPLnonum}

\SubtaskSolved   \javainputlisting[numbers=left]{examples/scalajava/Autoboxing2.java}

\SubtaskSolved   42 kommer läggas längst fram i listan istället för längst bak, då autounboxing kommer göra Integer(0) till 0 och tvärtom med variablen \code{pos}.

\SubtaskSolved   Om man ska undersöka om två int-variabler är lika ska man använda ==, men om variablerna är av typen Integer måste man använda \code{equals}.

JVM kommer inte varna om man vänder på \code{Integer} och \code{int}, som i \code{xs.add(0, pos)}.



\QUESTEND






\WHAT{CollectionConverters.}

\QUESTBEGIN

\Task  \what~  Med \code{import scala.jdk.CollectionConverters._} får man i sina Scalaprogram tillgång till de smidiga metoderna \code{asJava} och \code{asScala} som översätter mellan motsvarande samlingar i resp språks standardbibliotek. Kör nedan i REPL och gör efterföljande deluppgifter.

\begin{REPL}
scala> val sv = Vector(1,2,3)
scala> val ss = Set('a','b','c')
scala> val sm = Map("gurka" -> 42, "tomat" -> 0)
scala> val ja = new java.util.ArrayList[Int]
scala> ja.add(42)
scala> val js = new java.util.HashSet[Char]
scala> js.add('a')
scala> import scala.jdk.CollectionConverters._
\end{REPL}

\Subtask Till vilka typer konverteras Scalasamlingarna
\code{Vector[Int]}, \code{Set[Char]} och \\ \code{Map[String, Int]} om du anropar metoden \code{asJava} på dessa?

\Subtask Till vilka typer konverteras Javasamlingarna \code{ArrayList[Int]} och \code{HashSet[Char]}  om du anropar metoden \code{asScala} på dessa? Blir det föränderliga eller oföränderliga motsvarigheter?

\Subtask Vad får resultatet för typ om du kör \code{toSet} på en samling av typen \code{mutable.Set}?

\Subtask Undersök hur du kan efter att du gjort \code{sm.asJava.asScala} anropa ytterligare en metod för att få tillbaka en oföränderlig \code{immutable.Map}.

\Subtask Läs mer i dokumentationen om CollectionConverters\footnote{\href{https://docs.scala-lang.org/overviews/collections-2.13/conversions-between-java-and-scala-collections.html}{docs.scala-lang.org/overviews/collections-2.13/conversions-between-java-and-scala-collections.html}}
och prova några fler konverteringar.



\SOLUTION


\TaskSolved \what
     %%%TODO number  4 %%%starts with: \emph{CollectionConverters.} Med \cod%%%

\SubtaskSolved

Vector[Int] -> java.util.List[Int]

Set[Char] -> java.util.Set[Char]

Map[String, Int] -> java.util.Map[String, Int]

\SubtaskSolved

ArrayList[Int] -> scala.collection.mutable.Buffer[Int]

HashSet[Char] -> scala.collection.mutable.Set[Char]

Båda blir föränderliga motsvarigheter. Det visas genom att de till hör \code{scaka.collection.mutable} och både \code{ArrayList} och \code{HashSet} är förändrliga i Java.

\SubtaskSolved   \code{scala.collection.immutable.Set}

\SubtaskSolved   \code{sm.asJava.asScala} ger typen \code{scala.collection.mutable.Map[String,Int]}

\code{sm.asJava.asScala.toMap} ger typen \code{scala.collection.immutable.Map[String,Int]}

\SubtaskSolved   -

\QUESTEND


\WHAT{Hur fungerar en \jcode{switch}-sats i Java (och flera andra språk)?}

\QUESTBEGIN

\Task \label{task:switch} \what~   Det händer ofta att man vill testa om ett värde är ett av många olika alternativ. Då kan man använda en sekvens av många \code{if}-\code{else}, ett för varje alternativ. Men det finns ett annat sätt i Java och många andra språk: man kan använda \jcode{switch} som kollar flera alternativ i en och samma sats, se t.ex. \href{https://en.wikipedia.org/wiki/Switch_statement}{en.wikipedia.org/wiki/Switch\_statement}.

\Subtask Skriv in nedan kod i en kodeditor. Spara med namnet \texttt{Switch.java} och kompilera filen med kommandot \texttt{javac Switch.java}. Kör den med \texttt{java Switch} och ange din favoritgrönsak som argument till programmet. Vad händer? Förklara hur \jcode{switch}-satsen fungerar.

\javainputlisting[numbers=left,basicstyle=\ttfamily\fontsize{9}{11}\selectfont]{examples/Switch.java}

\Subtask \label{subtask:break} Vad händer om du tar bort \jcode{break}-satsen på rad 16?




\SOLUTION


\TaskSolved \what


\SubtaskSolved  Beroende på första bokstaven i din favoritgrönsak får du olika svar såsom \textit{gurka är gott!} vid första bokstaven $g$.\\
Javas \jcode{switch}-sats testar den första bokstaven på favoritgrönsaken genom att stegvis jämföra den med \jcode{case}-uttrycken. Om första bokstaven \jcode{firstChar} matchar bokstaven efter ett \jcode{case} körs koden efter kolonet till \jcode{switch}-satsens slut eller tills ett \jcode{break} avbryter \jcode{switch}-satsen.\\
Matchar inte \jcode{firstChar} något \jcode{case} så finns även \jcode{default}, som körs oavsett vilken första bokstaven är, ett generellt fall.

\SubtaskSolved  Om \jcode{case 't'} körs kommer både  \textit{tomat är gott!} och \textit{broccoli är gott!} skrivas ut, man säger att koden $"$faller igenom$"$. Utan \jcode{break}-satsen i Java körs koden i efterkommande \jcode{case} tills ett \jcode{break} avbryter exekveringen eller \jcode{switch}-satsen tar slut.



\QUESTEND




\WHAT{Fånga undantantag i Java med en \jcode{try}-\jcode{catch}-sats.}

\QUESTBEGIN

\Task \label{task:javatry} \what~   Det finns som vi såg i förra uppgiften inbyggt stöd i JVM för att hantera när program avbryts på oväntade sätt, t.ex. på grund av division med noll eller ej förväntade indata från användaren. Spara koden nedan\footnote{\url{https://github.com/lunduniversity/introprog/blob/master/compendium/examples/TryCatch.java}} i en fil med namnet \texttt{TryCatch.java} och kompilera med \texttt{javac TryCatch.java} i terminalen.

\javainputlisting[numbers=left,basicstyle=\ttfamily\fontsize{11}{12}\selectfont]{examples/TryCatch.java}

\Subtask Förklara vad som händer när du kör programmet med olika indata:
\begin{REPL}
> java TryCatch 42
> java TryCatch 0
> java TryCatch safe 42
> java TryCatch safe 0
> java TryCatch
\end{REPL}

\Subtask Vad händer om du ''glömmer bort'' raden 15 och därmed missar att initialisera input? Hur lyder felmeddelandet? Är det ett körtidsfel eller kompileringsfel?

%\Subtask Beskriv några skillnader och likheter i syntax och semantik mellan \code{try}-\code{catch} i Java respektive Scala.



\SOLUTION


\TaskSolved \what


\SubtaskSolved  \begin{enumerate}
\item Eftersom första argumentet inte är strängen \textit{safe} görs en oskyddad division av 42 med 42 där slutsvaret 1 visas.
\item Eftersom första argumentet inte är strängen \textit{safe} görs en oskyddad division av 42 med 0 som ger \code{ArithmeticException} eftersom ett tal inte kan delas med noll.
\item Eftersom första argumentet är strängen \textit{safe} görs en skyddad division av 42 med 42 där slutsvaret 1 visas.
\item Eftersom första argumentet är strängen \textit{safe} görs en skyddad division av 42 med 0. Denna gång fångas \code{ArithmeticException} av \code{try-catch}-satsen vilket ersätter den gamla division med en säker division med 1 där slutsvaret 42 visas.
\item Eftersom inga argument givits kastas ett \code{ArrayIndexOutOfBoundsException} när programmet försöker anropa \code{equals} metoden hos en sträng som inte finns. Detta kunde också kontrollerats av en \code{try-catch}-sats.
\end{enumerate}

\SubtaskSolved  \begin{REPL}
TryCatch.java:16: error: variable input might not have been initialized
\end{REPL}
Ett kompileringsfel uppstår på grund av risken att \code{input} inte blivit definierad vid division.

% \SubtaskSolved  Den mest markanta skillnaden mellan språken är att Scala varken kräver att ett undantag fångas av en \code{catch} eller att ett undantag behöver deklareras innan det kastas med en \code{@throws}. Dessutom saknar \code{catch}-metoden hos Java de \code{match}-egenskaper Scala har. Inte heller returnerar \code{catch} hos Java något värde vilket gör det nödvändigt att definiera variabler för detta innan. I övrigt är semantiken och syntaxen väldigt lika mellan båda språken. De använder samma struktur och samma ord, dessutom har de en hel del \code{Exception} gemensamt.



\QUESTEND




\WHAT{Matriser med array i Java.}

\QUESTBEGIN

\Task \label{task:arraymatrix-java} \what~   Om man redan vid allokering vet hur många element en matris ska ha, använder man i Java gärna en array av arrayer. En heltalsmatris (en array av array av heltal) skrivs i Java med dubbla hakparentespar \jcode{int[][]} direkt efter typen. Vid allokering använder man nyckelordet \code{new} och antalet element i respektive dimension anges inom hakparenteserna; t.ex. så ger \jcode{new int[42][21]} en matris med 42 rader och 21 kolumner, vilket motsvarar att man i Scala skriver \code{Array.ofDim[Int](42,21)}%
\footnote{
Ett annat sätt att skriva detta i Scala där initialvärdet framgår explicit: \code{Array.fill(42,21)(0)}
}. Alla element får defaultvärdet för typen, här \code{0} för heltal.

\Subtask Skriv nedan program i en editor och spara koden i filen \texttt{JavaArrayTest.java} och kompilera med \texttt{javac JavaArrayTest.java} och kör i terminalen med \texttt{java JavaArrayTest} och undersök utskriften. Förklara vad som händer. Notera några skillnader i hur matriser används i Scala och Java.


\begin{Code}[language=Java]
public class JavaArrayTest {

    public static void showMatrix(int[][] m){
        System.out.println("\n--- showMatrix ---");
        for (int row = 0; row < m.length; row++){
            for (int col = 0; col < m[row].length; col++) {
                System.out.print("[" + row + "]");
                System.out.print("[" + col + "] = ");
                System.out.print(m[row][col] + "; ");
            }
            System.out.println();
        }
    }

    public static void main(String[] args) {
        System.out.println("Hello JavaArrayTest!");
        int[][] xss = new int[10][5];
        showMatrix(xss);
    }
}
\end{Code}

\Subtask Implementera nedan metod \code{fillRnd} inuti klassen \code{JavaArrayTest}. Skriv kod som fyller matrisen \code{m} med slumptal mellan \code{1} och \code{n}.
\begin{Code}[language=Java]
    public static void fillRnd(int[][] m, int n){
        /* ??? */
    }
\end{Code}
\noindent \emph{Tips:} med detta uttryck skapas ett slumptal mellan 1 och 42 i Java:\\
\jcode{(int) (Math.random() * 42 + 1);} \\
där typkonverteringen \jcode{(int)} ger samma effekt som ett anrop av metoden \code{toInt} i Scala; alltså att dubbelprecisionsflyttal omvandlas till heltal genom avkortning av alla decimaler.


Ändra huvudprogrammet så det anropar \jcode{fillRnd(xss, 6)}. %
% \begin{Code}[language=Java]
%     public static void main(String[] args) {
%         System.out.println("Hello JavaArrayTest!");
%         int[][] xss = new int[10][5];
%         fillRnd(xss, 6);
%         showMatrix(xss);
%     }
% \end{Code}
Programmet ska ge en utskrift som liknar följande:
\begin{REPL}
Hello JavaArrayTest!

--- showMatrix ---
[0][0] = 6; [0][1] = 2; [0][2] = 6; [0][3] = 3; [0][4] = 5;
[1][0] = 2; [1][1] = 4; [1][2] = 6; [1][3] = 1; [1][4] = 1;
[2][0] = 5; [2][1] = 4; [2][2] = 4; [2][3] = 1; [2][4] = 5;
[3][0] = 4; [3][1] = 6; [3][2] = 6; [3][3] = 1; [3][4] = 3;
[4][0] = 4; [4][1] = 6; [4][2] = 2; [4][3] = 3; [4][4] = 2;
[5][0] = 2; [5][1] = 4; [5][2] = 5; [5][3] = 5; [5][4] = 3;
[6][0] = 6; [6][1] = 5; [6][2] = 2; [6][3] = 4; [6][4] = 3;
[7][0] = 1; [7][1] = 6; [7][2] = 1; [7][3] = 6; [7][4] = 2;
[8][0] = 1; [8][1] = 1; [8][2] = 5; [8][3] = 3; [8][4] = 2;
[9][0] = 1; [9][1] = 1; [9][2] = 1; [9][3] = 5; [9][4] = 4;

\end{REPL}

\SOLUTION

\TaskSolved \what
     %starts with: \label{task:arraymatrix-java} \%%%

%6.a)
\SubtaskSolved  Vid initialisering fylls alla element i \code{xss} med standardvärdet för typen, \code{0} i fallet med \code{int}. Den yttre \code{for}-loopen i \code{showMatrix()} itererar över raderna i \code{xss}. Den inre \code{for}-loopen itererar i sin tur längs med elementen på den aktuella raden och skriver ut rad, kolumn och innehåll. Efter varje rad sker en radbrytning, så att en rad i utskriften även motsvarar en rad i matrisen.\\
Exempel på skillnader mellan användning av matriser i scala och java:
\begin{itemize}
\item åtkomst: \code{minArray(rad)(kolumn)} respektive \code{minArray[rad][kolumn]}
\item typnamn: \code{Array[Array[elementTyp]]} respektive  \code{elementTyp[][]}
\item allokering: \code{Array.ofDim[typ](xDim,yDim)} respektive \code{new typ[xDim][yDim]}
\end{itemize}

%6.b)
\SubtaskSolved  \begin{Code}[language=Java]
public class JavaArrayTest {

	public static void showMatrix(int[][] m){
		System.out.println("\n--- showMatrix ---");
		for (int row = 0; row < m.length; row++){
			for (int col = 0; col < m[row].length; col++) {
				System.out.print("[" + row + "]");
				System.out.print("[" + col + "] = ");
				System.out.print(m[row][col] + ";");
			} System.out.println();
		}
	}

	public static void fillRnd(int[][] m, int n){
		for (int row = 0; row < m.length; row++){
			for (int col = 0; col < m[row].length; col++) {
				m[row][col] = (int) (Math.random() * n + 1);
			}
		}
	}

	public static void main(String[] args) {
    System.out.println("Hello JavaArrayTest!");
		int[][] xss = new int[10][5];
		fillRnd(xss, 6);
		showMatrix(xss);
	}
}
\end{Code}

\QUESTEND


%\ExtraTasks %%%%%%%%%%%%%%%%%%%


\WHAT{Översätta från Java till Scala.}

\QUESTBEGIN

\Task  \what~ Översätt nedan kod från Java till Scala. Skriv koden i en fil som heter \texttt{showInt.scala} och kalla Scala-objektet med \code{main}-metoden för \code{showInt}. Läs tipsen som följer efter koden innan du börjar.

\javainputlisting[numbers=left]{examples/scalajava/JShowInt.java}

\emph{Tips:}
\begin{itemize}[nolistsep, noitemsep]
\item En Javaklass med bara statiska medlemmar motsvaras av ett singeltonobjekt i Scala, alltså en \code{object}-deklaration. Scala har därför inte nyckelordet \jcode{static}.
\item Typen \jcode{Object} i Java motsvaras av Scalas \code{Any}.
\item Du kan använda Scalas möjlighet med default-argument (som saknas i Java) för att bara definiera en enda \code{show}-metod med en tom sträng som default \code{msg}-argument.
\item I Scala har objekt av typen \code{Char} en metod \code{def *(n: Int): String} som skapar en sträng med tecknet repeterat \code{n} gånger. Men du kan ju välja att ändå implementera metoden \code{repeatChar} med \code{StringBuilder} som nedan om du vill träna på att översätta en \code{for}-loop från Java till Scala.
\item I stället för \code{Scanner.nextLine} kan du använda \code{scala.io.StdIn.readLine} som tar en prompt som parameter, men du kan också använda \code{Scanner} i Scala om du vill träna på det.
\item I Java \emph{måste} man använda nyckelordet \jcode{return} om metoden inte är en \jcode{void}-metod, medan man i Scala faktiskt \emph{får} använda \code{return} även om man brukar undvika det och i stället utnyttja att satser i Scala också är uttryck.
\end{itemize}
Kompilera din Scala-kod och kör i terminalen och testa så att allt funkar. Vill du även kompilera Java-koden så finns den i kursens repo i filen\\ \texttt{compendium/examples/scalajava/JShowInt.java}


\SOLUTION


\TaskSolved \what


\begin{Code}[numbers=left]
object showInt {
  def show(obj: Any, msg: String = ""): Unit = println(msg + obj)

  def repeatChar(ch: Char, n: Int): String = ch.toString * n

  def showInt(i: Int): Unit = {
    val leading = Integer.numberOfLeadingZeros(i)
    val binaryString = repeatChar('0', leading) + i.toBinaryString
    show(i,               "Heltal : ")
    show(i.asInstanceOf[Char],         "Tecken : ")
    show(binaryString,    "Binärt : ")
    show(i.toHexString,   "Hex    : ")
    show(i.toOctalString, "Oktal  : ")
  }


  import scala.io.StdIn.readLine
  import scala.util.{Try,Success,Failure}

  def loop: Unit =
    Try { readLine("Heltal annars pang: ").toInt } match {
      case Failure(e) => show(e); show("PANG!")
      case Success(i) => showInt(i); loop
    }

  def main(args: Array[String]): Unit =
    if(args.length > 0) args.foreach(i => showInt(i.toInt))
    else loop
}
\end{Code}



\QUESTEND






\WHAT{Innehållslikhet och referenslikhet i Java.}

\QUESTBEGIN

\Task  \what~ Studera och prova denna fallgrop med innehållslikhet: \href{https://github.com/bjornregnell/lth-eda016-2015/blob/master/lectures/examples/eclipse-ws/lecture-examples/src/week10/generics/TestPitfall3.java}{TestPitfall3.java}







\SOLUTION


\TaskSolved \what
     %%%TODO number  6 %%%starts with: \TODO Fallgrop med Point som in%%%



\QUESTEND




%\AdvancedTasks %%%%%%%%%%%%%%%%%


\WHAT{Implementera innehållslikhet i Java.}

\QUESTBEGIN

\Task  \what~\Pen Studera fallgropar för hur man skriver en \code{equals}-metod i Java här:
\href{http://www.artima.com/lejava/articles/equality.html}{www.artima.com/lejava/articles/equality.html} och jämför med  det fullständiga receptet för hur man skriver en välfungerande \code{equals} och \code{hashcode} i Scala här: \href{http://www.artima.com/pins1ed/object-equality.html}{www.artima.com/pins1ed/object-equality.html}

\Subtask Vilka skillnader och likheter finns vid överskuggning av equals i Java respektive Scala, som ska ge en fungerande innehållstest för en hierarki med bastyper och subtyper?

\Subtask Vilka fallgropar är gemensamma för Java och Scala?\SOLUTION


\TaskSolved \what
     %%%TODO number  7 %%%starts with: \TODO \emph{Gränssnitt i Scala %%%



\QUESTEND



%\chapter{Snabbreferens}\label{chapter:quickref}
%
%Detta appendix innehåller en snabbreferens för Scala och Java. Snabbreferensen är enda tillåtna hjälpmedel under kursens skriftliga tentamen.
%
%Lär dig vad som finns i snabbreferensen så att du snabbt hittar det du behöver och träna på hur du  effektivt kan dra nytta av den när du skriver program med papper och penna utan datorhjälpmedel.
%
%\clearpage
%~
%\clearpage
%
%\includepdf[pages={1-12}, scale=0.77, frame]{../quickref/quickref.pdf}


\end{document}
