%!TEX encoding = UTF-8 Unicode
\documentclass[a4paper]{compendium}
\usepackage[swedish]{babel}
\addto\captionsswedish{%
  \renewcommand{\appendixname}{Appendix}%
}
%TODO: Glossary
%http://tex.stackexchange.com/questions/5821/creating-a-standalone-glossary/5837#5837

\setlength{\columnsep}{16mm}

\title{
{\vspace{-3.0cm}\bf\sffamily\Huge\selectfont  Introduktion till programmering med Scala och Java}
\\ \vspace{1em}%\hspace*{1.5cm}\inputgraphics[width=0.6\textwidth]{../img/gurka} \\
{\sffamily  Kompendium 2: Uppgifter}\\\vspace{2cm}
%\includegraphics[height=4cm]{../img/scala-logo.png}
%\includegraphics[height=4cm]{../img/java-logo.png}
\includegraphics[height=12cm]{cover/gurka.jpg}
}

%\author{Redaktör: Björn Regnell}
\date{\raggedbottom%
\vspace{-2em}\begin{minipage}{1.0\textwidth}\centering
EDAA45, Lp1-2, HT 2016\\
Datavetenskap, LTH\\
Lunds Universitet\\
~\\
Kompileringsdatum: \today \\
\url{http://cs.lth.se/pgk}
\end{minipage}
}

\usepackage{multicol}

\usepackage{pgffor}  %% http://stackoverflow.com/questions/2561791/iteration-in-latex
                     %  allows:  \foreach \n in {1,...,4}{ do something with \n }

\usepackage{framed}  %  allows:   \begin{framed}\end{framed}
%\newenvironment{Slide}[2][]
%  {\begin{framed}\setlist{noitemsep}\section*{#2}}
%  {\end{framed}}

\newcommand{\SlideHeading}[1]{} %ignore slide headings
\newcommand{\Subsection}[1]{} %ignore slide sections
\newcommand{\SlideOnly}[1]{} %ignore slide font size

\newif\ifkompendium  % to allow conditional text in slides only showing up in compendium
\kompendiumtrue      % in slides: \kompendiumfalse


%!TEX encoding = UTF-8 Unicode
\newcommand{\ExeWeekONE}{expressions}
\newcommand{\LabWeekONE}{kojo}

\newcommand{\ExeWeekTWO}{programs}
\newcommand{\LabWeekTWO}{--}

\newcommand{\ExeWeekTHREE}{functions}
\newcommand{\LabWeekTHREE}{irritext}

\newcommand{\ExeWeekFOUR}{objects}
\newcommand{\LabWeekFOUR}{blockmole}

\newcommand{\ExeWeekFIVE}{classes}
\newcommand{\LabWeekFIVE}{turtlegraphics}

\newcommand{\ExeWeekSIX}{sequences}
\newcommand{\LabWeekSIX}{shuffle}

\newcommand{\ExeWeekSEVEN}{sets-maps}
\newcommand{\LabWeekSEVEN}{words}

\newcommand{\ExeWeekEIGHT}{matrices}
\newcommand{\LabWeekEIGHT}{maze}

\newcommand{\ExeWeekNINE}{inheritance}
\newcommand{\LabWeekNINE}{turtlerace-team}

\newcommand{\ExeWeekTEN}{patterns}
\newcommand{\LabWeekTEN}{chords-team}

\newcommand{\ExeWeekELEVEN}{scala-java}
\newcommand{\LabWeekELEVEN}{lthopoly-team}

\newcommand{\ExeWeekTWELVE}{sorting}
\newcommand{\LabWeekTWELVE}{survey}

\newcommand{\ExeWeekTHIRTEEN}{--}
\newcommand{\LabWeekTHIRTEEN}{Projekt}

\newcommand{\ExeWeekFOURTEEN}{threads}
\newcommand{\LabWeekFOURTEEN}{--}


\begin{document}

\pagenumbering{roman}

\frontmatter
\maketitle
%!TEX root = ../compendium.tex

\clearpage\null\thispagestyle{empty}
\vfill

{
\setlength{\parindent}{0pt}
\emph{Editor}: Björn Regnell, Faculty of Engineering LTH, Lund University. \\ 

\emph{Contributors}: 
Björn Regnell,
Per Holm,
Sandra Nilsson,
Patrik Andersson,
Gustav Cedersjö,
Maj Stenmark,
Anna Axelsson,
Roy Andersson,
Markus Borg,
Anton Klarén.
\\

\emph{Repo}: \url{https://github.com/lunduniversity/introprog} \\ \newline

This manuscript is on-going work. Contributions are welcome! \\ 
\emph{Contact}: \url{bjorn.regnell@cs.lth.se}
\\ \newline

\emph{LICENCE}: CC BY-NC-SA 4.0 \\
\url{http://creativecommons.org/licenses/by-nc-sa/4.0/}
\\ \newline
Copyright \copyright~Computer Science, LTH \& Björn Regnell. 2016. Lund. Sweden.\\
}

%%!TEX encoding = UTF-8 Unicode
%!TEX root = ../compendium.tex

\ChapterUnnum{Framstegsprotokoll}\label{progress-protocoll}


\section*{Genomförda övningar}

\vspace{1em}\noindent
{Till varje laboration hör en övning med uppgifter som utgör förberedelse inför labben. Du behöver minst behärska grunduppgifterna för att klara labben inom rimlig tid. Om du känner att du behöver öva mer på grunderna, gör då även extrauppgifterna. Om du vill fördjupa dig, gör fördjupningsuppgifterna som är på mer avancerad nivå. Kryssa för nedan vilka övningar du har gjort, så blir det lättare för din handledare att anpassa dialogen till de kunskaper du förvärvat hittills.}

\newcommand{\TickBox}{\raisebox{-.50ex}{\Large$\square$}}
\newcommand{\ExeRow}[1]{\hyperref[section:exe:#1]{\texttt{#1}} & \TickBox  &  \TickBox &  \TickBox  \\ \addlinespace }

\begin{table}[h]
%\centering
\vspace{2em}
\begin{tabular}{lccc}
\toprule \addlinespace
{\sffamily Övning} &
{\sffamily Grund} &
{\sffamily Extra} &
{\sffamily Fördjupning}\\ \addlinespace \midrule \\[-0.7em]
\ExeRow{expressions}
\ExeRow{programs}
\ExeRow{functions}
\ExeRow{data}
\ExeRow{vectors}
\ExeRow{classes}
\ExeRow{traits}
\ExeRow{matching}
\ExeRow{matrices}
\ExeRow{sorting}
\ExeRow{scalajava}
\ExeRow{threads}
\bottomrule
\end{tabular}
\end{table}

\newpage

\section*{Godkända obligatoriska moment}

\vspace{1em}\noindent
För att bli godkänd på laborationsuppgifterna och projektuppgiften måste du lösa deluppgifterna och diskutera dina lösningar med en handledare. Denna diskussion är din möjlighet att få feedback på dina lösningar. Ta vara på den!
Se till att handledaren noterar nedan när du blivit godkänd på respektive obligatorisk moment. Spara detta blad tills du fått slutbetyg i kursen.


\vspace{2.5em}\noindent Namn: \dotfill\\

\vspace{1em}\noindent Namnteckning: \dotfill\\

\newcommand{\LabRow}[1]{\\[-1.1em] \hyperref[section:lab:#1]{\texttt{#1}} & \dotfill &  \dotfill  \\ \addlinespace }

\begin{table}[h]
%\centering
\vspace{1em}
\begin{tabular}{lcc}
\toprule \addlinespace
{\sffamily\bfseries\small Lab} & {\sffamily\small Datum gk} &	
{\sffamily\small Handledares signatur + namnförtydligande}\\ \addlinespace 
%\midrule 
\\[-0.5em]
%!TEX encoding = UTF-8 Unicode
%!TEX root = ../compendium2.tex
\LabRow{kojo}
\LabRow{irritext}
\LabRow{blockmole}
\LabRow{blockbattle}
\LabRow{shuffle}
\LabRow{words}
\LabRow{life}
\LabRow{snake}
\LabRow{tabular}
\LabRow{javatext}
%\toprule
\addlinespace 
%\midrule 
\addlinespace\addlinespace
{\sffamily\small {\bfseries Projektuppgift} (välj en)	} & \dotfill &  \dotfill  \\
\addlinespace\addlinespace %\midrule
{\Large$\square$}\texttt{~~~\hyperref[section:proj:bank]{bank}} &
\multicolumn{2}{c}{\textit{Om egendef., ge kort beskrivning här:}}  \\ \addlinespace
{\Large$\square$}\texttt{~~~\hyperref[section:proj:tabular]{tabular}} \\ \addlinespace
{\Large$\square$}\texttt{~~~\hyperref[section:proj:music]{music}} \\ \addlinespace
{\Large$\square$}\texttt{~~~\hyperref[section:proj:photo]{photo}}  \\ \addlinespace
{\Large$\square$}\texttt{~~~}\textit{egendefinerad}  \\
%\dotfill  \\
\addlinespace\addlinespace
%\midrule
\addlinespace
{\sffamily\small {\bfseries Muntligt prov}} &  & \\
\addlinespace\addlinespace{}
{\Large$\square$}\texttt{~~~} godkänd & \dotfill &  \dotfill \\
\addlinespace\addlinespace\bottomrule
\end{tabular}
\end{table}

%%!TEX root = ../compendium.tex


\ChapterUnnum{Förord} 

Programmering är inte bara ett sätt att ta makten över systemen som styr vårt samhälle. Det är också ett kraftfullt verktyg för tanken. Att lära sig programmering och systemutveckling är första steget på en livslång resa av kontinuerligt lärande. Programmeringsspråk och utvecklingsverktyg kommer och går, men de grundläggande koncepten sekvens, alternativ, repetition och abstraktion som ligger bakom all mjukvara består. 

Detta kompendium utgör kursmaterial för studier i grundläggande programmering, med syfte att ge en solid bas för ingenjörsstudenter och andra som utvecklar system som innehåller mjukvara. 

Kompendiet är framtaget av, med och för studenter och lärare på universitetsnivå, och distribueras som öppen källkod. Det får användas fritt så länge erkännande ges och eventuella ändringar också publiceras som öppen källkod under samma licens som ursprungsmaterialet. På kursens hemsida \href{http://cs.lth.se/pgk}{cs.lth.se/pgk} och repo \href{http://github.com/lunduniversity/introprog}{github.com/lunduniversity/introprog} finns instruktioner om hur du kan bidra till kursmaterialet.

Läromaterialet fokuserar på lärande genom eget arbete och innehåller övningar och laborationer som är organiserade i moduler. Varje modul har ett tema och tillhörande föreläsningsanteckningar.

I kursen används språken Scala och Java för att illustrera grunderna i imperativ och objektorienterad programmering, tillsammans med elementär funktionsprogrammering. Mer avancerad objektorientering och funktionsprogrammering och  lämnas till fortsättningskurser. 



Den kanske viktigaste framgångsfaktorn vid studier i programmering är att bejaka din egen upptäckarglädje och experimentlusta. Det fantastiska med programmering är att dina egna intellektuella konstruktioner faktiskt \emph{gör} något som just \emph{du} har bestämt! Ta vara på det och prova dig fram genom att koda egna idéer -- det är kul när det funkar men minst lika lärorikt är felsökning, buggrättande och alla misslyckade försök som efter hårt arbete vänds till lyckade lösningar och bestående lärdomar. 

Välkommen i programmeringens fascinerande värld och hjärtligt lycka till med dina studier!




\setcounter{tocdepth}{1} % set headings level in table of contents
\tableofcontents
\mainmatter

\pagenumbering{arabic}

%\renewcommand{\SlideHeading}[1]{\subsection{#1}}  %numbering sections in compendium slides

\part{Uppgifter}

%!TEX encoding = UTF-8 Unicode
\chapter{Introduktion}\label{chapter:W01}
Begrepp som ingår i denna veckas studier:
\begin{multicols}{2}\begin{itemize}[noitemsep,label={$\square$},leftmargin=*]
\item sekvens
\item alternativ
\item repetition
\item abstraktion
\item editera
\item kompilera
\item exekvera
\item datorns delar
\item virtuell maskin
\item litteral
\item värde
\item uttryck
\item identifierare
\item variabel
\item typ
\item tilldelning
\item namn
\item val
\item var
\item def
\item definera och anropa funktion
\item funktionshuvud
\item funktionskropp
\item procedur
\item inbyggda grundtyper
\item Int
\item Long
\item Short
\item Double
\item Float
\item Byte
\item Char
\item String
\item println
\item typen Unit
\item enhetsvärdet ()
\item stränginterpolatorn s
\item if
\item else
\item true
\item false
\item MinValue
\item MaxValue
\item aritmetik
\item slumptal
\item math.random
\item logiska uttryck
\item de Morgans lagar
\item while-sats
\item for-sats\end{itemize}\end{multicols}

%!TEX encoding = UTF-8 Unicode
%!TEX root = ../exercises.tex

\ifPreSolution
\Exercise{\ExeWeekONE}\label{exe:W01}

\begin{Goals}
%!TEX encoding = UTF-8 Unicode

\item Förstå vad som händer när satser exekveras och uttryck evalueras.
\item Förstå sekvens, alternativ och repetition.
\item Känna till litteralerna för enkla värden, deras typer och omfång.
\item Kunna deklarera och använda variabler och tilldelning, samt kunna rita bilder av minnessituationen då variablers värden förändras.
\item Förstå skillnaden mellan olika numeriska typer, kunna omvandla mellan dessa och vara medveten om noggrannhetsproblem som kan uppstå.
\item Förstå booleska uttryck och värdena \code{true} och \code{false}, samt kunna förenkla booleska uttryck.
\item Förstå skillnaden mellan heltalsdivision och flyttalsdivision, samt användning av rest vid heltalsdivision.
\item Förstå precedensregler och användning av parenteser i uttryck.
\item Kunna använda \code{if}-satser och \code{if}-uttryck.
\item Kunna använda \code{for}-satser och \code{while}-satser.
\item Kunna använda \code{math.random()} för att generera slumptal i olika intervaller.
\item Kunna beskriva skillnader och likheter mellan en procedur och en funktion.

\end{Goals}

\begin{Preparations}
\item \StudyTheory{01}
\item Du behöver en dator med Scala och Kojo installerad, se appendix~\ref{appendix:compile} och  \ref{appendix:kojo}.
\end{Preparations}

\else

\ExerciseSolution{\ExeWeekONE}

\fi  %%% END \ifPreSolution


\BasicTasks
%%%% TODO Strukturera övningen annorlunda: atomer, sammansatta uttryck, funktiner, kojo ??}


\def\what{\emph{Para ihop begrepp med beskrivning.}}

\QUESTBEGIN

\Task \what

\vspace{1em}\noindent Koppla varje begrepp med den (förenklade) beskrivning som passar bäst: 

\begin{ConceptConnections}
  litteral & 1 & & A & att översätta kod till exekverbar form \\ 
  sträng & 2 & & B & anger ett specifikt datavärde \\ 
  sats & 3 & & C & decimaltal med begränsad noggrannhet \\ 
  uttryck & 4 & & D & bra då antalet repetitioner ej är bestämt i förväg \\ 
  funktion & 5 & & E & vid anrop sker (sido)effekt; returvärdet är tomt \\ 
  procedur & 6 & & F & sker innan exekveringen startat \\ 
  exekveringsfel & 7 & & G & bra då antalet repetitioner är bestämt i förväg \\ 
  kompileringsfel & 8 & & H & för att ändra en variabels värde \\ 
  abstrahera & 9 & & I & sker medan programmet kör \\ 
  kompilera & 10 & & J & beskriver vad data kan användas till \\ 
  typ & 11 & & K & antingen sann eller falsk \\ 
  for-sats & 12 & & L & vid anrop beräknas ett returvärde \\ 
  while-sats & 13 & & M & en kodrad som gör något; kan särskiljas med semikolon \\ 
  tilldelning & 14 & & N & kombinerar värden och funktioner till ett nytt värde \\ 
  flyttal & 15 & & O & en sekvens av tecken \\ 
  boolesk & 16 & & P & att införa nya begrepp som förenklar kodningen \\ 
\end{ConceptConnections}

\SOLUTION

\TaskSolved \what

\begin{ConceptConnections}
  litteral & 1 & ~~\Large$\leadsto$~~ &  C & anger ett specifikt datavärde \\ 
  sträng & 2 & ~~\Large$\leadsto$~~ &  G & en sekvens av tecken \\ 
  sats & 3 & ~~\Large$\leadsto$~~ &  K & en kodrad som gör något; kan särskiljas med semikolon \\ 
  uttryck & 4 & ~~\Large$\leadsto$~~ &  N & kombinerar värden och funktioner till ett nytt värde \\ 
  funktion & 5 & ~~\Large$\leadsto$~~ &  L & vid anrop beräknas ett returvärde \\ 
  procedur & 6 & ~~\Large$\leadsto$~~ &  H & vid anrop sker (sido)effekt; returvärdet är tomt \\ 
  exekveringsfel & 7 & ~~\Large$\leadsto$~~ &  P & kan inträffa medan programmet kör \\ 
  kompileringsfel & 8 & ~~\Large$\leadsto$~~ &  A & kan inträffa innan exekveringen startat \\ 
  abstrahera & 9 & ~~\Large$\leadsto$~~ &  F & att införa nya begrepp som förenklar kodningen \\ 
  kompilera & 10 & ~~\Large$\leadsto$~~ &  B & att översätta kod till exekverbar form \\ 
  typ & 11 & ~~\Large$\leadsto$~~ &  D & beskriver vad data kan användas till \\ 
  for-sats & 12 & ~~\Large$\leadsto$~~ &  O & bra då antalet repetitioner är bestämt i förväg \\ 
  while-sats & 13 & ~~\Large$\leadsto$~~ &  E & bra då antalet repetitioner ej är bestämt i förväg \\ 
  tilldelning & 14 & ~~\Large$\leadsto$~~ &  I & för att ändra en variabels värde \\ 
  flyttal & 15 & ~~\Large$\leadsto$~~ &  M & decimaltal med begränsad noggrannhet \\ 
  boolesk & 16 & ~~\Large$\leadsto$~~ &  J & antingen sann eller falsk \\ 
\end{ConceptConnections}

\QUESTEND






\def\what{\emph{Utskrift i Scala REPL.}}

\QUESTBEGIN

\Task \what 

\vspace{1em}\noindent Starta Scala REPL \Eng{Read-Evaluate-Print-Loop}.

\begin{REPLnonum}
$ scala
Welcome to Scala version 2.11.8 (Java HotSpot(TM) 64-Bit Server VM, Java 1.8).
Type in expressions to have them evaluated.
Type :help for more information.

scala> 
\end{REPLnonum}

\Subtask Skriv efter prompten \code{scala>} en sats som skriver ut en valfri (bruklig/knasig) hälsningsfras, genom anrop av proceduren \code{println} med något strängargument. Tryck på \textit{Enter} så att satsen kompileras och exekveras. 

\Subtask Skriv samma sats igen (eller tryck pil-upp) men ''glöm bort'' att skriva högerparentesen efter argumentet innan du trycker på \textit{Enter}. Vad händer?

\begin{framed}
\noindent\emph{Tips inför fortsättningen:} Det finns många användbara kortkommandon och andra trix för att jobba snabbt i REPL. Be gärna någon som kan dessa trix att visa dig hur man kan jobba snabbare. Läs appendix \ref{appendix:compile:REPL} och prova sedan att kopiera och klistra in text. Använd piltangenterna för att bläddra i historiken och Ctrl+A för att komma till början av raden, Ctrl+K för att radera resten av raden, etc.
\end{framed}



\SOLUTION 
\TaskSolved \what

\SubtaskSolved Till exempel:
\begin{REPLnonum}
scala> println("hejsan svejsan")
\end{REPLnonum}

\SubtaskSolved Om högerparentes fattas får man fortsätta skriva på nästa rad. Detta indikeras med vertikalstreck i början av varje ny rad:
\begin{REPLnonum}
scala> println("hejsan svejsan"
     | + "!" 
     | )
hejsan svejsan!
\end{REPLnonum}

\QUESTEND



\def\what{\emph{Konkatenering av strängar.}}

\QUESTBEGIN

\Task \what

\Subtask Skriv ett uttryck som konkatenerar två strängar, t.ex. \code{"gurk"} och \code{"burk"}, med hjälp av operatorn \code{+} och studera resultatet. Vad har uttrycket för värde och typ? Vilken siffra står efter ordet \code{res} i variabeln som lagrar resultatet?

\Subtask Använd resultatet från konkateneringen, t.ex. \code{res0} (byt ev. ut \code{0}:an mot siffran efter \code{res} i utskriften från förra evalueringen), och skriv ett uttryck med hjälp av operatorn \code{*} som upprepar resultatet från förra deluppgiften 42 gånger. 


\SOLUTION

\TaskSolved \what

\SubtaskSolved 
\begin{REPLnonum}
scala> "gurk" + "burk"
res1: String = gurkburk
\end{REPLnonum}
värde: \code{"gurkburk"}, typ:  \code{String}

\SubtaskSolved
\begin{REPLnonum}
scala> res1 * 42
res2: String = gurkatomatgurkatomatgurkatomatgurkatomatgurkatomatgurkatomatgurkatomatgurkatomatgurkatomatgurkatomatgurkatomatgurkatomatgurkatomatgurkatomatgurkatomatgurkatomatgurkatomatgurkatomatgurkatomatgurkatomatgurkatomatgurkatomatgurkatomatgurkatomatgurkatomatgurkatomatgurkatomatgurkatomatgurkatomatgurkatomatgurkatomatgurkatomatgurkatomatgurkatomatgurkatomatgurkatomatgurkatomatgurkatomatgurkatomatgurkatomatgurkatomatgurkatomat
\end{REPLnonum}

\QUESTEND




\def\what{\emph{När upptäcks felet?}}

\QUESTBEGIN

\Task \what 

\Subtask Vad har uttrycket \code{ "hej" * 3 } för typ och värde? Testa i REPL.

\Subtask Byt ut 3:an ovan mot ett så pass stort heltal så att minnet blir fullt. Hur börjar felmeddelandet? Är detta ett körtidsfel eller ett kompileringsfel?

\Subtask Välj ett värde på argumentet efter operatorn \code{*} så att ett typfel genereras. Hur börjar felmeddelandet? Är detta ett körtidsfel eller ett kompileringsfel?

\begin{framed}
\noindent\emph{Tips inför fortsättningen:} Gör gärna fel när du kodar så lär du dig mer! Träna på att tolka olika felmeddelanden och fråga någon om hjälp om du inte förstår. Kompilatorns utskrifter kan vara till stor hjälp, men är ibland kryptiska. Om du kör fast och inte kommer vidare själv så be om hjälp, \emph{men be om tips snarare än färdiga lösningar} så att du behåller initiativet själv och tar kontroll över nästa steg i ditt lärande.
\end{framed}


\SOLUTION

\TaskSolved \what

\SubtaskSolved Typ: \code{String}, värde: \code{"hejhejhej"}

\SubtaskSolved Körtiddsfel:
\begin{REPLnonum}
scala> "hej" * Int.MaxValue
java.lang.OutOfMemoryError: Java heap space
\end{REPLnonum}

\SubtaskSolved Kompileringsfel: (indikeras av texten \code{<console> ... error:})
\begin{REPLnonum}
scala> "hej" * true
<console>:12: error: type mismatch;
 found   : Boolean(true)
 required: Int
       "hej" * true
\end{REPLnonum}


\QUESTEND




\def\what{\emph{Litteraler och typer.}}

\QUESTBEGIN

\Task \what

\Subtask Ta hjälp av REPL-kommadot \verb+:type+ (kan förkortas \code{:t}) vid behov för att para ihop nedan litteraler med rätt typ. 

\begin{ConceptConnections}[0.35\textwidth]
  \code|1    | & 1 & & A & \code|Char   | \\ 
  \code|1L   | & 2 & & B & \code|Double | \\ 
  \code|1.0  | & 3 & & C & \code|Boolean| \\ 
  \code|1D   | & 4 & & D & \code|Int    | \\ 
  \code|1F   | & 5 & & E & \code|Boolean| \\ 
  \code|'1'  | & 6 & & F & \code|Double | \\ 
  \code|"1"| & 7 & & G & \code|Long   | \\ 
  \code|true | & 8 & & H & \code|Float  | \\ 
  \code|false| & 9 & & I & \code|String | \\ 
  \code|()   | & 10 & & J & \code|Unit   | \\ 
%\Connect{\code|1      |}  {\code|Int    |}
%\Connect{\code|1L     |}  {\code|Long   |}
%\Connect{\code|1.0    |}  {\code|Double |}
%\Connect{\code|1D     |}  {\code|Double |}
%\Connect{\code|1F     |}  {\code|Float  |}
%\Connect{\code|'1'    |}  {\code|Char   |}
%\Connect{\code|\"1\"  |}  {\code|String |}
%\Connect{\code|true   |}  {\code|Boolean|} 
%\Connect{\code|false  |}  {\code|Boolean|} 
%\Connect{\code|()     |}  {\code|Unit   |} 
\end{ConceptConnections}

\Subtask Vad händer om du adderar 1 till det största möjliga värdet av typen \code{Int}? 
\\\emph{Tips:} se snabbreferensen \footnote{\url{http://cs.lth.se/pgk/quickref/}} under rubriken ''The Scala type system'' avsnitt ''Methods on numbers''.

\Subtask Vad är skillnaden mellan typerna \code{Long} och \code{Int}?

\Subtask Vad är skillnaden mellan typerna \code{Double} och \code{Float}?


\SOLUTION

\TaskSolved \what

\SubtaskSolved 

\begin{ConceptConnections}
  \code|1    | & 1 & ~~\Large$\leadsto$~~ &  D & \code|Int    | \\ 
  \code|1L   | & 2 & ~~\Large$\leadsto$~~ &  J & \code|Long   | \\ 
  \code|1.0  | & 3 & ~~\Large$\leadsto$~~ &  G & \code|Double | \\ 
  \code|1D   | & 4 & ~~\Large$\leadsto$~~ &  C & \code|Double | \\ 
  \code|1F   | & 5 & ~~\Large$\leadsto$~~ &  B & \code|Float  | \\ 
  \code|'1'  | & 6 & ~~\Large$\leadsto$~~ &  H & \code|Char   | \\ 
  \code|"1"| & 7 & ~~\Large$\leadsto$~~ &  A & \code|String | \\ 
  \code|true | & 8 & ~~\Large$\leadsto$~~ &  E & \code|Boolean| \\ 
  \code|false| & 9 & ~~\Large$\leadsto$~~ &  F & \code|Boolean| \\ 
  \code|()   | & 10 & ~~\Large$\leadsto$~~ &  I & \code|Unit   | \\ 
%\ConnectSolved{\code|1      |}  {\code|Int    |}
%\ConnectSolved{\code|1L     |}  {\code|Long   |}
%\ConnectSolved{\code|1.0    |}  {\code|Double |}
%\ConnectSolved{\code|1D     |}  {\code|Double |}
%\ConnectSolved{\code|1F     |}  {\code|Float  |}
%\ConnectSolved{\code|'1'    |}  {\code|Char   |}
%\ConnectSolved{\code|\"1\"  |}  {\code|String |}
%\ConnectSolved{\code|true   |}  {\code|Boolean|} 
%\ConnectSolved{\code|false  |}  {\code|Boolean|} 
\end{ConceptConnections}

\SubtaskSolved Värdet går över gränsen för vad som får plats i ett 32 bitars heltal och ''börjar om'' på det minsta möjliga heltalet \code{Int.MinValue}
\begin{REPL}
scala> Int.MaxValue + 1
res3: Int = -2147483648

scala> Int.MinValue
res4: Int = -2147483648
\end{REPL}

\SubtaskSolved Båda är heltal men \code{Long} kan representera större tal än \code{Int}.

\SubtaskSolved Båda är flyttal men \code{Double} har dubbel precision och kan representera större tal med fler decimaler.



\QUESTEND





\def\what{\emph{Matematiska funktioner. Scaladoc.}}

\QUESTBEGIN

\Task \what

\Subtask Antag att du har ett schackbräde med 64 rutor. Tänk dig att du börjar med ett enda riskorn på första rutan och sedan lägger dubbelt så många riskorn i en ny hög för varje efterföljande ruta: 1, 2, 4, 8, ...  etc. Hur många riskorn\footnote{\url{https://en.wikipedia.org/wiki/Wheat_and_chessboard_problem}} blir det då i den sextiofjärde rishögen?

\emph{Tips:} Du ska beräkna $2^{64} - 1$. Om du skriver \code{math.} i REPL och trycker TAB får du se inbyggda matematiska funktioner i Scalas standardbibliotek:
\begin{REPL}
scala> math.    // Tryck TAB direkt efter punkten och betrakta listan
\end{REPL}
Använd funktionen \code{math.pow} och lämpliga argument. Om du skriver \code{math.pow} och trycker TAB \emph{två gånger} får du se funktionshuvudet med parameterlistan. 

Om du surfar till \url{http://www.scala-lang.org/api/current/} och skriver \code{math} i sökrutan och sedan, efter att du klickat på \textbf{\textsf{\small scala.math}}, skriver \textbf{\textsf{\small pow}} i rutan längre ner, så filtreras sidan och du hittar dokumentationen av \code{ def pow } som du kan klicka på och läsa mer om.   

\Subtask Definiera funktionen \code{omkrets} nedan i REPL. Går det bra att utelämna returtyp-annoteringen? Varför? Finns det anledning att ha den kvar?
\begin{Code}
def omkrets(radie: Double): Double = 2 * math.Pi * radie
\end{Code}

\Subtask Jordens (genomsnittliga) diameter (vid ekvatorn) är ca $12 750$ $km$. Anropa funktionen \code{omkrets} ovan för att beräkna hur många kilometer per dag man ungefär måste färdas om man vill åka jorden runt på 80 dagar. 

\SOLUTION

\TaskSolved \what

\SubtaskSolved Ja, returtyp-annoteringen \code{: Double} kan utelämnas. 

\begin{itemize}
\item Varför kan returtyp utelämnas?\\Eftersom kompilatorns typhärledning kan härleda returtypen. 
\item Varför kan man vilja utelämna den?\\Det blir kortare att skriva utan. 
\item Anledningar att ange returtyp: 
\begin{itemize}
\item  Med explicit returtyp får du hjälp av kompilatorn att redan under kompileringen kontrollera att uttrycket till höger om likhetstecknet har den typ som förväntas. 

\item Genom att du anger returtypen explicit får de som enbart läser metodhuvudet (och inte implementationen)
 tydligt se vad som returneras.
\end{itemize}
\end{itemize}	


\SubtaskSolved Beräkning av $2^{64} - 1$ med \code{math.pow} enligt nedan ger ungefär $1.8 \cdot 10^{19}$
\begin{REPL}
scala> math.pow(2, 64) - 1
res0: Double = 1.8446744073709552E19
\end{REPL}


\SubtaskSolved Ca $500$ $km$.
\begin{REPL}
scala> omkrets(12750 / 2) / 80
res0: Double = 500.6913291658733
\end{REPL}

\QUESTEND




\def\what{\emph{Förändringsbara variabler och tilldelning.}}

\QUESTBEGIN

\Task \what~Rita en \emph{ny} bild av datorns minne efter \emph{varje} exekverad rad 1--6 nedan. Varje bild ska visa alla variabler som finns i minnet och deras variabelnamn, typ och värde.

\begin{REPL}[numbers=left, numberstyle=\color{black}\ttfamily\scriptsize\selectfont]
scala> var a = 13
scala> var b = a + 1
scala> var c = (a + b) * 2.0
scala> b = 0
scala> a = 0
scala> c = c + 1
\end{REPL}
Efter första raden ser minnessituationen ut så här:

\MEM{a}{Int}{13}

\SOLUTION

\TaskSolved \what

\begin{tabular}{l l l}
\MEM{{\it Efter rad1:~~~~} a}{Int}{13}\\
\MEM{{\it Efter rad2:~~~~} a}{Int}{13} & \MEM{b}{Int}{14}\\
\MEM{{\it Efter rad3:~~~~} a}{Int}{13} & \MEM{b}{Int}{14} & \MEM{c}{Double}{54.0}\\
\MEM{{\it Efter rad4:~~~~} a}{Int}{13} & \MEM{b}{Int}{0} & \MEM{c}{Double}{54.0}\\
\MEM{{\it Efter rad5:~~~~} a}{Int}{0} & \MEM{b}{Int}{0} & \MEM{c}{Double}{54.0}\\
\MEM{{\it Efter rad6:~~~~} a}{Int}{0} & \MEM{b}{Int}{0} & \MEM{c}{Double}{55.0}\\
\end{tabular}

\QUESTEND


\def\what{\emph{Slumptal med \code{math.random}.}}

\QUESTBEGIN

\Task\label{exercise:expressions:roll} \what

\Subtask Vad ger funktionen \code{math.random} för resultatvärde? Vilken typ? Vad är största och minsta möjliga värde?
\\\emph{Tips:} Se scaladoc här: \Scaladoc och prova i REPL.

\Subtask Deklarera den parameterlösa funktionen \code{def roll: Int = ???} som ska representera ett tärningskast och ge ett slumpmässigt heltal mellan 1 och 6. Testa funktionen genom att anropa den många gånger. \\\emph{Tips:} Använd \code{math.random} och multiplicera och addera med lämpliga heltal. Omge beräkningen med parenteser och avsluta med \code{.toInt} för att avkorta decimaler och omvandla typen från \code{Double} till \code{Int}.

\SOLUTION

\TaskSolved \what

\SubtaskSolved Ur dokumentationen:
\begin{Code}
/** Returns a Double value with a positive sign, 
 *  greater than or equal to 0.0 and less than 1.0.
 */
def random(): Double
\end{Code}


\SubtaskSolved 
\begin{REPL}
scala> def roll: Int = (math.random * 6 + 1).toInt

scala> roll
res0: Int = 4

scala> roll
res1: Int = 1
\end{REPL}

\QUESTEND




\def\what{\emph{Repetition med \code{for}, \code{foreach} och \code{while}.}}

\QUESTBEGIN

\Task \what

\Subtask Så här kan en \code{for}-sats ser ut: 
\begin{Code}
for (i <- 1 to 10) print(i + ", ")
\end{Code}
Använd en \code{for}-sats för att skriva ut resultatet av 100 tärningskast med funktionen \code{roll} från uppgift \ref{exercise:expressions:roll}. 

\Subtask Så här kan en \code{foreach}-sats ser ut: 
\begin{Code}
(1 to 10).foreach { i => print(i + ",") }
\end{Code}
Använd en \code{foreach}-sats för att skriva ut resultatet av 100 tärningskast med funktionen \code{roll} från uppgift \ref{exercise:expressions:roll}. 

\Subtask Så här kan en \code{while}-sats ser ut: 
\begin{Code}
var i = 1
while (i <= 10) { print(i + ","); i = i + 1 }
\end{Code}
Använd en \code{while}-sats för att skriva ut resultatet av 100 tärningskast med funktionen \code{roll} från uppgift \ref{exercise:expressions:roll}. Vad händer om du glömmer \code{i = i + 1} ?


\SOLUTION

\TaskSolved \what

\SubtaskSolved \TODO

\QUESTEND


\def\what{\emph{Alternativ med \code{if}-sats och \code{if}-uttryck.}}

\QUESTBEGIN

\Task \what

\Subtask Så här kan en \code{if}-sats se ut (notera dubbla likhetstecken):
\begin{Code}
if (roll == 3) println("TRE") else println("INTE TRE") 
\end{Code}
Testa ovan i REPL. Skriv sedan en \code{for}-sats som kastar 100 tärningar och skriver ut strängen \code{"GRATTIS!"} om det blir en sexa, annars en ledsen smiley: \code{":("} 

\Subtask Så här kan ett \code{if}-uttryck se ut:
\begin{Code}
if (roll < 6) 0 else 1 
\end{Code}
Testa ovan i REPL. Skriv sedan en \code{while}-sats som kastar 100 tärningar och räknar antalet sexor. 

\SOLUTION

\TaskSolved \what

\SubtaskSolved \TODO

\QUESTEND



\def\what{\emph{Sekvens, sats och procedur.}}

\QUESTBEGIN

\Task \what

\Subtask Vad gör dessa satser? 
\begin{REPLnonum}
scala> def p = { print("san"); print("!"); println("hej")}
scala> p;p;p;p
\end{REPLnonum}

\Subtask
Använd pil-upp för att få tillbaka raden du skrev med definitionen av proceduren \code{p}. Byt plats på strängarna i utskriftsanropen i proceduren \code{p} så att utskriften blir: 
\begin{REPLnonum}
hejsan!
hejsan!
hejsan!
hejsan!
\end{REPLnonum}

\Subtask Hur tolkar kompilatorn klammerparenteser och semikolon?

\SOLUTION

\TaskSolved \what

\SubtaskSolved 
Satserna skapar denna utskrift:
\begin{REPLnonum}
san!hej
san!hej
san!hej
san!hej
\end{REPLnonum}

\SubtaskSolved 
\begin{REPLnonum}
scala> def p = { print("hej"); print("san"); println("!")}
scala> p;p;p;p
\end{REPLnonum}

\SubtaskSolved 
\begin{itemize}
\item Klammerparenteser används för att gruppera flera satser. Klammerparenteser behövs om man vill definiera en funktion som består av mer än en sats.  

\item Semikolon särskiljer flera satser. Semikolon behövs om man vill skriva många satser på samma rad.


\end{itemize}

\QUESTEND




\def\what{\emph{Heltalsdivision.}}

\QUESTBEGIN

\Task \what~Vilket värde och vilken typ hör till vilket uttryck?  Är du osäker på svaret, testa i REPL.

\begin{ConceptConnections}[0.3\textwidth]
  \code| 4 / 42      | & 1 & & A & \code|true : Boolean  | \\ 
  \code| 42.0 / 2    | & 2 & & B & \code|    2: Int      | \\ 
  \code| 42 / 4      | & 3 & & C & \code| 10.5: Double   | \\ 
  \code| 42 % 4      | & 4 & & D & \code|   10: Int      | \\ 
  \code| 4 % 42      | & 5 & & E & \code|    0: Int      | \\ 
  \code| 40 % 4 == 0 | & 6 & & F & \code|false: Boolean  | \\ 
  \code| 42 % 4 == 0 | & 7 & & G & \code|    4: Int      | \\ 
\end{ConceptConnections}

\SOLUTION

\TaskSolved \what

\begin{ConceptConnections}[0.3\textwidth]
  \code| 4 / 42      | & 1 & ~~\Large$\leadsto$~~ &  A & \code|    0: Int      | \\ 
  \code| 42.0 / 2    | & 2 & ~~\Large$\leadsto$~~ &  G & \code| 10.5: Double   | \\ 
  \code| 42 / 4      | & 3 & ~~\Large$\leadsto$~~ &  E & \code|   10: Int      | \\ 
  \code| 42 % 4      | & 4 & ~~\Large$\leadsto$~~ &  C & \code|    2: Int      | \\ 
  \code| 4 % 42      | & 5 & ~~\Large$\leadsto$~~ &  F & \code|    4: Int      | \\ 
  \code| 40 % 4 == 0 | & 6 & ~~\Large$\leadsto$~~ &  D & \code|true : Boolean  | \\ 
  \code| 42 % 4 == 0 | & 7 & ~~\Large$\leadsto$~~ &  B & \code|false: Boolean  | \\ 
\end{ConceptConnections}

\QUESTEND





\def\what{\emph{Booleska värden.}}

\QUESTBEGIN

\Task \what~Vilket värde har dessa uttryck?  % Uppgift 13

\Subtask \code{true && true}

\Subtask \code{false && true}

\Subtask \code{true || true}

\Subtask \code{false || true}

\Subtask \code{false || false}

\Subtask \code{true == true}

\Subtask \code{true != false}

\Subtask \code{true > false}

\Subtask \code{true && (1 / 0 > 1)}

\Subtask \code{false && (1 / 0 > 1)}

\SOLUTION

\TaskSolved \what

\SubtaskSolved \code{true}

\SubtaskSolved \code{false}

\SubtaskSolved \code{false}

\SubtaskSolved \code{true}

\SubtaskSolved \code{true}

\SubtaskSolved \code{false}

\SubtaskSolved \code{true}

\SubtaskSolved \code{true}

\SubtaskSolved Undantag kastas: \code{java.lang.ArithmeticException: / by zero}

\SubtaskSolved \code{false}

\QUESTEND





\def\what{\emph{Booleska variabler.}}

\QUESTBEGIN

\Task \what~Vad skrivs ut på rad 2 och 4 nedan?

\begin{REPL}
scala> var monster = false
scala> if (monster) println("akta dig!!!")
scala> monster = true
scala> if (monster) println("akta dig!!!")
\end{REPL}

\SOLUTION

\TaskSolved \what

\begin{itemize}
\item[Rad 2:] Ingenting skrivs ut.
\item[Rad 4:] \code{akta dig!!!}
\end{itemize}


\QUESTEND






\def\what{\emph{Turtle graphics med Kojo.}}

\QUESTBEGIN

\Task \what~På veckans laboration ska du använda Kojo för att verifiera att du kan använda sekvens, alternativ, repetition och abstraktion. Med Kojo kan du rita färgglada figurer med hjälp av ett lättanvänt Scala-bibliotek för \emph{turtle graphics}\footnote{\url{https://en.wikipedia.org/wiki/Turtle_graphics}}. 

Starta Kojo (se appendix \ref{appendix:kojo}). Om du inte redan har svenska menyer: välj svenska i språkmenyn och starta om Kojo.  Skriv in nedan program och tryck på den \emph{gröna} play-knappen. Notera kopplingen mellan satssekvensen och vad som händer i ritfönstret.

\begin{Code}
sudda

fram; höger
fram; vänster
färg(grön)
fram
\end{Code}
\noindent


\Subtask Vad händer om du \emph{inte} börjar programmet med \code{sudda} och kör samma program upprepade gånger? Varför är det bra att börja programmet med \code{sudda}?

\Subtask Skriv kod som ritar en kvadrat enligt bilden nedan.
\vspace{1em}\\\includegraphics[width=0.47\textwidth]{../img/kojo/kvadrat}

\noindent Prova gärna olika sätt att skriva din kod \emph{utan} att resultatet ändras: skriv satser i sekvens på flera rader eller satser i sekvens på samma rad med semikolon emellan; använd blanktecken och blanka rader i koden. Hur vill du gruppera dina satser så att de är lätta för en människa att läsa?
%Prova att ändra på \emph{ordningen} mellan satserna och studera hur resultatet påverkas. Använd den \emph{gula} play-knappen  (programspårning) för att studera exekveringen i detalj. Vad händer du klickar på satser i ditt program och på rutor i programspårningen?


\Subtask Rita en trappa enligt bilden nedan.

\includegraphics[width=0.3\textwidth]{../img/kojo/stairs}

\Subtask Rita valfri bild på valfri bakgrund med hjälp av några av procedurerna i tabellen nedan. Du kan till exempel rita en rosa triangel med lila konturer mot svart bakgrund. % \ref{lab:kojo:kojo-procedures}. 
Försök att underlätta läsbarheten av din kod med hjälp av lämpliga radbrytningar och gruppering av satser. 


\begin{table}[H]
\begin{longtable}{l l}\small
\code|fram(100)| & Paddan går framåt 100 steg (25 om argument saknas).\\
\code|färg(rosa)| & Sätter pennans färg till rosa. \\
\code|fyll(lila)| & Sätter ifyllnadsfärgen till lila. \\
\code|fyll(genomskinlig)| & Gör så att paddan \emph{inte} fyller i något när den ritar. \\
\code|bredd(20)| & Gör så att pennan får bredden 20. \\
\code|bakgrund(svart)| & Bakgrundsfärgen blir svart. \\
\code|bakgrund2(grön,gul)| & Bakgrund med övergång från grönt till gult. \\
\code|pennaNer|  & Sätter ner paddans penna så att den ritar när den går. \\
\code|pennaUpp|  & Sänker paddans penna så att den \emph{inte} ritar när den går. \\
\code|höger(45)|   & Paddan vrider sig 45 grader åt höger. \\
\code|vänster(45)| & Paddan vrider sig 45 grader åt vänster. \\
\code|hoppa|       & Paddan hoppar 25 steg utan att rita. \\
\code|hoppa(100)|  & Paddan hoppar 100 steg utan att rita. \\
\code|hoppaTill(100, 200)| & Paddan hoppar till läget (100, 200) utan att rita. \\
\code|gåTill(100, 200)|    & Paddan vrider sig och går till läget (100, 200). \\
\code|öster|   & Paddan vrider sig så att nosen pekar åt höger. \\
\code|väster|  & Paddan vrider sig så att nosen pekar åt vänster. \\
\code|norr|    & Paddan vrider sig så att nosen pekar uppåt. \\
\code|söder|   & Paddan vrider sig så att nosen pekar neråt. \\
\code|mot(100,200)|   & Paddan vrider sig så att nosen pekar mot läget (100, 200) \\
\code|sättVinkel(90)| & Paddan vrider nosen till vinkeln 90 grader. \\
\end{longtable}
%\label{lab:kojo:kojo-procedures}
%\caption{Några användbara procedurer i Kojo.}
\end{table}

\begin{framed}
\noindent\emph{Tips inför fortsättningen:} Ha gärna både REPL och Kojo igång samtidigt. Då kan du undersöka hur olika kodkonstruktioner fungerar i REPL, medan du stegvis skapar allt större program i editorn i Kojo. Detta sätt att jobba har du nytta av under resten av kursen, både om du använder en texteditor och kompilerar i terminalen, och om du använder en professionell integrerad utvecklingsmiljö. Oavsett vilka andra verktyg du kör är det användbart att ha REPL igång i ett eget fönster som hjälp i den kreativa processen, medan du jagar buggar och medan du lär dig nya koncept. Så fort du undrar hur något fungerar i Scala: fram med REPL och testa!
\end{framed}


\SOLUTION

\TaskSolved \what
 
\SubtaskSolved Genom att börja din Kojo-program med \code{sudda} så startar du exekveringen i samma utgångsläge: en tom rityta \Eng{canvas} där paddan pekar uppåt, pennan är nere och pennans färg är röd.  Då blir det lättare att resonera om vad programmet gör från början till slut, jämfört med om exekveringen beror på resultatet av tidigare exekveringar.


\SubtaskSolved
\begin{Code}
sudda

fram; vänster
fram; vänster
fram; vänster
fram; vänster
\end{Code}


\SubtaskSolved
\begin{Code}
sudda

fram; vänster
fram; höger

fram; vänster
fram; höger

fram; vänster
fram; höger

fram; vänster
\end{Code}


\QUESTEND









\clearpage

\ExtraTasks %%%%%%%%%%%%%%%%%% EXTRAUPPGIFTER



\def\what{\emph{Typ och värde.}}

\QUESTBEGIN

\Task \what~Vilket värde och vilken typ hör till vilket uttryck?  Är du osäker på svaret, testa i REPL.

\begin{ConceptConnections}[0.3\textwidth]
  \code|1.0 + 18          | & 1 & & A & \code|" ": String   | \\ 
  \code|(41 + 1).toDouble | & 2 & & B & \code|19.0: Double    | \\ 
  \code|1.042e42 + 1      | & 3 & & C & \code|57: Int         | \\ 
  \code|12E6.toLong       | & 4 & & D & \code|42.0: Double    | \\ 
  \code|32.toChar.toString| & 5 & & E & \code|48: Int         | \\ 
  \code|'A'.toInt         | & 6 & & F & \code|0: Int          | \\ 
  \code|0.toInt           | & 7 & & G & \code|1.042E42: Double| \\ 
  \code|'0'.toInt         | & 8 & & H & \code|'*': Char       | \\ 
  \code|'9'.toInt         | & 9 & & I & \code|12000000: Long  | \\ 
  \code|'A' + '0'         | & 10 & & J & \code|65: Int         | \\ 
  \code|('A' + '0').toChar| & 11 & & K & \code|'q': Char       | \\ 
  \code|"*!%#".charAt(0)| & 12 & & L & \code|113: Int        | \\ 
\end{ConceptConnections}

\SOLUTION

\TaskSolved \what

\begin{ConceptConnections}
  \code|1.0 + 18          | & 1 & ~~\Large$\leadsto$~~ &  B & \code|19.0: Double    | \\ 
  \code|(41 + 1).toDouble | & 2 & ~~\Large$\leadsto$~~ &  D & \code|42.0: Double    | \\ 
  \code|1.042e42 + 1      | & 3 & ~~\Large$\leadsto$~~ &  G & \code|1.042E42: Double| \\ 
  \code|12E6.toLong       | & 4 & ~~\Large$\leadsto$~~ &  I & \code|12000000: Long  | \\ 
  \code|32.toChar.toString| & 5 & ~~\Large$\leadsto$~~ &  A & \code|" ": String   | \\ 
  \code|'A'.toInt         | & 6 & ~~\Large$\leadsto$~~ &  J & \code|65: Int         | \\ 
  \code|0.toInt           | & 7 & ~~\Large$\leadsto$~~ &  F & \code|0: Int          | \\ 
  \code|'0'.toInt         | & 8 & ~~\Large$\leadsto$~~ &  E & \code|48: Int         | \\ 
  \code|'9'.toInt         | & 9 & ~~\Large$\leadsto$~~ &  C & \code|57: Int         | \\ 
  \code|'A' + '0'         | & 10 & ~~\Large$\leadsto$~~ &  L & \code|113: Int        | \\ 
  \code|('A' + '0').toChar| & 11 & ~~\Large$\leadsto$~~ &  K & \code|'q': Char       | \\ 
  \code|"*!%#".charAt(0)| & 12 & ~~\Large$\leadsto$~~ &  H & \code|'*': Char       | \\ 
\end{ConceptConnections}

%\Subtask \code{1.0 + 18}
%
%\Subtask \code{(41 + 1).toDouble}
%
%\Subtask \code{1.042e42 + 1}
%
%\Subtask \code{12E6.toLong}
%
%\Subtask \code{"gurk" + 'a'}
%
%\Subtask \code{32.toChar.toString}
%
%\Subtask \code{'A'.toInt}
%
%\Subtask \linebreak[0] \code{'0'.toInt}
%
%\Subtask \code{'0'.toInt}
%
%\Subtask \code{'9'.toInt}
%
%\Subtask \code{'A' + '0'}
%
%\Subtask \code{('A' + '0').toChar}
%
%\Subtask \code{"*!%#".charAt(0)}
%%%%%%%%%%%%%%%%%%%%%%%%%%%%%%%%%%%%%%%%%%%%%%%%
%\SubtaskSolved \code{Double, 19}
%
%\SubtaskSolved \code{Double, 42}
%
%\SubtaskSolved \code{Double, 1.042E42}
%
%\SubtaskSolved \code{Long, 12000000}
%
%\SubtaskSolved \code{String, gurka}
%
%\SubtaskSolved \code{String, " "}
%
%\SubtaskSolved \code{Int, 65}
%
%\SubtaskSolved \code{Int, 48}
%
%\SubtaskSolved \code{Int,49}
%
%\SubtaskSolved \code{Int,57}
%
%\SubtaskSolved \code{Int, 113}
%
%\SubtaskSolved \code{Char, 'q'}
%
%\SubtaskSolved \code{Char, '*'}


\QUESTEND




\def\what{\emph{Satser och uttryck.}}

\QUESTBEGIN

\Task \what

\Subtask Vad är det för skillnad på en sats och ett uttryck?

\Subtask Ge exempel på satser som inte är uttryck?

\Subtask Förklara vad som händer för varje evaluerad rad:
\begin{REPL}
scala> def värdeSaknas = ()
scala> värdeSaknas
scala> värdeSaknas.toString
scala> println(värdeSaknas)
scala> println(println("hej"))
\end{REPL}

\Subtask Vilken typ har literalen \code{()}?

\Subtask Vilken returtyp har \code{println}?

\SOLUTION

\TaskSolved \what

\SubtaskSolved  Ett utryck kan evalueras och resulterar då i ett användbart värde. En sats \emph{gör} något (t.ex. skriver ut något), men resulterat inte i något användbart värde.

\SubtaskSolved \code{println()}

\SubtaskSolved 

 Värdesaknas innehåller Unit

 Skriver ut \code{Unit}

 Skriver ut \code{"()"}

 Skriver ut \code{"()"}

 Skriver först ut hej med det innersta anropet och sen \code{()} med det yttre anropet

\SubtaskSolved  \code{Unit}

\SubtaskSolved  \code{Unit}

\QUESTEND



\def\what{\emph{Procedur med parameter.} \TODO}

\QUESTBEGIN

\Task \what~En procedur är en funktion som orsakar en effekt, till exempel en utskrift eller en variabeltilldelning, men som inte returnerar något intressant resultatvärde. \footnote{I Scala är procedurer funktioner som returnerar det \emph{tomma värdet}, vilket skrivs \code{()} och är av typen \code{Unit}. I Java och flera andra språk finns inget tomt värde och man har en specialsyntax för procedurer som använder nyckelordet \code{void}. }

\Subtask Deklarera en förändringsbar variabel \code{highscore} som initieras till 0.

\Subtask Deklarera en procedur \code{updateHighscore} som tar en parameter \code{points} och tilldelar \code{highscore} \TODO ...


\SOLUTION

\TaskSolved \what

\SubtaskSolved 

\QUESTEND





\def\what{\emph{\code{if}\textit{-sats}.}}

\QUESTBEGIN

\Task \what~För varje rad nedan, beskriv vad som skrivs ut.  % Uppgift 18
\begin{REPL}
scala> if (!true) println("sant") else println("falskt")
scala> if (!false) println("sant") else println("falskt")
scala> def singlaSlant = if (math.random > 0.5) "krona" else "klave"
scala> for (i <- 1 to 5) print(s"$i:$singlaSlant ")
\end{REPL}

\SOLUTION

\TaskSolved \what

\begin{enumerate}
\item Utskrift: \code{falskt}
\item Utskrift: \code{sant}
\item Inget skrivs ut, funktionen deklareras men körs ej.
\item Utskrift: code{1:krona 2:klave 3:krona 4:krona 5:klave }
\end{enumerate}

\QUESTEND





\def\what{\emph{\code{if}\textit{-uttryck}.}}

\QUESTBEGIN

\Task  Deklarera följande variabler med nedan initialvärden:  

\begin{REPLnonum}
scala> var grönsak = "gurka"
scala> var frukt = "banan"
\end{REPLnonum}

Ange för varje rad nedan vad uttrycket har för värde och typ:
\begin{REPLnonum}
scala> if (grönsak == "tomat") "gott" else "inte gott" 
scala> if (frukt == "banan") "gott" else "inte gott" 
scala> if (true) grönsak else 42 
scala> if (false) grönsak else 42 
\end{REPLnonum}

\SOLUTION


\TaskSolved \what~Notera typen \code{Any} på de sista två uttrycken.

\begin{REPLnonum}
scala> if (grönsak == "tomat") "gott" else "inte gott"
res0: String = inte gott

scala> if (frukt == "banan") "gott" else "inte gott"
res1: String = gott

scala> if (true) grönsak else 42
res2: Any = gurka

scala> if (false) grönsak else 42
res3: Any = 42
\end{REPLnonum}


\QUESTEND






\def\what{\emph{QUESTTEMPLATE}}

\QUESTBEGIN

\Task \what

\Subtask

\SOLUTION

\TaskSolved \what

\SubtaskSolved 

\QUESTEND




\clearpage

\AdvancedTasks   %%%%%%%%%%%%%%%%%%% FÖRDJUPNINGSUPPGIFTER




\def\what{\emph{Stränginterpolatorn \code{s}.}}

\QUESTBEGIN

\Task \what~Med ett \code{s} framför en strängliteral får man hjälp av kompilatorn att, på ett typsäkert sätt, infoga variabelvärden i en sträng. 
Variablernas namn ska föregås med ett dollartecken, t.ex. \code{s"Hej $namn"}.  
Om man vill evaluera ett uttryck placeras detta inom klammer direkt efter dollartecknet, t.ex.
\code/s"Dubbla längden: ${namn.size * 2}"/  

\Subtask Vad skrivs ut nedan?
\begin{REPL}
scala> val f = "Kim"
scala> val e = "Finkodare"
scala> println(s"Namnet '$f $e' har ${f.size + e.size} bokstäver.")
\end{REPL}

\Subtask Skapa följande utskrifter med hjälp av stränginterpolatorn \code{s} och variablerna \code{f} och \code{e} i föregående deluppgift.
\begin{REPL}
Kim har 3 bokstäver.
Finkodare har 9 bokstäver.
\end{REPL}

\SOLUTION

\TaskSolved \what

\SubtaskSolved 
\begin{REPLnonum}
Namnet 'Kim Finkodare' har 12 bokstäver.
\end{REPLnonum}

\SubtaskSolved 
\begin{REPLnonum}
println(s"$f har  ${f.size} bokstäver.")
println(s"$e har  ${e.size} bokstäver.")
\end{REPLnonum}

\QUESTEND






\def\what{\emph{Flyttalsaritmetik}}

\QUESTBEGIN

\Task \what

\Subtask Vilket är det minsta positiva värdet av typen \code{Double}?

\Subtask Vad är värdet av detta uttryck? Varför blir det så?
\begin{REPL}
scala> Double.MaxValue + Double.MinPositiveValue == Double.MaxValue
\end{REPL}

\SOLUTION

\TaskSolved \what

\SubtaskSolved 

\begin{REPL}
scala> Double.MinPositiveValue
res0: Double = 4.9E-324
\end{REPL}

\SubtaskSolved 

\begin{REPL}
scala> Double.MaxValue + Double.MinPositiveValue == Double.MaxValue
res2: Boolean = true
\end{REPL}

\QUESTEND




\def\what{\emph{Stora tal.}}

\QUESTBEGIN

\Task \what~Om vi vill beräkna $2^{64} -1$ som ett exakt heltal\footnote{\url{https://en.wikipedia.org/wiki/Wheat_and_chessboard_problem}} blir det större än \code{Int.MaxValue}, så vi kan tyvärr inte använda snabba \code{Int}. Till vår räddning: \code{BigInt} 

\Subtask Läs om \code{BigInt} och \code{BigDecimal} på \Scaladoc \\ Notera vad de kan användas till. 

\Subtask Du skapar ett \code{BigInt}-heltal med \code{BigInt(2)} och kan anropa funktionen \code{pow} på en \code{BigInt} med punktnotation. Beräkna $2^{64} -1$ som ett exakt heltal.

\Subtask Vilka nackdelar finns med \code{BigInt} och \code{BigDecimal}?

\SOLUTION

\TaskSolved \what

\SubtaskSolved \code{BigInt} kan användas i stället för \code{Int} vid mycket stora heltal. \code{BigDecimal} kan användas i stället för \code{Double} vid mycket stora decimaltal.

\SubtaskSolved 
\begin{REPL}
scala> BigInt(2).pow(64)
res0: scala.math.BigInt = 18446744073709551616
\end{REPL}

\SubtaskSolved Beräkningar går mycket långsammare och de är lite krångligare att använda.

\QUESTEND





\def\what{\emph{Precedensregler}}

\QUESTBEGIN

\Task \what~Evalueringsordningen kan styras med parenteser. Vilket värde och vilken typ har följande uttryck? 

\Subtask \code{23 + 2 * 2 + (23 + 2) * 2}

\Subtask \code{(-(2 - 42)) / (1 + 1 + 1)}

\Subtask \code{(-(2 - 42)) / (-1)/(1 + 1 + 1)}

\SOLUTION

\TaskSolved \what

\SubtaskSolved \code{77:  Int}

\SubtaskSolved \code{13: Int}

\SubtaskSolved \code{-13: Int}

\QUESTEND






\def\what{\emph{QUESTTEMPLATE}}

\QUESTBEGIN

\Task \what

\Subtask

\SOLUTION

\TaskSolved \what

\SubtaskSolved 

\QUESTEND




\subsection{TODO}

\TODO{SAKERNA NEDAN SKA FLYTTAS/UPPDATERAS/TAS BORT???} 
%%%%%%%%%%%%%%%%%%%%%%%%%%%%%%%%%%%%%%%%%%%%%%%%
%%%%%%%%%%%%%%%%%%%%%%%%%%%%%%%%%%%%%%%%%%%%%%%%
%%%%%%%%%%%%%%%%%%%%%%%%%%%%%%%%%%%%%%%%%%%%%%%%
%%%%%%%%%%%%%%%%%%%%%%%%%%%%%%%%%%%%%%%%%%%%%%%%
%%%%%%%%%%%%%%%%%%%%%%%%%%%%%%%%%%%%%%%%%%%%%%%%
%%%%%%%%%%%%%%%%%%%%%%%%%%%%%%%%%%%%%%%%%%%%%%%%
%%%%%%%%%%%%%%%%%%%%%%%%%%%%%%%%%%%%%%%%%%%%%%%%
%%%%%%%%%%%%%%%%%%%%%%%%%%%%%%%%%%%%%%%%%%%%%%%%
%%%%%%%%%%%%%%%%%%%%%%%%%%%%%%%%%%%%%%%%%%%%%%%%


\ifPreSolution  %%% TODO remove \fi at end of file and break sultions into pieces





\Task Klassen \code{java.lang.Math} och paketobjektet \code{scala.math}. % Uppgift 11
Genom att trycka på tab tagenten kan man se vad som finns i olika paket.

\begin{REPL}
scala> java.    //tryck TAB efter punkten
applet   awt   beans   io   lang   math   net   nio   rmi   security   sql

scala>
\end{REPL}

\Subtask Undersök genom att trycka på Tab-tangenten, vilka funktioner som finns i \code{Math} och \code{math}. Vad heter konstanten $\pi$ i java.lang.Math respektive scala.math?

\begin{REPL}
scala> java.lang.Math.    //tryck TAB efter punkten
scala> scala.math.        //tryck TAB efter punkten
\end{REPL}

\Subtask Undersök dokumentationen för klassen \code{java.lang.Math} här: \\ \url{https://docs.oracle.com/javase/8/docs/api/java/lang/Math.html} \\
Vad gör \code{java.lang.Math.hypot}?

\Subtask Undersök dokumentationen för paketobjektet \code{scala.math} här: \\
\url{http://www.scala-lang.org/api/current/#scala.math.package} \\
Ge exempel på någon funktion i \code{java.lang.Math} som inte finns i \code{scala.math}.

%\TaskSection{Noggrannhet och undantag i aritmetiska uttryck}

\Task Vad händer här? Notera undantag \Eng{exceptions} och noggrannhetsproblem. % Uppgift 12

\Subtask \code{Int.MaxValue} + 1

\Subtask \code{1 / 0}

\Subtask \code{1E8 + 1E-8}

\Subtask \code{1E9 + 1E-9}

\Subtask \code{math.pow(math.hypot(3,6), 2)}

\Subtask \code{1.0 / 0}

\Subtask \code{(1.0 / 0).toInt}

\Subtask \code{math.sqrt(-1)}

\Subtask \code{math.sqrt(Double.NaN)}

\Subtask \code{throw new Exception("PANG!!!")}





\Task \textit{Deklarationer: \code{var}, \code{val}, \code{def}}. Evaluera varje rad nedan i tur och ordning i Scala REPL.  % Uppgift 15
\begin{REPL}[numbers=left, numberstyle=\color{black}\ttfamily\scriptsize\selectfont]
scala> var x = 30
scala> x + 1
scala> x
scala> x = x + 1
scala> x
scala> x == x + 1
scala> val y = 20
scala> y = y + 1
scala> var z = {println("gurka"); 10}
scala> def w = {println("gurka"); 10}
scala> z
scala> z
scala> z = z + 1
scala> w
scala> w
scala> w = w + 1
\end{REPL}

\Subtask För varje rad ovan: förklara för vad som händer.

\Subtask Vilka rader ger kompileringsfel och i så fall vilket och varför?

\Subtask\Pen Vad är det för skillnad på \code{var}, \code{val} och \code{def}?

\Subtask\Pen Tilldela variabeln \code{val even } värdet av ett uttryck som med modulo-operatorn \code
och olikhetsoperatorn \code{!=} testar om ett tal \code{n} är udda.


\Task\Pen \emph{Tilldelningsoperatorer.} Man kan förkorta en tilldelningssats som förändrar en variabel, t.ex. \code{x = x + 1}, genom att använda så kallade tilldelningsoperatorer och skriva \code{x += 1} som betyder samma sak. Rita en ny bild av datorns minne efter varje evaluerad rad nedan. Bilderna ska visa variablers namn, typ och värde.  % Uppgift 16

\begin{REPL}
scala> var a = 40
scala> var b = a + 40
scala> a += 10
scala> b -= 10
scala> a *= 2
scala> b /= 2
\end{REPL}



\Task \emph{Stränginterpolatorn \code{s}.} Man behöver ofta skapa strängar som innehåller variabelvärden. Med ett \code{s} framför en strängliteral får man hjälp av kompilatorn att, på ett typsäkert sätt, infoga variabelvärden i en sträng. Variablernas namn ska föregås med ett dollartecken, t.ex. \code{s"Hej $namn"}.  Om man vill evaluera ett uttryck placeras detta inom klammer direkt efter dollartecknet, t.ex.
\code/s"Dubbla längden: ${namn.size * 2}"/  % Uppgift 17

\begin{REPL}
scala> val f = "Kim"
scala> val e = "Finkodare"
scala> val tot = f.size + e.size
scala> println(s"Namnet '$f $e' har $tot bokstäver.")
scala> println(s"Efternamnet '$e' har ${e.size} bokstäver.")
\end{REPL}

\Subtask Vad skrivs ut ovan?

\Subtask Skapa följande utskrifter med hjälp av stränginterpolatorn \code{s} och lämpliga variabler.
\begin{REPL}
Namnet 'Kim' har 3 bokstäver.
Namnet 'Finkodare' har 9 bokstäver.
\end{REPL}



\Task \code{if}\textit{-sats}.För varje rad nedan; förklara vad som händer.  % Uppgift 18
\begin{REPL}
scala> if (true) println("sant") else println("falskt")
scala> if (false) println("sant") else println("falskt")
scala> if (!true) println("sant") else println("falskt")
scala> if (!false) println("sant") else println("falskt")
scala> def singlaSlant =
scala> 	 if (math.random > 0.5) print(" krona") else print(" klave")
scala> singlaSlant; singlaSlant; singlaSlant
\end{REPL}


\Task \code{if}\textit{-uttryck}. Deklarera följande variabler med nedan initialvärden:  % Uppgift 19

\begin{REPLnonum}
scala> var grönsak = "gurka"
scala> var frukt = "banan"
\end{REPLnonum}

Vad har följande uttryck för värden och typ?

\Subtask \code{if (grönsak == "tomat") "gott" else "inte gott" }

\Subtask \code{if (frukt == "banan") "gott" else "inte gott" }

\Subtask \code{if (frukt.size == grönsak.size) "lika stora" else "olika stora" }

\Subtask \code{if (true) grönsak else frukt }

\Subtask \code{if (false) grönsak else frukt }


\Task \code{for}\textit{-sats}.  Med bakåtpilen \texttt{<-} kan man i en \code{for}-sats ange vilka värden som ska gås igenom i sekvens. Vid varje runda i loopen får en lokal variabel ett nytt värde i sekvensen. % Uppgift 20

\Subtask Vad ger nedan \code{for}-satser för utskrift?

\begin{REPL}
scala> for (i <- 1 to 10) print(i + ", ")
scala> for (i <- 1 until 10) print(i + ", ")
scala> for (i <- 1 to 5) print((i * 2) + ", ")
scala> for (i <- 1 to 92 by 10) print(i + ", ")
scala> for (i <- 10 to 1 by -1) print(i + ", ")
\end{REPL}

\Subtask Skriv en \code{for}-sats som ger följande utskrift:
\begin{REPLnonum}
A1, A4, A7, A10, A13, A16, A19, A22, A25, A28, A31, A34, A37, A40, A43,
\end{REPLnonum}

\Task Repetition med metoden \code{foreach}. Efter framåtpilen \texttt{=>} (se nedan) anges vad som ska hända för varje element som gås igenom sekventiellt. Vid varje runda i loopen får en lokal variabel ett nytt värde i sekvensen.   % Uppgift 21

\Subtask Vad ger nedan satser för utskrifter?

\begin{REPL}
scala> (9 to 19).foreach{i => print(i + ", ")}
scala> (1 until 20).foreach{i => print(i + ", ")}
scala> (0 to 33 by 3).foreach{i => print(i + ", ")}
\end{REPL}

\Subtask Använd \code{foreach} och skriv ut följande:
\begin{REPLnonum}
B33, B30, B27, B24, B21, B18, B15, B12, B9, B6, B3, B0,
\end{REPLnonum}

\Task \code{while}\textit{-sats}. En sats eller ett block med satser upprepas så länge ett villkor är sant.  % Uppgift 22

\Subtask Vad ger nedan satser för utskrifter?
\begin{REPL}
scala> var i = 0
scala> while (i < 10) { println(i); i = i + 1 }
scala> var j = 0; while (j <= 10) { println(j); j = j + 2 }; println(j)
\end{REPL}

\Subtask Skriv en \code{while}-sats som ger följande utskrift. Använd en variabel \code{k} som initialiseras till 1.
\begin{REPLnonum}
A1, A4, A7, A10, A13, A16, A19, A22, A25, A28, A31, A34, A37, A40, A43,
\end{REPLnonum}

\Subtask\Pen Vilken av \code{for}, \code{while} och \code{foreach} är kortast att skriva om man vill repetera mer än en sats 100 gånger? Vilken tycker du är lättast att läsa?

\Task \textit{Slumptal}. Undersök vad dokumentationen säger om funktionen \code{scala.math.random}:\\  % Uppgift 23
\url{http://www.scala-lang.org/api/current/#scala.math.package}

\Subtask\Pen Vilken typ har värdet som returneras av funktionen \code{random}?

\Subtask\Pen Vilket är det minsta respektive största värde som kan returneras?

\Subtask\Pen Är \code{random} en \textit{äkta} funktion \Eng{pure function} i matematisk mening?

\Subtask Anropa funktionen \code{math.random} upprepade gånger och notera vad som händer. Använd pil-upp-tangenten.
\begin{REPLnonum}
scala> math.random
\end{REPLnonum}


\Subtask Vad händer? Använd \textit{pil-upp} och kör nedan \code{for}-sats flera gånger. Förklara vad som sker.

\begin{REPLnonum}
scala> for (i <- 1 to 20) println((math.random * 3 + 1).toInt)
\end{REPLnonum}

\Subtask Skriv en \code{for}-sats som skriver ut 100 slumpmässiga heltal från 0 till och med 9 på var sin rad.

\begin{REPLnonum}
scala> for (i <- 1 to 100) println(???)
\end{REPLnonum}

\Subtask Skriv en \code{for}-sats som skriver ut 100 slumpmässiga heltal från 1 till och med 6 på samma rad.

\begin{REPLnonum}
scala> for (i <- 1 to 100) print(???)
\end{REPLnonum}


\Subtask Använd \textit{pil-upp} och kör nedan \code{while}-sats flera gånger. Förklara vad som sker.

\begin{REPLnonum}
scala> while (math.random > 0.2) println("gurka")
\end{REPLnonum}

\Subtask Ändra i \code{while}-satsen ovan så att sannolikheten ökar att riktigt många strängar ska skrivas ut.

\Subtask Förklara vad som händer nedan.
\begin{REPL}
scala> var slumptal = math.random
scala> while (slumptal > 0.2) { println(slumptal); slumptal = math.random }
\end{REPL}

\Task\Pen \textit{Logik och De Morgans Lagar}. Förenkla följande uttryck. Antag att \code{poäng} och \code{highscore} är heltalsvariabler medan \code{klar} är av typen \code{Boolean}.
  % Uppgift 24

\Subtask \code{poäng > 100 && poäng > 1000}

\Subtask \code{poäng > 100 || poäng > 1000}

\Subtask \code{!(poäng > highscore)}

\Subtask \code{!(poäng > 0 && poäng < highscore) }

\Subtask \code{!(poäng < 0 || poäng > highscore) }

\Subtask \code{klar == true}

\Subtask \code{klar == false}


\clearpage

\ExtraTasks

\Task \textit{Slumptal}.

\Subtask Ersätt \code{???} nedan med literaler så att \code{tärning} returnerar ett slumpmässigt heltal mellan 1 och 6.
\begin{REPLnonum}
scala> def tärning = (math.random * ??? + ???).toInt
\end{REPLnonum}

\Subtask Ersätt \code{???} med literaler så att \code{rnd} blir ett decimaltal med max en decimal mellan 0.0 och 1.0.
\begin{REPLnonum}
scala> def rnd = math.round(math.random * ???) / ???
\end{REPLnonum}

\Subtask Vad blir det för skillnad om \code{math.round} ersätts med \code{math.floor} ovan? (Se dokumentationen av \code{java.lang.Math.round} och \code{java.lang.Math.floor}.)

\Task Undersök vad som finns i paketet \code{scala.math} genom att studera dess dokumentation: \href{http://www.scala-lang.org/api/current/#scala.math.package}{www.scala-lang.org/api/current/\#scala.math.package} och gör några matematiska beräkningar i REPL som använder olika funktioner i \code{math}-paketet.

\Task\Pen Antag att du byter plats mellan satsen efter villkoret och satsen efter \code{else} i \code{if}-satsen nedan. Hur kan du ändra i villkoret så att det ändå skrivs ut samma sak som före bytet?
\begin{Code}
if (x == 42) println("the meaning of it all") else println(":(")
\end{Code}

\Task\Pen Rita en ny bild av datorns minne efter varje evaluerad rad nedan. Bilderna ska visa variablers namn, typ och värde.
\begin{REPL}
scala> var x = 42
scala> var y = x + 1
scala> x += -1
scala> y -= 1
\end{REPL}

\Task Skapa med hjälp av \code{while} några olika oändliga loopar som skriver ut olika saker vid varje loop-runda.

\Task Hitta på några egna övningar för att träna mer på De Morgans lagar.



\clearpage

\AdvancedTasks

\Task Läs om moduloräkning här \href{https://en.wikipedia.org/wiki/Modulo\_operation}{en.wikipedia.org/wiki/Modulo\_operation} och undersök hur det blir med olika tecken (positivt resp. negativt) på divisor och dividend.



\Task Läs om identifierare i Scala och speciellt \emph{literal identifiers} här: \url{http://www.artima.com/pins1ed/functional-objects.html#6.10}.

\Subtask Förklara vad som händer nedan:
\begin{REPLnonum}
scala> val `konstig val` = 42
scala> println(`konstig val`)
\end{REPLnonum}

\Subtask Scala och Java har olika uppsättningar med reserverade ord. På vilket sätt kan ''backticks'' vara använbart med anledning av detta?


\Task Sök upp dokumentationen för \code{java.lang.Integer}.

\Subtask Undersök i REPL hur metoderna \code{toBinaryString} och \code{toHexString} fungerar.

\Subtask Vad betyder literalen \code{0x2a}?

\Task Typannoteringar skapas genom att i ett uttryck placera ett kolon följt av en typ, vid behov  omslutet av en parentes. Skapa ett större uttryck med typannoteringar och försök få kompilatorn att kontrollera typen på intressanta ställen. Märk att typannoteringar också ibland kan användas för att konvertera mellan numeriska typer.


\Task Förklara vad som händer nedan:
\begin{REPL}
scala> var i = 42
scala> i += 1
scala> i *= 2
scala> i /= 3
\end{REPL}


\Task Läs om BigInt och BigDecimal här: \href{http://alvinalexander.com/scala/how-to-use-large-integer-decimal-numbers-in-scala-bigint-bigdecimal}{alvinalexander.com/scala/how-to-use-large-integer-decimal-numbers-in-scala-bigint-bigdecimal} och prova att skapa riktigt stora tal med hjälp av metoden \code{pow} på BigInt och tal med riktigt många decimaler med BigDecimal dess metod \code{pow}.

\Task Sök upp dokumentationtionen för \code{java.lang.Math.multiplyExact} och läs om vad den metoden gör.

\Subtask Vad händer här?
\begin{REPLnonum}
scala> Math.multiplyExact(2, 42)
scala> Math.multiplyExact(Int.MaxValue, Int.MaxValue)
\end{REPLnonum}

\Subtask\Pen Varför kan man vilja använda \code{java.lang.Math.multiplyExact} i stället för ''vanlig'' multiplikation?



\Subtask\Pen Sök med Ctrl+F i webbläsaren och efter förekomster av texten \textit{''overflow''} i javadoc för klassen \code{java.lang.Math} i JDK 8. Vad är ''overflow''? Vilka metoder finns i \code{java.lang.Math} som hjälper dig att upptäcka om det blir overflow?

\Task Använda Scala REPL för att undersöka konstanterna nedan. Vilket av dessa värden är negativt? Vad kan man ha för praktisk nytta av dessa värden i ett program som gör flyttalsberäkningar?

\Subtask \code{java.lang.Double.MIN_VALUE}

\Subtask \code{scala.Double.MinValue}

\Subtask \code{scala.Double.MinPositiveValue}

\Task För typerna \code{Byte}, \code{Short}, \code{Char}, \code{Int}, \code{Long}, \code{Float}, \code{Double}: Undersök hur många bitar som behövs för att representera varje typs omfång? \\*
\textit{Tips:} Några användbara uttryck: \\*
 \code{Integer.toBinaryString(Int.MaxValue + 1).size} \\*
 \code{Integer.toBinaryString((math.pow(2,16) - 1).toInt).size} \\*
 \code{1 + math.log(Long.MaxValue)/math.log(2)}
Se även språkspecifikationen för Scala, kapitlet om heltalsliteraler: \\
\url{http://www.scala-lang.org/files/archive/spec/2.11/01-lexical-syntax.html#integer-literals}

\Subtask Undersök källkoden för paketobjektet \code{scala.math} här: \\
\url{https://github.com/scala/scala/blob/v2.11.7/src/library/scala/math/package.scala} \\
Hur många olika överlagrade varianter av funktionen \code{abs} finns det och för vilka parametertyper är den definierad?

\Task Läs mer om stränginterpolatorer här:\\ \href{http://docs.scala-lang.org/overviews/core/string-interpolation.html}{docs.scala-lang.org/overviews/core/string-interpolation.html} \\ Hur kan du använda \code{f}-interpolatorn för att göra följande utskrift i REPL? Byt ut \code{???} mot lämpliga tecken.
\begin{REPLnonum}
scala> val g: Double = 1 / 3.0
scala> val s: String = f"Gurkan är ??? meter lång"
scala> println(s)
Gurkan är 0.333 meter lång
\end{REPLnonum}

\fi %%% TODO fix solutions




%!TEX encoding = UTF-8 Unicode
%!TEX root = ../labs.tex

\Lab{\LabWeekONE}
%\externaldocument{compendium}
\begin{Goals}
%!TEX encoding = UTF-8 Unicode
%!TEX root = ../compendium2.tex

\item Kunna tillämpa och kombinera principerna sekvens, alternativ, repetition, och abstraktion i skapandet av egna program om minst 20 rader kod.
\item Kunna förklara vad ett program gör i termer av sekvens, alternativ, repetition, och abstraktion.
\item Kunna formatera egna program så att de blir lätta att läsa och förstå.
\item Kunna förklara vad en variabel är och kunna deklarera oföränderliga och förändringsbara variabler, samt göra tilldelningar.
\item Kunna genomföra upprepade varv i cykeln \emph{editera-exekvera-felsöka/förbättra} för att stegvis bygga upp allt mer utvecklade program.

\end{Goals}

\begin{Preparations}
\item Repetera veckans föreläsningsmaterial.
\item \DoExercise{\ExeWeekONE}{01}%Gör övning {\tt \ExeWeekONE} i kapitel \ref{exe:W01}.
\item Läs om Kojo i appendix \ref{appendix:kojo}. Kojo Desktop är förinstallerat på LTH:s datorer; om du vill installera Kojo Desktop på din egen dator, följ instruktionerna i \ref{appendix:ide:kojo:install}.
\item Läs igenom hela laborationen nedan. Fundera på möjliga lösningar till de uppgifter som är markerade med en penna i marginalen.
\item Hämta given kod via \href{https://github.com/lunduniversity/introprog/tree/master/workspace/}{kursen github-plats} eller via hemsidan under \href{https://cs.lth.se/pgk/download/}{Download}.
% \item Ladda hem och studera översiktligt detta dokument (25 sidor, det räcker att du bläddrar igenom dokumentet och får en uppfattning om hur Kojo kan användas): \\ ''Introduction to Kojo'' \url{http://www.kogics.net/kojo-ebooks#intro}
\end{Preparations}

\subsection{Obligatoriska uppgifter}

Om det förekommer en penna i marginalen ska du anteckna något inför redovisningen.


%%%%%%%%%%%%%%NEDAN ÄR FLYTTAT TILL ÖVNING 1 FÖR ATT GÖRA TYDLIGARE KOPPLING MELLAN LABBAR OCH ÖVN
%\Task \textit{Sekvens}.
%
%\Subtask Starta Kojo. Om du inte redan har svenska menyer: välj svenska i språkmenyn och starta om Kojo.  Skriv in nedan program och tryck på den \emph{gröna} play-knappen.
%
%\begin{Code}
%sudda
%
%fram; höger
%fram; vänster
%färg(grön)
%fram
%\end{Code}
%\noindent
%%Genom att börja din Kojo-program med \code{sudda} så startar du exekveringen i samma utgångsläge: en tom canvas där paddan pekar uppåt, pennan är nere och pennans färg är röd.
%%Då blir det lättare att resonera om vad programmet gör från början till slut, jämfört med om exekveringen beror på resultatet av tidigare exekveringar.
%
%\Subtask\Pen Vad händer om du \emph{inte} börjar programmet med \code{sudda} och kör samma program upprepade gånger? Varför är det bra att börja programmet med \code{sudda}?
%
%\Subtask Rita en kvadrat enligt bilden nedan.
%\vspace{1em}\\\includegraphics[width=0.45\textwidth]{../img/kojo/kvadrat}
%
%\Subtask Prova olika sätt att skriva din kod \emph{utan} att resultatet ändras: skriv satser i sekvens på flera rader eller satser i sekvens på samma rad med semikolon emellan; använd blanktecken och blanka rader i koden. Hur vill du gruppera dina satser så att de är lätta för en människa att läsa?
%
%\Subtask Prova att ändra på \emph{ordningen} mellan satserna och studera hur resultatet påverkas. Använd den \emph{gula} play-knappen  (programspårning) för att studera exekveringen i detalj. Klicka på satser i ditt program och på rutor i programspårningen och se vad som händer.
%
%
%\Subtask Rita en trappa enligt bilden nedan.
%
%\includegraphics[width=0.2\textwidth]{../img/kojo/stairs}
%
%\Subtask Rita valfri bild på valfri bakgrund med hjälp av några av procedurerna i tabellen nedan. Du kan till exempel rita en rosa triangel med lila konturer mot svart bakgrund. % \ref{lab:kojo:kojo-procedures}.
%Försök att underlätta läsbarheten av din kod med hjälp av lämpliga radbrytningar och gruppering av satser. Undersök hur ordningen av satserna i din kod påverkar resultatet.
%
%
%
%\begin{table}[H]
%\begin{tabular}{l l}\small
%\code|fram(100)| & Paddan går framåt 100 steg (25 om argument saknas).\\
%\code|färg(rosa)| & Sätter pennans färg till rosa. \\
%\code|fyll(lila)| & Sätter ifyllnadsfärgen till lila. \\
%\code|fyll(genomskinlig)| & Gör så att paddan \emph{inte} fyller i något när den ritar. \\
%\code|bredd(20)| & Gör så att pennan får bredden 20. \\
%\code|bakgrund(svart)| & Bakgrundsfärgen blir svart. \\
%\code|bakgrund2(grön,gul)| & Bakgrund med övergång från grönt till gult. \\
%\code|pennaNer|  & Sätter ner paddans penna så att den ritar när den går. \\
%\code|pennaUpp|  & Sänker paddans penna så att den \emph{inte} ritar när den går. \\
%\code|höger(45)|   & Paddan vrider sig 45 grader åt höger. \\
%\code|vänster(45)| & Paddan vrider sig 45 grader åt vänster. \\
%\code|hoppa|       & Paddan hoppar 25 steg utan att rita. \\
%\code|hoppa(100)|  & Paddan hoppar 100 steg utan att rita. \\
%\code|hoppaTill(100, 200)| & Paddan hoppar till läget (100, 200) utan att rita. \\
%\code|gåTill(100, 200)|    & Paddan vrider sig och går till läget (100, 200). \\
%\code|öster|   & Paddan vrider sig så att nosen pekar åt höger. \\
%\code|väster|  & Paddan vrider sig så att nosen pekar åt vänster. \\
%\code|norr|    & Paddan vrider sig så att nosen pekar uppåt. \\
%\code|söder|   & Paddan vrider sig så att nosen pekar neråt. \\
%\code|mot(100,200)|   & Paddan vrider sig så att nosen pekar mot läget (100, 200) \\
%\code|sättVinkel(90)| & Paddan vrider nosen till vinkeln 90 grader. \\
%\end{tabular}
%%\label{lab:kojo:kojo-procedures}
%%\caption{Några användbara procedurer i Kojo.}
%\end{table}


%%% NEDAN ÄR BORTTAGEN FÖR ATT MINSKA MÄNGDEN ARBETE

%\Subtask \emph{Rita och mät}.
%\begin{itemize}[noitemsep]
%\item Börja ditt program med dessa satser:\\ \code{sudda; axesOn; gridOn; sakta(0); osynlig}
%\item Rita sedan en kvadrat som har 444 längdenheter i omkrets.
%\item Ta fram linjalen med höger-klick i ritfönstret och mät så exakt du kan hur lång diagonalen i kvadraten är. Skriv ner resultatet. \\ \emph{Tips:} Du kan zooma med mushjulet om du håller nere Ctrl-knappen. Du kan flytta linjalen om du klick-drar på linjalens skalstreck. Du kan vrida linjalen om du klickar på skalstrecken och håller nere Shift-tangenten.
%\item Kontrollera med hjälp av \code{math.hypot} och \code{println} vad det exakta svaret är. Skriv ner svaret med 3 decimalers noggrannhet. Du kan t.e.x. använda REPL i ett terminalfönster bredvid, eller öppna ett nytt extra Kojo-fönster i Arkiv-menyn, eller lägga in utskrifterna sist i ditt befintliga program. Utskrifter med \code{println} i Kojo sker i utdatafönstret.
%\end{itemize}
%
%\Subtask Rita en liksidig triangel med sidan 300 längdenheter genom att ge lämpliga argument till \code{fram} och \code{höger}. Vinklar anges i grader.
%
%\Subtask\Checkpoint Visa dina resultat för en handledare och diskutera hur uppgifterna ovan illustrerar principen om sekvens.

\vspace{1em}

% \Task Läs om hur du gör grafikprogram med Kojo i Appendix \ref{appendix:kojo} och övning {\tt \ExeWeekONE} i kapitel \ref{exe:W01}.


\Task \textit{Sekvens och repetition}. Rita en kvadrat med hjälp av \code+upprepa(n){ ??? }+ där du ersätter \code{n} med antalet repetitioner och \code{???} med de satser som ska repeteras.

%\Subtask Om du kör Kojo Desktop: Prova att köra ditt program med den \emph{gula} play-knappen för programspårning. Studera exekveringssekvensen. Klicka på anropen i programspårningsfönstret och studera markeringarna i ritfönstret.





\Task \textit{Variabel och repetition}.

\Subtask Funktionen \code{System.currentTimeMillis} ingår i Javas standardbibliotek och ger ett heltal av typen \code{Long} med det nuvarande antalet millisekunder sedan midnatt den första januari 1970.  Med Kojo-proceduren \code{sakta(0)} blir det ingen fördröjning när paddan ritar och utritningen sker så snabbt som möjligt. Prova nedan program och förklara vad som händer.
\begin{Code}
sakta(0)
val n = 800 * 4
val t1 = System.currentTimeMillis
upprepa(n){ upprepa(4){ fram; höger } }
val t2 = System.currentTimeMillis
println(s"$n kvadratvarv tog ${t2 - t1} millisekunder")
\end{Code}
\noindent Om du kör Kojo Desktop är det bra att börja programmet med \code{sudda}. (Varför?)

\Subtask\Pen Anteckna ungefär hur många kvadratvarv per sekund som paddan kan rita när den är som snabbast. Kör flera gånger eftersom den virtuella maskinen behöver ''värmas upp'' för att maskinkoden ska optimeras. Vissa körningar kan gå långsammare om skräpsamlaren behöver lägga tid på att frigöra minne.

\Subtask\Pen Vad har variablerna i koden ovan för namn? Vad har variablerna för värden?

\Subtask Rita en kvadrat igen, men nu med hjälp av en \code{while}-sats och en loopvariabel. %Studera exekveringen med programspårning (den gula play-knappen).

\begin{Code}
sakta(100)
var i = 0
while (???) { fram; höger; i = ??? }
\end{Code}

\Subtask\Pen Vad är det för skillnad på variabler som deklareras med \code{val} respektive \code{var}?

\Subtask Rita en kvadrat igen, men nu med hjälp av en \code{for}-sats. Skriv ut värdet på den lokala variabeln \code{i} i varje loop-runda.

\begin{Code}
for (i <- 1 to ???) { ??? }
\end{Code}

\Subtask\Pen Går det att tilldela variabeln \code{i} ett nytt värde i loopen?

\Subtask\Pen Går det att referera till namnet \code{i} utanför loopen?


\Subtask Rita en kvadrat igen, men nu med hjälp av \code{foreach}. Skriv ut loopvariabelns värde i varje runda.

\begin{Code}
(1 to ???).foreach{ i => ??? }
\end{Code}

%\Subtask\Pen För var och en av de fyra repetitionskonstruktionerna du sett ovan, \code{upprepa}, \code{while}, \code{for} och \code{foreach}: skriv kod med penna på papper som skriver ut de första 100 jämna heltalen med blanktecken emellan: \code{2 4 6 8 10 12 ...} etc.\\ Vilken typ av loop tycker du är enklast att använda i detta fall?


\Task \textit{Abstraktion}.

\Subtask Använd en repetition för att abstrahera nedan sekvens, så att programmet blir kortare:
\begin{Code}
fram; höger; hoppa; fram; vänster; hoppa; fram; höger;
hoppa; fram; vänster; hoppa; fram; höger; hoppa; fram;
vänster; hoppa; fram; höger; hoppa; fram; vänster; hoppa;
fram; höger; hoppa; fram; vänster; hoppa
\end{Code}

%\Subtask\Pen Sök på nätet efter ''DRY principle programming'' och beskriv med egna ord vad DRY betyder och varför det är en viktig princip.

\Subtask Definiera en egen procedur som heter \code{kvadrat} med hjälp av nyckelordet \code{def} som vid anrop ritar en kvadrat med hjälp av en \code{for}-loop.

\begin{Code}
def kvadrat = for (???) {???}
\end{Code}


\Subtask Anropa din abstraktion efter att den deklarerats och efter att du exekverat:\\\code{sakta(100)}


\Subtask Anropa din abstraktion inuti en \code{for}-loop så att paddan ritar en stapel som är 10 kvadrater hög enligt bilden nedan.

\begin{figure}
  \begin{multicols}{2}

  \includegraphics[scale=0.6]{../img/kojo/square-column}

  \columnbreak

  \begin{Code}
  def kvadrat = for (???) {???}
  for (???) {???}
  \end{Code}

  \end{multicols}
  \caption{En kvadratstapel.\label{fig:kojo-lab:column}}
\end{figure}

\Subtask %Kör ditt program med den \emph{gula} play-knappen. 
Studera hur anrop av proceduren \code{kvadrat} påverkar exekveringssekvensen av dina satser genom att göra lämpliga utskrifter så att du kan se när olika delar av koden exekveras. Vid vilka punkter i programmet sker ett ''hopp'' i sekvensen i stället för att efterföljande sats exekveras?  Använd lämpligt argument till \code{sakta} för att du ska hinna studera exekveringen.


\Subtask Rita samma bild med 10 staplade kvadrater (se bild \ref{fig:kojo-lab:column} på sidan \pageref{fig:kojo-lab:column}), men nu \emph{utan} att använda abstraktionen \code{kvadrat} -- använd i stället en nästlad repetition (alltså en upprepning inuti en upprepning). Vilket av de två sätten (med och utan abstraktionen \code{kvadrat}) är lättast att läsa? %\emph{Tips:} Varje gång du trycker på någon av play-knapparna, sparas ditt program. Du kan se dina sparade program om du klickar på \emph{Historik}-fliken. Du kan också stega bakåt och framåt i historiken med de blå pilarna bredvid play-knapparna.

\Subtask Generalisera din abstraktion \code{kvadrat} genom att ge den en parameter \code{sida: Double} som anger kvadratens storlek. Rita flera kvadrater i likhet med bild \ref{fig:kojo-lab:resize} på sidan \pageref{fig:kojo-lab:resize}).

\begin{figure}[H]
\includegraphics{../img/kojo/square-param}
  \caption{Olika stora kvadrater.\label{fig:kojo-lab:resize}}

\end{figure}



%\Subtask\Pen%\Checkpoint
%Se över ditt program i föregående uppgift och säkerställ att det är lättläst och följer en struktur som börjar med alla definitioner i logisk ordning och därefter fortsätter med huvudprogrammet.
%%Diskutera ditt program med en handledare.



%\Subtask\Pen Spara ditt program i en fil men lämpligt namn och ha programmet redo när det är din tur att redovisa vad du gjort under laborationen.
%Anteckna några åtgärder du vidtagit för att göra programmet mer lättläst.







\Task \emph{Alternativ.} \label{kojo:alt}

\Subtask Kör programmet nedan. Förklara vad som händer. %Använd den gula play-knappen för att studera exekveringen.

\begin{Code}
sakta(5000)

def move(key: Int): Unit = {
  println("key: " + key)
  if (key == 87) fram(10)
  else if (key == 83) fram(-10)
}

move(87); move('W'); move('W')
move(83); move('S'); move('S'); move('S')
\end{Code}

\Subtask \label{subtask:keypress}  Kör programmet nedan. Notera \code{activateCanvas()} för att du ska slippa klicka i ritfönstret innan du kan styra paddan. Anropet \code{onKeyPress(move)} gör så att \code{move} kommer att anropas då en tangent trycks ned. Lägg till kod i \code{move} som gör att tangenten A ger en vridning moturs med 5 grader medan tangenten D ger en vridning medurs 5 grader. Med \code{onKeyPress} bestämmer man vilken procedur som ska köras vid tangenttryck.

\begin{Code}
sakta(0); activateCanvas()

def move(key: Int): Unit = {
  println("key: " + key)
  if (key == 'W') fram(10)
  else if (key == 'S') fram(-10)
}

onKeyPress(move)
\end{Code}



%\Subtask Spara ditt program i en fil men lämpligt namn och ha programmet redo när det är din tur att redovisa vad du gjort under laborationen.


\subsection{Kontrollfrågor}\Checkpoint

\noindent Repetera teorin för denna vecka och var beredd på att kunna svara på dessa frågor när det blir din tur att redovisa vad du gjort under laborationen:

\begin{enumerate}
\item Vad innebär sekventiell exekvering av satser?
\item Vad är skillnaden mellan en sats och ett uttryck?
\item Vad är skillnaden mellan en procedur och en funktion?
\item Spelar ordningen mellan argument någon roll vid anrop av en funktion med flera parametrar?
\item Vad är en variabel? Ge exempel på deklaration, initialisering och tilldelning av variabler, samt användning av variabler i uttryck.
\item Vad är ett logiskt uttryck? Ge exempel på användning av logiska uttryck.
\item Vad är abstraktion? Ge exempel på användning av abstraktion.
\item Vad är nyttan med abstraktion?
\item Hur deklareras och initialiseras en variabel vars värde är förändringsbart?
\item Hur deklareras och initialiseras en variabel vars värde är oföränderligt?
\item Är det ett körtidsfel eller kompileringsfel att tilldela en oföränderlig variabel ett nytt värde?
\item Ange vilken av \code{for} och \code{while} som är lämpligast i dessa fall:
\begin{itemize}[noitemsep, nolistsep]
\item[A.] Summera de hundra första heltalen.
\item[B.] Räkna antal tecken i en sträng innan första blanktecken.
\item[C.] Dra 100 slumptal mellan 1 och 6 och summera de tal som är mindre än 3.
\item[D.] Summera de första heltalen från 1 och uppåt tills summan är minst 100.
\end{itemize}
\end{enumerate}


\subsection{Frivilliga extrauppgifter}

\noindent Gör i mån intresse och träningsbehov nedan uppgifter i valfri ordning.

\Task \emph{Abstraktion och generalisering}.

\Subtask Skapa en abstraktion \code{def stapel = ???} som använder din abstraktion \code{kvadrat}.

\Subtask Du ska nu \emph{generalisera} din procedur så att den inte bara kan rita exakt 10 kvadrater i en stapel. Ge proceduren \code{stapel} en parameter \code{n} som styr hur många kvadrater som ritas.
\begin{Code}
def kvadrat = ???
def stapel(n: Int) = ???

sakta(100)
stapel(42)
\end{Code}



\Subtask Rita nedan bild med hjälp av abstraktionen \code{stapel}. Det är totalt 100 kvadrater och varje kvadrat har sidan 25. \emph{Tips:} Med ett negativt argument till proceduren \code{hoppa} kan du få sköldpaddan att hoppa baklänges utan att rita, t.ex. \code{hoppa(-10*25)}

\includegraphics[width=0.3\textwidth]{../img/kojo/square-grid}

\Subtask Generalisera dina abstraktioner \code{kvadrat} och \code{stapel} så att man kan påverka storleken på kvadraterna som ritas ut.

\Subtask Skapa en abstraktion \code{rutnät} med lämpliga parametrar som gör att man kan rita rutnät med olika stora kvadrater och olika många kvadrater i både x- och y-led.

\Subtask Generalisera dina abstraktioner \code{kvadrat} och \code{stapel} så att man kan påverka fyllfärgen och pennfärgen för kvadraterna som ritas ut. 

Färgerna i Kojo är av typen \code{java.awt.Color}. Typen är tillgänglig under namnet \code{Color} eftersom namnet är importerat med \code{export java.awt.Color} i filen \code{kojo.scala} (mer om nyckelorden \code{export} och \code{import} läsvecka 4).


\Task \emph{Växling med booleska värden.}

\Subtask Bygg vidare på programmet i uppgift \ref{kojo:alt} och lägg till nedan kod i början av programmet. Lägg även till kod som gör så att om man trycker på tangenten G så sätts rutnätet omväxlande på och av. Observera att det är exakt \emph{en} procedur som anropas vid \code{onKeyPress}.

\begin{Code}
var isGridOn = false

def toggleGrid =
  if (isGridOn) {
    gridOff
    isGridOn = false
  } else {
    gridOn
    isGridOn = true
  }
\end{Code}

\Subtask Gör så att när man trycker på tangenten X så sätter man omväxlande på och av koordinataxlarna. Använd en variabel \code{isAxesOn} och definiera en abstraktion \code{toggleAxes} som anropar \code{axesOn} och \code{axesOff} på liknande sätt som i föregående uppgift.


\Task \emph{Repetition.}~Skriv en procedur \code{randomWalk} med detta huvud: \\
\code{def randomWalk(n: Int, maxStep: Int, maxAngle: Int): Unit}\\ som gör så att paddan tar \code{n} steg av slumpmässig längd mellan \code{0} och \code{maxStep}, samt efter varje steg vrider sig åt vänster en slumpmässig vinkel mellan \code{0} och \code{maxAngle}. Anropa din procedur med olika argument och undersök hur dess värden påverkar bildens utseende. \emph{Tips:} Uttrycket \code{math.random() * 100} ger ett tal från 0 till (nästan) 100. Du kan styra hur långsamt paddan ritar genom anrop av \code{sakta(???)} (prova dig fram till något  lämpligt heltalsargument i stället för \code{???}).
\vspace{2em}\\\includegraphics[width=\textwidth]{../img/kojo/random-walk.png}


\Task \emph{Variabler, namngivning och formatering.}

\Subtask Klistra in nedan konstigt formatterade program \emph{exakt} som det står med blanktecken, indragningar och radbrytningar. Kör programmet och förklara vad som händer.

\begin{figure}[H]
\begin{Code}
// Ett konstigt formaterat program med en del konstiga namn.

def gurka(x: Double,
y: Double, namn: String,
typ: String,
värde:String) = {
val tomat = 15
val h = 30
hoppaTill(x,y)
norr
skriv(namn+": "+typ)
hoppaTill(x+tomat*(namn.size+typ.size),y)
skriv(värde); söder; fram(h); vänster
fram(tomat * värde.size); vänster
fram(h); vänster
fram(tomat * värde.size); vänster }
sudda; färg(svart); val s = 130
val h = 40
var x = 42; gurka(10, s-h*0, "x","Int", x.toString)
var y = x; gurka(10, s-h*1, "y","Int", y.toString)
x = x + 1; gurka(10, s-h*2, "x","Int", x.toString)
gurka(10, s-h*3, "y","Int", y.toString); osynlig
\end{Code}
\end{figure}

\Subtask\Pen Skriv ner namnet på alla variabler som förekommer i programmet.

\Subtask\Pen Vilka av dessa variabler är lokala?

\Subtask\Pen Vilka av dessa variabler kan förändras efter initialisering?

\Subtask\Pen Föreslå tre förändringar av programmet ovan (till exempel namnbyten) som gör att det blir lättare att läsa och förstå.

\Subtask Gör sök-ersätt av \code{gurka} till ett bättre namn. \emph{Tips:} undersök kontextmenyn i editorn i Kojo genom att högerklicka. Använd kortkommandot för Sök/Ersätt.

\Subtask Gör automatisk formatering av koden med hjälp av lämpligt kortkommando. Notera skillnaderna. Vilka autoformateringar gör programmet lättare att läsa? Vilka manuella formateringar tycker du bör göras för att öka läsbarheten? Ge funktionen \code{gurka} ett bättre namn.  Diskutera läsbarheten med en handledare.



\Task \label{task:measuretime} \emph{Tidmätning.} Hur snabb är din dator?

\Subtask \label{task:timer} Skriv in koden nedan i Kojos editor och kör upprepade gånger med den gröna play-knappen. Tar det lika lång tid varje gång? Varför?

\begin{Code}
object timer {
  def now: Long = System.currentTimeMillis
  var saved: Long = now
  def elapsedMillis: Long = now - saved
  def elapsedSeconds: Double = elapsedMillis / 1000.0
  def reset: Unit = { saved = now }
}

// HUVUDPROGRAM:
timer.reset
var i = 0L
while (i < 1e8.toLong) { i += 1 }
val t = timer.elapsedSeconds
println("Räknade till " + i + " på " + t + " sekunder.")
\end{Code}


\Subtask Ändra i loopen i uppgift \ref{task:timer}) så att den räknar till 4.4 miljarder. Hur lång tid tar det för din dator att räkna så långt?\footnote{Det går att göra ungefär en heltalsaddition per klockcykel per kärna. Den första elektroniska datorn \href{https://sv.wikipedia.org/wiki/ENIAC}{Eniac} hade en klockfrekvens motsvarande 5 kHz. Den dator på vilken denna övningsuppgift skapades hade en i7-4790K turboklockad upp till 4.4 GHz.
%\href{http://www.extremetech.com/computing/185512-overclocking-intels-core-i7-4790k-can-devils-canyon-fix-haswells-low-clock-speeds/2}{www.extremetech.com/computing/185512-overclocking-intels-core-i7-4790k-can-devils-canyon-fix-haswells-low-clock-speeds/2}
}

\Subtask  Om du kör på en Linux-maskin: Kör nedan Linux-kommando upprepade gånger i ett terminalfönster. Med hur många MHz kör din dators klocka för tillfället? Hur förhåller sig klockfrekvensen till antalet rundor i while-loopen i föregående uppgift? (Det kan hända att din dator kan variera centralprocessorns klockfrekvens. Prova både medan du kör tidmätningen i Kojo och då din dator ''vilar''. Vad är det för poäng med att en processor kan variera sin klockfrekvens?)
\begin{REPLnonum}
> lscpu | grep MHz
\end{REPLnonum}


\Subtask Ändra i koden i uppgift \ref{task:timer}) så att \code{while}-loopen bara kör 5 gånger. %Kör programmet med den \emph{gula} play-knappen. Scrolla i programspårningen och förklara vad som händer. Klicka på \code{CALL}-rutorna och se vilken rad som markeras i ditt program.

\Subtask Lägg till koden nedan i ditt program och försök ta reda på ungefär hur långt din dator hinner räkna till på en sekund för \code{Long}- respektive \code{Int}-variabler. Använd den gröna play-knappen.
\begin{CodeSmall}
def timeLong(n: Long): Double = {
  timer.reset
  var i = 0L
  while (i < n) { i += 1 }
  timer.elapsedSeconds
}

def timeInt(n: Int): Double = {
  timer.reset
  var i = 0
  while (i < n) { i += 1 }
  timer.elapsedSeconds
}

def show(msg: String, sec: Double): Unit = {
  print(msg + ": ")
  println(sec + " seconds")
}

def report(n: Long): Unit = {
  show("Long " + n, timeLong(n))
  if (n <= Int.MaxValue) show("Int  " + n, timeInt(n.toInt))
}

// HUVUDPROGRAM, mätningar:

report(Int.MaxValue)
for (i <- 1 to 10) report(4.26e9.toLong)
\end{CodeSmall}

\Subtask Hur mycket snabbare går det att räkna med \code{Int}-variabler jämfört med \code{Long}-variabler? Diskutera gärna svaret med en handledare.

\Task Lek med färg i Kojo. Sök på internet efter dokumentationen för klassen \code{java.awt.Color} och studera vilka heltalsparametrar den sista konstruktorn i listan med konstruktorer tar för att skapa sRGB-färger. Om du högerklickar i editorn i Kojo och väljer ''Välj färg...'' får du fram färgväljaren och med den kan du välja fördefinierade färger eller blanda egna färger. När du har valt färg får du se vilka parametrar till \code{java.awt.Color} som skapar färgen. Testa detta i REPL:

\begin{REPL}
scala> val c = new java.awt.Color(124,10,78,100)
c: java.awt.Color = java.awt.Color[r=124,g=10,b=78]

scala> c.  // tryck på TAB
asInstanceOf    getColorComponents      getRGBComponents
brighter        getColorSpace           getRed
createContext   getComponents           getTransparency
darker          getGreen                isInstanceOf
getAlpha        getRGB                  toString
getBlue         getRGBColorComponents

scala> c.getAlpha
res3: Int = 100
\end{REPL}
Skriv ett program som ritar många figurer med olika färger, till exempel cirklar som nedan. Om du använder alfakanalen blir färgerna genomskinliga.

\includegraphics[width=0.82\textwidth]{../img/kojo/random-color-circles.png}


\Task Ladda ner ''Uppdrag med Kojo'' från \href{http://lth.se/programmera/uppdrag}{lth.se/programmera/uppdrag}  och gör några uppgifter som du tycker verkar intressanta.

%\Subtask ''Programming Fundamentals with Kojo'' som kan laddas ner här:\\
%\href{http://wiki.kogics.net/kojo-codeactive-books}{wiki.kogics.net/kojo-codeactive-books}

\Task Om du vill jobba med att hjälpa skolbarn att lära sig programmera med Kojo, kontakta \url{http://www.vattenhallen.lth.se} och anmäl ditt intresse att vara handledare.


\chapter{Kodstrukturer}
\begin{itemize}[nosep]
\item while-sats
\item for-sats
\item algoritm: min/max
\item MIN_VALUE
\item MAX_VALUE
\item paket
\item import
\item filstruktur
\item jar
\item dokumentation
\item programlayout
\item JDK
\item konstanter vs föränderlighet
\item objektorientering
\item klasser
\item objekt
\item referensvariabler
\item referenstilldelning
\item anropa metoder
\item SimpleWindow\end{itemize}

%!TEX encoding = UTF-8 Unicode
%!TEX root = ../exercises.tex

\ifPreSolution

\Exercise{\ExeWeekTWO}\label{exe:W02}
\begin{Goals}
%!TEX encoding = UTF-8 Unicode
%!TEX root = ../exercises.tex

\item Kunna skapa, kompilera och köra en enkel applikation i terminalen.
\item Kunna skapa samlingarna Range, Array och Vector med heltal och strängar.
\item Kunna indexera i en indexerbar samling, t.ex. Array och Vector.
\item Kunna anropa operationerna size, mkString, sum, min, max på samlingar som innehåller heltal.
\item Känna till skillnader och likheter mellan samlingarna Range, Array och Vector.
\item Förstå skillnaden mellan en while-sats och ett for-uttryck.
\item Kunna skapa samlingar med heltalsvärden som resultat av enkla for-uttryck.
\item Förstå skillnaden mellan en algoritm i pseudo-kod och dess implementation.
\item Kunna implementera algoritmerna SUM, MIN, MAX med en indexerbar samling och en while-sats.

\end{Goals}

\begin{Preparations}
\item \StudyTheory{02}
\item Bekanta dig med grundläggande terminalkommandon, se appendix~\ref{appendix:terminal}.
\item Bekanta dig med den editor du vill använda, se appendix~\ref{appendix:compile}.
\end{Preparations}

\else

\ExerciseSolution{\ExeWeekTWO}

\fi


% terminalkommando
% scalac -> hello world; scala som script; javac
% paket, import, jar, main,


\BasicTasksNoLab %%%%%%%%%%%%%%%%




\WHAT{Para ihop begrepp med beskrivning.}

\QUESTBEGIN

\Task \what

\vspace{1em}\noindent Koppla varje begrepp med den (förenklade) beskrivning som passar bäst: 

\begin{ConceptConnections}
  kompilerad & 1 & & A & där exekveringen av kompilerad app startar \\ 
  skript & 2 & & B & en samling som representerar ett intervall av heltal \\ 
  objekt & 3 & & C & maskinkod sparad och kan köras igen utan kompilering \\ 
  main & 4 & & D & en oföränderlig, indexerbar sekvenssamling \\ 
  programargument & 5 & & E & applicerar en funktion på varje element i en samling \\ 
  datastruktur & 6 & & F & stegvis beskrivning av en lösning på ett problem \\ 
  samling & 7 & & G & maskinkod sparas ej utan skapas vid varje körning \\ 
  sekvenssamling & 8 & & H & samlar variabler och funktioner \\ 
  Array & 9 & & I & överförs via parametern args i main \\ 
  Vector & 10 & & J & en specifik realisering av en algoritm \\ 
  Range & 11 & & K & används i for-uttryck för att skapa ny samling \\ 
  yield & 12 & & L & en förändringsbar, indexerbar sekvenssamling \\ 
  map & 13 & & M & datastruktur med element av samma typ \\ 
  algoritm & 14 & & N & många olika element i en helhet; elementvis åtkomst \\ 
  implementation & 15 & & O & datastruktur med element i en viss ordning \\ 
\end{ConceptConnections}

\SOLUTION

\TaskSolved \what

\begin{ConceptConnections}
  kompilerad & 1 & ~~\Large$\leadsto$~~ &  C & maskinkod sparad och kan köras igen utan kompilering \\ 
  skript & 2 & ~~\Large$\leadsto$~~ &  G & maskinkod sparas ej utan skapas vid varje körning \\ 
  objekt & 3 & ~~\Large$\leadsto$~~ &  H & samlar variabler och funktioner \\ 
  main & 4 & ~~\Large$\leadsto$~~ &  A & där exekveringen av kompilerad app startar \\ 
  programargument & 5 & ~~\Large$\leadsto$~~ &  I & överförs via parametern args i main \\ 
  datastruktur & 6 & ~~\Large$\leadsto$~~ &  N & många olika element i en helhet; elementvis åtkomst \\ 
  samling & 7 & ~~\Large$\leadsto$~~ &  M & datastruktur med element av samma typ \\ 
  sekvenssamling & 8 & ~~\Large$\leadsto$~~ &  O & datastruktur med element i en viss ordning \\ 
  Array & 9 & ~~\Large$\leadsto$~~ &  L & en förändringsbar, indexerbar sekvenssamling \\ 
  Vector & 10 & ~~\Large$\leadsto$~~ &  D & en oföränderlig, indexerbar sekvenssamling \\ 
  Range & 11 & ~~\Large$\leadsto$~~ &  B & en samling som representerar ett intervall av heltal \\ 
  yield & 12 & ~~\Large$\leadsto$~~ &  K & används i for-uttryck för att skapa ny samling \\ 
  map & 13 & ~~\Large$\leadsto$~~ &  E & applicerar en funktion på varje element i en samling \\ 
  algoritm & 14 & ~~\Large$\leadsto$~~ &  F & stegvis beskrivning av en lösning på ett problem \\ 
  implementation & 15 & ~~\Large$\leadsto$~~ &  J & en specifik realisering av en algoritm \\ 
\end{ConceptConnections}

\QUESTEND






%%%%%%%%%%%%%%%%%%% SKA FIXAS:




\WHAT{Datastrukturen \code+Range+.}

\QUESTBEGIN

\Task  \what~Evaluera nedan uttryck i Scala REPL. Vad har respektive uttryck för värde och typ?

\Subtask \code{Range(1, 10)}

\Subtask \code{Range(1, 10).inclusive}

\Subtask \code{Range(0, 50, 5)}

\Subtask \code{Range(0, 50, 5).size}

\Subtask \code{Range(0, 50, 5).inclusive}

\Subtask \code{Range(0, 50, 5).inclusive.size}

\Subtask \code{0.until(10)}

\Subtask \code{0 until (10)}

\Subtask \code{0 until 10}

\Subtask \code{0.to(10)}

\Subtask \code{0 to 10}

\Subtask \code{0.until(50).by(5)}

\Subtask \code{0 to 50 by 5}

\Subtask \code{(0 to 50 by 5).size}

\Subtask \code{(1 to 1000).sum}


\SOLUTION


\TaskSolved \what
 

\SubtaskSolved  värde: \code{Range(1,2,3,4,5,6,7,8,9)}

typ: \code{scala.collection.immutable.Range}

\SubtaskSolved  värde: \code{Range(1,2,3,4,5,6,7,8,9,10)}

typ: \code{scala.collection.immutable.Range}

\SubtaskSolved  värde: \code{Range(0,5,10,15,20,25,30,35,40,45)}

 typ: \code{scala.collection.immutable.Range}

\SubtaskSolved  värde: \code{10}, typ: \code{Int}

\SubtaskSolved  värde: \code{Range(0,5,10,15,20,25,30,35,40,45,50)}

typ: \code{scala.collection.immutable.Range}

\SubtaskSolved  värde: \code{11}, typ: \code{Int}

\SubtaskSolved  värde: \code{Range(0,1,2,3,4,5,6,7,8,9)}

typ: \code{scala.collection.immutable.Range}

\SubtaskSolved  värde: \code{Range(0,1,2,3,4,5,6,7,8,9)}

typ: \code{scala.collection.immutable.Range}

\SubtaskSolved  värde: \code{Range(0,1,2,3,4,5,6,7,8,9)}

typ: \code{scala.collection.immutable.Range}

\SubtaskSolved  värde: \code{Range(0,1,2,3,4,5,6,7,8,9,10)}

typ: \code{scala.collection.immutable.Range.Inclusive}

\SubtaskSolved  värde: \code{Range(0,1,2,3,4,5,6,7,8,9,10)}

typ: \code{scala.collection.immutable.Range.Inclusive}

\SubtaskSolved  värde: \code{Range(0,5,10,15,20,25,30,35,40,45)}

typ: \code{scala.collection.immutable.Range}

\SubtaskSolved  värde: \code{Range(0,5,10,15,20,25,30,35,40,45,50)}

typ: \code{scala.collection.immutable.Range}

\SubtaskSolved  värde: \code{11}, typ: \code{Int}

\SubtaskSolved  värde: \code{500500}, typ: \code{Int}




\QUESTEND




%%<AUTOEXTRACTED by mergesolu>%%      %Uppgift 2




\WHAT{Datastrukturen \code+Array+.}

\QUESTBEGIN

\Task \label{task:array} \what~   Kör nedan kodrader i Scala REPL. Beskriv vad som händer.

\Subtask \code{val xs = Array("hej","på","dej", "!")}

\Subtask \code{xs(0)}

\Subtask \code{xs(3)}

\Subtask \code{xs(4)}

\Subtask \code{xs(1) + " " + xs(2)}

\Subtask \code{xs.mkString}

\Subtask \code{xs.mkString(" ")}

\Subtask \code{xs.mkString("(", ",", ")")}

\Subtask \code{xs.mkString("Array(", ", ", ")")}

\Subtask \code{xs(0) = 42}

\Subtask \code{xs(0) = "42"; println(xs(0))}

\Subtask \code{val ys = Array(42, 7, 3, 8)}

\Subtask \code{ys.sum}

\Subtask \code{ys.min}

\Subtask \code{ys.max}

\Subtask \code{val zs = Array.fill(10)(42)}

\Subtask \code{zs.sum}

\Subtask\Pen Datastrukturen \code{Range} håller reda på start- och slutvärde, samt stegstorleken för en uppräkning, men alla talen i uppräkningen genereras inte förrän så behövs. En \code{Int} tar 4 bytes i minnet. Ungefär hur mycket plats i minnet tar de objekt som variablerna \code{r} respektive \code{a} refererar till nedan?
\begin{REPL}
scala> val r = (1 to Int.MaxValue by 2)
scala> val a = r.toArray
\end{REPL}
\emph{Tips:} Använd uttrycket \code{ BigInt(Int.MaxValue) * 2 } i dina beräkningar.

\SOLUTION


\TaskSolved \what
 

\SubtaskSolved  Ett objekt av typen \code{Array[String]} skapas med värdet 

\code{Array(hej, på, dej, !)} och med namnet \code{xs}.

\SubtaskSolved  Returnerar en sträng med värdet \code{hej}.

\SubtaskSolved  Returnerar en sträng med värdet \code{!}.

\SubtaskSolved  Ett exception genereras. Skriver ut:

\code{java.lang.ArrayIndexOutOfBoundsException: 4}

\SubtaskSolved  Returnerar en sträng med värdet \code{på dej}.

\SubtaskSolved  Returnerar en sträng med värdet \code{hejpådej!}.

\SubtaskSolved  Returnerar en sträng med värdet \code{hej på dej !}.

\SubtaskSolved  Returnerar en sträng med värdet \code{(hej,på,dej,!)}.

\SubtaskSolved  Returnerar en sträng med värdet \code{Array(hej,på,dej,!)}.

\SubtaskSolved  Ett fel uppstår av typen \code{type mismatch}. Konsollen talar om för oss vad den fick, dvs värdet \code{42} av typen \code{Int}. Den talar även om för oss vad den ville ha, dvs något värde av typen \code{String}. Till sist skriver den ut vår kodrad och pekar ut felet.

\SubtaskSolved  Det första elementet i \code{xs} ändras till värdet \code{42}. Därefter skrivs det första värdet i \code{xs} ut.

\SubtaskSolved  Ett objekt av typen \code{Array[Int]} skapas med värdet \code{Array(42, 7, 3, 8)} och med namnet \code{ys}.

\SubtaskSolved  Returnerar summan av elementen i \code{ys}. Resultatet är \code{60}.

\SubtaskSolved  Returnerar det minsta värdet i \code{ys}. Resultatet är \code{3}.

\SubtaskSolved  Returnerar det största värdet i \code{ys}. Resultatet är \code{42}.

\SubtaskSolved  Ett nytt värde av typen \code{Array[Int]} skapas med \code{10} stycken element, alla med värdet \code{42}.

\SubtaskSolved  Returnerar summan av elementen i \code{zs}. Resultatet blir 420 (42 multiplicerat med 10).

\SubtaskSolved  \code{r} tar upp 12 bytes. \code{a} tar upp ca 4 miljarder bytes.



\QUESTEND




%%<AUTOEXTRACTED by mergesolu>%%      %Uppgift 3




\WHAT{Datastrukturen \code+Vector+.}

\QUESTBEGIN

\Task  \what~  Kör nedan kodrader i Scala REPL. Beskriv vad som händer.

\Subtask \code{val words = Vector("hej","på","dej", "!")}

\Subtask \code{words(0)}

\Subtask \code{words(3)}

\Subtask \code{words.mkString}

\Subtask \code{words.mkString(" ")}

\Subtask \code{words.mkString("(", ",", ")")}

\Subtask \code{words.mkString("Ord(", ", ", ")")}

\Subtask \code{words(0) = "42"}

\Subtask \code{val numbers = Vector(42, 7, 3, 8)}

\Subtask \code{numbers.sum}

\Subtask \code{numbers.min}

\Subtask \code{numbers.max}

\Subtask \code{val moreNumbers = Vector.fill(10000)(42)}

\Subtask \code{moreNumbers.sum}

\Subtask\Pen Jämför med uppgift \ref{task:array}. Vad kan man göra med en \code{Array} som man inte kan göra med en \code{Vector}?

\SOLUTION


\TaskSolved \what
 

\SubtaskSolved  Ett objekt av typen \code{scala.collection.immutable.Vector[String]} initieras med värdet \code{Vector(hej, på dej, !)}.

\SubtaskSolved  Returnerar det nollte elementet i \code{words}, dvs strängen \code{hej}.

\SubtaskSolved  Returnerar det tredje elementet i \code{words}, dvs strängen \code{!}.

\SubtaskSolved  Omvandlar vektorn till en Sträng.

\SubtaskSolved  Samma som ovan, fast den här gången används mellanrum för att seperera elementen.

\SubtaskSolved  Samma som ovan, fast den här gången sepereras elementen av kommatecken istället för mellanrum och dessutom börjar och slutar den resulterande strängen med parenteser.

\SubtaskSolved  Samma som ovan, fast med ordet \code{Ord} tillagt i början av den resulterande strängen.

\SubtaskSolved  Ett fel uppstår. Typen \code{Vector} är immutable. Dess element kan alltså inte bytas ut.

\SubtaskSolved  En ny \code{Vector[Int]} skapas med värdet \code{Vector(42, 7, 3, 8)}. 

\SubtaskSolved  Returnerar summan av vektorn \code{numbers}.

\SubtaskSolved  Returnerar vektorns minsta element.

\SubtaskSolved  Returnerar vektorns största element. 

\SubtaskSolved  En ny vektor skapas innehållandes tiotusen 42or.

\SubtaskSolved  Returnerar summan av vektorns element.

\SubtaskSolved  Byta ut element.



\QUESTEND




%%<AUTOEXTRACTED by mergesolu>%%      %Uppgift 4




\WHAT{\code+for+-uttryck}

\QUESTBEGIN

\Task  \what~ . Evaluera nedan uttryck i Scala REPL. Vad har respektive uttryck för värde och typ?

\Subtask \code{for (i <- Range(1,10)) yield i}

\Subtask \code{for (i <- 1 until 10) yield i}

\Subtask \code{for (i <- 1 until 10) yield i + 1}

\Subtask \code{for (i <- Range(1,10).inclusive) yield i}

\Subtask \code{for (i <- 1 to 10) yield i}

\Subtask \code{for (i <- 1 to 10) yield i + 1}

\Subtask \code{(for (i <- 1 to 10) yield i + 1).sum}

\Subtask \code{for (x <- 0.0 to 2 * math.Pi by math.Pi/4) yield math.sin(x)}


\SOLUTION


\TaskSolved \what
 

\SubtaskSolved  typ: \code{scala.collection.immutable.IndexedSeq[Int]}

värde: \code{Vector(1, 2, 3, 4, 5, 6, 7, 8, 9)}

\SubtaskSolved  typ: \code{scala.collection.immutable.IndexedSeq[Int]}

värde: \code{Vector(1, 2, 3, 4, 5, 6, 7, 8, 9)}

\SubtaskSolved  typ: \code{scala.collection.immutable.IndexedSeq[Int]}

värde: \code{Vector(2, 3, 4, 5, 6, 7, 8, 9, 10)}

\SubtaskSolved  typ: \code{scala.collection.immutable.IndexedSeq[Int]}

värde: \code{Vector(1, 2, 3, 4, 5, 6, 7, 8, 9, 10)}

\SubtaskSolved  typ: \code{scala.collection.immutable.IndexedSeq[Int]}

värde: \code{Vector(1, 2, 3, 4, 5, 6, 7, 8, 9, 10)}

\SubtaskSolved  typ: \code{scala.collection.immutable.IndexedSeq[Int]}

värde: \code{Vector(2, 3, 4, 5, 6, 7, 8, 9, 10, 11)}

\SubtaskSolved  typ: \code{Int}, värde: \code{Vector(65)}

\SubtaskSolved  typ: \code{scala.collection.immutable.IndexedSeq[Int]}

värde: \code{Vector(0.0, 0.707, 1.0, 0.707, 0.0, -0.707, -1.0, -0.707)}



\QUESTEND




%%<AUTOEXTRACTED by mergesolu>%%      %Uppgift 5




\WHAT{Metoden \code+map+ på en samling.}

\QUESTBEGIN

\Task  \what~  Evaluera nedan uttryck i Scala REPL. Vad har respektive uttryck för värde och typ?

\Subtask \code{Range(0,10).map(i => i + 1)}

\Subtask \code{(0 until 10).map(i => i + 1)}

\Subtask \code{(1 to 10).map(i => i * 2)}

\Subtask \code{(1 to 10).map(_ * 2)}

\Subtask \code{Vector.fill(10000)(42).map(_ + 43)}

\SOLUTION


\TaskSolved \what
 

\SubtaskSolved  typ: \code{scala.collection.immutable.IndexedSeq[Int]}

värde: \code{Vector(1, 2, 3, 4, 5, 6, 7, 8, 9, 10)}

\SubtaskSolved  typ: \code{scala.collection.immutable.IndexedSeq[Int]}

värde: \code{Vector(1, 2, 3, 4, 5, 6, 7, 8, 9, 10)}

\SubtaskSolved  typ: \code{scala.collection.immutable.IndexedSeq[Int]}

värde: \code{Vector(2, 4, 6, 8, 10, 12, 14, 16, 18, 20)}

\SubtaskSolved  typ: \code{scala.collection.immutable.IndexedSeq[Int]}

värde: \code{Vector(2, 4, 6, 8, 10, 12, 14, 16, 18, 20)}

\SubtaskSolved  typ: \code{scala.collection.immutable.Vector[Int]}

värde: En vector av tiotusen 85or (85 = 42 + 43).



\QUESTEND




%%<AUTOEXTRACTED by mergesolu>%%      %Uppgift 6




\WHAT{Metoden \code+foreach+ på en samling.}

\QUESTBEGIN

\Task  \what~  Kör nedan satser i Scala REPL. Vad händer?

\Subtask \code{Range(0,10).foreach(i => println(i))}

\Subtask \code{(0 until 10).foreach(i => println(i))}

\Subtask \code|(1 to 10).foreach{i => print("hej"); println(i * 2)}|

\Subtask \code{(1 to 10).foreach(println)}

\Subtask \code{Vector.fill(10000)(math.random).foreach(r => }\\
           \code{      if (r > 0.99) print("pling!"))}


\SOLUTION


\TaskSolved \what
 

\SubtaskSolved  En \code{Range} skapas och dess element skrivs ut ett och ett.

\SubtaskSolved  Samma sak händer.

\SubtaskSolved  De tio första jämna talen (noll ej inräknat) skrivs ut med ett "hej" framför.

\SubtaskSolved  Talen 1 till 10 skrivs ut.

\SubtaskSolved  Tiotusen slumptal mellan 0 och 1 genereras. Varje gång ett tal är större än 0.99 kommer det ett pling.



\QUESTEND




%%<AUTOEXTRACTED by mergesolu>%%      %Uppgift 7




\WHAT{Algoritm: SWAP.}

\QUESTBEGIN

\Task  \what~ 

\Subtask Skriv med \emph{pseudo-kod} algoritmen SWAP. Beskriv på vanlig svenska, steg för steg, hur en variabel $temp$ används för mellanlagring vid värdebytet:

\emph{Indata:} två heltalsvariabler $x$ och $y$

\emph{???}

\emph{Utdata:} variablerna $x$ och $y$ vars värden har bytt plats.

\Subtask Implementerar algoritmen SWAP. Ersätt \code{???} nedan med satser separerade av semikolon:

\begin{REPL}
scala> var (x, y) = (42, 43)
scala> ???
scala> println("x är " + x + ", y är " + y)
x är 43, y är 42
\end{REPL}



\SOLUTION


\TaskSolved \what
 

\SubtaskSolved  Pseudokoden kan se ut såhär:

Skapa heltalsvariabel temp. 
Flytta värdet från x till temp. 
Flytta värdet från y till x. 
Flytta värdet från temp till y.

\SubtaskSolved 
\begin{REPLnonum}
scala> var (x, y) = (42, 43)
x: Int = 42
y: Int = 43
scala> var temp = x; x = y; y = temp;
temp: Int = 42
x: Int = 43
y: Int = 42
scala> println("x är " + x + ", y är " + y)
x är 43, y är 42
\end{REPLnonum}



\QUESTEND




%%<AUTOEXTRACTED by mergesolu>%%      %Uppgift 8




\WHAT{Skript.}

\QUESTBEGIN

\Task  \what~  Skapa en fil med namn \texttt{hello-script.scala} med hjälp av en editor som innehåller denna enda rad:
\begin{Code}
println("hej skript")
\end{Code}
Spara filen och kör kommandot \code{scala hello-script.scala} i terminalen:
\begin{REPLnonum}
> scala hello-script.scala
\end{REPLnonum}

\Subtask Vad händer?

\Subtask Ändra i filen så att högerparentesen saknas. Spara och kör skriptfilen igen. Vad händer?

\Subtask Lägg till en sats sist i skriptet som skriver ut summan av de ett tusen stycken heltalen från och med 2 till och med 1001, så som visas nedan.
\begin{REPL}
> scala hello-script.scala
hej skript
501500
\end{REPL}

\Subtask Ändra i hello-script.scala genom att införa \code{val n = args(0).toInt} och använd \code{n} som övre gräns för summeringen av de n första heltalen.
\begin{REPL}
> scala hello-script.scala 5001
hej skript
12507501
\end{REPL}

\Subtask Vad blir det för felmeddelande om du glömmer ge programmet ett argument?


\SOLUTION


\TaskSolved \what
 

\SubtaskSolved  Skriver ut "hej skript".

\SubtaskSolved  Ett felmeddelande skrivs ut.

\SubtaskSolved  Lägg till raden:
\code{println((2 to 1001).sum)} 
eller motsvarande.

\SubtaskSolved  Filen ska se ut ungefär såhär: \\
\begin{Code} 
val n = args(0).toInt 
println("hej skript") 
println((1 to n).sum)
\end{Code}

\SubtaskSolved  \code{java.lang.ArrayIndexOutOfBoundsException: 0}



\QUESTEND




%%<AUTOEXTRACTED by mergesolu>%%      %Uppgift 9




\WHAT{Applikation med \code+main+-metod.}

\QUESTBEGIN

\Task  \what~  Skapa med hjälp av en editor en fil med namn \texttt{hello-app.scala}.
\begin{REPLnonum}
> gedit hello-app.scala
\end{REPLnonum}
Skriv dessa rader i filen:


\scalainputlisting{examples/hello-app.scala}

\Subtask Kompilera med \code{scalac hello-app.scala} och kör koden med \code{scala Hello}.
\begin{REPLnonum}
> scalac hello-app.scala
> ls
> scala Hello
\end{REPLnonum}
Vad heter filerna som kompilatorn skapar?

\Subtask Ändra i din kod så att kompilatorn ger följande felmeddelande: \\
\texttt{Missing closing brace}

\Subtask\Pen Varför behövs \code{main}-metoden?

\Subtask\Pen Vilket alternativ går snabbast att köra igång, ett skript eller en kompilerad applikation? Varför? Vilket alternativ kör snabbast när väl exekveringen är igång?


\SOLUTION


\TaskSolved \what
 

\SubtaskSolved  Hello.class och Hello\$.class

\SubtaskSolved  Ta bort en av krullparenteserna i slutet.

\SubtaskSolved  I ett skript behöver man inte skriva någon main-metod. Kompilatorn lägger till en automatiskt precis när koden ska köras. I en applikation behöver man däremot det. För att göra en applikation definierar vi ett objekt som vi i det här fallet kallar för \code{Hello}. Från början gör inte objekt någonting. De bara finns. För att objekt ska kunna göra något behövs det metoder. I vanliga fall utförs inte metoder förrän en annan metod "ropar" på metoden. main-metoden ropas dock automatiskt när en applikation startas. Annars hade ju ingenting hänt, eftersom alla metoderna väntar på att någon annan metod ska börja. \\
\SubtaskSolved  Första gången man ska köra en applikation måste den först kompileras innan den exekveras. Skript kompileras automatiskt samtidigt som de exekveras, vilket totalt sett görs på kortare tid. Därför tar det längre tid att starta en applikation första gången än att starta ett skript första gånge. När en applikation väl har kompileras och kan exekveras, går det dock mycket fortare. Fördelen med applikationer är att de kan exekveras flera gånger utan att kompileras om.



\QUESTEND




%%<AUTOEXTRACTED by mergesolu>%%      %Uppgift 10




\WHAT{Java-applikation.}

\QUESTBEGIN

\Task \label{task:java} \what~   Skapa med hjälp av en editor en fil med namn \texttt{Hi.java}.
\begin{REPLnonum}
> gedit Hi.java
\end{REPLnonum}
Skriv dessa rader i filen:

\javainputlisting{examples/Hi.java}

\noindent Kompilera med \code{javac Hi.java} och kör koden med \code{java Hi}.
\begin{REPLnonum}
> javac Hi.java
> ls
> java Hi
\end{REPLnonum}

\Subtask\Pen Vad heter filen som kompilatorn skapat?

\Subtask\Pen Jämför signaturen för Java-programmets main-metod med signaturen för Scala-programmets main-metod. De betyder samma sak men syntaxen är olika. Beskriv skillnader och likheter i syntaxen.

\Subtask\Pen Vad blir det för felmeddelande om källkodsfilen och klassnamnet inte överensstämmer i ett Java-program?


\SOLUTION


\TaskSolved \what
 

\SubtaskSolved  Hi.class

\SubtaskSolved  I javas syntax börjar man med orden \code{public static}. I scala uteblir dessa. I scala är alla metoder automatiskt publika om inget annat används. Därför behövs aldrig ordet \code{public} i scala. I scala finns det tekniskt sett inga statiska metoder. Men i praktiken fungerar vanliga metoder i ett scala-objekt på ungefär samma sätt som statiska metoder i en java-klass. I scala används ordet \code{def} varje gång en funktion ska definieras. I java slipper man det. I java skriver man returtypen (\code{void}) innan parametrarna. I scala kommer istället metodens returtyp (\code{Unit}) i slutet. Javas \code{void} motsvarar scalas \code{Unit}. I scalas syntax kommer parameterns namn (\code{args}) före parameterns typ (\code{Array[String]}), separerat med ett kolon. I java kommer typen (\code{String[]}) först och sen kommer namnet (\code{args}). \code{String[]} i java betyder ungefär samma sak som \code{Array[String]} i scala.

\SubtaskSolved  -



\QUESTEND




%%<AUTOEXTRACTED by mergesolu>%%      %Uppgift 11




\WHAT{Algoritm: SUMBUG}

\QUESTBEGIN

\Task  \what~ . Nedan återfinns pseudo-koden för SUMBUG.

\begin{algorithm}[H]
 \SetKwInOut{Input}{Indata}\SetKwInOut{Output}{Resultat}

 \Input{heltalet $n$}
 \Output{utskrift av summan av de första $n$ heltalen }
 $sum \leftarrow 0$ \\
 $i \leftarrow 1$  \\
 \While{$i \leq n$}{
  $sum \leftarrow sum + 1$
 }
 skriv ut $sum$
\end{algorithm}

\Subtask\Pen Kör algoritmen steg för steg med penna och papper, där du skriver upp hur värdena för respektive variabel ändras. Det finns två buggar i algoritmen. Vilka? Rätta buggarna och test igen genom att ''köra'' algoritmen med penna på papper och kontrollera så att algoritmen fungerar för $n=0$, $n=1$, och $n=5$. Vad händer om $n=-1$?

\Subtask Skapa med hjälp av en editor filen \code{sumn.scala}. Implementera algoritmen SUM enligt den rättade pseudokoden och placera implementationen i en main-metod i ett objekt med namnet \code{sumn}. Du kan skapa indata \code{n} till algoritmen med denna deklaration i början av din main-metod: \\ \code{val n = args(0).toInt} \\ Vad ger applikationen för utskrift om du kör den med argumentet 8888?

\begin{REPLnonum}
> scalac sumn.scala
> scala sumn 8888
\end{REPLnonum}

\Subtask Kontrollera att din implementation räknar rätt genom att jämföra svaret med detta uttrycks värde, evaluerat i Scala REPL:
\begin{REPLnonum}
scala> (1 to 8888).sum
\end{REPLnonum}

\Subtask Implementera algoritmen SUM enligt pseudokoden ovan, men nu i Java. Skapa filen \code{SumN.java} och använd koden från uppgift \ref{task:java} som mall för att deklarera den publika klassen \code{SumN} med en main-metod. Några tips om Java-syntax och standarfunktioner i Java:

\begin{itemize}[noitemsep, nolistsep]
\item Alla satser i Java måste avslutas med semikolon.
\item Heltalsvariabler deklareras med nyckelordet \lstinline[language=Java]{int} (litet i).
\item Typnamnet ska stå \emph{före} namnet på variabeln. Exempel: \\ \lstinline[language=Java]{int sum = 0;}
\item Indexering i en array görs i Java med hakparenteser: \code{args[0]}
\item I stället för Scala-uttrycket \code{args(0).toInt}, använd Java-uttrycket: \\ \code{Integer.parseInt(args[0])}
\item \code{while}-satser i Scala och Java har samma syntax.
\item Utskrift i Java görs med \code{System.out.println}
\end{itemize}


\SOLUTION


\TaskSolved \what
 

\SubtaskSolved  Bugg: Eftersom \code{i} inte ökar, fastnar programmet i en oändlig loop. Fix: Lägg till en sats i slutet av while-blocket som ökar värdet på i med 1.
Bugg: Eftersom man bara ökar summan med 1 varje gång, kommer resultatet att bli summan av n stycken 1or, inte de n första heltalen. Fix: Ändra så att summan ökar med \code{i} varje gång, istället för 1.
För -1, blir resultatet 0. Förklaring: i börjar på 1 och är alltså aldrig mindre än n som ju är -1. while-blocket genomförs alltså noll gånger, och efter att \code{sum} får sitt ursprungsvärde förändras den aldrig.
\SubtaskSolved  39502716
\SubtaskSolved  -
\SubtaskSolved  Såhär kan implementationen se ut:
\begin{Code}
public class SumN {
  public static void main(String[] args) {
    int n = Integer.parseInt(args[0]);
    int sum = 0;
    int i = 1;
    while(i <= n){
      sum = sum + i;
      i = i + 1;
      }
    }
    System.out.println(sum);
}
\end{Code}



\QUESTEND




%%<AUTOEXTRACTED by mergesolu>%%      %Uppgift 12




\WHAT{Algoritm: MAXBUG}

\QUESTBEGIN

\Task  \what~ . Nedan återfinns pseudo-koden för MAXBUG.

\begin{algorithm}[H]
 \SetKwInOut{Input}{Indata}\SetKwInOut{Output}{Resultat}

 \Input{Array $args$ med strängar som alla innehåller heltal}
 \Output{utskrift av största heltalet }
 $max \leftarrow$ det minsta heltalet som kan uppkomma  \\
 $n \leftarrow $ antalet heltal \\
 $i \leftarrow 0$ \\
 \While{$i < n$}{
   $x \leftarrow args(i).toInt$ \\
   \If{( x > $max$)}{$max \leftarrow x$}
  % $i \leftarrow i + 1$
 }
 skriv ut $max$
\end{algorithm}

\Subtask\Pen Kör med penna och papper. Det finns en bugg i algoritmen ovan. Vilken? Rätta buggen.

\Subtask Implementera algoritmen MAX (utan bugg) som en Scala-applikation. Tips:
\begin{itemize}[noitemsep, nolistsep]
\item Det minsta \code{Int}-värdet som någonsin kan uppkomma: \code{Int.MinValue}
\item Antalet element i $args$ ges av: \code{args.size}
\end{itemize}

\begin{REPL}
> gedit maxn.scala
> scalac maxn.scala
> scala maxn 7 42 1 -5 9
42
\end{REPL}

\Subtask\Pen \label{subtask:arg0} Skriv om algoritmen så att variabeln $max$ initialiseras med det första talet i sekvensen.

\Subtask Implementera den nya algoritmvarianten från uppgift \ref{subtask:arg0} och prova programmet. Vad händer om $args$ är tom?

\SOLUTION


\TaskSolved \what
 

\SubtaskSolved  Bugg: i ökar aldrig. Programmet fastnar i en oändlig loop. Fix: Lägg till en sats som ökar i med 1, i slutet av while-blocket.

\SubtaskSolved  Så här kan implementationen se ut:
\begin{Code}
object Max {
  def main(args: Array[String]): Unit = {
    var max = Int.MinValue
    val n = args.size
    var i = 0
    while(i < n) {
      val x = args(i).toInt
      if(x > max) {
        max = x
      }
      i = i + 1
    }
    println(max)
  }
}
\end{Code}
\SubtaskSolved  Raden där max initieras ändras till \code{var max = args(0).toInt} 

\SubtaskSolved  \code{java.lang.ArrayIndexOutOfBoundsException: 0}



\QUESTEND




%%<AUTOEXTRACTED by mergesolu>%%      %Uppgift 13




\WHAT{Block, namnsynlighet, namnöverskuggning}

\QUESTBEGIN

\Task  \what~ . Kör nedan kod i Scala REPL eller i Kojo. Vad händer nedan? Varför?

\Subtask \code|val a = {1 + 1; 2 + 2; 3 + 3; 4 + 4}; println(a)|

\Subtask \code|val b = {1; 2; 3; {val b = 4; b + b; b + 1}}; println(b)|

\Subtask \code|{val a = 42; println(a)}|

\Subtask \code|{val a = 42}; println(a)|

\Subtask \code|{val a = 42; {val a = 43; println(a)}; println(a)}|

\Subtask \code|{var a = 42; {a = a + 1}; var a = 43}|

\Subtask \code|{var a = 42; {a = a + b; var b = 43}; println(a)}|

\Subtask \code|{var a = 42; {var b = 43; a = a + b}; println(a)}|

\Subtask \code|{var a = 42; {a = a + b; def b = 43}; println(a)}|

\Subtask \code|{object a{var b=42;object a{var a=43}};println(a.b+a.a.a)}|

\Subtask

\begin{Code}
{
  object a {
    var b = 42
    object a {
      var a = 43
    }
  }
  println(a.b + a.a.a)
}
\end{Code}

\Subtask Vad är fördelen med att namn deklarerade inne i ett block är lokala i stället för globala?


\SOLUTION


\TaskSolved \what


\SubtaskSolved  Skriver ut talet 8. \code{a} får värdet \code{4 + 4} eftersom detta är den sista satsen i blocket. Man får också tre stycken varningar. Detta beror på att det förekommer tre satser i blocket som inte gör någon skillnad.

\SubtaskSolved  Skriver ut talet 5. De tre första satserna i det yttre blocket ignoreras. \code{b} får värdet som returneras av det yttre blocket. Det yttre blocket returnerar värdet som returneras i den sista satsen i blocket, som i sin tur är ett block. I det inre blocket skapas en ny \code{val} som också får namnet \code{b}. Notera att detta alltså inte är samma värde, även om det har samma namn. Den andra satsen räknar summan av \code{b} med sig själv. Eftersom vi nu befinner oss i det block där det andra \code{b}et precis har definieras så är det detta \code{b} som används och summan blir alltså åtta. Detta är dock helt irrelevant eftersom resultatet inte sparas någonstans. I den sista satsen blir resultatet 5 (eftersom \code{b} är fyra och vi adderar ett). Detta resultatet returneras från det innre blocket och vidare ur det yttre blocket.

\SubtaskSolved  Skriver ut talet 42. Blockets satser exekveras i ordning. 

\SubtaskSolved  Skriver inte ut 42. I blocket skapas ett \code{val} med namnet \code{a} och värdet \code{42}. Detta värde finns inte utanför blocket och kommer därför inte att skrivas ut. Om du däremot definierat \code{a} som något annat tidigare så kommer istället det värdet att skrivas ut.

\SubtaskSolved  Skriver först ut \code{43} och sedan \code{42}. Förklaring:

\code{a} initieras med värdet \code{42}. Ett nytt värde som också har namnet \code{a} initieras med värdet \code{43}. Eftersom detta sker innanför ett nytt block, befinner vi oss i ett annat "namespace" och det gör alltså inget att vi använder samma namn. \code{a} skrivs ut. Eftersom vi befinner oss i det inre blocket är det \code{43} som skrivs ut, inte \code{42}. Scala kollar först efter värden som heter \code{a} i det inre "namespacet". Det är först i andra hand som den skulle upptäcka att det finns ett \code{a} i det yttre blocket. Till sist körs den sista satsen i det yttre blocket. Då skrivs \code{a} ut. Eftersom vi nu befinner oss i det yttre blocket, vet inte ens scala om att det andra \code{a}:et existerar. Resultatet av den här utskriften blir alltså \code{42}.

\SubtaskSolved  Ett fel uppstår. Variabeln \code{a} initieras två gånger i samma namespace. Förklaring till felet:

I det yttre blockets första sats initieras variablen \code{a} med värdet \code{42}. I det yttre blockets tredje sats försöker vi definiera en ny variabel med samma namn. I och med att vi befinner oss i samma namespace, krockar namnen.

Förklaring till vad som händer i sats två:

I det inre blocket har vi inte definierat någon variabel \code{a}. Till en början hittar alltså inte scala något sådant. Då letar scala vidare i det namespace som finns utanför det inre blocket och hittar variabeln som vi definierade i det yttre blockets första sats. Denna variabel får sitt värde förändrat.

\SubtaskSolved  Fel. Framåtreferens. Förklaring:

Det är inte tillåtet att referera till variabler som initieras senare i koden.

\SubtaskSolved  Skriver ut \code{85}. Förklaring:

I och med att vi den här gången initierade variabeln \code{b} och gav den ett värde innan vi använder oss av den, slipper vi problemet ovan.

\SubtaskSolved  Skriver ut \code{85}. Förklaring:

Det är tillåtet att referera till funktioner som definieras senare i koden.

\SubtaskSolved  Skriver ut \code{85}. Förklaring:

\code{a.b} refererar till variabeln \code{b} som ingår i objektet \code{a}.
\code{a.a.a} refererar till variabeln \code{a}, som ingår i ett objekt som heter \code{a} som i sin tur befinner sig i ett annat objekt som också heter \code{a}.

\SubtaskSolved  Skriver ut \code{85}. Förklaring:

Koden är identisk med förra deluppgiften förutom att ny rad används istället för semikolon.

\SubtaskSolved  I stora projekt med mycket kod, kan det vara svårt att hitta unika namn till alla sina variabler. Då är det en fördel om man kan hålla sina variabler i begränsade namespaces, så att de bara är tillgängliga precis när de behöver användas. 



\QUESTEND




%%<AUTOEXTRACTED by mergesolu>%%      %Uppgift 14??? NUMMER I KOMMENTAR STÄMMER EJ MED GENERERAT NUMMER




\WHAT{Paket, \code{import} och klassfilstrukturer.}

\QUESTBEGIN

\Task \label{task:package} \what~   Med Java-8-plattformen kommer 4240 färdiga klasser, som är organiserade i 217 olika paket.\footnote{Se Stackoverflow: \href{http://stackoverflow.com/questions/3112882/how-many-classes-are-there-in-java-standard-edition}{how-many-classes-are-there-in-java-standard-edition}}

\Subtask Vilka paket finns i paketet javax som börjar på s?

\begin{REPLnonum}
scala> javax.s   //tryck på TAB-tangenten
\end{REPLnonum}

\Subtask Kör raderna nedan i REPL. Beskriv vad som händer för varje rad.
\begin{REPL}[numbers=left, numberstyle=\color{black}\ttfamily\scriptsize\selectfont]
scala> import javax.swing.JOptionPane
scala> def msg(s: String) = JOptionPane.showMessageDialog(null, s)
scala> msg("Hej på dej!")
scala> def input(msg: String) = JOptionPane.showInputDialog(null, msg)
scala> input("Vad heter du?")
scala> import JOptionPane.{showOptionDialog => optDlg}
scala> def inputOption(msg: String, opt: Array[Object]) =
         optDlg(null, msg, "Option", 0, 0, null, opt, opt(0))
scala> inputOption("Vad väljer du?", Array("Sten", "Sax", "Påse"))
\end{REPL}

\Subtask\Pen Vad hade du behövt ändra på efterföljande rader om import-satsen på rad 1 ovan ej hade gjorts?

\Subtask Skapa med en editor filen paket.scala och kompilera. Rita en bild av hur katalogstrukturen ser ut.

\begin{Code}
package gurka.tomat.banan

package p1 {
  package p11 {
    object hello {
      def hello = println("Hej paket p1.p11!")
    }
  }
  package p12 {
    object hello {
      def hello = println("Hej paket p1.p12!")
    }
  }
}

package p2 {
  package p21 {
    object hello {
      def hello = println("Hej paket p2.p21!")
    }
  }
}

object Main {
  def main(args: Array[String]): Unit = {
    import p1._
    p11.hello.hello
    p12.hello.hello
    import p2.{p21 => apelsin}
    apelsin.hello.hello
  }
}
\end{Code}

\begin{REPL}
> gedit paket.scala
> scalac paket.scala
> scala gurka.tomat.banan.Main
> ls -R
\end{REPL}

\SOLUTION


\TaskSolved \what
 

\SubtaskSolved  \code{script   security   smartcardio   sound   sql   swing}

\SubtaskSolved  Radernas funktion i ordning:

1. Importerar JOptionPane från javax.swing

2. Definierar en metod som tar en sträng och öppnar en dialogruta med strängen.

3. Testar funktionen med argumentet "Hej på dej!". En dialogruta öppnas med texten "Hej på dej!".

4. Definierar en metod som tar emot en sträng som argument och öppnar en input-dialogruta med strängen.

5. Testar funktionen med argumentet "Vad heter du?". En dialogruta öppnas med texten "Vad heter du?". I ett fält kan man fylla i sitt namn. Funktionen returnerar namnet.

6. Importerar showOptionDialog från JOptionPane under namnet optDlg.

7. Definierar en metod som tar emot en sträng och en Array som argument och öppnar en flervalsdialog. Strängen ska innehålla frågan som flervalsdialogen visar upp. Arrayn ska innehålla alternativen som användaren ska välja mellan.

8.Testar funktionen med argumenten \code{"Vad väljer du?"} och \\ \code{Array("Sten, "Sax", "Påse")}. En dialogruta kommer upp och man får möjlighet att välja sten sax eller påse. Funktionen returnerar valet som man gör.

\SubtaskSolved  På alla ställen där \code{JOptionPane} förekommer, hade man istället fått skriva \code{javax.swing.JOptionPane}.

\SubtaskSolved  -



\QUESTEND




%%<AUTOEXTRACTED by mergesolu>%%      %Uppgift 15




\WHAT{Skapa \code{jar}-filer och använda classpath}

\QUESTBEGIN

\Task  \what~ 

\Subtask Skriv kommandot \code{jar} i terminalen och undersök vad som finns för optioner. Se speciellt ''Example 1.'' i hjälputskriften. Vilket kommando ska du använda för att packa ihop flera filer i en enda jar-fil?

\Subtask Som en fortsättning på uppgift \ref{task:package}, packa ihop biblioteket \code{gurka} i en jar-fil med nedan kommando, samt kör igång REPL med jar-filen på classpath.

\begin{REPL}
> jar cvf mittpaket.jar gurka
> scala -cp mittpaket.jar
scala> gurka.tomat.banan.Main.main(Array())
\end{REPL}


\SOLUTION


\TaskSolved \what
 

\SubtaskSolved  jar cvf [namn på skapad fil] [namn på input-filer]

\SubtaskSolved  -



\QUESTEND




%%<AUTOEXTRACTED by mergesolu>%%      %Uppgift 16




\WHAT{Skapa dokumentation med \code{scaladoc}-kommandot}

\QUESTBEGIN

\Task  \what~ 

\Subtask Som en fortsättning på uppgift \ref{task:package}, kör nedan kommando i terminalen:

\begin{REPL}
> scaladoc paket.scala
> ls
> firefox index.html   # eller öppna index.html i valfri webbläsare
\end{REPL}

Vad händer?

\Subtask Lägg till några fler metoder i något av objekten i filen \code{paket.scala} och lägg även till några dokumentationskommentarer. Kompilera om och kör. Generera om dokumentationen.

\begin{verbatim}
//... ändra i filen paket.scala

/** min paketdokumentationskommentar p2 */
package p2 {
  /** min paketdokumentationskommentar p21 */
  package p21 {
    /** ett hälsningsobjekt */
    object hello {
      /** en hälsningsmetod i p2.p21 */
      def hello = println("Hej paket p2.p21!")

      /** en metod som skriver ut tiden */
      def date = println(new java.util.Date)
    }
  }
}

\end{verbatim}

\begin{REPL}
> gedit paket.scala
> scalac paket.scala
> jar cvf mittpaket.jar gurka
> scala -cp mittpaket.jar
scala> gurka.tomat.banan.p2.p21.hello.date
scala> :q
> scaladoc paket.scala
> firefox index.html
\end{REPL}

\newpage

\ExtraTasks %%%%%%%%%%%%%%%%%%%

\SOLUTION


\TaskSolved \what
 

\SubtaskSolved  -

\SubtaskSolved  -
\QUESTEND






\WHAT{NEEDS A TOPIC DESCRIPTION}

\QUESTBEGIN

\Task \label{task:minindex} \what~  Implementera algoritmen MININDEX som söker index för minsta heltalet i en sekvens. Pseudokod för algoritmen MININDEX:

\begin{algorithm}[H]
 \SetKwInOut{Input}{Indata}\SetKwInOut{Output}{Utdata}

 \Input{Sekvens $xs$ med $n$ st heltal.}
 \Output{Index för det minsta talet eller $-1$ om $xs$ är tom.  }
 $minPos \leftarrow 0 $\\
 $i \leftarrow 1$ \\
 \While{$i < n$}{
   \If{xs(i) < $xs(minPos)$}{$minPos \leftarrow i$}
   $i \leftarrow i + 1$
 }
 \eIf{$n > 0$}{\Return{$minPos$}}{\Return{$-1$}}
\end{algorithm}

\Subtask Prova algoritmen med penna och papper på sekvensen $(1, 2, -1, 4)$ och rita minnessituationen efter varje runda i loopen. Vad blir skillnaden i exekveringsförloppet om loopvariablen $i$  initialiserats till $0$ i stället för $1$?

\Subtask Implementera algoritmen MININDEX i Scala i en funktion med denna signatur:
\begin{Code}
def indexOfMin(xs: Array[Int]): Int = ???
\end{Code}
Testa för olika fall: tom sekvens; sekvens med endast ett tal; lång sekvens med det minsta talet först, någonstans mitt i, samt sist.

\begin{Code}
// kod till facit
def indexOfMin(xs: Array[Int]): Int = {
  var minPos = 0
  var i = 1
  while (i < xs.size) {
    if (xs(i) < xs(minPos)) minPos = i
    i += 1
  }
  if (xs.size > 0) minPos else -1
}


\end{Code}

\newpage

\AdvancedTasks %%%%%%%%%%%%%%%%%


\SOLUTION


\QUESTEND






\WHAT{NEEDS A TOPIC DESCRIPTION}

\QUESTBEGIN

\Task  \what~ Läs om krullparenteser och vanliga parenteser på stack overflow: \\ \href{http://stackoverflow.com/questions/4386127/what-is-the-formal-difference-in-scala-between-braces-and-parentheses-and-when}{stackoverflow.com/questions/4386127/what-is-the-formal-difference-in-scala-between-braces-and-parentheses-and-when} och prova själv i REPL hur du kan blanda dessa olika slags parenteser på olika vis.

\SOLUTION


\QUESTEND






\WHAT{Tips:}

\QUESTBEGIN

\Task  \what~ Gör jämförande studier av Scalas api-dokumentation för \code{ArrayBuffer}, \code{Array} och \code{Vector}. Ge exempel på metoder som finns på objekt av typen \code{Array} och \code{ArrayBuffer} men inte på objekt av typen \code{Vector}.  Kolla efter metoder som returnerar \code{Unit}. Prova några muterande metoder på \code{Array} och \code{ArrayBuffer} i REPL.

\SOLUTION


\QUESTEND






\WHAT{Tips:}

\QUESTBEGIN

\Task  \what~ Bygg vidare på koden nedan och gör ett Sten-Sax-Påse-spel\footnote{\href{https://sv.wikipedia.org/wiki/Sten,\_sax,\_p\%C3\%A5se}{sv.wikipedia.org/wiki/Sten,\_sax,\_p\%C3\%A5se}} som även meddelar vem som vinner. Koden fungerar att köra som den är, men funktionen \code{winnerMsg} är ej klar.  Du kan använda modulo-räkning med \code{%}-operatorn för att avgöra vem som vinner.

\begin{Code}[basicstyle=\ttfamily\footnotesize\selectfont]]
object Rock {
  import javax.swing.JOptionPane
  import JOptionPane.{showOptionDialog => optDlg}

  def inputOption(msg: String, opt: Vector[String]) =
    optDlg(null, msg, "Option", 0, 0, null, opt.toArray[Object], opt(0))

  def msg(s: String) = JOptionPane.showMessageDialog(null, s)

  val opt =  Vector("Sten", "Sax", "Påse")

  def userChoice = inputOption("Vad väljer du?", opt)

  def computerChoice = (math.random * 3).toInt

  def winnerMsg(user: Int, computer: Int) = "??? vann!"

  def main(args: Array[String]): Unit = {
    var keepPlaying = true
    while (keepPlaying) {
      val u = userChoice
      val c = computerChoice
      msg("Du valde " + opt(u) + "\n" +
          "Datorn valde " + opt(c) + "\n" +
          winnerMsg(u, c))
      if (u != c) keepPlaying = false
    }
  }
}
\end{Code}\SOLUTION


\QUESTEND


\input{modules/w02-programs-lab.tex}

\chapter{Funktioner, Objekt}\label{chapter:W03}
\begin{multicols}{2}\begin{itemize}[nosep,label={$\square$}]
\item definera funktion
\item anropa funktion
\item parameter
\item returtyp
\item värdeandrop
\item namnanrop
\item default-argument
\item namngivna argument
\item applicera funktion på alla element i en samling
\item procedur
\item typen Unit och värdet ()
\item värdeanrop vs namnanrop
\item uppdelad parameterlista
\item skapa egen kontrollstruktur
\item objekt
\item modul
\item punktnotation
\item tillstånd
\item funktionsvärde
\item funktionstyp
\item äkta funktion
\item stegad funktion
\item apply
\item lazy val
\item aktiveringspost
\item rekursion
\item basfall
\item anropsstacken
\item objektheapen\end{itemize}\end{multicols}

%!TEX encoding = UTF-8 Unicode
%!TEX root = ../compendium1.tex

\ifPreSolution

\Exercise{\ExeWeekTHREE}\label{exe:W03}
\begin{Goals}
%!TEX encoding = UTF-8 Unicode
%!TEX root = ../exercises.tex

\item Kunna skapa och använda funktioner med en eller flera parametrar, default-argument, namngivna argument, och uppdelad parameterlista.
\item Kunna använda funktioner som äkta värden.
\item Kunna skapa och använda anonyma funktioner (ä.k. lambda-funktioner).
\item Kunna applicera en funktion på element i en samling.
\item Förstå skillnader och likheter mellan en funktion och en procedur.
\item Förstå vad ett block och en lokal variabel är.
\item Kunna skapa och använda lokala funktioner och förklara nyttan med dessa.
\item Förstå skillnader och likheter mellan värdeanrop och namnanrop.
\item Kunna skapa en enkel kontrollstruktur med fördröjd evaluering av ett block.
\item Förstå skillnaden mellan äkta funktioner och funktioner med sidoeffekter.
%\item Kunna skapa och använda variabler med fördröjd initialisering och förstå när de är användbara.
\item Kunna förklara hur nästlade funktionsanrop sker med   aktiveringsposter.
\item Känna till rekursion och kunna förklara hur rekursiva funktioner fungerar.
\item Känna till att det går att partiellt applicera argument på funktioner med uppdelad parameterlista för att skapa s.k. stegade funktioner (ä.k. curry-funktioner).

%\item Känna till svansrekursion och att svansrekursiva funktioner kan optimeras till loopar.

\end{Goals}

\begin{Preparations}
\item \StudyTheory{03}
\end{Preparations}

\BasicTasks %%%%%%%%%%%%%%%%

\else

\ExerciseSolution{\ExeWeekTHREE}

\fi





\WHAT{Para ihop begrepp med beskrivning.}

\QUESTBEGIN

\Task \what~Koppla varje begrepp med den (förenklade) beskrivning som passar bäst:

\begin{ConceptConnections}
  funktionshuvud & 1 & & A & har parameterlista och eventuellt en returtyp \\ 
  funktionskropp & 2 & & B & beskriver namn och typ på parametrar \\ 
  parameterlista & 3 & & C & argumentet evalueras innan anrop \\ 
  block & 4 & & D & en funktion som anropar sig själv \\ 
  namngivna argument & 5 & & E & gör att argument kan utelämnas \\ 
  defaultargument & 6 & & F & koden som exekveras vid funktionsanrop \\ 
  värdeanrop & 7 & & G & gör att en funktion kan flera resultatvärden \\ 
  namnanrop & 8 & & H & gör att argument kan ges i valfri ordning \\ 
  tupel & 9 & & I & fördröjd evaluering av argument \\ 
  tupelreturtyp & 10 & & J & kan ha lokala namn; sista raden ger värdet \\ 
  äkta funktion & 11 & & K & funktion utan namn; kallas även lambda \\ 
  predikat & 12 & & L & ger alltid samma resultat om samma argument \\ 
  slumptalsfrö & 13 & & M & lista med bestämt antal (heterogena) värden \\ 
  anonym funktion & 14 & & N & ger återupprepningsbar sekvens av pseudoslumptal \\ 
  rekursiv funktion & 15 & & O & en funktion som ger ett booleskt värde \\ 
\end{ConceptConnections}

\SOLUTION

\TaskSolved \what

\begin{ConceptConnections}
  funktionshuvud & 1 & ~~\Large$\leadsto$~~ &  K & har parameterlista och eventuellt returtyp \\ 
  funktionskropp & 2 & ~~\Large$\leadsto$~~ &  M & koden som exekveras vid funktionsanrop \\ 
  parameterlista & 3 & ~~\Large$\leadsto$~~ &  I & beskriver namn och typ på parametrar \\ 
  parameter & 4 & ~~\Large$\leadsto$~~ &  N & namn i funktionshuvud; binds till argument \\ 
  argument & 5 & ~~\Large$\leadsto$~~ &  E & uttryck som är invärde vid funktionsanrop \\ 
  block & 6 & ~~\Large$\leadsto$~~ &  G & kan ha lokala namn; sista raden ger värdet \\ 
  namngivna argument & 7 & ~~\Large$\leadsto$~~ &  H & gör att argument kan ges i valfri ordning \\ 
  default-argument & 8 & ~~\Large$\leadsto$~~ &  L & gör att argument kan utelämnas \\ 
  värdeanrop & 9 & ~~\Large$\leadsto$~~ &  B & argumentet evalueras innan anrop \\ 
  namnanrop & 10 & ~~\Large$\leadsto$~~ &  A & fördröjd evaluering av argument \\ 
  tupel & 11 & ~~\Large$\leadsto$~~ &  J & lista med bestämt antal (heterogena) värden \\ 
  tupelreturtyp & 12 & ~~\Large$\leadsto$~~ &  D & gör att en funktion kan flera resultatvärden \\ 
  anonym funktion & 13 & ~~\Large$\leadsto$~~ &  F & funktion utan namn; kallas även lambda \\ 
  rekursiv funktion & 14 & ~~\Large$\leadsto$~~ &  C & en funktion som anropar sig själv \\ 
\end{ConceptConnections}

\QUESTEND





\WHAT{Definiera och anropa funktioner.}

\QUESTBEGIN

\Task \label{task:funcall} \what~
En funktion med en parameter definieras med följande syntax i Scala:
\vspace{0.5em} \\
\texttt{\code{def} \textit{namn}(\textit{parameter}: \textit{Typ} = \textit{defaultArgument}): \textit{Returtyp} = \textit{returvärde}}

% En funktion med två parametrar definieras med följande syntax i Scala: \vspace{0.5em} \\  \texttt{\code{def} \textit{namn}(\textit{parameter1}: \textit{Typ1}, \textit{parameter2}: \textit{Typ2}): \textit{Returtyp} = \textit{returvärde}}

\Subtask Definiera funktionen \code{öka} som har en heltalsparameter \code{x} och vars returvärde är argumentet plus 1. Defaultargument ska vara 1. Ange returtypen explicit.

\Subtask Vad har uttrycket \code{öka(öka(öka(öka())))} för värde?

\Subtask Definiera funktionen \code{minska} som har en heltalsparameter \code{x} och vars returvärde är argumentet minus 1. Defaultargument ska vara 1. Ange returtypen explicit.

\Subtask Vad är värdet av uttrycket \code{öka(minska(öka(öka(minska(minska())))))}

\Subtask Vad är det för skillnad mellan parameter och argument?

\SOLUTION

\TaskSolved \what

\SubtaskSolved
\begin{Code}
def öka(x: Int = 1): Int = x + 1
\end{Code}

\SubtaskSolved  \code{5}

\SubtaskSolved
\begin{Code}
def minska(x: Int = 1): Int = x - 1
\end{Code}

\SubtaskSolved  \code{1}

\SubtaskSolved
\begin{itemize}
  \item \emph{Kort, förenklad förklaring:} Parametern i funktionshuvudet är ett lokalt namn på indata som kan användas i funktionskroppen, medan argumentet är själva värdet på parametern som skickas med vid anrop.
  \item \emph{Längre, mer exakt förklaring:} En \textbf{parameter} är en deklaration av en oföränderlig variabel i ett funktionshuvud vars namn finns tillgängligt lokalt i funktionskroppen. Vid anrop \emph{binds} parameternamnet till ett specifikt argument. Ett \textbf{argument} är ett uttryck som  appliceras på en funktion vid anrop. Normalt evalueras argumentet innan anropet sker, men om parametertypen föregås av \code{=>} fördröjs evalueringen av argumentet och sker i stället \emph{varje gång} parameternamnet förekommer i funktionskroppen.
\end{itemize}

\QUESTEND



\WHAT{Implementera funktion på olika sätt.}

\QUESTBEGIN

\Task \label{task:funcsumfirst} \what~
Skapa en funktion som kan summera de första \code{n} positiva heltalen.

\Subtask Skriv först funktionshuvudet med \code{???} som funktionskropp. Ge funktionen ett bra namn. Ange returtyp. Kontrollera att din funktion kompilerar utan kompileringsfel innan du går vidare.

\Subtask Implementera funktionen med hjälp av ett intervall och metoden \code{sum}. Testa så att funktionen fungerar. Vad händer om du ger ett negativt argument?

\Subtask Implementera funktionen med hjälp av \code{while}-\code{do}. Vad händer om du ger ett negativt argument?

\SOLUTION

\TaskSolved \what

\SubtaskSolved
\begin{Code}
def sumFirst(n: Int): Int = ???
\end{Code}

\SubtaskSolved
\begin{Code}
def sumFirst(n: Int): Int = (1 to n).sum
\end{Code}
\begin{REPL}
scala> sumFirst(-1)
val res0: Int = 0
\end{REPL}

\SubtaskSolved
\begin{Code}
def sumFirst(n: Int): Int = 
  var result = 0
  var i = 1
  while i <= n do 
    result += i
    i += 1
  end while
  result
end sumFirst
\end{Code}
\begin{REPL}
scala> sumFirst(-1)
val res1: Int = 0
\end{REPL}

\QUESTEND




\WHAT{Textspelet AliensOnEarth.}

\QUESTBEGIN

\Task  \what~Ladda ner spelet nedan \footnote{
\url{https://raw.githubusercontent.com/lunduniversity/introprog/master/compendium/examples/AliensOnEarth.scala}} och studera koden.

\scalainputlisting[basicstyle=\ttfamily\fontsize{10}{12}\selectfont,numbers=left]{examples/AliensOnEarth.scala}

% def randomDistribution(weights: Vector[Int]): Int = {
%   require(weights.size > 0)
%   require(weights.forall(_ >= 0))
%
%   val probabilities = for (w <- weights) yield w / weights.sum.toDouble
%   val rnd = math.random()
%   var i = 0
%   var sum = probabilities(i)
%   while (i < probabilities.size - 1 && rnd > sum) {
%     i += 1
%     sum += probabilities(i)
%   }
%   i
% }

\Subtask Medan du läser koden, försök lista ut vilket som är bästa strategin för att få så mycket poäng som möjligt. Kompilera och kör spelet i terminalen med ditt favoritnamn som argument. Vilket av de tre objekten på planeten jorden har störst sannolikhet att vara bästa alternativet?

\Subtask Para ihop kodsnuttarna nedan med bästa beskrivningen.\footnote{Gör så gott du kan även om allt inte är solklart. Vissa saker kommer vi att gå igenom i detalj först under senare kursmoduler.}

\begin{ConceptConnections}
  \code|options.indices| & 1 & & A & fångar undantag för att förhindra krasch \\ 
  \code|"1X2".toLowercase| & 2 & & B & gör om en sträng till små bokstäver \\ 
  \code|Random.nextInt(n)| & 3 & & C & slumptal i intervallet \code|0 until n| \\ 
  \code|try { } catch { }| & 4 & & D & sträng som kan sträcka sig över flera kodrader \\ 
  \code|""" ... """| & 5 & & E & heltalssekvens med alla index i en sekvens \\ 
  \code|s.stripMargin| & 6 & & F & tar bort marginal till och med vertikalstreck \\ 
  \code|e.printStackTrace| & 7 & & G & skriver ut information om ett undantag \\ 
\end{ConceptConnections}

\noindent\emph{Tips:} Med hjälp av REPL kan du ta reda på hur olika delar fungerar, t.ex.:

\begin{REPL}
scala> val xs = Vector("p", "w", "a")
scala> xs.indices
scala> xs.indices.foreach(i => println(i))
scala> xs.indexOf("w")
scala> xs.indexOf("gurka")
scala> Vector("hej", "hejsan", "hej").indexOf("hej")
scala> try 1 / 0 catch case e: Exception => println(e)
\end{REPL}
%Kolla även dokumentationen för \code{nextInt}, \code{readLine}, m.fl genom att söka här: \\ \url{http://www.scala-lang.org/api/current/index.html}


%\begin{framed}
\noindent\emph{Tips inför fortsättningen:}

\begin{itemize}[nolistsep]
  \item När jag hittade på \code{AliensOnEarth} började jag med ett mycket litet program med en enkel \code{main}-funktion som bara skrev ut något kul. Sedan byggde jag vidare på programmet steg för steg och kompilerade och testade efter varje liten ändring.

  \item När jag kodar har jag REPL igång i ett eget terminalfönster och min kodeditor i ett annat fönster. I ett tredje fönster har jag en terminal med kompilering i \textit{watch mode}, se appendix \ref{appendix:build-scala-cli-watch-mode}. Fråga en handledare om hur du kan arbeta effektivt med stegvisa experimentering i REPL för att bygga upp ett allt större program i små steg.

  \item Detta arbetssätt tar ett tag att komma in i, men är ett bra sätt att uppfinna allt större och bättre program. Ett stort program byggs lättast i små steg och felsökning blir mycket lättare om man bara gör små tillägg åt gången.

  \item Du får också det mycket lättare att förstå ditt program om du delar upp koden i många korta funktioner med bra namn. Du kan sedan lättare hitta på mer avancerade funktioner genom att återanvända befintliga.

  \item Under veckans laboration ska du utveckla ditt eget textspel. Då har du nytta av att återanvända funktionerna för indata och slumpdragning från exempelprogrammet \code{AliensOnEarth}.
\end{itemize}

%\end{framed}


\SOLUTION

\TaskSolved \what~

\SubtaskSolved \code{"penguin"} är bästa alternativ med sannolikheten $\frac{1}{2} + \frac{1}{2}\cdot\frac{1}{3} = \frac{2}{3}$

\SubtaskSolved

\begin{ConceptConnections}
    \code|options.indices| & 1 & ~~\Large$\leadsto$~~ &  F & heltalssekvens med alla index i en sekvens \\ 
  \code|"1X2".toLowercase| & 2 & ~~\Large$\leadsto$~~ &  C & gör om en sträng till små bokstäver \\ 
  \code|Random.nextInt(n)| & 3 & ~~\Large$\leadsto$~~ &  D & slumptal i intervallet \code|0 until n| \\ 
  \code|try { } catch { }| & 4 & ~~\Large$\leadsto$~~ &  B & fångar undantag för att förhindra krasch \\ 
  \code|""" ... """| & 5 & ~~\Large$\leadsto$~~ &  G & sträng som kan sträcka sig över flera kodrader \\ 
  \code|s.stripMargin| & 6 & ~~\Large$\leadsto$~~ &  A & tar bort marginal till och med vertikalstreck \\ 
  \code|e.printStackTrace| & 7 & ~~\Large$\leadsto$~~ &  E & skriver ut information om ett undantag \\ 
\end{ConceptConnections}

\QUESTEND



\WHAT{Äkta funktioner.}

\QUESTBEGIN

\Task  \what~  En äkta funktion%
\footnote{Äkta funktioner uppfyller per definition  \textit{referentiell transparens} \Eng{referential transparency} som du kan läsa mer om här:  \href{https://simple.wikipedia.org/wiki/Referential_transparency}{simple.wikipedia.org/wiki/Referential\_transparency}}
\Eng{pure function} ger alltid samma resultat med samma argument (så som vi är vana vid inom matematiken) och har inga externt observerbara sidoeffekter (till exempel utskrifter).

Vilka funktioner nedan är äkta funktioner?
\begin{Code}
var x = 0
val y = x

def inc(i: Int) = i + 1

def nöff(i: Int) = 
  x = x + i
  "nöff " * x
end nöff

def addX(i: Int) = x + i

def addY(i: Int) = y + i

def isPalindrome(s: String) = s == s.reverse

def rnd(min: Int, max: Int) = math.random() * max + min
\end{Code}


\noindent\emph{Tips:} Skriv av och testa funktionerna i REPL en och en, så att du förstår exakt vad som händer.

\SOLUTION

\TaskSolved \what

\begin{itemize}
  \item Funktionerna  \code{inc}, \code{addY} och \code{isPalindrome} är äkta. Notera att \code{y}-variablen initialiseras till \code{0} och kan sedan inte ändras eftersom den är deklarerad med nyckelordet \code{val}.
\end{itemize}

\QUESTEND


\WHAT{Applicera funktion på varje element i en samling. Funktion som argument.}

\QUESTBEGIN

\Task  \what~

\noindent Deklarera funktionen \code{öka} och variabeln \code{xs} enligt nedan i REPL:
\begin{REPL}
scala> def öka(x: Int) = x + 1
scala> val xs = Vector(3, 4, 5)
\end{REPL}
\noindent Para ihop nedan uttryck till vänster med det uttryck till höger som har samma värde. Om du undrar något, testa uttrycken och olika varianter av dem i REPL.

\begin{ConceptConnections}
  \code|for (i <- 1 to 3) yield öka(i)| & 1 & & A & \code|Vector(5, 6, 7)| \\ 
  \code|Vector(2, 3, 4).map(i => öka(i))| & 2 & & B & \code|Vector(4, 5, 6)| \\ 
  \code|xs.map(öka)| & 3 & & C & \code|Vector(2, 3, 4)| \\ 
  \code|xs.map(öka).map(öka)| & 4 & & D & \code|()| \\ 
  \code|xs.foreach(öka)| & 5 & & E & \code|xs| \\ 
\end{ConceptConnections}

\SOLUTION

\TaskSolved \what

\begin{ConceptConnections}
    \code|for (i <- 1 to 3) yield öka(i)| & 1 & ~~\Large$\leadsto$~~ &  D & \code|Vector(2, 3, 4)| \\ 
  \code|Vector(2, 3, 4).map(i => öka(i))| & 2 & ~~\Large$\leadsto$~~ &  C & \code|xs| \\ 
  \code|xs.map(öka)| & 3 & ~~\Large$\leadsto$~~ &  E & \code|Vector(4, 5, 6)| \\ 
  \code|xs.map(öka).map(öka)| & 4 & ~~\Large$\leadsto$~~ &  A & \code|Vector(5, 6, 7)| \\ 
  \code|xs.foreach(öka)| & 5 & ~~\Large$\leadsto$~~ &  B & \code|()| \\ 
\end{ConceptConnections}

\QUESTEND




\WHAT{Anonyma funktioner.}

\QUESTBEGIN

\Task  \what~  Vi har flera gånger sett syntaxen \code{i => i + 1}, till exempel i en loop \code{(1 to 10).map(i => i + 1)} där funktionen \code{i => i + 1} appliceras på alla heltal från 1 till och med 10 och resultatet blir en ny sekvenssamling.

Syntaxen \code{(i: Int) => i + 1} är en litteral för att skapa ett \emph{funktionsvärde} (kallas även \emph{anonym funktion} eller \emph{lambda-uttryck}). Syntaxen liknar den för funktionsdeklarationer, men nyckelordet \code{def} saknas i funktionshuvudet och i stället för likhetstecken används \code{=>} för att avskilja parameterlistan från funktionskroppen.
Om kompilatorn kan härleda typen ur sammanhanget kan kortformen \code{i => i + 1} användas.

Det finns ett \emph{ännu} kortare sätt att skriva en anonym funktion \emph{om} typen kan härledas \emph{och} den bara använder sin parameter \emph{en enda gång}; då går funktionslitteraler att skriva med s.k. \emph{platshållarsyntax} som använder understreck, till exempel \code{ _ + 1} och som automatiskt expanderas av kompilatorn till \code{ngtnamn => ngtnamn + 1} (namnet på parametern spelar ingen roll; kompilatorn väljer något eget, internt namn).

Para ihop uttryck till vänster med uttryck till höger som har samma värde:

\begin{ConceptConnections}
\input{generated/quiz-w03-lambda-taskrows-generated.tex}
\end{ConceptConnections}

\noindent
Funktionslitteraler kallas \textit{anonyma funktioner}, eftersom de inte har något namn, till skillnad från t.ex. \code{def öka(i: Int): Int = i + 1}, som ju heter \code{öka}. Ett annat vanligt namn är \textit{lambda-uttryck} efter det datalogiska matematikverktyget \href{https://sv.wikipedia.org/wiki/Lambdakalkyl}{lambdakalkyl}.

\SOLUTION

\TaskSolved \what

\begin{ConceptConnections}
    \code|(0 to 2).map(i => i + 1)           | & 1 & ~~\Large$\leadsto$~~ &  B & \code|(2 to 4).map(i => i - 1)| \\ 
  \code|(1 to 3).map(_ + 1)                | & 2 & ~~\Large$\leadsto$~~ &  D & \code|Vector(2, 3, 4)         | \\ 
  \code|(2 to 4).map(math.pow(2, _))       | & 3 & ~~\Large$\leadsto$~~ &  A & \code|Vector(4.0, 8.0, 16.0)  | \\ 
  \code|(3 to 5).map(math.pow(_, 2))       | & 4 & ~~\Large$\leadsto$~~ &  C & \code|Vector(9.0, 16.0, 25.0) | \\ 
  \code|(4 to 6).map(_.toDouble).map(_ / 2)| & 5 & ~~\Large$\leadsto$~~ &  E & \code|Vector(2.0, 2.5, 3.0)   | \\ 
\end{ConceptConnections}

\QUESTEND




\WHAT{Skapa din egen kontrollstruktur med hjälp av namnanrop.}\label{func:upprepa}

\QUESTBEGIN

\Task  \what~Namnanrop skrivs med en raket efter kolon före parametertypen och innebär att argumentet evalueras på plats varje gång.

\Subtask Använd namnanrop i kombination med en uppdelad parameterlista och skapa din egen kontrollstruktur enligt nedan.\footnote{Det är så loopen \code{upprepa} i Kojo är definierad.}
\begin{Code}
def upprepa(n: Int)(block: => Unit): Unit =
  var i = 0
  while i < n do 
    ???
  end while
\end{Code}

\Subtask
Testa din kontrollstruktur i REPL. Låt upprepa 100 gånger att ett slumptal mellan 1 och 6 dras och sedan skrivs ut. Prova även att använda färre klammerparenteser med hjälp av kolon.

\Subtask
Varför behövs namnanrop här?

\SOLUTION

\TaskSolved \what

\SubtaskSolved
\begin{Code}
def upprepa(n: Int)(block: => Unit): Unit =
  var i = 0
  while i < n do
    block
    i += 1
  end while
\end{Code}

\SubtaskSolved
\begin{Code}
upprepa(100):
  val tärningskast = (math.random() * 6 + 1).toInt
  print(s"\$tärningskast ")
\end{Code}

\SubtaskSolved Om parametern \code{block} inte vore deklarerad med namnanrop så hade argumentet evaluerats en gång innan anropet och sedan hade det blivit samma resultat vid varje iteration. Med namnanrop kan block innehålla kod som t.ex. uppdaterar en variabel som vi vill ska ske vid varje iteration. Namn-anrop liknar att koden för argumentet ''klistras in'' på varje plats i funktionskroppen där parameternamnet förekommer. 

\QUESTEND



\WHAT{Lär dig läsa en stack trace.}

\QUESTBEGIN

\Task  \what~  Skriv ett program i filen \texttt{fel.scala} som orsakar ett \emph{körtidsfel} och kör igång det i terminalen med \code{scala-cli run fel.scala}. Studera den stack trace som skrivs ut. Vad innehåller en \code{stack trace}? Diskutera med handledare hur du kan ha nytta av en stack trace när du felsöker.

\SOLUTION

\TaskSolved \what En stack trace innehåller följande information:
\begin{enumerate}
  \item ett felmeddelande
  \item namn på alla funktioner som anropats vid tiden för körtidsfelet, enligt alla aktiveringsposter som ligger på anropsstacken 
  \item aktuell namnrymnd för varje funktionen, alltså paket/singelobjekt
  \item namnet på kodfilen för varje funktion
  \item radnummer i varje funktion 
  \item den funktion som kommer först är den funktion där felet inträffade
  \item eventuellt kan felet inträffa i standardbibliotekets funktioner och då är din egen funktion tidigare i anropskedjan
\end{enumerate}

Exempel på en stack trace:
\begin{REPLnonum}
> cat fel.scala 
@main def run = 
  println("Hej Scala!" + Vector().head)
> scala-cli run fel.scala
Compiling project (Scala 3.3.0, JVM)
Compiled project (Scala 3.3.0, JVM)
Exception in thread "main" java.util.NoSuchElementException: empty.head
	at scala.collection.immutable.Vector.head(Vector.scala:279)
	at fel$package$.run(fel.scala:2)
	at run.main(fel.scala:1)
>
\end{REPLnonum}

\QUESTEND


\ExtraTasks %%%%%%%%%%%%%%%%%%%%%%%%%%%%%%%%%%%%%%%%%%%%%%%%%%%%%%%%%%



\WHAT{Funktion med flera parametrar.}

\QUESTBEGIN

\Task  \what~  

\Subtask Definiera i REPL två funktioner \code{sum} och \code{diff} med två heltalsparametrar som returnerar summan respektive differensen av argumenten:
\begin{Code}
def sum(x: Int, y: Int): Int = ???

def diff(x: Int, y: Int): Int = ???
\end{Code}
Vad har nedan uttryck för värden? Förklara vad som händer.

\Subtask \code{diff(0, 100)}

\Subtask \code{diff(100, sum(42, 43))}

\Subtask \code{sum(sum(42, 43), diff(100, sum(0, 0)))}

\Subtask \code{sum(diff(Byte.MaxValue, Byte.MinValue), 1)}

\SOLUTION

\TaskSolved \what

\SubtaskSolved
\begin{Code}
  def sum(x: Int, y: Int): Int = x + y
  
  def diff(x: Int, y: Int): Int = x - y
\end{Code}
  

\SubtaskSolved  Det blir \code{-100} efter som \code{0 - 100 == -100} 

\SubtaskSolved  Det blir \code{15} eftersom det nästlade anropet motsvarar \\\code{diff(100, 42 + 43) == (100 - 85)}

\SubtaskSolved  Det blir \code{185} eftersom det nästlade anropet motsvarar \\\code{sum(42 + 43, 100 - 0) == (85 + 100)}

\SubtaskSolved  Det blir \code{256} eftersom \code{Byte.MaxValue == 127} och \  code{Byte.MinValue == -128} och \code{sum(127 + 128, 1) == 256}

\QUESTEND



\WHAT{Medelvärde.}

\QUESTBEGIN

\Task  \what~ Skriv och testa en funktion \code{avg} som räknar ut medelvärdet mellan två heltal och returnerar en \code{Double}.

\SOLUTION

\TaskSolved \what

\begin{Code}
def avg(x: Int, y: Int): Double = (x + y) / 2.0
\end{Code}

\QUESTEND




\WHAT{Funktionsanrop med namngivna argument.}

\QUESTBEGIN

\Task  \what~
\begin{REPL}
scala> def skrivNamn(efternamn: String, förnamn: String) =
         println(s"Namn: $efternamn, $förnamn")
scala> skrivNamn(förnamn = "Stina", efternamn = "Triangelsson")
scala> skrivNamn(efternamn = "Oval", "Viktor")

\end{REPL}

\Subtask Vad skrivs ut efter rad 3 resp. rad 4 ovan?

\Subtask Nämn tre fördelar med namngivna argument.

\SOLUTION

\TaskSolved \what~

\SubtaskSolved
\begin{REPL}
Namn: Triangelsson, Stina
Namn: Oval, Viktor
\end{REPL}

\SubtaskSolved
\begin{itemize}
  \item Anroparen kan själv välja ordning.
  \item Koden blir lättare att begripa om parameternamnen är självbeskrivande.
  \item Hjälper till att förhindra buggar som beror på förväxlade parametrar.
\end{itemize}

\QUESTEND



\WHAT{Funktion som äkta värde.}

\QUESTBEGIN

\Task  \what~  Funktioner är \emph{äkta värden} i Scala%\footnote{I likhet med t.ex. Javascript, men till skillnad från t.ex. Java.}
. Det betyder att variabler kan ha funktioner som värden och funktionsvärden kan vara argument till funktioner som har funktionsparametrar. Funktioner som tar funktioner som argument kallas \emph{högre ordningens funktioner}.

En funktion som har en heltalsparameter och ett heltalsresultat är av funktionstypen \code{Int => Int} (uttalas \emph{int-till-int}) och värdet av funktionen utgör ett objekt som har en metod som heter \code{apply} med motsvarande funktionstyp.

\Subtask \label{subtask:funcval} Deklarera nedan funktioner och variabler i REPL. Para sedan ihop nedan uttryck till vänster med det uttryck till höger som skapar samma utskrift. Om du undrar något, testa uttrycken och olika varianter av dem i REPL.

\begin{REPL}
scala> def hälsa(): Unit = println("Hej!")
scala> def fleraAnrop(antal: Int, f: () => Unit): Unit =
         for _ <- 1 to antal do f()
scala> val f1 = () => hälsa()
scala> var f2 = (s: String) => println(s)
scala> val f3 = () => f2("Thunk")
\end{REPL}

\begin{ConceptConnections}
  \code| fleraAnrop(1, hälsa) | & 1 & & A & \code| f2("Hej!\nHej!")| \\ 
  \code| fleraAnrop(3, hälsa) | & 2 & & B & \code| fleraAnrop(3, f1)  | \\ 
  \code| fleraAnrop(2, f1)    | & 3 & & C & \code| f3()               | \\ 
  \code| fleraAnrop(1, f3)    | & 4 & & D & \code| f2("Hej!")       | \\ 
\end{ConceptConnections}


\Subtask Vilka typer har variablerna \code{f1}, \code{f2} och \code{f3}?

\Subtask Funkar detta? Varför? \code{f2 = f1}

\Subtask Funkar detta? Varför? \code{val f4 = fleraAnrop}

\Subtask Funkar detta? Varför? \code{val f4 = hälsa}

\Subtask Funkar detta? Varför? \code{val f4: () => Unit = hälsa}

\SOLUTION

\TaskSolved \what

\SubtaskSolved

\begin{ConceptConnections}
    \code| fleraAnrop(1, hälsa) | & 1 & ~~\Large$\leadsto$~~ &  D & \code| f2("Hej!")       | \\ 
  \code| fleraAnrop(3, hälsa) | & 2 & ~~\Large$\leadsto$~~ &  B & \code| fleraAnrop(3, f1)  | \\ 
  \code| fleraAnrop(2, f1)    | & 3 & ~~\Large$\leadsto$~~ &  A & \code| f2("Hej!\nHej!")| \\ 
  \code| fleraAnrop(1, f3)    | & 4 & ~~\Large$\leadsto$~~ &  C & \code| f3()               | \\ 
\end{ConceptConnections}

\SubtaskSolved \code{f1} och \code{f3} är av typen \code{() => Unit} och \code{f2} av typen \code{String => Unit}.

\SubtaskSolved  Nej. \code{f1} och \code{f2} är av två olika funktionstyper.

\SubtaskSolved  Ja, det går fint.

\SubtaskSolved  Nej. När funktionen inte har någon parameter behöver kompilatorn mer information för att vara säker på att det är ett funktionsvärde du vill ha.

\SubtaskSolved Ja! Nu med typinformationen på plats är kompilatorn säker på vad du vill göra.

\QUESTEND



\WHAT{Bortkastade resultatvärden och returtypen \code{Unit}.}

\QUESTBEGIN

\Task  \what~ Undersök nedan kod i REPL och förklara vad som händer.

\Subtask
\begin{REPL}
scala> def tom = println("")
scala> println(tom)
\end{REPL}

\Subtask
\begin{REPL}
scala> def bortkastad: Unit = 1 + 1
scala> println(bortkastad)
\end{REPL}

\Subtask
\begin{REPL}
scala> def bortkastad2 = { val x = 1 + 1 }
scala> println(bortkastad2)
\end{REPL}

\Subtask Varför är det bra att explicit ange \code{Unit} som returtyp för procedurer?

\SOLUTION

\TaskSolved \what

\SubtaskSolved Procedurer returnerar tomma värdet och \code{println} är en procedur. När tomma värdet skrivs ut visas \code{()}.

\SubtaskSolved Procedurer returnerar tomma värdet. Om du anger returtyp \code{Unit} explicit, har du bättre chans att kompilatorn kan ge varning då uträkningar kommer att kastas bort. En varning avbryter inte exekveringen, utan är ett sätt för kompilatorn att ge dig tips om saker som kan behöva fixas till i din kod.

\SubtaskSolved I Scala är variabeldeklaration, precis som en tilldelningssats, och inte ett uttryck och saknar värde.

\SubtaskSolved  Koden blir lättare att läsa och kompilatorn får bättre möjlighet att hjälpa till med varningar om resultatvärden riskerar att bli bortkastade.

\QUESTEND


\WHAT{Namnanrop.}

\QUESTBEGIN

\Task  \what~

Deklarera denna procedur i REPL:
\begin{Code}
def görDettaTvåGånger(b: => Unit): Unit = { b; b }
\end{Code}

Anropa \code{görDettaTvåGånger} med ett block som parameter. Blocket ska innehålla en utskriftssats. Förklara vad som händer.

\SOLUTION

\TaskSolved \what

Blocket är ett uttryck som har värdet \code{(): Unit}. Evalueringen av blocket sker där namnet \code{b} förekommer i procedurkroppen, vilket är två gånger.
\begin{REPL}
scala> görDettaTvåGånger { println("goddag") }
goddag
goddag
\end{REPL}

\QUESTEND




\clearpage

\AdvancedTasks %%%%%%%%%%%%%%%%%%%%%%%%%%%%%%%%%%%%%%%%%%%%%%%%%%%%%%%%%%%




\WHAT{Föränderlighet av parametrar.}

\QUESTBEGIN

\Task \what~Vad tror du om detta: Är en parameter förändringsbar i funktionskroppen ...

\Subtask ... i Scala?  (Ja/Nej)

\Subtask ... i Java?  (Ja/Nej)

\Subtask ... i Python?  (Ja/Nej)


\SOLUTION

\TaskSolved \what~

\Subtask Nej, i Scala är parametern oföränderlig och det blir kompileringsfel om man försöker tilldela den ett nytt värde i funktionskroppen.

\Subtask \Subtask Ja det går utmärkt i både Java och Python att ändra värdet på parametern i funktionskroppen med tilldelning, men koden riskerar att bli förvirrande.\\
\url{https://stackoverflow.com/questions/2970984}

\QUESTEND



\WHAT{Värdeanrop och namnanrop.}

\QUESTBEGIN

\Task  \what~Normalt sker i Scala (och i Java) s.k. \emph{värdeanrop} vid anrop av funktioner, vilket innebär att argumentuttrycket evalueras \emph{före} bindningen till parameternamnet sker.

Man kan också i Scala (men inte i Java) med syntaxen \code{=>} framför parametertypen deklarera att \emph{namnanrop} ska ske, vilket innebär att evalueringen av argumentuttrycket \emph{fördröjs} och sker \emph{varje gång} namnet används i metodkroppen.

Deklarera nedan funktioner i REPL.

\begin{Code}
def snark: Int = { print("snark "); Thread.sleep(1000); 42 }
def callByValue(x: Int):   Int = x + x
def callByName(x: => Int): Int = x + x
lazy val zzz = snark
\end{Code}

\noindent Förklara vad som händer när nedan uttryck evalueras.

\Subtask \code{snark + snark}

\Subtask \code{callByValue(snark)}

\Subtask \code{callByName(snark)}

\Subtask \code{callByName(zzz)}

\SOLUTION

\TaskSolved \what

\SubtaskSolved Vid varje anrop av \code{snark} sker en utskrift och en fördröjnig innan $42$ returneras. \\\code{42 + 42 == 84} vilket blir värdet av uttrycket.
\begin{REPL}
scala> snark + snark
snark snark val res1: Int = 84
\end{REPL}

\SubtaskSolved Uttrycket \code{snark} evalueras direkt vid anropet och parametern \code{x} binds till värdet $42$ och i funktionskroppen beräknas $42+42$. Utskriften sker bara en gång.
\begin{REPL}
callByValue(snark)
snark val res2: Int = 84
\end{REPL}

\SubtaskSolved Evalueringen av uttrycket \code{snark} fördröjs tills varje förekomst av parametern \code{x} i funktionskroppen. Utskriften sker två gånger.
\begin{REPL}
callByName(snark)
snark snark val res3: Int = 84
\end{REPL}

\SubtaskSolved Evalueringen av uttrycket \code{zzz} fördröjs tills varje förekomst av parametern \code{x} i funktionskroppen. Utskriften sker en gång eftersom \code{val}-variabler tilldelas sitt värde en gång för alla vid den fördröjda initialiseringen.
\begin{REPL}
callByName(zzz)
snark val res4: Int = 84
\end{REPL}

\QUESTEND



\WHAT{Skapa egen kontrollstruktur för iteration med loop-variabel.}

\QUESTBEGIN

\Task  \what~

\Subtask Fördelen med \code{upprepa} i uppgift \ref{func:upprepa} är att den är koncis och lättanvänd. Men den är inte lika lätt att använda om man behöver tillgång till en loopvariabel. Implementera därför nedan kontrollstruktur.

\begin{Code}
def repeat(n: Int)(p: Int => Unit): Unit = 
  var i = 0
  while i < n do
    ??? 
\end{Code}

\Subtask Använd \code{repeat} för att 100 gånger skriva ut loopvariabeln och ett slumpdecimaltal mellan 0 och 1.


\SOLUTION

\TaskSolved \what

\SubtaskSolved
\begin{Code}
def repeat(n: Int)(p: Int => Unit): Unit = 
  var i = 0
  while i < n do
    p(i)
    i += 1
  end while
end repeat
\end{Code}

\SubtaskSolved

\begin{Code}
repeat(100){ i =>
  print("i ")
  println(math.random())
}
\end{Code}
Du kan använda färre klammerparenteser med hjälp av kolon:
\begin{Code}
repeat(100): i =>
  print("i ")
  println(math.random())
\end{Code}

\QUESTEND






\WHAT{Uppdelad parameterlista och stegade funktioner.}

\QUESTBEGIN

\Task \what~Man kan dela upp parametrarna till en funktion i flera parameterlistor. Funktionen \code{add1} nedan har en parameterlista med två parametrar medan \code{add2} har två parameterlistor med en parameter vardera:
\begin{Code}
  def add1(a: Int, b: Int) = a + b
  def add2(a: Int)(b: Int) = a + b
\end{Code}

\Subtask  När man anropar funktionen \code{add2} ska argumenten skrivas inom två olika parentespar. Hur kan du använda \code{add2} för att räkna ut \code{1 + 1}?

\Subtask En fördel med uppdelade parameterlistor är att man kan skapa s.k. \emph{stegade funktioner}\footnote{Kallas även Curry-funktioner efter matematikern och logikern Haskell Brooks Curry.} där argumenten är partiellt applicerade. Prova det stegade funktionsvärdet \code{singLa} nedan. Vad skrivs ut på efter raderna 3 och 5?

\begin{REPL}
scala> def repeat(s: String)(n: Int): String = s * n
scala> val song = repeat("doremi ")(3)
scala> println(song)
scala> val singLa = repeat("la")
scala> println(singLa(7))
\end{REPL}

\SOLUTION

\TaskSolved \what

\SubtaskSolved
\begin{REPL}
scala> def add2(a: Int)(b: Int) = a + b
def add2(a: Int)(b: Int): Int

scala> add2(1)(1)
val res0: Int = 2
\end{REPL}

\SubtaskSolved
\begin{itemize}

\item Rad 3:
\begin{REPLnonum}
doremi doremi doremi 
\end{REPLnonum}

\item Rad 5:
\begin{REPLnonum}
lalalalalalala
\end{REPLnonum}

\end{itemize}


\QUESTEND




\WHAT{Rekursion.}

\QUESTBEGIN

\Task\Uberkurs  \what~  En rekursiv funktion anropar sig själv.

\Subtask Förklara vad som händer nedan.

\begin{REPL}
scala> def countdown(x: Int): Unit = 
         if x > 0 then {println(x); countdown(x - 1)}
scala> countdown(10)
scala> countdown(-1)
scala> def finalCountdown(x: Byte): Unit =
         {println(x); Thread.sleep(100); finalCountdown((x-1).toByte); 1 / x}
scala> finalCountdown(Byte.MaxValue)
\end{REPL}

\Subtask Vad händer om du gör satsen som riskerar division med noll \emph{före} det rekursiva anropet i funktionen \code{finalCountdown} ovan?

\Subtask Förklara vad som händer nedan. Varför tar sista raden längre tid än näst sista raden?
\begin{REPL}
scala> def signum(a: Int): Int = if a >= 0 then 1 else -1
scala> def add(x: Int, y: Int): Int =
         if y == 0 then x else add(x + 1, y - signum(y))
scala> add(100, 100)
scala> add(Int.MaxValue, 0)
scala> add(0, Int.MaxValue)
\end{REPL}

\SOLUTION

\TaskSolved \what

\SubtaskSolved
\code{countdown} skriver ut x och gör ett rekursivt anrop med \code{x - 1} som argument, men bara om basvillkoret \code{x > 0} är uppfyllt. Resultatet blir en ändlig  repetition.
\code{finalCountdown} anropar sig själv rekursivt men saknar ett basvillkor som kan avbryta rekursionen, vilket genererar en oändlig repetition. Vid -128 blir det \emph{overflow} eftersom bitarna inte räcker till för större negativa tal och räkningen börjar om på 127. (Om minskar fördröjningen till \code{Thread.sleep(1)} blir det ganska snabbt \emph{stack overflow})

\SubtaskSolved
Eftersom vi hade \code{1/x} \emph{efter} det rekursiva anropet i föregående deluppgift, så kom vi aldrig till denna (potentiellt ödesdigra) beräkning, utan lade bara aktiveringsposter på hög på stacken vid varje anrop. Om vi placerar \code{1/x} \emph{före} det rekursiva anropet, så når vi detta uttryck direkt och det kastas ett undantag p.g.a. division med noll.

\SubtaskSolved
Den sista raden leder till många fler rekursiva anrop, så som basvillkoret och det rekursiva anropet är konstruerade. Lägg gärna in en \code{println}-sats före det rekursiva anropet och undersök i detalj vad som sker.

\QUESTEND



\WHAT{Undersök svansrekursion genom att kasta undantag.}

\QUESTBEGIN

\Task\Uberkurs  \what~  Förklara vad som händer. Kan du hitta bevis för att kompilatorn kan optimera rekursionen till en vanlig loop?

\begin{REPL}
scala> def explode = throw Exception("BANG!!!")
scala> explode
scala> def countdown(n: Int): Unit =
         if n == 0 then explode else countdown(n-1)
scala> countdown(10)
scala> countdown(10000)
scala> def countdown2(n: Int): Unit =
         if n == 0 then explode else {countdown2(n-1); print("no tailrec")}
scala> countdown2(10)
scala> countdown2(10000)
\end{REPL}

\SOLUTION

\TaskSolved \what~\code{countdown} är svansrekursiv eftersom det rekursiva anropet står \emph{sist} och kan då optimeras till en \code{while}-loop av kompilatorn. Det går fint att köra ända till det exploderar, även med 10000 anrop, och i felmeddelandet finns det endast ett anrop till \code{countdown}.

\code{countdown2} är inte svansrekursiv eftersom den har ett uttryck \code{efter} det rekursiva anropet. I felutskriften syns alla rekursiva anrop till \code{countdown2} innan basvillkoret inträffade. Vid \code{countdown2(10000)} uppfylls inte basvillkoret innan det blir \code{StackOverflowError}.

\QUESTEND



\WHAT{\code{@tailrec}-annotering.}

\QUESTBEGIN

\Task\Uberkurs  \what~  Du kan be kompilatorn att ge felmeddelande om den inte kan optimera koden till en motsvarande while-loop. Detta kan användas i de fall man vill vara helt säker på att kompilatorn kan optimera koden och det inte kan finnas risk för en överfull stack \Eng{stack overflow} på grund av för djup anropsnästling.

Prova nedan rader i REPL och förklara vad som händer.
\begin{REPL}
scala> def countNoTailrec(n: Long): Unit =
         if n <= 0L then println("Klar! " + n) else {countNoTailrec(n-1L); ()}
scala> countNoTailrec(1000L)
scala> countNoTailrec(100000L)
scala> import scala.annotation.tailrec
scala> @tailrec def countNoTailrec(n: Long): Unit =
         if n <= 0L then println("Klar! " + n) else {countNoTailrec(n-1L); ()}
scala> @tailrec def countTailrec(n: Long): Unit =
         if n <= 0L then println("Klar! " + n) else countTailrec(n-1L)
scala> countTailrec(1000L)
scala> countTailrec(100000L)
scala> countTailrec(Int.MaxValue.toLong * 2L)
\end{REPL}

\SOLUTION

\TaskSolved \what~Första gången \code{countNoTailrec(100000L)} anropas blir det \code{StackOverflowError}. Med annoteringen \code{@tailrec} får vi ett kompileringsfel eftersom kompilatorn inte kan optimera en icke svansrekursiv funktion. Om funktionen skrivs om kan kompilatorn optimera funktionen så att rekursionen byts ut mot en \code{while}-loop och vi kan köra så länge vi orkar utan att stacken flödar över. Och himla snabbt går det!!

\QUESTEND

%!TEX encoding = UTF-8 Unicode
%!TEX root = ../compendium2.tex

\Lab{\LabWeekTHREE}
\begin{Goals}
%!TEX encoding = UTF-8 Unicode
%!TEX root = ../compendium2.tex

%\item Kunna kompilera Scalaprogram med \texttt{scalac}.
%\item Kunna köra Scalaprogram med \texttt{scala}.
%\item Kunna definiera och anropa funktioner.
%\item Kunna använda och förstå default-argument.
%\item Kunna ange argument med parameternamn.
\item Kunna skapa ett större program med din egen kod efter dina egna idéer.
\item Kunna använda en editor och terminalen för att iterativt editera, kompilera, och testa din kod.
\item Kunna använda variabler i kombination med alternativ och repetetition i flera nivåer.
\item Kunna stegvis förbättra din kod för att underlätta förändring och öka läsbarhet.
\item Kunna skapa och använda abstraktioner för att generalisera och möjliggöra återanvändning av kod.

\end{Goals}

\begin{Preparations}
\item Gör övning \texttt{\ExeWeekTHREE} och repetera övning \texttt{\ExeWeekTWO} innan du påbörjar laborationen.
\item Läs appendix~\ref{appendix:terminal} och~\ref{appendix:compile}.
\item Hämta given kod via \href{https://github.com/lunduniversity/introprog/tree/master/workspace/}{kursen github-plats}.
\item Utveckla en första, spelbar version av ditt textspel, som du kan jobba vidare på under laborationen.
\item Hitta någon som spelar en tidig version av ditt spel och läser din kod och ger återkoppling på kodens läsbarhet. Skriv ner den återkoppling du får.
\item Spela någon annans textspel och ge återkoppling på kodens läsbarhet.
\end{Preparations}


\subsection{Krav}

\begin{itemize}
\item Du ska skapa ett lagom irriterande textspel med hjälp av en editor, till exempel VS \texttt{code} (se appendix~\ref{appendix:compile:edit}). Spelet ska köras i terminalen.

\item Under redovisningen av laborationen ska du redogöra för vilka programmeringskoncept du tränat på under utvecklingen av ditt textspel. Du ska också för handledaren beskriva hur du har förbättrat din kod genom den återkoppling du fått från någon som spelat ditt spel och läst koden.

\item Ditt textspel ska vara \emph{lagom} irriterande om den som spelar har läst koden, medan spelet gärna får vara orimligt irriterande för den som \emph{inte} läst koden. Det ska gå att klara spelet (du väljer själv vad det innebär) och därmed avsluta programmet inom rimlig tid med kännedom om koden.

\item Försök göra din kod \textit{lätt att läsa och förstå}, även om själva spelet stundtals kan vara mer eller mindre obegripligt, knasigt, eller besvärligt, för den spelare som inte har tillgång till koden... Observera att din kod inte behöver vara ''perfekt'' från början. Börja fritt och förbättra efterhand.

\item Allteftersom ditt program blir längre ska du omforma och dela upp din kod i många, korta abstraktioner med väl valda namn för att öka läsbarheten.

\item Din kod ska använda de viktiga begrepp som kursen hittills har behandlat, med speciellt fokus på det som just du behöver träna mest på.

%\item Slumptal ska ingå i ditt spel och styra valfria delar av exekveringen. Det ska även gå att ge ett valfritt slumptalsfrö som argument vid exekveringen av ditt program. Om fröargument ges ska exekveringen bli återupprepningsbar för en given indatasekvens, annars ska utfallet kunna bli olika vid upprepade körningar med samma indata.
\end{itemize}

\subsection{Tips för att komma igång}

\begin{itemize}
\item Skapa en katalog som innehåller en scala-kodfil med valfritt namn.
\item Skriv en enkel \code{@main}-metod i den nyskapade kodfilen som endast skriver ut strängen \code{"Hello World!"}.
\item Kompilera och kör, rätta eventuella fel tills programmet fungerar korrekt.
\item När programmet fungerar, börja utöka \code{@main}-metoden i din kodfil och implementera mer funktionalitet, ta en titt under inspiration nedan.
\item Börja enkelt och försök formulera vad ditt program ska göra med \emph{psuedokod} som kommentarer innan du skriver koden.
\item Kompilera och kör vid varje tillägg och håll varje tillägg så litet som möjligt, så slipper du reda ut en massa svåra följdfel vid kompilering och eventuella körtidsfel blir mer begripliga.  
\item Fortsätt utöka tills kraven för labben har uppnåtts.
\end{itemize}

\subsection{Inspiration}

Här följer en lista med olika förslag på funktioner som du kan välja bland, kombinera och variera på olika vis. Du kan också låta helt andra funktioner ingå i ditt spel. Det viktigaste är att du kombinerar kodglädje med lärorika utmaningar :)

\begin{itemize}
\item Be användaren logga in. Ge knasiga felmeddelande om användaren inte kan lösenordet.
\item Låt användaren hamna i en irriterande oändlig loop av meningslösa frågor om den gör ''fel''.
\item Beskriv en läskig fantasiplats där användaren befinner sig, till exempel en grotta | en källare | ett rymdskepp | Kemicentrum.
\item Låt användaren välja mellan fåniga vapen, till exempel golvmopp | örontops | foliehatt | förgiftad kexchoklad.
\item Låt användaren välja mellan olika vägar | dörrar | tunnlar | sektionscaféer. Låt valet styra vilka monster som påträffas. Låt användaren bekämpa monstret med olika vapen.
\item Inför någon slags poäng som redovisas under spelets gång och i slutet.
\item Inför olika sorters poäng för hälsa, stridskraft, uppnådd skicklighetsnivå, etc.
\item Fråga användaren om mer eller mindre relevanta detaljer: namn | skonummer | favorithusdjur. Ge knasiga kommentarer där dessa detaljer ingår som delsträngar.
\item Spela sten | sax | påse med användaren.
\item Spela ''gissa talet'' och ge ledtrådar om talet är för litet eller för stort.
\item Mät hur lång tid det tar för användaren att klara ditt spel och ge poäng därefter.
\item Kolla reaktionstiden hos användaren genom att mäta tiden det tar att trycka Enter efter att man fått vänta en slumpmässig tid på att strängen \code{"NU!"} skrivs ut. Om man trycker Enter innan startutskriften ges blir den uppmätta tiden 0 och på så sätt kan ditt program detektera att användaren har tryckt för tidigt. Mät reaktionstiden upprepade gånger och ge poäng efter medelvärdet.
\item Låt användaren på tid så snabbt som möjligt skriva olika ord baklänges.
\item Be användaren skriva en palindrom. Ge poäng efter längd.
\item Träna användaren i multiplikationstabellen på tid.
\item Låt användaren svara på flervalsfrågor om din favoritfilm.
\item Gör det möjligt att ge ett extra argument med en ''fuskkod'' som ger användaren speciella förmågor eller på annat sätt underlättar för användaren under spelets gång.
\end{itemize}

%\subsection{Tips}

%\begin{itemize}
%\item Du kommer åt första argumentet till ditt program genom att indexera i en array som heter \code{args} på plats noll så här: \code{args(0)}.
%\item Du kan kontrollera om det finns minst ett argument med hjälp av det booelska uttrycket \code{args(0).length > 0}.
%\item Metoden \code{toInt} kan göra om en sträng till ett heltal. Du kan vid felaktiga heltal ge ett defaultvärde med \code{scala.util.Try(args(0).toInt).getOrElse(42)}.
%\item Du läser från \textit{standard in} med \code{scala.io.StdIn.readline(prompt)} där \code{prompt} är en sträng som skrivs till \textit{standard out} innan inläsning sker.
%\item Sök upp och studera dokumentationen för klassen \code{scala.util.Random}.
%\item Du kan vänta i t.ex. 3 sekunder med hjälp av Thread.sleep(3000).
%\end{itemize}


\chapter{Datastrukturer}\label{chapter:W04}
\begin{itemize}[nosep]
\item tupler
\item case-klasser
\item case-object
\item enum i java ???
\item Array
\item Map
\item List
\item Vector
\item föränderlighet
\item iterering
\item vektorer i Java vs Scala
\item Complex
\item Rational
\item läsa/skriva textfiler
\item Source.fromFile
\item java.nio.file
\end{itemize}
%!TEX encoding = UTF-8 Unicode
%!TEX root = ../compendium2.tex

\Exercise{\ExeWeekFOUR}\label{exe:W04}
\begin{Goals}
%!TEX encoding = UTF-8 Unicode
%!TEX root = ../compendium2.tex

\item Kunna skapa och använda objekt som moduler.
\item Känna till att funktioner är objekt med en \code{apply}-metod.
\item Förstå begreppen synlighet, \code{private}, \code{import}, namnrymd och skuggning.
\item \TODO{FLER MÅL OM OBJEKT HÄR}

%\item Känna till svansrekursion och att svansrekursiva funktioner kan optimeras till loopar.

\end{Goals}

\begin{Preparations}
\item \StudyTheory{04}
\end{Preparations}

\BasicTasks %%%%%%%%%%%%%%%%

\TODO{ÖVNINGAR OM OBJEKT}

%!TEX encoding = UTF-8 Unicode
%!TEX root = ../compendium2.tex

\Lab{\LabWeekFOUR}
\begin{Goals}
%!TEX encoding = UTF-8 Unicode
%!TEX root = ../labs.tex

\item Kunna förklara hur singelobjekt kan användas som moduler.
\item Kunna förklara hur åtkomst av medlemmar i singelobjekt sker.
\item Kunna skapa kod som reagerar på och förändrar objekts tillstånd.
\item Kunna förklara nyttan med att samla namngivna konstanter i egen modul.
\item Kunna förklara hur import påverkar synlighet av namn.
\item Kunna ge exempel på en situation där man har nytta av namnbyte vid import.
\item Kunna redogöra för skillnaden mellan paket och singelobjekt.
\item Kunna skapa och använda tupler.

\end{Goals}

\begin{Preparations}
\item \DoExercise{\ExeWeekTHREE}{03}
\item \DoExercise{\ExeWeekFOUR}{04}
\end{Preparations}



\subsection{Obligatoriska uppgifter}


\begin{quote}
\textbf{Blockmullvad} (\textit{Talpa laterculus}) är ett fantasidjur i familjen mullvadsdjur.
Den är känd för sitt karaktäristiska kvadratiska utseende.
Den lever mest ensam i sina underjordiska gångar som till skillnad från mullvadens (\emph{Talpa europaea}) har helt raka väggar.
\end{quote}

\begin{figure}
\end{figure}

\Task
Du ska skriva ett Scala-program med en vanlig texteditor och kompilera ditt program med kommandot \texttt{scalac} och sedan köra programmet med kommandot \texttt{scala}.

\Subtask
Öppna en texteditor, till exempel gedit eller Atom (se appendix~\ref{appendix:edit} för hjälp).
Skapa en ny fil med namnet \texttt{Mole.scala} och spara den i en ny katalog i din hemkatalog, till exempel \texttt{\textasciitilde/pgk/mole/Mole.scala}, där \texttt{\textasciitilde} är din hemkatalog.

\Subtask
Öppna ett terminalfönster (se appendix~\ref{appendix:terminal} för hjälp).
Navigera till din nya katalog med \texttt{cd}-kommandot \Eng{change directory} och kontrollera med \texttt{ls}-kommandot \Eng{list} att din nya fil finns där.
\begin{REPLnonum}
> cd ~/pgk/mole
> ls
\end{REPLnonum}
Om allt går bra ska \texttt{ls}-kommandot skriva ut \texttt{Mole.scala}.

\Subtask
Gå tillbaka till din texteditor och skriv in ett objekt med namnet \code{Mole} i din fil.
Lägg till en \code{main}-funktion i objektet som skriver ut texten \emph{Keep on digging!} med hjälp av funktionen \code{println}.
Behöver du hjälp kan du gå tillbaka till övningarna i kapitel~\ref{exe:W03}.

\Subtask
Kör kommandot \texttt{scalac Mole.scala} i terminalfönstret för att kompilera ditt program.
Om kompilatorn rapporterar några fel rättar du till det i din texteditor kompilerar igen.
Kontrollera sedan med \texttt{ls}-kommandot att några filer som slutar på \texttt{class} har skapats.

\Subtask
Kör kommandot \texttt{scala Mole} för att köra ditt program.
Om att går bra ska texten du angivit skrivas ut i terminalfönstret.


\Task
Nu har du skrivit ett Scala-program som skriver ut en uppmaning till en mullvad att fortsätta gräva.
Det programmet är inte så användbart, eftersom mullvadar inte kan inte läsa.
Nästa steg är att skriva ett grafiskt program, snarare än ett textbaserat.

Funktionen \code{println} som anropas i \code{main}-funktionen ingår i Scalas standardbibliotek.
Ett programbibliotek innehåller kod eller kompilerade programsnuttar som kan användas av andra program, och för de flesta programspråk ingår ett standardbibliotek som alla program kan nyttja.
Till grafiken i denna uppgift ska du använda ett bibliotek som kallas \emph{cslib} och som kommer att användas även i senare labbar.

\Subtask

Ladda ner \texttt{cslib.jar} via länken \url{http://cs.lth.se/pgk/cslib} och lägg jar-filen i samma katalog som ditt Scala-program.
En jar-fil används för att paketera färdigkompilerade program, kod, dokumentation, resursfiler, etc, och är komprimerad på samma sätt som en zip-fil.

\Subtask
Byt ut \code{main}-funktionens kropp mot följande block:
\begin{Code}
{
	val w = new cslib.window.SimpleWindow(300, 500, "Digging")
	w.moveTo(10, 10)
	w.lineTo(10, 20)
	w.lineTo(20, 20)
	w.lineTo(20, 10)
	w.lineTo(10, 10)
}
\end{Code}
Den första raden skapar ett nytt \code{SimpleWindow} som ritar upp ett fönster som är 300 bildpunkter brett och 500 bildpunkter högt med titeln \emph{Digging}.
\code{SimpleWindow} har en \emph{penna} som kan flyttas runt och rita linjer.
Anropet \code{w.moveTo(10, 10)} flyttar pennan för fönstret \code{w} till position $(10,10)$ utan att rita något, och anropet \code{w.lineTo(10, 20)} ritar en linje därifrån till position $(10, 20)$.

\Subtask
Nu ska du kompilera ditt program, men eftersom \code{SimpleWindow} inte finns i Scalas standardbibliotek utan i \texttt{cslib.jar} behöver du visa kompilatorn var den ska leta.
Det gör du genom att ange en \emph{classpath}, dvs. en sökväg till \texttt{class}-filer, när du kompilerar.
Använd flaggan \texttt{-cp cslib.jar} för att ange \texttt{cslib.jar} som classpath och kompilera ditt Scala-program igen:
\begin{REPLnonum}
> scalac -cp cslib.jar Mole.scala
\end{REPLnonum}

\Subtask
Nu ska du köra ditt program, och då behöver du också ange var \texttt{class}-filerna ligger.
Du ska ange den katalog där \texttt{class}-filerna för \code{Mole} ligger, som du just kompilerat, men du ska också ange \texttt{cslib.jar}, och det gör du med en kolon-separerad lista\footnote{Kolon används i Linux och macOS, medan Windows använder semikolon.}, till exempel \code{"sökväg1:sökväg2:sökväg3"}.
Katalogen du står i, där dina \texttt{class}-filer ligger, kan anges med en punkt (\texttt{.}).
Kör programmet med följande kommando (om Windows använd semikolon):
\begin{REPLnonum}
> scala -cp ".:cslib.jar" Mole
\end{REPLnonum}
Du ska nu få upp ett fönster med en liten kvadrat utritad i övre vänstra hörnet.


\Task
Hela ditt program är för tillfället samlat i en och samma funktion, vilket fungerar bra för väldigt små program.
Nu ska vi strukturera programmet så det blir lättare att återanvända samma kodsnuttar.

\Subtask
Lägg till ett objekt med namnet \code{Graphics} i \texttt{Mole.scala} och flytta dit deklarationen av fönstret \code{w}.
Skapa en ny funktion med namnet \code{square} i det nya objektet och flytta dit koden som ritar kvadraten.
Anropa \code{square} i din \code{main}-funktion.
Filen \texttt{Mole.scala} ska se ut såhär (förutom \code{???}):
\begin{Code}
object Graphics {
	val w = new cslib.window.SimpleWindow(300, 500, "Digging")
	def square(): Unit = ???
}
object Mole {
	def main(args: Array[String]): Unit = {
		Graphics.square()
	}
}
\end{Code}
Observera att du inte kan anropa \code{square} direkt i funktionen \code{main}, utan måste ange att det är \code{square}-funktionen inuti \code{Graphics} du vill anropa.

\Subtask
Kompilera \texttt{Mole.scala} med \texttt{scalac}.
Glöm inte att ange korrekt classpath.
(\emph{Tips:} Du kan trycka uppåtpil för att komma till tidigare kommandon i terminalen.)
Kontrollera med \texttt{ls} att det nu också finns \texttt{class}-filer för \code{Graphics}-objektet.

\Subtask
Kör programmet \code{Mole} med \texttt{scala}.
Glöm inte att ange korrekt classpath.
Om allt fungerar ska programmet göra samma sak som innan.

\Task
Nu har du gjort ett grafiskt program, men ännu syns ingen mullvad.
Det är dags att ta reda på hur koordinatsystemet fungerar i denna grafiska miljö, så vi kan få mullvaden att hitta rätt.

\Subtask
Ändra i \code{Graphics.square} så att kvadraten ritas upp i \emph{övre högra} hörnet istället.
Prova dig fram för att ta reda på hur koordinatsystemet fungerar genom att ändra i koden, kompilera och köra programmet tills du får rätt på det.

\Subtask\Checkpoint
Visa kvadraten för din labbhandledare och förklara vad de två parametrarna gör genom att peka ut ungefär var positionerna $(0,0)$, $(300, 0)$, $(0, 300)$ och $(300, 300)$ ligger.

\Subtask
Ta bort anropet till funktionen \code{square} när du har visat den för din labbhandledare.

\Task
Nu ska du skapa ett nytt koordinatsystem för \code{Graphics} som har \emph{stora} bildpunkter.
Vi kallar \code{Graphics} stora bildpunkter för \emph{block} för att lättare skilja dem från \code{SimpleWindow}s bildpunkter.
Om blockstorleken är $b$, så ligger koordinaten $(x, y)$ i \code{Graphics} på koordinaten $(bx, by)$ i \code{SimpleWindow}.

\Subtask
Lägg till följande deklarationer överst i objektet \code{Graphics}.
\begin{Code}
val width = 30
val height = 50
val blockSize = 10
\end{Code}
Ändra bredden på ditt \code{SimpleWindow} till \code{width * blockSize} och ändra höjden till \code{height * blockSize}.

\Subtask
Skapa en ny funktion i \code{Graphics} med namnet \code{block} och två parametrar \code{x} och \code{y} av typen \code{Int} och returtypen \code{Unit}.
Metodens \emph{kropp} ska se ut såhär:
\begin{Code}
{
    val left = x * blockSize
    val right = left + blockSize - 1
    val top = y * blockSize
    val bottom = top + blockSize - 1

    for (row <- top to bottom) {
      w.moveTo(left, row)
      w.lineTo(right, row)
    }
}
\end{Code}

\Subtask\Pen
Metoden \code{block} ritar ett antal linjer.
Hur många linjer ritas ut?
I vilken ordning ritas linjerna?

\Subtask
Anropa funktionen \code{Graphics.block} några gånger i \code{Mole.main} så att några block ritas upp i fönstret när programmet körs.
Kompilera och kör ditt program.


\Task
Det finns många sätt att beskriva färger.
I naturligt språk har vi olika namn på färgerna, till exempel \emph{vitt}, \emph{rosa} och \emph{magenta}.
I datorn är det vanligt att beskriva färgerna som en blandning av \emph{rött}, \emph{grönt} och \emph{blått} i det så kallade RGB-systemet.
\code{SimpleWindow} använder typen \code{java.awt.Color} för att beskriva färger och \code{java.awt.Color} bygger på RGB.
Det finns några fördefinierade färger i \code{java.awt.Color}, till exempel \code{java.awt.Color.black} för svart och \code{java.awt.Color.green} för grönt.
Andra färger kan skapas genom att ange mängden rött, grönt och blått.

\Subtask
Skapa ett nytt objekt i \texttt{Mole.scala} med namnet \code{Colors} och lägg in följande definitioner:
\begin{Code}
val mole = new java.awt.Color(51, 51, 0)
val soil = new java.awt.Color(153, 102, 51)
val tunnel = new java.awt.Color(204, 153, 102)
\end{Code}
% val sky = new java.awt.Color(51, 51, 204)
% val grass = new java.awt.Color(51, 204, 51)
Den tre parametrarna till \code{new java.awt.Color(r, g, b)} anger hur mycket \emph{rött}, \emph{grönt} respektive \emph{blått} som färgen ska innehålla, och mängderna ska vara i intervallet 0--255.
Färgen $(153, 102, 51)$ innebär ganska mycket rött, lite mindre grönt och ännu mindre blått och det upplevs som brunt.
Objektet \code{Colors} är en färgpallett, men vi har inte ritat något med färg ännu.
Kompilera och kör ditt program ändå, för att se så programmet fungerar likadant som sist.

\Subtask
Lägg till en parameter till \code{Graphics.block} sist i parameterlistan med namnet \code{color} och typen \code{java.awt.Color}.
Låt \emph{default-argumentet} för den nya parametern vara \code{java.awt.Color.black}.
(Kommer du inte ihåg hur man gör default-argument kan du titta på övningarna i kapitel~\ref{exe:W03}.)
För att ändra färgen på blocket kan du byta linjefärg innan du ritar.
Lägg till följande rad i början på \code{Graphics.block}:
\begin{Code}
w.setLineColor(color)
\end{Code}
Kompilera och kör ditt program igen för att se om det fortfarande fungerar.

\Subtask\Pen
Funktionen \code{Graphics.block} har tre parametrar, men den anropas bara med två parametrar i \code{Mole.main}.
Varför är det tillåtet?
Vilket värde har den tredje parametern om ingen anges?

\Subtask
Ändra i \code{Mole.main} och lägg till en av definitionerna från objektet \code{Colors} som tredje parameter till \code{Graphics.block}.
Kompilera och kör ditt program och upplev världen i färg.

\Task
I programmet används många långa namn med punkter, som till exempel \code{java.awt.Color} och \code{Graphics.block}.
Dessa punkt-separerade namn kallas \emph{kvalificerade} namn.
För att slippa skriva dessa långa namn hela tiden kan man \emph{importera} en definition och sen använda bara den sista delen av namnet.

\Subtask
Importera namnet \code{java.awt.Color} i objektet \code{Colors}. Ändra sen alla \code{new java.awt.Color(...)} i objektet till \code{new Color(...)}.
(Har du glömt hur man importerar ett namn kan du gå tillbaka till övningarna i kapitel~\ref{exe:W02}.)

\Subtask\Pen
I vilka av objekten \code{Mole}, \code{Colors} och \code{Graphics} kan du använda det korta respektive det kvalificerade namnet av \code{java.awt.Color}?

\Subtask
Importera namnet \code{java.awt.Color} så att det korta namnet \code{Color} kan användas i objekten \code{Colors} och \code{Graphics} men inte i \code{Mole}.
Byt sedan ut de långa namnen mot de korta i \code{Graphics}.

\Task
Nu ska du skriva en funktion för att rita en rektangel. Rektangeln ska ritas med hjälp av funktionen \code{block}.
Sen ska du rita upp mullvadens underjordiska värld med hjälp av denna funktion.

\Subtask
Lägg till en funktion i objektet \code{Graphics} med namnet \code{rectangle} som tar fem parametrar \code{x}, \code{y}, \code{width} och \code{height} av typen \code{Int} och \code{color} av typen \code{Color}.
Parametrarna \code{x} och \code{y} anger \code{Graphics}-koordinaten för rektangelns övre vänstra hörn och \code{width} och \code{height} anger bredden respektive höjden.
Använd följande \code{for}-satser för att rita ut rektangeln.
\begin{Code}
for (yy <- y until (y + height)) {
	for (xx <- x until (x + width)) {
		block(xx, yy, color)
	}
}
\end{Code}

\Subtask\Pen
I vilken ordning ritas blocken ut?

% \Subtask\Pen (Fråga något om skuggning gällande \code{width} och \code{height}.)

\Subtask
Skriv en funktion i objektet \code{Mole} med namnet \code{drawWorld} som ritar ut mullvadens värld, det vill säga en massa jord där den kan gräva sina tunnlar.
\code{Mole.drawWorld} ska inte ha några parametrar och returtypen ska vara \code{Unit} och den ska anropa \code{Graphics.rectangle} för att rita en rektangel med färgen \code{Colors.soil} som precis täcker fönstret.
Eftersom funktionen har många parametrar som lätt kan blandas ihop ska du använda namngivna argument vid anropet.
(Om du har glömt hur man använder namngivna argument kan du titta på övningarna i kapitel~\ref{exe:W03}.)

\Subtask
Anropa \code{Mole.drawWorld} i \code{Mole.main} och testa så att det fungerar genom att kompilera och köra.

\Task
I \code{SimpleWindow} finns funktioner för att känna av tangenttryckningar och musklick.
Du ska använda de funktionerna för att styra en liten blockmullvad.

\Subtask
Importera \code{cslib.window.SimpleWindow} i objektet \code{Graphics} och lägg till följande funktion:
\begin{Code}
def waitForKey(): Char = {
	do {
		w.waitForEvent()
	} while (w.getEventType() != SimpleWindow.KEY_EVENT)
	w.getKey()
}
\end{Code}
Det finns olika sorters händelser som ett \code{SimpleWindow} kan reagera på, till exempel tangenttryckningar och musklick.
Funktionen som du precis lagt in väntar på en händelse i ditt \code{SimpleWindow} (\code{w.waitForEvent}) ända tills det kommer en tangenttryckning (\code{KEY_EVENT}).
När det kommit en tangenttryckning anropas \code{w.getKey} för att ta reda på vilken bokstav eller vilket tecken det blev, och det resultatet blir också resultatet av \code{waitForKey}, eftersom det ligger sist i det yttre \texttt{\{\}}-blocket.

\Subtask
Lägg till en funktion i objektet \code{Mole} med namnet \code{dig}, utan parametrar och med returtypen \code{Unit}.
Funktionens kropp ska se ut såhär (fast utan \code{???}):
\begin{Code}
{
  var x = Graphics.width / 2
  var y = Graphics.height / 2
  while (true) {
    Graphics.block(x, y, Colors.mole)
    val key = Graphics.waitForKey()
    if (key == 'w') ???
    else if (key == 'a') ???
    else if (key == 's') ???
    else if (key == 'd') ???
  }
}
\end{Code}
Fyll i alla \code{???} så att \code{'w'} styr mullvaden ett steg uppåt, \code{'a'} ett steg åt vänster, \code{'s'} ett steg nedåt och \code{'d'} ett steg åt höger.

\Subtask
Ändra \code{Mole.main} så att den bara innehåller två anrop: ett till \code{drawWorld} och ett till \code{dig}.
Kompilera och kör ditt program för att se om programmet reagerar på w, a, s och d.

\Subtask
Om programmet fungerar kommer det bli många mullvadar som tillsammans bildar en lång mask, och det är ju lite underligt.
Lägg till ett anrop i \code{Mole.dig} som ritar ut en bit tunnel på position $(x, y)$ efter anropet till \code{Graphics.waitForKey} men innan \code{if}-satserna.
Kompilera och kör ditt program för att gräva tunnlar med din blockmullvad.

\subsection{Frivilliga extrauppgifter}

\Task
Mullvaden kan för tillfället gräva sig utanför fönstret.
Lägg till några \code{if}-satser i början av \code{while}-satsen som upptäcker om \code{x} eller \code{y} ligger utanför fönstrets kant och flyttar i så fall tillbaka mullvaden precis innanför kanten.

\Task
Mullvadar är inte så intresserade av livet ovanför jord, men det kan vara trevligt att se hur långt ner mullvaden grävt sig.
Lägg till en himmelsfärg och en gräsfärg i objektet \code{Colors} och rita ut himmel och gräs i \code{Mole.drawWorld}.
Justera också det du gjorde i föregående uppgift, så mullvaden håller sig under jord.
(\emph{Tips:} Den andra parametern till \code{Color} reglerar mängden grönt och den tredje parametern reglerar mängden blått.)

\Task
Ändra så att mullvaden kan springa uppe på gräset också, men se till så att ingen tunnel ritas ut där.


\foreach \n in {5,...,9}{%
%  \input{modules/w0\n-chapter.tex}
  \input{generated/w0\n-chaphead-generated.tex}
  \input{modules/w0\n-exercise.tex}
  \input{modules/w0\n-lab.tex}
}
\foreach \n in {10,...,12}{%
%  \input{modules/w\n-chapter.tex}
  \input{generated/w\n-chaphead-generated.tex}
  \input{modules/w\n-exercise.tex}
  \input{modules/w\n-lab.tex}
}
%
%!TEX root = ../compendium.tex

%!TEX encoding = UTF-8 Unicode
\chapter{Design}\label{chapter:W13}
Koncept du ska lära dig denna vecka:
\begin{multicols}{2}\begin{itemize}[nosep,label={$\square$},leftmargin=*]
\item\end{itemize}\end{multicols}

    
%!TEX encoding = UTF-8 Unicode
\chapter{Design}\label{chapter:W13}
Koncept du ska lära dig denna vecka:
\begin{multicols}{2}\begin{itemize}[nosep,label={$\square$},leftmargin=*]
\item\end{itemize}\end{multicols}

\input{modules/w13-exercise.tex}
\input{modules/w13-assignment-life.tex}
%!TEX encoding = UTF-8 Unicode
%!TEX root = ../compendium.tex

\Assignment{bank}

\subsection{Obligatoriska uppgifter}

\Task En uppgift.

\Subtask En underuppgift.

\Subtask En underuppgift.

\subsection{Frivilliga extrauppgifter}

\Task En uppgift.

\Subtask En underuppgift.

\Subtask En underuppgift.


%!TEX encoding = UTF-8 Unicode
%!TEX root = ../compendium.tex

\Assignment{tictactoe}
I detta projekt ska du implementera din egen version av spelet tic-tac-toe (eller som vi på svenska kallar det, tre i rad)! Du kommer börja med att implementera en version där du kan spela mot en kursare och sen gå vidare till att implementera en datorspelare som lägger sin pjäs slumpmässigt och till slut en som inte kan förlora!

\subsection{Regler}
%Om du känner dig säker på hur reglerna i tic-tac-toe funkar kan du skippa detta. 
\begin{itemize}
	\item Spelplanen består av ett rutnät av storlek 3x3.
	\item Det finns två spelare: \texttt{x} och \texttt{o}.
	\item Spelarna placerar ut en pjäs var i växlande ordning där \texttt{x} börjar.
	\item Spelet tar slut om en spelare har fått antingen en rad, diagonal eller kolumn ifylld av sin spelpjäs eller om spelplanen är fylld.
\end{itemize}
\textit{Notera att pjäserna INTE får flyttas när de väl ligger på spelplanen.}

\subsection{Teori}
Representationen är vald till en endimensionell vektor av typen Int av storlek 9 där elementen 0 till och med 2  representerar den första raden, [3, 5] andra raden och [6, 8] den tredje. Anledningen till detta är att vi vill ha en representation så att spelaren kan svara vilket drag den vill göra med ett heltal.
Varje element i vektorn ska kunna representera en tom plats, en plats allokerad av \texttt{x} och en plats allokerad av \texttt{o}. Detta innebär att en vektor av typen Boolean inte räcker till. Istället väjs den (kanske lite minnesöverflödiga) typen Int. Hint: en bra representation är 0 för tom plats, 1 för \texttt{x} och -1 för \texttt{o}. 
 
\subsection{Obligatoriska uppgifter}

\Task Implementera ett fungerande spel genom att utöka kodskeletten i klasserna Player, HumanPlayer och Game.

\Subtask Implementera funktionen gameWon i klassen Player som testar huruvida spelaren \code{who} vunnit spelet.

\Subtask Implementera en HumanPlayer.

\Subtask Implementera första version av Game.

\Task Randomized player

\Subtask Implementera en spelare som väljer ett slumpmässigt giltigt drag.

\Subtask Ändra Game så att användaren tillåts stänga av ritfunktionen och tillåts spela många spel.

\Subtask Vad är sannolikheterna för att \texttt{x} vinner, \texttt{o} vinner och att det blir oavgjort om två randomized players spelar mot varandra?

Hamnar man i närheten av dessa resultat tror vi på er randomized player.
\begin{itemize}
	\item P(\texttt{x} vinner) = 0.586
	\item P(\texttt{o} vinner) = 0.288
	\item P(lika) = 0.126
\end{itemize}


\Subtask Varför är det större sannolikhet för \texttt{x} att vinna än \texttt{o}?

\Task Optimal Player

\Subtask Läs igenom eval-funktionen och Appendix om max-min-evaluering.

\Subtask Implementera Optimal Players move-funktion.

\Subtask testa att spela mot din Optimal player med en human player, kan du spela lika? Kan du vinna?

\Subtask Vad händer om du sätter en random player mot Optimal player? Blir det någonsin oavgjort, hur ofta?

\subsection{Frivilliga extrauppgifter}

\Task Hashning.

\Subtask En underuppgift.

\Subtask En underuppgift.
%!TEX encoding = UTF-8 Unicode
%!TEX root = ../compendium.tex

\Assignment{imageprocessing}

\subsection{Obligatoriska uppgifter}

\Task En uppgift.

\Subtask En underuppgift.

\Subtask En underuppgift.

\subsection{Frivilliga extrauppgifter}

\Task En uppgift.

\Subtask En underuppgift.

\Subtask En underuppgift.
%%!TEX encoding = UTF-8 Unicode

%!TEX root = ../compendium.tex

%!TEX encoding = UTF-8 Unicode
\chapter{Muntlig examen}\label{chapter:W14}


%!TEX encoding = UTF-8 Unicode
%!TEX root = ../lect-week14.tex

%%%

\Subsection{Tentatips}
\begin{Slide}{Före tentan:}\SlideFontSmall
\begin{enumerate}
\item Repetera övningar och labbar i kompendiet. 
\item Läs igenom föreläsningsanteckningar.
\item Studera \Emph{snabbref} \Alert{mycket noga} så att du vet vad som är givet och var det står, så att du kan hitta det du behöver snabbt.
\item Skapa och \Emph{memorera} en personlig \Emph{checklista} med programmeringsfel du brukar göra, som även inkluderar småfel, så som glömda parenteser och semikolon, och annat som en kompilator/IDE normalt hittar.
\item Tänk igenom hur du ska disponera dina 5 timmar på tentan.
\item Gör den minst en extenta som om det vore \Alert{skarpt läge}: 
\begin{enumerate}\SlideFontTiny
\item Avsätt 5 ostörda timmar (stäng av telefon, dator etc).
\item Inga hjälpmedel. Bara snabbref.
\item Förbered dryck och tilltugg.
\end{enumerate}
\end{enumerate}
\end{Slide}

\begin{Slide}{På tentan:} \SlideFontTiny
\begin{enumerate}
\item Läs igenom \Alert{hela} tentan först. \\ \Emph{Varför?} Förstå helheten. Delarna hänger ihop.
\item Notera och begrunda specifika begrepp och definitioner. \\ \Emph{Varför?} Begreppen är avgörande för förståelsen av uppgiften.
\item Notera förenklingar, antaganden och specialfall. \\ \Emph{Varför?} Uppgiften blir mkt enklare om du inte behöver hantera dessa.
\item \Alert{Fråga} tentamensansvarig om du inte förstår uppgiften -- speciellt om det finns misstänkta felaktigheter eller förmodat oavsiktliga oklarheter. \\ \Emph{Varför?} Det är inte lätt att konstruera en ''perfekt'' tenta. \\ Du får fråga vad du vill, men det är inte säkert du får svar :)
\item Läs specifikationskommentarerna och metodsignaturerna i alla givna klass-specifikationer \Alert{mycket noga}. \\ \Emph{Varför?} Det är ett vanligt misstag att förbise de ledtrådar som ges där.
\item Återskapa din memorerade personliga checklista för vanliga fel som du brukar göra och avsätt tid till att gå igenom den på tentan. Varje fix plockar poäng!
\item Lämna in ett försök även om du vet att lösningen inte är fullständig. Det gäller att ''plocka poäng'' på så mycket som möjligt. En dålig lösning kan ändå ge poäng.

\item Om du har svårigheter kan det bli kamp mot klockan. Försök hålla huvudet kallt och prioritera utifrån var du kan plocka flest poäng. Ge inte upp! Ta en kort äta-dricka-paus för att få mer energi!

\end{enumerate}
\end{Slide}

\ifkompendium\else

\begin{Slide}{Planeringstips}\SlideFontTiny
Exempel på saker som du kan lägga in tid för i din julpluggkalender:
\begin{enumerate}
\item Välja ut övningar att repetera
\item Repetera övning X, Y, Z, ... Både läsa och skriva kod. Fundera på typ och värde.
\item Välja ut labbar att repetera
\item Repetera labb X, Y, Z, ... Lär dig ''trick'' och ''mönster''.
\item Träna på att skriva program med papper och penna
\item Göra checklista med vanliga fel
\item Läsa igenom extentor i Java
\item Välja ut minst en Java-extenta att göra som i skarpt läge i Scala
\item Gör Java-extentor X, Y, Z, ... implementera (delar) i Scala
\item Gör utvalda delar av extenta X, Y, Z, ... i Java
\end{enumerate}
\end{Slide}

\Subsection{Avgränsning}

\begin{Slide}{Tentans struktur}
\begin{itemize}
\item Del A 20\%:\\\Emph{Läsa uttryck} där du ska \Alert{ange typ och värde}
\begin{itemize}\SlideFontTiny
\item Du kommer att behöva skriva ner delsteg och variablers värden (minnet)
\item Testar förståelse av variabler, uttryck, samlingar, algoritmer, arv, etc.
\item Uppdaterad (mildare) regel om ''rättningströskel'': \\
Ur senaste compendium.pdf kapitel 0.8: \textit{Om du på del A erhåller färre poäng än vad som krävs för att nå upp till en bestämd ''rättningströskel'', kan din tentamen komma att underkännas utan att del B bedöms.}
\item Liknar kompendiets övningar; rimlig att lösa och dubbelkolla på 1h
\end{itemize}


\item Del B 80\%:\\\Emph{Skriva kod} som uppfyller \Alert{krav och designspecifikation}
\begin{itemize}\SlideFontTiny
\item Testar att du själv kan skapa kod med delar som samverkar
\item Testar förmåga att gå från indata-utdata till algoritm \\
 givet: ledtrådar, design, ev. skiss på lösning, ev. pseudokod etc.
\item Liknar kompendiets labbar; rimlig att lösa och dubbelkolla på 4h 
\end{itemize}

\end{itemize}
\end{Slide}



\begin{Slide}{Vad kommer på tentan? (1 av 3)}\SlideFontTiny
\hspace{-1em}\begin{minipage}{1.0\textwidth}
Allmänt: 
\begin{itemize}\SlideFontTiny
\item Begrepp som är ''fördjupning'' krävs ej på tentan (men ökar förståelse)
\item Ok om du väljer en enklare lösning med basala begrepp som fungerar bra, \\i stället för en kortare/elegantare/mer avancerad lösning
\item Dessa moduler ingår ej på tentan: ''Trådar, webb'', ''Design, api''
\end{itemize}

\vspace{1em}\begin{tabular}{l | l | l}
\textbf{Modul} & \textit{Ingår t.ex.}& \textit{Avgränsning} (ej krav; ok anv. om lämpl.)\\\hline
Introduktion & uttryck, aritmetik, slumptal, & kan ha nytta av deMorgan men ej krav\\
             & strängar, typer, Unit   & skriva egna \code|s"$x"| (men kunna läsa)  \\
             & skillnad mellan heltal \& flyttal & Float, Byte, Short\\
             & variabler, for, while, if & hex-literaler, backticks\\ 
\hline
Kodstrukturer & iterering, SWAP, SUM, MIN/MAX & import, paketnamn\\             
              & loopar, Range, sats vs uttryck & ok att välja vilken loop du tycker passar\\
              & namn, synlighet, skuggning & scaladoc, javadoc, jar \\        
\hline
Funtioner,    & definiera, anropa, parameter& skapa egen kontrollstruktur\\
objekt        & returtyp, namnarop, defaultarg & stegad funktion, rekursion\\        
              & punktnotation, objekt vs static & lazy val\\        
              & map/foreach med egen funktion & \\
              & anonyma funktioner (lambda)  & \\                              

\end{tabular}
\end{minipage}
\end{Slide}


\begin{Slide}{Vad kommer på tentan? (2 av 3)}\SlideFontTiny
\hspace{-2em}\begin{minipage}{1.0\textwidth}
\begin{tabular}{l | l | l}
\textbf{Modul} & \textit{Ingår t.ex.}& \textit{Avgränsning} (ej krav; ok anv. om lämpl.)\\\hline
Datastrukt. & attribut, medlem, metod, klass & isInstanceOf (anv. match istället) \\
            & tupler, Vector, Set, Map & List (oftast Vector istället)\\
            & Source.fromFile          & java.nio.file \\
\hline
Sekvensalg. &  skapa ny samling från befintlig &  \\      
            &  registrering, Scanner, ArrayBuffer & StringBuilder\\
            &  uppdatera Array, ArrayBuffer, Vector & \\
            &  slumptalsfrö, scala.util.Random  &  \\
\hline

Klasser     &  new, this, synlighet  & null \\
            &  inkapsling, accessregler, private  & private[this] \\
            &  klassparameter, fabriksmetod  & \\
            &  class vs case class    & \\
            &  referenslikhet vs innehållslikhet    & \\
            &  föränderlig vs oföränderlig klass & \\
\hline
Arv         &  bastyp, subtyp, trait, extends  & \\
            &  överskuggning,                  & inmixning, \\
            &  Any, AnyVal, AnyRef, Object     & Null, Nothing\\
            &  accessregler vid arv, protected & final\\
            &  abstract class, case object     & \\
            
\end{tabular}
\end{minipage}
\end{Slide}


\begin{Slide}{Vad kommer på tentan? (3 av 3)}\SlideFontTiny
\hspace{-2em}\begin{minipage}{1.0\textwidth}
\begin{tabular}{l | l | l}
\textbf{Modul} & \textit{Ingår t.ex.}& \textit{Avgränsning} (ej krav; ok anv. om lämpl.)\\\hline

Mönster     & match, Option, Try & try catch, unapply\\
            & flatten, sealed            & flatMap, partiella funktioner\\
            & enkel equals utan arv     & hashcode, fullständig equals   \\ 
            & wildcard-mönster  & variabelbindn. i mönster, sekvensmönster\\
\hline

Matriser,     & indexering i nästlade strukturer & \\
typparametrar & nästlad for-sats  & \\ 
              & matriser i Java med array  & \\
              & använda generiska strukturer & skapa generiska strukturer\\ 
\hline

Sök, sortera & linjärsökning, binärsökning & algoritmisk komplexitet\\
            & compareTo, strängjämförelse & Ordering, Ordered\\
            & insättningssortering & räcker kunna en valfri sortering \\
\hline


Scala/Java & översätta enkel Java/Scala & try catch i Java \\
           & implemenetera Java-klass     &  arv i Java med super vid konstr.\\
           & grundläggande syntaxskillnader & \\
           & ArrayList vs ArrayBuffer & java.util.\{List, Set\}\\
           & Autoboxing vid ArrayList<Integer> & \\
\multicolumn{3}{c}{OBS! Java-övningar finns även här och där i andra moduler}\\
\hline           
     
\end{tabular}
\end{minipage}
\end{Slide}


\Subsection{Tips vid val av lösningar}


\begin{Slide}{Tips om val av klass/trait}\SlideFontSmall
Ofta ger tentan en specifik design, men du kan ha stor nytta av egna abstraktioner, speciellt \Emph{lokala funktioner} för att göra enklare dellösningar!

\pause\vspace{1em}Om du skulle behöva samla både attribut och metoder utöver givan specifikationer:
Singelobjekt, case-klass, klass, trait eller abstrakt klass?
\begin{itemize}\SlideFontTiny
\item Använd \code{object} om du behöver samla metoder (och variabler) i en modul som bara finns i en upplaga. Du får lokal namnrymd och punktnotation på köpet.
\item Använd en \code{case class} om du har \Emph{oföränderlig data}. Du får då även innehållslikhet, möjlighet till mönstermatchning, etc. på köpet! 
\item Behöver du \Alert{föränderligt tillstånd} använd en vanlig \code{class}.\\ Det normala är att tillståndet (alla attribut) är \code{private} eller \code{protected} och att all uppdatering och avläsning av tillståndet sker indirekt genom metoder (getters/setters/...). 
\item Behöver du en abstrakt bastyp utan konstruktorparametrar använd en \code{trait}. \\(Du får inmixningsmöjlighet med \code{with} på köpet. Inmixning kommer ej på tenta.)
\item Behöver du en abstrakt bastyp med konstruktorparametrar använd en \code{abstract class}. (Går dock ej att använda vid inmixning med \code{with}.)
\end{itemize}
\end{Slide}


\begin{Slide}{Tips om hur man läser en specifikation}\SlideFontSmall
När du läser en specifikation av en klass, en trait, eller ett singelobjekt:
\begin{itemize}
\item Tänk igenom vilket ansvar olika delar av koden har
\item Vad håller klassen reda på? \\$\rightarrow$ Ledtrådar till attribut
\item Vad ska klassen göra/räkna ut? \\$\rightarrow$ Ledtrådar till metoder och deras algoritm
\item Vilka andra klasser har nytta av denna metod? \\$\rightarrow$ Ledtrådar till hur klasserna samverkar för att lösa uppgiften
\end{itemize}
Rita gärna en bild med ett specifikt exempel på vilken data som olika instanser håller reda på och fundera på hur data skapas/beräknas/förändras
\end{Slide}


\begin{Slide}{Tips om val av samling}\SlideFontSmall

Generellt: Det är ofta enklare med oföränderliga samlingar med oföränderliga element och skapa nya samlingar vid förändring. Men ibland blir det enklare om man har föränderliga samlingar.

\begin{itemize}
\item Behöver du hantera värden \code{x} av t.ex. typen String med \Emph{heltalsindex}?
\begin{itemize}\SlideFontTiny
\item Om du klarar dig utan förändring av innehållet:\\ \code{ val xs: Vector[String]}
\item Om du behöver ändra innehåll men \Alert{inte} antal element: \\ \code{ val xs: Array[String]} 
\item Om du behöver ändra innehåll \Alert{och} antal element: 
\\ \code{ var xs: Vector[String] } (se metoden \code{patch}) eller \\ \code{ val xs: ArrayBuffer[String]} (har metoden \code{insert})
\end{itemize}

\item Behöver du hantera värden \code{x} som ska vara unika?
\begin{itemize}\SlideFontTiny
\item Oföränderlig: \code{  val xs: Set[String] }
\item Förändringsbar: \code{val xs: scala.collection.mutable.Set[String]}
\end{itemize}

\item Behöver du leta upp värden \code{x:Int} utifrån en nyckel av t.ex. String?
\begin{itemize}\SlideFontTiny
\item Oföränderlig: \code{   val xs: Map[String, Int] }
\item Förändringsbar: \code{val xs: scala.collection.mutable.Map[String, Int]}
\end{itemize}


\end{itemize}
\end{Slide}

\begin{Slide}{Tillåtna uppdateringar i din QuickRef}
Du får med egen penna göra dessa fixar i din QuickRef:
\begin{itemize}
\item Grundtypernas implementation, sid 4: 
\begin{itemize}

\item omfång för Int ska ha exponent 31 (inte 15), 
\item omfång för Long ska ha exponent 63 (inte 15).
\end{itemize}

\item Saknade samlingsmetoder: 
\begin{itemize}
\item Under rubriken "Methods in trait Map[K, V]" saknas metoderna keySet och mapValues. 
\item Saknade metoderna för mutable.ArrayBuffer[T]: \\ \code{update} \code{insert} \code{remove} \code{append} \code{prepend}, etc. \\ lägg till beskrivning på lediga platsen på sista sidan \\ 
(se vidare commit \href{https://github.com/lunduniversity/introprog/commit/a5e29d000062}{a5e29d000062a} i kursrepot)
\end{itemize}
\end{itemize}
\end{Slide}


\begin{Slide}{ArrayBuffer}
Viktigast att känna till: update, insert, remove, append
{\SlideFontTiny

\vspace{2.5em}\begin{tabular}{@{}p{4.2cm}  p{6.5cm}}
\texttt{xs(i) = x \newline xs.update(i, x)} & Replace element at index i with x. \newline Return type Unit.\\   \cline{1-2}

\texttt{xs.insert(i, x)\newline xs.remove(i)} & Insert x at index \texttt{i}. Remove element at i. \newline Return type Unit.\\   \cline{1-2}

\texttt{xs.append(x)~~~xs~+=~x} & Insert x at end.  Return type Unit.\\   \cline{1-2}

\texttt{xs.prepend(x)~~x~+=:~xs} & Insert x in front.  Return type Unit.\\   \cline{1-2}

\texttt{xs -= x} & Remove first occurance of x (if exists). \newline Returns xs itself. \\\cline{1-2}

\texttt{xs ++= ys} & Appends all elements in ys to xs and returns xs itself. \\

\end{tabular}
}
\end{Slide}



\Subsection{Genomgång av extenta}
\begin{Slide}{Extenta 2016-08-24 TimePlanner}\SlideFontSmall
\url{http://cs.lth.se/pgk/examination/}

\vspace{1em}\Alert{TimePlanner}: 
\begin{itemize}
\item \href{http://fileadmin.cs.lth.se/cs//Education/grundkurs/extentor/160824.pdf}{tentamen 160824} 
\item \href{http://fileadmin.cs.lth.se/cs//Education/grundkurs/extentor/sol-160824.pdf}{lösningsförslag Java} 
\item \href{https://github.com/lunduniversity/introprog/tree/master/compendium/examples/exam/re-impl-java-exams/timeplanner-160824}{översättning av lösning till Scala}
\end{itemize}
\end{Slide}


\Subsection{Avslutning av kursen}

\begin{Slide}{Obligatoriska moment}\SlideFontSmall
\begin{itemize}
\item Kolla vilka oblikatoriska moment du har kvar här:
\url{http://fileadmin.cs.lth.se/pgk/SAM-EDAA45-snapshot.html}
\item Sök på din födelsemånad/dag, tex 0401 för första April.
\item OBS! Kan ännu saknas rapportering av det som hände i fredags.
\item Läs \Alert{alla} instruktioner \Alert{noga} och \Alert{anmäl dig} här: \\
\href{http://www.student.lth.se/studieinformation/anonyma-tentor/}{www.student.lth.se/studieinformation/anonyma-tentor}
\item Du ska vara godkänd på alla labbar+projekt för att få tenta pgk EDAA45!
\item Du ska vara godkänd på alla labbar+projekt för att få gå pfk \href{http://cs.lth.se/edaa01vt}{EDAA01}!
\item Använd återstående \Emph{resurstider} för \Alert{redovisning av labbar/projekt}.
\end{itemize}
\end{Slide}
%
%\begin{Slide}{CEQ -- Course Experience Questionnaire}\SlideFontSmall
%\begin{itemize}
%\item Görs på hela LTH på samma sätt. Alla får länkar via mejl.
%\item Snälla fyll i CEQ! Jag är \Alert{mycket tacksam} för all konstruktiv feedback! \\ Hög svarsfrekvens är viktigt för att kunna dra slutsatser om variationen i svaren och signifikansen i sammanställningen.
%\item Del 1: Generella påståenden, alla med 5-gradig skala: \\ tar helt avstånd ... instämmer helt
%\item Del 2: \Emph{Fritextfrågor}: \\
%''Vad  tycker  du  var  det  bästa  med  den här  kursen?'' \\
%''Vad  tycker  du  främst  behöver  förbättras?''
%\item Fördel med CEQ: Samma alla kurser alla år medger jämförelse över tid.
%\item Begränsning med CEQ: Saknar frågor kopplat till specifika kursmoment.
%\item Mer om CEQ här: \url{https://www.ceq.lth.se/}
%\end{itemize}
%\end{Slide}
%
%\begin{Slide}{Kursspecifik utvärdering om specifika kursmoment}\SlideFontSmall
%\begin{itemize}
%\item Jag vill gärna att alla gör den LTH-gemensamma, anonyma kursutvärderingsenkäten \href{https://www.ceq.lth.se/}{CEQ}. Dina fritext-kommentarer om vad som är det bästa med kursen och vad som främst behöver förbättra emottages mycket tacksamt i CEQ-utvärderingen!
%\item Jag kommer att komplettera CEQ med en \Emph{kursspecifik} utvärdering av specifika kursmoment i denna kurs och jag är därför \Alert{mycket tacksam} om alla fyller enkäten när länk kommer via mejl. 
%\item Jag behandlar dina svar konfidentiellt, men ber om din STiL-id så att jag kan återkomma om jag mot förmodan undrar något mer.
%\item Din input är mycket värdefull vid framtida kursutveckling!
%\end{itemize}
%\end{Slide}
%
%\begin{Slide}{Intresserad av att arbeta som handledare?}\SlideFontSmall
%\begin{itemize}
%\item 
%\end{itemize}
%\end{Slide}
%
%\begin{Slide}{Utblick}\SlideFontSmall
%Framtiden för \Emph{Scala}:
%\begin{itemize}
%\item Scala 2.12 bättre bytekod, Scala 2.13 bättre standardbibliotek
%\item dotty och tasty
%\item Scala.JS: dela kod+kompetens mellan backend och frontend
%\item Scala native: kör Scala kompilerat direkt ''på metallen''
%\end{itemize}
%Några framtida \Emph{kurser} som direkt bygger på pgk:
%\begin{itemize}
%\item Fördjupningskursen (Java)
%\item Utvärdering av programvarusystem (R)
%\item Diskreta strukturer (Clojure)
%\item Programvaruutveckling i grupp 
%\item Objekt-orienterad modellering och design
%\item Funktionsprogrammering 
%\end{itemize}
%
%\end{Slide}
%
%
%\begin{Slide}{Hoppas att kursen varit kul och lärorik!}
%\includegraphics[width=5cm]{../img/gurka.jpg}\includegraphics[width=5cm]{../img/ukulele.jpg}
%\end{Slide}
%
%\begin{Slide}{Ett stort TACK för...}
%\begin{itemize}
%\item ... att ni kämpat så hårt!
%\item ... att ni ställt massor med frågor!
%\item ... att det har varit så hög närvaro på föreläsningarna!
%\item ... att ni är så konstruktiva och verkligen vill lära er!
%\end{itemize}
%\vspace{2em} \pause
%
%\Alert{Ett stort LYCKA TILL på vägen till att bli en \\ kompetent och innovativ systemutvecklare!}
%\end{Slide}


\fi
    


%\part{Appendix}
%\appendix
%%!TEX root = ../compendium.tex

\chapter{Terminalfönster och kommandoskal}

\section{Vad är ett terminalfönster?}

I ett terminalfönster kan man skriva kommandon som till exempel kör program och hanterar filer på din dator. När man programmerar använder man ofta terminalkommando för att kompilera och exekvera sina program.   
 
\subsubsection{Terminal i Linux}

\subsubsection{PowerShell i Microsoft Windows}
Microsoft Windows är inte Unix-baserat, men i kommandotolken PowerShell finns alias definierat för en del vanliga unix-kommandon. Du startar Powershell t.ex. genom att genom att trycka på Windows-knappen och skriva \texttt{powershell}.

\subsubsection{Terminal i Apple OS X}
Apple OS X är ett Unix-baserat operativsystem. Många kommandon som fungerar under Linux fungerar också under Apple OS X.

\section{Några viktiga terminalkommando}
%%!TEX encoding = UTF-8 Unicode
%!TEX root = ../compendium.tex

\chapter{Editera}\label{appendix:edit}
\section{Vad är en editor?}

En editor används för att redigera programkod. Det finns många olika editorer att välja på. Erfarna utvecklare lägger ofta mycket energi på att lära sig att använda favoriteditorns kortkommandon och specialfunktioner, eftersom detta påverkar stort hur snabbt kodredigeringen kan göras. 

En bra editor har \textbf{syntaxfärgning} för språket du använder, så att olika delar av koden visas i olika färger. Då går det lättare att läsa och hitta i koden. 

I en integrerad utvecklingsmiljö (se appendix \ref{appendix:ide}) finns en inbyggd editor som, förutom syntaxfärgning, har fler avancerade funktioner. 

\section{Välj editor}

I tabell \ref{editor:popular-editors} visas en lista med några populära editorer. Det är en stor fördel om en editor finns på flera plattformar så att du har nytta av dina inövade färdigheter när du behöver växla mellan olika operativsystem. 

Om du inte vet vilken du ska välja, börja med \textit{gedit}, som inte är så avancerad, men därför lätt att kommna igång med. När du sedan är redo att investera din lärtid i en mer avancerad editor rekommenderas \textit{Atom}, eftersom den är öppen, gratis och finns för Linux, Windows och macOS. 

Det är är också bra att lära sig åtminståne de mest basala kommandona i editorn \textit{vim} eftersom denna  editor kan köras direkt i terminalen, även vid fjärrinloggning, och finns förinstallerad i de flesta Linux-system.
%%!TEX encoding = UTF-8 Unicode
%!TEX root = ../compendium.tex

\chapter{Kompilera och exekvera}\label{appendix:compile}

\section{Vad är en kompilator?}

\section{Java JDK}

\subsection{Installera Java JDK}

\section{Scala}

\subsection{Installera Scala-kompilatorn}

\subsection{Scala Read-Evaluate-Print-Loop (REPL)}\label{appendix:compile:REPL}

För många språk, t.ex. Scala och Python, finns det en interaktiv tolk som gör det möjligt att exekvera enstaka programrader och direkt se effekten. En sådan tolk kallas Read-Evaluate-Print-Loop eftersom den läser en rad i taget och översätter till maskinkod som körs direkt.    

\TODO Kortkommandon: Ctrl+K etc.

\TODO :paste


%%!TEX root = ../compendium.tex

\chapter{Integrerad utvecklingsmiljö}

\section{Vad är en IDE?}

\section{ScalaIDE och Eclipse}

\subsection{Installera ScalaIDE}

\section{Handledning ScalaIDE/Eclipse}
%%!TEX encoding = UTF-8 Unicode
%!TEX root = ../compendium.tex

\chapter{Fixa fel}\label{appendix:debug}



\section{Vad är en bugg?}

En bugg är en oönskad egenskap hos ett program. 

\textbf{Varför heter det bugg?}


\textbf{Olika sorters fel?}

Kompileringsfel och exekveringsfel. 

Oändliga loopar eller bara långsamt... 

Det är viktigt att skilja på felorsak och felyttring.

\textbf{Bugg eller feature?} Är det verkligen ett fel eller är det egentligen ett avsett beteende?

\textbf{Felhanteringsverktyg} exempel Jira.
s
\section{Förebygga fel}

\begin{itemize}
\item \textbf{Begriplig kod}.
\item \textbf{Bra namn}.
\item \textbf{Typannoteringar}.
\item \textbf{Kontrollera villkor}.
\item \textbf{Hantera saknade värden}.
\item \textbf{Slänga undantag}.
\item \textbf{Granska kod}.
\item \textbf{Testa kod}.
\end{itemize}


\section{Vad är debugging?}

När fel identifierats, vid testning eller under användning av slutanvändare ''i produktion'' ska felorsaken hittas och felet åtgärdas. Detta kallas avlusning \Eng{debugging}.

Identifiering går vi inte vidare in på här. Testning är ett stort område....



\section{Hitta felorsaken}

När det blir fel som är svåra att hitta beror det ofta på att man har en felaktig hypotes om vad koden egentligen gör. Du stirrar dig blind på ett kodstycke och är övertygad om att en viss sak händer, men \emph{egentligen} är det \emph{inte} det du \emph{tror} händer som \emph{verkligen} händer. Exempelvis kanske du antar att en räknare räknas upp i en loop, men i själva verket saknas uppräkningen. 

En grundläggande princip i felsökning är att explicit formulera dina hypoteser och seda verifiera att de verkligen stämmer. Du ska alltså tydligt beskriva hur du tror att koden fungerar och sedan med olika former av instrumentering, t.ex. genom utskrifter i terminalen av variablers värden, kontrollera att så verkligen är fallet.

\subsection{Återskapa buggen med ett minimalt testfall}

\subsection{Instrumentering med utskrifter, ''print-debugging''}

\section{Använda en debugger}

\begin{itemize}
\item \textbf{Sätta brytpunkter}.
\item \textbf{Stegad exekvering}.
\item \textbf{Inspektera variabler}.
\end{itemize}

\subsection{Debuggern i Eclipse med ScalaIDE}
\subsubsection{Sätta brytpunkter i Eclipse}
\subsubsection{Stegad exekvering i Eclipse}
\subsubsection{Inspektera variabler i Eclipse}

\subsection{Debuggern i IntelliJ IDEA med Scala-plugin}
\subsubsection{Sätta brytpunkter i IntelliJ}
\subsubsection{Stegad exekvering i IntelliJ}
\subsubsection{Inspektera variabler i IntelliJ}



\section{Åtgärda fel}

Ibland är det det svåraste att \emph{hitta} buggen medan själva buggrättningen visar sig trivial. Har du, till exemple, väl hittat den saknade uppräkningen av din loop-variabel är det uppenbart vad du ska göra.

Men ibland är det riktigt knepigt att åtgärda felet. Nedan sammanfattas några av de situationer som kan uppkomma, som gör att felrättningen blir långt ifrån trivial. 

\begin{itemize}
\item Kanske är själva algoritmen i grunden feltänkt och en helt ny algoritm behöver konstrueras. Att skapa nya algoritmer från grunden kan visa sig mycket svårt i en del fall. I fortsättningskurser får du lära dig mer om algoritmkonstruktionens svåra konst.

\item Kanske algoritmen fungerar för olika normalfall, medan undantagsfallen inte hanteras korrekt. Att på ett bra sätt hantera alla upptänkliga fall kan visa sig väldigt knepigt. Tyvärr är det ofta undantagsfallen som öppnar för säkerhetsluckor som elaka hackare kan utnyttja för att få systemet att krascha eller smittas av virus.

\item Kanske är problemet i sig väldigt svårt att lösa på ett korrekt sätt. Algoritmen kan vara riktigt knepig med många villkor, loopar och nästlade datastrukturer. Blir det fel i en sådan algoritm kan det ta lång tid att få ändringar att fungera och alla villkor, loopar och nästlade datastrukter att passa ihop. 

\item Medan man rättar en bug kan man råka att, av misstag, skapa nya buggar. Risken för detta är speciellt stor om koden är komplex. Ibland låter man till och med bli att åtgärda ett fel om systemet ändå fungerar hjälpligt i andra avseenden och risken är för stor att ändra innan systemet strukturerats om så att det blir lättare att ändra i.

\item Kanske växer exekveringstiden exponetiellt med datamängden och det kan vara i praktiken omöjligt att skriva ett program som i alla lägen blir färdigt inom rimlig tid för tillräckigt stora datamängder. Då får man försöka tänka ut kluriga genvägar och det kan vara riktigt svårt och ibland kräva mycket avancerad programmeringsteknik.
 
\end{itemize}
%%!TEX root = ../compendium.tex

\chapter{Dokumentation}

\section{Vad gör ett dokumentationsverktyg?}

\section{scaladoc}

\section{javadoc}
%%!TEX root = ../compendium.tex

\chapter{Byggverktyg}

\section{Vad gör ett byggverktyg?}

\section{Byggverktyget sbt}

\subsection{Installera sbt}

\subsection{Använda sbt}
%%!TEX encoding = UTF-8 Unicode
%!TEX root = ../compendium2.tex


\chapter{Versionshantering och kodlagring}

\section{Vad är versionshantering?}

\textbf{Versionshantering}\footnote{\href{https://en.wikipedia.org/wiki/Version_control}{en.wikipedia.org/wiki/Version\_control}} \Eng{version control \textup{eller} revision control} av mjukvara innebär att hålla koll på olika versioner av koden i ett utvecklingsprojekt allteftersom koden ändras. Versionshantering är en deldisciplin inom \textbf{konfigurationshantering} \Eng{software configuration management} som inbegriper allt i processen för att identifiera, besluta, genomföra och följa upp ändringar.

En viktig del av versionshantering är att \textit{lagra} olika versioner av koden allt eftersom den utvecklas, så att tidigare versioner kan \textit{återskapas} vid behov. Ett bra verktygsstöd och en väldefinierad arbetsprocess för versionshanteringen, som alla i utvecklingsprojektet följer, möjliggör att flera utvecklare kan \textit{arbeta parallellt} med att sammanfoga \Eng{merge} varandras tillägg och ändringar i den gemensamma kodbasen utan att det blir kaos och förvirring.

God versionshantering är helt avgörande för utvecklarnas produktivitet, speciellt för stora projekt med många utvecklare som jobbar parallellt mot en omfattande kodbas med många olika interna och externa komponenter. 
Men även ett litet projekt med en enda utvecklare kan ha god nytta av ett versionshanteringsverktyg och ett disciplinerat förfarande för att namnge versioner, t.ex. för att kunna återskapa tidigare versioner av projektets olika kodfiler när en ändring visar sig mindre lyckad.   

Det finns flera olika modeller för hur kodlagringen sker:
\begin{itemize}
\item \textbf{lokal}; alla utvecklare jobbar i samma, lokala filsystem där alla olika versioner lagras.
\item \textbf{centraliserad}; ett repositorium (förk. repo), alltså en databas med koden, finns centralt på en server som alla jobbar mot med hjälp av en versionshanteringsklient.
\item \textbf{distribuerad}; alla utvecklare har sitt eget lokala repo och varje utvecklare initierar enskilt delning av ändringar mellan olika repo. 
\end{itemize}


\section{Versionshanteringsverktyget Git}

Det finns många olika versionshanteringsverktyg\footnote{\href{https://en.wikipedia.org/wiki/List_of_version_control_software}{https://en.wikipedia.org/wiki/List\_of\_version\_control\_software}}
 som använder olika modeller för kodlagring; lokal, centraliserad, distribuerad eller kombinationer därav. 
På senare tid har verktyget \textbf{Git}\footnote{\href{https://en.wikipedia.org/wiki/Git_(software)}{https://en.wikipedia.org/wiki/Git\_(software)}} fått en stark ställning, speciellt i öppenkällkodsvärlden. Git utvecklades ursprungligen av Linus Torvalds för att versionshantera Linuxkärnan, men har växt till ett omfattande öppenkällkodsprojekt med stor spridning och många användare och bidragsgivare. 

Git är skapad för \textbf{distribuerad} versionshantering där var och en kan jobba snabbt och smidigt i sitt eget lokala repo, utan att behöva vänta på att en klient ska synkronisera koden med ett centralt repo på en server över nätverket. Ändringar delas mellan repo på begäran av enskilda utvecklare. 

Varje ny version av koden lagras som en avgränsad mängd ändringar sedan förra versionen, en s.k. \textbf{commit}%
\footnote{På svenska kan t.ex. ''inlämning'' användas, men låneordet commit är redan etablerat.}%
, och hanteras internt av Git i en lokal databas i katalogen \code{.git} som ligger överst i din projektkatalog. Genom olika kommandon i terminalen, eller via en klient med ett grafiskt användargränssnitt, kan din kod överföras till och från den lokala koddatabasen, alternativt delas med andra repon via nätet. 

Det finns en välskriven bok kallad \textit{''Pro Git''} som förklarar Git på djupet och är tillgänglig fritt här: 
\url{https://git-scm.com/book/en/v2}.
Läs kapitel 1 och 2 så får du en bra grund att stå på. 

Dessa termer är bra att kunna utantill innan du kör igång med Git:
\newcommand{\TermItem}[3]{\item \textbf{#1} (\textit{substantiv}: #2, \textit{verb}: #3).}
\begin{itemize}

\item \textbf{repo} (\textit{substantiv}: ett repositorium, \textit{eng. a repository}) En koddatabas med ändringshistorik. 

\TermItem{commit}{en inlämning}{att lämna in} 
  En avgränsad mängd nya ändringar lämnas in i det lokala repot. Repots ändringshistorik utgörs av sekvensen av alla inlämningar.

\TermItem{push}{en leverans}{att leverera, att trycka upp} En eller flera inlämningar trycks upp till ett annat repo.

\TermItem{pull}{en hämtning}{att hämta, att dra ner} En eller flera inlämningar dras ner från ett annat repo.

\TermItem{merge}{en sammanfogning}{att sammanfoga} En eller flera inlämningar slås samman till en ny inlämning. 

\item \textbf{merge conflict} (\textit{substantiv}: en sammanfogningskonflikt, \textit{eng. a merge conflict}) Problem vid sammanfogning; ändringar kan inte enkelt sammanfogas på ett entydigt sätt.

\item \textbf{pull request} (förk. PR, \textit{substantiv}: en hämtningsbegäran, \textit{verb}: att begära en hämtning). Utvecklare A ber en annan utvecklare B att hämta en eller flera inlämningar från A:s repo och sammanfoga med B:s repo.

\end{itemize}

\subsection{Installera git}\label{subsection:install-git}

Git finns förinstallerat på LTH:s Linuxdatorer. Du kan kolla om Git redan finns på din maskin genom att skriva \code{git help} i terminalen. 

Det finns bra instruktioner om hur du installerar Git på din egen maskin här: \url{https://git-scm.com/book/en/v2/Getting-Started-Installing-Git}

VS Code har speciellt stöd för git och du kan inne ifrån VS Code göra t.ex. add, commit, push och pull via editorns grafiska gränssnitt. Läs mer här: \url{https://code.visualstudio.com/docs/editor/versioncontrol}

Det finns även många andra grafiska användargränssnitt till git, t.ex. \href{https://desktop.github.com/}{GitHub Desktop (Windows/Mac) eller \href{https://www.gitkraken.com/}{GitKraken (Linux/Windows/Mac)}}. Se fler exempel här: \url{https://git-scm.com/downloads/guis} 

%Om du inte vet vilken du ska välja, prova GitKraken som är gratis (men stängd) och finns för alla plattformar: \url{https://www.gitkraken.com/}.


\subsection{Anpassa Git}

Innan du börjar använda git, konfigurera ditt namn och din email med nedan terminalkommando, där du anger ditt namn i stället för \code{Förnamn Efternamn} och din mejladress i stället för \code{mejladr@plats.se}. Namnet och mejladressen kommer lagras i varje commit som du gör så att det går att se vem som har gjort en given ändring.
\begin{REPLnonum}
> git config --global user.name "Förnamn Efternamn"
> git config --global user.email mejladr@plats.se
\end{REPLnonum}

Läs mer om hur du gör andra inställningar här, t.ex. hur du anger vilken editor som git startar när du ska skriva commit-beskrivningar: \\ \url{https://git-scm.com/book/en/v2/Getting-Started-First-Time-Git-Setup}


\subsection{Använda git}

Nedan listas några vanliga terminalkommandon i Git.

\begin{itemize}[leftmargin=*]

\item Skapa ett repo i en katalog:
\begin{REPLnonum}
> cd myproject
> git init
\end{REPLnonum} 

\item Se vilka filer som ändrats och ännu ej lämnats in:
\begin{REPLnonum}
> git status
> git status -s
\end{REPLnonum} 

\item Se vilka ändringar som gjorts i filer som ännu ej lämnats in:
\begin{REPLnonum}
> git diff 
\end{REPLnonum} 

\item Se vilka inlämningar som finns i ändringshistoriken:
\begin{REPLnonum}
> git log 
> git log --oneline -5
\end{REPLnonum} 

\item Lägg till filer som ska ingå i nästa inlämning och gör sedan inlämningen; ge inlämningen en bra beskrivning som förklarar vad inlämningen omfattar:
\begin{REPLnonum}
> git add *.scala
> git commit -m 'initial project version'
\end{REPLnonum} 

\item Ångra alla tillägg inför inlämning (ändringarna finns kvar och kan läggas till igen om du vill):
\begin{REPLnonum}
> git reset 
\end{REPLnonum} 

\item Du kan skippa de senaste, ännu ej commitade, ändringar i filen \code{filename}, och göra ''\textit{undo}'', med kommandot \code{git checkout} på filen enligt nedan. Gör bara detta om du är helt säker på att du vill ångra dina senaste ändringar.
\\ \mbox{\colorbox{red!30}{VARNING!} Dina senaste ändringar i filen förloras för alltid; kan ej ångras!}   
\begin{REPLnonum}
> git checkout filename 
\end{REPLnonum} 

\item Man vill förhindra versionshantering av vissa filer, t.ex. binärkodsfiler så som \code{.class}-filer och andra genererade filer. Detta gör du genom att skapa en fil med namnet \code{.gitignore} och lägga in filändelser enligt nedan syntax, där \code{**/} avser alla kataloger och underkataloger och \code{*} kan vara vilken början på ett filnamn som helst. Symbolen \code{#} föregår en kommentarsrad.
\begin{Code}[language=]
# this is my .gitignore

# Java / Scala
**/*.class

# Sbt
**/target

\end{Code} 


\end{itemize}
 

\clearpage 
  
\section{Kodlagringsplatser på nätet}\label{section:code-hosting}

Många utvecklare använder kodlagringsplatser på nätet (''i molnet'') \Eng{code hosting} för att underlätta samarbete kring kod och för att dela med sig av öppen källkod. Det finns många olika kodlagringsplatser som kan användas gratis under vissa förutsättningar eller mot betalning med tillhörande extratjänster. 

\begin{oframed}
  \noindent \textbf{OBS!} Du får \emph{inte} lagra dina lösningar på kursens laborationer i ett öppet repo. Om du vill använda en kodlagringsplats måste du säkerställa att dina lösningar förblir i ett stängt repo utan att någon annan kan komma åt det.
\end{oframed}

Nedan beskrivs några vanliga nätplatser för öppen och sluten kodlagring, som alla är Git-baserade:

\begin{itemize}
\item  \textbf{GitHub}, \url{https://github.com}, är en av de mest populära kodlagringsplatserna för öppen källkod, men har även blivit en populär plats för jobbsökande utvecklare att visa upp sina  kodarbetsprover för framtida arbetsgivare. GitHub är gratis att använda för dig som privatperson. Många företag betalar GitHub för att lagra sin stängda kod med tilläggstjänster för att testa, bygga och driftsätta kod etc. Koden som styr själva kodlagringsplatsen GitHub är stängd, till skillnad från GitLab. GitHub köptes \href{https://computersweden.idg.se/2.2683/1.703485/microsoft-kop-github}{2018} av Microsoft för 65 miljarder kronor.

\item \textbf{GitLab}, \url{https://gitlab.com}, erbjuder gratis kodlagring för öppen källkod, men det är även gratis för privatpersoner och gemenskapsprojekt att ha stängda repo. Företag kan betala för stängd kodlagring med extratjänster för att testa, bygga och driftsätta kod etc. GitLab är i sig ett öppenkällkodsprojekt och koden som styr kodlagringsplatsen är öppen och fri. Detta innebär att du själv kan ladda ner koden och starta en kodlagringsplats. LTH har en GitLab-baserad kodlagringsplats här: \url{https://git.cs.lth.se}

\item \textbf{BitBucket}, \url{https://bitbucket.org}, är en populär kodlagringsplats både för öppen och stängd källkod och drivs av det australiensiska företaget Atlassian. Det är gratis för privatpersoner och små team att ha både öppna och slutna repon, men bara om det är få bidragsgivare. Kostnader tillkommer om antalet bidragsgivare kommer över en viss nivå. Universitetsanställda och studenter kan få mer gynnsamma villkor efter ansökan. Atlassian erbjuder en hel verktygssvit för att hantera buggar och samarbeta över nätet. BitBucket stödjer, förutom Git, även andra versionshanteringsverktyg.

\end{itemize}

\subsubsection{Använda kodlagringsplatser}

Om du inte redan gjort det är det bra om du registrerar ditt användarnamn, förslagsvis \code{fornamnefternamn} som ett ord utan svenska tecken med små bokstäver, på någon eller alla av ovan sajter, dels för att paxa ditt namn och dels för börja samarbeta med utvecklare världen över. Det är bra att välja \textit{ett} användarnamn för \textit{alla} kodlagringsplatser på nätet, speciellt om du jobbar med öppen källkod där ditt namn kommer associeras med alla de kodbidrag du gör under ditt yrkesliv.

Om du inte vet vilken sajt du ska välja, börja med \url{https://github.com}. Om du vill att även kodlagringssajten ska drivas av öppen källkod, testa \url{https://gitlab.com}.

Med en Git-baserad kodlagringsplats får du möjlighet att synka ditt lokala repo mot en server på nätet med hjälp av \code{git}-kommandon i terminalen eller via en Git-klient med grafiskt användargränssnitt, se avsnitt \ref{subsection:install-git}. 

Innan du börjar använda en kodlagringsplats är det bra att sätta sig in i begreppen nedan.

\begin{itemize}
\TermItem{clone}{en klon är kopia av ett (nätlagrat) repo}{att klona, att skapa en kopia} Genom att klona ett repo som ligger på en nätlagringsplats kan du bygga, undersöka och vidareutveckla koden lokalt på din dator. Om du har rättigheter att lämna in kod till det centrala originalet kan du pusha dina commits direkt via terminalkommando eller Git-klient.

\TermItem{fork}{en förgrening av ett helt repo}{att förgrena ett repo, att ''forka''} Genom att förgrena ett repo skapar du en kopia, normalt även den nätlagrad på en kodlagringsplats, som du kan utveckla separat från originalet. Det blir då möjligt för dig att lämna in ändringar och trycka upp dem, även om du inte har rättigheter att leverera (''pusha'') till originalet. Gör en ändringsbegäran (Pull Request, PR) om du vill bidra med dina ändringar, så kan ägaren av originalet sedan välja att sammanfoga (''merga'') dina ändringar med originalet. Många nätlagringsplatser, så som GitHub, har en speciell knapp som du trycker på för att enkelt skapa en fork av ett repo under din användare. 

\item \textbf{upstream} (\textit{preposition}: uppströms, \textit{substantiv}: uppströmsrepo) Ett uppströmsrepo utgör original till ett förgrenat repo (en ''fork''). 
\begin{itemize}[noitemsep,nolistsep]

\item Här beskrivs hur du länkar en förgrening uppströms: \\ 
{\small\url{https://help.github.com/articles/configuring-a-remote-for-a-fork/}}

\item Här beskrivs hur du synkar en förgrening uppströms:\\
{\small\url{https://help.github.com/articles/syncing-a-fork/}}

\end{itemize}

\end{itemize}

Om du vill bidra till ett öppenkällkodsprojekt, börja med att forka repot på kodlagringsplatsen och sedan klona repot till din lokala dator. Därefter kan du commita ändringar och pusha till din fork och slutligen göra en pull request från din fork till upstream. Läs om hur ett bidrag kan gå till i avsnitt \ref{section:OSS-contribution-example}.

Här följer några användbara kommandon:

\begin{itemize}
\item Skapa en lokal kopia av ett fjärran \Eng{remote} repo; här visas hur du klonar kursens repo från GitHub:
\begin{REPLnonum}
$ git clone --depth 1 https://github.com/lunduniversity/introprog
\end{REPLnonum} 

\item Dra ner nya inlämningar från ett fjärran repo:
\begin{REPLnonum}
$ git pull 
\end{REPLnonum} 

\item Trycka upp nya lokala inlämning till ett fjärran repo:
\begin{REPLnonum}
$ git push 
\end{REPLnonum} 

\end{itemize}



%%!TEX root = ../compendium.tex

\chapter{Virtuell maskin}\label{appendix:vbox}

\section{Vad är en virtuell maskin?}

Du kan köra alla kursens verktyg i en så kallad virtuell maskin (vm). Det är ett enkelt och säkert sätt att installera ett nytt operativsystem i en "sandlåda" som inte påverkar din dators ursprungliga operativsystem. 

\section{Installera kursens vm}
Det finns en virtuell maskin förberedd med alla verktyg som du behöver förinstallerade. Gör så här:
\begin{enumerate}
\item     Installera VirtualBox v5 här: \\ \url{https://www.virtualbox.org/wiki/Downloads}
\item     Ladda ner filen vbox.zip här: \\ \url{http://fileadmin.cs.lth.se/pgk/vbox.zip} \\ OBS! Då filen är på nästan 4GB kan nedladdningen ta mycket lång tid.
\item     Packa upp filen vbox.zip i biblioteket "VirtualBox VMs" som du fick i din hemkatalog när du installerade VirtualBox. Du får då 3 filer som heter något med "introprog-ubuntu-64bit".
\item     Kolla med hjälp av denna sida: \\ \url{https://md5file.com/calculator} \\ så att filen "introprog-ubuntu-64bit.vdi" har denna sha256-cheksumma: \\ --- ska-stå-checksumma-här-sen ---
\item     Öppna VirtualBox och lägg till maskinen introprog-ubuntu-64bit genom menyn ''add''.
\item     Starta maskinen.
\item     Öppna ett terminalfönster och skriv scala och du är igång och kan göra första övningen!
\end{enumerate}

\section{Vad innehåller kursens vm?}

Den virtuella maskinen kör Xubuntu 14.04 med fönstermiljön XFCE, vilket är samma miljö som E-husets linuxdatorer kör. 

I den virtuella maskinen finns detta förinstallerat:

\begin{itemize}
\item Java JDK 8
\item Scala 2.11.8
\item Kojo 2.4.08
\item Eclipse Mars.2 med ScalaIDE 4.3
\item gedit med syntaxfärgning för Scala och Java
\item git
\item sbt
\item Ammonite REPL
\end{itemize}
%%!TEX root = ../compendium.tex


\ChapterUnnum{Hur bidra till kursmaterialet?}


%\chapter{Ordlista}

\part{Lösningar till övningar}
\appendix

\chapter{Lösningar till övningarna}\label{chapter:solutions}

%!TEX encoding = UTF-8 Unicode
%!TEX root = ../solutions.tex


%\BasicTasks %%%%%%%%%%%
% uppgift 2

\noindent\TODO{HÄREFTER KOMMER GAMLA LÖSNINGAR FRÅN \code{-solutions.tex}}


\AdvancedTasks %%%%%%%%%



\Task % Uppgift 37

\Subtask 
Den första raden returnerar 84. Den andra kastar ett exception.

\Subtask
För att kunna hantera situationer när bydelängden på variabler inte är lång nog för värden.

\Subtask
Overflow är när en variabel inte kan inehålla ett värde då det är för stort och istället  blir ett värde som variabeln egentligen inte ska få.

\Task % Uppgift 38

\Subtask
\code{4.9E-3240}

\Subtask
\code{-1.7976931348623157E308}

\Subtask
\code{4.9E-324}

\Task % Uppgift 39

\Task % Uppgift 40

\begin{Code}
val s = f"Gurkan är $g meter lång"
\end{Code}







%!TEX encoding = UTF-8 Unicode
%!TEX root = ../solutions.tex

\ExerciseSolution{\ExeWeekTWO}

%Uppgift 1
\Task 

\Subtask värde: \code{Range(1,2,3,4,5,6,7,8,9)}

typ: \code{scala.collection.immutable.Range}

\Subtask värde: \code{Range(1,2,3,4,5,6,7,8,9,10)}

typ: \code{scala.collection.immutable.Range}

\Subtask värde: \code{Range(0,5,10,15,20,25,30,35,40,45)}

 typ: \code{scala.collection.immutable.Range}

\Subtask värde: \code{10}, typ: \code{Int}

\Subtask värde: \code{Range(0,5,10,15,20,25,30,35,40,45,50)}

typ: \code{scala.collection.immutable.Range}

\Subtask värde: \code{11}, typ: \code{Int}

\Subtask värde: \code{Range(0,1,2,3,4,5,6,7,8,9)}

typ: \code{scala.collection.immutable.Range}

\Subtask värde: \code{Range(0,1,2,3,4,5,6,7,8,9)}

typ: \code{scala.collection.immutable.Range}

\Subtask värde: \code{Range(0,1,2,3,4,5,6,7,8,9)}

typ: \code{scala.collection.immutable.Range}

\Subtask värde: \code{Range(0,1,2,3,4,5,6,7,8,9,10)}

typ: \code{scala.collection.immutable.Range.Inclusive}

\Subtask värde: \code{Range(0,1,2,3,4,5,6,7,8,9,10)}

typ: \code{scala.collection.immutable.Range.Inclusive}

\Subtask värde: \code{Range(0,5,10,15,20,25,30,35,40,45)}

typ: \code{scala.collection.immutable.Range}

\Subtask värde: \code{Range(0,5,10,15,20,25,30,35,40,45,50)}

typ: \code{scala.collection.immutable.Range}

\Subtask värde: \code{11}, typ: \code{Int}

\Subtask värde: \code{500500}, typ: \code{Int}


%Uppgift 2
\Task 

\Subtask Ett objekt av typen \code{Array[String]} skapas med värdet 

\code{Array(hej, på, dej, !)} och med namnet \code{xs}.

\Subtask Returnerar en sträng med värdet \code{hej}.

\Subtask Returnerar en sträng med värdet \code{!}.

\Subtask Ett exception genereras. Skriver ut:

\code{java.lang.ArrayIndexOutOfBoundsException: 4}

\Subtask Returnerar en sträng med värdet \code{på dej}.

\Subtask Returnerar en sträng med värdet \code{hejpådej!}.

\Subtask Returnerar en sträng med värdet \code{hej på dej !}.

\Subtask Returnerar en sträng med värdet \code{(hej,på,dej,!)}.

\Subtask Returnerar en sträng med värdet \code{Array(hej,på,dej,!)}.

\Subtask Ett fel uppstår av typen \code{type mismatch}. Konsollen talar om för oss vad den fick, dvs värdet \code{42} av typen \code{Int}. Den talar även om för oss vad den ville ha, dvs något värde av typen \code{String}. Till sist skriver den ut vår kodrad och pekar ut felet.

\Subtask Det första elementet i \code{xs} ändras till värdet \code{42}. Därefter skrivs det första värdet i \code{xs} ut.

\Subtask Ett objekt av typen \code{Array[Int]} skapas med värdet \code{Array(42, 7, 3, 8)} och med namnet \code{ys}.

\Subtask Returnerar summan av elementen i \code{ys}. Resultatet är \code{60}.

\Subtask Returnerar det minsta värdet i \code{ys}. Resultatet är \code{3}.

\Subtask Returnerar det största värdet i \code{ys}. Resultatet är \code{42}.

\Subtask Ett nytt värde av typen \code{Array[Int]} skapas med \code{10} stycken element, alla med värdet \code{42}.

\Subtask Returnerar summan av elementen i \code{zs}. Resultatet blir 420 (42 multiplicerat med 10).

\Subtask \code{r} tar upp 12 bytes. \code{a} tar upp ca 4 miljarder bytes.

%Uppgift 3
\Task 

\Subtask Ett objekt av typen \code{scala.collection.immutable.Vector[String]} initieras med värdet \code{Vector(hej, på dej, !)}.

\Subtask Returnerar det nollte elementet i \code{words}, dvs strängen \code{hej}.

\Subtask Returnerar det tredje elementet i \code{words}, dvs strängen \code{!}.

\Subtask Omvandlar vektorn till en Sträng.

\Subtask Samma som ovan, fast den här gången används mellanrum för att seperera elementen.

\Subtask Samma som ovan, fast den här gången sepereras elementen av kommatecken istället för mellanrum och dessutom börjar och slutar den resulterande strängen med parenteser.

\Subtask Samma som ovan, fast med ordet \code{Ord} tillagt i början av den resulterande strängen.

\Subtask Ett fel uppstår. Typen \code{Vector} är immutable. Dess element kan alltså inte bytas ut.

\Subtask En ny \code{Vector[Int]} skapas med värdet \code{Vector(42, 7, 3, 8)}. 

\Subtask Returnerar summan av vektorn \code{numbers}.

\Subtask Returnerar vektorns minsta element.

\Subtask Returnerar vektorns största element. 

\Subtask En ny vektor skapas innehållandes tiotusen 42or.

\Subtask Returnerar summan av vektorns element.

\Subtask Byta ut element.

%Uppgift 4
\Task 

\Subtask typ: \code{scala.collection.immutable.IndexedSeq[Int]}

värde: \code{Vector(1, 2, 3, 4, 5, 6, 7, 8, 9)}

\Subtask typ: \code{scala.collection.immutable.IndexedSeq[Int]}

värde: \code{Vector(1, 2, 3, 4, 5, 6, 7, 8, 9)}

\Subtask typ: \code{scala.collection.immutable.IndexedSeq[Int]}

värde: \code{Vector(2, 3, 4, 5, 6, 7, 8, 9, 10)}

\Subtask typ: \code{scala.collection.immutable.IndexedSeq[Int]}

värde: \code{Vector(1, 2, 3, 4, 5, 6, 7, 8, 9, 10)}

\Subtask typ: \code{scala.collection.immutable.IndexedSeq[Int]}

värde: \code{Vector(1, 2, 3, 4, 5, 6, 7, 8, 9, 10)}

\Subtask typ: \code{scala.collection.immutable.IndexedSeq[Int]}

värde: \code{Vector(2, 3, 4, 5, 6, 7, 8, 9, 10, 11)}

\Subtask typ: \code{Int}, värde: \code{Vector(65)}

\Subtask typ: \code{scala.collection.immutable.IndexedSeq[Int]}

värde: \code{Vector(0.0, 0.707, 1.0, 0.707, 0.0, -0.707, -1.0, -0.707)}

%Uppgift 5
\Task 

\Subtask typ: \code{scala.collection.immutable.IndexedSeq[Int]}

värde: \code{Vector(1, 2, 3, 4, 5, 6, 7, 8, 9, 10)}

\Subtask typ: \code{scala.collection.immutable.IndexedSeq[Int]}

värde: \code{Vector(1, 2, 3, 4, 5, 6, 7, 8, 9, 10)}

\Subtask typ: \code{scala.collection.immutable.IndexedSeq[Int]}

värde: \code{Vector(2, 4, 6, 8, 10, 12, 14, 16, 18, 20)}

\Subtask typ: \code{scala.collection.immutable.IndexedSeq[Int]}

värde: \code{Vector(2, 4, 6, 8, 10, 12, 14, 16, 18, 20)}

\Subtask typ: \code{scala.collection.immutable.Vector[Int]}

värde: En vector av tiotusen 85or (85 = 42 + 43).

%Uppgift 6
\Task 

\Subtask En \code{Range} skapas och dess element skrivs ut ett och ett.

\Subtask Samma sak händer.

\Subtask De tio första första jämna talen (noll ej inräknat) skrivs ut med ett "hej" framför.

\Subtask Talen 1 till 10 skrivs ut.

\Subtask Tiotusen slumptal mellan 0 och 1 genereras. Varje gång ett tal är större än 0.99 kommer det ett pling.

%Uppgift 7
\Task 

\Subtask Pseudokoden kan se ut såhär:

Skapa heltalsvariabel temp. 
Flytta värdet från x till temp. 
Flytta värdet från y till x. 
Flytta värdet från temp till y.

\Subtask
\begin{REPLnonum}
scala> var (x, y) = (42, 43)
x: Int = 42
y: Int = 43
scala> var temp = x; x = y; y = temp;
temp: Int = 42
x: Int = 43
y: Int = 42
scala> println("x är " + x + ", y är " + y)
x är 43, y är 42
\end{REPLnonum}

%Uppgift 8
\Task 

\Subtask Skriver ut "hej skript".

\Subtask Ett felmeddelande skrivs ut.

\Subtask Lägg till raden:
\code{println((2 to 1001).sum)} 
eller motsvarande.

\Subtask Filen ska se ut ungefär såhär: \\
\begin{Code} 
val n = args(0).toInt 
println("hej skript") 
println((1 to n).sum)
\end{Code}

\Subtask \code{java.lang.ArrayIndexOutOfBoundsException: 0}

%Uppgift 9
\Task 

\Subtask Hello.class och Hello\$.class

\Subtask Ta bort en av hakparenteserna i slutet.

\Subtask I ett skript behöver man inte skriva någon main-metod. Kompilatorn lägger till en automatiskt precis när koden ska köras. I en applikation behöver man däremot det. För att göra en applikation definierar vi ett objekt som vi i det här fallet kallar för \code{Hello}. Från början gör inte objekt någonting. De bara finns. För att objekt ska kunna göra något behövs det metoder. I vanliga fall utförs inte metoder förrän en annan metod "ropar" på metoden. main-metoden ropas dock automatiskt när en applikation startas. Annars hade ju ingenting hänt, eftersom alla metoderna väntar på att någon annan metod ska börja. \\
\Subtask Första gången man ska köra en applikation måste den först kompileras innan den exekveras. Skript kompileras automatiskt samtidigt som de exekveras, vilket totalt sett görs på kortare tid. Därför tar det längre tid att starta en applikation första gången än att starta ett skript första gånge. När en applikation väl har kompileras och kan exekveras, går det dock mycket fortare. Fördelen med applikationer är att de kan exekveras flera gånger utan att kompileras om.

%Uppgift 10
\Task 

\Subtask Hi.class

\Subtask I javas syntax börjar man med orden \code{public static}. I scala uteblir dessa. I scala är alla metoder automatiskt publika om inget annat används. Därför behövs aldrig ordet \code{public} i scala. I scala finns det tekniskt sett inga statiska metoder. Men i praktiken fungerar vanliga metoder i ett scala-objekt på ungefär samma sätt som statiska metoder i en java-klass. I scala används ordet \code{def} varje gång en funktion ska definieras. I java slipper man det. I java skriver man returtypen (\code{void}) innan parametrarna. I scala kommer istället metodens returtyp (\code{Unit}) i slutet. Javas \code{void} motsvarar scalas \code{Unit}. I scalas syntax kommer parameterns namn (\code{args}) före parameterns typ (\code{Array[String]}), separerat med ett kolon. I java kommer typen (\code{String[]}) först och sen kommer namnet (\code{args}). \code{String[]} i java betyder ungefär samma sak som \code{Array[String]} i scala.

\Subtask -

%Uppgift 11
\Task 

\Subtask Bugg: Eftersom \code{i} inte ökar, fastnar programmet i en oändlig loop. Fix: Lägg till en sats i slutet av while-blocket som ökar värdet på i med 1.
Bugg: Eftersom man bara ökar summan med 1 varje gång, kommer resultatet att bli summan av n stycken 1or, inte de n första heltalen. Fix: Ändra så att summan ökar med \code{i} varje gång, istället för 1.
För -1, blir resultatet 0. Förklaring: i börjar på 1 och är alltså aldrig mindre än n som ju är -1. while-blocket genomförs alltså noll gånger, och efter att \code{sum} får sitt ursprungsvärde förändras den aldrig.
\Subtask 39502716
\Subtask -
\Subtask Såhär kan implementationen se ut:
\begin{Code}
public class SumN {
  public static void main(String[] args) {
    int n = Integer.parseInt(args[0]);
    int sum = 0;
    int i = 1;
    while(i <= n){
      sum = sum + i;
      i = i + 1;
      }
    }
    System.out.println(sum);
}
\end{Code}

%Uppgift 12
\Task 

\Subtask Bugg: i ökar aldrig. Programmet fastnar i en oändlig loop. Fix: Lägg till en sats som ökar i med 1, i slutet av while-blocket.

\Subtask Så här kan implementationen se ut:
\begin{Code}
object Max {
  def main(args: Array[String]): Unit = {
    var max = Int.MinValue
    val n = args.size
    var i = 0
    while(i < n) {
      val x = args(i).toInt
      if(x > max) {
        max = x
      }
      i = i + 1
    }
    println(max)
  }
}
\end{Code}
\Subtask Raden där max initieras ändras till \code{var max = args(0).toInt} 

\Subtask \code{java.lang.ArrayIndexOutOfBoundsException: 0}

%Uppgift 13
\Task

\Subtask Skriver ut talet 8. \code{a} får värdet \code{4 + 4} eftersom detta är den sista satsen i blocket. Man får också tre stycken varningar. Detta beror på att det förekommer tre satser i blocket som inte gör någon skillnad.

\Subtask Skriver ut talet 5. De tre första satserna i det yttre blocket ignoreras. \code{b} får värdet som returneras av det yttre blocket. Det yttre blocket returnerar värdet som returneras i den sista satsen i blocket, som i sin tur är ett block. I det inre blocket skapas en ny \code{val} som också får namnet \code{b}. Notera att detta alltså inte är samma värde, även om det har samma namn. Den andra satsen räknar summan av \code{b} med sig själv. Eftersom vi nu befinner oss i det block där det andra \code{b}et precis har definieras så är det detta \code{b} som används och summan blir alltså åtta. Detta är dock helt irrelevant eftersom resultatet inte sparas någonstans. I den sista satsen blir resultatet 5 (eftersom \code{b} är fyra och vi adderar ett). Detta resultatet returneras från det innre blocket och vidare ur det yttre blocket.

\Subtask Skriver ut talet 42. Blockets satser exekveras i ordning. 

\Subtask Skriver inte ut 42. I blocket skapas ett \code{val} med namnet \code{a} och värdet \code{42}. Detta värde finns inte utanför blocket och kommer därför inte att skrivas ut. Om du däremot definierat \code{a} som något annat tidigare så kommer istället det värdet att skrivas ut.

\Subtask Skriver först ut \code{43} och sedan \code{42}. Förklaring:

\code{a} initieras med värdet \code{42}. Ett nytt värde som också har namnet \code{a} initieras med värdet \code{43}. Eftersom detta sker innanför ett nytt block, befinner vi oss i ett annat "namespace" och det gör alltså inget att vi använder samma namn. \code{a} skrivs ut. Eftersom vi befinner oss i det inre blocket är det \code{43} som skrivs ut, inte \code{42}. Scala kollar först efter värden som heter \code{a} i det inre "namespacet". Det är först i andra hand som den skulle upptäcka att det finns ett \code{a} i det yttre blocket. Till sist körs den sista satsen i det yttre blocket. Då skrivs \code{a} ut. Eftersom vi nu befinner oss i det yttre blocket, vet inte ens scala om att det andra \code{a}:et existerar. Resultatet av den här utskriften blir alltså \code{42}.

\Subtask Ett fel uppstår. Variabeln \code{a} initieras två gånger i samma namespace. Förklaring till felet:

I det yttre blockets första sats initieras variablen \code{a} med värdet \code{42}. I det yttre blockets tredje sats försöker vi definiera en ny variabel med samma namn. I och med att vi befinner oss i samma namespace, krockar namnen.

Förklaring till vad som händer i sats två:

I det inre blocket har vi inte definierat någon variabel \code{a}. Till en början hittar alltså inte scala något sådant. Då letar scala vidare i det namespace som finns utanför det inre blocket och hittar variabeln som vi definierade i det yttre blockets första sats. Denna variabel får sitt värde förändrat.

\Subtask Fel. Framåtreferens. Förklaring:

Det är inte tillåtet att referera till variabler som initieras senare i koden.

\Subtask Skriver ut \code{85}. Förklaring:

I och med att vi den här gången initierade variabeln \code{b} och gav den ett värde innan vi använder oss av den, slipper vi problemet ovan.

\Subtask Skriver ut \code{85}. Förklaring:

Det är tillåtet att referera till funktioner som definieras senare i koden.

\Subtask Skriver ut \code{85}. Förklaring:

\code{a.b} refererar till variabeln \code{b} som ingår i objektet \code{a}.
\code{a.a.a} refererar till variabeln \code{a}, som ingår i ett objekt som heter \code{a} som i sin tur befinner sig i ett annat objekt som också heter \code{a}.

\Subtask Skriver ut \code{85}. Förklaring:

Koden är identisk med förra deluppgiften förutom att ny rad används istället för semikolon.

\Subtask I stora projekt med mycket kod, kan det vara svårt att hitta unika namn till alla sina variabler. Då är det en fördel om man kan hålla sina variabler i begränsade namespaces, så att de bara är tillgängliga precis när de behöver användas. 

%Uppgift 14??? NUMMER I KOMMENTAR STÄMMER EJ MED GENERERAT NUMMER
\Task 

\Subtask \code{script   security   smartcardio   sound   sql   swing}

\Subtask Radernas funktion i ordning:

1. Importerar JOptionPane från javax.swing

2. Definierar en metod som tar en sträng och öppnar en dialogruta med strängen.

3. Testar funktionen med argumentet "Hej på dej!". En dialogruta öppnas med texten "Hej på dej!".

4. Definierar en metod som tar emot en sträng som argument och öppnar en input-dialogruta med strängen.

5. Testar funktionen med argumentet "Vad heter du?". En dialogruta öppnas med texten "Vad heter du?". I ett fält kan man fylla i sitt namn. Funktionen returnerar namnet.

6. Importerar showOptionDialog från JOptionPane under namnet optDlg.

7. Definierar en metod som tar emot en sträng och en Array som argument och öppnar en flervalsdialog. Strängen ska innehålla frågan som flervalsdialogen visar upp. Arrayn ska innehålla alternativen som användaren ska välja mellan.

8.Testar funktionen med argumenten \code{"Vad väljer du?"} och \\ \code{Array("Sten, "Sax", "Påse")}. En dialogruta kommer upp och man får möjlighet att välja sten sax eller påse. Funktionen returnerar valet som man gör.

\Subtask På alla ställen där \code{JOptionPane} förekommer, hade man istället fått skriva \code{javax.swing.JOptionPane}.

\Subtask -

%Uppgift 15
\Task 

\Subtask jar cvf [namn på skapad fil] [namn på input-filer]

\Subtask -

%Uppgift 16
\Task 

\Subtask -

\Subtask -

\input{modules/w03-functions-solutions.tex}

\foreach \n in {4,...,9}{%
  \input{modules/w0\n-solutions.tex}
}
\foreach \n in {10,...,14}{%
  \input{modules/w\n-solutions.tex}
}
%

%\chapter{Snabbreferens}\label{chapter:quickref}
%
%Detta appendix innehåller en snabbreferens för Scala och Java. Snabbreferensen är enda tillåtna hjälpmedel under kursens skriftliga tentamen.
%
%Lär dig vad som finns i snabbreferensen så att du snabbt hittar det du behöver och träna på hur du  effektivt kan dra nytta av den när du skriver program med papper och penna utan datorhjälpmedel.
%
%\clearpage
%~
%\clearpage
%
%\includepdf[pages={1-12}, scale=0.77, frame]{../quickref/quickref.pdf}


\end{document}
