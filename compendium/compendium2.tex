%!TEX encoding = UTF-8 Unicode
\documentclass[a4paper]{compendium}
%\usepackage{xr} %to crossreference to compendium1.tex
\externaldocument{compendium1}
\usepackage[swedish]{babel}


\addto\captionsswedish{%
  \renewcommand{\appendixname}{Appendix}%
}
%TODO: Glossary
%http://tex.stackexchange.com/questions/5821/creating-a-standalone-glossary/5837#5837

\setlength{\columnsep}{16mm}

\title{
{\vspace{-3.0cm}\bf\sffamily\Huge\selectfont  Introduktion till programmering med Scala}
\\ \vspace{2em}%\hspace*{1.5cm}\inputgraphics[width=0.6\textwidth]{../img/gurka} \\
{\sffamily \textbf{Kompendium 2}\\Andra läsperioden: Modul 8 -- 14}\\\vspace{2cm}
%\includegraphics[height=4cm]{../img/scala-logo.png}
%\includegraphics[height=4cm]{../img/java-logo.png}
\includegraphics[height=12cm]{cover/gurka.jpg}
}

%\author{Redaktör: Björn Regnell}
\date{\raggedbottom%
\vspace{-2em}\begin{minipage}{1.0\textwidth}\centering
EDAA45, Lp1-2, HT 2016\\
Datavetenskap, LTH\\
Lunds Universitet\\
~\\
Kompileringsdatum: \today \\
\url{http://cs.lth.se/pgk}
\end{minipage}
}

\usepackage{multicol}

\usepackage{pgffor}  %% http://stackoverflow.com/questions/2561791/iteration-in-latex
                     %  allows:  \foreach \n in {1,...,4}{ do something with \n }

\usepackage{framed}  %  allows:   \begin{framed}\end{framed}
%\newenvironment{Slide}[2][]
%  {\begin{framed}\setlist{noitemsep}\section*{#2}}
%  {\end{framed}}

\newcommand{\SlideHeading}[1]{\section*{#1}}

\usepackage[most]{tcolorbox}
\newenvironment{Slide}[2][]
  {\vspace{0.5em}\begin{tcolorbox}[left=1.5em,%width=1.05\textwidth,
  grow to right by=0.05\textwidth,grow to left by=0.05\textwidth,%
  %breakable,
  %frame hidden,
  colframe=gray!20,
  enhanced]\setlist{noitemsep}\SlideHeading{#2}}
  {\end{tcolorbox}\vspace{0.5em}}

\newcommand{\Subsection}[1]{} %ignore slide sections
\newcommand{\SlideOnly}[1]{} %ignore slide font size

\usepackage[framemethod=tikz]{mdframed}

\newcommand{\LibVersion}{1.1.5} % latest version of introlib at https://github.com/lunduniversity/introprog-scalalib
\newcommand{\LibJar}{\texttt{introprog\_3-\LibVersion.jar}}
\newcommand{\JDKApiUrl}{\url{https://docs.oracle.com/en/java/javase/11/docs/api/}}
\newcommand{\CurrentYear}{2021}
\newcommand{\VMName}{vm2020} %TODO: update vm
\newcommand{\VMPassword}{pgkBytMig\CurrentYear}
\newcommand{\VirtualBoxVersion}{6.1} %https://www.virtualbox.org/wiki/Downloads
\newcommand{\UbuntuVersion}{20.04}
\newcommand{\ScalaVersion}{3.0.1} %https://www.scala-lang.org/
\newcommand{\SbtVersion}{1.5.3} %https://eed3si9n.com/category/tags/sbt
\newcommand{\JDKVersion}{11} %https://adoptopenjdk.net/
\newcommand{\KojoVersion}{2.9.10} %https://www.kogics.net/kojo-download
\newcommand{\VSCodeVersion}{1.41} %https://code.visualstudio.com/updates
\newcommand{\MetalsVersion}{v1.10.6} %https://marketplace.visualstudio.com/items?itemName=scalameta.metals
\newcommand{\WindowsVersion}{10}
\newcommand{\ScalaIDEVersion}{4.7.0} %%DEPRECATED





\newif\ifkompendium  % to allow conditional text in slides only showing up in compendium
\kompendiumtrue      % in slides: \kompendiumfalse

\newif\ifPreSolution  % to allow tasks and solutions in same file
\PreSolutiontrue      % in solutions: \PreSolutionfalse

\let\QUESTBEGIN\ifPreSolution  % to mark formatting and numbering of exercises
\let\SOLUTION\else  % to mark solutions in the same file as questions
\let\QUESTEND\fi    % to mark end of exercise

%!TEX encoding = UTF-8 Unicode
\newcommand{\ExeWeekONE}{expressions}
\newcommand{\LabWeekONE}{kojo}

\newcommand{\ExeWeekTWO}{programs}
\newcommand{\LabWeekTWO}{--}

\newcommand{\ExeWeekTHREE}{functions}
\newcommand{\LabWeekTHREE}{irritext}

\newcommand{\ExeWeekFOUR}{objects}
\newcommand{\LabWeekFOUR}{blockmole}

\newcommand{\ExeWeekFIVE}{classes}
\newcommand{\LabWeekFIVE}{turtlegraphics}

\newcommand{\ExeWeekSIX}{sequences}
\newcommand{\LabWeekSIX}{shuffle}

\newcommand{\ExeWeekSEVEN}{sets-maps}
\newcommand{\LabWeekSEVEN}{words}

\newcommand{\ExeWeekEIGHT}{matrices}
\newcommand{\LabWeekEIGHT}{maze}

\newcommand{\ExeWeekNINE}{inheritance}
\newcommand{\LabWeekNINE}{turtlerace-team}

\newcommand{\ExeWeekTEN}{patterns}
\newcommand{\LabWeekTEN}{chords-team}

\newcommand{\ExeWeekELEVEN}{scala-java}
\newcommand{\LabWeekELEVEN}{lthopoly-team}

\newcommand{\ExeWeekTWELVE}{sorting}
\newcommand{\LabWeekTWELVE}{survey}

\newcommand{\ExeWeekTHIRTEEN}{--}
\newcommand{\LabWeekTHIRTEEN}{Projekt}

\newcommand{\ExeWeekFOURTEEN}{threads}
\newcommand{\LabWeekFOURTEEN}{--}



\begin{document}

\pagenumbering{roman}

\frontmatter
\maketitle
%!TEX root = ../compendium.tex

\clearpage\null\thispagestyle{empty}
\vfill

{
\setlength{\parindent}{0pt}
\emph{Editor}: Björn Regnell, Faculty of Engineering LTH, Lund University. \\ 

\emph{Contributors}: 
Björn Regnell,
Per Holm,
Sandra Nilsson,
Patrik Andersson,
Gustav Cedersjö,
Maj Stenmark,
Anna Axelsson,
Roy Andersson,
Markus Borg,
Anton Klarén.
\\

\emph{Repo}: \url{https://github.com/lunduniversity/introprog} \\ \newline

This manuscript is on-going work. Contributions are welcome! \\ 
\emph{Contact}: \url{bjorn.regnell@cs.lth.se}
\\ \newline

\emph{LICENCE}: CC BY-NC-SA 4.0 \\
\url{http://creativecommons.org/licenses/by-nc-sa/4.0/}
\\ \newline
Copyright \copyright~Computer Science, LTH \& Björn Regnell. 2016. Lund. Sweden.\\
}

%!TEX encoding = UTF-8 Unicode
%!TEX root = ../compendium.tex

\ChapterUnnum{Framstegsprotokoll}\label{progress-protocoll}


\section*{Genomförda övningar}

\vspace{1em}\noindent
{Till varje laboration hör en övning med uppgifter som utgör förberedelse inför labben. Du behöver minst behärska grunduppgifterna för att klara labben inom rimlig tid. Om du känner att du behöver öva mer på grunderna, gör då även extrauppgifterna. Om du vill fördjupa dig, gör fördjupningsuppgifterna som är på mer avancerad nivå. Kryssa för nedan vilka övningar du har gjort, så blir det lättare för din handledare att anpassa dialogen till de kunskaper du förvärvat hittills.}

\newcommand{\TickBox}{\raisebox{-.50ex}{\Large$\square$}}
\newcommand{\ExeRow}[1]{\hyperref[section:exe:#1]{\texttt{#1}} & \TickBox  &  \TickBox &  \TickBox  \\ \addlinespace }

\begin{table}[h]
%\centering
\vspace{2em}
\begin{tabular}{lccc}
\toprule \addlinespace
{\sffamily Övning} &
{\sffamily Grund} &
{\sffamily Extra} &
{\sffamily Fördjupning}\\ \addlinespace \midrule \\[-0.7em]
\ExeRow{expressions}
\ExeRow{programs}
\ExeRow{functions}
\ExeRow{data}
\ExeRow{vectors}
\ExeRow{classes}
\ExeRow{traits}
\ExeRow{matching}
\ExeRow{matrices}
\ExeRow{sorting}
\ExeRow{scalajava}
\ExeRow{threads}
\bottomrule
\end{tabular}
\end{table}

\newpage

\section*{Godkända obligatoriska moment}

\vspace{1em}\noindent
För att bli godkänd på laborationsuppgifterna och projektuppgiften måste du lösa deluppgifterna och diskutera dina lösningar med en handledare. Denna diskussion är din möjlighet att få feedback på dina lösningar. Ta vara på den!
Se till att handledaren noterar nedan när du blivit godkänd på respektive obligatorisk moment. Spara detta blad tills du fått slutbetyg i kursen.


\vspace{2.5em}\noindent Namn: \dotfill\\

\vspace{1em}\noindent Namnteckning: \dotfill\\

\newcommand{\LabRow}[1]{\\[-1.1em] \hyperref[section:lab:#1]{\texttt{#1}} & \dotfill &  \dotfill  \\ \addlinespace }

\begin{table}[h]
%\centering
\vspace{1em}
\begin{tabular}{lcc}
\toprule \addlinespace
{\sffamily\bfseries\small Lab} & {\sffamily\small Datum gk} &	
{\sffamily\small Handledares signatur + namnförtydligande}\\ \addlinespace 
%\midrule 
\\[-0.5em]
%!TEX encoding = UTF-8 Unicode
%!TEX root = ../compendium2.tex
\LabRow{kojo}
\LabRow{irritext}
\LabRow{blockmole}
\LabRow{blockbattle}
\LabRow{shuffle}
\LabRow{words}
\LabRow{life}
\LabRow{snake}
\LabRow{tabular}
\LabRow{javatext}
%\toprule
\addlinespace 
%\midrule 
\addlinespace\addlinespace
{\sffamily\small {\bfseries Projektuppgift} (välj en)	} & \dotfill &  \dotfill  \\
\addlinespace\addlinespace %\midrule
{\Large$\square$}\texttt{~~~\hyperref[section:proj:bank]{bank}} &
\multicolumn{2}{c}{\textit{Om egendef., ge kort beskrivning här:}}  \\ \addlinespace
{\Large$\square$}\texttt{~~~\hyperref[section:proj:tabular]{tabular}} \\ \addlinespace
{\Large$\square$}\texttt{~~~\hyperref[section:proj:music]{music}} \\ \addlinespace
{\Large$\square$}\texttt{~~~\hyperref[section:proj:photo]{photo}}  \\ \addlinespace
{\Large$\square$}\texttt{~~~}\textit{egendefinerad}  \\
%\dotfill  \\
\addlinespace\addlinespace
%\midrule
\addlinespace
{\sffamily\small {\bfseries Muntligt prov}} &  & \\
\addlinespace\addlinespace{}
{\Large$\square$}\texttt{~~~} godkänd & \dotfill &  \dotfill \\
\addlinespace\addlinespace\bottomrule
\end{tabular}
\end{table}

%!TEX encoding = UTF-8 Unicode
%!TEX root = ../compendium1.tex


\ChapterUnnum{Förord}

Detta kompendium innehåller övningar och laborationer och övningslösningar för andra läsperioden i LTH:s grundkurs i programmering för civilingenjörsprogrammet Datateknik.


Vi avslutade första läsperioden med en diagnostisk kontrollskrivning där du fick återkoppling på vad du lärt dig hittills. Det är viktigt att du använder dina lärdomar om vad du behöver träna mer på och direkt gör upp en plan för hur du kan befästa din förståelse för begreppen i första läsperioden, så att du hänger med under kommande läsperiod.

Det övergripande målet för den andra läsperioden är att du ska kunna skapa egna program som löser mer omfattande problem än tidigare, genom att kombinera flera abstraktionsmekanismer och begrepp. Vi inför även nya abstraktionsmekanismer (t.ex. arv), nya språkmekanismer (t.ex. mönstermatching), samt jämför och kombinerar Scala och Java. Läsperioden avslutas med ett individuellt projektarbete där du får möjlighet att fördjupa dig enligt dina egna intressen och önskemål.

Kompendiet är framtaget för och av studenter och lärare, och distribueras som öppen källkod. Det får användas fritt så länge erkännande ges och eventuella ändringar publiceras under samma licens som ursprungsmaterialet. På kurshemsidan \href{http://cs.lth.se/pgk}{cs.lth.se/pgk} och i kursrepot \href{http://github.com/lunduniversity/introprog}{github.com/lunduniversity/introprog} finns instruktioner om hur du kan bidra till kursmaterialet.

Välkommen till andra halvlek!

\vspace{1em}\noindent \textit{\hfill Lund, \today, Björn Regnell}


\setcounter{tocdepth}{2} % set headings level in table of contents
\tableofcontents
\mainmatter

\pagenumbering{arabic}


\part{Modulöversikt}

\begin{table}
\noindent\resizebox{1.0\columnwidth}{!}{
\renewcommand{\arraystretch}{2.0}
%!TEX encoding = UTF-8 Unicode
\begin{tabular}{l|l|l|l}
\textit{W} & \textit{Modul} & \textit{Övn} & \textit{Lab} \\ \hline \hline
W01 & Introduktion & expressions & kojo \\
W02 & Kodstrukturer & programs & -- \\
W03 & Funktioner, objekt & functions & blockmole \\
W04 & Datastrukturer & data & pirates \\
W05 & Sekvensalgoritmer & sequences & shuffle \\
W06 & Klasser & classes & turtlegraphics \\
W07 & Arv & traits & turtlerace-team \\
KS & KONTROLLSKRIVN. & -- & -- \\
W08 & Repetition, trösklar, luckor & reboot-init & reboot-check \\
W09 & Mönster, undantag & matching & chords-team \\
W10 & Matriser, typparametrar & matrices & maze \\
W11 & Sökning, sortering & sorting & survey \\
W12 & Scala och Java & scalajava & lthopoly-team \\
W13 & Extra: design, api, trådar, webb & threads & Projekt \\
W14 & Tentaträning & Extenta & -- \\
T & TENTAMEN & -- & -- \\
\end{tabular}

}
\end{table}
\clearpage

\hyphenation{intro-duktion sekvens-algoritmer kod-strukturer data-strukturer}
{\fontsize{11}{12}\selectfont
\renewcommand{\arraystretch}{1.75}
\begin{longtable}{@{}p{.05\textwidth} | >{\hspace{0.1em}\raggedright\bfseries\sffamily}p{.15\textwidth}  >{\raggedleft\arraybackslash\hspace{0.0em}%\fontsize{10.5}{12}\selectfont
}p{0.735\textwidth}}
W01 & Introduktion & sekvens, alternativ, repetition, abstraktion, programmeringsspråk, programmeringsparadigmer, editera-kompilera-exekvera, datorns delar, virtuell maskin, REPL, literal, värde, uttryck, identifierare, variabel, typ, tilldelning, namn, val, var, def, inbyggda grundtyper, Int, Long, Short, Double, Float, Byte, Char, String, println, typen Unit, enhetsvärdet (), stränginterpolatorn s, if, else, true, false, MinValue, MaxValue, aritmetik, slumptal, math.random, logiska uttryck, de Morgans lagar, while-sats, for-sats \\
W02 & Kodstrukturer & iterering, for-uttryck, map, foreach, Range, Array, Vector, algoritm vs implementation, pseudokod, algoritm: SWAP, algoritm: SUM, algoritm: MIN/MAX, algoritm: MININDEX, block, namnsynlighet, namnöverskuggning, lokala variabler, paket, import, filstruktur, jar, dokumentation, programlayout, JDK, main i Java vs Scala, java.lang.System.out.println \\
W03 & Funktioner & definera funktion, anropa funktion, parameter, returtyp, värdeandrop, namnanrop, default-argument, namngivna argument, applicera funktion på alla element i en samling, procedur, värdeanrop vs namnanrop, uppdelad parameterlista, skapa egen kontrollstruktur, funktionsvärde, funktionstyp, äkta funktion, stegad funktion, apply, lazy val, lokala funktioner, anonyma funktioner, lambda, aktiveringspost, anropsstacken, objektheapen, rekursion, cslib.window.SimpleWindow \\
W04 & Objekt & objekt, modul, paket, punktnotation, tillstånd, metod, medlem, funktioner är objekt, cslib.window.SimpleWindow \\
W05 & Klasser & objektorientering, klass, Point, Square, Complex, new, null, this, inkapsling, accessregler, private, private[this], kompanjonsobjekt, getters och setters, klassparameter, primär konstruktor, objektfabriksmetod, överlagring av metoder, referenslikhet vs strukturlikhet, eq vs == \\
W06 & Sekvensalgoritmer & sekvensalgoritm, algoritm: SEQ-COPY, in-place vs copy, algoritm: SEQ-REVERSE, algoritm: SEQ-REGISTER, sekvenser i Java vs Scala, for-sats i Java, java.util.Scanner, scala.collection.mutable.ArrayBuffer, StringBuilder, java.util.Random, slumptalsfrö \\
W07 & Datastrukturer & attribut (fält), medlem, metod, tupel, klass, Any, isInstanceOf, toString, case-klass, samling, scala.collection, föränderlighet vs oföränderlighet, List, Vector, Set, Map, typparameter, generisk samling som parameter, översikt samlingsmetoder, översikt strängmetoder, läsa/skriva textfiler, Source.fromFile, java.nio.file \\
KS & \multicolumn{2}{l}{KONTROLLSKRIVN.} \\
W08 & Matriser, typparametrar & matris, nästlad samling, nästlad for-sats, typparameter, generisk funktion, generisk klass, fri vs bunden typparameter, matriser i Java vs Scala, allokering av nästlade arrayer i Scala och Java \\
W09 & Arv & arv, polymorfism, trait, extends, asInstanceOf, with, inmixning, supertyp, subtyp, bastyp, override, klasshierarkin i Scala: Any AnyRef Object AnyVal Null Nothing, referenstyper vs värdetyper, klasshierarkin i scala.collection, Shape som bastyp till Rectangle och Circle, accessregler vid arv, protected, final, klass vs trait, abstract class, case-object, typer med uppräknade värden, gränssnitt, trait vs interface, programmeringsgränssnitt (api) \\
W10 & Mönster, undantag, likhet & mönstermatchning, match, Option, throw, try, catch, Try, unapply, sealed, flatten, flatMap, partiella funktioner, collect, speciella matchningar: wildcard pattern; variable binding; sequence wildcard; back-ticks, equals, hashcode, exempel: equals för klassen Complex, switch-sats i Java \\
W11 & Scala och Java & syntaxskillnader mellan Scala och Java, klasser i Scala vs Java, referensvariabler vs enkla värden i Java, referenstilldelning vs värdetilldelning i Java, alternativ konstruktor i Scala och Java, for-sats i Java, for-each-sats i Java, java.util.ArrayList, autoboxing i Java, primitiva typer i Java, wrapperklasser i Java, samlingar i Java vs Scala, scala.collection.JavaConverters, namnkonventioner för konstanter \\
W12 & Sökning, sortering, ordning & strängjämförelse, compareTo, implicit ordning, linjärsökning, binärsökning, algoritm: LINEAR-SEARCH, algoritm: BINARY-SEARCH, algoritmisk komplexitet, sortering till ny vektor, sortering på plats, insättningssortering, urvalssortering, algoritm: INSERTION-SORT, algoritm: SELECTION-SORT, Ordering[T], Ordered[T], Comparator[T], Comparable[T] \\
W13 & \multicolumn{2}{l}{Repetition, tentaträning, projekt} \\
W14 & Extra: jämlöpande exekvering & tråd, jämlöpande exekvering, icke-blockerande anrop, callback, java.lang.Thread, java.util.concurrent.atomic.AtomicInteger, scala.concurrent.Future, kort om html+css+javascript+scala.js och webbprogrammering \\
T & \multicolumn{2}{l}{TENTAMEN} \\
\end{longtable}
}

%\renewcommand{\SlideHeading}[1]{\subsection{#1}}  %numbering sections in compendium slides

\part{Moduler}

\setcounter{chapter}{7}

%%!TEX encoding = UTF-8 Unicode

%!TEX root = ../compendium2.tex

\chapter{Mönster, Undantag}\label{chapter:W08}
\begin{itemize}[nosep]
\item match
\item Option
\item null
\item try
\item catch
\item Try
\item unapply
\end{itemize}
\clearpage\section{Teori}
%!TEX encoding = UTF-8 Unicode
%!TEX root = ../lect-w08.tex

%%%

\Subsection{Veckans labb: \texttt{life}}

\begin{Slide}{Veckans labb: \texttt{life}}
\begin{minipage}{0.52\textwidth}
  \setlength{\leftmargini}{0pt}

\begin{itemize}
  \SlideFontSmall
\item Universum är en binär matris av \Emph{celler} där \Emph{levande} celler representeras med \code{true} och \Alert{döda} med \code{false}.
\item Följande regler gäller för \Emph{nästa generation} celler i universum:
\begin{itemize}\SlideFontTiny
  \item \textbf{Fortlevnad}: en levande cell med 2 eller 3 grannar \Emph{lever vidare}
  \item \textbf{Död}: en levande cell med färre än 2 eller fler än 3 grannar \Alert{dör}
  \item \textbf{Födelse}: en död cell med exakt tre grannar föds
\end{itemize}
\item Övning \code{matrices} uppgift 5: skapa en generisk \code{case class Matrix[T]}
\item På labben: använd \code{Matrix[Boolean]}
\end{itemize}

\end{minipage}%
\begin{minipage}{0.5\textwidth}
  \includegraphics[width=1.0\textwidth]{../img/glider-blinker-block}

  \begin{itemize}\SlideFontTiny
  \item Du ska simulera \emph{Game of Life} i ett \code{introprog.PixelWindow}
  \item Fördjupning:\\{\SlideFontTiny\url{https://en.wikipedia.org/wiki/Conway%27s_Game_of_Life}}
  \end{itemize}
\end{minipage}%

\end{Slide}






\Subsection{Matriser}

\begin{Slide}{Vad är en matris?}\SlideFontSmall
\begin{itemize}

\item En \Emph{matris} inom \Alert{matematiken} innehåller \Emph{rader} och \Emph{kolumner}\footnote{även kallade \emph{kolonner}} med tal.

\item I en \Alert{matematisk} matris har alla rader \Emph{lika många} element och

\item även alla kolumner har \Emph{lika många} element.

\item En matris av dimension $2\times{}5$ har $2 \cdot 5 = 10$ stycken element.

\item Exempel på en matematisk matris av dimension $2\times{}5$:
\[
M_{2,5}=
  \begin{pmatrix}
    5 & 2 & 42 & 4 & 5 \\
    3 & 4 & 18 & 6 & 7
  \end{pmatrix}
\]
\end{itemize}
\end{Slide}

\begin{Slide}{Indexering i en matris}\SlideFontSmall
\begin{itemize}

  \item En matris av dimension $m\times{}n$ har $m \cdot n$ stycken element.

  \item En matris $A_{m,n}$ av dimension $m\times{}n$ ritas inom matematiken ofta så här:

  \[
  A_{m,n} =
   \begin{pmatrix}
    a_{1,1} & a_{1,2} & \cdots & a_{1,n} \\
    a_{2,1} & a_{2,2} & \cdots & a_{2,n} \\
    \vdots  & \vdots  & \ddots & \vdots  \\
    a_{m,1} & a_{m,2} & \cdots & a_{m,n}
   \end{pmatrix}
  \]


\item Matrisindexering inom matematiken sker ofta från $1$, men ofta från $0$ i datorprogram.

\item Vad har talet $42$ för index i matrisen $M_{2,5}$ nedan?
\begin{itemize}\SlideFontTiny
  \item[--] Inom matematiken?
  \item[--] I Scala och Java och många andra språk?

  \[
  M_{2,5}=
    \begin{pmatrix}
      5 & 2 & 42 & 4 & 5 \\
      3 & 4 & 18 & 6 & 7
    \end{pmatrix}
  \]
\end{itemize}
\end{itemize}
\end{Slide}

\begin{Slide}{Hur skapa matriser?}
  \setlength{\leftmargini}{0pt}

  \begin{itemize}
  \item Inom programmering används ordet \Emph{matris} ofta för att beteckna en \Alert{nästlad struktur} i två dimensioner. Exempel:
  \begin{itemize}
   \item \Emph{Oföränderliga} sekvenser, t.ex. \code{Vector[Vector[Int]]} \\
   \code{val xss = Vector(Vector(0, 0, 0), Vector(0, 0, 0))} eller enklare: \\
      \code{val xss = Vector.fill(2,3)(0)}

    \item \Alert{Föränderliga} sekvens, t.ex. \code{Array[Array[Int]]} \\
    \code{val yss = Array(Array(0, 0, 0), Array(0, 0, 0))} eller enklare: \\
       \code{val yss = Array.fill(2,3)(0)}

  \end{itemize}

\end{itemize}
\end{Slide}

\begin{Slide}{Hur indexera i matriser?}
En matris med array av arrayer:
\begin{REPL}
scala> val xss = Array(Array(5,2,42,4,5),Array(3,4,18,6,7))
xss: Array[Array[Int]] = Array(Array(5, 2, 42, 4, 5), Array(3, 4, 18, 6, 7))
\end{REPL}
\pause
Man indexerar i en nästlad sekvens med upprepad \code{apply}:
\begin{REPL}
scala> xss(0)(2)
res0: ???

scala> xss.apply(0).apply(2)
res1: ???

scala> xss(0)
res2: ???
\end{REPL}
Övning: Vad är typ och värde vid \code{???} ovan?
\end{Slide}

\begin{Slide}{Hur indexera i matriser?}
En matris med array av arrayer:
\begin{REPL}
scala> val xss = Array(Array(5,2,42,4,5),Array(3,4,18,6,7))
xss: Array[Array[Int]] = Array(Array(5, 2, 42, 4, 5), Array(3, 4, 18, 6, 7))
\end{REPL}

Man indexerar i en nästlad sekvens med upprepad \code{apply}:
\begin{REPL}
scala> xss(0)(2)
res0: Int = 42

scala> xss.apply(0).apply(2)
res1: Int = 42

scala> xss(0)
res2: Array[Int] = Array(5, 2, 42, 4, 5)
\end{REPL}
Övning: Rita en bild av minnet som referensen \code{xss} refererar till.

\end{Slide}

\begin{Slide}{Uppdatering av en förändringsbar nästlad struktur}
Man kan förändra en array av arrayer ''på plats'' med tilldelning:
\begin{REPL}
scala> val xss = Array(Array(5,2,42,4,5),Array(3,4,18,6,7))

scala> xss(0)(0) = 100

scala> xss
res0: ???

scala> xss(0)(2) = xss(0)(2) - 1

scala> xss
res1: ???

scala> xss(1) = Array.fill(5)(-1)

scala> xss
res2: ???
\end{REPL}
\end{Slide}

\begin{Slide}{Uppdatering av en förändringsbar nästlad struktur}
Man kan förändra en array av arrayer ''på plats'' med tilldelning:
\begin{REPL}
scala> val xss = Array(Array(5,2,42,4,5),Array(3,4,18,6,7))

scala> xss(0)(0) = 100

scala> xss
res0: Array[Array[Int]]=Array(Array(100, 2, 42, 4, 5), Array(3, 4, 18, 6, 7))

scala> xss(0)(2) = xss(0)(2) - 1

scala> xss
res1: Array[Array[Int]]=Array(Array(100, 2, 41, 4, 5), Array(3, 4, 18, 6, 7))

scala> xss(1) = Array.fill(5)(-1)

scala> xss
res2: Array[Array[Int]]=Array(Array(100, 2, 41, 4, 5), Array(-1,-1,-1,-1,-1))
\end{REPL}
\end{Slide}

\begin{Slide}{Några olika sätt att skapa förändringsbara matriser}\SlideFontSmall
Det jobbiga, primitiva sättet:
\begin{REPL}
scala> val xss = new Array[Array[Int]](2)
xss: Array[Array[Int]] = Array(null, null)

scala> for (i <- xss.indices) {xss(i) = new Array[Int](5)}

scala> xss
res0: Array[Array[Int]] = Array(Array(0, 0, 0, 0, 0), Array(0, 0, 0, 0, 0))

scala> println(xss)
[[I@196a99d0
\end{REPL}
Enklare sätt:
\begin{REPL}
scala> val xss = Array.ofDim[Int](2,5)
xss: Array[Array[Int]] = Array(Array(0, 0, 0, 0, 0), Array(0, 0, 0, 0, 0))
\end{REPL}
Enklare och tydligare sätt, där initialvärdet anges explicit:
\begin{REPL}
scala> val xss = Array.fill(2,5)(0)
xss: Array[Array[Int]] = Array(Array(0, 0, 0, 0, 0), Array(0, 0, 0, 0, 0))
\end{REPL}

\end{Slide}

\begin{Slide}{Exempel på skapande av oföränderlig nästlad struktur}\SlideFontSmall
Om du kan beräkna initialvärde direkt, använd \code{Vector.fill}:\\
{\SlideFontTiny\code{def fill[A](n1: Int, n2: Int)(elem: => A): Vector[Vector[A]]}}
\begin{REPL}
scala> Vector.fill(2,5)(scala.util.Random.nextInt(6) + 1)
res0:
  typ???
  värde???

\end{REPL}
Om du kan beräkna initialvärde ur index, använd \code{Vector.tabulate}:\\
{\SlideFontTiny\code{def tabulate[A](n1: Int, n2: Int)(f: (Int, Int) => A): Vector[Vector[A]]}}
\begin{REPL}
scala> Vector.tabulate(5,2)((x,y) => x + y + 1)
res1:
  typ???
  värde???

\end{REPL}
\end{Slide}

\begin{Slide}{Exempel på skapande av oföränderlig nästlad struktur}\SlideFontSmall
Om du kan beräkna initialvärde direkt, använd \code{Vector.fill}:\\
{\SlideFontTiny\code{def fill[A](n1: Int, n2: Int)(elem: => A): Vector[Vector[A]]}}
\begin{REPL}
scala> Vector.fill(2,5)(scala.util.Random.nextInt(6) + 1)
res0: Vector[Vector[Int]] =
  Vector(Vector(1, 2, 6, 2, 1), Vector(1, 4, 3, 3, 2))

\end{REPL}
Om du kan beräkna initialvärde ur index, använd \code{Vector.tabulate}:\\
{\SlideFontTiny\code{def tabulate[A](n1: Int, n2: Int)(f: (Int, Int) => A): Vector[Vector[A]]}}
\begin{REPL}
scala> Vector.tabulate(5,2)((x,y) => x + y + 1)
res1: Vector[Vector[Int]] =
  Vector(Vector(1,2), Vector(2,3), Vector(3,4), Vector(4,5), Vector(5,	6))

\end{REPL}
\end{Slide}



\begin{Slide}{Uppdatering av en oföränderlig nästlad struktur}\SlideFontSmall
Uppdatering av endimensionell struktur med \code{xs.updated}:\\
{\SlideFontTiny\code{def updated[A](index: Int, elem: A): Vector[A]} }
\begin{REPL}
scala> var xs = Vector.tabulate(5)(x => x + 1)
xs: typ??? = värde???

scala> xs = xs.updated(1, 42)
xs: typ??? = värde???
\end{REPL}

Uppdatering av nästlad struktur i två dimensioner:
\begin{REPL}
scala> var xss = Vector.tabulate(2, 5)((x,y) => x + y + 1)
xss:
  typ??? =
  värde???

scala> xss = xss.updated(0, xss(0).updated(1, 42))
xss:
  typ??? =
  värde???
\end{REPL}

\end{Slide}



\begin{Slide}{Uppdatering av en oföränderlig nästlad struktur}\SlideFontSmall
Uppdatering av endimensionell struktur med \code{xs.updated}:\\
{\SlideFontTiny\code{def updated[A](index: Int, elem: A): Vector[A]} }
\begin{REPL}
scala> var xs = Vector.tabulate(5)(x => x + 1)
xs: Vector[Int] = Vector(1, 2, 3, 4, 5)

scala> xs = xs.updated(1, 42)
xs: Vector[Int] = Vector(1, 42, 3, 4, 5)
\end{REPL}

Uppdatering av nästlad struktur i två dimensioner:
\begin{REPL}
scala> var xss = Vector.tabulate(2, 5)((x,y) => x + y + 1)
xss: Vector[Vector[Int]] =
  Vector(Vector(1, 2, 3, 4, 5), Vector(2, 3, 4, 5, 6))

scala> xss = xss.updated(0, xss(0).updated(1, 42))
xss:
  Vector[Vector[Int]] =
  Vector(Vector(1, 42, 3, 4, 5), Vector(2, 3, 4, 5, 6))
\end{REPL}

\end{Slide}


\begin{Slide}{Iterera över nästlad struktur}\SlideFontSmall
Behandling av nästlade strukturer kräver ofta algoritmer med nästlad iterering. \\
Exempel: iterera med nästlad \code{for}-sats för utskrift av denna matris\\
\code{val xss = Vector.tabulate(2,5)((x,y) => x + y + 1)}
\pause
\begin{REPL}
scala> for ??? do
         for ??? do 
           print(xss(i)(j))
           print(" ")
         println

1 2 3 4 5
2 3 4 5 6
\end{REPL}
Övning: \\Vad ska det stå vid \code{???} för att alla element ska skrivas ut?
\end{Slide}

\begin{Slide}{Iterera över nästlad struktur}\SlideFontSmall
  \vspace{1em}
  Behandling av nästlade strukturer kräver ofta algoritmer med nästlad iterering. \\
  Exempel: iterera med nästlad \code{for}-sats för utskrift av denna matris \\
  \code{val xss = Vector.tabulate(2,5)((x,y) => x + y + 1)}

  \begin{REPL}
scala> for xs <- xss do
         for x <- xs do 
           print(x)
           print(" ")
         end for
         println()
       end for

1 2 3 4 5
2 3 4 5 6
\end{REPL}
Övning: skriv ut matrisen med nästlad \code{foreach}\\
\pause
\begin{Code}
xss.foreach { xs => 
  xs.foreach { x => print(x); print(" ") }
  println()
}
\end{Code}
\end{Slide}


\begin{Slide}{Övningsexempel: Yatzy}\SlideFontSmall
Skapa en funktion \code{roll} som ger utfallet av n st tärningskast:
\begin{REPL}
scala> import scala.util.Random

scala> def roll(n: Int): Vector[Int] = ???
\end{REPL}

Skapa en funktion \code{isYatzy} som ger \code{true} om alla utfall är lika:
\begin{REPL}
scala> def isYatzy(xs: Vector[Int]): Boolean = ???
\end{REPL}
Du kan anta att xs.length > 0\\
Tips: använd metoden xs.forall: \\
\code{def forall[A](p: A => Boolean): Boolean }
\end{Slide}


\begin{Slide}{Övningsexempel: Yatzy}\SlideFontSmall
Skapa en funktion \code{roll} som ger utfallet av n st tärningskast:
\begin{REPL}
scala> import scala.util.Random

scala> def roll(n: Int): Vector[Int] = Vector.fill(n)(Random.nextInt(6) + 1)
\end{REPL}

Skapa en funktion \code{isYatzy} som ger \code{true} om alla utfall är lika:
\begin{REPL}
scala> def isYatzy(xs: Vector[Int]): Boolean = xs.forall(x => x == xs(0))
\end{REPL}
Du kan anta att xs.length > 0\\
Tips: använd metoden xs.forall: \\
\code{def forall[A](p: A => Boolean): Boolean }
\end{Slide}

\begin{Slide}{Iterera över nästlad struktur: for-sats}\SlideFontSmall
Iterera med nästlad for-sats: (vad har xss för typ?)
\begin{REPL}
scala> val xss = Vector.fill(100)(roll(5))

scala> for (i <- ???) do 
         for (j <- ???) do
           print(s"($i)($j): ${xss(i)(j)} ")
         println(s" YATZY: ${isYatzy(xss(i))}")

(0)(0): 3 (0)(1): 6 (0)(2): 4 (0)(3): 4 (0)(4): 6  YATZY: false
(1)(0): 4 (1)(1): 1 (1)(2): 5 (1)(3): 2 (1)(4): 6  YATZY: false
(2)(0): 1 (2)(1): 3 (2)(2): 5 (2)(3): 6 (2)(4): 2  YATZY: false
(3)(0): 2 (3)(1): 1 (3)(2): 1 (3)(3): 5 (3)(4): 4  YATZY: false
(4)(0): 4 (4)(1): 4 (4)(2): 1 (4)(3): 6 (4)(4): 5  YATZY: false
(5)(0): 3 (5)(1): 3 (5)(2): 2 (5)(3): 3 (5)(4): 6  YATZY: false
(6)(0): 3 (6)(1): 6 (6)(2): 1 (6)(3): 1 (6)(4): 4  YATZY: false
(7)(0): 6 (7)(1): 2 (7)(2): 4 (7)(3): 4 (7)(4): 3  YATZY: false
(8)(0): 1 (8)(1): 5 (8)(2): 4 (8)(3): 2 (8)(4): 4  YATZY: false
(9)(0): 1 (9)(1): 1 (9)(2): 3 (9)(3): 6 (9)(4): 6  YATZY: false
(10)(0): 2 (10)(1): 5 (10)(2): 2 (10)(3): 4 (10)(4): 5  YATZY: false
(11)(0): 3 (11)(1): 4 (11)(2): 2 (11)(3): 5 (11)(4): 6  YATZY: false
...
\end{REPL}
\end{Slide}

\begin{Slide}{Iterera över nästlad struktur: for-sats}\SlideFontSmall
Iterera med nästlad for-sats: (xss är en \code{Vector[Vector[Int]]})
\begin{REPL}
scala> val xss = Vector.fill(100)(roll(5))

scala> for (i <- xss.indices) do 
         for (j <- xss(i).indices) do
           print(s"($i)($j): ${xss(i)(j)} ")
         println(s" YATZY: ${isYatzy(xss(i))}")

(0)(0): 3 (0)(1): 6 (0)(2): 4 (0)(3): 4 (0)(4): 6  YATZY: false
(1)(0): 4 (1)(1): 1 (1)(2): 5 (1)(3): 2 (1)(4): 6  YATZY: false
(2)(0): 1 (2)(1): 3 (2)(2): 5 (2)(3): 6 (2)(4): 2  YATZY: false
(3)(0): 2 (3)(1): 1 (3)(2): 1 (3)(3): 5 (3)(4): 4  YATZY: false
(4)(0): 4 (4)(1): 4 (4)(2): 1 (4)(3): 6 (4)(4): 5  YATZY: false
(5)(0): 3 (5)(1): 3 (5)(2): 2 (5)(3): 3 (5)(4): 6  YATZY: false
(6)(0): 3 (6)(1): 6 (6)(2): 1 (6)(3): 1 (6)(4): 4  YATZY: false
(7)(0): 6 (7)(1): 2 (7)(2): 4 (7)(3): 4 (7)(4): 3  YATZY: false
(8)(0): 1 (8)(1): 5 (8)(2): 4 (8)(3): 2 (8)(4): 4  YATZY: false
(9)(0): 1 (9)(1): 1 (9)(2): 3 (9)(3): 6 (9)(4): 6  YATZY: false
(10)(0): 2 (10)(1): 5 (10)(2): 2 (10)(3): 4 (10)(4): 5  YATZY: false
(11)(0): 3 (11)(1): 4 (11)(2): 2 (11)(3): 5 (11)(4): 6  YATZY: false
...
\end{REPL}
\end{Slide}


% \begin{Slide}{Iterera över nästlad struktur med nästlad foreach}\SlideFontSmall
% Iterera med nästlad foreach-sats:
% \begin{REPL}
% scala> val xss = Vector.tabulate(2,5)((x,y) => x + y + 1)

% xss.foreach{ xs => ??? ; println }

% 1 2 3 4 5
% 2 3 4 5 6
% \end{REPL}
% \end{Slide}


% \begin{Slide}{Iterera över nästlad struktur med nästlad foreach}\SlideFontSmall
% Iterera med nästlad foreach-sats:
% \begin{REPL}
% scala> val xss = Vector.tabulate(2,5)((x,y) => x + y + 1)

% xss.foreach{ xs => xs.foreach{ x => print(x + " ") }; println }

% 1 2 3 4 5
% 2 3 4 5 6
% \end{REPL}
% \end{Slide}


\begin{Slide}{Nästlade for-uttryck}\SlideFontSmall
Iterera med \Emph{nästlad for-yield}:\\
%Statisk typ: \code{IndexedSeq[IndexedSeq[[Int]]} \\
%Dynamisk typ: \code{Vector[Vector[[Int]]}

\begin{REPL}
scala> val xss = for (i <- 1 to 2) yield 
                   for (j <- 1 to 5) yield i + j + 1
                 
val xss: IndexedSeq[IndexedSeq[Int]] =
      ???

\end{REPL}
\pause Om man skriver så här får man en endimensionell struktur:
\begin{REPL}
scala> val xs = for (i <- 1 to 2; j <- 1 to 5) yield i + j + 1
val xs: IndexedSeq[Int] =
    ???

\end{REPL}
\end{Slide}

\begin{Slide}{Nästlade for-uttryck}\SlideFontSmall
Iterera med \Emph{nästlad for-yield}:\\
\begin{REPL}
scala> val xss = for (i <- 1 to 2) yield {
                   for (j <- 1 to 5) yield i + j + 1
                 }
val xss: IndexedSeq[IndexedSeq[Int]] =
    Vector(Vector(3, 4, 5, 6, 7), Vector(4, 5, 6, 7, 8))

\end{REPL}
\pause Om man skriver så här får man en endimensionell struktur:
\begin{REPL}
scala> val xs = for (i <- 1 to 2; j <- 1 to 5) yield i + j + 1
val xs: IndexedSeq[Int] =
    Vector(3, 4, 5, 6, 7, 4, 5, 6, 7, 8)

\end{REPL}
\end{Slide}



\begin{Slide}{Nästlade map-uttryck}\SlideFontSmall
Iterera med \Emph{nästlade map-uttryck}:\\
\begin{REPL}
scala> val xss = (1 to 2).map(i => (1 to 5).map(j => i + j + 1))
xss: IndexedSeq[IndexedSeq[Int]] =
      ???
\end{REPL}
\end{Slide}

\begin{Slide}{Nästlade map-uttryck}\SlideFontSmall
Iterera med \Emph{nästlade map-uttryck}:\\
\begin{REPL}
scala> val xss = (1 to 2).map(i => (1 to 5).map(j => i + j + 1))
xss: IndexedSeq[IndexedSeq[Int]] =
      Vector(Vector(3, 4, 5, 6, 7), Vector(4, 5, 6, 7, 8))
\end{REPL}
\end{Slide}



\ifkompendium\else
\begin{Slide}{Fallgrop: likhet av array}
\begin{REPL}
scala> Vector.fill(5, 2)(42) == Vector.fill(5, 2)(42)
val res0: ???

scala> Array.fill(5, 2)(42) == Array.fill(5, 2)(42)
val res1: ???
\end{REPL}
\end{Slide}
\fi

\begin{Slide}{Fallgrop: likhet av array}
\begin{REPL}
scala> Vector.fill(5, 2)(42) == Vector.fill(5, 2)(42)
val res0: Boolean = true

scala> Array.fill(5, 2)(42) == Array.fill(5, 2)(42)
val res1: Boolean = false  // AAAARRGH!!! :(
\end{REPL}
Primitiva arrayer har en equals-metod som ger referenslikhet, \Alert{inte} innehållslikhet. Och det fungerar följaktligen ej heller på nästlade strukturer. 
\end{Slide}

\ifkompendium\else
\begin{Slide}{Övning: Kolla likhet av array (uppfinner hjulet)}
\begin{Code}
def isEqual(xss: Array[Array[Int]], yss: Array[Array[Int]]) = 
  var i = 0
  var foundUnequal = false
  while ??? do                          // VILKET VILLKOR?
    var j = 0
    while ??? do                        // VILKET VILLKOR?
      if xss(i)(j) != yss(i)(j) then ???   // VAD SKA UPPDATERAS? 
      j += 1
    end while
    i += 1
  end while
  !foundUnequal
end isEqual
\end{Code}
\begin{REPL}
scala> val (xss, yss) = (Array.fill(5,2)(42), Array.fill(5,2)(42))

scala> isEqual(xss, yss)

scala> yss(4)(1) = 0

scala> isEqual(xss, yss)
\end{REPL}
\end{Slide}
\fi


\begin{Slide}{Övning: Kolla likhet av array (uppfinner hjulet)}
\begin{Code}
def isEqual(xss: Array[Array[Int]], yss: Array[Array[Int]]) = 
  var i = 0
  var foundUnequal = false
  while i < xss.length && !foundUnequal do
    var j = 0
    while j < xss(i).length && !foundUnequal do
      if xss(i)(j) != yss(i)(j) then foundUnequal = true
      j += 1
    end while
    i += 1
  end while
  !foundUnequal
end isEqual
\end{Code}
\begin{REPL}
scala> val (xss, yss) = (Array.fill(5,2)(42), Array.fill(5,2)(42))

scala> isEqual(xss, yss)  // true

scala> yss(4)(1) = 0

scala> isEqual(xss, yss)  // false
\end{REPL}
\end{Slide}

\begin{Slide}{Använd \texttt{sameElements} för test av innehållslikhet men bara på icke-nästlade arrayer}

  I Scala kan du använda metoden \code{sameElements} på arrayer för innehållslikhet, men det funkar \Alert{INTE} på nästlade strukturer.

\begin{REPL}
scala> val xs = Array(1,2,3)
xs: Array[Int] = Array(1, 2, 3)

scala> val ys = Array(1,2,3)
ys: Array[Int] = Array(1, 2, 3)

scala> xs.sameElements(ys)
res0: Boolean = true

scala> Array(Array(1)) sameElements Array(Array(1))  
res1: Boolean = false

\end{REPL}
\pause Använd i stället: \code{java.util.Arrays.deepEquals(xs, ys)}\\
men det kan då behövas \code{.asInstanceOf[Array[Object]]} på argumenten om kompilatorn inte klarar typkonverteringen.
\end{Slide}

% \begin{Slide}{Matris som Array med Array med heltal i Java}\SlideFontTiny
% \begin{CodeSmall}[language=Java]
% public class ArrayMatrix {

%     public static void showMatrix(int[][] m){
%         System.out.println("\n--- showMatrix ---");
%         for (int row = 0; row < m.length; row++){
%             for (int col = 0; col < m[row].length; col++) {
%                 System.out.print("[" + row + "]");
%                 System.out.print("[" + col + "] = ");
%                 System.out.print(m[row][col] + "; ");
%             }
%             System.out.println();
%         }
%     }

%     public static void main(String[] args) {
%         int[][] xss = new int[10][5];
%         showMatrix(xss);
%     }
% }
% \end{CodeSmall}
% \pause
% Övning: skriv en metod \code{fillRnd} som fyller en heltalsmatris med slumptal 1 till n:\\
% \pause
% \jcode|public static void fillRnd(int[][] m, int n){ /* ??? */ }| \\
% \pause
% Tips: använd en nästlad for-sats och detta uttryck: \\
% \jcode{(int) (Math.random() * n + 1) // (int) motsvarar Scalas asInstanceOf[Int]}

% \end{Slide}

\begin{Slide}{Om veckans övningar}\SlideFontSmall
\begin{itemize}
\item Träna på att iterera över nästlade strukurer

\item Fortsätt jobba med Yatzy-exemplet

\item träna på att skapa \Emph{imperativa} algoritmer: \\
lös \code{isYatzy} med \code{while}-sats 

\item Extrauppgift där du ska bygga ett enkelt yatzy-spel i terminalen (kunde varit en tentauppgift...)

\end{itemize}
\end{Slide}

% \begin{Slide}{Övning extrauppgift, utgör början på labb \code{survey}}\SlideFontSmall
%
% \begin{ScalaSpec}{Table}
% object Table {
%   /** Creates a new Table from fileName with columns split by sep */
%   def fromFile(fileName: String, separator: Char = ';'): Table = ???
% }
% case class Table(
%   data: Vector[Vector[String]],
%   headings: Vector[String],
%   sep: String){
%   /** A 2-tuple with (number of rows, number of columns) in data */
%   val dim: (Int, Int) = ???
%
%   /** The element in row r an column c of data, counting from 0 */
%   def apply(r: Int, c: Int): String = ???
%
%   /** The row-vector r in data, counting from 0 */
%   def row(r: Int): Vector[String]= ???
%
%   /** The column-vector c in data, counting from 0 */
%   def col(c: Int): Vector[String] = ???
%
%   /** A map from heading to index counting from 0 */
%   lazy val indexOfHeading: Map[String, Int] = ???
%
%   /** The column-vector with heading h in data */
%   def col(h: String): Vector[String] = ???
%
%   /** A vector with the distinct, sorted values of col with heading h */
%   def values(h: String): Vector[String] = ???
%
%   /** Headings and data with columns separated by sep */
%   override lazy val toString: String = ???
% }
% \end{ScalaSpec}
% \end{Slide}


% \begin{Slide}{Övn. fördjupn. uppg.: skapa en generisk matris-klass}\SlideFontSmall
% \vspace{-0.7em}
% \begin{Code}[basicstyle=\SlideFontSize{6}{6.8}\ttfamily\selectfont]
% case class Matrix[T](data: Vector[Vector[T]]){
%
%   def foreachRowCol(f: (Int, Int, T) => Unit): Unit =
%     for (r <- data.indices) {
%       for (c <- data(r).indices) {
%         f(r, c, data(r)(c))
%       }
%     }
%
%   def map[U](f: T => U): Matrix[U] = Matrix(data.map(_.map(f)))
%
%   /** The element at row r and column c */
%   def apply(r: Int, c: Int): T = ???
%
%   /** Gives Some[T](element) at index (r, c) if within index bounds, else None */
%   def get(r: Int, c: Int): Option[T] = ???
%
%   /** The row vector of row r */
%   def row(r: Int): Vector[T] = ???
%
%   /** The column vector of column c */
%   def col(c: Int): Vector[T] = ???
%
%   /** A new Matrix with element at row r and col c updated */
%   def updated(r: Int, c: Int, value: T): Matrix[T] = ???
% }
% object Matrix {
%   def fill[T](rowSize: Int, colSize: Int)(init: T): Matrix[T] =
%     new Matrix(Vector.fill(rowSize)(Vector.fill(colSize)(init)))
% }
% \end{Code}
% \end{Slide}

%!TEX encoding = UTF-8 Unicode
%!TEX root = ../lect-w08.tex

\Subsection{Typparametrar}



\begin{Slide}{Exempel: Icke-generisk case-klass med heltalsmatris}
  En \emph{icke-generisk} datastruktur har inga obundna typparametrar; alla typer är \Emph{konkreta} (alltså specifika). \\~\\ En icke-generisk case-class med en \code{Vector[Vector[Int]]}:
  \begin{Code}
  case class Matrix(data: Vector[Vector[Int]]):
    def apply(x: Int, y: Int): Int = data(x)(y)
  \end{Code}

  \begin{REPL}
  scala> Matrix(Vector(Vector(5, 2, 42, 4, 5),Vector(3, 4, 18, 6, 7)))
  res0: Matrix =
    Matrix(Vector(Vector(5, 2, 42, 4, 5), Vector(3, 4, 18, 6, 7)))
  \end{REPL}

\end{Slide}





\begin{Slide}{Exempel: Generisk case-klass med generell matris}
  En \emph{generisk} datastruktur har minst en obunden \Emph{typparameter} som kan bindas  till ett \Alert{konkret} \Emph{typargument}.
  
  \begin{Code}
  case class Matrix[T](data: Vector[Vector[T]]):
    def apply(x: Int, y: Int): T = data(x)(y)
  \end{Code}
  \code{Matrix} i exemplet ovan är en \Emph{generisk} case-class där \code{T} är obunden, eftersom \code{T} är en typparameter deklarerad inom \code{[]} \Alert{efter} klassens namn men \Alert{före} klassparameterlistan. \\

  \vspace{0.5em} Användning där \code{T} binds till \code{Int} via kompilatorns typhärledning:
  \begin{REPL}
  scala> Matrix(Vector(Vector(5, 2, 42, 4, 5),Vector(3, 4, 18, 6, 7)))
  res1: Matrix[Int] =
    Matrix(Vector(Vector(5, 2, 42, 4, 5), Vector(3, 4, 18, 6, 7)))
  \end{REPL}

\end{Slide}




\begin{Slide}{Vad är en typparameter?}\SlideFontSmall
  \setlength{\leftmargini}{0pt}

\begin{itemize}
\item En \Emph{typparameter} gör det möjligt att ge ett \Emph{typargument}.
\item Detta kallas \Emph{parametrisk polymorfism} \Eng{paramteric polymorphism}.
\item Exempel: \Emph{generisk} \Alert{funktion}:
\begin{Code}
def tnirp[A](x: A):Unit = println(x.toString.reverse)
\end{Code}
\pause
\item En \Emph{fri} typparameter kan bindas till vilken typ som helst.
\item Bindingen av typargument till typparametrar sker vid \Alert{kompileringstid}.
\item En typparameter är \Emph{fri} om den \Alert{inte} fått något värde, annars \Emph{bunden}. 
\pause
\item Exempel: \Emph{generisk} \Alert{klass} med \Emph{generiska} \Alert{metoder}:
\begin{Code}
class Cell[A](   // [A] är fri (måste bindas vid användning) 
    var value: A):                              // A är bunden
  def update(a: A): Unit = value = a            // A är bunden
  def replaced[B](b: B): Cell[B] = new Cell(b)  // första [B] är fri
\end{Code}
\pause
\item \Alert{Skuggning kan förekomma}: Om \code{replaced} i \code{Cell} hade använt namnet A på sin typparameter hade den \Emph{skuggat} klassens typparameter och tolkats som en  fri typparameter, alltså en godtycklig typ och \Alert{inte} klassens typparameter. (jämför  namnöverskuggning vid \Emph{lokala} namn i nästlade block)
\end{itemize}

\end{Slide}

\ifkompendium\else
\begin{Slide}{Exempel: Generisk funktion}
Vad händer här?
\begin{REPL}

scala> def skrikBaklänges(x: T): String = x.toString.toUpperCase.reverse
???



scala> def skrikBaklänges[T](x: T): String = x.toString.toUpperCase.reverse

scala> skrikBaklänges("gurka är gott")
val res0: ???

\end{REPL}
\end{Slide}


\begin{Slide}{Exempel: Generisk funktion}
Vad händer här?
\begin{REPL}

scala> def skrikBaklänges(x: T): String = x.toString.toUpperCase.reverse
1 |def skrikBaklänges(x: T): String = x.toString.toUpperCase.reverse
  |                      ^
  |                      Not found: type T
                             ^

scala> def skrikBaklänges[T](x: T): String = x.toString.toUpperCase.reverse

scala> skrikBaklänges("gurka är gott")
val res0: ???
\end{REPL}
\end{Slide}
\fi

\begin{Slide}{Exempel: Generisk funktion}
Vad händer här?
\begin{REPL}

scala> def skrikBaklänges(x: T): String = x.toString.toUpperCase.reverse
1 |def skrikBaklänges(x: T): String = x.toString.toUpperCase.reverse
  |                      ^
  |                      Not found: type T
                             ^

scala> def skrikBaklänges[T](x: T): String = x.toString.toUpperCase.reverse

scala> skrikBaklänges("gurka är gott")
val res0: String = TTOG RÄ AKRUG
\end{REPL}
Om ingen typparameter deklareras inom hakparenteser efter funktionens namns så vet inte kompilatorn vad \code{T} är för en typ. Men med en typparameter \code{[T]} efter funktionsnamnet tolkar kompilatorn funktionen som \Emph{generisk} och typen \code{T} bestäms av argumentets typ \Alert{vid anrop} och \code{T} kan bindas till godtycklig typ.
\end{Slide}


\begin{Slide}{Exempel: Generisk case-klass}
\SlideFontSmall
En generisk klass har en eller flera typparametrar efter klassnamnet:
\begin{Code}
case class Box[A](value: A)  
\end{Code}

Kompilatorn härleder typparameterarnas typ utifrån givna värden. 
\begin{REPL}
scala> Box("gurka")  
val res1: Box[String] = Box(gurka)
\end{REPL}

Du kan också ge typpparametern en typ explicit:
\begin{REPL}
scala> Box[Int](42)  // 
val res3: Box[Int] = Box(42)
\end{REPL}

Om typen inte stämmer får du hjälp av kompilatorn att hitta felet:
\begin{REPL}
scala> Box[String](42)
-- Error:
1 |Box[String](42)
  |            ^^
  |            Found:    (42 : Int)
  |            Required: String
\end{REPL}
\end{Slide}






\begin{Slide}{Fallgrop: Typradering \Eng{type erasure}}\SlideFontSmall
Informationen om typerna i typparametrar raderas innan kodgenerering för JVM av prestandaskäl och \Alert{typparametrar saknas vid runtime} i bytekoden.
\vspace{-0.25em}\begin{REPL}
scala> def isIntVector[T](xs: Vector[T]) = xs.isInstanceOf[Vector[Int]]
-- Warning:
1 |def isIntVector[T](xs: Vector[T]) = xs.isInstanceOf[Vector[Int]]
  |                                    ^^^^^^^^^^^^^^^^^^^^^^^^^^^^
  |                the type test for Vector[Int] cannot be checked at runtime
def isIntVector[T](xs: Vector[T]): Boolean

scala> isIntVector(Vector("hej"))
res42: Boolean = true  // AAAARGHH!! :(
\end{REPL}
Måste ''packa upp'' samlingen och typtesta alla element:
\begin{REPL}
scala> def isIntVector[T](xs: Vector[T]) = xs.forall(_.isInstanceOf[Int])

scala> isIntVector(Vector("hej"))
res43: Boolean = false  // FUNKAR :)

\end{REPL}
Typkontroll vid körtid görs oftast hellre med \code{match}.

\end{Slide}

\Subsection{Upptäcka och åtgärda buggar}

\begin{Slide}{Testning och avlusning}
%\TODO 
\begin{itemize}
\item Läs om testning och avlusning \Eng{debugging} i Appendix D: ''Fixa buggar'' 
\item Träna på println-debugging
\item Prova debuggern i VS code
%\item Visa hur testramverket ska funka som du ska skapa på övning och använda på labb
%\item sbt testOnly och andra sätt att köra testfall
%\item Visa hur en fördröjning kan skapas med en s.k. thunk 
%\item Visa hur printlndebugging
%\item Visa hur debugga i vs code
\end{itemize}
\end{Slide}


% \ifkompendium\else


% \begin{Slide}{Typparametrar på tentan?}
% \begin{itemize}
% \item Det ingår att kunna använda färdiga generiska strukturer med specifika typer, t.ex. \code{Vector[Int]}

% \item Det ingår att kunna skapa abstraktioner med specifika typparametrar, t.ex. metoder eller klasser som tar en vektor med en specifik typ som parameter:\\
% \code{case class X(x: Vector[Int])}


% \item Det ingår \Alert{inte} på tentan att kunna skapa generiska metoder eller klasser, t.ex.: \\
% \code{def f[T](x: Vector[T]): Vector[T] = ???} \\
% Mer om generiska strukturer i fördjupningskursen!
% \end{itemize}
% \end{Slide}

% \fi

%!TEX encoding = UTF-8 Unicode
%!TEX root = ../lect-w08.tex

\Subsection{Upptäcka och åtgärda buggar}

\begin{Slide}{Debugging -- Appendix D}
\begin{itemize}
\item Läs om testning och avlusning \Eng{debugging} i Appendix D: ''Fixa buggar'' 
\item Träna på println-debugging
\item Prova debuggern i VS code

% TODO?
%\item munit ?
%\item sbt testOnly och andra sätt att köra testfall???

\end{itemize}
\end{Slide}


\begin{Slide}{Den första buggen}\SlideFontSmall
En nattfjäril i ett relä i datorn Mark II hittad av Grace Hopper i logg från 1940:

\includegraphics[width=0.70\textwidth]{../img/bug}

\url{https://en.wikipedia.org/wiki/Debugging}

\end{Slide}



\begin{Slide}{Olika sorters fel?}
\begin{itemize}
\item Kravfel  
\item Designfel  
\item Implementeringsfel  
\item Testfel  
\item Operatörsfel  
\item Användarfel  
\end{itemize}
\end{Slide}

\begin{Slide}{När upptäcks fel?}
\begin{itemize}
\item Vid granskning av människor  
\item Kompileringsfel -- tack kompilatorn!
\item Exekveringsfel \pause 
\begin{itemize}
\item Exekveringen ger oönskat resultat 
\begin{itemize}
\item Vid testning -- eller är det fel på testfallet?
\item I produktion -- ledsna användare \code{:(}  
\end{itemize}
\item Exekveringen hänger sig \Eng{hang}
\begin{itemize}
\item oändlig loop  
\item väldigt långsamt  
\item väntar på indata
\item dödläge
\end{itemize}
\item Exekveringen kraschar \Eng{crash}
\begin{itemize}
\item minnet är slut
\item null-referens
\item undantag \Eng{exception}
\end{itemize}
\end{itemize}
\end{itemize}
\end{Slide}

\begin{Slide}{Förebygga fel}
\begin{itemize}
\item Skapa begriplig kod.
\item Tänk ut bra namn.
\item Kontrollera parametrar och variabler.
\item Kontrollera typer.
\item Hantera saknade värden.
\item Hantera undantag.
\item Granska kod.
\item Testa kod.
\item Lär av användarnas upplevelser.
\end{itemize}
\end{Slide}

\begin{Slide}{Hitta felorsaken: debugging (avlusning)}
\begin{itemize}
\item Återskapa buggen med ett minimalt testfall.
\item Formulera och verifiera hypoteser om buggen.
\item Instrumentering med utskrifter, "println-debugging".
\end{itemize}
\end{Slide}

\begin{Slide}{Åtgärda fel}
\begin{itemize}
\item Algoritmen i grunden feltänkt: skapa ny algoritm
\item Undantagsfall hanteras ej korrekt.
\item En knepig algoritm är extra svår att fixa till.
\item Medan man rättar en bug kan man råka att, av misstag, skapa nya buggar.
\item Exekveringstiden växer alltför snabbt ökad datamängd.
\end{itemize}
\end{Slide}

\begin{Slide}{Använda en debugger}
\begin{minipage}{0.42\textwidth}
\begin{itemize}
\item Sätta brytpunkter.
\item Stegad exekvering.
\item Inspektera variabler.
\end{itemize}
\end{minipage}%
\begin{minipage}{0.65\textwidth}
\includegraphics[width=1.0\textwidth]{../img/vscode-debug}
\end{minipage}

\vspace{2em}
Läs mer i Appendix H om debuggern i VSCode.
\end{Slide}



\ifkompendium\else

\begin{SlideExtra}{Om veckans övning: \code{matrices}}
\SlideFontSmall
\begin{itemize}

%!TEX encoding = UTF-8 Unicode
%!TEX root = ../exercises.tex


\item Kunna skapa och använda matriser med nästlade strukturer av \code{Vector}.
\item Kunna iterera över elementen i en matris med nästlade \code{for}-satser och \code{for}-\code{yield}-uttryck, samt nästlad applicering av \code{map} respektive \code{foreach}.
\item Kunna skapa och använda funktioner som tar matriser som parametrar.
\item Kunna skapa en enkel generisk klass och enkla generiska funktioner med hjälp av en typparameter.
\item Kunna beskriva skillnader och likheter mellan Scala och Java vad gäller indexering och iterering i matriser implementerade med nästlade arrayer.
%\item Kunna skapa och använda matriser med hjälp inbyggda arrayer i Java.
%\item Kunna använda nästlade \code{for}-satser i Java för att iterera över elementen i en matris.

\end{itemize}

\end{SlideExtra}

\begin{SlideExtra}{Om veckans labb: \code{life}}
\SlideFontSmall
\begin{itemize}
%!TEX encoding = UTF-8 Unicode
%!TEX root = ../compendium2.tex

\item Kunna skapa och använda matriser med hjälp av en generisk datatyp.
\item Kunna iterera över alla element i en matris.
\item Träna på algoritmkonstruktion.
\item Träna på hantering av både oföränderliga och förändringsbara objekt.
\item Prova på att använda en avlusare \Eng{debugger} i en integrerad utvecklingsmiljö (IDE), t.ex. VS code.

\end{itemize}
\end{SlideExtra}

\fi



\chapter{Mönster, Undantag}\label{chapter:W08}
\begin{itemize}[nosep]
\item match
\item Option
\item null
\item try
\item catch
\item Try
\item unapply
\end{itemize}

%!TEX encoding = UTF-8 Unicode
%!TEX root = ../exercises.tex

\ifPreSolution

\Exercise{\ExeWeekEIGHT}\label{exe:W08}

\begin{Goals}
\item Kunna skapa och använda matriser med nästlade strukturer av \code{Vector}.
\item Kunna iterera över elementen i en matris med nästlade \code{for}-satser och \code{for}-\code{yield}-uttryck, samt nästlad applicering av \code{map} respektive \code{foreach}.
\item Kunna skapa och använda funktioner som tar matriser som parametrar.
\item Kunna skapa en enkel generisk klass och enkla generiska funktioner med hjälp av en typparameter.
\item Kunna beskriva skillnader och likheter mellan Scala och Java vad gäller indexering och iterering i matriser implementerade med nästlade arrayer.
%\item Kunna skapa och använda matriser med hjälp inbyggda arrayer i Java.
%\item Kunna använda nästlade \code{for}-satser i Java för att iterera över elementen i en matris.
\end{Goals}

\begin{Preparations}
\item \StudyTheory{08}
\end{Preparations}

\BasicTasks

\else

\ExerciseSolution{\ExeWeekEIGHT}

\BasicTasks

\fi



\WHAT{Para ihop begrepp med beskrivning.}

\QUESTBEGIN

\Task \what

\vspace{1em}\noindent Koppla varje begrepp med den (förenklade) beskrivning som passar bäst:

\begin{ConceptConnections}
  matris & 1 & & A & konkret typ, binds till typparameter vid kompilering \\ 
  generisk & 2 & & B & indexerbar datastruktur i två dimensioner \\ 
  typargument & 3 & & C & har abstrakt typparameter, typen är generell \\ 
  typhärledning & 4 & & D & kompilatorn beräknar typ ur sammanhanget \\ 
\end{ConceptConnections}

\SOLUTION

\TaskSolved \what

\begin{ConceptConnections}
  matris & 1 & ~~\Large$\leadsto$~~ &  A & indexerbar datastruktur i två dimensioner \\ 
  radvektor & 2 & ~~\Large$\leadsto$~~ &  F & matris av dimension $1\times{}m$ med $m$ horisontella värden \\ 
  kolumnvektor & 3 & ~~\Large$\leadsto$~~ &  G & matris av dimension $m\times{}1$ med $m$ vertikala värden \\ 
  kolonn & 4 & ~~\Large$\leadsto$~~ &  C & annat ord för kolumn \\ 
  generisk & 5 & ~~\Large$\leadsto$~~ &  B & har abstrakt typparameter, typen är generell \\ 
  typargument & 6 & ~~\Large$\leadsto$~~ &  D & konkret typ, binds till typparameter vid kompilering \\ 
  typhärledning & 7 & ~~\Large$\leadsto$~~ &  E & kompilatorn beräknar typ ur sammanhanget \\ 
\end{ConceptConnections}

\QUESTEND




\WHAT{Skapa matriser med hjälp av nästlade samlingar.}

\QUESTBEGIN

\Task  \what~  Man kan i ett datorprogram, med hjälp av samlingar som innehåller samlingar, skapa nästlade strukturer som kan indexeras i två dimensioner och på så sätt representera en  \textbf{matris}.\footnote{\href{https://sv.wikipedia.org/wiki/Matris}{sv.wikipedia.org/wiki/Matris}}

\Subtask Rita minnessituationen efter tilldelningen på rad 1 nedan. Vad har \code{m} för typ och värde? Vad har \code{m} för dimensioner? Hur sker indexeringen i ett datorprogram jämfört med i matematiken?

\begin{REPL}
scala> val m = Vector((1 to 5).toVector, (3 to 7).toVector)
scala> m.apply(0).apply(1)
scala> m(1)
scala> m(1)(4)
\end{REPL}

\Subtask Vad ger uttrycken på raderna 2, 3 och 4 ovan för värden och typ?

\Subtask Man kan i ett datorprogram mycket väl skapa tvådimensionella, nästlade strukturer där raderna \emph{inte} innehåller samma antal element. Det blir då ingen äkta matris i strikt matematisk mening, men man kallar ofta ändå en sådan struktur för en ''matris''. Vilken typ har variablerna \code{m2}, \code{m3}, \code{m4} och \code{m5} nedan?

\begin{REPL}
scala> val m2 = Vector(Vector(1,2,3),Vector(4,5),Vector(42))
scala> val m3 = Vector(Vector(1,2), Vector(1.0, 2.0, 3.0))
scala> val m4 = m3(1) +: Vector("a") +: m3
scala> val m5 = Vector.fill(42){ m2(1).map(e => (e * math.random()).toInt) }
\end{REPL}

\Subtask Vilken av variablerna \code{m2}, \code{m3}, \code{m4} och \code{m5} ovan representerar en äkta matris i matematisk mening? Vilken är dess dimensioner?

\SOLUTION

\TaskSolved \what

\SubtaskSolved   \includegraphics{../img/w09-solutions/1a} \\
Typ: \code{Vector[Vector[Int]]}\\
Värde: \code{Vector(Vector(1, 2, 3, 4, 5), Vector(3, 4, 5, 6, 7))} \\
Dimensioner: $2 \times 5$\\
Inom matematiken sker indexering enligt konvention med 1 som lägsta index. I scala är lägsta index 0, man använder s.k. 0-indexering. \footnote{Detta är inte fallet i alla programmeringsspråk, vilket du kan läsa mer om på \url{https://en.wikipedia.org/wiki/Array\_data\_type\#Index\_origin}}

\SubtaskSolved
\begin{REPL}
scala> val m = Vector((1 to 5).toVector, (3 to 7).toVector)
m: Vector[Vector[Int]] = Vector(Vector(1, 2, 3, 4, 5), Vector(3, 4, 5, 6, 7))

scala> m.apply(0).apply(1)
res4: Int = 2

scala> m(1)
res5: Vector[Int] = Vector(3, 4, 5, 6, 7)

scala> m(1)(4)
res6: Int = 7
\end{REPL}

\SubtaskSolved  \\
m2: \code{Vector[Vector[Int]]}\\
m3: \code{Vector[Vector[Int | Double]]}\\
m4: \code{Vector[Vector[Int | Double | String]]}\\
m5: \code{Vector[Vector[Int]]}

\SubtaskSolved  m5, $42 \times 2$

\QUESTEND





\WHAT{Skapa och iterera över matriser.}

\QUESTBEGIN

\Task  \label{matrices:task:yatzy} \what~  Du ska skapa matriser där varje rad representerar 5 kast med en tärning i spelet Yatzy.\footnote{\href{https://sv.wikipedia.org/wiki/Yatzy}{sv.wikipedia.org/wiki/Yatzy}}


\Subtask Definiera i REPL en funktion \code{def throwDie: Int = ???} som returnerar ett slumptal mellan 1 och 6.

\Subtask Skapa nedan heltalsmatris i REPL. Vilken dimension får matrisen?
\begin{REPL}
scala> val ds1 = for (i <- 1 to 1000) yield 
            for (j <- 1 to 5) yield throwDie
          
\end{REPL}

\Subtask Man kan också använda nedan varianter för att skapa en heltalsmatris. Vilken av varianterna \code{ds1} ... \code{ds6} tycker du är lättast att läsa och förstå? Prova respektive variant i REPL och ange vilken typ på \code{ds1} ... \code{ds6} som härleds av kompilatorn.
\begin{REPL}
val ds2 = (1 to 1000).map(i => (1 to 5).map(j => throwDie))
val ds3 = (1 to 1000).map(i => Vector.fill(5)(throwDie))
val ds4 = for (i <- 1 to 1000) yield Vector.fill(5)(throwDie)
val ds5 = Vector.fill(1000)(Vector.fill(5)(throwDie))
val ds6 = Vector.fill(1000, 5)(throwDie)
\end{REPL}


\Subtask Definiera en funktion \\ \code{def roll(n: Int): Vector[Int] = ???}\\ som ger en heltalsvektor med $n$ stycken slumpvisa tärningskast. Kasten ska vara sorterade i växande ordning; använd för detta ändamål samlingsmetoden \code{sorted}.


\Subtask \label{matrices:subtask:isyatzyforall} Definera i REPL en funktion \code{isYatzy(xs: Vector[Int]): Boolean = ???} som testar om alla elementen i en heltalsvektor är samma. Använd samlingsmetoden \code{forall}.


\Subtask Skapa en funktion  \\ \code{def diceMatrix(m: Int, n: Int): Vector[Vector[Int]] = ???} \\ som med hjälp av funktionen \code{roll} skapar en matris med \code{m} st vektorer med vardera \code{n} slumpvisa tärningskast.


\Subtask \label{matrices:subtask:diceMatrixToString} Skapa en funktion som returnerar en utskriftsvänlig sträng \\ \code{def diceMatrixToString(xss: Vector[Vector[Int]]): String = ???} \\med hjälp av \code{map} och \code{mkString}, som fungerar enligt nedan.
\begin{REPL}
scala> val dm2s = diceMatrixToString(diceMatrix(4, 5))
val dm2s: String = 1 4 4 6 6
1 1 2 6 6
2 4 4 5 6
1 1 5 6 6

scala> println(dm2s)
1 4 4 6 6
1 1 2 6 6
2 4 4 5 6
1 1 5 6 6
\end{REPL}



\Subtask Implementera funktionen \\ \code{def filterYatzy(xss: Vector[Vector[Int]]): Vector[Vector[Int]]} \\ som filtrerar fram alla yatzy-rader i matrisen \code{xss} enligt nedan. Använd din funktion \code{isYatzy} och samlingsmetoden \code{filter}.
\begin{REPL}
scala> println(diceMatrixToString(filterYatzy(diceMatrix(10000, 5))))
4 4 4 4 4
6 6 6 6 6
4 4 4 4 4
6 6 6 6 6
4 4 4 4 4
4 4 4 4 4
2 2 2 2 2
\end{REPL}



\Subtask Implementera funktionen \\
\code{def yatzyPips(xss: Vector[Vector[Int]]): Vector[Int] = ???}\\
som ska ge en vektor med de tärningsvärden som gav yatzy, för kasten i matrisen \code{xss} enligt nedan. Använd din funktion \code{filterYatzy}.
\begin{REPL}
scala> val dm = Vector(Vector(1,2,3,4,5),Vector(4,4,4,4,4),Vector(3,3,3,3,3))
scala> yatzyPips(dm)
val res42: Vector[Int] = Vector(4, 3)
\end{REPL}

\SOLUTION

\TaskSolved \what

\SubtaskSolved
\begin{Code}
def throwDie: Int = (math.random() * 6).toInt + 1
\end{Code}
Eller:
\begin{Code}
def throwDie: Int = scala.util.Random.nextInt(6) + 1
\end{Code}

\SubtaskSolved  Matrisdimension i matematisk notation: $1000 \times 5$, vilket motsvarar en matris med 1000 rader och 5 kolumner.

\SubtaskSolved
\begin{Code}
ds1: IndexedSeq[IndexedSeq[Int]]
ds2: IndexedSeq[IndexedSeq[Int]]
ds3: IndexedSeq[Vector[Int]]
ds4: IndexedSeq[Vector[Int]]
ds5: Vector[Vector[Int]]
ds6: Vector[Vector[Int]]
\end{Code}
\code{IndexedSeq} och \code{Vector} ovan finns i paketet \code{scala.collection.immutable}

\SubtaskSolved  \begin{Code}
def roll(n: Int) = Vector.fill(n)(throwDie).sorted
\end{Code}

\SubtaskSolved  \begin{Code}
def isYatzy(xs: Vector[Int]): Boolean = xs.forall(_ == xs(0))
\end{Code}



%2.g)
\SubtaskSolved  \begin{Code}
def diceMatrix(m: Int, n: Int): Vector[Vector[Int]] =
  Vector.fill(m)(roll(n))
\end{Code}

\SubtaskSolved  \begin{Code}
def diceMatrixToString(xss: Vector[Vector[Int]]): String =
  xss.map(_.mkString(" ")).mkString("\n")
\end{Code}


%2.j)
\SubtaskSolved
\begin{Code}
def filterYatzy(xss: Vector[Vector[Int]]): Vector[Vector[Int]] =
  xss.filter(isYatzy)
\end{Code}



%2.m)
\SubtaskSolved  \begin{Code}
def yatzyPips(xss: Vector[Vector[Int]]): Vector[Int] =
  filterYatzy(xss).map(_.head)
\end{Code}

\QUESTEND








\WHAT{En oföränderlig, generisk matris-klass till veckans laboration \hyperref[section:lab:\LabWeekEIGHT]{\texttt{\LabWeekEIGHT}}.}

\QUESTBEGIN

\Task\label{exe:matrices:labprep}  \what~Under veckans laboration ska du simulera en enkel form av ''liv'' som består av celler i ett rutnät. För detta ändamål har vi nytta av en matris-klass som du ska implementera steg för steg i denna övning.
Skapa case-klassen nedan med en editor i filen \code{Matrix.scala}. Testa din lösning med hjälp av valfri \hyperref[appendix:ide]{IDE}, t.ex. \code{scalaide} eller \code{idea}.
\begin{Code}
case class Matrix(data: Vector[Vector[String]]){
  def apply(row: Int, col: Int): String = data(row)(col)
}
object Matrix {
  def fill(dim: (Int, Int))(value: String): Matrix =
    Matrix(Vector.fill(dim._1, dim._2)(value))
}
\end{Code}

\begin{REPLnonum}
scala> val m = Matrix.fill(3,4)("hej")
scala> val e = m(2, 2)
\end{REPLnonum}

\Subtask Vad får \code{m} ovan för typ?

\Subtask Vad får \code{e} ovan för typ?

\Subtask På hur många ställen måste du ändra i \code{Matrix} ovan för att den i stället ska representera en matris av heltal?

\Subtask Du ska nu med hjälp av en \textbf{typparameter} göra \code{Matrix} \textbf{generisk} \Eng{generic}, så att den blir en mer användbar matrisklass som kan innehålla element av vilken typ som helst. Genomför följande ändringar i \code{Matrix.scala}:

\begin{itemize}[noitemsep, nolistsep]
  \item Lägg till en typparameter \code{T} inom klammerparenteser efter namnet \code{Matrix} på alla ställen där det förekommer \emph{utom} efter namnet på kompanjonsobjektet\footnote{Singelobjekt kan inte ha typparametrar, men deras medlemmar kan.}.
  \item Byt ut \code{String} mot \code{T} på alla ställen där \code{String} förekommer.
  \item Lägg till en typparameter \code{T} inom klammerparenteser efter \code{def fill}.
\end{itemize}
Testa din generiska klass i REPL genom att skapa en boolesk matris:
\begin{REPLnonum}
scala> val bm = Matrix.fill(3,4)(false)
scala> val be = bm(0, 0)
\end{REPLnonum}

\Subtask Vad får \code{bm} ovan för typ?

\Subtask Vad får \code{be} ovan för typ?

\Subtask Lägg en kodrad i början av klasskroppen som med hjälp av \code{require} garanterar att alla rader i matrisen är lika långa.

\Subtask Lägg till en medlem \code{val dim: (Int, Int)} i klasskroppen efter \code{require}-satsen som ger ett par (alltså en 2-tupel) med antalet rader resp. kolumner i matrisen.

\Subtask Lägg till en metod \code{def updated(row: Int, col: Int)(value: T): Matrix[T]} som ger en ny matris där element på platsen \code{(row, col)} har uppdaterats till \code{value}.

\Subtask Lägg till en metod \code{def foreachIndex(f: (Int, Int) => Unit): Unit} som för varje index i \code{data} applicerar funktionen \code{f}.

\Subtask Lägg till en metod \code{override def toString} som så att en instans av \code{Matrix} visas enligt följande:
\begin{REPLnonum}
scala> val dm = Matrix.fill(3,4)(42.0)
val dm: Matrix[Double] =
Matrix of dim (3,4):
42.0 42.0 42.0 42.0
42.0 42.0 42.0 42.0
42.0 42.0 42.0 42.0
\end{REPLnonum}


\SOLUTION


\TaskSolved \what

\SubtaskSolved Typen på \code{m} blir \code{Matrix}.

\SubtaskSolved Typen på \code{e} blir \code{String}.

\SubtaskSolved Man behöver ändra på 3 ställen från \code{String} till \code{Int}.

\SubtaskSolved Generisk matris \code{Matrix[T]} för element av godtycklig typ \code{T}:

\begin{CodeSmall}
case class Matrix[T](data: Vector[Vector[T]]):
  def apply(row: Int, col: Int): T = data(row)(col)

object Matrix:
  def fill[T](dim: (Int, Int))(value: T): Matrix[T] =
    Matrix[T](Vector.fill(dim._1, dim._2)(value))
\end{CodeSmall}

\SubtaskSolved Tack vare kompilatorns typinferens så får \code{bm} typen \code{Matrix[Boolean]}.

\SubtaskSolved Typen på \code{be} blir \code{Boolean}.

\noindent \SubtaskSolved \SubtaskSolved \SubtaskSolved \SubtaskSolved \SubtaskSolved är alla implementerade i koden nedan: \vspace{-0.5em}
\begin{CodeSmall}
case class Matrix[T](data: Vector[Vector[T]]):
  require(data.forall(row => row.length == data(0).length))

  val dim: (Int, Int) = (data.length, data(0).length)

  def apply(row: Int, col: Int): T = data(row)(col)

  def updated(row: Int, col: Int)(value: T): Matrix[T] =
    Matrix(data.updated(row, data(row).updated(col, value)))

  def foreachIndex(f: (Int, Int) => Unit): Unit =
    for r <- data.indices; c <- data(r).indices do f(r, c)

  override def toString =
    s"""Matrix of dim $dim:\n${ data.map(_.mkString(" ")).mkString("\n") }"""

object Matrix:
  def fill[T](dim: (Int, Int))(value: T): Matrix[T] =
    Matrix[T](Vector.fill(dim._1, dim._2)(value))

\end{CodeSmall}

\QUESTEND


\clearpage

\ExtraTasks %%%%%%%%%%%%%%%%%%%%%%%%%%%%%%%%%%%%%%%%%%%%%%%%%


\WHAT{Imperativa matrisalgoritmer.}

\QUESTBEGIN

\Task  \what~Imperativa angreppssätt är nödvändiga att kunna när du stöter på samlingar och/eller språk som saknar funktionella metoder och/eller funktionsprogrammeringsmöjligheter. Genom att studera imperativa lösningar till de ofta mer koncisa funktionella lösningarna, får du träning i att skapa algoritmer som använder förändring genom tilldelning vid iterering.

\Subtask Implementera \code{isYatzy} från uppgift \ref{matrices:task:yatzy}\ref{matrices:subtask:isyatzyforall} igen, men nu med ett imperativt angreppssätt som använder en \code{while}-sats i stället för funktionella \code{forall}. Ta hjälp av en variabel \code{i} som håller reda på index och en variabel \code{foundDiff} som håller reda på om ett avvikande värde upptäcks. Funktionen kräver ca 9 rader, så det kan vara lämpligt att öppna en editor att skriva i medan du klurar ut lösningen. Börja med att skriva pseudokod, gärna med penna på papper. Prova genom att klistra in i REPL.

\Subtask En imperativ implementation av \code{diceMatrixToString} från uppgift \ref{matrices:task:yatzy}\ref{matrices:subtask:diceMatrixToString} med hjälp av förändringsbara  \code{StringBuilder}\footnote{\url{https://www.scala-lang.org/api/2.12.9/scala/collection/mutable/StringBuilder.html}} visas nedan. Förklara hur nedan kod fungerar. Vad händer om \code{xss} är tom? Vad händer om \code{xss} bara innehåller tomma vektorer? Nämn en fördel och en nackdel med att använda \code{val sb: StringBuilder} och \code{append}, jämfört med en vanlig, oföränderlig \code{var s: String} och \code{+} för tillägg i slutet.
\begin{Code}
def diceMatrixToString(xss: Vector[Vector[Int]]): String = 
  val sb = new StringBuilder()
  for(m <- xss.indices) do
    for(n <- xss(m).indices) do
      sb.append(xss(m)(n).toString)
      if n < xss(m).size - 1 then sb.append(" ")
      else if m < xss.size - 1 then sb.append("\n")
    end for
  end for
  sb.toString
\end{Code}

\Subtask Gör som träning en imperativ implementation av \code{filterYatzy} med en \code{for}-\code{do}-sats (alltså utan att använda \code{filter}, och utan att använda \code{yield}).


\Subtask Förklara hur nedan funktionella implementation av \code{filterYatzy} med \code{for}-\code{yield}-uttryck fungerar. Tycker du din imperativa lösning är lättare eller svårare att läsa och förstå jämfört nedan funktionella lösning?
\begin{CodeSmall}
def filterYatzy(xss: Vector[Vector[Int]]): Vector[Vector[Int]] = 
  (for i <- xss.indices if isYatzy(xss(i)) yield xss(i)).toVector
\end{CodeSmall}


\SOLUTION

\TaskSolved \what

\SubtaskSolved  \begin{Code}
def isYatzy(xs: Vector[Int]): Boolean = 
  var foundDiff = false
  var i = 0
  while (i < xs.size && !foundDiff) do
    foundDiff = xs(i) != xs(0)
    i += 1
  end while
  !foundDiff
\end{Code}


\SubtaskSolved  Funktionen går igenom varje matrisrad, där den i sin tur går igenom
varje element på raden och lägger till i \code{StringBuilder}-objektet. Om det inte är
det sista elementet på raden läggs även ett blanktecken till, annars läggs ett
nyradstecken till. Undantaget är sista raden, där inget nyradstecken läggs till.
Slutligen konverteras \code{StringBuilder}-objektet till en \code{String} som
returneras.


Är \code{xss} tom blir \code{xss.indices} en tom \code{Range} och den yttre \code{for}-loopen hoppas över och en tom sträng returneras.
Är alla rader tomma hoppas i stället de inre \code{for}-looparna över, med samma resultat.

\emph{Fördel:} \code{StringBuilder} är snabbare vid tillägg på slutet vid stora strängar (men här kommer det inte märkas eftersom strängen är så liten).

\emph{Nackdel:} StringBuilder-koden uppfattas av många som svårare att läsa.

\SubtaskSolved
\begin{Code}
def filterYatzy(xss: Vector[Vector[Int]]): Vector[Vector[Int]] = 
  var result: Vector[Vector[Int]] = Vector()
  for i <- xss.indices if isYatzy(xss(i)) do result = result :+ xss(i)
  result
\end{Code}

\SubtaskSolved  Varje looprunda ger en vektor \code{xss(i)} om filtervillkoret är uppfyllt och resultatet av \code{for}-uttrycket blir en vektor med vektorer som är yatzyslag.

\QUESTEND



\WHAT{Strängtabell med kolumnrubriker.}

\QUESTBEGIN

\Task  \what~  %Denna övning utgör en början på laboration \hyperref[section:lab:survey]{\texttt{survey}} i avsnitt \ref{section:lab:survey} på sidan \pageref{section:lab:survey}.

\Subtask Implementera case-klassen \code{Table} enligt specifikationen nedan. Du kan förutsätta att alla rader har lika många kolumner som antalet element i \code{headings}, samt att alla rubrikerna i \code{headings} är unika. Parametern \code{sep} anger det tecken som används för att separera kolumner. Detta förutsätts också gälla för indatafiler som läses in med \code{fromFile}.

\emph{Tips:}
\begin{itemize}%[nolistsep,noitemsep]
\item Värdet \code{indexOfHeading} kan skapas med hjälp av metoden \code{zipWithIndex} som fungerar på alla sekvenssamlingar, samt metoden \code{toMap} som fungerar på sekvenser av 2-tupler. Undersök först hur metoderna fungerar i REPL och sök upp deras dokumentation.
\item Skapa en indatafil som du kan använda för att testa att \code{Table} fungerar.
\end{itemize}


\begin{CodeSmall}
case class Table(
  data: Vector[Vector[String]],
  headings: Vector[String],
  sep: Char
):
  /** A 2-tuple with (number of rows, number of columns) in data */
  val dim: (Int, Int) = ???

  /** The element in row r and column c of data, counting from 0 */
  def apply(r: Int, c: Int): String = ???

  /** The row-vector r in data, counting from 0 */
  def row(r: Int): Vector[String]= ???

  /** The column-vector c in data, counting from 0 */
  def col(c: Int): Vector[String] = ???

  /** A map from heading to index counting from 0 */
  lazy val indexOfHeading: Map[String, Int] = ???

  /** The column-vector with heading h in data */
  def col(h: String): Vector[String] = ???

  /** A vector with the distinct, sorted values of col with heading h */
  def values(h: String): Vector[String] = ???

  /** Headings and data with columns separated by sep */
  override lazy val toString: String = ???

object Table:
  /** Creates a new Table from fileName with columns split by sep */
  def fromFile(fileName: String, sep: Char = ';'): Table = ???
\end{CodeSmall}

\Subtask Skapa med hjälp av \code{Table} ett program som kan köras från terminalen med \texttt{scala run infile.csv ';'} som ger en utskrift av antalet förekomster av olika värden i respektive kolumn (alltså en variant av registrering).



\SOLUTION

\TaskSolved \what

\SubtaskSolved  \begin{CodeSmall}
case class Table(
  data: Vector[Vector[String]],
  headings: Vector[String],
  sep: Char
):

  val dim: (Int, Int) = (data.size, headings.size)

  def apply(r: Int, c: Int): String = data(r)(c)

  def row(r: Int): Vector[String]= data(r)

  def col(c: Int): Vector[String] = data.map(r => r(c))

  lazy val indexOfHeading: Map[String, Int] = headings.zipWithIndex.toMap

  def col(h: String): Vector[String] = col(indexOfHeading(h))

  def values(h: String): Vector[String] = col(h).distinct.sorted

  override def toString: String =
    val s = sep.toString
    headings.mkString(s) + "\n" +data.map(_.mkString(s)).mkString("\n")

object Table:
  def fromFile(fileName: String, sep: Char = ';'): Table = 
    val lines = scala.io.Source.fromFile(fileName).getLines.toVector
    val matrix= lines.map(_.split(sep).toVector)
    new Table(matrix.tail, matrix.head, sep)
\end{CodeSmall}

\SubtaskSolved  \begin{CodeSmall}
@main 
def run(fileName: String, separator: String): Unit = 
  require(separator.length == 1, "separator ska vara exakt ett tecken")
  val t = Table.fromFile(fileName, separator.head)
  val counts: Vector[Vector[String]] =
    (0 until t.dim._2)
      .map(i => t.values(t.headings(i))
      .map(x => s"$x: ${t.col(i).count(_ == x)}"))
      .toVector
  for (i <- 0 until t.dim._2) do
    println(s"\nColumn: ${i + 1}, ${t.headings(i)}:")
    for (j <- 0 until counts(i).length) do
      println(counts(i)(j))
\end{CodeSmall}

\QUESTEND




\WHAT{Skapa ett yatzy-spel för användning i terminalen.}

\QUESTBEGIN

\Task  \what~%
% \Subtask Skapa en yatzy-matris enligt nedan specifikation. Läs om hur de olika predikaten för att kolla olika giltiga kombinationer i Yatzy ska fungera här: \href{https://en.wikipedia.org/wiki/Yahtzee}{en.wikipedia.org/wiki/Yahtzee}. Bygg ett huvudprogram som testar dina funktioner. Kompilera och testa i terminalen allteftersom du lägger till nya funktioner.
%
% \begin{CodeSmall}
% /** En skiss på en klass som kan användas till ett förenklat yatzy-spel */
% case class YatzyRows(val rows: Vector[Vector[Int]]) {
%   /** A new YatzyRows with a new row of 5 dice rolls appended to rows  */
%   def roll: YatzyRows = ???
%
%   /** A new YatzyRows with some indices of the last row re-rolled  */
%   def reroll(indices: Vector[Int]): YatzyRows = ???
% }
%
% object YatzyRows {
%   def isYatzy(xs: Vector[Int]): Boolean = ???
%   def isThreeOfAKind(xs: Vector[Int]): Boolean = ???
%   def isFourOfAKind(xs: Vector[Int]): Boolean = ???
%   def isFullHouse(xs: Vector[Int]): Boolean = ???
%   def isSmallStraight(xs: Vector[Int]): Boolean = ???
%   def isLargeStraight(xs: Vector[Int]): Boolean = ???
% }
% \end{CodeSmall}
%
%
% \Subtask Använd \code{YatzyRows} för att med hjälp av många tärningskast beräkna sannolikheter för några olika giltiga kombinationer. Använd, om du vill, möjligheten som reglerna ger att slå om tärningar i två ytterliggare kast, där de tärningar som slås om väljs slumpmässigt.
%
%\Subtask
Bygg ett förenklat yatzy-spel i terminalen där användaren kan bestämma vilka tärningar som ska slås om. Börja med något riktigt enkelt och bygg sedan vidare på ditt spel genom att införa fler och fler funktioner.

\SOLUTION


\TaskSolved \what
     %starts with: \emph{Skapa ett yatzy-spel för %%%

 --

% \SubtaskSolved   \begin{CodeSmall}
% /** En skiss på en klass som kan användas till ett förenklat yatzy-spel */
% case class YatzyRows(val rows: Vector[Vector[Int]]) {
%
%   private def throwDie: Int = (math.random() * 6).toInt + 1
%
%   /** A new YatzyRows with a new row of 5 dice rolls appended to rows */
%   def roll: YatzyRows = new YatzyRows(rows :+ Vector.fill(5)(throwDie))
%
%   /** A new YatzyRow with some indices of the last row re-rolled */
%   def reroll(indices: Vector[Int]): YatzyRows =
%     new YatzyRows(rows :+ rows(rows.length - 1).zipWithIndex.map {
%       case (x, i) => if (indices.contains(i)) throwDie else x
%     })
% }
% object YatzyRows {
%
%   def isYatzy(xs: Vector[Int]): Boolean = xs.forall(_ == xs(0))
%
%   def isThreeOfAKind(xs: Vector[Int]): Boolean =
%     xs.exists(x => xs.count(_ == x) >= 3)
%
%   def isFourOfAKind(xs: Vector[Int]): Boolean =
%     xs.exists(x => xs.count(_ == x) >= 4)
%
%   def isFullHouse(xs: Vector[Int]): Boolean =
%     xs.exists(x => xs.count(_ == x) == 3) &&
%     xs.exists(x => xs.count(_ == x) == 2)
%
%   def isSmallStraight(xs: Vector[Int]): Boolean =
%     xs.forall(x => xs.count(_ == x) == 1) && !xs.exists(_ == 6)
%
%   def isLargeStraight(xs: Vector[Int]): Boolean =
%     xs.forall(x => xs.count(_ == x) == 1) && !xs.exists(_ == 1)
% }
%
% \end{CodeSmall}
% Observera att fem stycken 2:or uppfyller kraven för Yatzy, men även för triss och fyrtal.
%
% \SubtaskSolved   Slumpen gör att utfallet inte kommer stämma exakt överens med teorin, men för ett stort antal kast bör resultaten hamna ganska nära. De teoretiska sannolikheterna (utan omkast) finns i \ref{yatzyProb}.
% \begin{table}[h]
% \centering
% \caption{Sannolikhet för olika Yatzy-resultat}
% \label{yatzyProb}
% \begin{tabular}{ll}
% Yatzy&  $0,077\%$  \\
% $\geq3$ av samma& $21\%$\\
% $\geq4$ av samma& $2,0\%$\\
% Kåk& $3,9\%$\\
% Liten stege& $1,5\%$\\
% Stor stege& $1,5\%$
% \end{tabular}
% \end{table}
%
% Kodexempel:
% \begin{CodeSmall}
% import YatzyRows._
%
% object YatzyStats extends App {
%   val n = 1000000.0
%   var yr = YatzyRows(Vector(Vector[Int]()))
%   for (i <- 1 to n.toInt) yr = yr.roll
%   println(s"Yatzy: ${yr.rows.count(isYatzy(_)) / n * 100}%")
%   println(s"Three of a kind: ${yr.rows.count(isThreeOfAKind(_)) / n * 100}%")
%   println(s"Four of a kind: ${yr.rows.count(isFourOfAKind(_)) / n * 100}%")
%   println(s"Full house: ${yr.rows.count(isFullHouse(_)) / n * 100}%")
%   println(s"Small straight: ${yr.rows.count(isSmallStraight(_)) / n * 100}%")
%   println(s"Large straight: ${yr.rows.count(isLargeStraight(_)) / n * 100}%")
% }
% \end{CodeSmall}
%
% \SubtaskSolved  --

\QUESTEND






\clearpage

\AdvancedTasks %%%%%%%%%%%%%%%%%


\WHAT{Generiska funktioner.}

\QUESTBEGIN

\Task  \what~  En generisk funktion har (minst) en typparameter inom klammerparenteser efter namnet, till exempel \code{[T]}. Denna typ förekommer sedan som typ på (någon av) parametrarna i parameterlistan. Kompilatorn härleder en konkret typ vid kompileringstid och ersätter typparametern med denna konkreta typ. På så sätt kan en funktion fungera för många olika typer.

\Subtask Förklara för varje rad nedan vad som händer.

\begin{REPL}
scala> def tnirp[T](x: T): Unit = println(x.toString.reverse)
scala> tnirp(42)
scala> tnirp("hej")
scala> case class Gurka(vikt: Int)
scala> tnirp(Gurka(42))
scala> tnirp[String](42)
scala> tnirp[Double](42)
\end{REPL}

\Subtask Man kan kombinera generiska funktioner med funktioner som tar funktioner som parametrar. Det är så \code{map} och \code{foreach} är implementerade. Förklara för varje rad nedan vad som händer.

\begin{REPL}
scala> def compose[A, B, C](f: A => B, g: B => C)(x: A): C = g(f(x))
scala> def inc(x: Int): Int = x + 1
scala> def half(x: Int): Double = x / 2.0
scala> compose(inc, half)(42)
scala> compose(half, inc)(42)
\end{REPL}

\Subtask Hur lyder felmeddelandet på sista raden ovan? Ändra \code{inc} och/eller \code{half} så att typerna passar.

\SOLUTION

\TaskSolved \what
     %starts with: \emph{Generiska funkioner.} En %%%

%4.a)
\SubtaskSolved   \begin{enumerate}
\item --
\item Strängrepresentationen av \code{42} spegelvänds
\item \code{"hej"} spegelvänds - \code{toString} av en sträng ger en likadan sträng
\item --
\item Gurk-objektets strängrepresentation spegelvänds
\item Funktionens typparameter matchar inte parameterns typ: \code{42} är ingen sträng
\item Implicit typkonvertering till \code{Double} sker för att stämma överens med typparametern, vilket ger en strängrepresentation med decimal
\end{enumerate}

%4.b)
\SubtaskSolved   \begin{enumerate}
\item En funktion definieras så att den tar emot två andra funktioner som argument, sätter ihop dem, och matar in ett tredje argument till den den sammansatta funktionen.
\item En funktion som inkrementerar ett heltal med 1 definieras.
\item En funktion som halverar ett flyttal definieras.
\item \code{42} matas in i \code{inc()} och resultatet (\code{43}) matas vidare till \code{half()}. Inuti \code{half()} sker implicit typkonvertering till \code{Double} då talet divideras med ett flyttal (\code{2.0}) och resultatet blir \code{43.0 / 2.0}, alltså \code{21.5}.
\item Resultatet från \code{half()} är av typ \code{Double}, medan \code{inc()} tar emot ett argument av typ \code{Int}. Då flyttal generellt inte kan konverteras till heltal utan informationsförlust sker ingen implicit konvertering, istället sker ett kompileringsfel.
\end{enumerate}

%4.c)
\SubtaskSolved  \begin{Code}
def inc(x: Double): Double = x + 1.0
\end{Code}
Nu ges kompileringsfel på rad 4 istället, vilket kan lösas med följande ändring:
\begin{Code}
def half(x: Double): Double = x / 2.0
\end{Code}

\QUESTEND




\WHAT{Generiska klasser.}

\QUESTBEGIN

\Task  \what~  Även klasser kan vara generiska. En generisk klass har (minst) en typparameter inom klammerparenteser efter klassens namn.

\Subtask Testa nedan generiska klass \code{Cell[T]} i REPL. Skapa instanser av klassen \code{Cell[T]} där typparametern \code{T} binds till olika konkreta typer och förklara vad som händer.

\begin{REPL}
scala> class Cell[T](var value: T):
         override def toString = "Cell(" + value + ")"
       
scala> new Cell(42)
scala> new Cell("hej")
scala> new Cell(new Cell(math.Pi))
scala> new Cell[String](42)
scala> new Cell[Double](42)
\end{REPL}

\Subtask Lägg till metoden \code{def concat[U](that: Cell[U]):Cell[String]} i klassen \code{Cell} som konkatenerar strängrepresentationerna av de båda cellvärdena.

\begin{REPL}
scala> val a = new Cell("hej")
scala> val b = new Cell(42)
scala> a concat b
\end{REPL}

\Subtask Vilken sorts celler kan du konkatenera om du tar bort typparameternamnet \code{U} i \code{concat} samtidigt som du använder \code{Cell[T]} som typ på värdeparametern \code{that}? Vad ger det för konsekvenser för celler av annan typ än \code{Cell[String]}?

\SOLUTION

\TaskSolved \what

%5.a)
\SubtaskSolved  --

%5.b)
\SubtaskSolved  \begin{Code}
class Cell[T](var value: T):
  override def toString = "Cell(" + value + ")"
  def concat[U](that: Cell[U]): Cell[String] = 
    Cell(s"$value${that.value}")
\end{Code}

%5.c)
\SubtaskSolved   Endast celler med samma typparameter kan nu konkateneras. Eftersom \code{concat()} returnerar ett objekt av typ \code{Cell[String]} kan ett ojämnt antal celler med någon annan typparameter än \code{String} alltså inte längre konkateneras. Är antalet jämnt går det att konkatenera dem parvis och sedan konkatenera de returnerade \code{Cell[String]}-objekten, men det är något omständigt.

\QUESTEND

\WHAT{Implementera fler generiska metoder i \code{Matrix[T]}.}

\QUESTBEGIN

\Task \what~ Bygg vidare på uppgift \ref{exe:matrices:labprep} och implementera nedan specifikation. Skapa egna tester som kontrollerar att alla metoder fungerar som förväntat.

\begin{ScalaSpec}{Matrix[T]}
/** En oföränderlig, generisk Matris-klass. */
case class Matrix[T](data: Vector[Vector[T]]):
  require(???)  // garantera att alla rader har lika många kolumner

  /** Ger ett par med antal rader och kolumner. */
  val dim: (Int, Int) = ???

  /** Ger elementet på plats (row, col). */
  def apply(row: Int, col: Int): T = ???

  /** Ger en ny matris där elementet på plats (row, col) har värdet value. */
  def updated(row: Int, col: Int)(value: T): Matrix[T] =  ???

  /** Applicerar f på alla element. */
  def foreach(f: T => Unit): Unit = ???

  /** Applicerar f på alla index. */
  def foreachIndex(f: (Int, Int) => Unit): Unit = ???

  /** Ger en ny matris med resultaten av elementvis applicering av f. */
  def map[U](f: T => U): Matrix[U] = ???

  /** Ger en ny matris med resultaten av applicering av f på varje index. */
  def mapIndex[U](f: (Int, Int) => U): Matrix[U] = ???

  /** Ger en utskriftsvänlig strängrepresentation av matrisen. */
  override def toString = ???

object Matrix:
  /** Ger en matris med dimension dim där alla element har värdet value. */
  def fill[T](dim: (Int, Int))(value: T): Matrix[T] = ???
\end{ScalaSpec}

\SOLUTION


\TaskSolved \what

\begin{CodeSmall}
case class Matrix[T](data: Vector[Vector[T]]):
  require(data.forall(row => row.size == data(0).size))

  val dim: (Int, Int) = (data.length, data(0).length)

  def apply(row: Int, col: Int): T = data(row)(col)

  def updated(row: Int, col: Int)(value: T): Matrix[T] =
    Matrix(data.updated(row, data(row).updated(col, value)))

  def foreach(f: T => Unit): Unit = data.foreach(_.foreach(f))

  def foreachIndex(f: (Int, Int) => Unit): Unit =
    for r <- data.indices; c <- data(r).indices do f(r, c)

  def map[U](f: T => U): Matrix[U] = Matrix(data.map(_.map(f)))

  def mapIndex[U](f: (Int, Int) => U): Matrix[U] =
    var result = Matrix.fill(dim)(f(0,0))
    for 
      r <- data.indices
      c <- data(r).indices 
    do
      result = result.updated(r, c)(f(r, c))
    end for
    result

  override def toString =
    s"""Matrix of dim $dim:\n${ data.map(_.mkString(" ")).mkString("\n") }"""

object Matrix:
  def fill[T](dim: (Int, Int))(value: T): Matrix[T] =
    Matrix[T](Vector.fill(dim._1, dim._2)(value))
\end{CodeSmall}


\QUESTEND





% \WHAT{Skapa en generisk, oföränderlig matrisklass.}
%
% \QUESTBEGIN
%
% \Task \label{task:generic-matrix} \what~   Med hjälp av en typparameter kan vi skapa en matrisklass som kan innehålla vilka element som helst. Implementera nedan specifikation. Testa din matrisklass i REPL för olika typer av element.
%
% \begin{ScalaSpec}{Matrix[T]}
% case class Matrix[T](data: Vector[Vector[T]]){
%
%   def foreachRowCol(f: (Int, Int, T) => Unit): Unit =
%     for (r <- 0 until data.size) {
%       for (c <- 0 until data(r).size) {
%         f(r, c, data(r)(c))
%       }
%     }
%
%   def map[U](f: T => U): Matrix[U] = Matrix(data.map(_.map(f)))
%
%   /** The element at row r and column c */
%   def apply(r: Int, c: Int): T = ???
%
%   /** Gives Some[T](element) at row r and column c
%    *  if r and c are within index bounds, else None */
%   def get(r: Int, c: Int): Option[T] = ???
%
%   /** The row vector of row r */
%   def row(r: Int): Vector[T] = ???
%
%   /** The column vector of column c */
%   def col(c: Int): Vector[T] = ???
%
%   /** A new Matrix with element at row r and col c updated */
%   def updated(r: Int, c: Int, value: T): Matrix[T] = ???
% }
% object Matrix {
%   def fill[T](rowSize: Int, colSize: Int)(init: T): Matrix[T] =
%     new Matrix(Vector.fill(rowSize)(Vector.fill(colSize)(init)))
% }
% \end{ScalaSpec}
%
% \SOLUTION
%
%
% \TaskSolved \what
%      %%%TODO number  8 %%%starts with: \label{task:generic-matrix} \em%%%
%
% \SubtaskSolved  -- %%%TODO in task 8 %%%
%
%
%
% \QUESTEND
%

% \clearpage
%
% \WHAT{Skapa en Sprite-editor.}
%
% \QUESTBEGIN
%
% \Task  \what~ Använd matrisklassen från uppgift \ref{task:generic-matrix} för att göra en SpriteEditor med JColorChoser enligt nedan skiss.
%
% \begin{Code}
% object ColorChooser {
%   import java.awt.Color
%   import javax.swing.JColorChooser
%
%   var title = "Pick Color"
%   private val chooser = new JColorChooser(Color.BLACK)
%   private val dialog = JColorChooser.
%     createDialog(null, title, true, jcs, null, null)
%
%   def getColor(initColor: Color = Color.BLACK): Color = {
%     chooser.setColor(initColor)
%     dialog.setVisible(true)
%     chooser.getColor
%   }
% }
%
% class Sprite(// en bild med många lager av pixlar i olika färger
%   val id: String,
%   val size: (Int, Int),
%   val pixels: Matrix[Int],   // färg i colors, -1 betyder genomskinlig
%   var scale: Int,            // uppskalning av storlek i pixlar
%   var colors: Vector[Color], // tillgängliga färger
%   var pos: (Int, Int, Int)   // (row, col, layer)
% ){
%   def row = pos._1
%   def col = pos._2
%   def layer = pos._3
% }
%
% class SpriteEditor(
%     rows: Int = 64, cols: Int = 64,
%     scale: Int = 16, nColors: Int = 16) {
%   private val w = new SimpleWindow(???)
%   def edit: Unit = ???
% }
%
% \end{Code}
%
%
%
% \SOLUTION
%
%
% \TaskSolved \what
%      %%%TODO number  9 %%%starts with: \TODO \emph{Klasser för täta oc%%%
%
% \SubtaskSolved  -- %%%TODO in task 9 %%%
%
% \SubtaskSolved  -- %%%TODO in task 9 %%%
%
% \SubtaskSolved  -- %%%TODO in task 9 %%%
%
% \SubtaskSolved  -- %%%TODO in task 9 %%%
%
% \SubtaskSolved  -- %%%TODO in task 9 %%%
%
% \SubtaskSolved  -- %%%TODO in task 9 %%%
%
%
%
% \QUESTEND




% \WHAT{Klasser för täta och glesa matematiska matriser med flyttal.}
%
% \QUESTBEGIN
%
% \Task  \what~   Läs om matrisräkning här: \href{https://sv.wikipedia.org/wiki/Matris}{sv.wikipedia.org/wiki/Matris}
%
% \Subtask Skapa en oföränderlig klass \code{DenseMatrix} för matematiska matriser med dubbelprecisionsflyttal. \code{DenseMatrix} ska internt lagra elementen i en privat \emph{endimensionell} array av flyttal av typen \code{Array[Double]}.
%
% Klassen ska inte vara en case-klass. Det ska gå att skapa matriser med uttryck så som  \code{DenseMatrix.ofDim(3,7)(1.0,42,3.2,1.0,2.2,3)} tack vare ett kompanjonsobjekt med lämplig fabriksmetod som anropar den privata konstruktorn.  Om antalet element är för litet i förhållande till den angivna dimensionen så fyll på med nollor.
%
% \Subtask Överskugga metoderna equals och hashcode och ge \code{DenseMatrix} innehållslikhet i stället för referenslikhet.
%
% \Subtask Implementera egna innehålllikhetsmetoder med namnet \code{===} på \code{DenseMatrix} som är typsäker, d.v.s. bara tillåter innehållsjämförelse mellan täta matriser.
%
% \Subtask Läs om glesa matriser här: \href{https://sv.wikipedia.org/wiki/Gles_matris}{https://sv.wikipedia.org/wiki/Gles\_matris} och implementera \code{SparseMatrix} med ett privat attribut av typen \\ \code{mutable.Map[(Int, Int), Double]} som bara lagrar index som inte är noll.
%
% \Subtask Skapa ett \code{trait Matrix} som både \code{DenseMatrix} och \code{SparseMatrix} ärver, med lämpliga abstrakta och konkreta medlemmar. Implementera addition, subtraktion och multiplikation av täta och glesa matriser.
%
% %\Task \emph{Matriser med \jcode{ArrayList} i Java.} Om man i Java inte vet antalet element i matrisen från början kan man använda en lista av typen \jcode{ArrayList}, där varje element i sin tur innehåller en lista av typen\jcode{ArrayList}. Javas \jcode{ArrayList} är en generisk samling som motsvaras av Scalas \code{ArrayBuffer}. Generiska samlingar i Java kan endast innehålla referenstyper; vill man ha en primitiv typ, t.ex. \jcode{int}, behöver man packa in denna i en s.k. wrapper-klass, t.ex.  klassen \jcode{Integer}. Det finns en wrapper-klass för varje primitiv typ i Java. Matristypen för en heltalstyp i Java skrivs \jcode{ArrayList<ArrayList<Integer>>} där alltså \code{<T>} motsvarar Scalas hakparenteser \code{[T]} för typparametern T.
% %
% %
%
% \SOLUTION
%
% \TaskSolved \what
%      %%%TODO number  10 %%%starts with: \emph{Matriser med \jcode{Array%%%
%
% \SubtaskSolved  -- %%%TODO in task 10 %%%
% \QUESTEND

%!TEX encoding = UTF-8 Unicode
%!TEX root = ../compendium2.tex

\Lab{\LabWeekEIGHT}

\begin{Goals}
\item Kunna skapa och använda matriser.
\item Kunna iterera över matriser med nästlade for-loopar.
\item Träna på algoritmkonstruktion.
\item Använda en integrerad utvecklingsmiljö (IDE).
\end{Goals}

\begin{Preparations}
\item Gör övning {\tt \ExeWeekEIGHT} i kapitel \ref{chapter:W08}, speciellt övning \ref{exe:matrices:labprep}.

\item Läs igenom hela laborationen och studera den befintliga koden i \TODO \code{workspace}.

\item Läs appendix \ref{appendix:ide} och välj vilken IDE du ska använda (ScalaIDE/Eclipse eller IntelliJ IDEA). Säkerställ att du får igång en av dessa utvecklingsmijöer genom att köra hello-world-exemplet och sedan ladda ner och importera kursens \TODO workspace enligt instruktionerna i appendix \ref{appendix:ide}.
\end{Preparations}

\subsection{Bakgrund}



\subsection{Obligatoriska krav}



\subsection{Valbara krav -- välj minst ett}


%%!TEX encoding = UTF-8 Unicode

%!TEX root = ../compendium1.tex

%!TEX encoding = UTF-8 Unicode
\chapter{Arv}\label{chapter:W09}
Begrepp som ingår i denna veckas studier:
\begin{itemize}[noitemsep,label={$\square$},leftmargin=*]
\item arv
\item polymorfism
\item trait
\item extends
\item asInstanceOf
\item with
\item inmixning
\item supertyp
\item subtyp
\item bastyp
\item override
\item klasshierarkin i Scala: Any AnyRef Object AnyVal Null Nothing
\item referenstyper vs värdetyper
\item klasshierarkin i scala.collection
\item Shape som bastyp till Rectangle och Circle
\item accessregler vid arv
\item protected
\item final
\item klass vs trait
\item abstract class
\item case-object
\item typer med uppräknade värden
\item gränssnitt
\item trait vs interface
\item programmeringsgränssnitt (api)\end{itemize}

\clearpage
\input{../slides/body/lect-w09-extends.tex}
\input{../slides/body/lect-w09-override.tex}
%!TEX encoding = UTF-8 Unicode
\chapter{Arv}\label{chapter:W09}
Begrepp som ingår i denna veckas studier:
\begin{itemize}[noitemsep,label={$\square$},leftmargin=*]
\item arv
\item polymorfism
\item trait
\item extends
\item asInstanceOf
\item with
\item inmixning
\item supertyp
\item subtyp
\item bastyp
\item override
\item klasshierarkin i Scala: Any AnyRef Object AnyVal Null Nothing
\item referenstyper vs värdetyper
\item klasshierarkin i scala.collection
\item Shape som bastyp till Rectangle och Circle
\item accessregler vid arv
\item protected
\item final
\item klass vs trait
\item abstract class
\item case-object
\item typer med uppräknade värden
\item gränssnitt
\item trait vs interface
\item programmeringsgränssnitt (api)\end{itemize}

%!TEX encoding = UTF-8 Unicode

%!TEX root = ../compendium2.tex

\Exercise{\ExeWeekNINE}\label{exe:W09}

\begin{Goals}
%!TEX encoding = UTF-8 Unicode

%!TEX root = ../compendium2.tex

\item Känna till begrepp:
bastyp,
sypertyp,
subtyp,
körtidstyp,
dynamisk bindning,
polymorfism,
trait,
inmixning,
överskuggad medlem,
anonym klass,
skyddad medlem,
abstrakt medlem,
abstrakt klass,
referenstyp,
värdetyp.

\item Kunna deklarera och använda en arvshierarki i flera nivåer.

\item Känna till synlighetsregler vid arv och nyttan med privata och skyddade medlemmar.

\item Kunna deklarera och använda skyddade medlemmar.

\item Kunna deklarera och använda överskuggade medlemmar.

\item Känna till reglerna som gäller vid överskuggning av olika sorters medlemmar.

\item Kunna deklarera och använda en hierarki av klasser där konstruktorparametrar överförs till superklasser.

\item Kunna deklarera och använda uppräknade värden med case-objekt och gemensam bastyp.

\item Kunna deklarera och känna till nyttan med finala klasser och finala attribut och nyckelordet \code{final}.

%TODO KOLLA PÅ NEDAN MÅL OCH BESTÄM HUR DE SKA IN I ÖVNINGARNA

%\item Känna till hur typtester och typkonvertering under körtid kan göras med metoderna \code{isInstanceOf} och \code{asInstanceOf} och känna till att detta görs bättre med \code{match}.

%\item Kunna deklarera och använda inmixning med flera traits och nyckelordet \code{with}.

%\item Kunna referera till medlem i superklassen med referensen \code{super} och känna till när detta nyckel ord behövs.

%\item Känna till begreppet anonym klass.

\end{Goals}

\begin{Preparations}
\item \StudyTheory{09}
\end{Preparations}

\BasicTasks %%%%%%%%%%%%%%%%


\Task \emph{Gemensam bastyp.} Man vill ofta lägga in objekt av olika typ i samma samling.
\begin{REPL}
scala> class Gurka(val vikt: Int)
scala> class Tomat(val vikt: Int)
scala> val gurkor = Vector(new Gurka(100), new Gurka(200))
scala> val grönsaker = Vector(new Gurka(300), new Tomat(42))
\end{REPL}
\Subtask Om en samling innehåller objekt av flera olika typer försöker kompilatorn härleda den mest specifika typen som objekten har gemensamt. Vad blir det för typ på värdet \code{grönsaker} ovan?

\Subtask Försök ta reda på summan av vikterna enligt nedan. Vad ger andra raden för felmeddelande? Varför?

\begin{REPL}
scala> gurkor.map(_.vikt).sum
scala> grönsaker.map(_.vikt).sum
\end{REPL}

\Subtask Vi kan göra så att vi kan komma åt vikten på alla grönsaker genom att ge gurkor och tomater en gemensam bastyp som de olika konkreta grönsakstyperna utvidgar med nyckelordet \code{extends}. Man säger att subtyperna \code{Gurka} och \code{Tomat} \textbf{ärver} egenskaperna hos supertypen \code{Grönsak}.

Attributet \code{vikt} i traiten \code{Grönsak} nedan initialiseras inte förrän konstruktorerna anropas när vi gör \code{new} på någon av klasserna \code{Gurka} eller \code{Tomat}.

\begin{REPL}
scala> trait Grönsak { val vikt: Int }
scala> class Gurka(val vikt: Int) extends Grönsak
scala> class Tomat(val vikt: Int) extends Grönsak
scala> val gurkor = Vector(new Gurka(100), new Gurka(200))
scala> val grönsaker = Vector(new Gurka(300), new Tomat(42))
\end{REPL}

\Subtask Vad blir det nu för typ på variabeln \code{grönsaker} ovan?

\Subtask Fungerar det nu att räkna ut summan av vikterna i \code{grönsaker} med \code{grönsaker.map(_.vikt).sum}?


\Subtask En trait liknar en klass, men man kan inte instansiera den och den kan inte ha några parametrar. En typ som inte kan instansieras kallas \textbf{abstrakt} \Eng{abstract}. Vad blir det för felmeddelande om du försöker göra \code{new} på en trait enligt nedan?
\begin{REPL}
scala> trait Grönsak { val vikt: Int }
scala> new Grönsak
\end{REPL}


\Subtask Traiten \code{Grönsak} har en abstrakt medlem \code{vikt}. Den sägs vara abstrakt eftersom den saknar definition -- medlemmen har bara ett namn och en typ men inget värde. Du kan instansiera den abstrakta traiten \code{Grönsak} om du fyller i det som ''fattas'', nämligen ett värde på \code{vikt}. Man kan fylla på det som fattas i genom att ''hänga på'' ett block efter typens namn vid instansiering. Man får då vad som kallas en \textbf{anonym} klass, i detta fall en ganska konstig grönsak som inte är någon speciell sorts grönsak med som ändå har en vikt.

Vad får \code{anonymGrönsak} nedan för typ och strängrepresenation?
\begin{REPL}
scala> val anonymGrönsak = new Grönsak { val vikt = 42 }
\end{REPL}



\Task \emph{Polymorfism i samband med arv.} Polymorfism betyder ''många skepnader''. I samband med arv  innebär det att flera subtyper, till exempel \code{Ko} och \code{Gris}, kan hanteras gemensamt som om de vore instanser av samma supertyp, så som \code{Djur}. Subklasser kan implementera en metod med samma namn på olika sätt. Vilken metod som exekveras bestäms vid körtid beroende på vilken subtyp som instansieras. På så sätt kan djur komma i många skepnader.

\Subtask Implementera funktionen \code{skapaDjur} nedan så att den returnerar antingen en ny Ko eller en ny Gris med lika sannolikhet.

\begin{REPL}
scala> trait Djur { def väsnas: Unit }
scala> class Ko   extends Djur { def väsnas = println("Muuuuuuu") }
scala> class Gris extends Djur { def väsnas = println("Nöffnöff") }
scala> def skapaDjur: Djur = ???
scala> val bondgård = Vector.fill(42)(skapaDjur)
scala> bondgård.foreach(_.väsnas)
\end{REPL}

\Subtask Lägg till ett djur av typen Häst som väsnas på lämpligt sätt och modifiera \code{skapaDjur} så att det skapas kor, grisar och hästar med lika sannolikhet.



\Task \emph{Bastypen \code{Shape} och subtyperna \code{Rectangle} och \code{Circle}.} Du ska nu skapa ett litet bibliotek för geometriska former med oföränderliga objekt implementerade med hjälp av case-klasser. De geometriska formerna har en gemensam bastyp kallad \code{Shape}. Skriv nedan kod i en editor och klistra sedan in den i REPL med kommandot \code{:paste}.
\begin{Code}
case class Point(x: Double, y: Double) {
  def move(dx: Double, dy: Double): Point = Point(x + dx, y + dy)
}

trait Shape {
  def pos: Point
  def move(dx: Double, dy: Double): Shape
}

case class Rectangle(
  pos: Point,
  dx: Double,
  dy: Double
) extends Shape {
  override def move(dx: Double, dy: Double): Rectangle =
    Rectangle(pos.move(dx, dy), this.dx, this.dy)
}

case class Circle(pos: Point, radius: Double) extends Shape {
  override def move(dx: Double, dy: Double): Circle =
    Circle(pos.move(dx, dy), radius)
}
\end{Code}

\Subtask Instansiera några cirklar och rektanglar och gör några relativa förflyttningar av dina instanser genom att anropa \code{move}.

\Subtask Lägg till metoden \code{moveTo} i \code{Point}, \code{Shape}, \code{Rectangle} och \code{Circle} som gör en absolut förflyttning till koordinaterna \code{x} och \code{y}. Klistra in i REPL och testa på några instanser av \code{Rectangle} och \code{Circle}.

\Subtask Lägg till metoden \code{distanceTo(that: Point): Double } i case-klassen \code{Point} som räknar ut avståndet till en annan punkt med hjälp av \code{math.hypot}. Klistra in i REPL och testa på några instanser av \code{Point}.

\Subtask Lägg till en konkret metod \code{distanceTo(that: Shape): Double } i traiten \code{Shape} som räknar ut avståndet till positionen för en annan Shape. Klistra in i REPL och testa på några instanser av \code{Rectangle} och \code{Circle}.







\Task \label{task:fyle} \emph{Inmixning.} Man kan utvidga en klass med multipla traits med nyckelordet \code{with}. På så sätt kan man fördela medlemmar i olika traits och återanvända gemensamma koddelar genom så kallad \textbf{inmixning}, så som nedan exempel visar.

En alternativ fågeltaxonomi, speciellt populär i Skåne, delar in alla fåglar i två specifika kategorier: Kråga respektive Ånka. Krågor kan flyga men inte simma, medan Ånkor kan simma och oftast även flyga. Fågel i generell, kollektiv bemärkelse kallas på gammal skånska för Fyle.%
\footnote{\href{http://www.klangfix.se/ordlista.htm}{www.klangfix.se/ordlista.htm}}
Skriv in nedan kod i en editor och spara den för kommande uppgifter. Klistra in koden i REPL med kommandot \code{:paste}.

\begin{Code}
trait Fyle {
  val läte: String
  def väsnas: Unit = print(läte * 2)
  val ärSimkunnig: Boolean
  val ärFlygkunnig: Boolean
}

trait KanSimma       { val ärSimkunnig = true }
trait KanInteSimma   { val ärSimkunnig = false }
trait KanFlyga       { val ärFlygkunnig = true }
trait KanKanskeFlyga { val ärFlygkunnig = math.random < 0.8 }

class Kråga extends Fyle with KanFlyga with KanInteSimma {
  val läte = "krax"
}

class Ånka extends Fyle with KanSimma with KanKanskeFlyga {
  val läte = "kvack"
  override def väsnas = print(läte * 4)
}
\end{Code}

\Subtask En flitig ornitolog hittar 42 fåglar i en perfekt skog där alla fågelsorter är lika sannolika, representerat av vektorn \code{fyle} nedan. Skriv i REPL ett uttryck som undersöker hur många av dessa som är flygkunniga Ånkor, genom att använda metoderna \code{filter}, \code{isInstanceOf}, \code{ärFlygkunnig} och \code{size}.

\begin{REPL}
scala> val fyle =
         Vector.fill(42)(if (math.random > 0.5) new Kråga else new Ånka)
scala> fyle.foreach(_.väsnas)
scala> val antalFlygånkor: Int = ???
\end{REPL}

\Subtask \label{subtask:fyle:sound} Om alla de fåglar som ornitologen hittade skulle väsnas exakt en gång var, hur många krax och hur många kvack skulle då höras? Använd metoderna \code{filter} och \code{size}, samt predikatet \code{ärSimkunnig} för att beräkna antalet krax respektive kvack.
\begin{REPL}
scala> val antalKrax: Int = ???
scala> val antalKvack: Int = ???
\end{REPL}

\Task \emph{Finala klasser.} Om man vill förhindra att man kan göra \code{extends} på en klass kan man göra den final genom att placera nyckelordet \code{final} före nyckelordet \code{class}.

\Subtask Eftersom klassificeringen av fåglar i uppgiften ovan i antingen Ånkor eller Krågor är fullständig och det inte finns några subtyper till varken Ånkor eller Krågor är det lämpligt att göra dessa finala. Ändra detta i din kod.

\Subtask Testa att ändå försöka göra en subklass \code{Simkråga extends Kråga}. Vad ger kompilatorn för felmeddelande om man försöker utvidga en final klass?


\Task \emph{Accessregler vid arv och nyckelordet \code{protected}.} Om en medlem i en supertyp är privat så kan man inte komma åt den i en subklass. Ibland vill man att subklassen ska kunna komma åt en medlem även om den ska vara otillgänglig i annan kod.

\begin{REPL}
trait Super {
  private val minHemlis = 42
  protected val vårHemlis = 42
}
class Sub extends Super {
  def avslöja = minHemlis
  def kryptisk = vårHemlis * math.Pi
}
\end{REPL}

\Subtask Vad blir felmeddelandet när klassen \code{Sub} försöker komma åt \code{minHemlis}?

\Subtask Deklarera \code{Sub} på nytt, men nu utan den förbjudna metoden \code{avslöja}. Vad blir felmeddelandet om du försöker komma åt \code{vårHemlis} via en instans av klassen \code{Sub}? Prova till exempel med \code{(new Sub).vårHemlis}

\Subtask Fungerar det att anropa metoden \code{kryptisk} på instanser av klassen \code{Sub}?

\Task \emph{Använding av \code{protected}.} Den flitige ornitologen från uppgift \ref{task:fyle} ska ringmärka alla 42 fåglar hen hittat i skogen. När hen ändå håller på bestämmer hen att även försöka ta reda på hur mycket oväsen som skapas av respektive fågelsort. För detta ändamål apterar den flitige ornitologen en linuxdator på allt infångat fyle. Du ska hjälpa ornitologen att skriva programmet.

\Subtask Inför en \code{protected var räknaLäte} i traiten \code{Fyle} och skriv kod på lämpliga ställen för att räkna hur många läten som respektive fågelinstans yttrar.

\Subtask Inför en metod \code{antalLäten} som returnerar antalet krax respektive kvack som en viss fågel yttrat sedan dess skapelse.

\Subtask\Pen Varför inte använda \code{private} i stället for \code{protected}?

\Subtask\Pen Varför är det bra att göra räknar-variabeln oåtkomlig från ''utsidan''?



\Task \emph{Typtester med \code{isInstanceOf} och typkonvertering med \code{asInstanceOf}.} Gör nedan deklarationer.
\begin{REPL}
scala> trait A; trait B extends A; class C extends B; class D extends B
scala> val (c, d) = (new C, new D)
scala> val a: A = c
scala> val b: B = d
\end{REPL}

\Subtask Rita en bild över vilka typer som ärver vilka.

\Subtask Vilket resultat ger dessa typtester? Varför?
\begin{REPL}
scala> c.isInstanceOf[C]
scala> c.isInstanceOf[D]
scala> d.isInstanceOf[B]
scala> c.isInstanceOf[A]
scala> b.isInstanceOf[A]
scala> b.isInstanceOf[D]
scala> a.isInstanceOf[B]
scala> c.isInstanceOf[AnyRef]
scala> c.isInstanceOf[Any]
scala> c.isInstanceOf[AnyVal]
scala> c.isInstanceOf[Object]
scala> 42.isInstanceOf[Object]
scala> 42.isInstanceOf[Any]
\end{REPL}

\Subtask Vilka av dessa typkonverteringar ger felmeddelande? Vilket och varför?
\begin{REPL}
scala> c.asInstanceOf[B]
scala> c.asInstanceOf[A]
scala> d.asInstanceOf[C]
scala> a.asInstanceOf[B]
scala> a.asInstanceOf[C]
scala> a.asInstanceOf[D]
scala> a.asInstanceOf[E]
scala> b.asInstanceOf[A]
\end{REPL}



\Task \emph{Regler för \code{override}, \code{private} och \code{final}.}

\Subtask \label{subtask:overriderules} Undersök överskuggningning av abstrakta, konkreta, privata och finala medlemmar genom att skriva in raderna nedan en i taget i REPL. Vilka rader ger felmeddelande? Varför? Vid felmeddelande: notera hur felmeddelandet lyder och ändra deklarationen av den felande medlemmen så att koden blir kompilerbar (eller om det är enda rimliga lösningen: ta bort den felaktiga medlemmen), innan du provar efterkommande rad.

\begin{REPL}
trait Super1 { def a: Int; def b = 42; private def c = "hemlis" }
class Sub2 extends Super1 { def a = 43; def b = 43; def c = 43 }
class Sub3 extends Super1 { def a = 43; override def b = 43 }
class Sub4 extends Super1 { def a = 43; override def c = "43" }
trait Super5 { final def a: Int; final def b = 42 }
class Sub6 extends Super5 { override def a = 43; def b = 43 }
class Sub7 extends Super5 { def a = 43; override def b = 43 }
class Sub8 extends Super5 { def a = 43; override def c = "43" }
trait Super9 { val a: Int; val b = 42; lazy val c: String = "lazy" }
class Sub10 extends Super9 { override def a = 43; override val b = 43 }
class Sub11 extends Super9 { val a = 43; override lazy val b = 43 }
class Sub12 extends Super9 { val a = 43; override var b = 43 }
class Sub13 extends Super9 { val a = 43; override lazy val c = "still lazy" }
class SubSub extends Sub13 { override val a = 44}
trait Super14 { var a: Int; var b = 42; var c: String }
class Sub15 extends Super14 { def a = 43; override var b = 43; val c = "?" }
\end{REPL}

\Subtask Skapa instanser av klasserna \code{Sub3}, \code{Sub13} och \code{SubSub} från ovan deluppgift och undersök alla medlemmarnas värden för respektive instans. Förklara varför de har dessa värden.

\Subtask Läs igenom reglerna i kapitel  \ref{slideW07:overriderules} om vad som gäller vid arv och överskuggning av medlemmar. Gör några egna undersökningar i REPL som försöker bryta mot någon regel som inte testades i deluppgift \ref{subtask:overriderules}.

\Task \emph{Supertyp med parameter.} En trait kan inte ha någon parameter. Vill man ha en parameter till supertypen måste man använda en klass istället, enligt nedan exempel.

Utbildningsdepartementet vill i sitt system hålla koll på vissa personer och skapar därför en klasshierarki enligt nedan. Skriv in koden i en editor och klipp sedan in den i REPL.
\begin{Code}
class Person(val namn: String)

class Akademiker(
  namn: String,
  val universitet: String) extends Person(namn)

class Student(
  namn: String,
  universitet: String,
  program: String) extends Akademiker(namn, universitet)

class Forskare(
  namn: String,
  universitet: String,
  titel: String) extends Akademiker(namn, universitet)
\end{Code}


\Subtask Deklarera fyra olika \code{val}-variabler med lämpliga namn som refererar till olika instanser av alla olika klasser ovan och ge attributen valfria initialvärden genom olika parametrar till konstruktorerna.

\Subtask Skriv två satser: en som först stoppar in instanserna i en \code{Vector} och en som sedan loopar igenom vektorn och skriv ut alla instansers \code{toString} och \code{namn}.


\Subtask Utbildningsdepartementet vill att det inte ska gå att instansiera objekt av typerna \code{Person} och \code{Akademiker}. Det kan åstadkommas genom att placera nyckelordet \code{abstract} före \code{class}. Uppdatera koden i enlighet med detta. Vilket blir felmeddelande om man försöker instansiera en \code{abstract class}?

\Subtask Utbildningsdeparetementet vill slippa implementera \code{toString} och slippa skriva \code{new} vid instansiering. Gör därför om typerna \code{Student} och \code{Forskare} till case-klasser. \emph{Tips:} För att undkomma ett kompileringsfel (vilket?) behöver du använda \code{override val} på lämpligt ställe.

Skapa instanser av de nya case-klasserna \code{Student} och \code{Forskare} och skriv ut deras \code{toString}. Hur ser utskriften ut?

\Subtask Eftersom \code{Person} och \code{Akademiker} nu är abstrakta, vill utbildningsdepartementet att du gör om dessa typer till traits med abstrakta attribut istället för klasser. Du kan då undvika \code{override val} i klassparametrarna till de konkreta case-klasserna.

Man inför också en case-klass \code{IckeAkademiker} som man tänker använda i olika statistiska medborgarundersökningar.

Dessutom förser man alla personer med ett personnummer representerat som en \code{Int}.

Hur ser utbildningsdepartementets kod ut nu, efter alla ändringar? Skriv ett testprogram som skapar några instanser och skriver ut deras attribut.

\Subtask\Pen I vilka sammanhang är det nödvändigt att använda en \code{trait} respektive en \code{class}?




\Task \emph{Uppräknade värden.} Ett sätt att hålla reda på uppräknade värden, t.ex. färgen på olika kort i en kortlek, är att använda olika heltal som får representera de olika värdena, till exempel så här:\footnote{Om namnkonventioner för konstanter i Scala: läs under rubriken ''Constants, Values, Variable and Methods'' här \href{http://docs.scala-lang.org/style/naming-conventions.html}{docs.scala-lang.org/style/naming-conventions.html}}
\begin{Code}
object Färg {
  val Spader = 1
  val Hjärter = 2
  val Ruter = 3
  val Klöver = 4
}
\end{Code}
Dessa kan sedan användas som parametrar till denna case-klass vid skapande av kortobjekt:
\begin{lstlisting}[language=,keywords={case,class}]
case class Kort(färg: Int, valör: Int)
\end{lstlisting}
Man kan hålla reda på färgen med en variabel av typen \code{Int} och tilldela den en viss färg med ovan konstanter. Och när man skapar ett kort behöver man inte komma ihåg vilket numret är.
\begin{REPL}
scala> val f = Färg.Spader
scala> import Färg._
scala> Kort(Ruter, 7)
\end{REPL}
En annan fördelen med detta är att man lätt kan loopa från 1 till 4 för att gå igenom alla färger.
\begin{REPL}
scala> val kortlek = for (f <- 1 to 4; v <- 1 to 13) yield Kort(f, v)
\end{REPL}
Nackdelen är att kompilatorn vid kompileringstid inte kollar om variablerna av misstag råkar ges något värde utanför det giltiga intervallet, t.ex. 42. Detta får vi själv hålla koll på vid körtid, till exempel med funktionen \code{require} eller \code{if}-satser, etc.

Istället kan man använda case-objekt enligt nedan deluppgifter och få hjälp av kompilatorn att hitta eventuella fel vid kompileringstid.  Ett case-objekt är som ett vanligt singelton-objekt men det får automatiskt en \code{toString} samma som namnet och kan användas i matchningar etc. (mer om match i kapitel \ref{chapter:W08}).

\Subtask Deklarera följande uppräknade värden som singelton objekt med gemensam bastyp i en editor och klistra in i REPL med kommandot \code{:paste}. Med nyckelordet \code{sealed} så ''förseglas'' klassen och inga andra direkta subtyper tillåts förutom de som finns i samma kod-fil eller block. I detta exempel  med kortfärger vet vi ju att det inte finns fler än dessa fyra färger.
\begin{Code}
sealed trait Färg
case object Spader extends Färg
case object Hjärter extends Färg
case object Ruter extends Färg
case object Klöver extends Färg
\end{Code}
Dessa kan sedan användas som parametrar till denna case-klass vid skapande av kortobjekt:
\begin{lstlisting}[language=,keywords={case,class}]
case class Kort(färg: Färg, valör: Int)
\end{lstlisting}
Skapa därefter några exempelinstanser av klassen \code{Kort}. Vad är fördelen med ovan angreppssätt jämfört med att använda heltalskonstanter?

\Subtask Om man vill kunna iterera över alla värden är det bra om de finns i en samling med alla värden. Vi kan lägga en sådan i ett kompanjonsobjekt till bastypen. Uppdatera koden enligt nedan och klistra in på nytt i REPL med kommandot \code{:paste}. Skriv ut alla färgvärden med en \code{for}-sats.

\begin{Code}
sealed trait Färg
object Färg {
  val values = Vector(Spader, Hjärter, Ruter, Klöver)
}
case object Spader extends Färg
case object Hjärter extends Färg
case object Ruter extends Färg
case object Klöver extends Färg
\end{Code}
Skapa en kortlek med 52 olika kort och blanda den sedan med \code{Random.shuffle} enligt nedan. Använd en dubbel \code{for}-sats och \code{yield}.
\begin{REPL}
scala> val kortlek: Vector[Kort] = ???
scala> val blandad = scala.util.Random.shuffle(kortlek)
\end{REPL}

\Subtask Skriv en funktion \code{ def blandadKortlek: Vector[Kort] = ???} som ger en ny blandad kortlek varje gång den anropas med metoden i föregående uppgift.

%%%%%%%%%%%%%%%%%%% FEEEEEELLL \end{Code}



\Subtask Om man även vill ha en heltalsrepresentation med en medlem \code{toInt} för alla värden, kan man ge bastypen en parameter och i stället för en trait (som inte kan ha några parametrar) använda en abstrakt klass.

\begin{Code}
sealed abstract class Färg(final val toInt: Int)
object Färg {
  val values = Vector(Spader, Hjärter, Ruter, Klöver)
}
case object Spader  extends Färg(0)
case object Hjärter extends Färg(1)
case object Ruter   extends Färg(2)
case object Klöver  extends Färg(3)
\end{Code}
Skapa en funktion \code{färgPoäng} som räknar ut summan av heltalsrepresentationen av alla färger hos en vektor med kort, och använd den sedan för att räkna ut \code{färgPoäng} för de första fem korten.
\begin{REPL}
scala> def blandadKortlek: Vector[Kort] = ???
scala> def färgPoäng(xs: Vector[Kort]): Int = ???
scala> färgPoäng(blandadKortlek.take(5))
\end{REPL}


\ExtraTasks %%%%%%%%%%%%%%%%%%%

\Task Det visar sig att vår flitige ornitolog från uppgift \ref{task:fyle} på sidan \pageref{task:fyle} sov på en av föreläsningarna i zoologi när hen var nolla på Natfak, och därför helt missat fylekategorin \code{Pjodd}. Hjälp vår stackars ornitolog så att fylehierarkin nu även omfattar Pjoddar. En Pjodd kan flyga som en Kråga men den \code{ÄrLiten} medan en Kråga \code{ÄrStor}. En Pjodd kvittrar dubbelt så många gånger som en Ånka kvackar. En Pjodd \code{KanKanskeSimma} där simkunnighetssannolikheten är $0.2$. Låt ornitologen ånyo finna 42 slumpmässiga fåglar i skogen och filtrera fram lämpliga arter. Undersök sedan hur dessa väsnas, i likhet med deluppgift \ref{task:fyle}\ref{subtask:fyle:sound}.


\clearpage

\AdvancedTasks %%%%%%%%%%%%%%%%%

\Task Hitta på en egen fördjupningsuppgift inspirerat av denna artikel på Stackoverflow: \url{http://stackoverflow.com/questions/16173477/usages-of-null-nothing-unit-in-scala}

\Task Studera den djupa arvshierarkin i paketet \code{numbers} nedan som modellerar olika sorters tal i matematiken. Du kan även ladda ner koden här: \\
\href{https://github.com/lunduniversity/introprog/blob/master/compendium/examples/numbers.scala}{github.com/lunduniversity/introprog/blob/master/compendium/examples/numbers.scala}
\\ Notera metoden \code{reduce} som reducerar ett tal till sin enklaste form och dess implementation överskuggas på lämpliga ställen med relevant reduktion.

\Subtask Skriv kod som använder de olika konkreta klasserna i \code{package numbers}. Om du kompilerar koden i samma bibliotek som du kör igång REPL är det bara att använda paketet direkt:
\begin{REPL}
$ scalac numbers.scala
$ scala
scala> numbers.  // Tryck Tab
AbstractComplex   AbstractNatural    AbstractReal   Frac    Nat      Polar
AbstractInteger   AbstractRational   Complex        Integ   Number   Real

scala> numbers.Integ(12)
res0: numbers.Integ = Integ(12)

scala> import numbers.Syntax._
import numbers.Syntax._

scala> 42.j
res1: numbers.Complex = Complex(Real(0),Real(42))

scala> 42.42.j
res2: numbers.Complex = Complex(Real(0),Real(42.42))

\end{REPL}

\Subtask Ändra på metoden \code{+} i \code{trait Number} så att den blir abstrakt och implementera den i alla konkreta klasser.

\Subtask Implementera fler räknesätt och bygg vidare på koden så som du finner intressant.

\Subtask Gör så att metoden \code{reduce} i klassen \code{AbstractRational} använder algoritmen Greatest Common Divisor (GCD)\footnote{\href{https://sv.wikipedia.org/wiki/St\%C3\%B6rsta_gemensamma_delare}{https://sv.wikipedia.org/wiki/St\%C3\%B6rsta\_gemensamma\_delare}} så som beskrivs här: \\ \href{http://www.artima.com/pins1ed/functional-objects.html#6.8}{www.artima.com/pins1ed/functional-objects.html\#6.8} \\ så att täljare och nämnare blir så små som möjligt.

\scalainputlisting[numbers=left, basicstyle=\ttfamily\fontsize{9}{11}\selectfont]{examples/numbers.scala}

%!TEX encoding = UTF-8 Unicode
%!TEX root = ../compendium2.tex

\Teamlab{\LabWeekNINE}

\begin{Goals}
%!TEX encoding = UTF-8 Unicode
%!TEX root = ../compendium2.tex

\item Kunna använda arv.
\item Kunna göra överskuggning av medlemmar i en supertyp med \code{override}.
\item Kunna referera till medlemmar i superklassen med \code{super} vid överskuggning.
\item Kunna förklara begreppet dynamisk bindning.

\end{Goals}

\begin{Preparations}
\item Gör övning {\tt \ExeWeekNINE} i kapitel \ref{exe:W09}, speciellt uppgift \ref{exe:inheritance:labprep-pair}.
\item Läs dokumentationen för \code{introprog.BlockGame}.
\item Läs igenom hela laborationen och förbered dig inför första gruppmötet.
%!TEX encoding = UTF-8 Unicode
%!TEX root = compendium.tex
\item
Diskutera i din samarbetsgrupp hur ni ska dela upp koden mellan er i flera olika delar, som ni kan arbeta med var för sig. En sådan del kan vara en klass, en trait, ett objekt, ett paket, eller en funktion.
\item
Varje del ska ha en \textbf{huvudansvarig} individ.
\item
Arbetsfördelningen ska vara någorlunda jämnt fördelad mellan gruppmedlemmarna.
\item
Den som är huvudansvarig för en viss del redovisar den delen.
\item 
Ni ska ta fram en gruppgemensam checklista för kodgranskning. Alla ska granska minst en annan gruppmedlems kod enligt checklistan. 
\item
Grupplaborationen görs över \textbf{två veckor} uppdelat på två delredovisningar. Vid första redovisningen ska arbetsupplägget och pågående utveckling redovisas. Vid andra tillfället ska de färdig lösningarna presenteras av respektive huvudansvarig individ.
\item
Vid första redovisningen ska du redogöra för handledaren hur ni delat upp koden och vem som är huvudansvarig för vad och vad ditt ansvar omfattar, samt hur ni jobbar praktiskt med att synkronisera er utveckling.
\item Grupplaborationen är en \textbf{extra stor uppgift} och grupparbetet behöver ledtid för att ni ska hinna koordinera er sinsemellan. Du behöver därför planera för att arbeta med något i grupplabben i stort sett varje dag under de tillgängliga veckorna, och vara redo att bidra i diskussioner.

\item Träffas i din samarbetsgrupp och diskutera ert arbetssätt utifrån följande frågeställningar:
\begin{itemize}
  \item Vilken krav ska ni implementera?
  \item Hur ska ni jobba med gemensamma koddelar?
  \item Hur ska ni dela med er av de koddelar som ni utvecklar var för sig?
\end{itemize}

\end{Preparations}

\subsection{Bakgrund}

Spelet \emph{Snake}\footnote{Även kallat ''masken''. \url{https://sv.wikipedia.org/wiki/Snake}}

\TODO screendumps one+two-player game
\subsection{Obligatoriska funktionella krav}

Följande funktionella krav ska uppfyllas av ert program om ni är sex personer i gruppen. Om ni är färre kan ni välja att skippa krav enligt tabell \ref{lab:snak:table-reqt}.
%\footnote{Om någon student, p.g.a. långvarig sjukdom eller annat giltigt skäl, genomför laborationen själv i efterhand som en individuell laboration ska följande krav implementeras på egen hand: \code{Player}, \code{OnePlayerGame}, \code{Snake}, \code{Apple}.}
\begin{itemize}[nosep, label={$\square$},]
\item \textbf{\texttt{Player}}. Det ska finnas spelare som motsvarar mänskliga användare och som har ett namn och fyra knappar som den kan spela med. Varje spelare har en egen orm som den kan styra med sina knappar.

\item \textbf{\texttt{Snake}}. Det ska finnas ormar. En orm består av ett antal block, där det främsta blocket kallas huvud och resten av blocken kallas svans. Huvudet har en ljusare färg än kroppen. Svansens längd ökar under spelets gång. En orm rör sig i en viss riktning och varje spelare kan ändra riktningen på sin orm med sina knappar, i en av fyra riktningar \code{North}, \code{South}, \code{East} eller \code{West}.

\item \textbf{\texttt{Apple}}. Det ska finnas äpple. Ett äpple består av ett rött block och finns på en slumpvis position. Ett äpple kan ätas av en orm om ormens huvud träffar äpplet. Varje gång ett äpple äts upp av en orm så teleporteras äpplet till en ny position och kan ätas igen.

\item \textbf{\texttt{Banana}}. Det ska finnas bananer. En banan består av tre vertikala gula block och finns på en slumpvis position. En banan kan ätas av en orm om ormens huvud träffar bananen. Varje gång en banan äts upp av en orm så teleporteras bananen till en ny position och kan ätas igen.

\item \textbf{\texttt{OneplayerGame}}. Det ska gå att spela ensam. I varianten med en spelare finns en orm och ett äpple. Varje gång användarens orm lyckas äta ett äpple får användaren poäng. När användaren ätit ett visst antal äpplen är spelet slut och poängen visas. Allteftersom tiden går blir svansen, vid jämna tidsintervall, längre.

\item \textbf{\texttt{TwoplayerGame}}. Det ska gå att spela två och två. I varianten med två spelare finns två ormar. Det finnas också flera äpplen och flera bananer. Om en orm äter en banan blir dess svans längre. Om en orm äter ett äpple får dess spelare poäng. Allteftersom tiden går blir båda ormarnas svansar, vid jämna tidsintervall, längre.

\end{itemize}
\begin{table}[H]
  \centering
  \caption{Krav som minst ska implementeras vid respektive gruppstorlek. \label{lab:snak:table-reqt}}

\begin{tabular}{r | c c c c c}
  Krav                   & 2       & 3       & 4       & 5       & 6 \\ \hline
  \texttt{Player}        & $\surd$ & $\surd$ & $\surd$ & $\surd$ & $\surd$ \\
  \texttt{OnePlayerGame} &         & $\surd$ &         & $\surd$ & $\surd$ \\
  \texttt{TwoPlayerGame} & $\surd$ & $\surd$ & $\surd$ & $\surd$ & $\surd$ \\
  \texttt{Snake}         & $\surd$ & $\surd$ & $\surd$ & $\surd$ & $\surd$ \\
  \texttt{Apple}         & $\surd$ &         & $\surd$ & $\surd$ & $\surd$ \\
  \texttt{Banana}        &         &         & $\surd$ &         & $\surd$ \\
\end{tabular}
\end{table}

\subsection{Obligatoriska design-krav}

\begin{enumerate}[label={$\square$}, leftmargin=*]

\item Snake-spel ska gå att starta med huvudprogrammet nedan. Huvudprogrammet får ändras vid behov i enlighet med minimikrav vad gäller gruppstorlek i tabell \ref{lab:snak:table-reqt}, samt valbara extrakrav i avsnitt \ref{lab:snake:extra-reqts}.
\begin{Code}
package snake

object Main {
  def main(args: Array[String]): Unit = {
    val buttons = Seq("One","Two", "Tournament", "Cancel")
    val selected =
      introprog.Dialog.select("Number of players?", buttons, "Snake")
    selected match {
      case "One" => (new OnePlayerGame).play("Pink")
      case "Two" => (new TwoPlayerGame).play("Purple", "Pink")
      case "Tournament" => ??? // valbart krav
      case _ =>
    }
  }
}
\end{Code}

\item Spelet ska bygga vidare på \code{introprog.BlockWindow} enligt typhierarkin i fig.~\ref{snake:fig:game-hierarchy}.

\begin{figure}[H]
\begin{center}
\newcommand{\TextBox}[1]{\raisebox{0pt}[1em][0.5em]{#1}}
\tikzstyle{umlclass}=[rectangle, draw=black,  thick, anchor=north, text width=3cm, rectangle split, rectangle split parts = 3]
\begin{tikzpicture}[inner sep=0.5em,scale=1.0, every node/.style={transform shape}]

  \node [umlclass, rectangle split parts = 1, xshift=0cm, yshift=4.5cm] (BaseType)  {
              \textit{\textbf{\centerline{\TextBox{\code{BlockGame}}}}}
%              \nodepart[align=left]{second}\code{def x: T} \newline \code{def y: T}
          };


  \node [umlclass, rectangle split parts = 1, xshift=0cm, yshift=3.0cm] (SubType)  {
              \textit{\textbf{\centerline{\TextBox{\code{SnakeGame}}}}}
%              \nodepart[align=left]{second}\code{val x: Int} \newline \code{val y: Int}
          };

\node [umlclass, rectangle split parts = 1, xshift=-3cm, yshift=1.0cm] (SubSubType1)  {
            {\textbf{\centerline{\TextBox{\code{OnePlayerGame}}}}}
%            \nodepart[]{second}\TextBox{\code{val dim: Int}}
        };

\node [umlclass, rectangle split parts = 1, xshift=3cm, yshift=1.0cm] (SubSubType2)  {
            {\textbf{\centerline{\TextBox{\code{TwoPlayerGame}}}}}
%            \nodepart[]{second}\TextBox{\code{val dim: Int}}
        };

\draw[umlarrow] (SubType.north) -- ++(0,0.5) -| (BaseType.south);
\draw[umlarrow] (SubSubType1.north) -- ++(0,0.5) -| (SubType.south);
\draw[umlarrow] (SubSubType2.north) -- ++(0,0.5) -| (SubType.south);

\end{tikzpicture}
\end{center}
\caption{Arvshierarki med klassen \code{introprog.BlockGame} som bastyp.}
\label{snake:fig:game-hierarchy}
\end{figure}


\item Ormar och frukt ska utgå från bastypen \code{Entity} enligt typhierarkin i ~\ref{snake:fig:entity-hierarchy}.

\begin{figure}[H]
\begin{center}
\newcommand{\TextBox}[1]{\raisebox{0pt}[1em][0.5em]{#1}}
\tikzstyle{umlclass}=[rectangle, draw=black,  thick, anchor=north, text width=2.5cm, rectangle split, rectangle split parts = 3]
\begin{tikzpicture}[inner sep=0.5em,scale=1.0, every node/.style={transform shape}]

  \node [umlclass, rectangle split parts = 1, xshift=-1.0cm, yshift=4.5cm] (BaseType)  {
              \textit{\textbf{\centerline{\TextBox{\code{Entity}}}}}
%              \nodepart[align=left]{second}\code{def x: T} \newline \code{def y: T}
          };


  \node [umlclass, rectangle split parts = 1, xshift=-3cm, yshift=2.5cm] (SubType1)  {
              \textit{\textbf{\centerline{\TextBox{\code{MovingEntity}}}}}
%              \nodepart[align=left]{second}\code{val x: Int} \newline \code{val y: Int}
          };

\node [umlclass, rectangle split parts = 1, xshift=-4.5cm, yshift=0.5cm] (SubSubType0)  {
            {\textbf{\centerline{\TextBox{\code{Snake}}}}}
%            \nodepart[]{second}\TextBox{\code{val dim: Int}}
};


\node [umlclass, rectangle split parts = 1, xshift=0.75cm, yshift=2.5cm] (SubType2)  {
            \textit{\textbf{\centerline{\TextBox{\code{Fruit}}}}}
%            \nodepart[]{second}\TextBox{\code{val dim: Int}}
        };

\node [umlclass, rectangle split parts = 1, xshift=-1.0cm, yshift=0.5cm] (SubSubType1)  {
            {\textbf{\centerline{\TextBox{\code{Apple}}}}}
%            \nodepart[]{second}\TextBox{\code{val dim: Int}}
        };

\node [umlclass, rectangle split parts = 1, xshift=2.5cm, yshift=0.5cm] (SubSubType2)  {
            {\textbf{\centerline{\TextBox{\code{Banana}}}}}
%            \nodepart[]{second}\TextBox{\code{val dim: Int}}
        };


\draw[umlarrow] (SubType1.north) -- ++(0,0.5) -| (BaseType.south);
\draw[umlarrow] (SubType2.north) -- ++(0,0.5) -| (BaseType.south);
\draw[umlarrow] (SubSubType1.north) -- ++(0,0.5) -| (SubType2.south);
\draw[umlarrow] (SubSubType2.north) -- ++(0,0.5) -| (SubType2.south);
\draw[umlarrow] (SubSubType0.north) -- ++(0,0.5) -| (SubType1.south);

\end{tikzpicture}
\end{center}
\caption{Arvshierarki med klassen \code{Entity} som bastyp.}
\label{snake:fig:entity-hierarchy}
\end{figure}


\item \code{Entity} representerar en varelse i ett spel och ska se ut så här:
\begin{Code}
trait Entity {
  def draw():   Unit
  def erase():  Unit
  def update(): Unit
  def reset():  Unit
}
\end{Code}
Metoderna \code{draw} resp. \code{erase} anropas vid ritning resp. radering. Metoden \code{reset} återställer ursprungstillståndet. Metoden \code{update} anropas en gång i varje runda i spel-loopen.

\item \code{MovingEntity} representerar en spelvarelse som kan röra sig i en viss hastighet och ska se ut så här:
\begin{Code}
trait MovingEntity extends Entity {
  def move(): Unit

  var movesPerSecond: Double = 20.0

  final def millisBetweenMoves(): Int =
    (1000 / movesPerSecond).round.toInt max 1

  private var _timestampLastMove: Long = System.currentTimeMillis
  final def timestampLastMove = _timestampLastMove

  override final def update(): Unit =
    if (System.currentTimeMillis > _
          timestampLastMove + millisBetweenMoves) {
      _timestampLastMove = System.currentTimeMillis
      move()
    }
}
\end{Code}

\item Det ska i \code{SnakeGame} finnas en typhierarki med \code{State} som bastyp som representerar spelets övergripande tillstånd enligt följande:
\begin{Code}
sealed trait State
case object Starting extends State
case object Playing  extends State
case object GameOver extends State
case object Quitting extends State
\end{Code}

\item Vid varje runda i spelloopen ska följande logik exekveras. Denna kod placeras förslagsvis i \code{gameLoopAction}, se vidare \code{SnakeGame} i avsnitt \ref{lab:snake:tips}.
\begin{Code}
if (state == Playing) {
  entities.foreach(_.erase)
  entities.foreach(_.update)
  entities.foreach(_.draw)
  if (isGameOver) enterGameOverState()
}
\end{Code}

\end{enumerate}



\subsection{Valbara krav -- välj minst ett}\label{lab:snake:extra-reqts}

Du ska implementera minst ett (gärna flera) av dessa krav:
\begin{itemize}[nosep, label={$\square$}]
\item Spelare kan ange sitt namn, t.ex. via en dialog eller genom argument till \code{main}.
\item Om en spelare kör in i sin egen orms svans så är spelet förlorat. %(Detta är vanligt i många Snake-spel.)
\item Om en spelare kör utanför spelplanen så är spelet förlorat. %(Detta är vanligt i många Snake-spel.)
\item \textbf{\code{Points}}. Inför ett poängsystem, där poängen beror på t.ex. längden på svansen, antalet svängar, antal uppätna äpplen, etc.
\item \textbf{\code{Highscore}}. Spelet ska visa en lista med de spelare som fått flest poäng.

\item \textbf{\code{TwoPlayerComp extends Competition}}. Två spelare ska kunna tävla i en bäst-av-$n$-matcher-tävling i en sekvens av \code{TwoPlayerGame.play}, där den som vinner flest matcher blir blir totalvinnare.

\item \textbf{\code{SinglePlayerComp extends Competition}}. Flera spelare ska kunna tävla i en-persons-Snake, där den som får flest poäng av $n$ \code{OnePlayerGame}-spel blir totalvinnare.


\item \textbf{\code{Tournament extends Competition}}. Många spelare ska kunna spela en turnering.\footnote{\url{https://en.wikipedia.org/wiki/Tournament}} Namnen på spelarna läses in från en textfil.
\begin{itemize}[nosep, label={$\square$}]
\item \textbf{\code{KnockOut extends Tournament}}. Det ska gå att spela en utslagsturnering, som avslutas med final efter semi-final, etc., beroende på antal spelare.
\item \textbf{\code{RoundRobin extends Tournament}}. Det ska gå att spela en alla-möter-alla-turnering, där alla möjliga par av spelare möts i slumpvis ordning.
\end{itemize}

\end{itemize}


\subsection{Tips och förslag}\label{lab:snake:tips}

Här följer en skiss på den abstrakta klassen \code{SnakeGame} med de abstrakta metoderna \code{isGameOver} och \code{play} som överskuggas i underklasserna \code{OnePlayerGame} och \code{TwoPlayerGame}:
\begin{CodeSmall}
package snake
import introprog.BlockGame

abstract class SnakeGame(title: String) extends BlockGame(
  title,
  dim = (60, 60),
  blockSize = 15,
  background = Colors.black,
  framesPerSecond = 50,
  messageAreaHeight = 3,
) {
  var entities: Vector[Entity] = Vector.empty
  var players: Vector[Player] = Vector.empty
  var state: State = Starting

  def enterStartingState(): Unit = {
    clearWindow
    drawCenteredText("Press SPACE to start!")
    state = Starting
  }

  def enterPlayingState(): Unit = {
    clearWindow
    entities.foreach(_.reset)
    entities.foreach(_.draw)
    state = Playing
  }

  def enterGameOverState(): Unit =  ???

  def enterQuittingState(): Unit =  ???

  override def handleKey(key: String): Unit = ???

  override def onClose(): Unit = ???

  def isGameOver(): Boolean

  override def gameLoopAction(): Unit =

  def startGameLoop(): Unit = {
    pixelWindow.open // möjliggör omstart även om fönstret stängts
    enterStartingState()
    gameLoop(stopWhen = state == Quitting)
  }

  def play(playerNames: String*): Unit
}
\end{CodeSmall}


\begin{enumerate}[leftmargin=*]
\item
\end{enumerate}


%\input{modules/w10-patterns-chapter.tex}
%!TEX encoding = UTF-8 Unicode
\chapter{Sökning, Sortering}\label{chapter:W10}
Begrepp du ska lära dig denna vecka:
\begin{itemize}[noitemsep,label={$\square$},leftmargin=*]
\item compareTo på strängar
\item trait Ordered[T]
\item algoritm: LINEAR-SEARCH
\item algortim: BINARY-SEARCH
\item algoritmisk komplexitet
\item sortering till ny vektor
\item sortering på plats
\item algoritm: INSERTION-SORT
\item algoritm: SELECTION-SORT\end{itemize}

%!TEX encoding = UTF-8 Unicode

%!TEX root = ../compendium2.tex

\Exercise{\ExeWeekTEN}\label{exe:W10}

\begin{Goals}
\item Kunna skapa och använda \code{match}-uttryck med konstanta värden, garder och mönstermatchning med case-klasser.
\item Kunna skapa och använda case-objekt för matchningar på uppräknade värden.
\item Känna till betydelsen av små och stora begynnelsebokstäver i case-grenar i en matchning, samt förstå hur namn binds till värden in en case-gren.
\item Kunna hantera saknade värden med hjälp av typen \code{Option} och mönstermatchning på \code{Some} och \code{None}.
\item Känna till hur metoden \code{unapply} används vid mönstermatchning.
\item Känna till nyckelordet \code{sealed} och förstå nyttan med förseglade typer.
\item Känna till \jcode{switch}-satser i Java.
\item Känna till \code{null}.
\item Kunna fånga undantag med \code{try}-\code{catch} och \code{scala.util.Try}.
\item Känna till skillnaderna mellan \code{try}-\code{catch} i Scala och java.
\item Kunna implementera \code{equals} med hjälp av en \code{match}-sats, som fungerar för finala klasser utan arv.
\item Känna till relationen mellan \code{hashcode} och \code{equals}.
\item Kunna använda \code{flatMap} tillsammans med \code{Option} och \code{Try}.
\item Kunna skapa partiella funktioner med case-uttryck.
\end{Goals}

\begin{Preparations}
\item \StudyTheory{10}
\end{Preparations}

\BasicTasks %%%%%%%%%%%%%%%%

\Task \label{task:switch} \emph{Hur fungerar en \jcode{switch}-sats i Java (och flera andra språk)?} Det händer ofta att man vill testa om ett värde är ett av många olika alternativ. Då kan man använda en sekvens av många \code{if}-\code{else}, ett för varje alternativ. Men det finns ett annat sätt i Java och många andra språk: man kan använda \jcode{switch} som kollar flera alternativ i en och samma sats, se t.ex. \href{https://en.wikipedia.org/wiki/Switch_statement}{en.wikipedia.org/wiki/Switch\_statement}.

\Subtask Skriv in nedan kod i en kodeditor. Spara med namnet \texttt{Switch.java} och kompilera filen med kommandot \texttt{javac Switch.java}. Kör den med \texttt{java Switch} och ange din favoritgrönsak som argument till programmet. Vad händer? Förklara hur \jcode{switch}-satsen fungerar.

\javainputlisting[numbers=left,basicstyle=\ttfamily\fontsize{11}{12}\selectfont]{examples/Switch.java}

\Subtask \label{subtask:break} Vad händer om du tar bort \jcode{break}-satsen på rad 16?




\Task \label{task:vegomatch} \emph{Matcha på konstanta värden.} I Scala finns ingen \jcode{switch}-sats. I stället har Scala ett \code{match}-uttryck som är mer kraftfullt. Dock saknar Scala nyckelordet \jcode{break} och Scalas \code{match}-uttryck kan inte ''falla igenom'' som skedde i uppgift \ref{task:switch}\ref{subtask:break}.

\Subtask \label{subtask:vegomatch} Skriv nedan program med en kodeditor och spara i filen \texttt{Match.scala}. Kompilera med \texttt{scalac Match.scala}. Kör med \texttt{scala Match} och ge som argument din favoritgrönsak. Vad händer? Förklara hur ett \code{match}-uttryck fungerar.

\scalainputlisting[numbers=left,basicstyle=\ttfamily\fontsize{11}{12}\selectfont]{examples/Match.scala}

\Subtask Vad blir det för felmeddelande om du tar bort case-grenen för defaultvärden och indata väljs så att inga case-grenar matchar? Är det ett exekveringsfel eller ett kompileringsfel?


\Subtask\Pen Beskriv några skillnader i syntax och semantik mellan Javas flervalssats \jcode{switch} och Scalas flervalsuttryck \code{match}.




\Task \emph{Gard i case-grenar.} Med hjälp en gard \Eng{guard} i en case-gren kan man begränsa med ett villkor om grenen ska väljas.

Utgå från koden i uppgift \ref{task:vegomatch}\ref{subtask:vegomatch} och byt ut case-grenen för \code{'g'}-matchning till nedan variant med en gard med nyckelordet \code{if} (notera att det inte behövs parenteser runt villkoret):
\begin{Code}
    case 'g' if math.random > 0.5 => "gurka är gott ibland..."
\end{Code}
Kompilera om och kör programmet upprepade gånger med olika indata tills alla grenar i \code{match}-uttrycket har exekverats. Förklara vad som händer.

\Task \label{task:match-caseclass} \emph{Mönstermatcha på attributen i case-klasser.} Scalas \code{match}-uttryck är extra kraftfulla om de används tillsammans med \code{case}-klasser: då kan attribut extraheras automatiskt och bindas till lokala variabler direkt i case-grenen som nedan exempel visar (notera att \code{v} och \code{rutten} inte behöver deklareras explicit). Detta kallas för \textbf{mönstermatchning}.

\Subtask \label{subtask:autobinding-match} Vad skrivs ut nedan? Varför? Prova att byta namn på \code{v} och \code{rutten}.
\begin{REPL}
scala> case class Gurka(vikt: Int, ärRutten: Boolean)
scala> val g = Gurka(100, true)
scala> g match { case Gurka(v,rutten) => println("G" + v + rutten) }
\end{REPL}

\Subtask Skriv sedan nedan i REPL och tryck TAB två gånger efter punkten. Vad har \code{unapply}-metoden för resultattyp?
\begin{REPL}
scala> Gurka.unapply   // Tryck TAB två gånger
\end{REPL}
\begin{Background}
Case-klasser får av kompilatorn automatiskt ett kompanjonsobjekt \Eng{companion object}, i detta fallet \code{object Gurka}. Det objektet får av kompilatorn automatiskt en \code{unapply}-metod. Det är \code{unapply} som anropas ''under huven'' när case-klassernas attribut extraheras vid mönstermatchning, men detta sker alltså automatiskt och man behöver inte explicit nyttja \code{unapply} om man inte själv vill implementera s.k. extraherare \Eng{extractors}; om du är nyfiken på detta, se fördjupningsuppgift \ref{task:extractor}.
\end{Background}

\Subtask Anropa \code{unapply}-metoden enligt nedan. Vad blir resultatet?
\begin{REPL}
scala> Gurka.unapply(g)
\end{REPL}
Vi ska i senare uppgifter undersöka hur typerna \code{Option} och \code{Some} fungerar och hur man kan ha nytta av dessa i andra sammanhang.

\Subtask Spara programmet nedan i filen \texttt{vegomatch.scala} och kompilera med \code{scalac vegomatch.scala} och kör med \code{scala vegomatch.Main 1000} i terminalen. Förklara hur predikatet \code{ärÄtvärd} fungerar.
\scalainputlisting[numbers=left,basicstyle=\ttfamily\fontsize{11}{12}\selectfont]{examples/vegomatch.scala}



\Task Man kan åstadkomma urskiljningen av de ätbara grönsakerna i uppgift \ref{task:match-caseclass} med polymorfism i stället för \code{match}.

\Subtask Gör en ny variant av ditt program enligt nedan riktlinjer och spara den modifierade koden i filen \texttt{vegopoly.scala} och kompilera och kör.
\begin{itemize}[noitemsep]
\item Ta bort predikatet \code{ärÄtvärd} i objektet \code{Main} och inför i stället en abstrakt metod \code{def ärÄtbar: Boolean} i traiten \code{Grönsak}.
\item Inför konkreta \code{val}-medlemmar i respektive grönsak som definierar ätbarheten.
\item Ändra i huvudprogrammet i enlighet med ovan ändringar så att \code{ärÄtvärd} anropas som en metod på de skördade grönsaksobjekten när de ätvärda ska filtreras ut.
\end{itemize}

\Subtask Lägg till en ny grönsak \code{case class Broccoli} och definiera dess ätbarhet. Ändra i slump-funktionerna så att broccoli blir ovanligare än gurka.

\Subtask\Pen Jämför lösningen med \code{match} i uppgift \ref{task:match-caseclass} och lösningen ovan med polymorfism. Vilka är för- och nackdelarna med respektive lösning? Diskutera två olika situationer på ett hypotetiskt företag som utvecklar mjukvara för jordbrukssektorn: 1) att uppsättningen grönsaker inte ändras särskilt ofta medan definitionerna av ätbarhet ändras väldigt ofta och 2) att uppsättningen grönsaker ändras väldigt ofta men att ätbarhetsdefinitionerna inte ändras särskilt ofta.



\Task \emph{Matcha på case-objekt och nyttan med \code{sealed}.} Skapa nedan kod i en editor, och klistra in i REPL med kommandot \code{:pa}. Notera nyckelordet \code{sealed} som används för att försegla en typ. En \textbf{förseglad typ} måste ha alla sina subtyper i en och samma kodfil.
\begin{Code}
sealed trait Färg
object Färg {
  val values = Vector(Spader, Hjärter, Ruter, Klöver)
}
case object Spader  extends Färg
case object Hjärter extends Färg
case object Ruter   extends Färg
case object Klöver  extends Färg
\end{Code}

\Subtask Skapa en funktion \code{def parafärg(f: Färg): Färg} i en editor, som med hjälp av ett match-uttryck returnerar parallellfärgen till en färg. Parallellfärgen till \code{Hjärter} är \code{Ruter} och vice versa, medan parallellfärgen till \code{Klöver} är \code{Spader} och vice versa. Klistra in funktionen i REPL.
\begin{REPL}
scala> parafärg(Spader)
scala> val xs = Vector.fill(5)(Färg.values((math.random * 4).toInt))
scala> xs.map(parafärg)
\end{REPL}

\Subtask Vi ska nu undersöka vad som händer om man glömmer en av case-grenarna i matchningen i \code{parafärg}? ''Glöm'' alltså avsiktligt en av case-grenarna och klistra in den nya \code{parafärg} med den ofullständiga matchningen. Hur lyder varningen? Kommer varningen vid körtid eller vid kompilering?

\Subtask Anropa \code{parafärg} med den ''glömda'' färgen. Hur lyder felmeddelandet? Är det ett kompileringsfel eller ett körtidsfel?

\Subtask\Pen Förklara vad nyckelordet \code{sealed} innebär och vilken nytta man kan ha av att \textbf{försegla} en supertyp.


\Task \emph{Betydelsen av små och stora begynnelsebokstäver vid matchning.} För att åstadkomma att namn kan bindas till variabler vid matchning utan att de behöver deklareras i förväg (som vi såg i uppgift \ref{task:match-caseclass}\ref{subtask:autobinding-match}) så har identifierare med liten begynnelsebokstav fått speciell betydelse: den tolkas av kompilatorn som att du vill att en variabel  binds till ett värde vid matchningen. En identifierare med stor begynnelsebokstav tolkas däremot som ett konstant värde (t.ex. ett case-objekt eller ett case-klass-mönster).

\Subtask \emph{En case-gren som fångar allt}. En case-gren med en identifierare med liten begynnelsebokstav som saknar gard kommer att matcha allt. Prova nedan i REPL, men försök lista ut i förväg vad som kommer att hända. Vad händer?
\begin{REPL}
scala> val x = "urka"
scala> x match {
         case str if str.startsWith("g") => println("kanske gurka")
         case vadsomhelst => println("ej gurka: " + vadsomhelst)
       }
scala> val g = "gurka"
scala> g match {
         case str if str.startsWith("g") => println("kanske gurka")
         case vadsomhelst => println("ej gurka: " + vadsomhelst)
       }
\end{REPL}

\Subtask \emph{Fallgrop med små begynnelsbokstäver.} Innan du provar nedan i REPL, försök gissa vad som kommer att hända. Vad händer? Hur lyder varningarna och vad innebär de?
\begin{REPL}
scala> val any: Any = "gurka"
scala> case object Gurka
scala> case object tomat
scala> any match {
         case Gurka => println("gurka")
         case tomat => println("tomat")
         case _ => println("allt annat")
       }
\end{REPL}

\Subtask \emph{Använd backticks för att tvinga fram match på konstant värde.} Det finns en utväg om man inte vill att kompilatorn ska skapa en ny lokal variabel: använd specialtecknet \emph{backtick}, som skrivs \`{} och kräver speciella tangentbordstryck.\footnote{Fråga någon om du inte hittar hur man gör backtick \`{} på ditt tangentbord.}  Gör om föregående uppgift men omgärda nu identifieraren \code{tomat} i tomat-case-grenen med backticks, så här: \code{  case `tomat` => ...}



\Task \emph{Använda \code{Option} och matcha på värden som kanske saknas.} Man behöver ofta skriva kod för att hantera värden som eventuellt saknas, t.ex. saknade telefonnummer i en persondatabas. Denna situation är så pass vanlig att många språk har speciellt stöd för saknande värden.

I Java\footnote{Scala har också \code{null} men det behövs bara vid samverkan med Java-kod.} används värdet \code{null} för att indikera att en referens saknar värde. Man får då komma ihåg att testa om värdet saknas varje gång sådana värden ska behandlas, t.ex. med \code+if (ref != null) { ...} else { ... }+. Ett annat vanligt trick är att låta \code{-1} indikera saknade positiva heltal, till exempel saknade index, som får behandlas med \code+if (i != -1) { ...} else { ... }+.

I Scala finns en speciell typ \code{Option} som möjliggör smidig och typsäker hantering av saknade värden. Om ett kanske saknat värde packas in i en \code{Option} \Eng{wrapped in an Option}, finns det i en speciell slags samling som bara kan innehålla \emph{inget} eller \emph{något} värde, och alltså har antingen storleken \code{0} eller \code{1}.

\Subtask Förklara vad som händer nedan.
\begin{REPL}
scala> var kanske: Option[Int] = None
scala> kanske.size
scala> kanske = Some(42)
scala> kanske.size
scala> kanske.isEmpty
scala> kanske.isDefined
scala> def ökaOmFinns(opt: Option[Int]): Option[Int] = opt match {
         case Some(i) => Some(i + 1)
         case None    => None
       }
scala> val annanKanske = ökaOmFinns(kanske)
scala> def öka(i: Int) = i + 1
scala> val merKanske = kanske.map(öka)
\end{REPL}

\Subtask Mönstermatchingen ovan är minst lika knölig som en \code{if}-sats, men tack vare att en \code{Option} är en slags (liten) samling finns det smidigare sätt. Förklara vad som händer nedan.
\begin{REPL}
val meningen = Some(42)
val ejMeningen = Option.empty[Int]
meningen.map(_ + 1)
ejMeningen.map(_ + 1)
ejMeningen.map(_ + 1).orElse(Some("saknas")).foreach(println)
meningen.map(_ + 1).orElse(Some("saknas")).foreach(println)
\end{REPL}

\Subtask \emph{Samlingsmetoder som ger en \code{Option}.} Förklara för varje rad nedan vad som händer. En av raderna ger ett felmeddelande; vilken rad och vilket felmeddelande?
\begin{REPL}
val xs = (42 to 84 by 5).toVector
val e = Vector.empty[Int]
xs.headOption
xs.headOption.get
xs.headOption.getOrElse(0)
xs.headOption.orElse(Some(0))
e.headOption
e.headOption.get
e.headOption.getOrElse(0)
e.headOption.orElse(Some(0))
Vector(xs, e, e, e)
Vector(xs, e, e, e).map(_.lastOption)
Vector(xs, e, e, e).map(_.lastOption).flatten
xs.lift(0)
xs.lift(1000)
e.lift(1000).getOrElse(0)
xs.find(_ > 50)
xs.find(_ < 42)
e.find(_ > 42).foreach(_ => println("HITTAT!"))
\end{REPL}

\Subtask\Pen Vilka är fördelerna med \code{Option} jämfört med \code{null} eller \code{-1} om man i sin kod glömmer hantera saknade värden?

\Task \emph{Kasta undantag.} Om man vill signalera att ett fel eller en onormal situtation uppstått så kan man \textbf{kasta} \Eng{throw} ett \textbf{undantag} \Eng{exception}. Då avbryts programmet direkt med ett felmeddelande, om man inte väljer att \textbf{fånga} \Eng{catch} undantaget.

\Subtask Vad händer nedan?
\begin{REPL}
scala> throw new Exception("PANG!")
scala> java.lang.   // Tryck TAB efter punkten
scala> throw new IllegalArgumentException("fel fel fel")
scala> val carola = try {
         throw new Exception("stormvind!")
         42
       } catch { case e: Throwable => println("Fångad av en " + e); -1 }
\end{REPL}
\Subtask\Pen Nämn ett par undantag som finns i paketet \code{java.lang} som du kan gissa vad de innebär och i vilka situationer de kastas.

\Subtask\Pen Vilken typ har variabeln \code{carola} ovan? Vad hade typen blivit om catch-grenen hade returnerat en sträng i stället?

\Task \label{task:javatry} \emph{Fånga undantantag i Java med en \jcode{try}-\jcode{catch}-sats.} Det finns som vi såg i förra uppgiften inbyggt stöd i JVM för att hantera när program avbryts på oväntade sätt, t.ex. på grund av division med noll eller ej förväntade indata från användaren. Skriv in nedan Java-program i en editor och spara i en fil med namnet \texttt{TryCatch.java} och kompilera med \texttt{javac TryCatch.java} i terminalen.

\javainputlisting[numbers=left,basicstyle=\ttfamily\fontsize{11}{12}\selectfont]{examples/TryCatch.java}

\Subtask Förklara vad som händer när du kör programmet med olika indata:
\begin{REPL}
$ java TryCatch 42
$ java TryCatch 0
$ java TryCatch safe 42
$ java TryCatch safe 0
$ java TryCatch
\end{REPL}

\Subtask Vad händer om du ''glömmer bort'' raden 15 och därmed missar att initialisera input? Hur lyder felmeddelandet? Är det ett körtidsfel eller kompileringsfel?

\Subtask\Pen Beskriv några skillnader och likheter i syntax och semantik mellan \code{try}-\code{catch} i Java respektive Scala.



\Task \emph{Fånga undantantag i Scala med \code{scala.util.Try}.} I paketet \code{scala.util} finns typen \code{Try} med stort T som är som en slags samling som kan innehålla antingen ett ''lyckat'' eller ''misslyckat'' värde. Om beräkningen av värdet lyckades och inga undantag kastas blir värdet inkapslat i en \code{Success}, annars blir undantaget inkapslat i en \code{Failure}. Man kan extrahera värdet, respektive undantaget, med mönstermatchning, men det är oftast smidigare att använda samlingsmetoderna \code{map} och \code{foreach}, i likhet med hur \code{Option} används. Det finns även en smidig metod \code{recover} på objekt av typen \code{Try} där man kan skicka med kod som körs om det uppstår en undantagssituation.

\Subtask Förklara vad som händer nedan.
\begin{REPL}
scala> def pang = throw new Exception("PANG!")
scala> import scala.util.{Try, Success, Failure}
scala> Try{pang}
scala> Try{pang}.recover{case e: Throwable => s"desarmerad bomb: $e"}
scala> Try{"tyst"}.recover{case e: Throwable => s"desarmerad bomb: $e"}
scala> def kanskePang = if (math.random > 0.5) "tyst" else pang
scala> def kanskeOk = Try{ kanskePang}
scala> val xs = Vector.fill(100)(kanskeOk)
scala> xs(13) match {
         case Success(x) => ":)"
         case Failure(e) => ":( " + e
       }
scala> x(13).isSuccess
scala> x(13).isFailure
scala> xs.count(_.isFailure)
scala> xs.find(_.isFailure)
scala> val badOpt = xs.find(_.isFailure)
scala> val goodOpt = xs.find(_.isSuccess)
scala> badOpt
scala> badOpt.get
scala> badOpt.get.get
scala> badOpt.map(_.getOrElse("bomben desarmerad!")).get
scala> goodOpt.map(_.getOrElse("bomben desarmerad!")).get
scala> xs.map(_.getOrElse("bomben desarmerad!")).foreach(println)
scala> xs.map(_.toOption)
scala> xs.map(_.toOption).flatten
scala> xs.map(_.toOption).flatten.size
\end{REPL}


\Subtask Vad har funktionen \code{pang} för returtyp?

\Subtask\Pen Varför får funktionen \code{kanskePang} den härledda returtypen \code{String}?

\Task \emph{Metoden \code{equals}.}  Om man överskuggar den befintliga metoden \code{equals} så kommer metoden \code{==} att fungera annorlunda. Man kan då själv åstadkomma innehållslikhet i stället för referenslikhet. Vi börjar att studera den befintliga equals med referenslikhet.

\Subtask \label{subtask:refequals} Vad händer nedan? Om du trycker TAB \emph{två} gånger efter ett metodnamn får du se metodens signatur. Vilken signatur har metoden \code{equals}?
\begin{REPL}
scala> class Gurka(val vikt: Int, ärÄtbar: Boolean)
scala> val g1 = new Gurka(42, true)
scala> val g2 = g1
scala> val g3 = new Gurka(42, true)
scala> g1 == g2
scala> g1 == g3
scala> g1.equals  // tryck TAB två gånger
\end{REPL}

\Subtask\Pen Rita minnessituationen efter rad 4.

\Subtask \emph{Överskugga metoderna \code{equals} och \code{hashCode}.}

\begin{Background}
Det visar sig förvånande komplicerat att implementera innehållslikhet med metoden \code{equals} så att den ger bra resultat under alla speciella omständigheter. Till exempel måste man även överskugga en metod vid namn \code{haschCode} om man överskuggar \code{equals}, eftersom dessa båda används gemensamt av effektivitetsskäl för att skapa den interna lagringen av objekten i vissa samlingar. Om man missar det kan objekt bli ''osynliga'' i \code{hashCode}-baserade samlingar -- men mer om detta i senare kurser. Om objekten ingår i en öppen arvshierarki blir det också mer komplicerat; det är enklare om man har att göra med finala klasser. Dessutom krävs speciella hänsyn om klassen har en typparameter.
\end{Background}

\noindent Definera klassen nedan i REPL med överskuggade \code{equals} och \code{hashCode}; den ärver inte något och är final.

\begin{Code}
// fungerar fint om klassen är final och inte ärver något
final class Gurka(val vikt: Int, ärÄtbar: Boolean) {
  override def equals(other: Any): Boolean = other match {
    case that: Gurka => this.vikt == that.vikt
    case _ => false
  }
  override def hashCode: Int = (vikt, ärÄtbar).## //förklaras sen
}
\end{Code}
\Subtask Vad händer nu nedan, där \code{Gurka} nu har en överskuggad \code{equals} med innehållslikhet?
\begin{REPL}
scala> val g1 = new Gurka(42, true)
scala> val g2 = g1
scala> val g3 = new Gurka(42, true)
scala> g1 == g2
scala> g1 == g3
\end{REPL}
\Subtask\Pen Hur märker man ovan att den överskuggade \code{equals} medför att \code{==} nu ger innehållslikhet? Jämför med deluppgift \ref{subtask:refequals}.

I uppgift \ref{task:equals:Complex} får du prova på att följa det fullständiga receptet i 8 steg för att överskugga en \code{equals} enligt konstens alla regler. I efterföljande kurs kommer mer träning i att hantera innehållslikhet och hash-koder. I Scala får man ett objekts hash-kod med metoden \code{##}.\footnote{Om du är nyfiken på hash-koder, läs mer här:
\href{https://en.wikipedia.org/wiki/Java_hashCode()}{en.wikipedia.org/wiki/Java\_hashCode()}.}



\clearpage
\ExtraTasks %%%%%%%%%%%%%%%%%%%

\Task \label{task:plynomial} \emph{Polynom}. Med hjälp av koden nedan, kan man göra följande:
\begin{REPL}
scala> :pa polynomial.scala

scala> import polynomial._

scala> Const(1) * x
res0: polynomial.Term = x

scala> (x*5)^2
res1: polynomial.Prod = 25x^2

scala> Poly(x*(-5), y^4, (z^2)*3)
res2: polynomial.Poly = -5x + y^4 + 3z^2

\end{REPL}

\Subtask\Pen Förklara vad som händer ovan genom att studera koden för \code{object polynomial} nedan i filen \code{polynomial.scala}.\footnote{Koden finns även här:\\ \href{https://github.com/lunduniversity/introprog/tree/master/compendium/examples/polynomial}{github.com/lunduniversity/introprog/tree/master/compendium/examples/polynomial}}

\scalainputlisting[numbers=left,basicstyle=\ttfamily\fontsize{10}{12}\selectfont]{examples/polynomial/polynomial.scala}

\Subtask Bygg vidare på \code{object polynomial} och implementera addition mellan olika termer.


\Task\Pen Studera dokumentationen för \code{Option} här och se om du känner igen några av metoderna som också finns på samlingen \code{Vector}:\\ \href{http://www.scala-lang.org/api/current/index.html#scala.Option}{www.scala-lang.org/api/current/index.html\#scala.Option}
\\Förklara hur metoden \code{contains} på en \code{Option} fungerar med hjälp av dokumentationens exempel.



\Task Gör motsvarande program i Scala som visas i uppgift \ref{task:javatry}, men utnyttja att Scalas \code{try}-\code{catch} är ett uttryck. Kompilera och kör och testa så att de ur användarens synvinkel fungerar precis på samma sätt. Notera de viktigaste skillnaderna mellan de båda programmen.




\clearpage

\AdvancedTasks %%%%%%%%%%%%%%%%%

\Task Bygg vidare på \code{object polynomial} i uppgift \ref{task:plynomial} på sidan \pageref{task:plynomial} och implementera metoden \code{def reduce: Poly} i case-klassen \code{Poly} som förenklar polynom om flera \code{Prod}-termer kan adderas.

\Task\Pen Läs om hash-koder här: \href{https://en.wikipedia.org/wiki/Java_hashCode()}{en.wikipedia.org/wiki/Java\_hashCode()} \\
Vad ger metoden \code{##} i scala.Any för resultat? Läs dokumentationen här: \\ \href{http://www.scala-lang.org/api/current/#scala.Any}{www.scala-lang.org/api/current/\#scala.Any}

\Task \emph{Typsäker innehållstest med metoden \code{===}.} Metoderna \code{equals} och \code{==} tillåter jämförelse med vad som helst. Ibland vill man ha en typsäker innehållsjämförelse som bara tillåter jämförelse av objekt av en mer specifik typ och ger kompileringsfel annars. Man brukar då definiera en metod \code{===} som har en parameter \code{that} som har en så specifik typ som önskas. Inför nedan abstrakta metod \code{===} i traiten \code{polynomial.Term} i uppgift \ref{task:plynomial} på sidan \pageref{task:plynomial} och överskugga den sedan i alla subklasser till Term. Testa så att du får kompileringsfel om du försöker jämföra en \code{Term} med något helt annat, t.ex. en \code{String} eller \code{Vector}.
\begin{Code}
  def ===(that: Term): Boolean
\end{Code}


\Task \label{task:equals:Complex} \emph{Överskugga \code{equals} med innehållslikhet även för icke-finala klasser.} Nedan visas delar av klassen \code{Complex} som representerar ett komplext tal med realdel och imaginärdel. I stället för att, som man ofta gör i Scala, använda en case-klass och en \code{equals}-metod som automatiskt ger innehållslikhet, ska du träna på att implementera en egen \code{equals}.
\begin{Code}
class Complex(re: Double, im: Double) {
  def abs: Double = math.hypot(re, im)
  override def toString = s"Complex($re, $im)"
  def canEqual(other: Any): Boolean = ???
  override def hashCode: Int  = ???
  override def equals(other: Any): Boolean = ???
}
case object Complex {
  def apply(re: Double, im: Double): Complex = new Complex(re, im)
}
\end{Code}
Följ detta \textbf{recept}\footnote{Detta recept bygger på \url{http://www.artima.com/pins1ed/object-equality.html}} i 8 steg för att överskugga \code{equals} med innehållslikhet som fungerar även för klasser som inte är \code{final}:

\begin{enumerate}[leftmargin=*]
\item Inför denna metod: \code{ def canEqual(other: Any): Boolean}\\Observera att typen på parametern ska vara \code{Any}. Om detta görs i en subklass till en klass som redan implementerat \code{canEqual}, behövs även \code{override}.

\item Metoden \code{canEqual} ska ge \code{true} om \code{other} är av samma typ som \code{this}, alltså till exempel: \\
\code{def canEqual(other: Any): Boolean = other.isInstanceOf[Complex]}

\item Inför metoden \code{equals} och var noga med att parametern har typen \code{Any}: \\ \code{override def equals(other: Any): Boolean}

\item Implementera metoden \code{equals} med ett match-uttryck som börjar så här:
\code|other match { ... } |

\item Match-uttrycket ska ha två grenar. Den första grenen ska ha ett typat mönster för den klass som ska jämföras: \\ \code{  case that: Complex =>}

\item Om du implementerar \code{equals} i den klass som inför \code{canEqual}, börja uttrycket med: \\ \code{(that canEqual this) &&} \\
och skapa därefter en fortsättning som baseras på innehållet i klassen, till exempel: \code{this.re == that.re && this.im == that.im} \\
Om du överskuggar en \textit{annan} equals än den standard-equals som finns i \code{AnyRef}, vill du förmodligen börja det logiska uttrycket med att anropa superklassens equals-metod:
 \code{super.equals(that) && } men du får fundera noga på vad likhet av underklasser egentligen ska innebära i ditt speciella fall.

\item Den andra grenen i matchningen ska vara:
\code{case _ => false}

\item Överskugga \code{hashCode}, till exempel genom att göra en tupel av innehållet i klassen och anropa metoden \code{##} på tupeln så får du i en bra hashcode: \\
\code{override def hashCode: Int  = (re, im).## }

\end{enumerate}


\Task Bygg vidare på exemplet nedan och överskugga equals vid arv, genom att följa receptet i uppgift \ref{task:equals:Complex}.
\begin{Code}
trait Number {
  override def equals(other: Any): Boolean = ???
}
class Complex(re: Double, im: Double) extends Number {
  override def equals(other: Any): Boolean = ???
}
class Rational(numerator: Int, denominator: Int) extends Number {
  override def equals(other: Any): Boolean = ???
}
\end{Code}


\Task Läs om olika speciella matchningar här: \\
\href{http://www.artima.com/pins1ed/case-classes-and-pattern-matching.html}{www.artima.com/pins1ed/case-classes-and-pattern-matching.html}

\Subtask Prova variabelbinding med \texttt{@} i en matchning i REPL.

\Subtask Prova sekvensmönster med \texttt{\_} och \texttt{\_*} i en matching i REPL.

\Task \label{task:extractor} Läs mer om extraktorer här: \\ \href{http://www.artima.com/pins1ed/extractors.html}{www.artima.com/pins1ed/extractors.html} \\
Skapa ditt eget extraktor-objekt för http-addresser som i t.ex.: \\
\texttt{http://my.host.domain/path/to/this} \\ extraherar \texttt{my.host.domain} och \texttt{path/to/this} med metoden \code{unapply} och testa i en matchning.

%\Task \TODO \emph{flatten och flatMap med Option och Try}
%Ska detta vara ordinarie uppgift eller fördjupning???


%\Task \TODO \emph{partiella funktioner och metoderna collect och collectFirst på samlingar}
%Ska detta vara ordinarie uppgift eller fördjupning???

\Task En rejäl utmaning: Implementera polynomdivision på lämpligt sätt genom att bygga vidare på  \code{object polynomial} i  uppgift \ref{task:plynomial} på sidan \pageref{task:plynomial}.  \\ Läs mer om polynomdivision här: \href{https://sv.wikipedia.org/wiki/Polynomdivision}{sv.wikipedia.org/wiki/Polynomdivision}

%!TEX encoding = UTF-8 Unicode
%!TEX root = ../compendium1.tex

% INGEN LAB DENNA VECKA


%%!TEX encoding = UTF-8 Unicode

%!TEX root = ../compendium1.tex

%!TEX encoding = UTF-8 Unicode
\chapter{TODO: Kontextuella abstraktioner}\label{chapter:W11}
Begrepp som ingår i denna veckas studier:
\begin{itemize}[noitemsep,label={$\square$},leftmargin=*]
\item syntaxskillnader mellan Scala och Java
\item klasser i Scala och Java
\item referensvariabler i Java
\item enkla värden i Java
\item primitiva typer i Java
\item referenstilldelning och värdetilldelning i Java
\item alternativ konstruktor i Scala och Java
\item for-sats i Java
\item for-each-sats i Java
\item java.util.ArrayList
\item autoboxing i Java
\item wrapperklasser i Java
\item samlingar i Java
\item scala.jdk.CollectionConverters
\item namnkonventioner för konstanter i Scala och Java
\item kodläsbarhet
\item idiom
\item kodningsstandard\end{itemize}


%!TEX encoding = UTF-8 Unicode
\chapter{TODO: Kontextuella abstraktioner}\label{chapter:W11}
Begrepp som ingår i denna veckas studier:
\begin{itemize}[noitemsep,label={$\square$},leftmargin=*]
\item syntaxskillnader mellan Scala och Java
\item klasser i Scala och Java
\item referensvariabler i Java
\item enkla värden i Java
\item primitiva typer i Java
\item referenstilldelning och värdetilldelning i Java
\item alternativ konstruktor i Scala och Java
\item for-sats i Java
\item for-each-sats i Java
\item java.util.ArrayList
\item autoboxing i Java
\item wrapperklasser i Java
\item samlingar i Java
\item scala.jdk.CollectionConverters
\item namnkonventioner för konstanter i Scala och Java
\item kodläsbarhet
\item idiom
\item kodningsstandard\end{itemize}


%!TEX encoding = UTF-8 Unicode
%!TEX root = ../exercises.tex

\ifPreSolution



\Exercise{\ExeWeekELEVEN}\label{exe:W11}

\begin{Goals}
\item Kunna förklara och beskriva viktiga skillnader mellan Scala och Java.
\item Kunna översätta enkla algoritmer, klasser och singeltonobjekt från Scala till Java och vice versa.
\item Känna till vad en case-klass innehåller i termer av en Javaklass.
%\item Förstå hur autoboxing fungerar.
\item Kunna använda Javatyperna \code{List}, \code{ArrayList}, \code{Set}, \code{HashSet} och översätta till deras Scalamotsvarigheter med \code{JavaConverters}.
\item Kunna förklara hur autoboxning fungerar i Java, samt beskriva fördelar och fallgropar.
\end{Goals}

\begin{Preparations}
\item \StudyTheory{11}
\end{Preparations}

\BasicTasks %%%%%%%%%%%%%%%%

\else



\ExerciseSolution{\ExeWeekELEVEN}

\BasicTasks %%%%%%%%%%%

\fi





\WHAT{Översätta metoder från Java till Scala.}

\QUESTBEGIN

\Task  \what~  I denna uppgift ska du översätta en Java-klass som används som en modul\footnote{\href{https://en.wikipedia.org/wiki/Modular_programming}{en.wikipedia.org/wiki/Modular\_programming}} och bara innehåller statiska metoder och inget förändringsbart tillstånd som kan ändras utifrån. (I nästa uppgift ska du sedan översätta klasser med förändringsbara  tillstånd.)

Vi börjar med att göra översättningen från Java till Scala rad för rad och du ska behålla så mycket som möjligt av syntax och semantik så att Scala-koden blir så Java-lik som möjligt. I efterföljande deluppgift ska du sedan omforma översättningen så att Scala-koden blir mer idiomatisk\footnote{\href{https://sv.wikipedia.org/wiki/Idiom_\%28programmering\%29}{sv.wikipedia.org/wiki/Idiom\_\%28programmering\%29}}.

\Subtask Studera klassen \code{Hangman} nedan. Du ska översätta den från Java till Scala enlig de riktlinjer och tips som följer efter koden. Läs igenom alla riktlinjer och tips innan du börjar.

\javainputlisting[numbers=left]{examples/scalajava/Hangman.java}

\noindent\emph{Riktlinjer och tips för översättningen:}

\begin{enumerate}[noitemsep]

\item Skriv Scala-koden med en texteditor i en fil som heter \texttt{hangman1.scala} och kompilera med \code{scalac hangman1.scala} i terminalen; använd alltså \emph{inte} en IDE, så som Eclipse eller IntelliJ, utan en ''vanlig'' texteditor, t.ex. \code{atom}.

\item Översätt i denna första deluppgift rad för rad så likt den ursprungliga Java-kodens utseende (syntax)  som möjligt, med så få ändringar som möjligt. Du ska alltså ha kvar dessa Scalaovanligheter, även om det inte alls blir som man brukar skriva i Scala:
\begin{enumerate}[nolistsep, noitemsep]
\item långa indrag, \item onödiga semikolon, \item onödiga \code{()}, \item onödiga \code|{}|, \item onödiga \code{System.out}, och \item onödiga \code{return}.
\end{enumerate}

\item Försök också i denna deluppgift göra så att betydelsen (semantiken) så långt som möjligt motsvarar den i Java, t.ex. genom att använda \code{var} överallt, även där man i Scala normalt använder \code{val}.

\item En Javaklass med bara statiska medlemmar motsvarar ett singeltonobjekt i Scala, alltså en \code{object}-deklaration innehållande ''vanliga'' medlemmar.

\item För att tydliggöra att du använder Javas \code{Set} och \code{HashSet} i din Scala-kod, använd följande import-satser i \code{hangman1.scala}, som därmed döper om dina importerade namn och gör så att de inte krockar med Scalas inbyggda \code{Set}. Denna form av import går inte att göra i Java.
\begin{Code}
import java.util.{Set => JSet};
import java.util.{HashSet => JHashSet};
\end{Code}

\item Javas \code{i++} fungerar inte i Scala; man får istället skriva \code{i += 1} eller mindre vanliga \code{i = i + 1}.

\item Typparametrar i Java skrivs inom \code{<>} medan Scalas syntax för typparametrar använder \code{[]}.

\item Till skillnad från Java så har Scalas metoddeklarationer ett tilldelningstecken \code{=} efter returtypen, före kroppen.

\item Du kan ladda ner Java-koden till \code{Hangman}-klassen nedan från kursens repo%
\footnote{\href{https://github.com/lunduniversity/introprog/blob/master/compendium/examples/scalajava/Hangman.java}{github.com/lunduniversity/introprog/blob/master/compendium/examples/scalajava/Hangman.java}}. I samma bibliotek ligger även lösningarna till översättningen i Scala, men kolla \emph{inte} på dessa förrän du gjort klart översättningarna och fått dem att kompilera och köra felfritt! Tanken är att du ska träna på att läsa felmeddelande från kompilatorn och åtgärda dem i en upprepad kompilera-testa-rätta-cykel.

\end{enumerate}







\Subtask Skapa en ny fil \code{hangman2.scala} som till att börja med innehåller en kopia av din direkt-översatta Java-kod från föregående deluppgift. Omforma koden så att den blir mer som man brukar skriva i Scala, alltså mer Scala-idiomatisk. Försök förenkla och förkorta så mycket du kan utan att göra avkall på läsbarheten.

\emph{Tips och riktlinjer:}

\begin{enumerate}[nolistsep, noitemsep]

\item Kalla Scala-objektet för \code{hangman}. När man använder ett Scalaobjekt som en modul (alltså en samling funktioner i en gemensam, avgränsad namnrymd) har man gärna liten begynnelsebokstav, i likhet med konventionen för paketnamn. Ett paket är ju också en slags modul och med en namngivningskonvention som är gemensam kan man senare, utan att behöva ändra koden som använder modulen, ändra från ett singelobjekt till ett paket och vice versa om man så önskar.

\item Gör alla metoder publikt tillgängliga och låt även strängvektorn \code{hangman} vara publikt tillgänglig. Deklarera \code{hangman} som en \code{val} och konstruera den med \code{Vector}. Eftersom \code{Vector} är oföränderlig och man inte kan ärva från singelobjekt och \code{hangman} är deklarerad med \code{val} finns inga speciella risker med att göra den konstanta vektorn publik om  vi inte har något emot att annan kod kan läsa (och eventuellt göra sig beroende av) vår hänggubbetext.

\item I metoden \code{renderHangman}, använd \code{take} och \code{mkString}.

\item I metoden \code{hideSecret}, använd \code{map} i stället för en \code{for}-sats.

\item Det går att ersätta metoden \code{findAll} med det kärnfulla uttrycket \\ \code{(secret forall found)} där \code{secret} är en sträng och \code{found} är en mängd av tecken (undersök gärna i REPL hur detta fungerar). Skippa därför den metoden helt och använd det kortare uttrycket direkt.

\item I metoden \code{makeGuess}, i stället för \code{Scanner}, använd \code{scala.io.StdIn.readLine}.

\item Om du vill träna på att använda rekursion i stället för imperativa loopar: Gör metoden \code{makeGuess} rekursiv i stället för att använda \code{do}-\code{while}.

\item I metoden \code{download}, i stället för \code{java.net.URL} och \code{java.util.ArrayList}, använd \code{scala.io.Source.fromURL(address, coding).getLines.toVector} och gör en lokal import av \code{scala.io.Source.fromURL} överst i det block där den används. Det går inte att ha lokala \code{import}-satser i Java.

\item Låt metoden \code{download} returnera en \code{Option[String]} som i fallet att nedladdningen misslyckas returnerar \code{None}.

\item I metoden \code{download}, i stället för \code{try}-\code{catch} använd \code{scala.util.Try} och dess smidiga metoder \code {recover} och \code{toOption}.

\item Om du vill träna på att använda rekursion i stället för imperativa loopar: Använd, i stället för \code{while}-satsen i metoden \code{play}, en lokal rekursiv funktion med denna signatur:
\begin{Code}
  def loop(found: Set[Char], bad: Int): (Int, Boolean)
\end{Code}
Funktionen \code{loop} returnerar en 2-tupel med antalet felgissningar och \code{true} om man hittat alla bokstäver eller \code{false} om man blev hängd.

\end{enumerate}





\SOLUTION


\TaskSolved \what
     %%%TODO number  1 %%%starts with: \emph{Översätta algoritmer och %%%

\SubtaskSolved  \scalainputlisting[numbers=left,basicstyle=\ttfamily\fontsize{10.3}{12}\selectfont]{examples/scalajava/hangman1.scala}

\SubtaskSolved  \scalainputlisting[numbers=left,basicstyle=\ttfamily\fontsize{11.2}{13}\selectfont]{examples/scalajava/hangman2.scala}



\QUESTEND






\WHAT{Översätta mellan klasser i Scala och klasser i Java.}

\QUESTBEGIN

\Task  \what~
Klassen \code{Point} nedan är en modell av en punkt som kan sparas på begäran i en lista. Listan är privat för kompanjonsobjektet och kan skrivas ut med en metod \code{showSaved}. I koden används en \code{ArrayBuffer}, men i framtiden vill man, vid behov, kunna ändra från \code{ArrayBuffer} till en annan sekvenssamlingsimplementation, t.ex. \code{ListBuffer}, som uppfyller egenskaperna hos supertypen \code{Buffer}, men har andra prestandaegenskaper för olika operationer. Därför är attributet \code{saved} i kompanjonsobjektet deklarerat med den mer generella typen.

\scalainputlisting[numbers=left]{examples/scalajava/Point.scala}

\Subtask Översätt klassen \code{Point} ovan från Scala till Java. Vi ska i nästa deluppgift kompilera både Scala-programmet ovan och ditt motsvarande Java-program i terminalen och testa i REPL att klasserna har motsvarande funktionalitet.

\emph{Tips och riktlinjer:}
\begin{enumerate}[nolistsep, noitemsep]
\item För att namnen inte ska krocka i våra kommande tester, kalla Javatypen för \code{JPoint}.
\item  I stället för Scalas \code{ArrayBuffer} och \code{Buffer}, använd Javas \code{ArrayList} och \code{List} som båda ligger i paketet \code{java.util}.
\item Undersök dokumentationen för \code{java.util.List} för att hitta en motsvarighet till \code{prepend} för att lägga till i början av listan.
\item I stället för default-argumentet i Scalas primärkonstruktor, använd en extra Java-konstruktor.
\item Det finns inga singelobjekt och inga kompanjonsobjekt i Java; istället kan man använda statiska klassmedlemmar. Placera kompanjonsobjektets medlemmars motsvarigheter \emph{inuti} Java-klassen och gör dem till \jcode{static}-medlemmar.
\item Kod i klasskroppen i Scalaklassen, så som if-satsen på rad 4, placeras i lämplig konstruktor i Javaklassen.
\item Utskrifter med \code{print} och \code{println} behöver i Java föregås av \code{System.out}.
\item Det finns inget nyckelord \code{override} i Java, men en s.k. annotering som ger samma kompilatorhjälp. Den skrivs med ett snabel-a och stor begynnelsebokstav, så här: \jcode{ @Override }  före metoddeklarationen.
\item I Java används konventionen att börja getter-metoder med ordet \code{get}, t.ex. \code{getX()}.
\item Det finns ingen motsvarighet till \code{mkString} för \code{List} så du behöver själv gå igenom listan och hämta elementreferenser för utskrift med en \jcode{for}-loop. Notera att efter sista elementet ska radbrytning göras i utskriften och att inget komma ska skrivas ut efter sista elementet.
\item I Java behövs en ny \jcode{import}-deklaration om man vill importera ännu en typ från samma paket. Man kan även i Java använda asterisk \code{*}, (motsvarande \code{_} i Scala), för att importera allt i ett paket, men då får man med alla möjliga namn och det vill man kanske inte.
\item Metoder i Java slutar med \code{()} om de saknar parametrar.
\item Alla satser i Java slutar med lättglömda semikolon. (Efter att man i skrivit mycket Javakod och växlar till Scalakod är det svårt att vänja sig av med att skriva semikolon...)
\end{enumerate}


\Subtask Starta REPL i samma bibliotek som du kompilerat kodfilerna. Testa så att klasserna \code{Point} och \code{JPoint} beter sig på samma vis enligt nedan. Skriv även testkod i REPL för att avläsa de attributvärden som har getters och undersök att allt funkar som det ska.
\begin{REPLnonum}
$ scalac Point.scala
$ javac JPoint.java
$ scala
scala> val (p, jp) = (new Point, new JPoint)
scala> p.distanceTo(new Point(3, 4))
scala> Point.showSaved
scala> jp.distanceTo(new JPoint(3, 4))
scala> JPoint.showSaved
scala> for (i <- 1 to 10) { new Point(i, i, true) }
scala> Point.showSaved
scala> for (i <- 1 to 10) { new JPoint(i, i, true) }
scala> JPoint.showSaved
\end{REPLnonum}


\Subtask Översätt nedan Javaklass \code{JPerson} till en \code{case class Person} i Scala med  motsvarande funktionalitet.


\javainputlisting[numbers=left]{examples/scalajava/JPerson.java}


\Subtask\Pen Undersök i REPL vilken funktionalitet i Scala-case-klassen \code{Person} som \emph{inte} är implementerad i Java-klassen \code{JPerson} ovan. Skriv upp namnen på några av case-klassens extra metoder samt deras signatur genom att för en \code{Person}-instans, och för kompanjonsobjektet \code{Person}, trycka på TAB-tangenten. Prova några av de extra metoderna i REPL och förklara vad de gör.

\begin{REPL}
scala> val p = Person("Björn", 49)
scala> p.      // tryck TAB en gång
scala> Person. // tryck TAB en gång
scala> p.copy  // tryck TAB en gång
scala> p.copy()
scala> p.copy(age = p.age + 1)
scala> Person.unapply(p)
\end{REPL}


\SOLUTION


\TaskSolved \what
     %%%TODO number  2 %%%starts with: \emph{Översätta mellan klasser %%%

\SubtaskSolved   \javainputlisting[numbers=left]{examples/scalajava/JPoint.java}

\SubtaskSolved   -

\SubtaskSolved   \begin{Code}
case class Person(name: String, age: Int = 0)
\end{Code}

\SubtaskSolved  p.*TAB* - copy, producArity, ProductIterator, productElement, productPrefix

Person.*TAB* - apply, curried, tupled, unapply

\begin{REPLnonum}
scala> p.copy
   def copy(name: String,age: Int): Person

scala> p.copy()
res0: Person = Person(Björn,49)

scala> p.copy(age = p.age + 1)
res1: Person = Person(Björn,50)

scala> Person.unapply(p)
res2: Option[(String, Int)] = Some((Björn,49))
\end{REPLnonum}



\QUESTEND






\WHAT{Auto(un)boxing.}

\QUESTBEGIN

\Task  \what~  I JVM måste typparametern för generiska klasser vara av referenstyp. I Scala löser kompilatorn detta åt oss så att vi ändå kan ha t.ex. \code{Int} som argument till en typparameter i Scala, medan man i Java \emph{inte} direkt kan ha den primitiva typen \jcode{int} som typparameter till t.ex. \code{ArrayList}.

I Java och i den underliggande plattformen JVM används s.k. wrapper-klasser för att lösa detta, t.ex. genom wrapper-klassen \code{Integer} som boxar den primitiva typen \jcode{int}. Java-kompilatorn har stöd för att automatiskt packa in värden av primitiv typ i sådana wrapper-klasser för att skapa referenstyper och kan även automatiskt packa upp dem.

\Subtask Studera hur Scala-kompilatorn låter oss arbeta med en \code{Cell[Int]} även om det underliggande JVM:ens körtidstyp \Eng{runtime type} är en wrapper-klass. Man kan se JVM-körtidstypen med metoderna \code{getClass} och \code{getTypeName} enligt nedan.
\begin{REPL}
scala> class Cell[T](var value: T){
         val typeName: String = value.getClass.getTypeName
         override def toString = "Cell[" + typeName + "](" + value + ")"
       }
scala> val c = new Cell[Int](42)
scala> c.value.getClass.getTypeName
\end{REPL}


\Subtask Vad är körtidstypen för \code{c.value} ovan? Förklara hur det kan komma sig trots att vi deklarerade med typargumentet \code{Int}?

\Subtask Studera dokumentationen för \code{java.lang.Integer}\footnote{\href{https://docs.oracle.com/javase/8/docs/api/java/lang/Integer.html}{docs.oracle.com/javase/8/docs/api/java/lang/Integer.html}} och testa i REPL några av \emph{klassmetoderna} (de som är \jcode{static} och därmed kan anropas med punktnotation direkt på klassens namn utan \code{new}) och några av \emph{instansmetoderna} (de som inte är \jcode{static}).
\begin{REPL}
scala> Integer.  //tryck TAB
scala> Integer.
scala> Integer.toBinaryString(42)
scala> Integer.valueOf(42)
scala> val i = new Integer(42)
scala> i.  // tryck TAB
scala> i.toString
scala> i.compareTo  // tryck TAB 2 gånger
scala> i.compareTo(Integer.valueOf(42))
scala> i.compareTo(42)  // varför fungerar detta?
\end{REPL}

\Subtask\Pen Enligt dokumentationen\footnote{\href{https://docs.oracle.com/javase/8/docs/api/java/lang/Integer.html\#compareTo-java.lang.Integer-}{docs.oracle.com/javase/8/docs/api/java/lang/Integer.html\#compareTo-java.lang.Integer-}} tar instansmetoden \code{compareTo} i klassen \code{Integer} en \code{Integer} som parameter. Hur kan det då komma sig att sista raden ovan fungerar med en \code{Int}?

\Subtask Studera nedan Java-program och beskriv vad som kommer att skrivas ut \emph{innan} du kompilerar och testkör.

\javainputlisting[numbers=left]{examples/scalajava/Autoboxing.java}

\Subtask Ändra i programmet ovan så att autoboxing och autounboxing utnyttjas på alla ställen där så är möjligt. Utnyttja även att \code{toString}-metoden på \code{Integer} ger samma stränrepresentation som \jcode{int} vid utskrift. Fixa också så att du undviker \emph{fallgropen} att i Java jämföra med referenslikhet i stället för att använda \code{equals}. Testa så att allt fungerar som det borde efter dina ändringar.


\Subtask\Pen Antag att du råkar skriva \jcode{xs.add(0, pos)} på rad 14 i ditt program från föregående uppgift. Förklara hur autoboxingen stjälper dig i en \emph{fallgrop} då.

\Subtask\Pen Med ledning av de båda tidigare deluppgifterna: sammanfatta de två nämnda fallgropar med autoboxing i Java i två generella punkter, så att du har nytta av att memorera dem inför din framtida Javakodning.


\SOLUTION


\TaskSolved \what
     %%%TODO number  3 %%%starts with: \emph{Auto(un)boxing.} I JVM må%%%

\SubtaskSolved   -

\SubtaskSolved   Cell har typen java.lang.Integer. När man hämtar ut värdet med \code{c.value} hämtas den primitiva typ \code{int} ut.

\SubtaskSolved   Med hjälp av autoboxing förvandlas 42 till typen \code{Integer} och kan därför jämföras med en annan \code{Integer}.

\SubtaskSolved   i.compareTo(42) fungerar på grund av autoboxing. Då JVM packar in den primitiva typ int i en Integer-objekt automatiskt.

\SubtaskSolved
\begin{REPLnonum}
0 10 20 30 40 50 60 ... 390 400 410

[0]: 0
[42]: 0
NOT EQUAL
\end{REPLnonum}

\SubtaskSolved   \javainputlisting[numbers=left]{examples/scalajava/Autoboxing2.java}

\SubtaskSolved   42 kommer läggas längst fram i listan istället för längst bak, då autounboxing kommer göra Integer(0) till 0 och tvärtom med variablen \code{pos}.

\SubtaskSolved   Om man ska undersöka om två int-variabler är lika ska man använda ==, men om variablerna är av typen Integer måste man använda \code{equals}.

JVM kommer inte varna om man vänder på \code{Integer} och \code{int}, som i \code{xs.add(0, pos)}.



\QUESTEND






\WHAT{JavaConverters.}

\QUESTBEGIN

\Task  \what~  Med \code{import scala.collection.JavaConverters._} får man i sina Scalaprogram tillgång till de smidiga metoderna \code{asJava} och \code{asScala} som översätter mellan motsvarande samlingar i resp språks standardbibliotek. Kör nedan i REPL och gör efterföljande deluppgifter.

\begin{REPL}
scala> val sv = Vector(1,2,3)
scala> val ss = Set('a','b','c')
scala> val sm = Map("gurka" -> 42, "tomat" -> 0)
scala> val ja = new java.util.ArrayList[Int]
scala> ja.add(42)
scala> val js = new java.util.HashSet[Char]
scala> js.add('a')
scala> import scala.collection.JavaConverters._
\end{REPL}

\Subtask Till vilka typer konverteras Scalasamlingarna
\code{Vector[Int]}, \code{Set[Char]} och \\ \code{Map[String, Int]} om du anropar metoden \code{asJava} på dessa?

\Subtask Till vilka typer konverteras Javasamlingarna \code{ArrayList[Int]} och \code{HashSet[Char]}  om du anropar metoden \code{asScala} på dessa? Blir det föränderliga eller oföränderliga motsvarigheter?

\Subtask Vad får resultatet för typ om du kör \code{toSet} på en samling av typen \code{mutable.Set}?

\Subtask Undersök hur du kan efter att du gjort \code{sm.asJava.asScala} anropa ytterligare en metod för att få tillbaka en oföränderlig \code{immutable.Map}.

\Subtask Läs mer i dokumentationen om JavaConverters\footnote{\href{http://docs.scala-lang.org/overviews/collections/conversions-between-java-and-scala-collections.html}{docs.scala-lang.org/overviews/collections/conversions-between-java-and-scala-collections.html}}
och prova några fler konverteringar.



\SOLUTION


\TaskSolved \what
     %%%TODO number  4 %%%starts with: \emph{JavaConverters.} Med \cod%%%

\SubtaskSolved

Vector[Int] -> java.util.List[Int]

Set[Char] -> java.util.Set[Char]

Map[String, Int] -> java.util.Map[String, Int]

\SubtaskSolved

ArrayList[Int] -> scala.collection.mutable.Buffer[Int]

HashSet[Char] -> scala.collection.mutable.Set[Char]

Båda blir föränderliga motsvarigheter. Det visas genom att de till hör \code{scaka.collection.mutable} och både \code{ArrayList} och \code{HashSet} är förändrliga i Java.

\SubtaskSolved   \code{scala.collection.immutable.Set}

\SubtaskSolved   \code{sm.asJava.asScala} ger typen \code{scala.collection.mutable.Map[String,Int]}

\code{sm.asJava.asScala.toMap} ger typen \code{scala.collection.immutable.Map[String,Int]}

\SubtaskSolved   -

\QUESTEND




\ExtraTasks %%%%%%%%%%%%%%%%%%%


\WHAT{Översätta från Java till Scala.}

\QUESTBEGIN

\Task  \what~ Översätt nedan kod från Java till Scala. Skriv koden i en fil som heter \texttt{showInt.scala} och kalla Scala-objektet med \code{main}-metoden för \code{showInt}. Läs tipsen som följer efter koden innan du börjar.

\javainputlisting[numbers=left]{examples/scalajava/JShowInt.java}

\emph{Tips:}
\begin{itemize}[nolistsep, noitemsep]
\item En Javaklass med bara statiska medlemmar motsvaras av ett singeltonobjekt i Scala, alltså en \code{object}-deklaration. Scala har därför inte nyckelordet \jcode{static}.
\item Typen \jcode{Object} i Java motsvaras av Scalas \code{Any}.
\item Du kan använda Scalas möjlighet med default-argument (som saknas i Java) för att bara definiera en enda \code{show}-metod med en tom sträng som default \code{msg}-argument.
\item I Scala har objekt av typen \code{Char} en metod \code{def *(n: Int): String} som skapar en sträng med tecknet repeterat \code{n} gånger. Men du kan ju välja att ändå implementera metoden \code{repeatChar} med \code{StringBuilder} som nedan om du vill träna på att översätta en \code{for}-loop från Java till Scala.
\item I stället för \code{Scanner.nextLine} kan du använda \code{scala.io.StdIn.readLine} som tar en prompt som parameter, men du kan också använda \code{Scanner} i Scala om du vill träna på det.
\item I Java \emph{måste} man använda nyckelordet \jcode{return} om metoden inte är en \jcode{void}-metod, medan man i Scala faktiskt \emph{får} använda \code{return} även om man brukar undvika det och i stället utnyttja att satser i Scala också är uttryck.
\end{itemize}
Kompilera din Scala-kod och kör i terminalen och testa så att allt funkar. Vill du även kompilera Java-koden så finns den i kursens repo i filen\\ \texttt{compendium/examples/scalajava/JShowInt.java}


\SOLUTION


\TaskSolved \what


\begin{Code}[numbers=left]
object showInt {
  def show(obj: Any, msg: String = ""): Unit = println(msg + obj)

  def repeatChar(ch: Char, n: Int): String = ch.toString * n

  def showInt(i: Int): Unit = {
    val leading = Integer.numberOfLeadingZeros(i)
    val binaryString = repeatChar('0', leading) + i.toBinaryString
    show(i,               "Heltal : ")
    show(i.asInstanceOf[Char],         "Tecken : ")
    show(binaryString,    "Binärt : ")
    show(i.toHexString,   "Hex    : ")
    show(i.toOctalString, "Oktal  : ")
  }


  import scala.io.StdIn.readLine
  import scala.util.{Try,Success,Failure}

  def loop: Unit =
    Try { readLine("Heltal annars pang: ").toInt } match {
      case Failure(e) => show(e); show("PANG!")
      case Success(i) => showInt(i); loop
    }

  def main(args: Array[String]): Unit =
    if(args.length > 0) args.foreach(i => showInt(i.toInt))
    else loop
}
\end{Code}



\QUESTEND






\WHAT{Innehållslikhet och referenslikhet i Java.}

\QUESTBEGIN

\Task  \what~ Studera och prova denna fallgrop med innehållslikhet: \href{https://github.com/bjornregnell/lth-eda016-2015/blob/master/lectures/examples/eclipse-ws/lecture-examples/src/week10/generics/TestPitfall3.java}{TestPitfall3.java}







\SOLUTION


\TaskSolved \what
     %%%TODO number  6 %%%starts with: \TODO Fallgrop med Point som in%%%



\QUESTEND




\AdvancedTasks %%%%%%%%%%%%%%%%%


\WHAT{Implementera innehållslikhet i Java.}

\QUESTBEGIN

\Task  \what~\Pen Studera fallgropar för hur man skriver en \code{equals}-metod i Java här:
\href{http://www.artima.com/lejava/articles/equality.html}{www.artima.com/lejava/articles/equality.html} och jämför med  det fullständiga receptet för hur man skriver en välfungerande \code{equals} och \code{hashcode} i Scala här: \href{http://www.artima.com/pins1ed/object-equality.html}{www.artima.com/pins1ed/object-equality.html}

\Subtask Vilka skillnader och likheter finns vid överskuggning av equals i Java respektive Scala, som ska ge en fungerande innehållstest för en hierarki med bastyper och subtyper?

\Subtask Vilka fallgropar är gemensamma för Java och Scala?\SOLUTION


\TaskSolved \what
     %%%TODO number  7 %%%starts with: \TODO \emph{Gränssnitt i Scala %%%



\QUESTEND

%!TEX encoding = UTF-8 Unicode
%!TEX root = ../compendium2.tex

\Lab{\LabWeekELEVEN}

\begin{Goals}
\item Förstå skillnaden mellan primitiva typer och objekt i Java.
\item Kunna förklara hur autoboxing fungerar i Java.
\item Kunna förklara vad statiska metoder och attribut i Java innebär.
\item Kunna använda \code{ArrayList} och \code{Array} i Java.
\item Kunna använda \code{Scanner} i Java.
\item Kunna skapa en for-sats i Java.
\item Känna till hur man kan förenkla användningen av Java-kod från Scala med hjälp av \code{scala.collection.JavaConverters}.
\end{Goals}

\begin{Preparations}
\item \DoExercise{\ExeWeekELEVEN}{11}
\item Läs igenom bakgrunden, kodstrukturen och alla kommentarer i kodskelettet.
\end{Preparations}

Denna vecka kan du välja mellan att göra ett individuellt textspel i Java+Scala eller att tillsammans med hela eller delar av din samarbetsgrupp göra en alternativ grupplaboration som beskrivs i avsnitt \ref{section:alt:lthopoly}.

\subsection{Krav}
Du ska skapa ett textspel för terminalen som är (lagom) intressant/roligt att spela, sparar poäng per spelomgång för olika spelare och mäter tiden. Ditt textspel ska köras i terminalen och uppfylla följande krav och riktlinjer:

\begin{enumerate}
  \item När ditt program kör ska man ska kunna starta flera spelomgångar efter varandra utan att programmet ska behöva avslutas.
  \item För varje spelomgång ska spelarens namn\footnote{eller spelar\emph{nas} namn om det är ett spel får två eller flera personer} sparas i en textfil med tillhörande resultat.
  \item Efter varje spelomgång ska en topplista med bästa poäng visas, och efter begäran ska samtliga spelresultat för en viss spelare visas.
  \item Speltiden för varje spelomgång ska mätas och sparas tillsammans med poängresultatet för respektive spelare. Poängen för varje spelomgång ska på något sätt bero av speltiden.
  \item Ditt spel ska i Java-kod använda minst en av datastrukturerna
  \code{ArrayList},
  \code{HashSet},
  \code{HashMap} ur paketet \code{java.util}, samt minst en array i Java-kod.
  \item Ditt spel ska huvudsakligen (minst ca 80\% av kodraderna) vara skrivet i Java, men några delar \emph{ska} vara i Scala enligt följande riktlinjer:
  \begin{enumerate}
    \item Huvudprogrammet ska vara i Scala och \code{main} ska vara max 10 rader.
    \item Skrivning till textfil ska ske via Scala, och inläsning från textfil ska ske med hjälp av \code{java.util.Scanner} i Scala-kod (du ska alltså för träningens skull utgå från JDK-dokumentationen av \code{java.util.Scanner} och \emph{inte} använda \code{scala.io.Source}).
    \item Du ska samla all filhantering i ett Scala-singelobjekt med namnet \code{Disk}.
    \item I någon del av Scala-koden i ditt program ska du
använda omvandlingsmetoderna \code{asScala} och \code{asJava} efter \code{import scala.collection.JavaConverters._} för att omvandla mellan \code{java.util.ArrayList} och lämpliga samlingar i Scalas standardbibliotek.
  \end{enumerate}
  \item Du ska spela någon annans halvfärdiga spel och, efter att du studerat koden, ge återkoppling på kodens läsbarhet.
  \item Du ska låta någon annan spela ditt halvfärdiga spel och visa din kod och fråga om återkoppling på läsbarheten. Du ska anteckna den återkoppling du får.
  \item Du ska inför redovisningen förbereda följande:
  \begin{enumerate}
    \item en kort genomgång spelets idé,
    \item en kort förklaring av kodens struktur och de olika Java-klassernas ansvar,
    \item en kort redogörelse för den återkoppling du fått på din kods läsbarhet och hur du arbetat med att förbättra läsbarheten under dina stegvisa utvidgningar av din kod,
    \item en lista med koncept som du tränat på när du skapat ditt textspel.
  \end{enumerate}
\end{enumerate}

\subsection{Inspiration och tips}

\begin{enumerate}
  \item Utgå från Hangman i veckans övning eller,
  \item Yatzy från tidigare övningar, eller
  \item skapa ett kortspel inspirerat av \code{shuffle}-labben, eller
  \item inspireras av listan med sällskapsspel på wikipedia:\\ \href{https://sv.wikipedia.org/wiki/Kategori:Sällskapsspel}{sv.wikipedia.org/wiki/Kategori:Sällskapsspel}
  \item eller hitta på ett eget textspel.
  \item Börja med förenklade varianter som du sedan bygger vidare på.
  \item Kompilera och testa efter varje ändring, så att du hela tiden har ett fungerande program.
  \item Det finns mycket information på nätet om hur man skriver Java-kod och använder JDK, t.ex. på \url{https://stackoverflow.com/}
  \item träna på att använda JDK8-dokumentationen här:\\ \url{https://docs.oracle.com/javase/8/docs/api/}
\end{enumerate}


\clearpage
\section{Alternativ grupplaboration: \textnormal{\texttt{javatext-team/lthopoly}}}\label{section:alt:lthopoly}

I stället för ett individuellt textspel i Java+Scala enligt föregående stycke, kan hela eller delar av en samarbetsgrupp välja att göra ett större textspel i Java+Scala som en grupplaboration, med nedan laborationsbeskrivning som utgångspunkt. Ni får lov att anpassa uppgiften efter antalet gruppmedlemmar och er gemensamma ambition.

Inför ert första gruppmöte: Läs igenom bakgrunden, kodstrukturen och alla kommentarer i kodskelettet.

Riktlinjer för grupplaborationen:
\begin{itemize}[nolistsep]
%!TEX encoding = UTF-8 Unicode
%!TEX root = compendium.tex
\item
Diskutera i din samarbetsgrupp hur ni ska dela upp koden mellan er i flera olika delar, som ni kan arbeta med var för sig. En sådan del kan vara en klass, en trait, ett objekt, ett paket, eller en funktion.
\item
Varje del ska ha en \textbf{huvudansvarig} individ.
\item
Arbetsfördelningen ska vara någorlunda jämnt fördelad mellan gruppmedlemmarna.
\item
Den som är huvudansvarig för en viss del redovisar den delen.
\item 
Ni ska ta fram en gruppgemensam checklista för kodgranskning. Alla ska granska minst en annan gruppmedlems kod enligt checklistan. 
\item
Grupplaborationen görs över \textbf{två veckor} uppdelat på två delredovisningar. Vid första redovisningen ska arbetsupplägget och pågående utveckling redovisas. Vid andra tillfället ska de färdig lösningarna presenteras av respektive huvudansvarig individ.
\item
Vid första redovisningen ska du redogöra för handledaren hur ni delat upp koden och vem som är huvudansvarig för vad och vad ditt ansvar omfattar, samt hur ni jobbar praktiskt med att synkronisera er utveckling.
\item Grupplaborationen är en \textbf{extra stor uppgift} och grupparbetet behöver ledtid för att ni ska hinna koordinera er sinsemellan. Du behöver därför planera för att arbeta med något i grupplabben i stort sett varje dag under de tillgängliga veckorna, och vara redo att bidra i diskussioner.

\end{itemize}

\subsection{Bakgrund}
I denna labb skall ni tillverka ett spel kallat lthopoly, en variant av det välkända brädspelet Monopol med några simplifieringar. Varje spelare börjar med en summa pengar och förflyttar sig längs spelplanen.
I början av en runda slår den aktiva spelaren en tärning och deras pjäs flyttas det antal steg som tärningen visar.
Beroende på vilken av de tre möjliga ruttyperna spelaren hamnar på sker olika saker:

\begin{itemize}
\item \code{MoveSpace}: Om en spelare hamnar på denna ruta får den dra ett \code{MoveCard}, varpå spelaren förflyttas antingen framåt eller bakåt det antal steg som kortet anger.
\item \code{MoneySpace}: Om en spelare hamnar på denna ruta får den dra ett \code{MoneyCard}, varpå spelaren antingen förlorar eller vinner pengar enligt beskrivningen på kortet.
\item \code{HouseSpace}: Dessa rutor kan ägas utav spelare. Om en spelare förflyttas till denna ruta och ingen annan spelare äger den ges möjligheten att köpa den. Äger en annan spelare rutan blir den nuvarande spelaren tvungen att betala hyra. Hyran är samma som inköpspriset.
\end{itemize}

\begin{figure}[H]
\centering
\includegraphics[width=\textwidth]{../img/w11-lab/lthopoly.png}
\caption {Flödesdiagram för en spelrunda}
\label{fig:scalajava:lthopoly-team:flowchart}
\end{figure}

Spelet ska uppfylla följande krav:
\begin{itemize}
\item Varje spelare måste alltid börja sin runda med att slå en tärning innan den utför någon annan spelhandling, d.v.s. någon annan handling som påverkar spelets tillstånd (exempelvis kan en alltid visa spelplanen eller avsluta spelet).
\item Om någon spelare har mindre än 0 SEK kvar skall spelet sluta.
\item Om någon spelare hamnar på en husruta som ägs av en annan spelare måste denne betala ägaren husets hyra i SEK. Om ingen äger huset ges spelaren möjlighet att köpa det för ett belopp motsvarande en hyra (förutsatt att den har råd).
\item Om en spelare hamnar på ett \code{MoveSpace} eller ett \code{MoneySpace} får spelaren möjligheten att dra ett kort. För \code{MoveCard} innebär detta en förflyttning (bakåt eller framåt) medan för \code{MoneyCard} en minskning eller ökning av pengar.
\item Spelplanen skall vara cyklisk, d.v.s. att rutan direkt efter sista rutan är den första rutan på spelplanen.
\end{itemize}

Spelet avslutas när någon spelare får slut på pengar och då vinner spelaren med mest pengar.

\subsection{Kodstruktur}

Klassen \code{Player} representerar en spelare. Varje spelare måste känna till sitt saldo och sin position på brädet.

\begin{JavaSpec}{class Player}
	/** Creates a new player */
	public Player(String name, int money, int pos);

	/** Returns this player's money */
	public int getMoney();

	/** Adjusts this player's money */
	public void adjustMoney(int money);

	/** Returns this player's position */
	public int getPosition();

	/** Returns a string representation of this player */
	public String toString();

	/** Sets this player's position */
	public void setPosition(int pos);

\end{JavaSpec}

\code{MoneyCard} och \code{MoveCard} är två liknande klasser som representerar de kort som spelare kan dra. \code{MoneyCard} ska spara information om hur mycket pengar som ska läggas till eller dras bort. \code{MoveCard} ska spara information om hur långt en spelare skall förflytta sig när kortet dras. Båda korten skall även innehålla en beskrivning utav varför detta händer. Informationen för dessa kort läses in från textfilerna moneycards.txt och movecards.txt.

\begin{JavaSpec}{class MoneyCard}
    /** Creates a new MoneyCard */
    public MoneyCard(String description, int money);

    /** Returns this card's money adjustment value */
    public int getMoney();

    /** Returns the description of why the money is adjusted */
    public String getDescription();
\end{JavaSpec}

\begin{JavaSpec}{class MoveCard}
    /** Creates a new MoveCard */
    public MoveCard(String description, int positionAdjustment);

    /** Returns the position adjustment */
    public int getPositionAdjustment();

    /** Returns the description of why the position is adjusted */
    public String getDescription();

\end{JavaSpec}

Klasserna \code{MoneySpace} och \code{MoveSpace} ska ärva från den abstrakta klassen \code{BoardSpace}. \code{MoneySpace} och \code{MoveSpace} ska använda sig av en array med respektive korttyp.

\begin{JavaSpec}{class BoardSpace}
    /** Returns an array of possible game actions permitted by this space */
    public abstract int[] getPossibleActions(GameBoard board);

    /** Performs a game action available while on this space */
    public abstract void action(GameBoard board, int action);

    /** Returns a string representation of this BoardSpace */
    public abstract String toString();
\end{JavaSpec}

\begin{JavaSpec}{class MoneySpace}
    /** Returns an array of possible game actions permitted by this space */
    public int[] getPossibleActions(GameBoard board);

    /** Performs a game action available while on this space */
    public void action(GameBoard board, int action);

    /** Returns a string representation of this MoneySpace */
   public String toString();
\end{JavaSpec}

\begin{JavaSpec}{class MoveSpace}
    /** Returns an array of possible game actions available on this space */
    public int[] getPossibleActions(GameBoard board);

    /** Performs a game action available while on this space */
    public void action(GameBoard board, int action);

    /** Returns a string representation of this MoveSpace */
   public String toString();
\end{JavaSpec}

\begin{JavaSpec}{class HouseSpace}
    /** Returns an array of possible game actions permitted by this space */
    public int[] getPossibleActions(GameBoard board);

    /** Performs a game action available while on this space */
    public void action(GameBoard board, int action);

    /** Returns a string representation of this HouseSpace
     *  using the format "HouseName [Owner] (Rent)" */
   public String toString();
\end{JavaSpec}

Spelbrädets utseende bestäms av textfilen board.txt. Denna textfil specifierar i vilken ordning de olika sorters rutorna kommer och även namn och hyra för varje \code{HouseSpace}. Klassen \code{DocumentParser} hanterar inläsning från fil och ska kunna läsa in \code{MoneyCards}, \code{MoveCards} samt hela spelplanen.

\begin{JavaSpec}{class DocumentParser}
	/** Returns an ArrayList of Boardspaces loaded from a file */
	public static ArrayList<BoardSpace> getBoard();

	/** Returns an array of MoneyCards loaded from a file */
	public static MoneyCard[] getMoneyCards();

	/** Returns an array of MoveCards loaded from a file */
	public static MoveCard[] getMoveCards();
\end{JavaSpec}


Klassen \code{GameBoard} håller koll på spelets tillstånd.
\code{GameBoard} kombinerar ovannämnda klasser för att bygga upp spelet. \code{GameBoard} har en metod \code{getPossibleActions()} som returnerar en lista över alla möjliga spelarhandlingar för den nuvarande spelaren. Denna används av \code{main}-metoden för att be användaren välja nästa handling.
Olika sorters spelarhandlingar representeras av statiska \code{int}-variabler i klassen \code{GameBoard}. Vid val av handling matar användaren in handlingens siffervärde i konsolen. Varje handling motsvaras då alltid av samma inmatningsvärde för användaren.


\begin{JavaSpec}{class GameBoard}
    /** Creates a new board ready to play */
    public GameBoard(List<Player> players);

    /** Returns an int array containing possible game actions.
     *  A game action can be any of the static constants in GameBoard */
    public int[] getPossibleActions() ;

    /** Checks whether the game is over or not */
    public boolean isGameOver();

    /** Returns the player with the most money */
    public Player getRichestPlayer();

    /** Returns a list of all players */
    public List<Player> getPlayers();

    /** Returns a list of all BoardSpaces */
    public List<BoardSpace> getBoardSpaces();

    /** Performs an action for the current player */
    public void doAction(int action);

    /** Returns the currently active player */
    public Player getCurrentPlayer();

    /** Returns the BoardSpace corresponding to the position
     *  of the current player. */
    public BoardSpace getCurrentBoardSpace();

    /** Moves the currently active player adjustments spaces forward.
     *  Negative adjustment moves the player backwards. */
    public void moveCurrentPlayer(int adjustment);

    /** Returns an ArrayList<Integer> where each element contains
     *  the total sum of all players' money at the end of a round.
     *  E.g. list.get(0) is the total amount
     *  of money in the game after the first round. */
    public ArrayList<Integer> getStatistics();

    /** Returns a string representation of this GameBoard */
    public String toString();
\end{JavaSpec}


Den visuella representationen av spelet sker via konsolfönstret med hjälp av klassen \code{TextUI}. \code{TextUI} är en färdigskriven klass med metoder som gör det enkelt att skriva ut spelplanen och en logg av spelhistoriken i konsolfönstret.
 När \code{addToLog()} anropas sparar den sitt argument i en lista. Hela listan skrivs ut varje gång \code{updateConsole} anropas.
Utöver loggen skriver \code{updateConsole} även ut ett statusfönster med kortfattad information om varje spelare.
Om spelaren väljer att visa spelplanen ska metoden \code{printBoard()} istället anropas.
Alla utskrifter som sker under spelets gång ska gå via \code{TextUI}.
\newline

\begin{figure}[H]
\centering

\begin{REPL}
Oskar slog en 2:a!
Oskar drog ett kort: Jädrans! Studiebidraget har sänkts. Förlora 40 SEK
Oskar har avslutat sin runda.
Nästa spelare: Jonas
Jonas slog en 3:a!
Jonas drog ett kort: Det lönade sig att leva på nudlar! Inkassera 50 SEK
Jonas har avslutat sin runda.
Nästa spelare: Valthor
Valthor slog en 5:a!
Grattis, Valthor är nu den stolta ägaren av V-Huset
Valthor har avslutat sin runda.
Nästa spelare: Oskar
Oskar slog en 2:a!
Namn----------------Position----------------------Pengar----
Oskar*              Moroten och piskan(40)        260
Jonas               MoneySpace                    350
Valthor             V-Huset [Valthor](45)         255
------------------------------------------------------------
Välj ett alternativ:

	3. Köp ett hus
	5. Avsluta din runda
	8. Visa standardvyn
	9. Visa spelplanen
	0. Avsluta Lthopoly

============================================================

\end{REPL}
\caption {Utskrift av standardvyn}
\label{fig:scalajava:lthopoly-team:defaultview}
\end{figure}

\begin{figure}[H]
\centering

\begin{REPL}
Rutans Namn [Ägare] (Pris/Hyra) (Spelare, Pengar)*
---------------------------------------------------------------
Studiecentrum(20)
A-huset(25)
MoneySpace
MoneySpace (Jonas,350)
Moroten och piskan(40) (Oskar,260)
V-Huset [Valthor](45) (Valthor,255)
MoveSpace
MoneySpace
LED-Cafe(70)
F-Huset(75)
MoneySpace
MoveSpace
Ideet(80)
MoneySpace
E-huset(100)
MoveSpace

Välj ett alternativ:

	3. Köp ett hus
	5. Avsluta din runda
	8. Visa standardvyn
	9. Visa spelplanen
	0. Avsluta Lthopoly

============================================================
\end{REPL}

\caption {Utskrift av spelplanen}
\label{fig:scalajava:lthopoly-team:boardview}
\end{figure}

\begin{ScalaSpec}{TextUI}
  /** Prints an ASCII plot of the total amount
      of money in the game as a function of the turn index */
  def plotStatistics(x: Buffer[Int]): Unit

  /** Appends the String s to the end of the UI's event log */
  def addToLog(s: String): Unit

  /** Reprints the current state of the UI using the given
      GameBoard to print the status bar */
  def updateConsole(board: GameBoard): Unit

  /** Asks the user to select an option from a list of options
   *
   * @param options An Array of tuples of the form (choice, description)
   *                where choice is the number the user must enter
   *                to select the choice represented by description. E.g.
   *                (0, "End Game") lets the user input 0 to end the game.
   *
   * @return The selected choice
   */
  def promptForInput(options: Array[(Int, String)]): Int

  /** Prints the entire GameBoard */
  def printBoard(board: GameBoard): Unit
\end{ScalaSpec}

Simuleringen av spelet sker i main-metoden i klassen \code{Main}, och skall implementeras i Scala (i uppgift 6).

\ScalaSpecInputListing{Main}{../workspace/w11_lthopoly_team/src/main/scala/lthopoly/Main.scala}

\subsection{Obligatoriska uppgifter}

\Task All information om olika kort och om spelplanens upplägg finns i textfiler som måste läsas in med hjälp av metoderna i klassen \code{DocumentParser}.
Textfilerna moneycards.txt och movecards.txt innehåller information om de olika korten som finns.
Varje rad innehåller en beskrivning för kortet följt av ett värde separerat med semikolon. Dessa sparas i en array vid inläsning, och när en spelare hamnar på en \code{MoveSpace} eller \code{MoneySpace} dras ett slumpmässigt kort från arrayen.
Arrayen av kort måste alltså skickas med till motsvarande ruta när rut-objektet skapas.
Se textfilerna moneycards.txt, movecards.txt och board.txt i mappen resources för förståelse för hur inläsningen bör gå till för korten.
Filen board.txt innehåller spelplanen, men behandlas först i uppgift 3.



\Subtask Implementera klasserna \code{MoveCard}. Notera att samtlig inläsning från textfiler kommer att implementeras i klassen \code{DocumentParser} i en senare uppgift.
\newline
\Subtask Implementera klassen \code{MoneyCard}.
\newline
\Subtask Implementera klassen \code{Player}.
\newline
\Subtask Implementera metoderna \code{getMoneyCards()} och \code{getMoveCards()} i \code{DocumentParser}. Metoderna ska klara av att läsa in ett variabelt antal kort och här kommer ni att behöva läsa från fil.
Använd er utav ett \code{Scanner}-objekt för att läsa in information från fil.
\newline
\newline
\textbf{Tips:}
\begin{itemize}
\item Ni kan nyttja metoden \code{String.split(String delimiter)} för att dela en sträng i en array av fält, där \code{delimiter} är den avgränsade strängen som används vid uppdelningen.

\item För att konvertera mellan \code{ArrayList}s och arrayer kan \code{ArrayList}s \code{.toArray}-metod användas enligt följande:

\begin{Code}
ArrayList<MyClass> list = new ArrayList<MyClass>();
MyClass[] arr = list.toArray(new MyClass[]{});
\end{Code}

\item För att skapa ett \code{Path}-objekt som kan användas som konstruktorargument för en \code{Scanner} kan följande kod användas:

\begin{Code}
Path p = Paths.get(DocumentParser.class.
        getResource("/moneycards.txt").toURI);
\end{Code}

Det går också att använda \code{File}-objekt istället för \code{Path}-objekt, men se upp för att använda strängar! En \code{Scanner} som konstrueras med en sträng kommer läsa själva strängens innehåll, inte filen som sökvägen i strängen leder till.

\item Ni kommer behöva hantera undantag \Eng{exceptions} vid inläsning, vilket ni kan göra antingen med hjälp av en try-catch-sats, eller genom att lägga till \code{throws FileNotFoundException} i metodsignaturen och även hantera undantag i koden som anropar metoden.

\end{itemize}
\Task I denna uppgift skall de tre olika subklasserna till \code{BoardSpace} implementeras. Tänk på att \code{MoveSpace} och \code{MoneySpace} behöver tillgång till respektive kortlekar.
För att skriva metoderna \code{action} och \code{getPossibleActions} kommer ni behöva nyttja att klassen \code{GameBoard} har statiska konstanter som representerar de olika spelarvalen.


\Subtask Implementera en klass för varje typ av spelruta.
\newline
\newline
\noindent
\textbf{Obs!} Än så länge kommer logiken inte fungera då inga metoder är implementerade i \code{BoardGame}. Det går trots detta bra att anropa metoderna utan kompileringsfel (i väntan på att de implementeras).
\newline
\newline
\noindent
\textbf{Obs!} Metoderna \code{getPossibleActions} tar som argument ett \code{GameBoard}-objekt. Detta är bara användbart för \code{HouseSpace}s eftersom de är de enda rutor som behöver känna till spelets tillstånd. Anledningen till att de övriga klassernas \code{getPossibleActions}-metoder kräver samma argument är för att de ärver från den abstrakta klassen \code{BoardSpace}, som definierar metoden med ett \code{GameBoard} som argument. Om ni väljer att vidareutveckla spelet öppnar detta upp för fler tillåtna actions som beror på spelets tillstånd.


\begin{figure}[H]
\centering
\includegraphics[width=0.8\textwidth]{../img/w11-lab/boarddiagram.png}
\caption {Illustration av spelbrädets representation i en \code{ArrayList}. Objekt av typen \code{MoveSpace} och \code{MoneySpace} har referenser till en gemensam array av de inlästa korten motsvarande sin typ. Notera att elementens ordning i listan här bara är ett exempel.}
\label{fig:scalajava:lthopoly-team:boarddiagram}
\end{figure}

\Task Nu är det dags att implementera \code{getBoard()} i klassen \code{DocumentParser}. I denna metod skall ni läsa in från filen board.txt och nyttja de metoder ni redan skrivit för att nu kunna skapa \code{MoveSpace}-objekt och \code{MoneySpace}-objekt. Radernas ordning i filen bestämmer deras ordning på spelplanen. Varje rad börjar antingen med orden ''Move'', ''Money'' eller ''House''. House-rader följs dessutom av husets hyra och dess namn separerat av semikolon. Baserat på radens första ord skall ett motsvarande objekt konstrueras och läggas till i en \code{ArrayList<BoardSpace>} som slutligen returneras.

\Subtask Implementera \code{getBoard()}.
\newline
\newline
\textbf{Fundera på}
\begin{itemize}
\item Behöver flera objekt skapas av varje ruttyp?
\end{itemize}

\Task Implementera alla metoder utom \code{getStatistics()} i \code{GameBoard} (se specifikationen).
\newline
\newline
\textbf{Tips:}

\begin{itemize}
\item Ni kan använda privata hjälpmetoder för att underlätta implementeringen.
\item Metoden \code{plotStatistics} i klassen \code{TextUI} tar en array av \code{int}-värden som inparameter, vilket är opassande då det underlättar att lagra penga\-historiken i en \code{ArrayList} (eftersom dess storlek inte är bestämd). Det är därför lämpligt att skriva en metod som flyttar över samtliga \code{Integer}-objekt från \code{ArrayList<Integer>} till en array av primitiva \code{int}-värden. Detta fungerar trots att de har olika typer p.g.a. autoboxing.
\item Tänk på att spelarna skall kunna gå runt spelplanen ett obegränsat antal gånger.
\item Glöm inte att alla utskrifter skall gå via  \code{TextUI}.
\item Se flödesdiagrammet på sida \pageref{fig:scalajava:lthopoly-team:flowchart} för att få en överblick över vilka actions som är tillåtna vid en given tidpunkt.
\end{itemize}

\textbf{Fundera på}
\begin{itemize}
\item Varför tar konstruktorn i \code{GameBoard} emot en \code{List<Player>} istället för en \code{ArrayList<Player>}?
\end{itemize}

\Task Med spelplanen implementerad behövs en \code{main}-metod för att kunna starta spelet. I \code{main}-metoden skapas \code{GameBoard} samt alla \code{Player}-objekt. Spelet körs sedan som en loop där \code{GameBoard} tillfrågas vilka spelarhandlingar som är tillåtna för den nuvarande spelaren, erbjuder spelaren möjligheten att välja något av dessa alternativ, och matar sedan in spelarens val tillbaka till \code{GameBoard} som hanterar valet. \code{GameBoard} ska alltså hantera all spellogik internt.  Spel-loopen skall köras tills dess att spelet är över enligt \code{GameBoard.isGameOver}.

\Subtask Implementera \code{getAction} i Scala. Metoden ska anropa \code{TextUI.promptForInput} med en lämplig lista av tupler för att begära input från användarna. Metoden \code{getAction} skall nyttja de statiska variablerna från \code{GameBoard} för att ge en lämplig utskrift.

\Subtask Implementera \code{main}-metoden i Scala.

\begin{figure}[H]
\centering

\begin{REPL}
Relativ graf över total mängd pengar under spelets gång:
880  *
846     *
813
780        *
747
713           *
680
647              *  *  *
614                       *  *
580
547                             *
514
481
447                                *
414
381                                   *  *
348                                         *  *  *
314                                                  *  *  *  *  *
281                                                                 *  *
248
215                                                                       *
     1  2  3  4  5  6  7  8  9  10 11 12 13 14 15 16 17 18 19 20 21 22 23 24
\end{REPL}
\caption {Graf över spelets totala mängd pengar som funktion av rundornas index.}
\label{fig:scalajava:lthopoly-team:statistics}
\end{figure}

\Task Vi skall nu lägga till möjligheten att se statistik över spelets monetära tillstånd. Metoden \code{getStatistics()} i \code{GameBoard} ska returnera en lista innehållande den totala summan av pengar i spelet i slutet av varje runda. Denna lista kan skickas vidare till metoden \code{TextUI.plotStatistics} där den sedan skrivs ut i en vacker plot (se utskrift ovan).

\Subtask Implementera metoden \code{getStatistics()} i \code{GameBoard}.

\Subtask Utöka \code{main}-metoden så att grafen skrivs ut efter spelets slut.
Själva uppritningen sker med hjälp av den färdigskrivna metoden \code{plotStatistics} i \code{TextUI} som kräver en \code{Buffer[Int]} innehållande varje rundas totalsumma.
Ett tips är att nyttja \code{scala.collection.JavaConverters} för att konvertera Javas datatyper till Scalas motsvarigheter.
\newline
\newline
\textbf{Tips:}

\begin{itemize}
\item  För \code{getAction} är syftet att metoden skall hämta alla tillåtna actions från det \code{GameBoard}-objekt som skickas med som argument. Dessa ska sedan skickas vidare till \code{TextUI.promptForInput} varpå de möjliga valen skrivs ut. I \code{getAction} definierar ni själva hur ni vill att texten för de olika valen skall se ut. Det viktiga är att \code{promptForInput} får en tuple som länkar ihop ett sifferval med en sträng. Slutligen returnerar metoden siffervalet som användaren har gjort så att det går att ta logiska beslut utifrån denna siffra på andra ställen i koden.

\item  För att hantera den historik som krävs för att generera statistiken, måste ni själva implementera den logik som krävs. Hur ni väljer att implementera detta är upp till er, så länge \code{getStatistics} returnerar det som anges i specifikationen. Det är dock lämpligt att använda egna privata metoder för att implementera detta, och det är också lämpligt att implementera ihopräkning av pengar för en runda i slutet av metoden \code{doAction}, när en spelare faktiskt avslutar sin runda.
\item Kom ihåg att när \code{updateConsole} anropas så skrivs statusfältet och loggen ut automatiskt. Därefter kan ni anropa metoden \code{TextUI.promptForInput} för att återge standardvyn på samma format som visas i figur \ref{fig:scalajava:lthopoly-team:defaultview} på sida \pageref{fig:scalajava:lthopoly-team:defaultview}.
Anropet till \code{TextUI.promptForInput} bör alltid ske efter ett utskriftsanrop, detta så att det skall gå att mata in ett nytt val efter att föregående val har utförts.

\end{itemize}

\subsection{Frivilliga extrauppgifter}

\Task Utöka spelet med ny spelmekanik.

\Subtask Implementera funktionalitet för att varje spelare ska få extra pengar då den passerar första spelrutan.

\Subtask Implementera funktionalitet för att varje spelare som hamnar på en ruta de äger sedan tidigare har möjlighet att öka hyran för rutan ifall någon annan spelare skulle hamna på den.

\Subtask Implementera funktionalitet så att ägaren av ett hus måste betala en andel av dess värde varje gång de passerat första spelrutan.


%\input{modules/w12-sorting-chapter.tex}
%!TEX encoding = UTF-8 Unicode
\chapter{Trådar}\label{chapter:W12}
Begrepp du ska lära dig denna vecka:
\begin{multicols}{2}\begin{itemize}[nosep,label={$\square$},leftmargin=*]
\item tråd
\item jämlöpande exekvering
\item icke-blockerande anrop
\item callback
\item java.lang.Thread
\item java.util.concurrent.atomic.AtomicInteger
\item scala.concurrent.Future\end{itemize}\end{multicols}

\input{modules/w12-sorting-exercise.tex}
\input{modules/w12-sorting-lab.tex}

%%!TEX encoding = UTF-8 Unicode

%!TEX root = ../compendium2.tex

%!TEX encoding = UTF-8 Unicode
\chapter{Design}\label{chapter:W13}
Koncept du ska lära dig denna vecka:
\begin{multicols}{2}\begin{itemize}[nosep,label={$\square$},leftmargin=*]
\item\end{itemize}\end{multicols}

\clearpage\section{Tips}
%!TEX encoding = UTF-8 Unicode
%!TEX root = ../lect-w13.tex
%%%



\Subsection{Repetition på begäran}

\newcommand{\Vecka}[1]{\hfill\href{https://fileadmin.cs.lth.se/pgk/lect-w#1.pdf}{w#1}}

% \begin{Slide}{Repetitionsämnen 2020}
% Gör en lista på saker du behöver repetera.\\Exempel på önskade repetitionsämnen från tidigare år:
% \begin{itemize}\SlideFontSmall
%   \item closure (''fångad variabelrymd'') \Vecka{03}
%   \item Skillnad på objekt och singelobjekt? \Vecka{04}
%   \item Mönstermatchning. \Vecka{06}
%   \item \code{Option}  \Vecka{06}
%   \item \code{Try} med stort T  \Vecka{06}
%   \item \code{enum}: när och hur? eller case-klass? \Vecka{07}
%   \item När använda vilken sekvenstyp? \Vecka{07}
%   \item Typhärledning. \Vecka{08}
%   \item komposition eller arv?  \Vecka{10}
% \end{itemize}  
% \end{Slide}

% \begin{Slide}{Repetitionsämnen på begäran från tidigare år}
% \begin{enumerate}\SlideFontSmall
%    \item namnanrop, värdeanrop \Vecka{03}
%    \item funktionsvärde, funktionstyp och thunk \Vecka{03}
%    \item \code{--classpath} \Vecka{04}
%    \item \code{import} \Vecka{04}
%    \item \code{Option} \Vecka{06}
%    \item \code{Try} \Vecka{06}
%    \item \code{enum} \Vecka{07}
%    \item avlusning, läsa felmeddelande \Vecka{08}
%    \item \code{given using} \Vecka{11}
% \end{enumerate}  
% \end{Slide}

% \begin{Slide}{Några extra önskemål från tidigare år (i mån av tid)}
% \begin{enumerate}\SlideFontSmall
%   \item Closure (''fångad variabelrymd'') \Vecka{03}
%   \item Skillnad på objekt och singelobjekt? \Vecka{04}
%   \item Mönstermatchning. \Vecka{06}
%   \item När använda vilken sekvenstyp? \Vecka{07}
%   \item Typhärledning. \Vecka{08}
%   \item Komposition eller arv?  \Vecka{10}
% \end{enumerate}  
% \end{Slide}



\begin{Slide}{På begäran 2025}
\Emph{Grumligt}
\begin{enumerate}\SlideFontSmall
  \item namnanrop och värdeanrop \Vecka{03}
  \item konstruktor \Vecka{05}
  \item mönstermatchning med \code{match} ... \code{case} \Vecka{06}
  \item enumerationer \Vecka{07}
  \item synlighet, import/export, private/protected \Vecka{10}

\end{enumerate}  
\vspace{1em}\Alert{Nyfiken-på}
\begin{enumerate}\SlideFontSmall
  \item säker kod och felhantering
\end{enumerate}  
\end{Slide}


\begin{Slide}{På begäran 2024}
\Emph{Grumligt}
\begin{enumerate}\SlideFontSmall
  \item När är det bra/dåligt att använda anonyma funktioner? \Vecka{03}
  \item Klasser och kompanjonsobjekt: vad passar bäst var? \Vecka{05}
  \item Hur göra felhantering med \code{Option} och \code{Try}? \Vecka{06}
  \item Skillnaden mellan sats \& uttryck, tex \code{if}, \code{for}? \Vecka{01}

\end{enumerate}  
\vspace{1em}\Alert{Nyfiken-på}
\begin{enumerate}\SlideFontSmall
  \item Flertrådad programmering
  \item Fönsterhantering i introprog under huven 
  \item Generiska typgränser \code{<:} \code{>:}
\end{enumerate}  
\end{Slide}
  


% \begin{Slide}{På begäran 2023}
% \Emph{Grumligt}
% \begin{enumerate}\SlideFontSmall
%   \item Jämför: \code{class}, \code{trait}, \code{enum} \Vecka{05}
%   \item Hur fungerar kompanjonsobjekt? \Vecka{05}
%   \item Jämför: \code{try catch finally} och \code{Try Success Failure} \Vecka{06}
%   \item Hur fungerar \code{enum}? \Vecka{07} 
%   \item \code{match} \code{case} \Vecka{06}
%   \item Vad händer i minnet? Aktiveringspost, stacken, heapen \Vecka{03}
% \end{enumerate}  
% \vspace{1em}\Alert{Nyfiken-på}
% \begin{enumerate}\SlideFontSmall
%   \item Auto-formatera kod \hfill \url{https://scalameta.org/scalafmt/}
%   \item Flertrådad programmering: Övning Extra Vecka 12 \\ Kap 12.2.2 Uppgifter om trådar och jämlöpande exekvering
%   \item Opaka typer \hfill \url{https://docs.scala-lang.org/scala3/reference/other-new-features/opaques.html} 
% \end{enumerate}  
% \end{Slide}


%!TEX encoding = UTF-8 Unicode
%!TEX root = ../lect-w12.tex

%%%


\begin{Slide}{Repetition: Vad är en algoritm? }\SlideFontTiny
En \href{https://sv.wikipedia.org/wiki/Algoritm}{algoritm} är en stegvis beskrivning av hur man löser ett problem. \\ 
Exempel: SWAP, MIN, Registrering, Sökning, Sortering \\
\pause\vspace{0.5em}
Problemlösningsprocessens olika steg (inte nödvändigtvis i denna ordning): 
\begin{itemize}
\item Dela upp problemet i enklare delproblem och sätt samman.
\item Finns redan färdig lösning på (del)problem?
\item Formulera (del)\Emph{problemet} och ange tydligt indata och utdata: \\ exempel MIN: indata: sekvens av heltal; utdata: minsta talet
\item Kom på en \Emph{lösningsidé}: (kan  vara mycket klurigt och svårt) \\ exempel MIN: iterera över talen och håll reda på ''minst hittills''
\item Formulera en \Emph{stegvis beskrivning} som löser problemet: \\ exempel: pseudo-kod med sekvens av instruktioner
\item Implementera en \Emph{körbar lösning} i ''riktig'' kod: \\ exempel: en Scala-metod i en klass eller i ett singelobjekt
\item Har algoritmen acceptabla tids- och minneskrav?
\end{itemize}
\pause\vspace{0.5em} Det krävs ofta \Emph{kreativitiet} i stegen ovan  -- även i att \Emph{känna igen} problemet!\\
Simpelt exempel: Du stöter på problemet MAX och ser likheten med MIN.\\
\pause\vspace{0.5em}\Emph{Övning}: Diskutera hur du löser detta problem i relation till stegen ovan: \\
\emph{Att räkna antalet förekomster av olika unika ord i en textsträng.} 
\end{Slide}















\begin{Slide}{Repetition: Tumregler/tips vid val av abstraktion}\SlideFontSmall
Extensionsmetod, singelobjekt, case-klass, klass, trait, eller enum?
\begin{itemize}\SlideFontTiny
\item Om du vill lägga till en metod på befintlig typ utan behov av nya attribut etc., använd \code{extension}.
\item Använd \code{object} om du behöver samla metoder (och variabler) i en modul som bara finns i en upplaga. Du får lokal namnrymd och punktnotation på köpet.
\item Behöver du modellera \Emph{oföränderlig data}, använd en \code{case class} eller \code{enum}.  
\item Om du vill ha uppräknade värden som du vill kunna iterera över och matcha på i förseglad struktur, med värden i egen namnrymd, använd \code{enum}.
\item Med \code{case class} och \code{enum} får du även innehållslikhet och en massa annat godis på köpet!
\item Behöver du \Alert{förändringsbart tillstånd} \Eng{mutable state} använd en vanlig \code{class}. Det normala är att det föränderliga tillståndet (de attribut som är föränderliga) är \code{private} eller \code{protected} och att all uppdatering och avläsning av tillståndet sker indirekt genom metoder (getters/setters/...).
\item Behöver du en abstrakt bastyp använd en \code{trait}, speciellt om du vill ha möjlighet till inmixning.  Om du vill förhindra inmixning eller underlätta användning från Java, använd \code{abstract class}. 
\end{itemize}
\end{Slide}


% \begin{Slide}{Tips om hur man läser en specifikation}\SlideFontSmall
% När du läser en specifikation av en klass, en trait, eller ett singelobjekt:
% \begin{itemize}
% \item Tänk igenom vilket ansvar olika delar av koden har
% \item Vad håller klassen reda på? \\$\rightarrow$ Ledtrådar till attribut
% \item Vad ska klassen göra/räkna ut? \\$\rightarrow$ Ledtrådar till metoder och deras algoritm
% \item Vilka andra klasser har nytta av denna metod? \\$\rightarrow$ Ledtrådar till hur klasserna samverkar för att lösa uppgiften
% \end{itemize}
% Rita gärna en bild med ett specifikt exempel på vilken data som olika instanser håller reda på och fundera på hur data skapas/beräknas/förändras
% \end{Slide}


\begin{Slide}{Repetition: Tips om val av samling}\SlideFontSmall

Det är ofta enklare med oföränderliga samlingar med oföränderliga element och skapa nya samlingar vid förändring. Men för vissa algoritmer blir det enklare eller effektivare om du ändrar på plats i förändringsbar samling.

\begin{itemize}
\item Behöver du hantera värden i sekvens?
\begin{itemize}\SlideFontTiny
\item Om du klarar dig utan förändring av innehållet efter konstruktion:\\
\code{val}-referens till \code{Vector}
\item Om du behöver ändra innehåll men \Alert{inte} antal element:\\
\code{val}-referens till \code{Array}
\item Om du behöver ändra innehåll \Alert{och} antal element:
\\ \code{var}-referens till \code{Vector} och t.ex. metoden \code{patch}, eller \\
\code{val}-referens till \code{ArrayBuffer} och t.ex. metoden \code{insert}
\end{itemize}

\item Behöver du hantera värden \code{x} som ska vara unika?
\begin{itemize}\SlideFontTiny
\item Oföränderlig: \code{  Set}
\item Förändringsbar: \code{val}-referens till \code{scala.collection.mutable.Set}
\end{itemize}

\item Behöver du leta upp värden \code{x:Int} utifrån en nyckel av t.ex. String?
\begin{itemize}\SlideFontTiny
\item Oföränderlig: \code{Map[String, Int] }
\item Förändringsbar: \code{val}-referens till \code{scala.collection.mutable.Map[String, Int]}
\end{itemize}


\end{itemize}
\end{Slide}

% \begin{Slide}{ArrayBuffer}
% Ändra på plats: update, insert, remove, append
% {\SlideFontTiny

% \vspace{2.5em}\begin{tabular}{@{}p{4.2cm}  p{6.5cm}}
% \texttt{xs(i) = x \newline xs.update(i, x)} & Replace element at index i with x. \newline Return type Unit.\\   \cline{1-2}

% \texttt{xs.insert(i, x)\newline xs.remove(i)} & Insert x at index \texttt{i}. Remove element at i. \newline Return type Unit.\\   \cline{1-2}

% \texttt{xs.append(x)~~~xs~+=~x} & Insert x at end.  Return type Unit.\\   \cline{1-2}

% \texttt{xs.prepend(x)~~x~+=:~xs} & Insert x in front.  Return type Unit.\\   \cline{1-2}

% \texttt{xs -= x} & Remove first occurance of x (if exists). \newline Returns xs itself. \\\cline{1-2}

% \texttt{xs ++= ys} & Appends all elements in ys to xs and returns xs itself. \\

% \end{tabular}
% }

% \end{Slide}


\Subsection{Tentatips}

\begin{Slide}{Före tentan:}\SlideFontSmall
\begin{enumerate}
\item Repetera övningar och labbar i kompendiet.
\item Läs igenom föreläsningsanteckningar.
\item Studera \Emph{snabbref} \Alert{mycket noga} så att du vet vad som är givet och var det står, så att du kan hitta det du behöver snabbt.
\item Skapa och \Emph{memorera} en personlig \Emph{checklista} med programmeringsfel du brukar göra, som även inkluderar småfel, så som glömda parenteser och semikolon, och annat som en kompilator/IDE normalt hittar.
\item Tänk igenom hur du ska disponera dina 5 timmar på tentan.
\item Gör minst en extenta som om det vore \Alert{skarpt läge}:
\begin{enumerate}\SlideFontTiny
\item Avsätt 5 ostörda timmar (stäng av telefon, dator etc).
\item Inga hjälpmedel. Bara snabbref.
\item Förbered dryck och tilltugg.
\end{enumerate}
\end{enumerate}
\end{Slide}

\begin{Slide}{På tentan:} \SlideFontTiny
\begin{enumerate}
\item Läs igenom \Alert{hela} tentan först. \\ \Emph{Varför?} Förstå helheten. Delarna hänger ihop.
\item Notera och begrunda specifika begrepp och definitioner. \\ \Emph{Varför?} Begreppen är avgörande för förståelsen av uppgiften.
\item Notera förenklingar, antaganden och specialfall. \\ \Emph{Varför?} Uppgiften blir mkt enklare om du inte behöver hantera dessa.
\item \Alert{Fråga} tentamensansvarig om du inte förstår uppgiften -- speciellt om det finns misstänkta felaktigheter eller förmodat oavsiktliga oklarheter. \\ \Emph{Varför?} Det är inte lätt att konstruera en ''perfekt'' tenta. \\ Du får fråga vad du vill, men det är inte säkert du får svar...
\item Läs specifikationskommentarerna och metodsignaturerna i alla givna klass-specifikationer \Alert{mycket noga}. \\ \Emph{Varför?} Det är ett vanligt misstag att förbise de ledtrådar som ges där.
\item Återskapa din memorerade personliga checklista för vanliga fel som du brukar göra och avsätt tid till att gå igenom den på tentan. Varje fix plockar poäng!
\item Lämna in ett försök även om du vet att lösningen inte är fullständig. Det gäller att plocka så många poäng det går. En ofullständig lösning kan ändå ge poäng.

\item Om du har svårigheter kan det bli kamp mot klockan. Försök hålla huvudet kallt och prioritera utifrån var du kan plocka flest poäng. Ge inte upp! Ta en kort äta-dricka-paus för att få mer energi!

\end{enumerate}
\end{Slide}

\ifkompendium\else

\begin{Slide}{Planeringstips}\SlideFontTiny
Exempel på saker som du kan lägga in tid för i din julpluggkalender:
\begin{enumerate}
\item Ta reda på vad just \Alert{du} behöver träna på!
\item Välja ut övningar att repetera.
\item Repetera övning X, Y, Z, ... Både läsa och skriva kod. Fundera på typ och värde.
\item Välja ut labbar att repetera.
\item Repetera labb X, Y, Z, ... Lär dig ''trick'' och ''mönster''.
\item Träna på att skriva program med papper och penna.
\item Gör så många extentor du orkar, simulera ''skarp läge''.
\item Gör en checklista med vanliga fel och misstag som du brukar göra.
%\item Det finns inte så många Scala-extentor, men du kan också göra Java-extentor och lösa vissa delar i Scala och vissa delar i Java beroende på vad du behöver träna på.

\item Läsa igenom alla de extentor som du väljer att inte göra ''i fiktivt skarpt läge'' och studera generella mönster och typiska trick.
\end{enumerate}
\end{Slide}

\begin{Slide}{Tentans struktur}
\begin{itemize}\SlideFontSmall
\item Del A 20\%:\\\Emph{Evaluera uttryck} där du ska \Alert{ange typ och värde}
\begin{itemize}\SlideFontTiny
\item Testar förståelse av variabler, uttryck, samlingar, algoritmer, arv, etc.
\item Det är bra/nödvändigt att anteckna delsteg och variablers värden, då det är mycket svårt att tänka ut svaren direkt i huvudet.
%\item Ev. ''rättningströskel'': \textit{Om du på del A erhåller färre poäng än vad som krävs för att nå upp till en bestämd ''rättningströskel'', kan din tentamen komma att underkännas utan att del B bedöms.}
\end{itemize}


\item Del B 80\%:\\\Emph{Skriva kod} som uppfyller \Alert{krav och design}
\begin{itemize}\SlideFontTiny
\item Testar att du själv kan skapa kod med delar som samverkar
\item Testar förmåga att gå från indata-utdata till algoritm \\
 givet: ledtrådar, design, ev. skiss på lösning, ev. pseudokod etc.
\end{itemize}
\item Blanka inlämningar ger 0 poäng; det är alltid bättre att försöka än att lämna in blankt. Lämna inte in kladdpapper eller dubbla lösningar.
\end{itemize}
\end{Slide}


\begin{Slide}{Vad kommer på tentan?}
\begin{itemize}
\item Grundläggande begrepp och det som tränas på grundövningar och labbar är basen för att bli godkänd.
\item Begrepp, föreläsningsbilder och övningar som är markerade \Emph{''fördjupning''} krävs ej för att klara tentan men ökar förståelsen och hjälper dig att nå högre betyg.
\item Det är helt ok på tentan om du väljer en \Emph{enkel lösning med basala begrepp} \Alert{som fungerar bra}, i stället för en kortare/elegantare/mer avancerad lösning.
\item Extra-övningarna i läsvecka 12 ingår ej på tentan.
\end{itemize}
\end{Slide}


\Subsection{Avslutning}

\begin{Slide}{När du om några år tänker tillbaka på pgk...}
...hoppas jag du uppskattar den allmänbildning du fick om:
\begin{itemize}
  \item sekvens -- alternativ -- repetition -- abstraktion 
  \item viktiga datavetenskapliga idéer\\namnrymd, datatyp, samling, uppdelning i delproblem, ...
  \item grundläggande algoritmer\\registrering, linjärsökning, insättningssortering, ...
  \item träning i att självständigt skapa lättläst kod
  \item färdighet i användning av programmeringsverktyg  
\end{itemize}

\vspace{1em} Bonus: ett värdefullt socialt sammanhang med framtida kollegor.
\end{Slide}

\begin{Slide}{Scala då, nu och i framtiden}\SlideFontSmall

% {\SlideFontSize{7}{10}\url{
% https://en.wikipedia.org/wiki/Scala_(programming_language)#Versions
% }}
{\SlideFontSize{12}{13} Scalas övergripande målsättning: \Emph{smidigt} OCH \Alert{säkert}}
\begin{itemize}
\item Scala 1.0 (2003) första pre-release
\item Scala 2.0-2.9 (2006-2011) pionjärer: Twitter, LinkedIn, The Guardian, ...
\item Scala 2.10 (2013) brett genombrott, viktiga språkutvidgningar
\item Scala 2.11 (2014) allmän industriell spridning, stabilitet, prestanda, \\
% {\SlideFontSize{7}{10}\url{
% https://en.wikipedia.org/wiki/Scala_(programming_language)#Companies
% }}
\item Scala 2.12 (2016) fokus på prestanda, snabbare bytekod: lambda i JVM
\item Scala 2.13 (2019) fokus på standardbiblioteket och \code{scala.collection}, migreringsverktyg för Scala 3
\item Scala 3.0 (2021): \Alert{stort} \Emph{tekniksprång} med många nya språkdelar %:\\enum, top-level defs, @main, trait params, given, export, creator applications, ...,\\ "uppstädning" + förenklingar baserat på lärdomar från Scala 2.
\item Scala 3.8 (2026): första experimentella steget mot ännu säkrare system (fångstkontroll, nullkontroll, strikt likhet, ...)  
\end{itemize}

% Historiker på wikipedia är tyvärr inte uppdaterad...
%Läs mer om historik här: \url{https://en.wikipedia.org/wiki/Scala_(programming_language)}

Läs mer om nya experimentella delar i Scala här: \url{https://docs.scala-lang.org/scala3/reference/experimental}
\end{Slide}

% \begin{Slide}{Scala 3}\SlideFontSmall
% \begin{itemize}
%   \item Scala 3 släpptes i början av 2021.
%   \item Nya språkkonstruktioner, t.ex.  optional braces, if then else, while do, enum, top-level defs, @main, trait params, extension methods, given, using, export, creator applications, union types, intersection types,  ...,
%   \item \Emph{Tasty}: nytt format för kodträd som kompletterar bytekod och möjliggör omkompilering i efterhand och korsvis användning Scala 2 \& 3. \\
%   \item Formell bas för Scala: DOT (en algebra för dependent object types)
% \item Några viktiga Scala-ramverk för stordata, massiv parallellism, AI:
% \begin{itemize}\SlideFontTiny
%   \item \href{https://akka.io/}{Akka} ramverk för skalbara parallella arkitekturer
%   \item \href{https://spark.apache.org/}{Apache Spark} för parallell behandling av stordata i molnet, för AI, ML ...
%   \item \href{https://en.wikipedia.org/wiki/Apache_Kafka}{Apache Kafka} för hantering av strömmande data (initierad av LinkedIn)
%   \item \href{https://www.playframework.com/}{Play framework} för moderna, skalbara webbappar
% \end{itemize}
% \item Flera ''backends'' som breddar Scalas användningsområde:
% \begin{itemize}\SlideFontTiny
%   \item \href{http://www.scala-js.org/}{scala-js.org}: dela kod+kompetens mellan backend och frontend
%   \item \href{http://scala-native.org}{scala-native.org}: kör Scala kompilerat direkt ''på metallen''
% \end{itemize}
% \end{itemize}
% \end{Slide}


% \begin{Slide}{Scala på JVM, Scala JS, Scala Native}
% \begin{multicols}{3}

% \begin{tikzpicture}[node distance=1.4cm]
% \node (input) [startstop] {Scala-kod};
% \node (compile) [process, below of=input] {\texttt{scalac}};
% \node (output) [startstop, below of=compile] {byte-kod};
% \node (interp) [process, below of=output] {JVM};
% %\node (output2) [startstop, below of=interp] {maskinkod};
% \draw [arrow] (input) -- (compile);
% \draw [arrow] (compile) -- (output);
% \draw [arrow] (output) -- (interp);
% %\draw [arrow] (interp) -- (output2);
% \end{tikzpicture}


% \columnbreak %---------

% %https://www.sharelatex.com/blog/2013/08/29/tikz-series-pt3.html
% \begin{tikzpicture}[node distance=1.4cm]
% \node (input) [startstop] {Scala-kod};
% \node (compile) [process, below of=input] {\texttt{Scala JS}};
% \node (output) [startstop, below of=compile] {Javascript};
% \node (interp) [process, below of=output] {Browser | NodeJS};
% %\node (output2) [process, right of=interp, minimum size=6mm] {NodeJS};
% \draw [arrow] (input) -- (compile);
% \draw [arrow] (compile) -- (output);
% \draw [arrow] (output) -- (interp);
% %\draw [arrow] (interp) -- (output2);
% \end{tikzpicture}

% \columnbreak

% \begin{tikzpicture}[node distance=1.4cm]
% \node (input) [startstop] {Scala-kod};
% \node (compile) [process, below of=input] {\texttt{Scala Native}};
% \node (output) [startstop, below of=compile] {Mellankod (IR)};
% \node (interp) [process, below of=output] {LLVM};
% \node (output2) [startstop, below of=interp] {maskinkod};
% \draw [arrow] (input) -- (compile);
% \draw [arrow] (compile) -- (output);
% \draw [arrow] (output) -- (interp);
% \draw [arrow] (interp) -- (output2);
% \end{tikzpicture}

% \end{multicols} 
% \end{Slide}



\begin{Slide}{Hur håller jag mig uppdaterad om Scalas utveckling?}
\begin{itemize}%\SlideFontTiny
  \item Officiell blog: \url{https://www.scala-lang.org/blog/}
  \item Scala-nyheter: \url{http://scalatimes.com/}
%  \item Open online courses: \\\url{https://www.coursera.org/courses?query=scala}
  \item User-forum: \url{https://users.scala-lang.org/}
  \item Tjatt: \url{https://discord.gg/h9452YPJ}
  \item Scala Center: \url{https://scala.epfl.ch/}
  \item Scala-bibliotek: \url{https://index.scala-lang.org/}
  \item Contributors: \url{https://contributors.scala-lang.org/}
  \item Scala-språkets pågående förbättring: \url{https://docs.scala-lang.org/sips}
  % \item Scala Improvement Process: \\
  % \url{http://docs.scala-lang.org/sips/all.html}
\end{itemize}
\end{Slide}



\begin{Slide}{CEQ -- Course Experience Questionnaire}\SlideFontSmall
\begin{itemize}
\item Görs på hela LTH på samma sätt. Alla får länkar via mejl.
\item Snälla fyll i CEQ! Jag är \Alert{mycket tacksam} för all konstruktiv feedback! \\ Hög svarsfrekvens är viktigt för att kunna dra slutsatser om variationen i svaren och signifikansen i sammanställningen.
\item Del 1: Generella påståenden, alla med 5-gradig skala: \\ tar helt avstånd ... instämmer helt
\item Del 2: \Emph{Fritextfrågor}: \\
''Vad  tycker  du  var  det  bästa  med  den här  kursen?'' \\
''Vad  tycker  du  främst  behöver  förbättras?''
\item Mer om CEQ här: \url{https://www.ceq.lth.se/}
\item \Emph{Fördel} med CEQ: Samma alla kurser alla år medger jämförelse över tid.
\item \Alert{Begränsning}: Saknar frågor kopplat till specifika kursmoment.
\end{itemize}
\end{Slide}

\begin{Slide}{Kursspecifik utvärdering om specifika kursmoment}\SlideFontSmall
\begin{itemize}
\item Jag vill gärna att \Alert{alla} gör den LTH-gemensamma, anonyma kursutvärderingsenkäten \href{https://www.ceq.lth.se/}{CEQ}. Dina fritext-kommentarer om vad som är det bästa med kursen och vad som främst behöver förbättra emottages mycket tacksamt i CEQ-utvärderingen!
\item Jag kommer att komplettera CEQ med en \Emph{kursspecifik utvärdering} av specifika kursmoment i denna kurs och jag är därför \Alert{mycket tacksam} om alla fyller enkäten när länk kommer via anslag i Canvas.
\item Jag behandlar dina svar \Alert{konfidentiellt}, men sparar din email så att jag kan återkomma om jag mot förmodan undrar något mer.
\item Din input är \Emph{mycket värdefull} vid framtida kursutveckling!
\end{itemize}
\end{Slide}

\begin{Slide}{Intresserad av att arbeta som handledare?}
\begin{itemize}
\item Vi har ständigt behov av nya handledare i våra kurser
\item Det är lärorikt att jobba som handledare
\item Information om ansökningsprocessen: 
\begin{itemize}
  \item \url{https://www.cs.lth.se/utbildning/amanuenser}
  \item Sista ansökningsdatum för pgk är senare delen av \Alert{maj} (se exakt datum via länk från ovan sida)
  \item Skriv i ansökan om du helst jobbar i pgk.
  \item Ange övriga relevanta meriter, tex. ledarskap i föreningar, pedagogiska uppdrag, etc.
  \item Prata också gärna med handledare om hur det är att jobba i pgk eller mejla bjorn.regnell@cs.lth.se om du har frågor.
\end{itemize}
\end{itemize}
\end{Slide}


\begin{Slide}{Ett stort TACK...}
\begin{itemize}
  \item
... till alla \Emph{handledare} som jobbat hårt för att ni ska lära er så mycket som möjligt!
\item ... till alla \Alert{studenter} som gått kursen för:
\begin{itemize}
\item ... att ni kämpat så hårt!
\item ... att ni ställt massor med frågor!
\item ... att det har varit så hög närvaro på föreläsningarna!
\item ... att ni hjälp till med värdefull återkoppling!
\item ... att ni är så konstruktiva och verkligen vill lära er!
\end{itemize}
\vspace{2em} \pause

\end{itemize}
\Alert{Ett stort LYCKA TILL på vägen till att bli en \\ kompetent och innovativ systemutvecklare!}
\end{Slide}

\begin{Slide}{Hoppas att pgk-kursen varit givande!}
\includegraphics[width=5cm]{../img/gurka.jpg}\includegraphics[width=5cm]{../img/ukulele.jpg}
\end{Slide}


% subjekt och predikat -> public static void: https://www.youtube.com/watch?v=1ZPaR_wH-R8


% \begin{Slide}{Koda i Scala}

%   {\footnotesize\it Melodi: McDonalds-låten}
% % https://youtu.be/cTVhZqNwn3Y

% \begin{verbatim}

%           E         A             E          B 
% Det finns stunder i livet som man alltid har kvar

%           E           A               B 
% Det finns villkor och uttryck som man spar 

%         F#           B             F#           C#
% Och när koden är öppen finns gemenskap för fler 

% F#      B      C#       F#
% Koda i Scala; det ger meeeeeeer! 
% \end{verbatim}

% \end{Slide}


% \begin{Slide}{Ljuvliga språk}
% \fontsize{7}{8}\selectfont
%   \begin{verbatim}  
% F        F9   Fmaj7 F9        Fmaj7
% Å detta språk detta ljuvliga språk

% F             Gm    Gm7     C7
% som vi kallar bella Scala

% Gm                 Gm7       C7
% se vilken syn alla uttryck i skyn

%       Gm    C7    F
% detta ljuva bella Scala

% Cm                  Cm7  F7     Bbmaj7        Bb
% fjärran från det du älskar blir koden ödsligt tom

%     Dm7   G7     Dm7      G7        Gm7         C
% men i din närhet sluts du in i dess trolska rikedom

% C7#5   F        Am           Eb        D
% åh åh detta språk det är ungdomens språk

%        Gm     C     F      C7       
% som vi kallar bella Scala
% \end{verbatim}
% \end{Slide}




\fi

%!TEX encoding = UTF-8 Unicode
\chapter{Design}\label{chapter:W13}
Koncept du ska lära dig denna vecka:
\begin{multicols}{2}\begin{itemize}[nosep,label={$\square$},leftmargin=*]
\item\end{itemize}\end{multicols}

%!TEX encoding = UTF-8 Unicode
%!TEX root = ../exercises.tex

\ifPreSolution

\Exercise{\ExeWeekTHIRTEEN}\label{exe:W13}
\begin{Goals}
\item Kunna skriva tentamenslika program med papper, penna och snabbreferens som enda hjälpmedel.
\item Förbereda projektredovisningen.
\item Kunna skapa dokumentation med \code{scaladoc} och \code{javadoc}.
\item Kunna skapa jar-filer.
\end{Goals}

% \begin{Preparations}
% \item \StudyTheory{13}
% \end{Preparations}

\else

\ExerciseSolution{\ExeWeekTHIRTEEN}

\fi


\subsection{Förberedelse inför examination}




\WHAT{Gör en extenta.} %%%%%%%%%%%%%%%%%%%%%%%%%%%%%%%%%%%%%%%%%%%%%%%%%%%%%%%%

\QUESTBEGIN

\Task \what~\TODO

\SOLUTION

\TaskSolved \what~\TODO

\QUESTEND




\WHAT{Förbered din projektredovisning.} %%%%%%%%%%%%%%%%%%%%%%%%%%%%%%%%%%%%%%%

\QUESTBEGIN

\Task \what~\TODO

\SOLUTION

\TaskSolved \what~\TODO

\QUESTEND



\WHAT{Skapa dokumentation.} %%%%%%%%%%%%%%%%%%%%%%%%%%%%%%%%%%%%%%%%%%%%%%%%%%%

\QUESTBEGIN

\Task  \what~

\Subtask \TODO kör nedan kommando i terminalen:

\begin{REPL}
> scaladoc paket.scala
> ls
> firefox index.html   # eller öppna index.html i valfri webbläsare
\end{REPL}

Vad händer?

\Subtask Lägg till några fler metoder i något av objekten i filen \code{paket.scala} och lägg även till några dokumentationskommentarer. Kompilera om och kör. Generera om dokumentationen.

\begin{verbatim}
//... ändra i filen paket.scala

/** min paketdokumentationskommentar p2 */
package p2 {
  /** min paketdokumentationskommentar p21 */
  package p21 {
    /** ett hälsningsobjekt */
    object hello {
      /** en hälsningsmetod i p2.p21 */
      def hello = println("Hej paket p2.p21!")

      /** en metod som skriver ut tiden */
      def date = println(new java.util.Date)
    }
  }
}

\end{verbatim}

\begin{REPL}
> gedit paket.scala
> scalac paket.scala
> jar cvf mittpaket.jar gurka
> scala -cp mittpaket.jar
scala> gurka.tomat.banan.p2.p21.hello.date
scala> :q
> scaladoc paket.scala
> firefox index.html
\end{REPL}

\SOLUTION


\TaskSolved \what

\SubtaskSolved  -

\SubtaskSolved  -

\QUESTEND



\WHAT{Repetera övningar och laborationer.} %%%%%%%%%%%%%%%%%%%%%%%%%%%%%%%%%%%%

\QUESTBEGIN

\Task \what~\TODO

\SOLUTION

\TaskSolved \what~\TODO

\QUESTEND

\input{modules/w13-assignment-life.tex}
%!TEX encoding = UTF-8 Unicode
%!TEX root = ../compendium.tex

\Assignment{bank}

\subsection{Obligatoriska uppgifter}

\Task En uppgift.

\Subtask En underuppgift.

\Subtask En underuppgift.

\subsection{Frivilliga extrauppgifter}

\Task En uppgift.

\Subtask En underuppgift.

\Subtask En underuppgift.


%!TEX encoding = UTF-8 Unicode
%!TEX root = ../compendium.tex

\Assignment{tictactoe}
I detta projekt ska du implementera din egen version av spelet tic-tac-toe (eller som vi på svenska kallar det, tre i rad)! Du kommer börja med att implementera en version där du kan spela mot en kursare och sen gå vidare till att implementera en datorspelare som lägger sin pjäs slumpmässigt och till slut en som inte kan förlora!

\subsection{Regler}
%Om du känner dig säker på hur reglerna i tic-tac-toe funkar kan du skippa detta. 
\begin{itemize}
	\item Spelplanen består av ett rutnät av storlek 3x3.
	\item Det finns två spelare: \texttt{x} och \texttt{o}.
	\item Spelarna placerar ut en pjäs var i växlande ordning där \texttt{x} börjar.
	\item Spelet tar slut om en spelare har fått antingen en rad, diagonal eller kolumn ifylld av sin spelpjäs eller om spelplanen är fylld.
\end{itemize}
\textit{Notera att pjäserna INTE får flyttas när de väl ligger på spelplanen.}

\subsection{Teori}
Representationen är vald till en endimensionell vektor av typen Int av storlek 9 där elementen 0 till och med 2  representerar den första raden, [3, 5] andra raden och [6, 8] den tredje. Anledningen till detta är att vi vill ha en representation så att spelaren kan svara vilket drag den vill göra med ett heltal.
Varje element i vektorn ska kunna representera en tom plats, en plats allokerad av \texttt{x} och en plats allokerad av \texttt{o}. Detta innebär att en vektor av typen Boolean inte räcker till. Istället väjs den (kanske lite minnesöverflödiga) typen Int. Hint: en bra representation är 0 för tom plats, 1 för \texttt{x} och -1 för \texttt{o}. 
 
\subsection{Obligatoriska uppgifter}

\Task Implementera ett fungerande spel genom att utöka kodskeletten i klasserna Player, HumanPlayer och Game.

\Subtask Implementera funktionen gameWon i klassen Player som testar huruvida spelaren \code{who} vunnit spelet.

\Subtask Implementera en HumanPlayer.

\Subtask Implementera första version av Game.

\Task Randomized player

\Subtask Implementera en spelare som väljer ett slumpmässigt giltigt drag.

\Subtask Ändra Game så att användaren tillåts stänga av ritfunktionen och tillåts spela många spel.

\Subtask Vad är sannolikheterna för att \texttt{x} vinner, \texttt{o} vinner och att det blir oavgjort om två randomized players spelar mot varandra?

Hamnar man i närheten av dessa resultat tror vi på er randomized player.
\begin{itemize}
	\item P(\texttt{x} vinner) = 0.586
	\item P(\texttt{o} vinner) = 0.288
	\item P(lika) = 0.126
\end{itemize}


\Subtask Varför är det större sannolikhet för \texttt{x} att vinna än \texttt{o}?

\Task Optimal Player

\Subtask Läs igenom eval-funktionen och Appendix om max-min-evaluering.

\Subtask Implementera Optimal Players move-funktion.

\Subtask testa att spela mot din Optimal player med en human player, kan du spela lika? Kan du vinna?

\Subtask Vad händer om du sätter en random player mot Optimal player? Blir det någonsin oavgjort, hur ofta?

\subsection{Frivilliga extrauppgifter}

\Task Hashning.

\Subtask En underuppgift.

\Subtask En underuppgift.
%!TEX encoding = UTF-8 Unicode
%!TEX root = ../compendium.tex

\Assignment{imageprocessing}

\subsection{Obligatoriska uppgifter}

\Task En uppgift.

\Subtask En underuppgift.

\Subtask En underuppgift.

\subsection{Frivilliga extrauppgifter}

\Task En uppgift.

\Subtask En underuppgift.

\Subtask En underuppgift.

%%!TEX encoding = UTF-8 Unicode

%!TEX root = ../compendium2.tex

%!TEX encoding = UTF-8 Unicode
\chapter{Muntlig examen}\label{chapter:W14}


\TODO Beskrivning av hur muntligt prov går till. Vad händer om du behöver visa mer av dina kunskaper?

%!TEX encoding = UTF-8 Unicode
\chapter{Muntlig examen}\label{chapter:W14}

%!TEX encoding = UTF-8 Unicode
%!TEX root = ../exercises.tex

\ifPreSolution

\Exercise{\ExeWeekFOURTEEN}\label{exe:W14}

\begin{Goals}
\item Känna till vad en tråd är och kunna förklara begreppet jämlöpande exekvering.
\item Känna till vad metoderna \code{run} och \code{start} gör i klassen \code{Thread}.
\item Kunna skapa och starta en tråd med överskuggad \code{run}-metod.
\item Kunna skapa ett enkelt program som från två trådar tävlar om att uppdatera en variabel och förklara varför beteendet kan bli oförutsägbart.
\item Kunna använda en \code{Future} för att köra igång flera parallella beräkningar.
\item Kunna registrera en callback på en \code{Future} med metoden \code{onComplete}.
%\item Känna till att webbsidor beskrivs av HTML-kod och kunna skapa en minimal webbsida.
%\item Kunna ladda ner en webbsida med \code{scala.io.Source.fromURL}.
\end{Goals}

% \begin{Preparations}
% \item \StudyTheory{14}
% \end{Preparations}

\else

\ExerciseSolution{\ExeWeekFOURTEEN}

\fi


\subsection{Frivilliga extrauppgifter}



\WHAT{Trådar.}

\QUESTBEGIN

\Task  \what~   Klassen \code{java.lang.Thread} används för att skapa  \textbf{trådar} med jämlöpande exekvering \Eng{concurrent execution}. På så sätt kan man få olika koddelar att köra samtidigt.

Klassen \code{Thread} definierar en tom \code{run}-metod. Vill man att tråden ska göra något vettigt får man överskugga \code{run} med det man vill ska göras.

En tråd körs igång med metoden \code{start} och då anropas automatiskt \code{run}-metoden och tråden exekverar koden i \code{run} jämlöpande med övriga trådar. Om man anropar \code{run} direkt blir det \emph{inte} jämlöpande exekvering.

\Subtask Skapa en tråd som gör något som tar lite tid och kör med \code{run} resp. \code{start}.
\begin{REPL}
def zzz = { print("zzzzzz"); Thread.sleep(5000); println(" VAKEN!")}
zzz
val t2 = new Thread{ override def run = zzz }
t2.run
t2.run; println("Gomorron!")
t2.start; println("Gomorron!")
t2.start
\end{REPL}

\Subtask Vad händer om man anropar \code{start} mer än en gång på samma tråd?

\Subtask Skapa två trådar med överskuggade \code{run}-metoder och kör igång dem samtidigt enligt nedan. Vilken ordning skrivs hälsningarna ut efter rad 3 resp. rad 4 nedan? Förklara vad som händer.
\begin{REPL}
val g = new Thread{ override def run = for (i <- 1 to 100) print("Gurka ") }
val t = new Thread{ override def run = for (i <- 1 to 100) print("Tomat ") }
g.run; t.run
g.start; t.start
\end{REPL}

\Subtask Använd \code{Thread.sleep} enligt nedan. Är beteendet helt förutsägbart (deterministiskt)? Förklara vad som händer. Du kan (om du kör Linux) avbryta REPL med Ctrl+C%
\footnote{\href{http://stackoverflow.com/questions/6248884/can-i-stop-the-execution-of-an-infinite-loop-in-scala-repl}{stackoverflow.com/questions/6248884/can-i-stop-the-execution-of-an-infinite-loop-in-scala-repl}}.
\begin{REPL}
def ibland(block: => Unit) = new Thread {
  override def run = while(true) { block; Thread.sleep(600) }
}.start
ibland(print("zzz ")); ibland(print("snark ")); ibland(println("hej!"))
\end{REPL}


\SOLUTION


\TaskSolved \what
     %%%TODO number  1 %%%starts with: \emph{Trådar.}  %%%

\SubtaskSolved   -

\SubtaskSolved  \code {java.lang.IllegalThreadStateException}. Det går inte att starta en tråd mer än en gång. Tråden kan därför inte startas om när den redan har exekverats.

\SubtaskSolved   När \code {start} anropas exekveras koden i \code{run} parallellt. Därför skrivs \code{Gurka} och \code{Tomat} ut omlöpande. Om istället \code{run} anropas direkt blir det inte jämnlöpande exekvering och \code{Gurka} skrivs ut 100 gånger, sedan skrivs \code{Tomat} ut 100 gånger.

\SubtaskSolved   \code{Thread.sleep} pausar inte tråden i exakt den tiden som angets. Alltså kommer det skrivas ut \code{zzz snark hej!} i de flesta fall, men det är inte garanterat.



\QUESTEND






\WHAT{Jämlöpande variabeluppdatering.}

\QUESTBEGIN

\Task \label{task:racecondition} \what~   Skriv klasserna \code{Bank} och \code{Kund} i en editor och klistra sedan in koden i REPL.

\begin{Code}
class Bank {
  private var saldo = 0;
  def visaSaldo: Unit = println("saldo: " + saldo)
  def sättIn: Unit = { saldo += 1 }
  def taUt: Unit   = { saldo -= 1 }
}

class Kund(bank: Bank) {
  def slösaSpara = {bank.taUt; Thread.sleep(1); bank.sättIn}
}
\end{Code}

\Subtask Använd funktionen \code{ibland} från föregående uppgift och kör nedan rader i REPL. Resultatet av jämlöpande variabeluppdatering blir här heltokigt och leder till mycket upprörda bankkunder och -ägare. Förklara vad som händer.

\begin{REPL}
val bank = new Bank
bank.visaSaldo
bank.sättIn
bank.visaSaldo
bank.taUt
bank.visaSaldo

val bamse = new Kund(bank)
val skutt = new Kund(bank)

bamse.slösaSpara
skutt.slösaSpara
bank.visaSaldo

def ofta(block: => Unit) = new Thread {
  override def run = while(true) { block; Thread.sleep(1) }
}.start

ofta(bamse.slösaSpara); ofta(skutt.slösaSpara)

ibland(bank.visaSaldo)
\end{REPL}


\SOLUTION


\TaskSolved \what
     %%%TODO number  2 %%%starts with: \emph{Jämlöpande variabeluppdat%%%

\SubtaskSolved  I \code{slösaSpara} hämtas saldot, ändras och placeras tillbaka i minnet -  fördröjs -  upprepas. Om \code{bamse} blir klar med att ladda, ändra och lagra innan skutt gör detsamma med den muterbara variablen hade det inte varit perfekt. Problemet ligger i  när en tråd laddar och innan den kan lagra det förändrade värdet laddar den andra tråden samma värde. Bara en av dessa trådar vinner racet och får lagra sitt ändrade tal. \code{skutt} och \code{bamse} blir alltså upprörda för att inte alla dess uttag och insättningar registreras.


\QUESTEND






\WHAT{Trådsäkra \code{AtomicInteger}.}

\QUESTBEGIN

\Task  \what~  Det finns stöd i JVM för att åstadkomma uppdateringar som inte kan avbrytas av andra trådar under pågånde minnesskrivning. En operation som inte kan avbrytas kallas \textbf{atomär} \Eng{atomic}. Studera dokumentationen för \code{AtomicInteger}\footnote{\href{https://docs.oracle.com/javase/8/docs/api/java/util/concurrent/atomic/AtomicInteger.html}{docs.oracle.com/javase/8/docs/api/java/util/concurrent/atomic/AtomicInteger.html}} och prova nedan kod. Förklara vad som händer.

Använd funktionerna \code{ofta} och \code{ibland} från tidigare uppgifter.
\begin{Code}
class SäkerBank {
  import java.util.concurrent.atomic.AtomicInteger
  private var saldo = new AtomicInteger
  def visaSaldo: Unit = println(s"saldo: ${saldo.get}")
  def sättIn: Unit = { saldo.incrementAndGet }
  def taUt: Unit   = { saldo.decrementAndGet }
}

class SäkerKund(bank: SäkerBank) {
  def slösaSpara = {bank.taUt; Thread.sleep(1); bank.sättIn}
}
\end{Code}
\begin{REPL}
val säkerBank = new SäkerBank
val farmor = new SäkerKund(säkerBank)
val vargen = new SäkerKund(säkerBank)

ofta(farmor.slösaSpara); ofta(vargen.slösaSpara)

ibland(säkerBank.visaSaldo)
\end{REPL}





\SOLUTION


\TaskSolved \what
     %%%TODO number  3 %%%starts with: \emph{Jämlöpande exekvering med%%%

Nu är \code{farmor} garanterad att kunna ladda saldot, ta ut pengar/ändra och lagra innan \code{vargen} kan överskriva resultatet. I \code{slösaSpara} pausas tråden i en millisekund så \code{vargen} kan fortfarande ta ut pengar innan \code{farmor} hinner sätta in pengar igen. Dock kommer alla uttag och insättningar registreras eftersom operationerna är atomära.


\QUESTEND






\WHAT{Jämlöpande exekvering med \code{scala.concurrent.Future}.}

\QUESTBEGIN

\Task \label{task:future} \what~   Att skapa och hålla reda på trådar kan bli ganska omständligt och knepigt att få rätt på.
Med hjälp av \code{scala.concurrent.Future} kan man på ett enklare sätta skapa jämlöpande exekvering.

\begin{Background}
Med en \code{Future} skapas jämlöpande exekvering som ''under huven'' använder ett ramverk som heter Akka\footnote{\url{http://akka.io/}}, skrivet i Scala och Java. Akka erbjuder automatisk  multitrådning med s.k. trådpooler och möjliggör avancerad parallellprogrammering på en hög  abstraktionsnivå, där man själv slipper skapa instanser av klassen \code{Thread}. I stället kan man helt enkelt placera sin kod inramad med \code|Future{ "körs parallellt" }| efter att man importerat det som behövs.
\end{Background}

\Subtask För att skapa jämlöpande exekvering med \code{Future} behöver man först göra import enligt nedan; då skapas ett exekveringssammanhang med trådpooler redo för användning. Starta om REPL och studera felmeddelandet efter rad 1 nedan. Importera därefter enligt nedan. Vad har \code{f} för typ?
\begin{REPL}
scala> concurrent.Future { Thread.sleep(1000); println("En sekund senare!") }
scala> import scala.concurrent._
scala> import ExecutionContext.Implicits.global
scala> val f = Future { Thread.sleep(1000); println("En sekund senare!") }
\end{REPL}

\Subtask Skapa en procedur \code{printLater} enligt nedan som skriver ut argumentet efter slumpmässig tid. Förklara vad som händer nedan.
\begin{REPL}
scala> def printLater(a: Any): Unit =
         Future { Thread.sleep((math.random * 10000).toInt); print(a + " ") }
scala> (1 to 42).foreach(i => printLater(i)); println("alla är igång!")
\end{REPL}

\Subtask Skapa enligt nedan en \code{Future} som räknar ut hur många siffror det är i ett väldigt stort tal. Med \code{onComplete} kan man ange vad som ska göras när den tunga beräkningen är färdig; detta kallas att ''registrera en callback''. Vilken returtyp har \code{big}? Hur många siffror har det stora talet? Vad har \code{r} för typ? Justera argumentet till \code{big} om du inte orkar vänta på resultatet...

\begin{REPL}
scala> BigInt(10).pow(100)
scala> BigInt(10).pow(100).toString.size
scala> def big(n: Int) = Future { BigInt(n).pow(n).toString.size }
scala> big(1234567).onComplete{r => println(r + " siffror") }
\end{REPL}

\Subtask Den stora vinsten med \code{Future} är att man kan köra vidare under tiden, varför anropet av \code{Future} kallas \textbf{icke-blockerande} \Eng{non-blocking}. Det händer ibland att man ändå vill blockera exekveringen i väntan på ett resultat. Man kan då använda objektet \code{scala.concurrent.Await} och dess metod \code{result} enligt nedan. Använd \code{big} från föregående uppgift och gör en blockerande väntan på resultatet enligt nedan. Vad händer? Vad händer om du väntar för kort tid?

\begin{REPL}
scala> import scala.concurrent.duration._
scala> Await.result(big(1234567), 20.seconds)
\end{REPL}



\SOLUTION


\TaskSolved \what
     %%%TODO number  4 %%%starts with: TODO  %%%%%%%%%%%%%%%%%%%\Advan%%%

\SubtaskSolved  error: Cannot find an implicit ExecutionContext. Future behöver en ExecutionContext för att kunna köras. \code{f} är av typen Future[Unit].

\SubtaskSolved  Funktionen \code{printLater} har en Future, vilket innebär att när både \code{printLater} och \code{println} anropas i foreach-loopen exekveras de jämnlöpande. Eftersom det tar längre tid att starta upp en Future för datorn är \code{println} snabbare och skriver ut att alla är igång först. Sedan skrivs siffrorna från 1 - 42 ut med oregelbundna mellanrum eftersom tråden pausas olika länge.

\SubtaskSolved  \code{big} är en Future[Int]. Det stora talet har 7 520 383 siffror. \code{r} är av typen Try[Int] (se dokumentationen för Future om du är osäker)

\SubtaskSolved  Eftersom exekveringen blockas tills den har fått ett resultat går det inte att fortsätta skriva i REPL medan uträkningen pågår. Väntar man för kort tid får man ett TimeOutException och uträkningen avbryts.


\QUESTEND






\WHAT{Använda \code{Future} för att göra flera saker samtidigt.}

\QUESTBEGIN

\Task  \what~
I denna uppgift ska du ladda ner webbsidor parallellt med hjälp av \code{Future}, så att en nedladdning kan avslutas under tiden en annan dröjer.

\Subtask Koden för en minimal webbsida ser ut som nedan. Du kan beskåda sidan här: \url{http://fileadmin.cs.lth.se/pgk/mini.html} eller skriva in nedan kod i en fil som heter något som slutar på \texttt{.html} och öppna filen i din webbläsare.

\begin{verbatim}
<!DOCTYPE html>
<html>
<body>
HELLO WORLD!
</body>
</html>
\end{verbatim}

\Subtask För att simulera slöa webbservrar kan man ladda ner en sida via sajten \texttt{http://deelay.me/}. Ladda ner ovan sida med 2 sekunders fördröjning:\\
\url{http://deelay.me/2000/http://fileadmin.cs.lth.se/pgk/mini.html}

\Subtask Man kan ladda ner webbsidor med \code{scala.io.Source}. Vad händer nedan? Försök, med ledning av hur \code{delay} beräknas, uppskatta hur lång tid du måste vänta i medeltal, i bästa fall, respektive värsta fall, innan du kan se första webbsidan i vektorn \code{laddningar} nedan?

\begin{REPL}
scala> def ladda(url: String) = scala.io.Source.fromURL(url).getLines.toVector
scala> def slöladda(url: String) = {
         val delay = (math.random * 1000 + 2000).toInt
         val delaySite = s"http://deelay.me/$delay/"
         ladda(delaySite+url)
      }
scala> ladda("http://fileadmin.cs.lth.se/pgk/mini.html")
scala> def seg = slöladda("http://fileadmin.cs.lth.se/pgk/mini.html")
scala> val laddningar = Vector.fill(10)(seg)
scala> laddningar(0)
\end{REPL}

\Subtask Innan vi kan köra igång en \code{Future} så måste vi, som visats i uppgift \ref{task:future} importera den underliggande exekveringsmiljön som är redo att parallelisera ditt program i trådar utan att du själv måste skapa dem. Vad händer nedan?
\begin{REPL}
scala> import scala.concurrent._
scala> import ExecutionContext.Implicits.global
scala> val f = Future{ seg }
scala> f   // kolla om den är klar annars prova igen senare
scala> f
\end{REPL}

\Subtask Ladda indata utan att blockera \Eng{non-blocking input}. Förklara vad som händer nedan.
\begin{REPL}
scala> val nonblock = Future{ Vector.fill(10)(seg) }
scala> nonblock   // kolla igen senare om ej klar
scala> nonblock
\end{REPL}

\Subtask Ladda indata separat i olika parallella trådar. Förklara vad som händer nedan. Kör uttrycket på rad 3 nedan upprepade gånger i snabb följd efter varandra med pil-upp+Enter i REPL.
\begin{REPL}
scala> val para = Vector.fill(10)(Future{ seg })
scala> para
scala> para.map(_.isCompleted)
scala> para.map(_.isCompleted) // studera hur de blir färdiga en efter en
scala> para(0)
\end{REPL}

\Subtask Registrera en callback med metoden \code{onComplete}. Förklara vad som händer nedan.

\begin{REPL}
scala> val action = Vector.fill(10)(Future{ seg })
scala> action(0).onComplete(xs => println(s"ready:$xs"))
scala> // vänta tills laddning på plats 0 är klar
\end{REPL}

\Subtask Registrera en callback för felhantering i händelse av undantag med metoden \code{onFailure}. Förklara vad som händer nedan.
\begin{REPL}
scala> def lycka  = { Thread.sleep(3000); println(":)") }
scala> def olycka = { Thread.sleep(3000); 42 / 0; lycka }
scala> Future{ lycka  }.onFailure{ case e => println(s":( $e") }
scala> Future{ olycka }.onFailure{ case e => println(s":( $e") }
\end{REPL}



\SOLUTION


\TaskSolved \what
     %%%TODO number  5 %%%starts with: Sök upp och studera dokumentati%%%

\SubtaskSolved  -

\SubtaskSolved  -

\SubtaskSolved  Varje sida fördröjs med mellan 2 upp till 3 sekunder (2000-3000 millisekunder). Så i medeltal tar det 2.5 sekunder för varje sida att laddas. Vektorn måste fyllas innan exekveringen kan fortsätta. Därför laddas alla 10 stycken sidor in innan man kan se första websidan. Det tar därför i medeltal 2.5 x 10 = 25 sekunder.

\SubtaskSolved  \code{f} ger en Vektor fylld med strängar där varje element ges av en rad på hemsidan. Då \code{f} körs i bakgrunden kan programmet fortlöpa medan innehållet räknas ut. Du kan därför skriva \code{f} i REPL:n men det är inte säkert att proccessen är klar och det slutgilltiga resultatet visas.

\SubtaskSolved  Samma som ovan, förutom att det blir en vektor där varje element är i sig en vektor med strängar.

\SubtaskSolved  Laddar in datan parallelt så nedladdingen sker samtidigt, men det går olika snabbt pga metoden seg.

\SubtaskSolved  Eftersom datan laddas i parallella trådar utan blockering blir de inte klara i ordning, utan i den ordningen tråden körs klart. Till slut blir alla klara och resultatet visar en vektor med \code{true} värden.

\SubtaskSolved  Metoden \code{lycka} är väldefinerad och kastar därför inga undantag. Den skriver alltid ut \code{:)}. Metoden \code{olycka} är inte väldefinerad då division med 0 ger \code{java.lang.ArithmeticException}. Detta fångas upp vid callbacken och det skrivs ut \code{:(} samt det specifierade undantaget.

\ExtraTasks %%%%%%%%%%%%


\QUESTEND






\WHAT{}

\QUESTBEGIN

\Task  \what~ Räkna ut stora primtal parallellt genom att använda nedan funktioner. Implementera \code{isPrime} enligt pseudokod från den engelska wikipediasidan om primtalstest\footnote{\href{https://en.wikipedia.org/wiki/Primality_test}{en.wikipedia.org/wiki/Primality\_test}} med den s.k. ''naiva algoritmen''.  Räkna ut 10 st slumpvisa primtal med 16 siffror vardera. Gör beräkningarna parallellt med hjälp av \code{Future}.

\begin{Code}
def isPrime(n: BigInt): Boolean = ???

def nextPrime(start: BigInt): BigInt = {
  var i = start
  while (!isPrime(i)) { i += 1 }
  i
}

def randomBigInt(nDigits: Int): BigInt = {
   def rndChar = ('0' + (math.random * 10).toInt).toChar
   val str = Array.fill(nDigits)(rndChar).mkString
   BigInt(str)
}
\end{Code}

\SOLUTION


\TaskSolved \what
  %%%TODO number  6 %%%

\begin{Code}
def isPrime(n: BigInt): Boolean = n match {
  case _ if (n <= 1) => false
  case _ if (n <= 3) => true
  case _ if n % 2 == 0 || n % 3 == 0 => false
  case _ =>
    var i = BigInt(5)
    while (i * i < n) {
      if (n % i == 0 || n % (i + 2) == 0) false
      i += 6
    }
    true
}

import scala.concurrent._
import ExecutionContext.Implicits.global

val primes = Vector.fill(10)(Future{nextPrime(randomBigInt(16))})
primes.foreach(_.onSuccess{case i => println(i)})
\end{Code}


\QUESTEND






\WHAT{Svara på teorifrågor.}

\QUESTBEGIN

\Task  \what~\Pen

\Subtask Vad är en tråd?

\Subtask Hur skapar man en tråd med klassen \code{Thread}?

\Subtask Hur startar man en tråd?

\Subtask Vilka problem kan man råka ut för om man uppdaterar samma resurs i flera olika trådar?

\Subtask Vad innbär det att kod är \emph{trådsäker}?

\Subtask Nämn några fördelar med att använda Future jämfört med att använda trådar direkt.


\SOLUTION


\TaskSolved \what
 %%%TODO number  7 %%%

\SubtaskSolved  Stackoverflow ger följande förklaring:

A thread is an independent set of values for the processor registers (for a single core). Since this includes the Instruction Pointer (aka Program Counter), it controls what executes in what order. It also includes the Stack Pointer, which had better point to a unique area of memory for each thread or else they will interfere with each other.

\SubtaskSolved

\begin{Code}
val thread = new Thread(new Runnable{
	def run(){println(''Det här är en tråd'')}
})
\end{Code}

\SubtaskSolved  \code{thread.start}

\SubtaskSolved  Det kan bli kapplöpning(race conditions) om vilken tråds resurser blir sparade. Vilket leder till att de andra trådarnas ändringar blir ignorerade.

\SubtaskSolved  Trådsäkerhet innebär att flera trådar kan köras parallellt utan felaktigheter i resultatet. Exempelvis får man vara väldigt försiktig om man vill ha en muterbar variabel som alla trådar ska ändra samtidigt.

\SubtaskSolved  Till exempel slipper man skapa instanser av klassen Thread eftersom man kan placera koden i en Future istället. Den löser även mycket under huven för kodaren.


\QUESTEND






\WHAT{Klasser med atomär uppdatering.}

\QUESTBEGIN

\Task  \what~ Läs om och testa klasserna AtomicBoolean, AtomicDouble och AtomicReference för atomär uppdatering i paketet \\ \code{java.util.concurrent.atomic}.

Använd några av dessa tillsammans med \code{scala.concurrent.Future}.


\SOLUTION

\TaskSolved --

\QUESTEND





\WHAT{Skapa din egen multitrådade webbserver.}

\QUESTBEGIN

\Task  \what~

\Subtask Skriv in\footnote{Eller ladda ner här: \href{https://github.com/lunduniversity/introprog/blob/master/compendium/examples/simple-web-server/webserver.scala}{github.com/lunduniversity/introprog/blob/master/compendium/examples/simple-web-server/webserver.scala}} nedan kod i en editor och spara i en fil med namn \texttt{webserver.scala} och kompilera och kör med \texttt{scala webserver.start} och beskriv vad som händer när du med din webbläsare surfar till adressen: \\ \url{http://localhost:8089/abbasillen}

\scalainputlisting[numbers=left,basicstyle=\ttfamily\fontsize{11}{12}\selectfont]{examples/simple-web-server/webserver.scala}

\Subtask Du ska nu skapa en webbserver som gör något lite mer intressant. Den ska svara med det 13:e Fibonacci-talet\footnote{\href{https://sv.wikipedia.org/wiki/Fibonaccital}{https://sv.wikipedia.org/wiki/Fibonaccital}} om du surfar till \url{http://localhost:8089/fib/13}.
Spara din webbserver från föregående deluppgift under det nya namnet \texttt{fibserver.scala} och använd koden nedan och lägg till och ändra så att din server kan svara med Fibonaccital. Vi börjar med att räkna ut Fibonaccital i funktionen \code{compute.fib} nedan på ett onödigt processorkrävande sätt med exponentiell tidskomplexitet så att webbservern verkligen får jobba, för att i senare deluppgifter implementera \code{compute.fib} med linjär tidskomplexitet och därmed undvika onödig planetuppvärmning.
\begin{CodeSmall}
  object compute {
    def fib(n: BigInt): BigInt = {
      if (n < 0) 0 else
      if (n == 1 || n == 2) 1
      else fib(n - 1) + fib(n -2)
    }
  }

  def fibResponse(num: String) = Try { num.toInt } match {
    case Success(n) => html.page(s"fib($n) == " + compute.fib(n))
    case Failure(e) => html.page(s"FEL $e: skriv heltal, inte $num")
  }

  def errorResponse(uri:String) = html.page("FATTAR NOLL: " + uri)

  def handleRequest(cmd: String, uri: String, socket: Socket): Unit = {
    val os = socket.getOutputStream
    val parts = uri.split('/').drop(1) // skip initial slash
    val response: String = (parts.head, parts.tail) match {
      case (head, Array(num)) => fibResponse(num)
      case _                  => errorResponse(uri)
    }
    os.write(html.header(response.size).getBytes("UTF-8"))
    os.write(response.getBytes("UTF-8"))
    os.close
    socket.close
  }
\end{CodeSmall}
Kör i terminalen med \texttt{scala fibserver.start} och beskriv vad som händer i din webbläsare när du surfar till servern.


%%%\textbf{KOD TILL FACIT:}
%%%\scalainputlisting[numbers=left,basicstyle=\ttfamily\fontsize{11}{12}\selectfont]{examples/simple-web-server/fibserver.scala}


\Subtask Surfa efter flera stora Fibonacci-tal samtidigt i olika flikar i din browser. Hur märks det att servern bara kör i en enda tråd?

\Subtask Gör din server multitrådad med hjälp av den nya server-loopen nedan.

\begin{CodeSmall}
import scala.concurrent._
import ExecutionContext.Implicits.global

  def serverLoop(server: ServerSocket): Unit = {
    println(s"http://localhost:${server.getLocalPort}/hej")
		while (true) {
  		Try {
  		  var socket = server.accept  // blocks thread until connect
	  	  val scan = new Scanner(socket.getInputStream, "UTF-8")
		    val (cmd, uri) = (scan.next, scan.next)
			  println(s"Request: $cmd $uri")
		    Future { handleRequest(cmd, uri, socket) }.onFailure {
		      case e => println(s"Reqest failed: $e")
		    }
		  }.recover{ case e: Throwable => s"Connection failed: $e" }
		}
  }
\end{CodeSmall}

\Subtask Surfa efter flera stora Fibonacci-tal samtidigt i olika flikar i din browser. Hur märks det att servern är multitrådad?


\Subtask Det är onödigt att räkna ut samma Fibonacci-tal flera gånger. Med hjälp av en cache i form av en föränderlig \code{Map} kan du spara undan redan uträknade värden. Det funkar dock inte med en vanlig \code{scala.collection.mutable.Map} i vår multitrådade webbserver, eftersom den inte är \textbf{trådsäker} \Eng{thread-safe}. Med trådosäkra föränderliga datastrukturer blir det samma besvärliga beteende som i uppgift \ref{task:racecondition}.

Du ska i stället använda \code{java.util.concurrent.ConcurrentHashMap}. Sök upp  dokumentationen för \code{ConcurrentHashMap} och försök förstå koden nedan. Hur fungerar metoderna \code{containsKey}, \code{put} och \code{get}?
\begin{Code}
object compute {
  import java.util.concurrent.ConcurrentHashMap
  val memcache = new ConcurrentHashMap[BigInt, BigInt]

  def fib(n: BigInt): BigInt =
    if (memcache.containsKey(n)) {
      println("CACHE HIT!!! no need to compute: " + n)
      memcache.get(n)
    } else {
      println("cache miss :( must compute fib:  " + n)
      val f = fastFib(n)
      memcache.put(n, f)
      f
    }

  private def fastFib(n: BigInt): BigInt = {
    if (n < 0) 0 else
    if (n == 1 || n == 2) 1
    else fib(n - 1) + fib(n -2)
  }
}
\end{Code}

\Subtask Använd ovan \code{fib}-objekt i en ny version av din webserver. Spara den i en ny kodfil med namnet \texttt{fibserver-memcached.scala}. Undersök hur snabbt det går med stora Fibonaccital med den nya varianten. Hur stora tal kan du räkna ut? Kan servern fortsätta efter överflödad stack? Förklara varför.

\Subtask Nu när vi kan få väldigt stora Fibonacci-tal kan det vara användbart att stoppa in radbrytningar på webbsidan. Html-taggen \texttt{</br>} ger en radbrytning.
\begin{Code}
  def insertBreak(s: String, n: Int = 80): String = {
    if (s.size < n) s
    else s.take(n) + "</br>" + insertBreak(s.drop(n),n)
  }
\end{Code}
Använd den rekursiva funktionen ovan för att pilla in radbrytningstaggar på var $n$:te position i långa strängar. Testa hur det ser ut på webbsidan med ovan funktion när din server svarar med väldigt stora tal.

\Subtask Vi ska nu använda det större heap-minnet i stället för stack-minnet och därmed inte begränsas av stackens max-storlek. Skriv om \code{fastFib} så att den använder en \code{while}-sats i stället för ett rekursivt anrop. Denna uppgift är ganska klurig, men om du kör fast kan du snegla i lösningarna i Appendix för inspiration.

Hur stora tal klarar din server nu? Vad händer med servern när minnet tar slut? Hur kan du skydda servern så att den inte kan hänga sig?

\SOLUTION


\TaskSolved \what
 %%%TODO number  9 %%%

\SubtaskSolved  \code{abbasillen} skrivs ut baklänges till \code{nellisabba}.

\SubtaskSolved

\SubtaskSolved

\SubtaskSolved

\SubtaskSolved

\SubtaskSolved

\SubtaskSolved

\SubtaskSolved

\SubtaskSolved

Lösningsförslag:
\scalainputlisting[numbers=left,basicstyle=\ttfamily\fontsize{11}{12}\selectfont]{examples/simple-web-server/fibserver-threaded-memcached-while.scala}


\QUESTEND






\WHAT{}

\QUESTBEGIN

\Task  \what~ Utöka din server med fler beräkningsintensiva funktioner. Exempelvis primtalsberäkningar eller beräkningar av valfritt antal decimaler av $\pi$ eller $e$. Utnyttja gärna det du lärt dig i  matematiken om summor och serieutvecklingar.

\SOLUTION


\TaskSolved \what
 %%%TODO number  10 %%%

---


\QUESTEND






\WHAT{}

\QUESTBEGIN

\Task  \what~ Läs mer om \code{Future} och jämlöpande exekvering i Scala här:\\
\href{http://alvinalexander.com/scala/future-example-scala-cookbook-oncomplete-callback}{alvinalexander.com/scala/future-example-scala-cookbook-oncomplete-callback}

\SOLUTION


\TaskSolved \what
 %%%TODO number  11 %%%

---


\QUESTEND






\WHAT{}

\QUESTBEGIN

\Task  \what~ Läs mer om jämlöpande exekvering och multitrådade program i Java här: \href{http://www.tutorialspoint.com/java/java_multithreading.htm}{www.tutorialspoint.com/java/java\_multithreading.htm}  \\
\noindent När man skriver program med jämlöpande exekvering finns det många fallgropar; det kan bli kapplöpning \Eng{race conditions} om gemensamma resurser och dödläge \Eng{deadlock} där inget händer för att trådar väntar på varandra. Mer om detta i senare kurser.


\SOLUTION


\TaskSolved \what
 %%%TODO number  12 %%%

---


\QUESTEND






\WHAT{Studera dokumentationen i \code{scala.concurrent}.}

\QUESTBEGIN

\Task  \what~\Pen

\Subtask Studera dokumentationen för \code{scala.concurrent.Future}\footnote{\href{http://www.scala-lang.org/files/archive/api/current/\#scala.concurrent.Future}{http://www.scala-lang.org/files/archive/api/current/\#scala.concurrent.Future}}. Hur samverkar \code{Future} med \code{Try} och \code{Option}? Vilka vanliga samlingsmetoder känner du igen?

\Subtask Studera dokumentationen för \code{scala.concurrent.duration.Duration}\footnote{\href{http://www.scala-lang.org/api/current/\#scala.concurrent.duration.Duration}{www.scala-lang.org/api/current/\#scala.concurrent.duration.Duration}}. Vilka tidsenheter kan användas?

\Subtask Vid import av \code{scala.concurrent.duration._ } dekoreras de numeriska klasserna med metoder för att skapa instanser av klassen \code{Duration}. Detta möjligörs med hjälp av klassen \code{scala.concurrent.duration.DurationConversions}. Studera dess dokumentation och testa att i REPL skapa några tidsperioder med metoderna på \code{Int}.



\SOLUTION


\TaskSolved \what
 %%%TODO number  13 %%%

\SubtaskSolved

\SubtaskSolved

\SubtaskSolved


\QUESTEND






\WHAT{}

\QUESTBEGIN

\Task  \what~ Fördjupa dig inom webbteknologi.

\Subtask Lär dig om HTML, CSS och JavaScript här: \url{https://developer.mozilla.org/en-US/docs/Learn}

\Subtask Lär dig om Scala.JS här: \url{http://www.scala-js.org/}\SOLUTION


\TaskSolved \what
 %%%TODO number  14 %%%

\SubtaskSolved  ---

\SubtaskSolved  ---

\SubtaskSolved  ---

\SubtaskSolved  ---
\QUESTEND

\input{modules/w14-extra-lab.tex}


\part{Appendix}
\appendix

\setcounter{chapter}{8} %next is I in \Alph

%!TEX root = ../compendium.tex

\chapter{Integrerad utvecklingsmiljö}

\section{Vad är en IDE?}

\section{ScalaIDE och Eclipse}

\subsection{Installera ScalaIDE}

\section{Handledning ScalaIDE/Eclipse}
%!TEX encoding = UTF-8 Unicode
%!TEX root = ../compendium2.tex

\chapter{Skapa webb-appar med ScalaJS}\label{appendix:scalajs}

\TODO

\noindent Läs först appendix \ref{appendix:build}

\begin{itemize}
\item \url{http://scala-js.org/}
\end{itemize}
%!TEX encoding = UTF-8 Unicode
%!TEX root = ../compendium2.tex

\chapter{Skapa Android-appar i Scala}\label{appendix:scala-android}

\TODO

\noindent Läs först appendix \ref{appendix:build}

\begin{itemize}
\item \url{http://scala-android.org/}
\end{itemize}

\part{Lösningar}

\setcounter{chapter}{11} %next is L in \Alph
\chapter{Lösningar till övningarna}\label{chapter:solutions}
\setcounter{section}{7}

\PreSolutionfalse

\let\QUESTBEGIN\ifPreSolution  % to mark formatting and numbering of exercises
\let\SOLUTION\else  % to mark solutions in the same file as questions
\let\QUESTEND\fi    % to mark end of exercise


%!TEX encoding = UTF-8 Unicode
%!TEX root = ../exercises.tex

\ifPreSolution

\Exercise{\ExeWeekEIGHT}\label{exe:W08}

\begin{Goals}
\item Kunna skapa och använda matriser med nästlade strukturer av \code{Vector}.
\item Kunna iterera över elementen i en matris med nästlade \code{for}-satser och \code{for}-\code{yield}-uttryck, samt nästlad applicering av \code{map} respektive \code{foreach}.
\item Kunna skapa och använda funktioner som tar matriser som parametrar.
\item Kunna skapa en enkel generisk klass och enkla generiska funktioner med hjälp av en typparameter.
\item Kunna beskriva skillnader och likheter mellan Scala och Java vad gäller indexering och iterering i matriser implementerade med nästlade arrayer.
%\item Kunna skapa och använda matriser med hjälp inbyggda arrayer i Java.
%\item Kunna använda nästlade \code{for}-satser i Java för att iterera över elementen i en matris.
\end{Goals}

\begin{Preparations}
\item \StudyTheory{08}
\end{Preparations}

\BasicTasks

\else

\ExerciseSolution{\ExeWeekEIGHT}

\BasicTasks

\fi



\WHAT{Para ihop begrepp med beskrivning.}

\QUESTBEGIN

\Task \what

\vspace{1em}\noindent Koppla varje begrepp med den (förenklade) beskrivning som passar bäst:

\begin{ConceptConnections}
  matris & 1 & & A & konkret typ, binds till typparameter vid kompilering \\ 
  generisk & 2 & & B & indexerbar datastruktur i två dimensioner \\ 
  typargument & 3 & & C & har abstrakt typparameter, typen är generell \\ 
  typhärledning & 4 & & D & kompilatorn beräknar typ ur sammanhanget \\ 
\end{ConceptConnections}

\SOLUTION

\TaskSolved \what

\begin{ConceptConnections}
  matris & 1 & ~~\Large$\leadsto$~~ &  A & indexerbar datastruktur i två dimensioner \\ 
  radvektor & 2 & ~~\Large$\leadsto$~~ &  F & matris av dimension $1\times{}m$ med $m$ horisontella värden \\ 
  kolumnvektor & 3 & ~~\Large$\leadsto$~~ &  G & matris av dimension $m\times{}1$ med $m$ vertikala värden \\ 
  kolonn & 4 & ~~\Large$\leadsto$~~ &  C & annat ord för kolumn \\ 
  generisk & 5 & ~~\Large$\leadsto$~~ &  B & har abstrakt typparameter, typen är generell \\ 
  typargument & 6 & ~~\Large$\leadsto$~~ &  D & konkret typ, binds till typparameter vid kompilering \\ 
  typhärledning & 7 & ~~\Large$\leadsto$~~ &  E & kompilatorn beräknar typ ur sammanhanget \\ 
\end{ConceptConnections}

\QUESTEND




\WHAT{Skapa matriser med hjälp av nästlade samlingar.}

\QUESTBEGIN

\Task  \what~  Man kan i ett datorprogram, med hjälp av samlingar som innehåller samlingar, skapa nästlade strukturer som kan indexeras i två dimensioner och på så sätt representera en  \textbf{matris}.\footnote{\href{https://sv.wikipedia.org/wiki/Matris}{sv.wikipedia.org/wiki/Matris}}

\Subtask Rita minnessituationen efter tilldelningen på rad 1 nedan. Vad har \code{m} för typ och värde? Vad har \code{m} för dimensioner? Hur sker indexeringen i ett datorprogram jämfört med i matematiken?

\begin{REPL}
scala> val m = Vector((1 to 5).toVector, (3 to 7).toVector)
scala> m.apply(0).apply(1)
scala> m(1)
scala> m(1)(4)
\end{REPL}

\Subtask Vad ger uttrycken på raderna 2, 3 och 4 ovan för värden och typ?

\Subtask Man kan i ett datorprogram mycket väl skapa tvådimensionella, nästlade strukturer där raderna \emph{inte} innehåller samma antal element. Det blir då ingen äkta matris i strikt matematisk mening, men man kallar ofta ändå en sådan struktur för en ''matris''. Vilken typ har variablerna \code{m2}, \code{m3}, \code{m4} och \code{m5} nedan?

\begin{REPL}
scala> val m2 = Vector(Vector(1,2,3),Vector(4,5),Vector(42))
scala> val m3 = Vector(Vector(1,2), Vector(1.0, 2.0, 3.0))
scala> val m4 = m3(1) +: Vector("a") +: m3
scala> val m5 = Vector.fill(42){ m2(1).map(e => (e * math.random()).toInt) }
\end{REPL}

\Subtask Vilken av variablerna \code{m2}, \code{m3}, \code{m4} och \code{m5} ovan representerar en äkta matris i matematisk mening? Vilken är dess dimensioner?

\SOLUTION

\TaskSolved \what

\SubtaskSolved   \includegraphics{../img/w09-solutions/1a} \\
Typ: \code{Vector[Vector[Int]]}\\
Värde: \code{Vector(Vector(1, 2, 3, 4, 5), Vector(3, 4, 5, 6, 7))} \\
Dimensioner: $2 \times 5$\\
Inom matematiken sker indexering enligt konvention med 1 som lägsta index. I scala är lägsta index 0, man använder s.k. 0-indexering. \footnote{Detta är inte fallet i alla programmeringsspråk, vilket du kan läsa mer om på \url{https://en.wikipedia.org/wiki/Array\_data\_type\#Index\_origin}}

\SubtaskSolved
\begin{REPL}
scala> val m = Vector((1 to 5).toVector, (3 to 7).toVector)
m: Vector[Vector[Int]] = Vector(Vector(1, 2, 3, 4, 5), Vector(3, 4, 5, 6, 7))

scala> m.apply(0).apply(1)
res4: Int = 2

scala> m(1)
res5: Vector[Int] = Vector(3, 4, 5, 6, 7)

scala> m(1)(4)
res6: Int = 7
\end{REPL}

\SubtaskSolved  \\
m2: \code{Vector[Vector[Int]]}\\
m3: \code{Vector[Vector[Int | Double]]}\\
m4: \code{Vector[Vector[Int | Double | String]]}\\
m5: \code{Vector[Vector[Int]]}

\SubtaskSolved  m5, $42 \times 2$

\QUESTEND





\WHAT{Skapa och iterera över matriser.}

\QUESTBEGIN

\Task  \label{matrices:task:yatzy} \what~  Du ska skapa matriser där varje rad representerar 5 kast med en tärning i spelet Yatzy.\footnote{\href{https://sv.wikipedia.org/wiki/Yatzy}{sv.wikipedia.org/wiki/Yatzy}}


\Subtask Definiera i REPL en funktion \code{def throwDie: Int = ???} som returnerar ett slumptal mellan 1 och 6.

\Subtask Skapa nedan heltalsmatris i REPL. Vilken dimension får matrisen?
\begin{REPL}
scala> val ds1 = for (i <- 1 to 1000) yield 
            for (j <- 1 to 5) yield throwDie
          
\end{REPL}

\Subtask Man kan också använda nedan varianter för att skapa en heltalsmatris. Vilken av varianterna \code{ds1} ... \code{ds6} tycker du är lättast att läsa och förstå? Prova respektive variant i REPL och ange vilken typ på \code{ds1} ... \code{ds6} som härleds av kompilatorn.
\begin{REPL}
val ds2 = (1 to 1000).map(i => (1 to 5).map(j => throwDie))
val ds3 = (1 to 1000).map(i => Vector.fill(5)(throwDie))
val ds4 = for (i <- 1 to 1000) yield Vector.fill(5)(throwDie)
val ds5 = Vector.fill(1000)(Vector.fill(5)(throwDie))
val ds6 = Vector.fill(1000, 5)(throwDie)
\end{REPL}


\Subtask Definiera en funktion \\ \code{def roll(n: Int): Vector[Int] = ???}\\ som ger en heltalsvektor med $n$ stycken slumpvisa tärningskast. Kasten ska vara sorterade i växande ordning; använd för detta ändamål samlingsmetoden \code{sorted}.


\Subtask \label{matrices:subtask:isyatzyforall} Definera i REPL en funktion \code{isYatzy(xs: Vector[Int]): Boolean = ???} som testar om alla elementen i en heltalsvektor är samma. Använd samlingsmetoden \code{forall}.


\Subtask Skapa en funktion  \\ \code{def diceMatrix(m: Int, n: Int): Vector[Vector[Int]] = ???} \\ som med hjälp av funktionen \code{roll} skapar en matris med \code{m} st vektorer med vardera \code{n} slumpvisa tärningskast.


\Subtask \label{matrices:subtask:diceMatrixToString} Skapa en funktion som returnerar en utskriftsvänlig sträng \\ \code{def diceMatrixToString(xss: Vector[Vector[Int]]): String = ???} \\med hjälp av \code{map} och \code{mkString}, som fungerar enligt nedan.
\begin{REPL}
scala> val dm2s = diceMatrixToString(diceMatrix(4, 5))
val dm2s: String = 1 4 4 6 6
1 1 2 6 6
2 4 4 5 6
1 1 5 6 6

scala> println(dm2s)
1 4 4 6 6
1 1 2 6 6
2 4 4 5 6
1 1 5 6 6
\end{REPL}



\Subtask Implementera funktionen \\ \code{def filterYatzy(xss: Vector[Vector[Int]]): Vector[Vector[Int]]} \\ som filtrerar fram alla yatzy-rader i matrisen \code{xss} enligt nedan. Använd din funktion \code{isYatzy} och samlingsmetoden \code{filter}.
\begin{REPL}
scala> println(diceMatrixToString(filterYatzy(diceMatrix(10000, 5))))
4 4 4 4 4
6 6 6 6 6
4 4 4 4 4
6 6 6 6 6
4 4 4 4 4
4 4 4 4 4
2 2 2 2 2
\end{REPL}



\Subtask Implementera funktionen \\
\code{def yatzyPips(xss: Vector[Vector[Int]]): Vector[Int] = ???}\\
som ska ge en vektor med de tärningsvärden som gav yatzy, för kasten i matrisen \code{xss} enligt nedan. Använd din funktion \code{filterYatzy}.
\begin{REPL}
scala> val dm = Vector(Vector(1,2,3,4,5),Vector(4,4,4,4,4),Vector(3,3,3,3,3))
scala> yatzyPips(dm)
val res42: Vector[Int] = Vector(4, 3)
\end{REPL}

\SOLUTION

\TaskSolved \what

\SubtaskSolved
\begin{Code}
def throwDie: Int = (math.random() * 6).toInt + 1
\end{Code}
Eller:
\begin{Code}
def throwDie: Int = scala.util.Random.nextInt(6) + 1
\end{Code}

\SubtaskSolved  Matrisdimension i matematisk notation: $1000 \times 5$, vilket motsvarar en matris med 1000 rader och 5 kolumner.

\SubtaskSolved
\begin{Code}
ds1: IndexedSeq[IndexedSeq[Int]]
ds2: IndexedSeq[IndexedSeq[Int]]
ds3: IndexedSeq[Vector[Int]]
ds4: IndexedSeq[Vector[Int]]
ds5: Vector[Vector[Int]]
ds6: Vector[Vector[Int]]
\end{Code}
\code{IndexedSeq} och \code{Vector} ovan finns i paketet \code{scala.collection.immutable}

\SubtaskSolved  \begin{Code}
def roll(n: Int) = Vector.fill(n)(throwDie).sorted
\end{Code}

\SubtaskSolved  \begin{Code}
def isYatzy(xs: Vector[Int]): Boolean = xs.forall(_ == xs(0))
\end{Code}



%2.g)
\SubtaskSolved  \begin{Code}
def diceMatrix(m: Int, n: Int): Vector[Vector[Int]] =
  Vector.fill(m)(roll(n))
\end{Code}

\SubtaskSolved  \begin{Code}
def diceMatrixToString(xss: Vector[Vector[Int]]): String =
  xss.map(_.mkString(" ")).mkString("\n")
\end{Code}


%2.j)
\SubtaskSolved
\begin{Code}
def filterYatzy(xss: Vector[Vector[Int]]): Vector[Vector[Int]] =
  xss.filter(isYatzy)
\end{Code}



%2.m)
\SubtaskSolved  \begin{Code}
def yatzyPips(xss: Vector[Vector[Int]]): Vector[Int] =
  filterYatzy(xss).map(_.head)
\end{Code}

\QUESTEND








\WHAT{En oföränderlig, generisk matris-klass till veckans laboration \hyperref[section:lab:\LabWeekEIGHT]{\texttt{\LabWeekEIGHT}}.}

\QUESTBEGIN

\Task\label{exe:matrices:labprep}  \what~Under veckans laboration ska du simulera en enkel form av ''liv'' som består av celler i ett rutnät. För detta ändamål har vi nytta av en matris-klass som du ska implementera steg för steg i denna övning.
Skapa case-klassen nedan med en editor i filen \code{Matrix.scala}. Testa din lösning med hjälp av valfri \hyperref[appendix:ide]{IDE}, t.ex. \code{scalaide} eller \code{idea}.
\begin{Code}
case class Matrix(data: Vector[Vector[String]]){
  def apply(row: Int, col: Int): String = data(row)(col)
}
object Matrix {
  def fill(dim: (Int, Int))(value: String): Matrix =
    Matrix(Vector.fill(dim._1, dim._2)(value))
}
\end{Code}

\begin{REPLnonum}
scala> val m = Matrix.fill(3,4)("hej")
scala> val e = m(2, 2)
\end{REPLnonum}

\Subtask Vad får \code{m} ovan för typ?

\Subtask Vad får \code{e} ovan för typ?

\Subtask På hur många ställen måste du ändra i \code{Matrix} ovan för att den i stället ska representera en matris av heltal?

\Subtask Du ska nu med hjälp av en \textbf{typparameter} göra \code{Matrix} \textbf{generisk} \Eng{generic}, så att den blir en mer användbar matrisklass som kan innehålla element av vilken typ som helst. Genomför följande ändringar i \code{Matrix.scala}:

\begin{itemize}[noitemsep, nolistsep]
  \item Lägg till en typparameter \code{T} inom klammerparenteser efter namnet \code{Matrix} på alla ställen där det förekommer \emph{utom} efter namnet på kompanjonsobjektet\footnote{Singelobjekt kan inte ha typparametrar, men deras medlemmar kan.}.
  \item Byt ut \code{String} mot \code{T} på alla ställen där \code{String} förekommer.
  \item Lägg till en typparameter \code{T} inom klammerparenteser efter \code{def fill}.
\end{itemize}
Testa din generiska klass i REPL genom att skapa en boolesk matris:
\begin{REPLnonum}
scala> val bm = Matrix.fill(3,4)(false)
scala> val be = bm(0, 0)
\end{REPLnonum}

\Subtask Vad får \code{bm} ovan för typ?

\Subtask Vad får \code{be} ovan för typ?

\Subtask Lägg en kodrad i början av klasskroppen som med hjälp av \code{require} garanterar att alla rader i matrisen är lika långa.

\Subtask Lägg till en medlem \code{val dim: (Int, Int)} i klasskroppen efter \code{require}-satsen som ger ett par (alltså en 2-tupel) med antalet rader resp. kolumner i matrisen.

\Subtask Lägg till en metod \code{def updated(row: Int, col: Int)(value: T): Matrix[T]} som ger en ny matris där element på platsen \code{(row, col)} har uppdaterats till \code{value}.

\Subtask Lägg till en metod \code{def foreachIndex(f: (Int, Int) => Unit): Unit} som för varje index i \code{data} applicerar funktionen \code{f}.

\Subtask Lägg till en metod \code{override def toString} som så att en instans av \code{Matrix} visas enligt följande:
\begin{REPLnonum}
scala> val dm = Matrix.fill(3,4)(42.0)
val dm: Matrix[Double] =
Matrix of dim (3,4):
42.0 42.0 42.0 42.0
42.0 42.0 42.0 42.0
42.0 42.0 42.0 42.0
\end{REPLnonum}


\SOLUTION


\TaskSolved \what

\SubtaskSolved Typen på \code{m} blir \code{Matrix}.

\SubtaskSolved Typen på \code{e} blir \code{String}.

\SubtaskSolved Man behöver ändra på 3 ställen från \code{String} till \code{Int}.

\SubtaskSolved Generisk matris \code{Matrix[T]} för element av godtycklig typ \code{T}:

\begin{CodeSmall}
case class Matrix[T](data: Vector[Vector[T]]):
  def apply(row: Int, col: Int): T = data(row)(col)

object Matrix:
  def fill[T](dim: (Int, Int))(value: T): Matrix[T] =
    Matrix[T](Vector.fill(dim._1, dim._2)(value))
\end{CodeSmall}

\SubtaskSolved Tack vare kompilatorns typinferens så får \code{bm} typen \code{Matrix[Boolean]}.

\SubtaskSolved Typen på \code{be} blir \code{Boolean}.

\noindent \SubtaskSolved \SubtaskSolved \SubtaskSolved \SubtaskSolved \SubtaskSolved är alla implementerade i koden nedan: \vspace{-0.5em}
\begin{CodeSmall}
case class Matrix[T](data: Vector[Vector[T]]):
  require(data.forall(row => row.length == data(0).length))

  val dim: (Int, Int) = (data.length, data(0).length)

  def apply(row: Int, col: Int): T = data(row)(col)

  def updated(row: Int, col: Int)(value: T): Matrix[T] =
    Matrix(data.updated(row, data(row).updated(col, value)))

  def foreachIndex(f: (Int, Int) => Unit): Unit =
    for r <- data.indices; c <- data(r).indices do f(r, c)

  override def toString =
    s"""Matrix of dim $dim:\n${ data.map(_.mkString(" ")).mkString("\n") }"""

object Matrix:
  def fill[T](dim: (Int, Int))(value: T): Matrix[T] =
    Matrix[T](Vector.fill(dim._1, dim._2)(value))

\end{CodeSmall}

\QUESTEND


\clearpage

\ExtraTasks %%%%%%%%%%%%%%%%%%%%%%%%%%%%%%%%%%%%%%%%%%%%%%%%%


\WHAT{Imperativa matrisalgoritmer.}

\QUESTBEGIN

\Task  \what~Imperativa angreppssätt är nödvändiga att kunna när du stöter på samlingar och/eller språk som saknar funktionella metoder och/eller funktionsprogrammeringsmöjligheter. Genom att studera imperativa lösningar till de ofta mer koncisa funktionella lösningarna, får du träning i att skapa algoritmer som använder förändring genom tilldelning vid iterering.

\Subtask Implementera \code{isYatzy} från uppgift \ref{matrices:task:yatzy}\ref{matrices:subtask:isyatzyforall} igen, men nu med ett imperativt angreppssätt som använder en \code{while}-sats i stället för funktionella \code{forall}. Ta hjälp av en variabel \code{i} som håller reda på index och en variabel \code{foundDiff} som håller reda på om ett avvikande värde upptäcks. Funktionen kräver ca 9 rader, så det kan vara lämpligt att öppna en editor att skriva i medan du klurar ut lösningen. Börja med att skriva pseudokod, gärna med penna på papper. Prova genom att klistra in i REPL.

\Subtask En imperativ implementation av \code{diceMatrixToString} från uppgift \ref{matrices:task:yatzy}\ref{matrices:subtask:diceMatrixToString} med hjälp av förändringsbara  \code{StringBuilder}\footnote{\url{https://www.scala-lang.org/api/2.12.9/scala/collection/mutable/StringBuilder.html}} visas nedan. Förklara hur nedan kod fungerar. Vad händer om \code{xss} är tom? Vad händer om \code{xss} bara innehåller tomma vektorer? Nämn en fördel och en nackdel med att använda \code{val sb: StringBuilder} och \code{append}, jämfört med en vanlig, oföränderlig \code{var s: String} och \code{+} för tillägg i slutet.
\begin{Code}
def diceMatrixToString(xss: Vector[Vector[Int]]): String = 
  val sb = new StringBuilder()
  for(m <- xss.indices) do
    for(n <- xss(m).indices) do
      sb.append(xss(m)(n).toString)
      if n < xss(m).size - 1 then sb.append(" ")
      else if m < xss.size - 1 then sb.append("\n")
    end for
  end for
  sb.toString
\end{Code}

\Subtask Gör som träning en imperativ implementation av \code{filterYatzy} med en \code{for}-\code{do}-sats (alltså utan att använda \code{filter}, och utan att använda \code{yield}).


\Subtask Förklara hur nedan funktionella implementation av \code{filterYatzy} med \code{for}-\code{yield}-uttryck fungerar. Tycker du din imperativa lösning är lättare eller svårare att läsa och förstå jämfört nedan funktionella lösning?
\begin{CodeSmall}
def filterYatzy(xss: Vector[Vector[Int]]): Vector[Vector[Int]] = 
  (for i <- xss.indices if isYatzy(xss(i)) yield xss(i)).toVector
\end{CodeSmall}


\SOLUTION

\TaskSolved \what

\SubtaskSolved  \begin{Code}
def isYatzy(xs: Vector[Int]): Boolean = 
  var foundDiff = false
  var i = 0
  while (i < xs.size && !foundDiff) do
    foundDiff = xs(i) != xs(0)
    i += 1
  end while
  !foundDiff
\end{Code}


\SubtaskSolved  Funktionen går igenom varje matrisrad, där den i sin tur går igenom
varje element på raden och lägger till i \code{StringBuilder}-objektet. Om det inte är
det sista elementet på raden läggs även ett blanktecken till, annars läggs ett
nyradstecken till. Undantaget är sista raden, där inget nyradstecken läggs till.
Slutligen konverteras \code{StringBuilder}-objektet till en \code{String} som
returneras.


Är \code{xss} tom blir \code{xss.indices} en tom \code{Range} och den yttre \code{for}-loopen hoppas över och en tom sträng returneras.
Är alla rader tomma hoppas i stället de inre \code{for}-looparna över, med samma resultat.

\emph{Fördel:} \code{StringBuilder} är snabbare vid tillägg på slutet vid stora strängar (men här kommer det inte märkas eftersom strängen är så liten).

\emph{Nackdel:} StringBuilder-koden uppfattas av många som svårare att läsa.

\SubtaskSolved
\begin{Code}
def filterYatzy(xss: Vector[Vector[Int]]): Vector[Vector[Int]] = 
  var result: Vector[Vector[Int]] = Vector()
  for i <- xss.indices if isYatzy(xss(i)) do result = result :+ xss(i)
  result
\end{Code}

\SubtaskSolved  Varje looprunda ger en vektor \code{xss(i)} om filtervillkoret är uppfyllt och resultatet av \code{for}-uttrycket blir en vektor med vektorer som är yatzyslag.

\QUESTEND



\WHAT{Strängtabell med kolumnrubriker.}

\QUESTBEGIN

\Task  \what~  %Denna övning utgör en början på laboration \hyperref[section:lab:survey]{\texttt{survey}} i avsnitt \ref{section:lab:survey} på sidan \pageref{section:lab:survey}.

\Subtask Implementera case-klassen \code{Table} enligt specifikationen nedan. Du kan förutsätta att alla rader har lika många kolumner som antalet element i \code{headings}, samt att alla rubrikerna i \code{headings} är unika. Parametern \code{sep} anger det tecken som används för att separera kolumner. Detta förutsätts också gälla för indatafiler som läses in med \code{fromFile}.

\emph{Tips:}
\begin{itemize}%[nolistsep,noitemsep]
\item Värdet \code{indexOfHeading} kan skapas med hjälp av metoden \code{zipWithIndex} som fungerar på alla sekvenssamlingar, samt metoden \code{toMap} som fungerar på sekvenser av 2-tupler. Undersök först hur metoderna fungerar i REPL och sök upp deras dokumentation.
\item Skapa en indatafil som du kan använda för att testa att \code{Table} fungerar.
\end{itemize}


\begin{CodeSmall}
case class Table(
  data: Vector[Vector[String]],
  headings: Vector[String],
  sep: Char
):
  /** A 2-tuple with (number of rows, number of columns) in data */
  val dim: (Int, Int) = ???

  /** The element in row r and column c of data, counting from 0 */
  def apply(r: Int, c: Int): String = ???

  /** The row-vector r in data, counting from 0 */
  def row(r: Int): Vector[String]= ???

  /** The column-vector c in data, counting from 0 */
  def col(c: Int): Vector[String] = ???

  /** A map from heading to index counting from 0 */
  lazy val indexOfHeading: Map[String, Int] = ???

  /** The column-vector with heading h in data */
  def col(h: String): Vector[String] = ???

  /** A vector with the distinct, sorted values of col with heading h */
  def values(h: String): Vector[String] = ???

  /** Headings and data with columns separated by sep */
  override lazy val toString: String = ???

object Table:
  /** Creates a new Table from fileName with columns split by sep */
  def fromFile(fileName: String, sep: Char = ';'): Table = ???
\end{CodeSmall}

\Subtask Skapa med hjälp av \code{Table} ett program som kan köras från terminalen med \texttt{scala run infile.csv ';'} som ger en utskrift av antalet förekomster av olika värden i respektive kolumn (alltså en variant av registrering).



\SOLUTION

\TaskSolved \what

\SubtaskSolved  \begin{CodeSmall}
case class Table(
  data: Vector[Vector[String]],
  headings: Vector[String],
  sep: Char
):

  val dim: (Int, Int) = (data.size, headings.size)

  def apply(r: Int, c: Int): String = data(r)(c)

  def row(r: Int): Vector[String]= data(r)

  def col(c: Int): Vector[String] = data.map(r => r(c))

  lazy val indexOfHeading: Map[String, Int] = headings.zipWithIndex.toMap

  def col(h: String): Vector[String] = col(indexOfHeading(h))

  def values(h: String): Vector[String] = col(h).distinct.sorted

  override def toString: String =
    val s = sep.toString
    headings.mkString(s) + "\n" +data.map(_.mkString(s)).mkString("\n")

object Table:
  def fromFile(fileName: String, sep: Char = ';'): Table = 
    val lines = scala.io.Source.fromFile(fileName).getLines.toVector
    val matrix= lines.map(_.split(sep).toVector)
    new Table(matrix.tail, matrix.head, sep)
\end{CodeSmall}

\SubtaskSolved  \begin{CodeSmall}
@main 
def run(fileName: String, separator: String): Unit = 
  require(separator.length == 1, "separator ska vara exakt ett tecken")
  val t = Table.fromFile(fileName, separator.head)
  val counts: Vector[Vector[String]] =
    (0 until t.dim._2)
      .map(i => t.values(t.headings(i))
      .map(x => s"$x: ${t.col(i).count(_ == x)}"))
      .toVector
  for (i <- 0 until t.dim._2) do
    println(s"\nColumn: ${i + 1}, ${t.headings(i)}:")
    for (j <- 0 until counts(i).length) do
      println(counts(i)(j))
\end{CodeSmall}

\QUESTEND




\WHAT{Skapa ett yatzy-spel för användning i terminalen.}

\QUESTBEGIN

\Task  \what~%
% \Subtask Skapa en yatzy-matris enligt nedan specifikation. Läs om hur de olika predikaten för att kolla olika giltiga kombinationer i Yatzy ska fungera här: \href{https://en.wikipedia.org/wiki/Yahtzee}{en.wikipedia.org/wiki/Yahtzee}. Bygg ett huvudprogram som testar dina funktioner. Kompilera och testa i terminalen allteftersom du lägger till nya funktioner.
%
% \begin{CodeSmall}
% /** En skiss på en klass som kan användas till ett förenklat yatzy-spel */
% case class YatzyRows(val rows: Vector[Vector[Int]]) {
%   /** A new YatzyRows with a new row of 5 dice rolls appended to rows  */
%   def roll: YatzyRows = ???
%
%   /** A new YatzyRows with some indices of the last row re-rolled  */
%   def reroll(indices: Vector[Int]): YatzyRows = ???
% }
%
% object YatzyRows {
%   def isYatzy(xs: Vector[Int]): Boolean = ???
%   def isThreeOfAKind(xs: Vector[Int]): Boolean = ???
%   def isFourOfAKind(xs: Vector[Int]): Boolean = ???
%   def isFullHouse(xs: Vector[Int]): Boolean = ???
%   def isSmallStraight(xs: Vector[Int]): Boolean = ???
%   def isLargeStraight(xs: Vector[Int]): Boolean = ???
% }
% \end{CodeSmall}
%
%
% \Subtask Använd \code{YatzyRows} för att med hjälp av många tärningskast beräkna sannolikheter för några olika giltiga kombinationer. Använd, om du vill, möjligheten som reglerna ger att slå om tärningar i två ytterliggare kast, där de tärningar som slås om väljs slumpmässigt.
%
%\Subtask
Bygg ett förenklat yatzy-spel i terminalen där användaren kan bestämma vilka tärningar som ska slås om. Börja med något riktigt enkelt och bygg sedan vidare på ditt spel genom att införa fler och fler funktioner.

\SOLUTION


\TaskSolved \what
     %starts with: \emph{Skapa ett yatzy-spel för %%%

 --

% \SubtaskSolved   \begin{CodeSmall}
% /** En skiss på en klass som kan användas till ett förenklat yatzy-spel */
% case class YatzyRows(val rows: Vector[Vector[Int]]) {
%
%   private def throwDie: Int = (math.random() * 6).toInt + 1
%
%   /** A new YatzyRows with a new row of 5 dice rolls appended to rows */
%   def roll: YatzyRows = new YatzyRows(rows :+ Vector.fill(5)(throwDie))
%
%   /** A new YatzyRow with some indices of the last row re-rolled */
%   def reroll(indices: Vector[Int]): YatzyRows =
%     new YatzyRows(rows :+ rows(rows.length - 1).zipWithIndex.map {
%       case (x, i) => if (indices.contains(i)) throwDie else x
%     })
% }
% object YatzyRows {
%
%   def isYatzy(xs: Vector[Int]): Boolean = xs.forall(_ == xs(0))
%
%   def isThreeOfAKind(xs: Vector[Int]): Boolean =
%     xs.exists(x => xs.count(_ == x) >= 3)
%
%   def isFourOfAKind(xs: Vector[Int]): Boolean =
%     xs.exists(x => xs.count(_ == x) >= 4)
%
%   def isFullHouse(xs: Vector[Int]): Boolean =
%     xs.exists(x => xs.count(_ == x) == 3) &&
%     xs.exists(x => xs.count(_ == x) == 2)
%
%   def isSmallStraight(xs: Vector[Int]): Boolean =
%     xs.forall(x => xs.count(_ == x) == 1) && !xs.exists(_ == 6)
%
%   def isLargeStraight(xs: Vector[Int]): Boolean =
%     xs.forall(x => xs.count(_ == x) == 1) && !xs.exists(_ == 1)
% }
%
% \end{CodeSmall}
% Observera att fem stycken 2:or uppfyller kraven för Yatzy, men även för triss och fyrtal.
%
% \SubtaskSolved   Slumpen gör att utfallet inte kommer stämma exakt överens med teorin, men för ett stort antal kast bör resultaten hamna ganska nära. De teoretiska sannolikheterna (utan omkast) finns i \ref{yatzyProb}.
% \begin{table}[h]
% \centering
% \caption{Sannolikhet för olika Yatzy-resultat}
% \label{yatzyProb}
% \begin{tabular}{ll}
% Yatzy&  $0,077\%$  \\
% $\geq3$ av samma& $21\%$\\
% $\geq4$ av samma& $2,0\%$\\
% Kåk& $3,9\%$\\
% Liten stege& $1,5\%$\\
% Stor stege& $1,5\%$
% \end{tabular}
% \end{table}
%
% Kodexempel:
% \begin{CodeSmall}
% import YatzyRows._
%
% object YatzyStats extends App {
%   val n = 1000000.0
%   var yr = YatzyRows(Vector(Vector[Int]()))
%   for (i <- 1 to n.toInt) yr = yr.roll
%   println(s"Yatzy: ${yr.rows.count(isYatzy(_)) / n * 100}%")
%   println(s"Three of a kind: ${yr.rows.count(isThreeOfAKind(_)) / n * 100}%")
%   println(s"Four of a kind: ${yr.rows.count(isFourOfAKind(_)) / n * 100}%")
%   println(s"Full house: ${yr.rows.count(isFullHouse(_)) / n * 100}%")
%   println(s"Small straight: ${yr.rows.count(isSmallStraight(_)) / n * 100}%")
%   println(s"Large straight: ${yr.rows.count(isLargeStraight(_)) / n * 100}%")
% }
% \end{CodeSmall}
%
% \SubtaskSolved  --

\QUESTEND






\clearpage

\AdvancedTasks %%%%%%%%%%%%%%%%%


\WHAT{Generiska funktioner.}

\QUESTBEGIN

\Task  \what~  En generisk funktion har (minst) en typparameter inom klammerparenteser efter namnet, till exempel \code{[T]}. Denna typ förekommer sedan som typ på (någon av) parametrarna i parameterlistan. Kompilatorn härleder en konkret typ vid kompileringstid och ersätter typparametern med denna konkreta typ. På så sätt kan en funktion fungera för många olika typer.

\Subtask Förklara för varje rad nedan vad som händer.

\begin{REPL}
scala> def tnirp[T](x: T): Unit = println(x.toString.reverse)
scala> tnirp(42)
scala> tnirp("hej")
scala> case class Gurka(vikt: Int)
scala> tnirp(Gurka(42))
scala> tnirp[String](42)
scala> tnirp[Double](42)
\end{REPL}

\Subtask Man kan kombinera generiska funktioner med funktioner som tar funktioner som parametrar. Det är så \code{map} och \code{foreach} är implementerade. Förklara för varje rad nedan vad som händer.

\begin{REPL}
scala> def compose[A, B, C](f: A => B, g: B => C)(x: A): C = g(f(x))
scala> def inc(x: Int): Int = x + 1
scala> def half(x: Int): Double = x / 2.0
scala> compose(inc, half)(42)
scala> compose(half, inc)(42)
\end{REPL}

\Subtask Hur lyder felmeddelandet på sista raden ovan? Ändra \code{inc} och/eller \code{half} så att typerna passar.

\SOLUTION

\TaskSolved \what
     %starts with: \emph{Generiska funkioner.} En %%%

%4.a)
\SubtaskSolved   \begin{enumerate}
\item --
\item Strängrepresentationen av \code{42} spegelvänds
\item \code{"hej"} spegelvänds - \code{toString} av en sträng ger en likadan sträng
\item --
\item Gurk-objektets strängrepresentation spegelvänds
\item Funktionens typparameter matchar inte parameterns typ: \code{42} är ingen sträng
\item Implicit typkonvertering till \code{Double} sker för att stämma överens med typparametern, vilket ger en strängrepresentation med decimal
\end{enumerate}

%4.b)
\SubtaskSolved   \begin{enumerate}
\item En funktion definieras så att den tar emot två andra funktioner som argument, sätter ihop dem, och matar in ett tredje argument till den den sammansatta funktionen.
\item En funktion som inkrementerar ett heltal med 1 definieras.
\item En funktion som halverar ett flyttal definieras.
\item \code{42} matas in i \code{inc()} och resultatet (\code{43}) matas vidare till \code{half()}. Inuti \code{half()} sker implicit typkonvertering till \code{Double} då talet divideras med ett flyttal (\code{2.0}) och resultatet blir \code{43.0 / 2.0}, alltså \code{21.5}.
\item Resultatet från \code{half()} är av typ \code{Double}, medan \code{inc()} tar emot ett argument av typ \code{Int}. Då flyttal generellt inte kan konverteras till heltal utan informationsförlust sker ingen implicit konvertering, istället sker ett kompileringsfel.
\end{enumerate}

%4.c)
\SubtaskSolved  \begin{Code}
def inc(x: Double): Double = x + 1.0
\end{Code}
Nu ges kompileringsfel på rad 4 istället, vilket kan lösas med följande ändring:
\begin{Code}
def half(x: Double): Double = x / 2.0
\end{Code}

\QUESTEND




\WHAT{Generiska klasser.}

\QUESTBEGIN

\Task  \what~  Även klasser kan vara generiska. En generisk klass har (minst) en typparameter inom klammerparenteser efter klassens namn.

\Subtask Testa nedan generiska klass \code{Cell[T]} i REPL. Skapa instanser av klassen \code{Cell[T]} där typparametern \code{T} binds till olika konkreta typer och förklara vad som händer.

\begin{REPL}
scala> class Cell[T](var value: T):
         override def toString = "Cell(" + value + ")"
       
scala> new Cell(42)
scala> new Cell("hej")
scala> new Cell(new Cell(math.Pi))
scala> new Cell[String](42)
scala> new Cell[Double](42)
\end{REPL}

\Subtask Lägg till metoden \code{def concat[U](that: Cell[U]):Cell[String]} i klassen \code{Cell} som konkatenerar strängrepresentationerna av de båda cellvärdena.

\begin{REPL}
scala> val a = new Cell("hej")
scala> val b = new Cell(42)
scala> a concat b
\end{REPL}

\Subtask Vilken sorts celler kan du konkatenera om du tar bort typparameternamnet \code{U} i \code{concat} samtidigt som du använder \code{Cell[T]} som typ på värdeparametern \code{that}? Vad ger det för konsekvenser för celler av annan typ än \code{Cell[String]}?

\SOLUTION

\TaskSolved \what

%5.a)
\SubtaskSolved  --

%5.b)
\SubtaskSolved  \begin{Code}
class Cell[T](var value: T):
  override def toString = "Cell(" + value + ")"
  def concat[U](that: Cell[U]): Cell[String] = 
    Cell(s"$value${that.value}")
\end{Code}

%5.c)
\SubtaskSolved   Endast celler med samma typparameter kan nu konkateneras. Eftersom \code{concat()} returnerar ett objekt av typ \code{Cell[String]} kan ett ojämnt antal celler med någon annan typparameter än \code{String} alltså inte längre konkateneras. Är antalet jämnt går det att konkatenera dem parvis och sedan konkatenera de returnerade \code{Cell[String]}-objekten, men det är något omständigt.

\QUESTEND

\WHAT{Implementera fler generiska metoder i \code{Matrix[T]}.}

\QUESTBEGIN

\Task \what~ Bygg vidare på uppgift \ref{exe:matrices:labprep} och implementera nedan specifikation. Skapa egna tester som kontrollerar att alla metoder fungerar som förväntat.

\begin{ScalaSpec}{Matrix[T]}
/** En oföränderlig, generisk Matris-klass. */
case class Matrix[T](data: Vector[Vector[T]]):
  require(???)  // garantera att alla rader har lika många kolumner

  /** Ger ett par med antal rader och kolumner. */
  val dim: (Int, Int) = ???

  /** Ger elementet på plats (row, col). */
  def apply(row: Int, col: Int): T = ???

  /** Ger en ny matris där elementet på plats (row, col) har värdet value. */
  def updated(row: Int, col: Int)(value: T): Matrix[T] =  ???

  /** Applicerar f på alla element. */
  def foreach(f: T => Unit): Unit = ???

  /** Applicerar f på alla index. */
  def foreachIndex(f: (Int, Int) => Unit): Unit = ???

  /** Ger en ny matris med resultaten av elementvis applicering av f. */
  def map[U](f: T => U): Matrix[U] = ???

  /** Ger en ny matris med resultaten av applicering av f på varje index. */
  def mapIndex[U](f: (Int, Int) => U): Matrix[U] = ???

  /** Ger en utskriftsvänlig strängrepresentation av matrisen. */
  override def toString = ???

object Matrix:
  /** Ger en matris med dimension dim där alla element har värdet value. */
  def fill[T](dim: (Int, Int))(value: T): Matrix[T] = ???
\end{ScalaSpec}

\SOLUTION


\TaskSolved \what

\begin{CodeSmall}
case class Matrix[T](data: Vector[Vector[T]]):
  require(data.forall(row => row.size == data(0).size))

  val dim: (Int, Int) = (data.length, data(0).length)

  def apply(row: Int, col: Int): T = data(row)(col)

  def updated(row: Int, col: Int)(value: T): Matrix[T] =
    Matrix(data.updated(row, data(row).updated(col, value)))

  def foreach(f: T => Unit): Unit = data.foreach(_.foreach(f))

  def foreachIndex(f: (Int, Int) => Unit): Unit =
    for r <- data.indices; c <- data(r).indices do f(r, c)

  def map[U](f: T => U): Matrix[U] = Matrix(data.map(_.map(f)))

  def mapIndex[U](f: (Int, Int) => U): Matrix[U] =
    var result = Matrix.fill(dim)(f(0,0))
    for 
      r <- data.indices
      c <- data(r).indices 
    do
      result = result.updated(r, c)(f(r, c))
    end for
    result

  override def toString =
    s"""Matrix of dim $dim:\n${ data.map(_.mkString(" ")).mkString("\n") }"""

object Matrix:
  def fill[T](dim: (Int, Int))(value: T): Matrix[T] =
    Matrix[T](Vector.fill(dim._1, dim._2)(value))
\end{CodeSmall}


\QUESTEND





% \WHAT{Skapa en generisk, oföränderlig matrisklass.}
%
% \QUESTBEGIN
%
% \Task \label{task:generic-matrix} \what~   Med hjälp av en typparameter kan vi skapa en matrisklass som kan innehålla vilka element som helst. Implementera nedan specifikation. Testa din matrisklass i REPL för olika typer av element.
%
% \begin{ScalaSpec}{Matrix[T]}
% case class Matrix[T](data: Vector[Vector[T]]){
%
%   def foreachRowCol(f: (Int, Int, T) => Unit): Unit =
%     for (r <- 0 until data.size) {
%       for (c <- 0 until data(r).size) {
%         f(r, c, data(r)(c))
%       }
%     }
%
%   def map[U](f: T => U): Matrix[U] = Matrix(data.map(_.map(f)))
%
%   /** The element at row r and column c */
%   def apply(r: Int, c: Int): T = ???
%
%   /** Gives Some[T](element) at row r and column c
%    *  if r and c are within index bounds, else None */
%   def get(r: Int, c: Int): Option[T] = ???
%
%   /** The row vector of row r */
%   def row(r: Int): Vector[T] = ???
%
%   /** The column vector of column c */
%   def col(c: Int): Vector[T] = ???
%
%   /** A new Matrix with element at row r and col c updated */
%   def updated(r: Int, c: Int, value: T): Matrix[T] = ???
% }
% object Matrix {
%   def fill[T](rowSize: Int, colSize: Int)(init: T): Matrix[T] =
%     new Matrix(Vector.fill(rowSize)(Vector.fill(colSize)(init)))
% }
% \end{ScalaSpec}
%
% \SOLUTION
%
%
% \TaskSolved \what
%      %%%TODO number  8 %%%starts with: \label{task:generic-matrix} \em%%%
%
% \SubtaskSolved  -- %%%TODO in task 8 %%%
%
%
%
% \QUESTEND
%

% \clearpage
%
% \WHAT{Skapa en Sprite-editor.}
%
% \QUESTBEGIN
%
% \Task  \what~ Använd matrisklassen från uppgift \ref{task:generic-matrix} för att göra en SpriteEditor med JColorChoser enligt nedan skiss.
%
% \begin{Code}
% object ColorChooser {
%   import java.awt.Color
%   import javax.swing.JColorChooser
%
%   var title = "Pick Color"
%   private val chooser = new JColorChooser(Color.BLACK)
%   private val dialog = JColorChooser.
%     createDialog(null, title, true, jcs, null, null)
%
%   def getColor(initColor: Color = Color.BLACK): Color = {
%     chooser.setColor(initColor)
%     dialog.setVisible(true)
%     chooser.getColor
%   }
% }
%
% class Sprite(// en bild med många lager av pixlar i olika färger
%   val id: String,
%   val size: (Int, Int),
%   val pixels: Matrix[Int],   // färg i colors, -1 betyder genomskinlig
%   var scale: Int,            // uppskalning av storlek i pixlar
%   var colors: Vector[Color], // tillgängliga färger
%   var pos: (Int, Int, Int)   // (row, col, layer)
% ){
%   def row = pos._1
%   def col = pos._2
%   def layer = pos._3
% }
%
% class SpriteEditor(
%     rows: Int = 64, cols: Int = 64,
%     scale: Int = 16, nColors: Int = 16) {
%   private val w = new SimpleWindow(???)
%   def edit: Unit = ???
% }
%
% \end{Code}
%
%
%
% \SOLUTION
%
%
% \TaskSolved \what
%      %%%TODO number  9 %%%starts with: \TODO \emph{Klasser för täta oc%%%
%
% \SubtaskSolved  -- %%%TODO in task 9 %%%
%
% \SubtaskSolved  -- %%%TODO in task 9 %%%
%
% \SubtaskSolved  -- %%%TODO in task 9 %%%
%
% \SubtaskSolved  -- %%%TODO in task 9 %%%
%
% \SubtaskSolved  -- %%%TODO in task 9 %%%
%
% \SubtaskSolved  -- %%%TODO in task 9 %%%
%
%
%
% \QUESTEND




% \WHAT{Klasser för täta och glesa matematiska matriser med flyttal.}
%
% \QUESTBEGIN
%
% \Task  \what~   Läs om matrisräkning här: \href{https://sv.wikipedia.org/wiki/Matris}{sv.wikipedia.org/wiki/Matris}
%
% \Subtask Skapa en oföränderlig klass \code{DenseMatrix} för matematiska matriser med dubbelprecisionsflyttal. \code{DenseMatrix} ska internt lagra elementen i en privat \emph{endimensionell} array av flyttal av typen \code{Array[Double]}.
%
% Klassen ska inte vara en case-klass. Det ska gå att skapa matriser med uttryck så som  \code{DenseMatrix.ofDim(3,7)(1.0,42,3.2,1.0,2.2,3)} tack vare ett kompanjonsobjekt med lämplig fabriksmetod som anropar den privata konstruktorn.  Om antalet element är för litet i förhållande till den angivna dimensionen så fyll på med nollor.
%
% \Subtask Överskugga metoderna equals och hashcode och ge \code{DenseMatrix} innehållslikhet i stället för referenslikhet.
%
% \Subtask Implementera egna innehålllikhetsmetoder med namnet \code{===} på \code{DenseMatrix} som är typsäker, d.v.s. bara tillåter innehållsjämförelse mellan täta matriser.
%
% \Subtask Läs om glesa matriser här: \href{https://sv.wikipedia.org/wiki/Gles_matris}{https://sv.wikipedia.org/wiki/Gles\_matris} och implementera \code{SparseMatrix} med ett privat attribut av typen \\ \code{mutable.Map[(Int, Int), Double]} som bara lagrar index som inte är noll.
%
% \Subtask Skapa ett \code{trait Matrix} som både \code{DenseMatrix} och \code{SparseMatrix} ärver, med lämpliga abstrakta och konkreta medlemmar. Implementera addition, subtraktion och multiplikation av täta och glesa matriser.
%
% %\Task \emph{Matriser med \jcode{ArrayList} i Java.} Om man i Java inte vet antalet element i matrisen från början kan man använda en lista av typen \jcode{ArrayList}, där varje element i sin tur innehåller en lista av typen\jcode{ArrayList}. Javas \jcode{ArrayList} är en generisk samling som motsvaras av Scalas \code{ArrayBuffer}. Generiska samlingar i Java kan endast innehålla referenstyper; vill man ha en primitiv typ, t.ex. \jcode{int}, behöver man packa in denna i en s.k. wrapper-klass, t.ex.  klassen \jcode{Integer}. Det finns en wrapper-klass för varje primitiv typ i Java. Matristypen för en heltalstyp i Java skrivs \jcode{ArrayList<ArrayList<Integer>>} där alltså \code{<T>} motsvarar Scalas hakparenteser \code{[T]} för typparametern T.
% %
% %
%
% \SOLUTION
%
% \TaskSolved \what
%      %%%TODO number  10 %%%starts with: \emph{Matriser med \jcode{Array%%%
%
% \SubtaskSolved  -- %%%TODO in task 10 %%%
% \QUESTEND

%!TEX encoding = UTF-8 Unicode

%!TEX root = ../compendium2.tex

\Exercise{\ExeWeekNINE}\label{exe:W09}

\begin{Goals}
%!TEX encoding = UTF-8 Unicode

%!TEX root = ../compendium2.tex

\item Känna till begrepp:
bastyp,
sypertyp,
subtyp,
körtidstyp,
dynamisk bindning,
polymorfism,
trait,
inmixning,
överskuggad medlem,
anonym klass,
skyddad medlem,
abstrakt medlem,
abstrakt klass,
referenstyp,
värdetyp.

\item Kunna deklarera och använda en arvshierarki i flera nivåer.

\item Känna till synlighetsregler vid arv och nyttan med privata och skyddade medlemmar.

\item Kunna deklarera och använda skyddade medlemmar.

\item Kunna deklarera och använda överskuggade medlemmar.

\item Känna till reglerna som gäller vid överskuggning av olika sorters medlemmar.

\item Kunna deklarera och använda en hierarki av klasser där konstruktorparametrar överförs till superklasser.

\item Kunna deklarera och använda uppräknade värden med case-objekt och gemensam bastyp.

\item Kunna deklarera och känna till nyttan med finala klasser och finala attribut och nyckelordet \code{final}.

%TODO KOLLA PÅ NEDAN MÅL OCH BESTÄM HUR DE SKA IN I ÖVNINGARNA

%\item Känna till hur typtester och typkonvertering under körtid kan göras med metoderna \code{isInstanceOf} och \code{asInstanceOf} och känna till att detta görs bättre med \code{match}.

%\item Kunna deklarera och använda inmixning med flera traits och nyckelordet \code{with}.

%\item Kunna referera till medlem i superklassen med referensen \code{super} och känna till när detta nyckel ord behövs.

%\item Känna till begreppet anonym klass.

\end{Goals}

\begin{Preparations}
\item \StudyTheory{09}
\end{Preparations}

\BasicTasks %%%%%%%%%%%%%%%%


\Task \emph{Gemensam bastyp.} Man vill ofta lägga in objekt av olika typ i samma samling.
\begin{REPL}
scala> class Gurka(val vikt: Int)
scala> class Tomat(val vikt: Int)
scala> val gurkor = Vector(new Gurka(100), new Gurka(200))
scala> val grönsaker = Vector(new Gurka(300), new Tomat(42))
\end{REPL}
\Subtask Om en samling innehåller objekt av flera olika typer försöker kompilatorn härleda den mest specifika typen som objekten har gemensamt. Vad blir det för typ på värdet \code{grönsaker} ovan?

\Subtask Försök ta reda på summan av vikterna enligt nedan. Vad ger andra raden för felmeddelande? Varför?

\begin{REPL}
scala> gurkor.map(_.vikt).sum
scala> grönsaker.map(_.vikt).sum
\end{REPL}

\Subtask Vi kan göra så att vi kan komma åt vikten på alla grönsaker genom att ge gurkor och tomater en gemensam bastyp som de olika konkreta grönsakstyperna utvidgar med nyckelordet \code{extends}. Man säger att subtyperna \code{Gurka} och \code{Tomat} \textbf{ärver} egenskaperna hos supertypen \code{Grönsak}.

Attributet \code{vikt} i traiten \code{Grönsak} nedan initialiseras inte förrän konstruktorerna anropas när vi gör \code{new} på någon av klasserna \code{Gurka} eller \code{Tomat}.

\begin{REPL}
scala> trait Grönsak { val vikt: Int }
scala> class Gurka(val vikt: Int) extends Grönsak
scala> class Tomat(val vikt: Int) extends Grönsak
scala> val gurkor = Vector(new Gurka(100), new Gurka(200))
scala> val grönsaker = Vector(new Gurka(300), new Tomat(42))
\end{REPL}

\Subtask Vad blir det nu för typ på variabeln \code{grönsaker} ovan?

\Subtask Fungerar det nu att räkna ut summan av vikterna i \code{grönsaker} med \code{grönsaker.map(_.vikt).sum}?


\Subtask En trait liknar en klass, men man kan inte instansiera den och den kan inte ha några parametrar. En typ som inte kan instansieras kallas \textbf{abstrakt} \Eng{abstract}. Vad blir det för felmeddelande om du försöker göra \code{new} på en trait enligt nedan?
\begin{REPL}
scala> trait Grönsak { val vikt: Int }
scala> new Grönsak
\end{REPL}


\Subtask Traiten \code{Grönsak} har en abstrakt medlem \code{vikt}. Den sägs vara abstrakt eftersom den saknar definition -- medlemmen har bara ett namn och en typ men inget värde. Du kan instansiera den abstrakta traiten \code{Grönsak} om du fyller i det som ''fattas'', nämligen ett värde på \code{vikt}. Man kan fylla på det som fattas i genom att ''hänga på'' ett block efter typens namn vid instansiering. Man får då vad som kallas en \textbf{anonym} klass, i detta fall en ganska konstig grönsak som inte är någon speciell sorts grönsak med som ändå har en vikt.

Vad får \code{anonymGrönsak} nedan för typ och strängrepresenation?
\begin{REPL}
scala> val anonymGrönsak = new Grönsak { val vikt = 42 }
\end{REPL}



\Task \emph{Polymorfism i samband med arv.} Polymorfism betyder ''många skepnader''. I samband med arv  innebär det att flera subtyper, till exempel \code{Ko} och \code{Gris}, kan hanteras gemensamt som om de vore instanser av samma supertyp, så som \code{Djur}. Subklasser kan implementera en metod med samma namn på olika sätt. Vilken metod som exekveras bestäms vid körtid beroende på vilken subtyp som instansieras. På så sätt kan djur komma i många skepnader.

\Subtask Implementera funktionen \code{skapaDjur} nedan så att den returnerar antingen en ny Ko eller en ny Gris med lika sannolikhet.

\begin{REPL}
scala> trait Djur { def väsnas: Unit }
scala> class Ko   extends Djur { def väsnas = println("Muuuuuuu") }
scala> class Gris extends Djur { def väsnas = println("Nöffnöff") }
scala> def skapaDjur: Djur = ???
scala> val bondgård = Vector.fill(42)(skapaDjur)
scala> bondgård.foreach(_.väsnas)
\end{REPL}

\Subtask Lägg till ett djur av typen Häst som väsnas på lämpligt sätt och modifiera \code{skapaDjur} så att det skapas kor, grisar och hästar med lika sannolikhet.



\Task \emph{Bastypen \code{Shape} och subtyperna \code{Rectangle} och \code{Circle}.} Du ska nu skapa ett litet bibliotek för geometriska former med oföränderliga objekt implementerade med hjälp av case-klasser. De geometriska formerna har en gemensam bastyp kallad \code{Shape}. Skriv nedan kod i en editor och klistra sedan in den i REPL med kommandot \code{:paste}.
\begin{Code}
case class Point(x: Double, y: Double) {
  def move(dx: Double, dy: Double): Point = Point(x + dx, y + dy)
}

trait Shape {
  def pos: Point
  def move(dx: Double, dy: Double): Shape
}

case class Rectangle(
  pos: Point,
  dx: Double,
  dy: Double
) extends Shape {
  override def move(dx: Double, dy: Double): Rectangle =
    Rectangle(pos.move(dx, dy), this.dx, this.dy)
}

case class Circle(pos: Point, radius: Double) extends Shape {
  override def move(dx: Double, dy: Double): Circle =
    Circle(pos.move(dx, dy), radius)
}
\end{Code}

\Subtask Instansiera några cirklar och rektanglar och gör några relativa förflyttningar av dina instanser genom att anropa \code{move}.

\Subtask Lägg till metoden \code{moveTo} i \code{Point}, \code{Shape}, \code{Rectangle} och \code{Circle} som gör en absolut förflyttning till koordinaterna \code{x} och \code{y}. Klistra in i REPL och testa på några instanser av \code{Rectangle} och \code{Circle}.

\Subtask Lägg till metoden \code{distanceTo(that: Point): Double } i case-klassen \code{Point} som räknar ut avståndet till en annan punkt med hjälp av \code{math.hypot}. Klistra in i REPL och testa på några instanser av \code{Point}.

\Subtask Lägg till en konkret metod \code{distanceTo(that: Shape): Double } i traiten \code{Shape} som räknar ut avståndet till positionen för en annan Shape. Klistra in i REPL och testa på några instanser av \code{Rectangle} och \code{Circle}.







\Task \label{task:fyle} \emph{Inmixning.} Man kan utvidga en klass med multipla traits med nyckelordet \code{with}. På så sätt kan man fördela medlemmar i olika traits och återanvända gemensamma koddelar genom så kallad \textbf{inmixning}, så som nedan exempel visar.

En alternativ fågeltaxonomi, speciellt populär i Skåne, delar in alla fåglar i två specifika kategorier: Kråga respektive Ånka. Krågor kan flyga men inte simma, medan Ånkor kan simma och oftast även flyga. Fågel i generell, kollektiv bemärkelse kallas på gammal skånska för Fyle.%
\footnote{\href{http://www.klangfix.se/ordlista.htm}{www.klangfix.se/ordlista.htm}}
Skriv in nedan kod i en editor och spara den för kommande uppgifter. Klistra in koden i REPL med kommandot \code{:paste}.

\begin{Code}
trait Fyle {
  val läte: String
  def väsnas: Unit = print(läte * 2)
  val ärSimkunnig: Boolean
  val ärFlygkunnig: Boolean
}

trait KanSimma       { val ärSimkunnig = true }
trait KanInteSimma   { val ärSimkunnig = false }
trait KanFlyga       { val ärFlygkunnig = true }
trait KanKanskeFlyga { val ärFlygkunnig = math.random < 0.8 }

class Kråga extends Fyle with KanFlyga with KanInteSimma {
  val läte = "krax"
}

class Ånka extends Fyle with KanSimma with KanKanskeFlyga {
  val läte = "kvack"
  override def väsnas = print(läte * 4)
}
\end{Code}

\Subtask En flitig ornitolog hittar 42 fåglar i en perfekt skog där alla fågelsorter är lika sannolika, representerat av vektorn \code{fyle} nedan. Skriv i REPL ett uttryck som undersöker hur många av dessa som är flygkunniga Ånkor, genom att använda metoderna \code{filter}, \code{isInstanceOf}, \code{ärFlygkunnig} och \code{size}.

\begin{REPL}
scala> val fyle =
         Vector.fill(42)(if (math.random > 0.5) new Kråga else new Ånka)
scala> fyle.foreach(_.väsnas)
scala> val antalFlygånkor: Int = ???
\end{REPL}

\Subtask \label{subtask:fyle:sound} Om alla de fåglar som ornitologen hittade skulle väsnas exakt en gång var, hur många krax och hur många kvack skulle då höras? Använd metoderna \code{filter} och \code{size}, samt predikatet \code{ärSimkunnig} för att beräkna antalet krax respektive kvack.
\begin{REPL}
scala> val antalKrax: Int = ???
scala> val antalKvack: Int = ???
\end{REPL}

\Task \emph{Finala klasser.} Om man vill förhindra att man kan göra \code{extends} på en klass kan man göra den final genom att placera nyckelordet \code{final} före nyckelordet \code{class}.

\Subtask Eftersom klassificeringen av fåglar i uppgiften ovan i antingen Ånkor eller Krågor är fullständig och det inte finns några subtyper till varken Ånkor eller Krågor är det lämpligt att göra dessa finala. Ändra detta i din kod.

\Subtask Testa att ändå försöka göra en subklass \code{Simkråga extends Kråga}. Vad ger kompilatorn för felmeddelande om man försöker utvidga en final klass?


\Task \emph{Accessregler vid arv och nyckelordet \code{protected}.} Om en medlem i en supertyp är privat så kan man inte komma åt den i en subklass. Ibland vill man att subklassen ska kunna komma åt en medlem även om den ska vara otillgänglig i annan kod.

\begin{REPL}
trait Super {
  private val minHemlis = 42
  protected val vårHemlis = 42
}
class Sub extends Super {
  def avslöja = minHemlis
  def kryptisk = vårHemlis * math.Pi
}
\end{REPL}

\Subtask Vad blir felmeddelandet när klassen \code{Sub} försöker komma åt \code{minHemlis}?

\Subtask Deklarera \code{Sub} på nytt, men nu utan den förbjudna metoden \code{avslöja}. Vad blir felmeddelandet om du försöker komma åt \code{vårHemlis} via en instans av klassen \code{Sub}? Prova till exempel med \code{(new Sub).vårHemlis}

\Subtask Fungerar det att anropa metoden \code{kryptisk} på instanser av klassen \code{Sub}?

\Task \emph{Använding av \code{protected}.} Den flitige ornitologen från uppgift \ref{task:fyle} ska ringmärka alla 42 fåglar hen hittat i skogen. När hen ändå håller på bestämmer hen att även försöka ta reda på hur mycket oväsen som skapas av respektive fågelsort. För detta ändamål apterar den flitige ornitologen en linuxdator på allt infångat fyle. Du ska hjälpa ornitologen att skriva programmet.

\Subtask Inför en \code{protected var räknaLäte} i traiten \code{Fyle} och skriv kod på lämpliga ställen för att räkna hur många läten som respektive fågelinstans yttrar.

\Subtask Inför en metod \code{antalLäten} som returnerar antalet krax respektive kvack som en viss fågel yttrat sedan dess skapelse.

\Subtask\Pen Varför inte använda \code{private} i stället for \code{protected}?

\Subtask\Pen Varför är det bra att göra räknar-variabeln oåtkomlig från ''utsidan''?



\Task \emph{Typtester med \code{isInstanceOf} och typkonvertering med \code{asInstanceOf}.} Gör nedan deklarationer.
\begin{REPL}
scala> trait A; trait B extends A; class C extends B; class D extends B
scala> val (c, d) = (new C, new D)
scala> val a: A = c
scala> val b: B = d
\end{REPL}

\Subtask Rita en bild över vilka typer som ärver vilka.

\Subtask Vilket resultat ger dessa typtester? Varför?
\begin{REPL}
scala> c.isInstanceOf[C]
scala> c.isInstanceOf[D]
scala> d.isInstanceOf[B]
scala> c.isInstanceOf[A]
scala> b.isInstanceOf[A]
scala> b.isInstanceOf[D]
scala> a.isInstanceOf[B]
scala> c.isInstanceOf[AnyRef]
scala> c.isInstanceOf[Any]
scala> c.isInstanceOf[AnyVal]
scala> c.isInstanceOf[Object]
scala> 42.isInstanceOf[Object]
scala> 42.isInstanceOf[Any]
\end{REPL}

\Subtask Vilka av dessa typkonverteringar ger felmeddelande? Vilket och varför?
\begin{REPL}
scala> c.asInstanceOf[B]
scala> c.asInstanceOf[A]
scala> d.asInstanceOf[C]
scala> a.asInstanceOf[B]
scala> a.asInstanceOf[C]
scala> a.asInstanceOf[D]
scala> a.asInstanceOf[E]
scala> b.asInstanceOf[A]
\end{REPL}



\Task \emph{Regler för \code{override}, \code{private} och \code{final}.}

\Subtask \label{subtask:overriderules} Undersök överskuggningning av abstrakta, konkreta, privata och finala medlemmar genom att skriva in raderna nedan en i taget i REPL. Vilka rader ger felmeddelande? Varför? Vid felmeddelande: notera hur felmeddelandet lyder och ändra deklarationen av den felande medlemmen så att koden blir kompilerbar (eller om det är enda rimliga lösningen: ta bort den felaktiga medlemmen), innan du provar efterkommande rad.

\begin{REPL}
trait Super1 { def a: Int; def b = 42; private def c = "hemlis" }
class Sub2 extends Super1 { def a = 43; def b = 43; def c = 43 }
class Sub3 extends Super1 { def a = 43; override def b = 43 }
class Sub4 extends Super1 { def a = 43; override def c = "43" }
trait Super5 { final def a: Int; final def b = 42 }
class Sub6 extends Super5 { override def a = 43; def b = 43 }
class Sub7 extends Super5 { def a = 43; override def b = 43 }
class Sub8 extends Super5 { def a = 43; override def c = "43" }
trait Super9 { val a: Int; val b = 42; lazy val c: String = "lazy" }
class Sub10 extends Super9 { override def a = 43; override val b = 43 }
class Sub11 extends Super9 { val a = 43; override lazy val b = 43 }
class Sub12 extends Super9 { val a = 43; override var b = 43 }
class Sub13 extends Super9 { val a = 43; override lazy val c = "still lazy" }
class SubSub extends Sub13 { override val a = 44}
trait Super14 { var a: Int; var b = 42; var c: String }
class Sub15 extends Super14 { def a = 43; override var b = 43; val c = "?" }
\end{REPL}

\Subtask Skapa instanser av klasserna \code{Sub3}, \code{Sub13} och \code{SubSub} från ovan deluppgift och undersök alla medlemmarnas värden för respektive instans. Förklara varför de har dessa värden.

\Subtask Läs igenom reglerna i kapitel  \ref{slideW07:overriderules} om vad som gäller vid arv och överskuggning av medlemmar. Gör några egna undersökningar i REPL som försöker bryta mot någon regel som inte testades i deluppgift \ref{subtask:overriderules}.

\Task \emph{Supertyp med parameter.} En trait kan inte ha någon parameter. Vill man ha en parameter till supertypen måste man använda en klass istället, enligt nedan exempel.

Utbildningsdepartementet vill i sitt system hålla koll på vissa personer och skapar därför en klasshierarki enligt nedan. Skriv in koden i en editor och klipp sedan in den i REPL.
\begin{Code}
class Person(val namn: String)

class Akademiker(
  namn: String,
  val universitet: String) extends Person(namn)

class Student(
  namn: String,
  universitet: String,
  program: String) extends Akademiker(namn, universitet)

class Forskare(
  namn: String,
  universitet: String,
  titel: String) extends Akademiker(namn, universitet)
\end{Code}


\Subtask Deklarera fyra olika \code{val}-variabler med lämpliga namn som refererar till olika instanser av alla olika klasser ovan och ge attributen valfria initialvärden genom olika parametrar till konstruktorerna.

\Subtask Skriv två satser: en som först stoppar in instanserna i en \code{Vector} och en som sedan loopar igenom vektorn och skriv ut alla instansers \code{toString} och \code{namn}.


\Subtask Utbildningsdepartementet vill att det inte ska gå att instansiera objekt av typerna \code{Person} och \code{Akademiker}. Det kan åstadkommas genom att placera nyckelordet \code{abstract} före \code{class}. Uppdatera koden i enlighet med detta. Vilket blir felmeddelande om man försöker instansiera en \code{abstract class}?

\Subtask Utbildningsdeparetementet vill slippa implementera \code{toString} och slippa skriva \code{new} vid instansiering. Gör därför om typerna \code{Student} och \code{Forskare} till case-klasser. \emph{Tips:} För att undkomma ett kompileringsfel (vilket?) behöver du använda \code{override val} på lämpligt ställe.

Skapa instanser av de nya case-klasserna \code{Student} och \code{Forskare} och skriv ut deras \code{toString}. Hur ser utskriften ut?

\Subtask Eftersom \code{Person} och \code{Akademiker} nu är abstrakta, vill utbildningsdepartementet att du gör om dessa typer till traits med abstrakta attribut istället för klasser. Du kan då undvika \code{override val} i klassparametrarna till de konkreta case-klasserna.

Man inför också en case-klass \code{IckeAkademiker} som man tänker använda i olika statistiska medborgarundersökningar.

Dessutom förser man alla personer med ett personnummer representerat som en \code{Int}.

Hur ser utbildningsdepartementets kod ut nu, efter alla ändringar? Skriv ett testprogram som skapar några instanser och skriver ut deras attribut.

\Subtask\Pen I vilka sammanhang är det nödvändigt att använda en \code{trait} respektive en \code{class}?




\Task \emph{Uppräknade värden.} Ett sätt att hålla reda på uppräknade värden, t.ex. färgen på olika kort i en kortlek, är att använda olika heltal som får representera de olika värdena, till exempel så här:\footnote{Om namnkonventioner för konstanter i Scala: läs under rubriken ''Constants, Values, Variable and Methods'' här \href{http://docs.scala-lang.org/style/naming-conventions.html}{docs.scala-lang.org/style/naming-conventions.html}}
\begin{Code}
object Färg {
  val Spader = 1
  val Hjärter = 2
  val Ruter = 3
  val Klöver = 4
}
\end{Code}
Dessa kan sedan användas som parametrar till denna case-klass vid skapande av kortobjekt:
\begin{lstlisting}[language=,keywords={case,class}]
case class Kort(färg: Int, valör: Int)
\end{lstlisting}
Man kan hålla reda på färgen med en variabel av typen \code{Int} och tilldela den en viss färg med ovan konstanter. Och när man skapar ett kort behöver man inte komma ihåg vilket numret är.
\begin{REPL}
scala> val f = Färg.Spader
scala> import Färg._
scala> Kort(Ruter, 7)
\end{REPL}
En annan fördelen med detta är att man lätt kan loopa från 1 till 4 för att gå igenom alla färger.
\begin{REPL}
scala> val kortlek = for (f <- 1 to 4; v <- 1 to 13) yield Kort(f, v)
\end{REPL}
Nackdelen är att kompilatorn vid kompileringstid inte kollar om variablerna av misstag råkar ges något värde utanför det giltiga intervallet, t.ex. 42. Detta får vi själv hålla koll på vid körtid, till exempel med funktionen \code{require} eller \code{if}-satser, etc.

Istället kan man använda case-objekt enligt nedan deluppgifter och få hjälp av kompilatorn att hitta eventuella fel vid kompileringstid.  Ett case-objekt är som ett vanligt singelton-objekt men det får automatiskt en \code{toString} samma som namnet och kan användas i matchningar etc. (mer om match i kapitel \ref{chapter:W08}).

\Subtask Deklarera följande uppräknade värden som singelton objekt med gemensam bastyp i en editor och klistra in i REPL med kommandot \code{:paste}. Med nyckelordet \code{sealed} så ''förseglas'' klassen och inga andra direkta subtyper tillåts förutom de som finns i samma kod-fil eller block. I detta exempel  med kortfärger vet vi ju att det inte finns fler än dessa fyra färger.
\begin{Code}
sealed trait Färg
case object Spader extends Färg
case object Hjärter extends Färg
case object Ruter extends Färg
case object Klöver extends Färg
\end{Code}
Dessa kan sedan användas som parametrar till denna case-klass vid skapande av kortobjekt:
\begin{lstlisting}[language=,keywords={case,class}]
case class Kort(färg: Färg, valör: Int)
\end{lstlisting}
Skapa därefter några exempelinstanser av klassen \code{Kort}. Vad är fördelen med ovan angreppssätt jämfört med att använda heltalskonstanter?

\Subtask Om man vill kunna iterera över alla värden är det bra om de finns i en samling med alla värden. Vi kan lägga en sådan i ett kompanjonsobjekt till bastypen. Uppdatera koden enligt nedan och klistra in på nytt i REPL med kommandot \code{:paste}. Skriv ut alla färgvärden med en \code{for}-sats.

\begin{Code}
sealed trait Färg
object Färg {
  val values = Vector(Spader, Hjärter, Ruter, Klöver)
}
case object Spader extends Färg
case object Hjärter extends Färg
case object Ruter extends Färg
case object Klöver extends Färg
\end{Code}
Skapa en kortlek med 52 olika kort och blanda den sedan med \code{Random.shuffle} enligt nedan. Använd en dubbel \code{for}-sats och \code{yield}.
\begin{REPL}
scala> val kortlek: Vector[Kort] = ???
scala> val blandad = scala.util.Random.shuffle(kortlek)
\end{REPL}

\Subtask Skriv en funktion \code{ def blandadKortlek: Vector[Kort] = ???} som ger en ny blandad kortlek varje gång den anropas med metoden i föregående uppgift.

%%%%%%%%%%%%%%%%%%% FEEEEEELLL \end{Code}



\Subtask Om man även vill ha en heltalsrepresentation med en medlem \code{toInt} för alla värden, kan man ge bastypen en parameter och i stället för en trait (som inte kan ha några parametrar) använda en abstrakt klass.

\begin{Code}
sealed abstract class Färg(final val toInt: Int)
object Färg {
  val values = Vector(Spader, Hjärter, Ruter, Klöver)
}
case object Spader  extends Färg(0)
case object Hjärter extends Färg(1)
case object Ruter   extends Färg(2)
case object Klöver  extends Färg(3)
\end{Code}
Skapa en funktion \code{färgPoäng} som räknar ut summan av heltalsrepresentationen av alla färger hos en vektor med kort, och använd den sedan för att räkna ut \code{färgPoäng} för de första fem korten.
\begin{REPL}
scala> def blandadKortlek: Vector[Kort] = ???
scala> def färgPoäng(xs: Vector[Kort]): Int = ???
scala> färgPoäng(blandadKortlek.take(5))
\end{REPL}


\ExtraTasks %%%%%%%%%%%%%%%%%%%

\Task Det visar sig att vår flitige ornitolog från uppgift \ref{task:fyle} på sidan \pageref{task:fyle} sov på en av föreläsningarna i zoologi när hen var nolla på Natfak, och därför helt missat fylekategorin \code{Pjodd}. Hjälp vår stackars ornitolog så att fylehierarkin nu även omfattar Pjoddar. En Pjodd kan flyga som en Kråga men den \code{ÄrLiten} medan en Kråga \code{ÄrStor}. En Pjodd kvittrar dubbelt så många gånger som en Ånka kvackar. En Pjodd \code{KanKanskeSimma} där simkunnighetssannolikheten är $0.2$. Låt ornitologen ånyo finna 42 slumpmässiga fåglar i skogen och filtrera fram lämpliga arter. Undersök sedan hur dessa väsnas, i likhet med deluppgift \ref{task:fyle}\ref{subtask:fyle:sound}.


\clearpage

\AdvancedTasks %%%%%%%%%%%%%%%%%

\Task Hitta på en egen fördjupningsuppgift inspirerat av denna artikel på Stackoverflow: \url{http://stackoverflow.com/questions/16173477/usages-of-null-nothing-unit-in-scala}

\Task Studera den djupa arvshierarkin i paketet \code{numbers} nedan som modellerar olika sorters tal i matematiken. Du kan även ladda ner koden här: \\
\href{https://github.com/lunduniversity/introprog/blob/master/compendium/examples/numbers.scala}{github.com/lunduniversity/introprog/blob/master/compendium/examples/numbers.scala}
\\ Notera metoden \code{reduce} som reducerar ett tal till sin enklaste form och dess implementation överskuggas på lämpliga ställen med relevant reduktion.

\Subtask Skriv kod som använder de olika konkreta klasserna i \code{package numbers}. Om du kompilerar koden i samma bibliotek som du kör igång REPL är det bara att använda paketet direkt:
\begin{REPL}
$ scalac numbers.scala
$ scala
scala> numbers.  // Tryck Tab
AbstractComplex   AbstractNatural    AbstractReal   Frac    Nat      Polar
AbstractInteger   AbstractRational   Complex        Integ   Number   Real

scala> numbers.Integ(12)
res0: numbers.Integ = Integ(12)

scala> import numbers.Syntax._
import numbers.Syntax._

scala> 42.j
res1: numbers.Complex = Complex(Real(0),Real(42))

scala> 42.42.j
res2: numbers.Complex = Complex(Real(0),Real(42.42))

\end{REPL}

\Subtask Ändra på metoden \code{+} i \code{trait Number} så att den blir abstrakt och implementera den i alla konkreta klasser.

\Subtask Implementera fler räknesätt och bygg vidare på koden så som du finner intressant.

\Subtask Gör så att metoden \code{reduce} i klassen \code{AbstractRational} använder algoritmen Greatest Common Divisor (GCD)\footnote{\href{https://sv.wikipedia.org/wiki/St\%C3\%B6rsta_gemensamma_delare}{https://sv.wikipedia.org/wiki/St\%C3\%B6rsta\_gemensamma\_delare}} så som beskrivs här: \\ \href{http://www.artima.com/pins1ed/functional-objects.html#6.8}{www.artima.com/pins1ed/functional-objects.html\#6.8} \\ så att täljare och nämnare blir så små som möjligt.

\scalainputlisting[numbers=left, basicstyle=\ttfamily\fontsize{9}{11}\selectfont]{examples/numbers.scala}

%!TEX encoding = UTF-8 Unicode

%!TEX root = ../compendium2.tex

\Exercise{\ExeWeekTEN}\label{exe:W10}

\begin{Goals}
\item Kunna skapa och använda \code{match}-uttryck med konstanta värden, garder och mönstermatchning med case-klasser.
\item Kunna skapa och använda case-objekt för matchningar på uppräknade värden.
\item Känna till betydelsen av små och stora begynnelsebokstäver i case-grenar i en matchning, samt förstå hur namn binds till värden in en case-gren.
\item Kunna hantera saknade värden med hjälp av typen \code{Option} och mönstermatchning på \code{Some} och \code{None}.
\item Känna till hur metoden \code{unapply} används vid mönstermatchning.
\item Känna till nyckelordet \code{sealed} och förstå nyttan med förseglade typer.
\item Känna till \jcode{switch}-satser i Java.
\item Känna till \code{null}.
\item Kunna fånga undantag med \code{try}-\code{catch} och \code{scala.util.Try}.
\item Känna till skillnaderna mellan \code{try}-\code{catch} i Scala och java.
\item Kunna implementera \code{equals} med hjälp av en \code{match}-sats, som fungerar för finala klasser utan arv.
\item Känna till relationen mellan \code{hashcode} och \code{equals}.
\item Kunna använda \code{flatMap} tillsammans med \code{Option} och \code{Try}.
\item Kunna skapa partiella funktioner med case-uttryck.
\end{Goals}

\begin{Preparations}
\item \StudyTheory{10}
\end{Preparations}

\BasicTasks %%%%%%%%%%%%%%%%

\Task \label{task:switch} \emph{Hur fungerar en \jcode{switch}-sats i Java (och flera andra språk)?} Det händer ofta att man vill testa om ett värde är ett av många olika alternativ. Då kan man använda en sekvens av många \code{if}-\code{else}, ett för varje alternativ. Men det finns ett annat sätt i Java och många andra språk: man kan använda \jcode{switch} som kollar flera alternativ i en och samma sats, se t.ex. \href{https://en.wikipedia.org/wiki/Switch_statement}{en.wikipedia.org/wiki/Switch\_statement}.

\Subtask Skriv in nedan kod i en kodeditor. Spara med namnet \texttt{Switch.java} och kompilera filen med kommandot \texttt{javac Switch.java}. Kör den med \texttt{java Switch} och ange din favoritgrönsak som argument till programmet. Vad händer? Förklara hur \jcode{switch}-satsen fungerar.

\javainputlisting[numbers=left,basicstyle=\ttfamily\fontsize{11}{12}\selectfont]{examples/Switch.java}

\Subtask \label{subtask:break} Vad händer om du tar bort \jcode{break}-satsen på rad 16?




\Task \label{task:vegomatch} \emph{Matcha på konstanta värden.} I Scala finns ingen \jcode{switch}-sats. I stället har Scala ett \code{match}-uttryck som är mer kraftfullt. Dock saknar Scala nyckelordet \jcode{break} och Scalas \code{match}-uttryck kan inte ''falla igenom'' som skedde i uppgift \ref{task:switch}\ref{subtask:break}.

\Subtask \label{subtask:vegomatch} Skriv nedan program med en kodeditor och spara i filen \texttt{Match.scala}. Kompilera med \texttt{scalac Match.scala}. Kör med \texttt{scala Match} och ge som argument din favoritgrönsak. Vad händer? Förklara hur ett \code{match}-uttryck fungerar.

\scalainputlisting[numbers=left,basicstyle=\ttfamily\fontsize{11}{12}\selectfont]{examples/Match.scala}

\Subtask Vad blir det för felmeddelande om du tar bort case-grenen för defaultvärden och indata väljs så att inga case-grenar matchar? Är det ett exekveringsfel eller ett kompileringsfel?


\Subtask\Pen Beskriv några skillnader i syntax och semantik mellan Javas flervalssats \jcode{switch} och Scalas flervalsuttryck \code{match}.




\Task \emph{Gard i case-grenar.} Med hjälp en gard \Eng{guard} i en case-gren kan man begränsa med ett villkor om grenen ska väljas.

Utgå från koden i uppgift \ref{task:vegomatch}\ref{subtask:vegomatch} och byt ut case-grenen för \code{'g'}-matchning till nedan variant med en gard med nyckelordet \code{if} (notera att det inte behövs parenteser runt villkoret):
\begin{Code}
    case 'g' if math.random > 0.5 => "gurka är gott ibland..."
\end{Code}
Kompilera om och kör programmet upprepade gånger med olika indata tills alla grenar i \code{match}-uttrycket har exekverats. Förklara vad som händer.

\Task \label{task:match-caseclass} \emph{Mönstermatcha på attributen i case-klasser.} Scalas \code{match}-uttryck är extra kraftfulla om de används tillsammans med \code{case}-klasser: då kan attribut extraheras automatiskt och bindas till lokala variabler direkt i case-grenen som nedan exempel visar (notera att \code{v} och \code{rutten} inte behöver deklareras explicit). Detta kallas för \textbf{mönstermatchning}.

\Subtask \label{subtask:autobinding-match} Vad skrivs ut nedan? Varför? Prova att byta namn på \code{v} och \code{rutten}.
\begin{REPL}
scala> case class Gurka(vikt: Int, ärRutten: Boolean)
scala> val g = Gurka(100, true)
scala> g match { case Gurka(v,rutten) => println("G" + v + rutten) }
\end{REPL}

\Subtask Skriv sedan nedan i REPL och tryck TAB två gånger efter punkten. Vad har \code{unapply}-metoden för resultattyp?
\begin{REPL}
scala> Gurka.unapply   // Tryck TAB två gånger
\end{REPL}
\begin{Background}
Case-klasser får av kompilatorn automatiskt ett kompanjonsobjekt \Eng{companion object}, i detta fallet \code{object Gurka}. Det objektet får av kompilatorn automatiskt en \code{unapply}-metod. Det är \code{unapply} som anropas ''under huven'' när case-klassernas attribut extraheras vid mönstermatchning, men detta sker alltså automatiskt och man behöver inte explicit nyttja \code{unapply} om man inte själv vill implementera s.k. extraherare \Eng{extractors}; om du är nyfiken på detta, se fördjupningsuppgift \ref{task:extractor}.
\end{Background}

\Subtask Anropa \code{unapply}-metoden enligt nedan. Vad blir resultatet?
\begin{REPL}
scala> Gurka.unapply(g)
\end{REPL}
Vi ska i senare uppgifter undersöka hur typerna \code{Option} och \code{Some} fungerar och hur man kan ha nytta av dessa i andra sammanhang.

\Subtask Spara programmet nedan i filen \texttt{vegomatch.scala} och kompilera med \code{scalac vegomatch.scala} och kör med \code{scala vegomatch.Main 1000} i terminalen. Förklara hur predikatet \code{ärÄtvärd} fungerar.
\scalainputlisting[numbers=left,basicstyle=\ttfamily\fontsize{11}{12}\selectfont]{examples/vegomatch.scala}



\Task Man kan åstadkomma urskiljningen av de ätbara grönsakerna i uppgift \ref{task:match-caseclass} med polymorfism i stället för \code{match}.

\Subtask Gör en ny variant av ditt program enligt nedan riktlinjer och spara den modifierade koden i filen \texttt{vegopoly.scala} och kompilera och kör.
\begin{itemize}[noitemsep]
\item Ta bort predikatet \code{ärÄtvärd} i objektet \code{Main} och inför i stället en abstrakt metod \code{def ärÄtbar: Boolean} i traiten \code{Grönsak}.
\item Inför konkreta \code{val}-medlemmar i respektive grönsak som definierar ätbarheten.
\item Ändra i huvudprogrammet i enlighet med ovan ändringar så att \code{ärÄtvärd} anropas som en metod på de skördade grönsaksobjekten när de ätvärda ska filtreras ut.
\end{itemize}

\Subtask Lägg till en ny grönsak \code{case class Broccoli} och definiera dess ätbarhet. Ändra i slump-funktionerna så att broccoli blir ovanligare än gurka.

\Subtask\Pen Jämför lösningen med \code{match} i uppgift \ref{task:match-caseclass} och lösningen ovan med polymorfism. Vilka är för- och nackdelarna med respektive lösning? Diskutera två olika situationer på ett hypotetiskt företag som utvecklar mjukvara för jordbrukssektorn: 1) att uppsättningen grönsaker inte ändras särskilt ofta medan definitionerna av ätbarhet ändras väldigt ofta och 2) att uppsättningen grönsaker ändras väldigt ofta men att ätbarhetsdefinitionerna inte ändras särskilt ofta.



\Task \emph{Matcha på case-objekt och nyttan med \code{sealed}.} Skapa nedan kod i en editor, och klistra in i REPL med kommandot \code{:pa}. Notera nyckelordet \code{sealed} som används för att försegla en typ. En \textbf{förseglad typ} måste ha alla sina subtyper i en och samma kodfil.
\begin{Code}
sealed trait Färg
object Färg {
  val values = Vector(Spader, Hjärter, Ruter, Klöver)
}
case object Spader  extends Färg
case object Hjärter extends Färg
case object Ruter   extends Färg
case object Klöver  extends Färg
\end{Code}

\Subtask Skapa en funktion \code{def parafärg(f: Färg): Färg} i en editor, som med hjälp av ett match-uttryck returnerar parallellfärgen till en färg. Parallellfärgen till \code{Hjärter} är \code{Ruter} och vice versa, medan parallellfärgen till \code{Klöver} är \code{Spader} och vice versa. Klistra in funktionen i REPL.
\begin{REPL}
scala> parafärg(Spader)
scala> val xs = Vector.fill(5)(Färg.values((math.random * 4).toInt))
scala> xs.map(parafärg)
\end{REPL}

\Subtask Vi ska nu undersöka vad som händer om man glömmer en av case-grenarna i matchningen i \code{parafärg}? ''Glöm'' alltså avsiktligt en av case-grenarna och klistra in den nya \code{parafärg} med den ofullständiga matchningen. Hur lyder varningen? Kommer varningen vid körtid eller vid kompilering?

\Subtask Anropa \code{parafärg} med den ''glömda'' färgen. Hur lyder felmeddelandet? Är det ett kompileringsfel eller ett körtidsfel?

\Subtask\Pen Förklara vad nyckelordet \code{sealed} innebär och vilken nytta man kan ha av att \textbf{försegla} en supertyp.


\Task \emph{Betydelsen av små och stora begynnelsebokstäver vid matchning.} För att åstadkomma att namn kan bindas till variabler vid matchning utan att de behöver deklareras i förväg (som vi såg i uppgift \ref{task:match-caseclass}\ref{subtask:autobinding-match}) så har identifierare med liten begynnelsebokstav fått speciell betydelse: den tolkas av kompilatorn som att du vill att en variabel  binds till ett värde vid matchningen. En identifierare med stor begynnelsebokstav tolkas däremot som ett konstant värde (t.ex. ett case-objekt eller ett case-klass-mönster).

\Subtask \emph{En case-gren som fångar allt}. En case-gren med en identifierare med liten begynnelsebokstav som saknar gard kommer att matcha allt. Prova nedan i REPL, men försök lista ut i förväg vad som kommer att hända. Vad händer?
\begin{REPL}
scala> val x = "urka"
scala> x match {
         case str if str.startsWith("g") => println("kanske gurka")
         case vadsomhelst => println("ej gurka: " + vadsomhelst)
       }
scala> val g = "gurka"
scala> g match {
         case str if str.startsWith("g") => println("kanske gurka")
         case vadsomhelst => println("ej gurka: " + vadsomhelst)
       }
\end{REPL}

\Subtask \emph{Fallgrop med små begynnelsbokstäver.} Innan du provar nedan i REPL, försök gissa vad som kommer att hända. Vad händer? Hur lyder varningarna och vad innebär de?
\begin{REPL}
scala> val any: Any = "gurka"
scala> case object Gurka
scala> case object tomat
scala> any match {
         case Gurka => println("gurka")
         case tomat => println("tomat")
         case _ => println("allt annat")
       }
\end{REPL}

\Subtask \emph{Använd backticks för att tvinga fram match på konstant värde.} Det finns en utväg om man inte vill att kompilatorn ska skapa en ny lokal variabel: använd specialtecknet \emph{backtick}, som skrivs \`{} och kräver speciella tangentbordstryck.\footnote{Fråga någon om du inte hittar hur man gör backtick \`{} på ditt tangentbord.}  Gör om föregående uppgift men omgärda nu identifieraren \code{tomat} i tomat-case-grenen med backticks, så här: \code{  case `tomat` => ...}



\Task \emph{Använda \code{Option} och matcha på värden som kanske saknas.} Man behöver ofta skriva kod för att hantera värden som eventuellt saknas, t.ex. saknade telefonnummer i en persondatabas. Denna situation är så pass vanlig att många språk har speciellt stöd för saknande värden.

I Java\footnote{Scala har också \code{null} men det behövs bara vid samverkan med Java-kod.} används värdet \code{null} för att indikera att en referens saknar värde. Man får då komma ihåg att testa om värdet saknas varje gång sådana värden ska behandlas, t.ex. med \code+if (ref != null) { ...} else { ... }+. Ett annat vanligt trick är att låta \code{-1} indikera saknade positiva heltal, till exempel saknade index, som får behandlas med \code+if (i != -1) { ...} else { ... }+.

I Scala finns en speciell typ \code{Option} som möjliggör smidig och typsäker hantering av saknade värden. Om ett kanske saknat värde packas in i en \code{Option} \Eng{wrapped in an Option}, finns det i en speciell slags samling som bara kan innehålla \emph{inget} eller \emph{något} värde, och alltså har antingen storleken \code{0} eller \code{1}.

\Subtask Förklara vad som händer nedan.
\begin{REPL}
scala> var kanske: Option[Int] = None
scala> kanske.size
scala> kanske = Some(42)
scala> kanske.size
scala> kanske.isEmpty
scala> kanske.isDefined
scala> def ökaOmFinns(opt: Option[Int]): Option[Int] = opt match {
         case Some(i) => Some(i + 1)
         case None    => None
       }
scala> val annanKanske = ökaOmFinns(kanske)
scala> def öka(i: Int) = i + 1
scala> val merKanske = kanske.map(öka)
\end{REPL}

\Subtask Mönstermatchingen ovan är minst lika knölig som en \code{if}-sats, men tack vare att en \code{Option} är en slags (liten) samling finns det smidigare sätt. Förklara vad som händer nedan.
\begin{REPL}
val meningen = Some(42)
val ejMeningen = Option.empty[Int]
meningen.map(_ + 1)
ejMeningen.map(_ + 1)
ejMeningen.map(_ + 1).orElse(Some("saknas")).foreach(println)
meningen.map(_ + 1).orElse(Some("saknas")).foreach(println)
\end{REPL}

\Subtask \emph{Samlingsmetoder som ger en \code{Option}.} Förklara för varje rad nedan vad som händer. En av raderna ger ett felmeddelande; vilken rad och vilket felmeddelande?
\begin{REPL}
val xs = (42 to 84 by 5).toVector
val e = Vector.empty[Int]
xs.headOption
xs.headOption.get
xs.headOption.getOrElse(0)
xs.headOption.orElse(Some(0))
e.headOption
e.headOption.get
e.headOption.getOrElse(0)
e.headOption.orElse(Some(0))
Vector(xs, e, e, e)
Vector(xs, e, e, e).map(_.lastOption)
Vector(xs, e, e, e).map(_.lastOption).flatten
xs.lift(0)
xs.lift(1000)
e.lift(1000).getOrElse(0)
xs.find(_ > 50)
xs.find(_ < 42)
e.find(_ > 42).foreach(_ => println("HITTAT!"))
\end{REPL}

\Subtask\Pen Vilka är fördelerna med \code{Option} jämfört med \code{null} eller \code{-1} om man i sin kod glömmer hantera saknade värden?

\Task \emph{Kasta undantag.} Om man vill signalera att ett fel eller en onormal situtation uppstått så kan man \textbf{kasta} \Eng{throw} ett \textbf{undantag} \Eng{exception}. Då avbryts programmet direkt med ett felmeddelande, om man inte väljer att \textbf{fånga} \Eng{catch} undantaget.

\Subtask Vad händer nedan?
\begin{REPL}
scala> throw new Exception("PANG!")
scala> java.lang.   // Tryck TAB efter punkten
scala> throw new IllegalArgumentException("fel fel fel")
scala> val carola = try {
         throw new Exception("stormvind!")
         42
       } catch { case e: Throwable => println("Fångad av en " + e); -1 }
\end{REPL}
\Subtask\Pen Nämn ett par undantag som finns i paketet \code{java.lang} som du kan gissa vad de innebär och i vilka situationer de kastas.

\Subtask\Pen Vilken typ har variabeln \code{carola} ovan? Vad hade typen blivit om catch-grenen hade returnerat en sträng i stället?

\Task \label{task:javatry} \emph{Fånga undantantag i Java med en \jcode{try}-\jcode{catch}-sats.} Det finns som vi såg i förra uppgiften inbyggt stöd i JVM för att hantera när program avbryts på oväntade sätt, t.ex. på grund av division med noll eller ej förväntade indata från användaren. Skriv in nedan Java-program i en editor och spara i en fil med namnet \texttt{TryCatch.java} och kompilera med \texttt{javac TryCatch.java} i terminalen.

\javainputlisting[numbers=left,basicstyle=\ttfamily\fontsize{11}{12}\selectfont]{examples/TryCatch.java}

\Subtask Förklara vad som händer när du kör programmet med olika indata:
\begin{REPL}
$ java TryCatch 42
$ java TryCatch 0
$ java TryCatch safe 42
$ java TryCatch safe 0
$ java TryCatch
\end{REPL}

\Subtask Vad händer om du ''glömmer bort'' raden 15 och därmed missar att initialisera input? Hur lyder felmeddelandet? Är det ett körtidsfel eller kompileringsfel?

\Subtask\Pen Beskriv några skillnader och likheter i syntax och semantik mellan \code{try}-\code{catch} i Java respektive Scala.



\Task \emph{Fånga undantantag i Scala med \code{scala.util.Try}.} I paketet \code{scala.util} finns typen \code{Try} med stort T som är som en slags samling som kan innehålla antingen ett ''lyckat'' eller ''misslyckat'' värde. Om beräkningen av värdet lyckades och inga undantag kastas blir värdet inkapslat i en \code{Success}, annars blir undantaget inkapslat i en \code{Failure}. Man kan extrahera värdet, respektive undantaget, med mönstermatchning, men det är oftast smidigare att använda samlingsmetoderna \code{map} och \code{foreach}, i likhet med hur \code{Option} används. Det finns även en smidig metod \code{recover} på objekt av typen \code{Try} där man kan skicka med kod som körs om det uppstår en undantagssituation.

\Subtask Förklara vad som händer nedan.
\begin{REPL}
scala> def pang = throw new Exception("PANG!")
scala> import scala.util.{Try, Success, Failure}
scala> Try{pang}
scala> Try{pang}.recover{case e: Throwable => s"desarmerad bomb: $e"}
scala> Try{"tyst"}.recover{case e: Throwable => s"desarmerad bomb: $e"}
scala> def kanskePang = if (math.random > 0.5) "tyst" else pang
scala> def kanskeOk = Try{ kanskePang}
scala> val xs = Vector.fill(100)(kanskeOk)
scala> xs(13) match {
         case Success(x) => ":)"
         case Failure(e) => ":( " + e
       }
scala> x(13).isSuccess
scala> x(13).isFailure
scala> xs.count(_.isFailure)
scala> xs.find(_.isFailure)
scala> val badOpt = xs.find(_.isFailure)
scala> val goodOpt = xs.find(_.isSuccess)
scala> badOpt
scala> badOpt.get
scala> badOpt.get.get
scala> badOpt.map(_.getOrElse("bomben desarmerad!")).get
scala> goodOpt.map(_.getOrElse("bomben desarmerad!")).get
scala> xs.map(_.getOrElse("bomben desarmerad!")).foreach(println)
scala> xs.map(_.toOption)
scala> xs.map(_.toOption).flatten
scala> xs.map(_.toOption).flatten.size
\end{REPL}


\Subtask Vad har funktionen \code{pang} för returtyp?

\Subtask\Pen Varför får funktionen \code{kanskePang} den härledda returtypen \code{String}?

\Task \emph{Metoden \code{equals}.}  Om man överskuggar den befintliga metoden \code{equals} så kommer metoden \code{==} att fungera annorlunda. Man kan då själv åstadkomma innehållslikhet i stället för referenslikhet. Vi börjar att studera den befintliga equals med referenslikhet.

\Subtask \label{subtask:refequals} Vad händer nedan? Om du trycker TAB \emph{två} gånger efter ett metodnamn får du se metodens signatur. Vilken signatur har metoden \code{equals}?
\begin{REPL}
scala> class Gurka(val vikt: Int, ärÄtbar: Boolean)
scala> val g1 = new Gurka(42, true)
scala> val g2 = g1
scala> val g3 = new Gurka(42, true)
scala> g1 == g2
scala> g1 == g3
scala> g1.equals  // tryck TAB två gånger
\end{REPL}

\Subtask\Pen Rita minnessituationen efter rad 4.

\Subtask \emph{Överskugga metoderna \code{equals} och \code{hashCode}.}

\begin{Background}
Det visar sig förvånande komplicerat att implementera innehållslikhet med metoden \code{equals} så att den ger bra resultat under alla speciella omständigheter. Till exempel måste man även överskugga en metod vid namn \code{haschCode} om man överskuggar \code{equals}, eftersom dessa båda används gemensamt av effektivitetsskäl för att skapa den interna lagringen av objekten i vissa samlingar. Om man missar det kan objekt bli ''osynliga'' i \code{hashCode}-baserade samlingar -- men mer om detta i senare kurser. Om objekten ingår i en öppen arvshierarki blir det också mer komplicerat; det är enklare om man har att göra med finala klasser. Dessutom krävs speciella hänsyn om klassen har en typparameter.
\end{Background}

\noindent Definera klassen nedan i REPL med överskuggade \code{equals} och \code{hashCode}; den ärver inte något och är final.

\begin{Code}
// fungerar fint om klassen är final och inte ärver något
final class Gurka(val vikt: Int, ärÄtbar: Boolean) {
  override def equals(other: Any): Boolean = other match {
    case that: Gurka => this.vikt == that.vikt
    case _ => false
  }
  override def hashCode: Int = (vikt, ärÄtbar).## //förklaras sen
}
\end{Code}
\Subtask Vad händer nu nedan, där \code{Gurka} nu har en överskuggad \code{equals} med innehållslikhet?
\begin{REPL}
scala> val g1 = new Gurka(42, true)
scala> val g2 = g1
scala> val g3 = new Gurka(42, true)
scala> g1 == g2
scala> g1 == g3
\end{REPL}
\Subtask\Pen Hur märker man ovan att den överskuggade \code{equals} medför att \code{==} nu ger innehållslikhet? Jämför med deluppgift \ref{subtask:refequals}.

I uppgift \ref{task:equals:Complex} får du prova på att följa det fullständiga receptet i 8 steg för att överskugga en \code{equals} enligt konstens alla regler. I efterföljande kurs kommer mer träning i att hantera innehållslikhet och hash-koder. I Scala får man ett objekts hash-kod med metoden \code{##}.\footnote{Om du är nyfiken på hash-koder, läs mer här:
\href{https://en.wikipedia.org/wiki/Java_hashCode()}{en.wikipedia.org/wiki/Java\_hashCode()}.}



\clearpage
\ExtraTasks %%%%%%%%%%%%%%%%%%%

\Task \label{task:plynomial} \emph{Polynom}. Med hjälp av koden nedan, kan man göra följande:
\begin{REPL}
scala> :pa polynomial.scala

scala> import polynomial._

scala> Const(1) * x
res0: polynomial.Term = x

scala> (x*5)^2
res1: polynomial.Prod = 25x^2

scala> Poly(x*(-5), y^4, (z^2)*3)
res2: polynomial.Poly = -5x + y^4 + 3z^2

\end{REPL}

\Subtask\Pen Förklara vad som händer ovan genom att studera koden för \code{object polynomial} nedan i filen \code{polynomial.scala}.\footnote{Koden finns även här:\\ \href{https://github.com/lunduniversity/introprog/tree/master/compendium/examples/polynomial}{github.com/lunduniversity/introprog/tree/master/compendium/examples/polynomial}}

\scalainputlisting[numbers=left,basicstyle=\ttfamily\fontsize{10}{12}\selectfont]{examples/polynomial/polynomial.scala}

\Subtask Bygg vidare på \code{object polynomial} och implementera addition mellan olika termer.


\Task\Pen Studera dokumentationen för \code{Option} här och se om du känner igen några av metoderna som också finns på samlingen \code{Vector}:\\ \href{http://www.scala-lang.org/api/current/index.html#scala.Option}{www.scala-lang.org/api/current/index.html\#scala.Option}
\\Förklara hur metoden \code{contains} på en \code{Option} fungerar med hjälp av dokumentationens exempel.



\Task Gör motsvarande program i Scala som visas i uppgift \ref{task:javatry}, men utnyttja att Scalas \code{try}-\code{catch} är ett uttryck. Kompilera och kör och testa så att de ur användarens synvinkel fungerar precis på samma sätt. Notera de viktigaste skillnaderna mellan de båda programmen.




\clearpage

\AdvancedTasks %%%%%%%%%%%%%%%%%

\Task Bygg vidare på \code{object polynomial} i uppgift \ref{task:plynomial} på sidan \pageref{task:plynomial} och implementera metoden \code{def reduce: Poly} i case-klassen \code{Poly} som förenklar polynom om flera \code{Prod}-termer kan adderas.

\Task\Pen Läs om hash-koder här: \href{https://en.wikipedia.org/wiki/Java_hashCode()}{en.wikipedia.org/wiki/Java\_hashCode()} \\
Vad ger metoden \code{##} i scala.Any för resultat? Läs dokumentationen här: \\ \href{http://www.scala-lang.org/api/current/#scala.Any}{www.scala-lang.org/api/current/\#scala.Any}

\Task \emph{Typsäker innehållstest med metoden \code{===}.} Metoderna \code{equals} och \code{==} tillåter jämförelse med vad som helst. Ibland vill man ha en typsäker innehållsjämförelse som bara tillåter jämförelse av objekt av en mer specifik typ och ger kompileringsfel annars. Man brukar då definiera en metod \code{===} som har en parameter \code{that} som har en så specifik typ som önskas. Inför nedan abstrakta metod \code{===} i traiten \code{polynomial.Term} i uppgift \ref{task:plynomial} på sidan \pageref{task:plynomial} och överskugga den sedan i alla subklasser till Term. Testa så att du får kompileringsfel om du försöker jämföra en \code{Term} med något helt annat, t.ex. en \code{String} eller \code{Vector}.
\begin{Code}
  def ===(that: Term): Boolean
\end{Code}


\Task \label{task:equals:Complex} \emph{Överskugga \code{equals} med innehållslikhet även för icke-finala klasser.} Nedan visas delar av klassen \code{Complex} som representerar ett komplext tal med realdel och imaginärdel. I stället för att, som man ofta gör i Scala, använda en case-klass och en \code{equals}-metod som automatiskt ger innehållslikhet, ska du träna på att implementera en egen \code{equals}.
\begin{Code}
class Complex(re: Double, im: Double) {
  def abs: Double = math.hypot(re, im)
  override def toString = s"Complex($re, $im)"
  def canEqual(other: Any): Boolean = ???
  override def hashCode: Int  = ???
  override def equals(other: Any): Boolean = ???
}
case object Complex {
  def apply(re: Double, im: Double): Complex = new Complex(re, im)
}
\end{Code}
Följ detta \textbf{recept}\footnote{Detta recept bygger på \url{http://www.artima.com/pins1ed/object-equality.html}} i 8 steg för att överskugga \code{equals} med innehållslikhet som fungerar även för klasser som inte är \code{final}:

\begin{enumerate}[leftmargin=*]
\item Inför denna metod: \code{ def canEqual(other: Any): Boolean}\\Observera att typen på parametern ska vara \code{Any}. Om detta görs i en subklass till en klass som redan implementerat \code{canEqual}, behövs även \code{override}.

\item Metoden \code{canEqual} ska ge \code{true} om \code{other} är av samma typ som \code{this}, alltså till exempel: \\
\code{def canEqual(other: Any): Boolean = other.isInstanceOf[Complex]}

\item Inför metoden \code{equals} och var noga med att parametern har typen \code{Any}: \\ \code{override def equals(other: Any): Boolean}

\item Implementera metoden \code{equals} med ett match-uttryck som börjar så här:
\code|other match { ... } |

\item Match-uttrycket ska ha två grenar. Den första grenen ska ha ett typat mönster för den klass som ska jämföras: \\ \code{  case that: Complex =>}

\item Om du implementerar \code{equals} i den klass som inför \code{canEqual}, börja uttrycket med: \\ \code{(that canEqual this) &&} \\
och skapa därefter en fortsättning som baseras på innehållet i klassen, till exempel: \code{this.re == that.re && this.im == that.im} \\
Om du överskuggar en \textit{annan} equals än den standard-equals som finns i \code{AnyRef}, vill du förmodligen börja det logiska uttrycket med att anropa superklassens equals-metod:
 \code{super.equals(that) && } men du får fundera noga på vad likhet av underklasser egentligen ska innebära i ditt speciella fall.

\item Den andra grenen i matchningen ska vara:
\code{case _ => false}

\item Överskugga \code{hashCode}, till exempel genom att göra en tupel av innehållet i klassen och anropa metoden \code{##} på tupeln så får du i en bra hashcode: \\
\code{override def hashCode: Int  = (re, im).## }

\end{enumerate}


\Task Bygg vidare på exemplet nedan och överskugga equals vid arv, genom att följa receptet i uppgift \ref{task:equals:Complex}.
\begin{Code}
trait Number {
  override def equals(other: Any): Boolean = ???
}
class Complex(re: Double, im: Double) extends Number {
  override def equals(other: Any): Boolean = ???
}
class Rational(numerator: Int, denominator: Int) extends Number {
  override def equals(other: Any): Boolean = ???
}
\end{Code}


\Task Läs om olika speciella matchningar här: \\
\href{http://www.artima.com/pins1ed/case-classes-and-pattern-matching.html}{www.artima.com/pins1ed/case-classes-and-pattern-matching.html}

\Subtask Prova variabelbinding med \texttt{@} i en matchning i REPL.

\Subtask Prova sekvensmönster med \texttt{\_} och \texttt{\_*} i en matching i REPL.

\Task \label{task:extractor} Läs mer om extraktorer här: \\ \href{http://www.artima.com/pins1ed/extractors.html}{www.artima.com/pins1ed/extractors.html} \\
Skapa ditt eget extraktor-objekt för http-addresser som i t.ex.: \\
\texttt{http://my.host.domain/path/to/this} \\ extraherar \texttt{my.host.domain} och \texttt{path/to/this} med metoden \code{unapply} och testa i en matchning.

%\Task \TODO \emph{flatten och flatMap med Option och Try}
%Ska detta vara ordinarie uppgift eller fördjupning???


%\Task \TODO \emph{partiella funktioner och metoderna collect och collectFirst på samlingar}
%Ska detta vara ordinarie uppgift eller fördjupning???

\Task En rejäl utmaning: Implementera polynomdivision på lämpligt sätt genom att bygga vidare på  \code{object polynomial} i  uppgift \ref{task:plynomial} på sidan \pageref{task:plynomial}.  \\ Läs mer om polynomdivision här: \href{https://sv.wikipedia.org/wiki/Polynomdivision}{sv.wikipedia.org/wiki/Polynomdivision}


%!TEX encoding = UTF-8 Unicode
%!TEX root = ../exercises.tex

\ifPreSolution



\Exercise{\ExeWeekELEVEN}\label{exe:W11}

\begin{Goals}
\item Kunna förklara och beskriva viktiga skillnader mellan Scala och Java.
\item Kunna översätta enkla algoritmer, klasser och singeltonobjekt från Scala till Java och vice versa.
\item Känna till vad en case-klass innehåller i termer av en Javaklass.
%\item Förstå hur autoboxing fungerar.
\item Kunna använda Javatyperna \code{List}, \code{ArrayList}, \code{Set}, \code{HashSet} och översätta till deras Scalamotsvarigheter med \code{JavaConverters}.
\item Kunna förklara hur autoboxning fungerar i Java, samt beskriva fördelar och fallgropar.
\end{Goals}

\begin{Preparations}
\item \StudyTheory{11}
\end{Preparations}

\BasicTasks %%%%%%%%%%%%%%%%

\else



\ExerciseSolution{\ExeWeekELEVEN}

\BasicTasks %%%%%%%%%%%

\fi





\WHAT{Översätta metoder från Java till Scala.}

\QUESTBEGIN

\Task  \what~  I denna uppgift ska du översätta en Java-klass som används som en modul\footnote{\href{https://en.wikipedia.org/wiki/Modular_programming}{en.wikipedia.org/wiki/Modular\_programming}} och bara innehåller statiska metoder och inget förändringsbart tillstånd som kan ändras utifrån. (I nästa uppgift ska du sedan översätta klasser med förändringsbara  tillstånd.)

Vi börjar med att göra översättningen från Java till Scala rad för rad och du ska behålla så mycket som möjligt av syntax och semantik så att Scala-koden blir så Java-lik som möjligt. I efterföljande deluppgift ska du sedan omforma översättningen så att Scala-koden blir mer idiomatisk\footnote{\href{https://sv.wikipedia.org/wiki/Idiom_\%28programmering\%29}{sv.wikipedia.org/wiki/Idiom\_\%28programmering\%29}}.

\Subtask Studera klassen \code{Hangman} nedan. Du ska översätta den från Java till Scala enlig de riktlinjer och tips som följer efter koden. Läs igenom alla riktlinjer och tips innan du börjar.

\javainputlisting[numbers=left]{examples/scalajava/Hangman.java}

\noindent\emph{Riktlinjer och tips för översättningen:}

\begin{enumerate}[noitemsep]

\item Skriv Scala-koden med en texteditor i en fil som heter \texttt{hangman1.scala} och kompilera med \code{scalac hangman1.scala} i terminalen; använd alltså \emph{inte} en IDE, så som Eclipse eller IntelliJ, utan en ''vanlig'' texteditor, t.ex. \code{atom}.

\item Översätt i denna första deluppgift rad för rad så likt den ursprungliga Java-kodens utseende (syntax)  som möjligt, med så få ändringar som möjligt. Du ska alltså ha kvar dessa Scalaovanligheter, även om det inte alls blir som man brukar skriva i Scala:
\begin{enumerate}[nolistsep, noitemsep]
\item långa indrag, \item onödiga semikolon, \item onödiga \code{()}, \item onödiga \code|{}|, \item onödiga \code{System.out}, och \item onödiga \code{return}.
\end{enumerate}

\item Försök också i denna deluppgift göra så att betydelsen (semantiken) så långt som möjligt motsvarar den i Java, t.ex. genom att använda \code{var} överallt, även där man i Scala normalt använder \code{val}.

\item En Javaklass med bara statiska medlemmar motsvarar ett singeltonobjekt i Scala, alltså en \code{object}-deklaration innehållande ''vanliga'' medlemmar.

\item För att tydliggöra att du använder Javas \code{Set} och \code{HashSet} i din Scala-kod, använd följande import-satser i \code{hangman1.scala}, som därmed döper om dina importerade namn och gör så att de inte krockar med Scalas inbyggda \code{Set}. Denna form av import går inte att göra i Java.
\begin{Code}
import java.util.{Set => JSet};
import java.util.{HashSet => JHashSet};
\end{Code}

\item Javas \code{i++} fungerar inte i Scala; man får istället skriva \code{i += 1} eller mindre vanliga \code{i = i + 1}.

\item Typparametrar i Java skrivs inom \code{<>} medan Scalas syntax för typparametrar använder \code{[]}.

\item Till skillnad från Java så har Scalas metoddeklarationer ett tilldelningstecken \code{=} efter returtypen, före kroppen.

\item Du kan ladda ner Java-koden till \code{Hangman}-klassen nedan från kursens repo%
\footnote{\href{https://github.com/lunduniversity/introprog/blob/master/compendium/examples/scalajava/Hangman.java}{github.com/lunduniversity/introprog/blob/master/compendium/examples/scalajava/Hangman.java}}. I samma bibliotek ligger även lösningarna till översättningen i Scala, men kolla \emph{inte} på dessa förrän du gjort klart översättningarna och fått dem att kompilera och köra felfritt! Tanken är att du ska träna på att läsa felmeddelande från kompilatorn och åtgärda dem i en upprepad kompilera-testa-rätta-cykel.

\end{enumerate}







\Subtask Skapa en ny fil \code{hangman2.scala} som till att börja med innehåller en kopia av din direkt-översatta Java-kod från föregående deluppgift. Omforma koden så att den blir mer som man brukar skriva i Scala, alltså mer Scala-idiomatisk. Försök förenkla och förkorta så mycket du kan utan att göra avkall på läsbarheten.

\emph{Tips och riktlinjer:}

\begin{enumerate}[nolistsep, noitemsep]

\item Kalla Scala-objektet för \code{hangman}. När man använder ett Scalaobjekt som en modul (alltså en samling funktioner i en gemensam, avgränsad namnrymd) har man gärna liten begynnelsebokstav, i likhet med konventionen för paketnamn. Ett paket är ju också en slags modul och med en namngivningskonvention som är gemensam kan man senare, utan att behöva ändra koden som använder modulen, ändra från ett singelobjekt till ett paket och vice versa om man så önskar.

\item Gör alla metoder publikt tillgängliga och låt även strängvektorn \code{hangman} vara publikt tillgänglig. Deklarera \code{hangman} som en \code{val} och konstruera den med \code{Vector}. Eftersom \code{Vector} är oföränderlig och man inte kan ärva från singelobjekt och \code{hangman} är deklarerad med \code{val} finns inga speciella risker med att göra den konstanta vektorn publik om  vi inte har något emot att annan kod kan läsa (och eventuellt göra sig beroende av) vår hänggubbetext.

\item I metoden \code{renderHangman}, använd \code{take} och \code{mkString}.

\item I metoden \code{hideSecret}, använd \code{map} i stället för en \code{for}-sats.

\item Det går att ersätta metoden \code{findAll} med det kärnfulla uttrycket \\ \code{(secret forall found)} där \code{secret} är en sträng och \code{found} är en mängd av tecken (undersök gärna i REPL hur detta fungerar). Skippa därför den metoden helt och använd det kortare uttrycket direkt.

\item I metoden \code{makeGuess}, i stället för \code{Scanner}, använd \code{scala.io.StdIn.readLine}.

\item Om du vill träna på att använda rekursion i stället för imperativa loopar: Gör metoden \code{makeGuess} rekursiv i stället för att använda \code{do}-\code{while}.

\item I metoden \code{download}, i stället för \code{java.net.URL} och \code{java.util.ArrayList}, använd \code{scala.io.Source.fromURL(address, coding).getLines.toVector} och gör en lokal import av \code{scala.io.Source.fromURL} överst i det block där den används. Det går inte att ha lokala \code{import}-satser i Java.

\item Låt metoden \code{download} returnera en \code{Option[String]} som i fallet att nedladdningen misslyckas returnerar \code{None}.

\item I metoden \code{download}, i stället för \code{try}-\code{catch} använd \code{scala.util.Try} och dess smidiga metoder \code {recover} och \code{toOption}.

\item Om du vill träna på att använda rekursion i stället för imperativa loopar: Använd, i stället för \code{while}-satsen i metoden \code{play}, en lokal rekursiv funktion med denna signatur:
\begin{Code}
  def loop(found: Set[Char], bad: Int): (Int, Boolean)
\end{Code}
Funktionen \code{loop} returnerar en 2-tupel med antalet felgissningar och \code{true} om man hittat alla bokstäver eller \code{false} om man blev hängd.

\end{enumerate}





\SOLUTION


\TaskSolved \what
     %%%TODO number  1 %%%starts with: \emph{Översätta algoritmer och %%%

\SubtaskSolved  \scalainputlisting[numbers=left,basicstyle=\ttfamily\fontsize{10.3}{12}\selectfont]{examples/scalajava/hangman1.scala}

\SubtaskSolved  \scalainputlisting[numbers=left,basicstyle=\ttfamily\fontsize{11.2}{13}\selectfont]{examples/scalajava/hangman2.scala}



\QUESTEND






\WHAT{Översätta mellan klasser i Scala och klasser i Java.}

\QUESTBEGIN

\Task  \what~
Klassen \code{Point} nedan är en modell av en punkt som kan sparas på begäran i en lista. Listan är privat för kompanjonsobjektet och kan skrivas ut med en metod \code{showSaved}. I koden används en \code{ArrayBuffer}, men i framtiden vill man, vid behov, kunna ändra från \code{ArrayBuffer} till en annan sekvenssamlingsimplementation, t.ex. \code{ListBuffer}, som uppfyller egenskaperna hos supertypen \code{Buffer}, men har andra prestandaegenskaper för olika operationer. Därför är attributet \code{saved} i kompanjonsobjektet deklarerat med den mer generella typen.

\scalainputlisting[numbers=left]{examples/scalajava/Point.scala}

\Subtask Översätt klassen \code{Point} ovan från Scala till Java. Vi ska i nästa deluppgift kompilera både Scala-programmet ovan och ditt motsvarande Java-program i terminalen och testa i REPL att klasserna har motsvarande funktionalitet.

\emph{Tips och riktlinjer:}
\begin{enumerate}[nolistsep, noitemsep]
\item För att namnen inte ska krocka i våra kommande tester, kalla Javatypen för \code{JPoint}.
\item  I stället för Scalas \code{ArrayBuffer} och \code{Buffer}, använd Javas \code{ArrayList} och \code{List} som båda ligger i paketet \code{java.util}.
\item Undersök dokumentationen för \code{java.util.List} för att hitta en motsvarighet till \code{prepend} för att lägga till i början av listan.
\item I stället för default-argumentet i Scalas primärkonstruktor, använd en extra Java-konstruktor.
\item Det finns inga singelobjekt och inga kompanjonsobjekt i Java; istället kan man använda statiska klassmedlemmar. Placera kompanjonsobjektets medlemmars motsvarigheter \emph{inuti} Java-klassen och gör dem till \jcode{static}-medlemmar.
\item Kod i klasskroppen i Scalaklassen, så som if-satsen på rad 4, placeras i lämplig konstruktor i Javaklassen.
\item Utskrifter med \code{print} och \code{println} behöver i Java föregås av \code{System.out}.
\item Det finns inget nyckelord \code{override} i Java, men en s.k. annotering som ger samma kompilatorhjälp. Den skrivs med ett snabel-a och stor begynnelsebokstav, så här: \jcode{ @Override }  före metoddeklarationen.
\item I Java används konventionen att börja getter-metoder med ordet \code{get}, t.ex. \code{getX()}.
\item Det finns ingen motsvarighet till \code{mkString} för \code{List} så du behöver själv gå igenom listan och hämta elementreferenser för utskrift med en \jcode{for}-loop. Notera att efter sista elementet ska radbrytning göras i utskriften och att inget komma ska skrivas ut efter sista elementet.
\item I Java behövs en ny \jcode{import}-deklaration om man vill importera ännu en typ från samma paket. Man kan även i Java använda asterisk \code{*}, (motsvarande \code{_} i Scala), för att importera allt i ett paket, men då får man med alla möjliga namn och det vill man kanske inte.
\item Metoder i Java slutar med \code{()} om de saknar parametrar.
\item Alla satser i Java slutar med lättglömda semikolon. (Efter att man i skrivit mycket Javakod och växlar till Scalakod är det svårt att vänja sig av med att skriva semikolon...)
\end{enumerate}


\Subtask Starta REPL i samma bibliotek som du kompilerat kodfilerna. Testa så att klasserna \code{Point} och \code{JPoint} beter sig på samma vis enligt nedan. Skriv även testkod i REPL för att avläsa de attributvärden som har getters och undersök att allt funkar som det ska.
\begin{REPLnonum}
$ scalac Point.scala
$ javac JPoint.java
$ scala
scala> val (p, jp) = (new Point, new JPoint)
scala> p.distanceTo(new Point(3, 4))
scala> Point.showSaved
scala> jp.distanceTo(new JPoint(3, 4))
scala> JPoint.showSaved
scala> for (i <- 1 to 10) { new Point(i, i, true) }
scala> Point.showSaved
scala> for (i <- 1 to 10) { new JPoint(i, i, true) }
scala> JPoint.showSaved
\end{REPLnonum}


\Subtask Översätt nedan Javaklass \code{JPerson} till en \code{case class Person} i Scala med  motsvarande funktionalitet.


\javainputlisting[numbers=left]{examples/scalajava/JPerson.java}


\Subtask\Pen Undersök i REPL vilken funktionalitet i Scala-case-klassen \code{Person} som \emph{inte} är implementerad i Java-klassen \code{JPerson} ovan. Skriv upp namnen på några av case-klassens extra metoder samt deras signatur genom att för en \code{Person}-instans, och för kompanjonsobjektet \code{Person}, trycka på TAB-tangenten. Prova några av de extra metoderna i REPL och förklara vad de gör.

\begin{REPL}
scala> val p = Person("Björn", 49)
scala> p.      // tryck TAB en gång
scala> Person. // tryck TAB en gång
scala> p.copy  // tryck TAB en gång
scala> p.copy()
scala> p.copy(age = p.age + 1)
scala> Person.unapply(p)
\end{REPL}


\SOLUTION


\TaskSolved \what
     %%%TODO number  2 %%%starts with: \emph{Översätta mellan klasser %%%

\SubtaskSolved   \javainputlisting[numbers=left]{examples/scalajava/JPoint.java}

\SubtaskSolved   -

\SubtaskSolved   \begin{Code}
case class Person(name: String, age: Int = 0)
\end{Code}

\SubtaskSolved  p.*TAB* - copy, producArity, ProductIterator, productElement, productPrefix

Person.*TAB* - apply, curried, tupled, unapply

\begin{REPLnonum}
scala> p.copy
   def copy(name: String,age: Int): Person

scala> p.copy()
res0: Person = Person(Björn,49)

scala> p.copy(age = p.age + 1)
res1: Person = Person(Björn,50)

scala> Person.unapply(p)
res2: Option[(String, Int)] = Some((Björn,49))
\end{REPLnonum}



\QUESTEND






\WHAT{Auto(un)boxing.}

\QUESTBEGIN

\Task  \what~  I JVM måste typparametern för generiska klasser vara av referenstyp. I Scala löser kompilatorn detta åt oss så att vi ändå kan ha t.ex. \code{Int} som argument till en typparameter i Scala, medan man i Java \emph{inte} direkt kan ha den primitiva typen \jcode{int} som typparameter till t.ex. \code{ArrayList}.

I Java och i den underliggande plattformen JVM används s.k. wrapper-klasser för att lösa detta, t.ex. genom wrapper-klassen \code{Integer} som boxar den primitiva typen \jcode{int}. Java-kompilatorn har stöd för att automatiskt packa in värden av primitiv typ i sådana wrapper-klasser för att skapa referenstyper och kan även automatiskt packa upp dem.

\Subtask Studera hur Scala-kompilatorn låter oss arbeta med en \code{Cell[Int]} även om det underliggande JVM:ens körtidstyp \Eng{runtime type} är en wrapper-klass. Man kan se JVM-körtidstypen med metoderna \code{getClass} och \code{getTypeName} enligt nedan.
\begin{REPL}
scala> class Cell[T](var value: T){
         val typeName: String = value.getClass.getTypeName
         override def toString = "Cell[" + typeName + "](" + value + ")"
       }
scala> val c = new Cell[Int](42)
scala> c.value.getClass.getTypeName
\end{REPL}


\Subtask Vad är körtidstypen för \code{c.value} ovan? Förklara hur det kan komma sig trots att vi deklarerade med typargumentet \code{Int}?

\Subtask Studera dokumentationen för \code{java.lang.Integer}\footnote{\href{https://docs.oracle.com/javase/8/docs/api/java/lang/Integer.html}{docs.oracle.com/javase/8/docs/api/java/lang/Integer.html}} och testa i REPL några av \emph{klassmetoderna} (de som är \jcode{static} och därmed kan anropas med punktnotation direkt på klassens namn utan \code{new}) och några av \emph{instansmetoderna} (de som inte är \jcode{static}).
\begin{REPL}
scala> Integer.  //tryck TAB
scala> Integer.
scala> Integer.toBinaryString(42)
scala> Integer.valueOf(42)
scala> val i = new Integer(42)
scala> i.  // tryck TAB
scala> i.toString
scala> i.compareTo  // tryck TAB 2 gånger
scala> i.compareTo(Integer.valueOf(42))
scala> i.compareTo(42)  // varför fungerar detta?
\end{REPL}

\Subtask\Pen Enligt dokumentationen\footnote{\href{https://docs.oracle.com/javase/8/docs/api/java/lang/Integer.html\#compareTo-java.lang.Integer-}{docs.oracle.com/javase/8/docs/api/java/lang/Integer.html\#compareTo-java.lang.Integer-}} tar instansmetoden \code{compareTo} i klassen \code{Integer} en \code{Integer} som parameter. Hur kan det då komma sig att sista raden ovan fungerar med en \code{Int}?

\Subtask Studera nedan Java-program och beskriv vad som kommer att skrivas ut \emph{innan} du kompilerar och testkör.

\javainputlisting[numbers=left]{examples/scalajava/Autoboxing.java}

\Subtask Ändra i programmet ovan så att autoboxing och autounboxing utnyttjas på alla ställen där så är möjligt. Utnyttja även att \code{toString}-metoden på \code{Integer} ger samma stränrepresentation som \jcode{int} vid utskrift. Fixa också så att du undviker \emph{fallgropen} att i Java jämföra med referenslikhet i stället för att använda \code{equals}. Testa så att allt fungerar som det borde efter dina ändringar.


\Subtask\Pen Antag att du råkar skriva \jcode{xs.add(0, pos)} på rad 14 i ditt program från föregående uppgift. Förklara hur autoboxingen stjälper dig i en \emph{fallgrop} då.

\Subtask\Pen Med ledning av de båda tidigare deluppgifterna: sammanfatta de två nämnda fallgropar med autoboxing i Java i två generella punkter, så att du har nytta av att memorera dem inför din framtida Javakodning.


\SOLUTION


\TaskSolved \what
     %%%TODO number  3 %%%starts with: \emph{Auto(un)boxing.} I JVM må%%%

\SubtaskSolved   -

\SubtaskSolved   Cell har typen java.lang.Integer. När man hämtar ut värdet med \code{c.value} hämtas den primitiva typ \code{int} ut.

\SubtaskSolved   Med hjälp av autoboxing förvandlas 42 till typen \code{Integer} och kan därför jämföras med en annan \code{Integer}.

\SubtaskSolved   i.compareTo(42) fungerar på grund av autoboxing. Då JVM packar in den primitiva typ int i en Integer-objekt automatiskt.

\SubtaskSolved
\begin{REPLnonum}
0 10 20 30 40 50 60 ... 390 400 410

[0]: 0
[42]: 0
NOT EQUAL
\end{REPLnonum}

\SubtaskSolved   \javainputlisting[numbers=left]{examples/scalajava/Autoboxing2.java}

\SubtaskSolved   42 kommer läggas längst fram i listan istället för längst bak, då autounboxing kommer göra Integer(0) till 0 och tvärtom med variablen \code{pos}.

\SubtaskSolved   Om man ska undersöka om två int-variabler är lika ska man använda ==, men om variablerna är av typen Integer måste man använda \code{equals}.

JVM kommer inte varna om man vänder på \code{Integer} och \code{int}, som i \code{xs.add(0, pos)}.



\QUESTEND






\WHAT{JavaConverters.}

\QUESTBEGIN

\Task  \what~  Med \code{import scala.collection.JavaConverters._} får man i sina Scalaprogram tillgång till de smidiga metoderna \code{asJava} och \code{asScala} som översätter mellan motsvarande samlingar i resp språks standardbibliotek. Kör nedan i REPL och gör efterföljande deluppgifter.

\begin{REPL}
scala> val sv = Vector(1,2,3)
scala> val ss = Set('a','b','c')
scala> val sm = Map("gurka" -> 42, "tomat" -> 0)
scala> val ja = new java.util.ArrayList[Int]
scala> ja.add(42)
scala> val js = new java.util.HashSet[Char]
scala> js.add('a')
scala> import scala.collection.JavaConverters._
\end{REPL}

\Subtask Till vilka typer konverteras Scalasamlingarna
\code{Vector[Int]}, \code{Set[Char]} och \\ \code{Map[String, Int]} om du anropar metoden \code{asJava} på dessa?

\Subtask Till vilka typer konverteras Javasamlingarna \code{ArrayList[Int]} och \code{HashSet[Char]}  om du anropar metoden \code{asScala} på dessa? Blir det föränderliga eller oföränderliga motsvarigheter?

\Subtask Vad får resultatet för typ om du kör \code{toSet} på en samling av typen \code{mutable.Set}?

\Subtask Undersök hur du kan efter att du gjort \code{sm.asJava.asScala} anropa ytterligare en metod för att få tillbaka en oföränderlig \code{immutable.Map}.

\Subtask Läs mer i dokumentationen om JavaConverters\footnote{\href{http://docs.scala-lang.org/overviews/collections/conversions-between-java-and-scala-collections.html}{docs.scala-lang.org/overviews/collections/conversions-between-java-and-scala-collections.html}}
och prova några fler konverteringar.



\SOLUTION


\TaskSolved \what
     %%%TODO number  4 %%%starts with: \emph{JavaConverters.} Med \cod%%%

\SubtaskSolved

Vector[Int] -> java.util.List[Int]

Set[Char] -> java.util.Set[Char]

Map[String, Int] -> java.util.Map[String, Int]

\SubtaskSolved

ArrayList[Int] -> scala.collection.mutable.Buffer[Int]

HashSet[Char] -> scala.collection.mutable.Set[Char]

Båda blir föränderliga motsvarigheter. Det visas genom att de till hör \code{scaka.collection.mutable} och både \code{ArrayList} och \code{HashSet} är förändrliga i Java.

\SubtaskSolved   \code{scala.collection.immutable.Set}

\SubtaskSolved   \code{sm.asJava.asScala} ger typen \code{scala.collection.mutable.Map[String,Int]}

\code{sm.asJava.asScala.toMap} ger typen \code{scala.collection.immutable.Map[String,Int]}

\SubtaskSolved   -

\QUESTEND




\ExtraTasks %%%%%%%%%%%%%%%%%%%


\WHAT{Översätta från Java till Scala.}

\QUESTBEGIN

\Task  \what~ Översätt nedan kod från Java till Scala. Skriv koden i en fil som heter \texttt{showInt.scala} och kalla Scala-objektet med \code{main}-metoden för \code{showInt}. Läs tipsen som följer efter koden innan du börjar.

\javainputlisting[numbers=left]{examples/scalajava/JShowInt.java}

\emph{Tips:}
\begin{itemize}[nolistsep, noitemsep]
\item En Javaklass med bara statiska medlemmar motsvaras av ett singeltonobjekt i Scala, alltså en \code{object}-deklaration. Scala har därför inte nyckelordet \jcode{static}.
\item Typen \jcode{Object} i Java motsvaras av Scalas \code{Any}.
\item Du kan använda Scalas möjlighet med default-argument (som saknas i Java) för att bara definiera en enda \code{show}-metod med en tom sträng som default \code{msg}-argument.
\item I Scala har objekt av typen \code{Char} en metod \code{def *(n: Int): String} som skapar en sträng med tecknet repeterat \code{n} gånger. Men du kan ju välja att ändå implementera metoden \code{repeatChar} med \code{StringBuilder} som nedan om du vill träna på att översätta en \code{for}-loop från Java till Scala.
\item I stället för \code{Scanner.nextLine} kan du använda \code{scala.io.StdIn.readLine} som tar en prompt som parameter, men du kan också använda \code{Scanner} i Scala om du vill träna på det.
\item I Java \emph{måste} man använda nyckelordet \jcode{return} om metoden inte är en \jcode{void}-metod, medan man i Scala faktiskt \emph{får} använda \code{return} även om man brukar undvika det och i stället utnyttja att satser i Scala också är uttryck.
\end{itemize}
Kompilera din Scala-kod och kör i terminalen och testa så att allt funkar. Vill du även kompilera Java-koden så finns den i kursens repo i filen\\ \texttt{compendium/examples/scalajava/JShowInt.java}


\SOLUTION


\TaskSolved \what


\begin{Code}[numbers=left]
object showInt {
  def show(obj: Any, msg: String = ""): Unit = println(msg + obj)

  def repeatChar(ch: Char, n: Int): String = ch.toString * n

  def showInt(i: Int): Unit = {
    val leading = Integer.numberOfLeadingZeros(i)
    val binaryString = repeatChar('0', leading) + i.toBinaryString
    show(i,               "Heltal : ")
    show(i.asInstanceOf[Char],         "Tecken : ")
    show(binaryString,    "Binärt : ")
    show(i.toHexString,   "Hex    : ")
    show(i.toOctalString, "Oktal  : ")
  }


  import scala.io.StdIn.readLine
  import scala.util.{Try,Success,Failure}

  def loop: Unit =
    Try { readLine("Heltal annars pang: ").toInt } match {
      case Failure(e) => show(e); show("PANG!")
      case Success(i) => showInt(i); loop
    }

  def main(args: Array[String]): Unit =
    if(args.length > 0) args.foreach(i => showInt(i.toInt))
    else loop
}
\end{Code}



\QUESTEND






\WHAT{Innehållslikhet och referenslikhet i Java.}

\QUESTBEGIN

\Task  \what~ Studera och prova denna fallgrop med innehållslikhet: \href{https://github.com/bjornregnell/lth-eda016-2015/blob/master/lectures/examples/eclipse-ws/lecture-examples/src/week10/generics/TestPitfall3.java}{TestPitfall3.java}







\SOLUTION


\TaskSolved \what
     %%%TODO number  6 %%%starts with: \TODO Fallgrop med Point som in%%%



\QUESTEND




\AdvancedTasks %%%%%%%%%%%%%%%%%


\WHAT{Implementera innehållslikhet i Java.}

\QUESTBEGIN

\Task  \what~\Pen Studera fallgropar för hur man skriver en \code{equals}-metod i Java här:
\href{http://www.artima.com/lejava/articles/equality.html}{www.artima.com/lejava/articles/equality.html} och jämför med  det fullständiga receptet för hur man skriver en välfungerande \code{equals} och \code{hashcode} i Scala här: \href{http://www.artima.com/pins1ed/object-equality.html}{www.artima.com/pins1ed/object-equality.html}

\Subtask Vilka skillnader och likheter finns vid överskuggning av equals i Java respektive Scala, som ska ge en fungerande innehållstest för en hierarki med bastyper och subtyper?

\Subtask Vilka fallgropar är gemensamma för Java och Scala?\SOLUTION


\TaskSolved \what
     %%%TODO number  7 %%%starts with: \TODO \emph{Gränssnitt i Scala %%%



\QUESTEND

\input{modules/w12-sorting-exercise.tex}
%!TEX encoding = UTF-8 Unicode
%!TEX root = ../exercises.tex

\ifPreSolution

\Exercise{\ExeWeekTHIRTEEN}\label{exe:W13}
\begin{Goals}
\item Kunna skriva tentamenslika program med papper, penna och snabbreferens som enda hjälpmedel.
\item Förbereda projektredovisningen.
\item Kunna skapa dokumentation med \code{scaladoc} och \code{javadoc}.
\item Kunna skapa jar-filer.
\end{Goals}

% \begin{Preparations}
% \item \StudyTheory{13}
% \end{Preparations}

\else

\ExerciseSolution{\ExeWeekTHIRTEEN}

\fi


\subsection{Förberedelse inför examination}




\WHAT{Gör en extenta.} %%%%%%%%%%%%%%%%%%%%%%%%%%%%%%%%%%%%%%%%%%%%%%%%%%%%%%%%

\QUESTBEGIN

\Task \what~\TODO

\SOLUTION

\TaskSolved \what~\TODO

\QUESTEND




\WHAT{Förbered din projektredovisning.} %%%%%%%%%%%%%%%%%%%%%%%%%%%%%%%%%%%%%%%

\QUESTBEGIN

\Task \what~\TODO

\SOLUTION

\TaskSolved \what~\TODO

\QUESTEND



\WHAT{Skapa dokumentation.} %%%%%%%%%%%%%%%%%%%%%%%%%%%%%%%%%%%%%%%%%%%%%%%%%%%

\QUESTBEGIN

\Task  \what~

\Subtask \TODO kör nedan kommando i terminalen:

\begin{REPL}
> scaladoc paket.scala
> ls
> firefox index.html   # eller öppna index.html i valfri webbläsare
\end{REPL}

Vad händer?

\Subtask Lägg till några fler metoder i något av objekten i filen \code{paket.scala} och lägg även till några dokumentationskommentarer. Kompilera om och kör. Generera om dokumentationen.

\begin{verbatim}
//... ändra i filen paket.scala

/** min paketdokumentationskommentar p2 */
package p2 {
  /** min paketdokumentationskommentar p21 */
  package p21 {
    /** ett hälsningsobjekt */
    object hello {
      /** en hälsningsmetod i p2.p21 */
      def hello = println("Hej paket p2.p21!")

      /** en metod som skriver ut tiden */
      def date = println(new java.util.Date)
    }
  }
}

\end{verbatim}

\begin{REPL}
> gedit paket.scala
> scalac paket.scala
> jar cvf mittpaket.jar gurka
> scala -cp mittpaket.jar
scala> gurka.tomat.banan.p2.p21.hello.date
scala> :q
> scaladoc paket.scala
> firefox index.html
\end{REPL}

\SOLUTION


\TaskSolved \what

\SubtaskSolved  -

\SubtaskSolved  -

\QUESTEND



\WHAT{Repetera övningar och laborationer.} %%%%%%%%%%%%%%%%%%%%%%%%%%%%%%%%%%%%

\QUESTBEGIN

\Task \what~\TODO

\SOLUTION

\TaskSolved \what~\TODO

\QUESTEND

%!TEX encoding = UTF-8 Unicode
%!TEX root = ../exercises.tex

\ifPreSolution

\Exercise{\ExeWeekFOURTEEN}\label{exe:W14}

\begin{Goals}
\item Känna till vad en tråd är och kunna förklara begreppet jämlöpande exekvering.
\item Känna till vad metoderna \code{run} och \code{start} gör i klassen \code{Thread}.
\item Kunna skapa och starta en tråd med överskuggad \code{run}-metod.
\item Kunna skapa ett enkelt program som från två trådar tävlar om att uppdatera en variabel och förklara varför beteendet kan bli oförutsägbart.
\item Kunna använda en \code{Future} för att köra igång flera parallella beräkningar.
\item Kunna registrera en callback på en \code{Future} med metoden \code{onComplete}.
%\item Känna till att webbsidor beskrivs av HTML-kod och kunna skapa en minimal webbsida.
%\item Kunna ladda ner en webbsida med \code{scala.io.Source.fromURL}.
\end{Goals}

% \begin{Preparations}
% \item \StudyTheory{14}
% \end{Preparations}

\else

\ExerciseSolution{\ExeWeekFOURTEEN}

\fi


\subsection{Frivilliga extrauppgifter}



\WHAT{Trådar.}

\QUESTBEGIN

\Task  \what~   Klassen \code{java.lang.Thread} används för att skapa  \textbf{trådar} med jämlöpande exekvering \Eng{concurrent execution}. På så sätt kan man få olika koddelar att köra samtidigt.

Klassen \code{Thread} definierar en tom \code{run}-metod. Vill man att tråden ska göra något vettigt får man överskugga \code{run} med det man vill ska göras.

En tråd körs igång med metoden \code{start} och då anropas automatiskt \code{run}-metoden och tråden exekverar koden i \code{run} jämlöpande med övriga trådar. Om man anropar \code{run} direkt blir det \emph{inte} jämlöpande exekvering.

\Subtask Skapa en tråd som gör något som tar lite tid och kör med \code{run} resp. \code{start}.
\begin{REPL}
def zzz = { print("zzzzzz"); Thread.sleep(5000); println(" VAKEN!")}
zzz
val t2 = new Thread{ override def run = zzz }
t2.run
t2.run; println("Gomorron!")
t2.start; println("Gomorron!")
t2.start
\end{REPL}

\Subtask Vad händer om man anropar \code{start} mer än en gång på samma tråd?

\Subtask Skapa två trådar med överskuggade \code{run}-metoder och kör igång dem samtidigt enligt nedan. Vilken ordning skrivs hälsningarna ut efter rad 3 resp. rad 4 nedan? Förklara vad som händer.
\begin{REPL}
val g = new Thread{ override def run = for (i <- 1 to 100) print("Gurka ") }
val t = new Thread{ override def run = for (i <- 1 to 100) print("Tomat ") }
g.run; t.run
g.start; t.start
\end{REPL}

\Subtask Använd \code{Thread.sleep} enligt nedan. Är beteendet helt förutsägbart (deterministiskt)? Förklara vad som händer. Du kan (om du kör Linux) avbryta REPL med Ctrl+C%
\footnote{\href{http://stackoverflow.com/questions/6248884/can-i-stop-the-execution-of-an-infinite-loop-in-scala-repl}{stackoverflow.com/questions/6248884/can-i-stop-the-execution-of-an-infinite-loop-in-scala-repl}}.
\begin{REPL}
def ibland(block: => Unit) = new Thread {
  override def run = while(true) { block; Thread.sleep(600) }
}.start
ibland(print("zzz ")); ibland(print("snark ")); ibland(println("hej!"))
\end{REPL}


\SOLUTION


\TaskSolved \what
     %%%TODO number  1 %%%starts with: \emph{Trådar.}  %%%

\SubtaskSolved   -

\SubtaskSolved  \code {java.lang.IllegalThreadStateException}. Det går inte att starta en tråd mer än en gång. Tråden kan därför inte startas om när den redan har exekverats.

\SubtaskSolved   När \code {start} anropas exekveras koden i \code{run} parallellt. Därför skrivs \code{Gurka} och \code{Tomat} ut omlöpande. Om istället \code{run} anropas direkt blir det inte jämnlöpande exekvering och \code{Gurka} skrivs ut 100 gånger, sedan skrivs \code{Tomat} ut 100 gånger.

\SubtaskSolved   \code{Thread.sleep} pausar inte tråden i exakt den tiden som angets. Alltså kommer det skrivas ut \code{zzz snark hej!} i de flesta fall, men det är inte garanterat.



\QUESTEND






\WHAT{Jämlöpande variabeluppdatering.}

\QUESTBEGIN

\Task \label{task:racecondition} \what~   Skriv klasserna \code{Bank} och \code{Kund} i en editor och klistra sedan in koden i REPL.

\begin{Code}
class Bank {
  private var saldo = 0;
  def visaSaldo: Unit = println("saldo: " + saldo)
  def sättIn: Unit = { saldo += 1 }
  def taUt: Unit   = { saldo -= 1 }
}

class Kund(bank: Bank) {
  def slösaSpara = {bank.taUt; Thread.sleep(1); bank.sättIn}
}
\end{Code}

\Subtask Använd funktionen \code{ibland} från föregående uppgift och kör nedan rader i REPL. Resultatet av jämlöpande variabeluppdatering blir här heltokigt och leder till mycket upprörda bankkunder och -ägare. Förklara vad som händer.

\begin{REPL}
val bank = new Bank
bank.visaSaldo
bank.sättIn
bank.visaSaldo
bank.taUt
bank.visaSaldo

val bamse = new Kund(bank)
val skutt = new Kund(bank)

bamse.slösaSpara
skutt.slösaSpara
bank.visaSaldo

def ofta(block: => Unit) = new Thread {
  override def run = while(true) { block; Thread.sleep(1) }
}.start

ofta(bamse.slösaSpara); ofta(skutt.slösaSpara)

ibland(bank.visaSaldo)
\end{REPL}


\SOLUTION


\TaskSolved \what
     %%%TODO number  2 %%%starts with: \emph{Jämlöpande variabeluppdat%%%

\SubtaskSolved  I \code{slösaSpara} hämtas saldot, ändras och placeras tillbaka i minnet -  fördröjs -  upprepas. Om \code{bamse} blir klar med att ladda, ändra och lagra innan skutt gör detsamma med den muterbara variablen hade det inte varit perfekt. Problemet ligger i  när en tråd laddar och innan den kan lagra det förändrade värdet laddar den andra tråden samma värde. Bara en av dessa trådar vinner racet och får lagra sitt ändrade tal. \code{skutt} och \code{bamse} blir alltså upprörda för att inte alla dess uttag och insättningar registreras.


\QUESTEND






\WHAT{Trådsäkra \code{AtomicInteger}.}

\QUESTBEGIN

\Task  \what~  Det finns stöd i JVM för att åstadkomma uppdateringar som inte kan avbrytas av andra trådar under pågånde minnesskrivning. En operation som inte kan avbrytas kallas \textbf{atomär} \Eng{atomic}. Studera dokumentationen för \code{AtomicInteger}\footnote{\href{https://docs.oracle.com/javase/8/docs/api/java/util/concurrent/atomic/AtomicInteger.html}{docs.oracle.com/javase/8/docs/api/java/util/concurrent/atomic/AtomicInteger.html}} och prova nedan kod. Förklara vad som händer.

Använd funktionerna \code{ofta} och \code{ibland} från tidigare uppgifter.
\begin{Code}
class SäkerBank {
  import java.util.concurrent.atomic.AtomicInteger
  private var saldo = new AtomicInteger
  def visaSaldo: Unit = println(s"saldo: ${saldo.get}")
  def sättIn: Unit = { saldo.incrementAndGet }
  def taUt: Unit   = { saldo.decrementAndGet }
}

class SäkerKund(bank: SäkerBank) {
  def slösaSpara = {bank.taUt; Thread.sleep(1); bank.sättIn}
}
\end{Code}
\begin{REPL}
val säkerBank = new SäkerBank
val farmor = new SäkerKund(säkerBank)
val vargen = new SäkerKund(säkerBank)

ofta(farmor.slösaSpara); ofta(vargen.slösaSpara)

ibland(säkerBank.visaSaldo)
\end{REPL}





\SOLUTION


\TaskSolved \what
     %%%TODO number  3 %%%starts with: \emph{Jämlöpande exekvering med%%%

Nu är \code{farmor} garanterad att kunna ladda saldot, ta ut pengar/ändra och lagra innan \code{vargen} kan överskriva resultatet. I \code{slösaSpara} pausas tråden i en millisekund så \code{vargen} kan fortfarande ta ut pengar innan \code{farmor} hinner sätta in pengar igen. Dock kommer alla uttag och insättningar registreras eftersom operationerna är atomära.


\QUESTEND






\WHAT{Jämlöpande exekvering med \code{scala.concurrent.Future}.}

\QUESTBEGIN

\Task \label{task:future} \what~   Att skapa och hålla reda på trådar kan bli ganska omständligt och knepigt att få rätt på.
Med hjälp av \code{scala.concurrent.Future} kan man på ett enklare sätta skapa jämlöpande exekvering.

\begin{Background}
Med en \code{Future} skapas jämlöpande exekvering som ''under huven'' använder ett ramverk som heter Akka\footnote{\url{http://akka.io/}}, skrivet i Scala och Java. Akka erbjuder automatisk  multitrådning med s.k. trådpooler och möjliggör avancerad parallellprogrammering på en hög  abstraktionsnivå, där man själv slipper skapa instanser av klassen \code{Thread}. I stället kan man helt enkelt placera sin kod inramad med \code|Future{ "körs parallellt" }| efter att man importerat det som behövs.
\end{Background}

\Subtask För att skapa jämlöpande exekvering med \code{Future} behöver man först göra import enligt nedan; då skapas ett exekveringssammanhang med trådpooler redo för användning. Starta om REPL och studera felmeddelandet efter rad 1 nedan. Importera därefter enligt nedan. Vad har \code{f} för typ?
\begin{REPL}
scala> concurrent.Future { Thread.sleep(1000); println("En sekund senare!") }
scala> import scala.concurrent._
scala> import ExecutionContext.Implicits.global
scala> val f = Future { Thread.sleep(1000); println("En sekund senare!") }
\end{REPL}

\Subtask Skapa en procedur \code{printLater} enligt nedan som skriver ut argumentet efter slumpmässig tid. Förklara vad som händer nedan.
\begin{REPL}
scala> def printLater(a: Any): Unit =
         Future { Thread.sleep((math.random * 10000).toInt); print(a + " ") }
scala> (1 to 42).foreach(i => printLater(i)); println("alla är igång!")
\end{REPL}

\Subtask Skapa enligt nedan en \code{Future} som räknar ut hur många siffror det är i ett väldigt stort tal. Med \code{onComplete} kan man ange vad som ska göras när den tunga beräkningen är färdig; detta kallas att ''registrera en callback''. Vilken returtyp har \code{big}? Hur många siffror har det stora talet? Vad har \code{r} för typ? Justera argumentet till \code{big} om du inte orkar vänta på resultatet...

\begin{REPL}
scala> BigInt(10).pow(100)
scala> BigInt(10).pow(100).toString.size
scala> def big(n: Int) = Future { BigInt(n).pow(n).toString.size }
scala> big(1234567).onComplete{r => println(r + " siffror") }
\end{REPL}

\Subtask Den stora vinsten med \code{Future} är att man kan köra vidare under tiden, varför anropet av \code{Future} kallas \textbf{icke-blockerande} \Eng{non-blocking}. Det händer ibland att man ändå vill blockera exekveringen i väntan på ett resultat. Man kan då använda objektet \code{scala.concurrent.Await} och dess metod \code{result} enligt nedan. Använd \code{big} från föregående uppgift och gör en blockerande väntan på resultatet enligt nedan. Vad händer? Vad händer om du väntar för kort tid?

\begin{REPL}
scala> import scala.concurrent.duration._
scala> Await.result(big(1234567), 20.seconds)
\end{REPL}



\SOLUTION


\TaskSolved \what
     %%%TODO number  4 %%%starts with: TODO  %%%%%%%%%%%%%%%%%%%\Advan%%%

\SubtaskSolved  error: Cannot find an implicit ExecutionContext. Future behöver en ExecutionContext för att kunna köras. \code{f} är av typen Future[Unit].

\SubtaskSolved  Funktionen \code{printLater} har en Future, vilket innebär att när både \code{printLater} och \code{println} anropas i foreach-loopen exekveras de jämnlöpande. Eftersom det tar längre tid att starta upp en Future för datorn är \code{println} snabbare och skriver ut att alla är igång först. Sedan skrivs siffrorna från 1 - 42 ut med oregelbundna mellanrum eftersom tråden pausas olika länge.

\SubtaskSolved  \code{big} är en Future[Int]. Det stora talet har 7 520 383 siffror. \code{r} är av typen Try[Int] (se dokumentationen för Future om du är osäker)

\SubtaskSolved  Eftersom exekveringen blockas tills den har fått ett resultat går det inte att fortsätta skriva i REPL medan uträkningen pågår. Väntar man för kort tid får man ett TimeOutException och uträkningen avbryts.


\QUESTEND






\WHAT{Använda \code{Future} för att göra flera saker samtidigt.}

\QUESTBEGIN

\Task  \what~
I denna uppgift ska du ladda ner webbsidor parallellt med hjälp av \code{Future}, så att en nedladdning kan avslutas under tiden en annan dröjer.

\Subtask Koden för en minimal webbsida ser ut som nedan. Du kan beskåda sidan här: \url{http://fileadmin.cs.lth.se/pgk/mini.html} eller skriva in nedan kod i en fil som heter något som slutar på \texttt{.html} och öppna filen i din webbläsare.

\begin{verbatim}
<!DOCTYPE html>
<html>
<body>
HELLO WORLD!
</body>
</html>
\end{verbatim}

\Subtask För att simulera slöa webbservrar kan man ladda ner en sida via sajten \texttt{http://deelay.me/}. Ladda ner ovan sida med 2 sekunders fördröjning:\\
\url{http://deelay.me/2000/http://fileadmin.cs.lth.se/pgk/mini.html}

\Subtask Man kan ladda ner webbsidor med \code{scala.io.Source}. Vad händer nedan? Försök, med ledning av hur \code{delay} beräknas, uppskatta hur lång tid du måste vänta i medeltal, i bästa fall, respektive värsta fall, innan du kan se första webbsidan i vektorn \code{laddningar} nedan?

\begin{REPL}
scala> def ladda(url: String) = scala.io.Source.fromURL(url).getLines.toVector
scala> def slöladda(url: String) = {
         val delay = (math.random * 1000 + 2000).toInt
         val delaySite = s"http://deelay.me/$delay/"
         ladda(delaySite+url)
      }
scala> ladda("http://fileadmin.cs.lth.se/pgk/mini.html")
scala> def seg = slöladda("http://fileadmin.cs.lth.se/pgk/mini.html")
scala> val laddningar = Vector.fill(10)(seg)
scala> laddningar(0)
\end{REPL}

\Subtask Innan vi kan köra igång en \code{Future} så måste vi, som visats i uppgift \ref{task:future} importera den underliggande exekveringsmiljön som är redo att parallelisera ditt program i trådar utan att du själv måste skapa dem. Vad händer nedan?
\begin{REPL}
scala> import scala.concurrent._
scala> import ExecutionContext.Implicits.global
scala> val f = Future{ seg }
scala> f   // kolla om den är klar annars prova igen senare
scala> f
\end{REPL}

\Subtask Ladda indata utan att blockera \Eng{non-blocking input}. Förklara vad som händer nedan.
\begin{REPL}
scala> val nonblock = Future{ Vector.fill(10)(seg) }
scala> nonblock   // kolla igen senare om ej klar
scala> nonblock
\end{REPL}

\Subtask Ladda indata separat i olika parallella trådar. Förklara vad som händer nedan. Kör uttrycket på rad 3 nedan upprepade gånger i snabb följd efter varandra med pil-upp+Enter i REPL.
\begin{REPL}
scala> val para = Vector.fill(10)(Future{ seg })
scala> para
scala> para.map(_.isCompleted)
scala> para.map(_.isCompleted) // studera hur de blir färdiga en efter en
scala> para(0)
\end{REPL}

\Subtask Registrera en callback med metoden \code{onComplete}. Förklara vad som händer nedan.

\begin{REPL}
scala> val action = Vector.fill(10)(Future{ seg })
scala> action(0).onComplete(xs => println(s"ready:$xs"))
scala> // vänta tills laddning på plats 0 är klar
\end{REPL}

\Subtask Registrera en callback för felhantering i händelse av undantag med metoden \code{onFailure}. Förklara vad som händer nedan.
\begin{REPL}
scala> def lycka  = { Thread.sleep(3000); println(":)") }
scala> def olycka = { Thread.sleep(3000); 42 / 0; lycka }
scala> Future{ lycka  }.onFailure{ case e => println(s":( $e") }
scala> Future{ olycka }.onFailure{ case e => println(s":( $e") }
\end{REPL}



\SOLUTION


\TaskSolved \what
     %%%TODO number  5 %%%starts with: Sök upp och studera dokumentati%%%

\SubtaskSolved  -

\SubtaskSolved  -

\SubtaskSolved  Varje sida fördröjs med mellan 2 upp till 3 sekunder (2000-3000 millisekunder). Så i medeltal tar det 2.5 sekunder för varje sida att laddas. Vektorn måste fyllas innan exekveringen kan fortsätta. Därför laddas alla 10 stycken sidor in innan man kan se första websidan. Det tar därför i medeltal 2.5 x 10 = 25 sekunder.

\SubtaskSolved  \code{f} ger en Vektor fylld med strängar där varje element ges av en rad på hemsidan. Då \code{f} körs i bakgrunden kan programmet fortlöpa medan innehållet räknas ut. Du kan därför skriva \code{f} i REPL:n men det är inte säkert att proccessen är klar och det slutgilltiga resultatet visas.

\SubtaskSolved  Samma som ovan, förutom att det blir en vektor där varje element är i sig en vektor med strängar.

\SubtaskSolved  Laddar in datan parallelt så nedladdingen sker samtidigt, men det går olika snabbt pga metoden seg.

\SubtaskSolved  Eftersom datan laddas i parallella trådar utan blockering blir de inte klara i ordning, utan i den ordningen tråden körs klart. Till slut blir alla klara och resultatet visar en vektor med \code{true} värden.

\SubtaskSolved  Metoden \code{lycka} är väldefinerad och kastar därför inga undantag. Den skriver alltid ut \code{:)}. Metoden \code{olycka} är inte väldefinerad då division med 0 ger \code{java.lang.ArithmeticException}. Detta fångas upp vid callbacken och det skrivs ut \code{:(} samt det specifierade undantaget.

\ExtraTasks %%%%%%%%%%%%


\QUESTEND






\WHAT{}

\QUESTBEGIN

\Task  \what~ Räkna ut stora primtal parallellt genom att använda nedan funktioner. Implementera \code{isPrime} enligt pseudokod från den engelska wikipediasidan om primtalstest\footnote{\href{https://en.wikipedia.org/wiki/Primality_test}{en.wikipedia.org/wiki/Primality\_test}} med den s.k. ''naiva algoritmen''.  Räkna ut 10 st slumpvisa primtal med 16 siffror vardera. Gör beräkningarna parallellt med hjälp av \code{Future}.

\begin{Code}
def isPrime(n: BigInt): Boolean = ???

def nextPrime(start: BigInt): BigInt = {
  var i = start
  while (!isPrime(i)) { i += 1 }
  i
}

def randomBigInt(nDigits: Int): BigInt = {
   def rndChar = ('0' + (math.random * 10).toInt).toChar
   val str = Array.fill(nDigits)(rndChar).mkString
   BigInt(str)
}
\end{Code}

\SOLUTION


\TaskSolved \what
  %%%TODO number  6 %%%

\begin{Code}
def isPrime(n: BigInt): Boolean = n match {
  case _ if (n <= 1) => false
  case _ if (n <= 3) => true
  case _ if n % 2 == 0 || n % 3 == 0 => false
  case _ =>
    var i = BigInt(5)
    while (i * i < n) {
      if (n % i == 0 || n % (i + 2) == 0) false
      i += 6
    }
    true
}

import scala.concurrent._
import ExecutionContext.Implicits.global

val primes = Vector.fill(10)(Future{nextPrime(randomBigInt(16))})
primes.foreach(_.onSuccess{case i => println(i)})
\end{Code}


\QUESTEND






\WHAT{Svara på teorifrågor.}

\QUESTBEGIN

\Task  \what~\Pen

\Subtask Vad är en tråd?

\Subtask Hur skapar man en tråd med klassen \code{Thread}?

\Subtask Hur startar man en tråd?

\Subtask Vilka problem kan man råka ut för om man uppdaterar samma resurs i flera olika trådar?

\Subtask Vad innbär det att kod är \emph{trådsäker}?

\Subtask Nämn några fördelar med att använda Future jämfört med att använda trådar direkt.


\SOLUTION


\TaskSolved \what
 %%%TODO number  7 %%%

\SubtaskSolved  Stackoverflow ger följande förklaring:

A thread is an independent set of values for the processor registers (for a single core). Since this includes the Instruction Pointer (aka Program Counter), it controls what executes in what order. It also includes the Stack Pointer, which had better point to a unique area of memory for each thread or else they will interfere with each other.

\SubtaskSolved

\begin{Code}
val thread = new Thread(new Runnable{
	def run(){println(''Det här är en tråd'')}
})
\end{Code}

\SubtaskSolved  \code{thread.start}

\SubtaskSolved  Det kan bli kapplöpning(race conditions) om vilken tråds resurser blir sparade. Vilket leder till att de andra trådarnas ändringar blir ignorerade.

\SubtaskSolved  Trådsäkerhet innebär att flera trådar kan köras parallellt utan felaktigheter i resultatet. Exempelvis får man vara väldigt försiktig om man vill ha en muterbar variabel som alla trådar ska ändra samtidigt.

\SubtaskSolved  Till exempel slipper man skapa instanser av klassen Thread eftersom man kan placera koden i en Future istället. Den löser även mycket under huven för kodaren.


\QUESTEND






\WHAT{Klasser med atomär uppdatering.}

\QUESTBEGIN

\Task  \what~ Läs om och testa klasserna AtomicBoolean, AtomicDouble och AtomicReference för atomär uppdatering i paketet \\ \code{java.util.concurrent.atomic}.

Använd några av dessa tillsammans med \code{scala.concurrent.Future}.


\SOLUTION

\TaskSolved --

\QUESTEND





\WHAT{Skapa din egen multitrådade webbserver.}

\QUESTBEGIN

\Task  \what~

\Subtask Skriv in\footnote{Eller ladda ner här: \href{https://github.com/lunduniversity/introprog/blob/master/compendium/examples/simple-web-server/webserver.scala}{github.com/lunduniversity/introprog/blob/master/compendium/examples/simple-web-server/webserver.scala}} nedan kod i en editor och spara i en fil med namn \texttt{webserver.scala} och kompilera och kör med \texttt{scala webserver.start} och beskriv vad som händer när du med din webbläsare surfar till adressen: \\ \url{http://localhost:8089/abbasillen}

\scalainputlisting[numbers=left,basicstyle=\ttfamily\fontsize{11}{12}\selectfont]{examples/simple-web-server/webserver.scala}

\Subtask Du ska nu skapa en webbserver som gör något lite mer intressant. Den ska svara med det 13:e Fibonacci-talet\footnote{\href{https://sv.wikipedia.org/wiki/Fibonaccital}{https://sv.wikipedia.org/wiki/Fibonaccital}} om du surfar till \url{http://localhost:8089/fib/13}.
Spara din webbserver från föregående deluppgift under det nya namnet \texttt{fibserver.scala} och använd koden nedan och lägg till och ändra så att din server kan svara med Fibonaccital. Vi börjar med att räkna ut Fibonaccital i funktionen \code{compute.fib} nedan på ett onödigt processorkrävande sätt med exponentiell tidskomplexitet så att webbservern verkligen får jobba, för att i senare deluppgifter implementera \code{compute.fib} med linjär tidskomplexitet och därmed undvika onödig planetuppvärmning.
\begin{CodeSmall}
  object compute {
    def fib(n: BigInt): BigInt = {
      if (n < 0) 0 else
      if (n == 1 || n == 2) 1
      else fib(n - 1) + fib(n -2)
    }
  }

  def fibResponse(num: String) = Try { num.toInt } match {
    case Success(n) => html.page(s"fib($n) == " + compute.fib(n))
    case Failure(e) => html.page(s"FEL $e: skriv heltal, inte $num")
  }

  def errorResponse(uri:String) = html.page("FATTAR NOLL: " + uri)

  def handleRequest(cmd: String, uri: String, socket: Socket): Unit = {
    val os = socket.getOutputStream
    val parts = uri.split('/').drop(1) // skip initial slash
    val response: String = (parts.head, parts.tail) match {
      case (head, Array(num)) => fibResponse(num)
      case _                  => errorResponse(uri)
    }
    os.write(html.header(response.size).getBytes("UTF-8"))
    os.write(response.getBytes("UTF-8"))
    os.close
    socket.close
  }
\end{CodeSmall}
Kör i terminalen med \texttt{scala fibserver.start} och beskriv vad som händer i din webbläsare när du surfar till servern.


%%%\textbf{KOD TILL FACIT:}
%%%\scalainputlisting[numbers=left,basicstyle=\ttfamily\fontsize{11}{12}\selectfont]{examples/simple-web-server/fibserver.scala}


\Subtask Surfa efter flera stora Fibonacci-tal samtidigt i olika flikar i din browser. Hur märks det att servern bara kör i en enda tråd?

\Subtask Gör din server multitrådad med hjälp av den nya server-loopen nedan.

\begin{CodeSmall}
import scala.concurrent._
import ExecutionContext.Implicits.global

  def serverLoop(server: ServerSocket): Unit = {
    println(s"http://localhost:${server.getLocalPort}/hej")
		while (true) {
  		Try {
  		  var socket = server.accept  // blocks thread until connect
	  	  val scan = new Scanner(socket.getInputStream, "UTF-8")
		    val (cmd, uri) = (scan.next, scan.next)
			  println(s"Request: $cmd $uri")
		    Future { handleRequest(cmd, uri, socket) }.onFailure {
		      case e => println(s"Reqest failed: $e")
		    }
		  }.recover{ case e: Throwable => s"Connection failed: $e" }
		}
  }
\end{CodeSmall}

\Subtask Surfa efter flera stora Fibonacci-tal samtidigt i olika flikar i din browser. Hur märks det att servern är multitrådad?


\Subtask Det är onödigt att räkna ut samma Fibonacci-tal flera gånger. Med hjälp av en cache i form av en föränderlig \code{Map} kan du spara undan redan uträknade värden. Det funkar dock inte med en vanlig \code{scala.collection.mutable.Map} i vår multitrådade webbserver, eftersom den inte är \textbf{trådsäker} \Eng{thread-safe}. Med trådosäkra föränderliga datastrukturer blir det samma besvärliga beteende som i uppgift \ref{task:racecondition}.

Du ska i stället använda \code{java.util.concurrent.ConcurrentHashMap}. Sök upp  dokumentationen för \code{ConcurrentHashMap} och försök förstå koden nedan. Hur fungerar metoderna \code{containsKey}, \code{put} och \code{get}?
\begin{Code}
object compute {
  import java.util.concurrent.ConcurrentHashMap
  val memcache = new ConcurrentHashMap[BigInt, BigInt]

  def fib(n: BigInt): BigInt =
    if (memcache.containsKey(n)) {
      println("CACHE HIT!!! no need to compute: " + n)
      memcache.get(n)
    } else {
      println("cache miss :( must compute fib:  " + n)
      val f = fastFib(n)
      memcache.put(n, f)
      f
    }

  private def fastFib(n: BigInt): BigInt = {
    if (n < 0) 0 else
    if (n == 1 || n == 2) 1
    else fib(n - 1) + fib(n -2)
  }
}
\end{Code}

\Subtask Använd ovan \code{fib}-objekt i en ny version av din webserver. Spara den i en ny kodfil med namnet \texttt{fibserver-memcached.scala}. Undersök hur snabbt det går med stora Fibonaccital med den nya varianten. Hur stora tal kan du räkna ut? Kan servern fortsätta efter överflödad stack? Förklara varför.

\Subtask Nu när vi kan få väldigt stora Fibonacci-tal kan det vara användbart att stoppa in radbrytningar på webbsidan. Html-taggen \texttt{</br>} ger en radbrytning.
\begin{Code}
  def insertBreak(s: String, n: Int = 80): String = {
    if (s.size < n) s
    else s.take(n) + "</br>" + insertBreak(s.drop(n),n)
  }
\end{Code}
Använd den rekursiva funktionen ovan för att pilla in radbrytningstaggar på var $n$:te position i långa strängar. Testa hur det ser ut på webbsidan med ovan funktion när din server svarar med väldigt stora tal.

\Subtask Vi ska nu använda det större heap-minnet i stället för stack-minnet och därmed inte begränsas av stackens max-storlek. Skriv om \code{fastFib} så att den använder en \code{while}-sats i stället för ett rekursivt anrop. Denna uppgift är ganska klurig, men om du kör fast kan du snegla i lösningarna i Appendix för inspiration.

Hur stora tal klarar din server nu? Vad händer med servern när minnet tar slut? Hur kan du skydda servern så att den inte kan hänga sig?

\SOLUTION


\TaskSolved \what
 %%%TODO number  9 %%%

\SubtaskSolved  \code{abbasillen} skrivs ut baklänges till \code{nellisabba}.

\SubtaskSolved

\SubtaskSolved

\SubtaskSolved

\SubtaskSolved

\SubtaskSolved

\SubtaskSolved

\SubtaskSolved

\SubtaskSolved

Lösningsförslag:
\scalainputlisting[numbers=left,basicstyle=\ttfamily\fontsize{11}{12}\selectfont]{examples/simple-web-server/fibserver-threaded-memcached-while.scala}


\QUESTEND






\WHAT{}

\QUESTBEGIN

\Task  \what~ Utöka din server med fler beräkningsintensiva funktioner. Exempelvis primtalsberäkningar eller beräkningar av valfritt antal decimaler av $\pi$ eller $e$. Utnyttja gärna det du lärt dig i  matematiken om summor och serieutvecklingar.

\SOLUTION


\TaskSolved \what
 %%%TODO number  10 %%%

---


\QUESTEND






\WHAT{}

\QUESTBEGIN

\Task  \what~ Läs mer om \code{Future} och jämlöpande exekvering i Scala här:\\
\href{http://alvinalexander.com/scala/future-example-scala-cookbook-oncomplete-callback}{alvinalexander.com/scala/future-example-scala-cookbook-oncomplete-callback}

\SOLUTION


\TaskSolved \what
 %%%TODO number  11 %%%

---


\QUESTEND






\WHAT{}

\QUESTBEGIN

\Task  \what~ Läs mer om jämlöpande exekvering och multitrådade program i Java här: \href{http://www.tutorialspoint.com/java/java_multithreading.htm}{www.tutorialspoint.com/java/java\_multithreading.htm}  \\
\noindent När man skriver program med jämlöpande exekvering finns det många fallgropar; det kan bli kapplöpning \Eng{race conditions} om gemensamma resurser och dödläge \Eng{deadlock} där inget händer för att trådar väntar på varandra. Mer om detta i senare kurser.


\SOLUTION


\TaskSolved \what
 %%%TODO number  12 %%%

---


\QUESTEND






\WHAT{Studera dokumentationen i \code{scala.concurrent}.}

\QUESTBEGIN

\Task  \what~\Pen

\Subtask Studera dokumentationen för \code{scala.concurrent.Future}\footnote{\href{http://www.scala-lang.org/files/archive/api/current/\#scala.concurrent.Future}{http://www.scala-lang.org/files/archive/api/current/\#scala.concurrent.Future}}. Hur samverkar \code{Future} med \code{Try} och \code{Option}? Vilka vanliga samlingsmetoder känner du igen?

\Subtask Studera dokumentationen för \code{scala.concurrent.duration.Duration}\footnote{\href{http://www.scala-lang.org/api/current/\#scala.concurrent.duration.Duration}{www.scala-lang.org/api/current/\#scala.concurrent.duration.Duration}}. Vilka tidsenheter kan användas?

\Subtask Vid import av \code{scala.concurrent.duration._ } dekoreras de numeriska klasserna med metoder för att skapa instanser av klassen \code{Duration}. Detta möjligörs med hjälp av klassen \code{scala.concurrent.duration.DurationConversions}. Studera dess dokumentation och testa att i REPL skapa några tidsperioder med metoderna på \code{Int}.



\SOLUTION


\TaskSolved \what
 %%%TODO number  13 %%%

\SubtaskSolved

\SubtaskSolved

\SubtaskSolved


\QUESTEND






\WHAT{}

\QUESTBEGIN

\Task  \what~ Fördjupa dig inom webbteknologi.

\Subtask Lär dig om HTML, CSS och JavaScript här: \url{https://developer.mozilla.org/en-US/docs/Learn}

\Subtask Lär dig om Scala.JS här: \url{http://www.scala-js.org/}\SOLUTION


\TaskSolved \what
 %%%TODO number  14 %%%

\SubtaskSolved  ---

\SubtaskSolved  ---

\SubtaskSolved  ---

\SubtaskSolved  ---
\QUESTEND



%\chapter{Snabbreferens}\label{chapter:quickref}
%
%Detta appendix innehåller en snabbreferens för Scala och Java. Snabbreferensen är enda tillåtna hjälpmedel under kursens skriftliga tentamen.
%
%Lär dig vad som finns i snabbreferensen så att du snabbt hittar det du behöver och träna på hur du  effektivt kan dra nytta av den när du skriver program med papper och penna utan datorhjälpmedel.
%
%\clearpage
%~
%\clearpage
%
%\includepdf[pages={1-12}, scale=0.77, frame]{../quickref/quickref.pdf}


\end{document}
