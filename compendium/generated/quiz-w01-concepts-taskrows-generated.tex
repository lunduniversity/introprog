  litteral & 1 & & A & vid anrop sker (sido)effekt; returvärdet är tomt \\ 
  sträng & 2 & & B & för att ändra en variabels värde \\ 
  sats & 3 & & C & bra då antalet repetitioner är bestämt i förväg \\ 
  uttryck & 4 & & D & anger ett specifikt datavärde \\ 
  funktion & 5 & & E & antingen sann eller falsk \\ 
  procedur & 6 & & F & sker innan exekveringen startat \\ 
  exekveringsfel & 7 & & G & bra då antalet repetitioner ej är bestämt i förväg \\ 
  kompileringsfel & 8 & & H & beskriver vad data kan användas till \\ 
  abstrahera & 9 & & I & en kodrad som gör något; kan särskiljas med semikolon \\ 
  kompilera & 10 & & J & sker medan programmet kör \\ 
  typ & 11 & & K & kombinerar värden och funktioner till ett nytt värde \\ 
  for-sats & 12 & & L & att översätta kod till exekverbar form \\ 
  while-sats & 13 & & M & decimaltal med begränsad noggrannhet \\ 
  tilldelning & 14 & & N & en sekvens av tecken \\ 
  flyttal & 15 & & O & vid anrop beräknas ett returvärde \\ 
  boolesk & 16 & & P & att införa nya begrepp som förenklar kodningen \\ 