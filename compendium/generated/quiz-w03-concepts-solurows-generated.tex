  parameter & 1 & ~~\Large$\leadsto$~~ &  D & namn i funktionshuvud; binds till argument vid anrop \\ 
  argument & 2 & ~~\Large$\leadsto$~~ &  F & uttryck som är invärde vid funktionsapplicering \\ 
  block & 3 & ~~\Large$\leadsto$~~ &  C & kan ha lokala namn; sista uttrycket blir returvärde \\ 
  värdeanrop & 4 & ~~\Large$\leadsto$~~ &  A & argumentet evalueras innan funktionen appliceras \\ 
  namnanrop & 5 & ~~\Large$\leadsto$~~ &  E & fördröjd evaluering av argument \\ 
  namngivna argument & 6 & ~~\Large$\leadsto$~~ &  B & möjliggör att argument kan ges i valfri ordning \\ 
  tupel & 7 & ~~\Large$\leadsto$~~ &  H & en sekvens med ett visst antal värden, ev. av olika typ \\ 
  funktionshuvud & 8 & ~~\Large$\leadsto$~~ &  I & har en parameterlista och eventuellt en returtyp \\ 
  funktionskropp & 9 & ~~\Large$\leadsto$~~ &  G & koden som exekveras vid funktionsapplicering \\ 
  anonym funktion & 10 & ~~\Large$\leadsto$~~ &  L & funktion utan namn; kallas även lambda \\ 
  parameterlista & 11 & ~~\Large$\leadsto$~~ &  K & beskriver namn och typ på parametrar om fler än noll \\ 
  rekursiv funktion & 12 & ~~\Large$\leadsto$~~ &  J & en funktion som anropar sig själv \\ 