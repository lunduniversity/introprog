%!TEX root = ../compendium.tex

\Exercise{\ExeWeekTHREE}

\begin{Goals}
\item 
\end{Goals}

\begin{Preparations}
\item 
\end{Preparations}

\BasicTasks %%%%%%%%%%%%%%%%

\Task \emph{Definiera och anropa funktioner.} 
\\En funktion med två parametrar definieras med följande syntax i Scala: \\ \texttt{\code{def} \textit{namn}(\textit{parameter1}: \textit{Typ1}, \textit{parameter2}: \textit{Typ2}): \textit{Returtyp} = \textit{returvärde}}

\Subtask Definiera en funktion med namnet \code{öka} som har en heltalsparameter \code{x} och som returnerar \code{x + 1}. Ange returtypen explicit. Testa funktionen i REPL med argumentet 42.

\begin{REPL}
scala> ???  // definiera funktionen öka
scala> öka(42)
43
\end{REPL}

\Subtask\Pen Vad har funktionen \code{öka} i föregående uppgift för returtyp?

\Subtask\Pen Vad gör kompilatorn om du utelämnar returtypen?

\Subtask\Pen Varför kan det vara bra att ange returtypen explicit?

\Subtask\Pen Vad är det för skillnad mellan parameter och argument?
 
\Subtask Vad har uttrycket \code{öka(öka(öka(öka(42))))} för värde?

\Subtask Definera funktionen \code{minska(x: Int): Int} med returvärdet \code{x - 1}.

\Subtask Vad har uttrycket \code{öka(minska(öka(öka(minska(minska(42))))))}

\Task Föränderlighet och oföränderlighet.

\Subtask Innan du kör nedan kod: Försök lista ut vad som kommer skrivas ut. Rita minnessituationen efter varje tilldelning.

\begin{Code}
println("\n--- Example 1: mutable value assigmnent")
var x1 = 42
var y1 = x1
x1 = x1 + 42
println(x1)
println(y1)
\end{Code}

\Subtask Innan du kör nedan kod: Försök lista ut vad som kommer skrivas ut. Rita minnessituationen efter varje tilldelning.

\begin{Code}
println("\n--- Example 2: mutable object reference assignment")
class MutableInt(private var i: Int) {
  def +(a: Int): MutableInt = { i = i + a; this }
  override def toString = i.toString
}
var x2 = new MutableInt(42)
var y2 = x2
x2 = x2 + 42
println(x2)
println(y2)
\end{Code}

\Subtask Innan du kör nedan kod: Försök lista ut vad som kommer skrivas ut. Rita minnessituationen efter varje tilldelning.

\begin{Code}
println("\n--- Example 3: immutable object reference assignment")
class ImmutableInt(val i: Int) {
  def +(a: Int): ImmutableInt = new ImmutableInt(i + a) 
  override def toString = i.toString
}
var x3 = new ImmutableInt(42)
var y3 = x3
x3 = x3 + 42
println(x3)
println(y3)
\end{Code}

\Subtask Vad ger nedan kod för utskrift?

\Subtask Vad ger nedan kod för utskrift?


\ExtraTasks %%%%%%%%%%%%%%%%%%%

\Task 

\AdvancedTasks %%%%%%%%%%%%%%%%%

\Task 