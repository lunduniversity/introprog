%!TEX encoding = UTF-8 Unicode
%!TEX root = ../compendium2.tex

\Lab{\LabWeekTHREE}
\begin{Goals}
%!TEX encoding = UTF-8 Unicode
%!TEX root = ../compendium2.tex

%\item Kunna kompilera Scalaprogram med \texttt{scalac}.
%\item Kunna köra Scalaprogram med \texttt{scala}.
%\item Kunna definiera och anropa funktioner.
%\item Kunna använda och förstå default-argument.
%\item Kunna ange argument med parameternamn.
\item Kunna skapa ett större program med din egen kod efter dina egna idéer.
\item Kunna använda en editor och terminalen för att iterativt editera, kompilera, och testa din kod.
\item Kunna använda variabler i kombination med alternativ och repetetition i flera nivåer.
\item Kunna stegvis förbättra din kod för att underlätta förändring och öka läsbarhet.
\item Kunna skapa och använda abstraktioner för att generalisera och möjliggöra återanvändning av kod.

\end{Goals}

\begin{Preparations}
\item Gör övning \texttt{\ExeWeekTHREE} och repetera övning \texttt{\ExeWeekTWO} innan du påbörjar laborationen.
\item Läs appendix~\ref{appendix:terminal} och~\ref{appendix:compile}.
\item Hämta given kod via \href{https://github.com/lunduniversity/introprog/tree/master/workspace/}{kursen github-plats}.
\item Utveckla en första, spelbar version av ditt textspel, som du kan jobba vidare på under laborationen.
\item Hitta någon som spelar en tidig version av ditt spel och läser din kod och ger återkoppling på kodens läsbarhet. Skriv ner den återkoppling du får.
\item Spela någon annans textspel och ge återkoppling på kodens läsbarhet.
\end{Preparations}


\subsection{Krav}

\begin{itemize}
\item Du ska skapa ett lagom irriterande textspel med hjälp av en editor, till exempel VS \texttt{code} (se appendix~\ref{appendix:compile:edit}). Spelet ska köras i terminalen.

\item Under redovisningen av laborationen ska du redogöra för vilka programmeringskoncept du tränat på under utvecklingen av ditt textspel. Du ska också för handledaren beskriva hur du har förbättrat din kod genom den återkoppling du fått från någon som spelat ditt spel och läst koden.

\item Ditt textspel ska vara \emph{lagom} irriterande om den som spelar har läst koden, medan spelet gärna får vara orimligt irriterande för den som \emph{inte} läst koden. Det ska gå att klara spelet (du väljer själv vad det innebär) och därmed avsluta programmet inom rimlig tid med kännedom om koden.

\item Försök göra din kod \textit{lätt att läsa och förstå}, även om själva spelet stundtals kan vara mer eller mindre obegripligt, knasigt, eller besvärligt, för den spelare som inte har tillgång till koden... Observera att din kod inte behöver vara ''perfekt'' från början. Börja fritt och förbättra efterhand.

\item Allteftersom ditt program blir längre ska du omforma och dela upp din kod i många, korta abstraktioner med väl valda namn för att öka läsbarheten.

\item Din kod ska använda de viktiga begrepp som kursen hittills har behandlat, med speciellt fokus på det som just du behöver träna mest på.

%\item Slumptal ska ingå i ditt spel och styra valfria delar av exekveringen. Det ska även gå att ge ett valfritt slumptalsfrö som argument vid exekveringen av ditt program. Om fröargument ges ska exekveringen bli återupprepningsbar för en given indatasekvens, annars ska utfallet kunna bli olika vid upprepade körningar med samma indata.
\end{itemize}

\subsection{Tips för att komma igång}

\begin{itemize}
\item Skapa en katalog som innehåller en scala-kodfil med valfritt namn.
\item Skriv en enkel \code{@main}-metod i den nyskapade kodfilen som endast skriver ut strängen \code{"Hello World!"}.
\item Kompilera och kör, rätta eventuella fel tills programmet fungerar korrekt.
\item När programmet fungerar, börja utöka \code{@main}-metoden i din kodfil och implementera mer funktionalitet, ta en titt under inspiration nedan.
\item Börja enkelt och försök formulera vad ditt program ska göra med \emph{psuedokod} som kommentarer innan du skriver koden.
\item Kompilera och kör vid varje tillägg och håll varje tillägg så litet som möjligt, så slipper du reda ut en massa svåra följdfel vid kompilering och eventuella körtidsfel blir mer begripliga.  
\item Fortsätt utöka tills kraven för labben har uppnåtts.
\end{itemize}

\subsection{Inspiration}

Här följer en lista med olika förslag på funktioner som du kan välja bland, kombinera och variera på olika vis. Du kan också låta helt andra funktioner ingå i ditt spel. Det viktigaste är att du kombinerar kodglädje med lärorika utmaningar :)

\begin{itemize}
\item Be användaren logga in. Ge knasiga felmeddelande om användaren inte kan lösenordet.
\item Låt användaren hamna i en irriterande oändlig loop av meningslösa frågor om den gör ''fel''.
\item Beskriv en läskig fantasiplats där användaren befinner sig, till exempel en grotta | en källare | ett rymdskepp | Kemicentrum.
\item Låt användaren välja mellan fåniga vapen, till exempel golvmopp | örontops | foliehatt | förgiftad kexchoklad.
\item Låt användaren välja mellan olika vägar | dörrar | tunnlar | sektionscaféer. Låt valet styra vilka monster som påträffas. Låt användaren bekämpa monstret med olika vapen.
\item Inför någon slags poäng som redovisas under spelets gång och i slutet.
\item Inför olika sorters poäng för hälsa, stridskraft, uppnådd skicklighetsnivå, etc.
\item Fråga användaren om mer eller mindre relevanta detaljer: namn | skonummer | favorithusdjur. Ge knasiga kommentarer där dessa detaljer ingår som delsträngar.
\item Spela sten | sax | påse med användaren.
\item Spela ''gissa talet'' och ge ledtrådar om talet är för litet eller för stort.
\item Mät hur lång tid det tar för användaren att klara ditt spel och ge poäng därefter.
\item Kolla reaktionstiden hos användaren genom att mäta tiden det tar att trycka Enter efter att man fått vänta en slumpmässig tid på att strängen \code{"NU!"} skrivs ut. Om man trycker Enter innan startutskriften ges blir den uppmätta tiden 0 och på så sätt kan ditt program detektera att användaren har tryckt för tidigt. Mät reaktionstiden upprepade gånger och ge poäng efter medelvärdet.
\item Låt användaren på tid så snabbt som möjligt skriva olika ord baklänges.
\item Be användaren skriva en palindrom. Ge poäng efter längd.
\item Träna användaren i multiplikationstabellen på tid.
\item Låt användaren svara på flervalsfrågor om din favoritfilm.
\item Gör det möjligt att ge ett extra argument med en ''fuskkod'' som ger användaren speciella förmågor eller på annat sätt underlättar för användaren under spelets gång.
\end{itemize}

%\subsection{Tips}

%\begin{itemize}
%\item Du kommer åt första argumentet till ditt program genom att indexera i en array som heter \code{args} på plats noll så här: \code{args(0)}.
%\item Du kan kontrollera om det finns minst ett argument med hjälp av det booelska uttrycket \code{args(0).length > 0}.
%\item Metoden \code{toInt} kan göra om en sträng till ett heltal. Du kan vid felaktiga heltal ge ett defaultvärde med \code{scala.util.Try(args(0).toInt).getOrElse(42)}.
%\item Du läser från \textit{standard in} med \code{scala.io.StdIn.readline(prompt)} där \code{prompt} är en sträng som skrivs till \textit{standard out} innan inläsning sker.
%\item Sök upp och studera dokumentationen för klassen \code{scala.util.Random}.
%\item Du kan vänta i t.ex. 3 sekunder med hjälp av Thread.sleep(3000).
%\end{itemize}
