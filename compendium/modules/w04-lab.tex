%!TEX encoding = UTF-8 Unicode

%!TEX root = ../compendium.tex


\Lab{\LabWeekFOUR}

\begin{Goals}
\item Kunna använda utvecklingsmiljön Eclipse och dess Scala IDE.
\item Kunna använda case classer för att spara data.
\item Kunna spara data till fil.
\item Kunna l{\"a}sa in data fr{\aa}n fil med \code{scala.io}.
\item Kunna skapa och använda klasser för att behandla data.
\item Kunna använda samlingstyperna vektor och map samt Option.
\item Förstå och kunna använda Option, Some och None. 
\item Förstå skillnaden mellan kompileringsfel och exekveringsfel.
\item Kunna avlusa (debugga) program med hjälp av Eclipse.

%\item Att spara objekt till fil.
%\item Att l{\"a}sa in objekt fr{\aa}n en fil.
\end{Goals}

\begin{Preparations}
\item G{\"o}r {\"o}vning 4. % ref
\item Läs om Eclipse i Appendix.
\item Läs igenom laborationen och gör förberedelseuppgiften.
%\item {\"O}ppna Scala IDE i Eclipse enligt intruktionerna XX.
%\item Skapa ett projekt och skapa ett \code{object Hello} med en \code{main}-metod enligt XY.
%\item Skriv ut en h{\"a}lsning till terminalen med \code{println("...")} och testk{\"o}r programmet genom att markera filnamnet i projektmenyn och trycka p{\aa} den gr{\"o}na pilen. Kontrollera att h{\"a}lsningen skrivs ut!
\end{Preparations}


\subsection{Förberedelseuppgifter}
\Task Ladda hem zip-filen med kursens workspace från git-repot (XX) och packa upp det på valfritt ställe på din dator. Starta programmet Eclipse och följ instruktionerna i figurer för att skapa ett program och exekvera det.

% Change this to whatever environment we're supposed to use
\begin{center}

\includegraphics[width=0.7\textwidth]{../img/pirates/selectws.png} \\
1. Bläddra fram till kurs-workspacet. \\
\vspace{5mm}

\includegraphics[width=0.7\textwidth]{../img/pirates/selectws2.png} \\
2. Första gången Eclipse öppnas har den en välkomstskärm, klicka på {\bf Workbench}.
\vspace{5mm}

\includegraphics[width=0.7\textwidth]{../img/pirates/selectscala.png} \\
3. Kontrollera att du har Scala igång uppe i högra hörnet, annars klickar du på knapp nummer 1 och väljer Scala bland dina plugin (2) och {\bf OK}. 

\vspace{5mm}

\includegraphics[width=0.7\textwidth]{../img/pirates/createproject.png} \\
4. Skapa ett nytt projekt genom att höger-klicka i {\bf Package Explorer}.

\vspace{5mm}

\includegraphics[width=0.5\textwidth]{../img/pirates/nameproject.png} \\
5.  Döp ditt projekt till ett beskrivande namn.

\vspace{5mm}
\includegraphics[width=0.7\textwidth]{../img/pirates/createobject.png} \\
6. Genom att höger-klicka på projektet kan man lägga till filer, t ex skapa ett paket eller en klass. Börja med att skapa ett objekt ...

\vspace{5mm}

\includegraphics[width=0.4\textwidth]{../img/pirates/nameobject.png} \\
7 ... som du döper till nåt beskrivande!

\vspace{5mm}
\includegraphics[width=0.7\textwidth]{../img/pirates/exekvera.png} \\
8. Lägg till kod som skriver ut en hälsning. Exekvera sedan genom att trycka på den gröna pilen och se om utskriften kommer ut i  Eclipse-konsollen.
\end{center}

Prova att stava fel till \code{println}, då dyker det upp ett rött kryss till vänster på samma kodrad som berättar vad du gjort för fel. Kompilatorn kör i bakgrunden och innan koden kan exekveras måste alla kompileringsfel åtgärdas. Men betyder detta att programmet alltid kommer bete sig korrekt under körningen?


\subsection{Obligatoriska uppgifter}
Efter en rad olyckliga omständigheter har du blivit pirat i 1700-talets Karibien. Nu beh{\"o}ver du undvika galgen, hitta en skatt och f{\"o}rs{\"o}ka f{\"o}rutse dina f{\"o}r{\"a}diska skeppskamraters n{\"a}sta steg.

\Task \emph{Save your crew}. 

\Subtask Kung George {\"a}r villig att ben{\aa}da fem personer ur din bes{\"a}ttning! Skapa en lista (nåja, vektor) d{\"a}r personerna sparas med f{\"o}rnamn, efternamn och befattning genom att l{\"a}sa in dem fr{\aa}n konsolen i Eclipse. Inl{\"a}sning kan g{\"o}ras med \code{scala.io} med kodraden: 
\begin{Code}
val first = scala.io.StdIn.readLine("Förnamn: "). 
\end{Code}
Namnen och befattningen kan sparas som en \code{case class CrewMember}! Skriv ett program som läser in namn och befattning på dina besättningsmedlemmar och spara dem i en vektor.

\Subtask Vi vill skriva besättningen till en fil så att den faktiskt sparas. Till din hjälp får du följande kodsnutt som tar en sträng \code{s} och sparar den till en fil \code{fileName}:
\begin{Code}
def write(s: String, fileName: String): Unit = {
	import java.nio.file.{Paths, Files, StandardOpenOption}
	Files.write(Paths.get(fileName), s.getBytes("UTF-8"), 
	StandardOpenOption.CREATE, 
	StandardOpenOption.TRUNCATE_EXISTING) // will append
}
\end{Code}

Istället för att hårdkoda filnamnet går det att ange som argument genom att högerklicka på klassfilen i {\bf Project explorer}, välja {\bf Run As} -> {\bf Run Configurations} och under fliken {\bf Arguments} skriva in argument (separerade med mellanslag) och sen välja {\bf Run} eller {\bf Apply}. 
\begin{center}
\includegraphics[width=0.5\textwidth]{../img/pirates/args.png} \\
\end{center}

Lägg till kodrader i ditt program som sparar besättningen till filnamnet i det första argumentet om ett sådant har angivits, annars till filen \code{crew.txt}.

\Subtask Det går att överskugga toString() i \code{CrewMember} och på så sätt ändra utskriften genom att lägga till följande kodrad inne i klassen: 
\begin{Code}
override def toString(): String = ??? // add your code here. 
\end{Code}

\noindent Ändra utskriften så att den blir {\em snygg}, t ex med komma\\ 

\noindent Jack Sparrow, kapten \\
Anne Bonny, mordlysten matros \\
Ed Kenway, lönnmördare \\
... 

\Task{Avlusa din besättning.}

\Subtask Din moraliska kompass hindrar dig inte från att också jobba för kungen. Hjälp honom att läsa listan!

En fil \code{fileName} kan läsas rad för rad till en vektor med följande kodexempel: 

\begin{Code}
def readLines(fileName: String): Vector[String] = {
	   scala.io.Source.fromFile(fileName).getLines.toVector
	}
\end{Code}
Skapa en funktion \code{readCrewMember(s:String)} som tar en rad från filen och returnerar en \code{CrewMember}. 

\Subtask Skriv ett testprogram som läser in din besättning från filen och skriver ut dem i konsolen. Stämmer det med filinnehållet?

\Subtask Det går att följa exekveringen av programmet stegvis genom att köra det i \emph{debug mode} som startas med 
\includegraphics[width=0.05\textwidth]{../img/pirates/bug.png}

Då öppnas en debuggvy i Eclipse. Variablerna visas uppe i högra hörnet. Där går att klicka och se vilka värden varje variabel har för tillfället. 

Genom att lägga till \emph{break points} (klicka på sidan av koden för att lägga till och ta bort dem) kan programexekveringen pausas just innan raden exekveras  \\
\includegraphics[width=0.5\textwidth]{../img/pirates/breakpoint.png} \\
så att programmeraren kan kontrollera variablernas värden och sen köra vidare med \includegraphics[width=0.05\textwidth]{../img/pirates/next.png}.

\Subtask Den konkurrerande kaptenen Charles Vane betalar dig för att sabotera listan genom att lägga till nonsensrader i filen. Gör det. Vad händer då när du exekverar ditt testprogram?

\Subtask Ändra din funktion \code{readCrewMember(s:String)} så att en korrekt formaterad rad returner en CrewMember medan felaktiga rader ger \code{None}. Det går att göra genom att använda en behållartyp som heter \code{Option} som fungerar som en samling som kan innehålla \code{None} eller något \\ \code{Some(CrewMember(first, last, post))}. Returtypen kan då deklareras som
\begin{Code}
def readCrewMember(s:String): Option[CrewMember] = ...
\end{Code}
Resultatet från ett option-objekt \code{o} hämtas med t ex \code{o.get} eller genom att skriva ut felmeddelanden om det är \code{None}, \code{o.getOrElse("I'm no one")}. Testa ditt nya program och se om det blir som förväntat genom lämpliga break points och utskrifter. 

\Task{Lögner, förbannade lögner och statistik.}\\
\noindent \\
För att du inte ska bli överlistad av dina sluga, lögnaktiga skeppskamrater behöver du kunna gissa hur en pirat tänker. Därför förkovrar du dig i Robert Louis Stevensons \emph{Skattkammarön}\footnote{Vars copyright har gått ut så du behöver inte piratkopiera den.} som finns i filen skattkammarön.txt i workspacet.  Genom att för varje ord spara det mest frekventa nästkommande ordet går det att förutse vad som kommer sägas\footnote{Detta används till exempel i Swiftkey på smarttelefoner.}. 

\Subtask Det går att läsa in ord från en fil med \code{scala.io.Source.fromFile} genom att först läsa in alla rader, ta bort allt som {\em inte} är svenska bokstäver och sen göra en split på \emph{white space}: 
\begin{Code}
def readWords(fileName: String): Vector[String] = {
	   scala.io.Source.fromFile(fileName).getLines.
	   map(_.replaceAll("[^a-zA-ZåäöÅÄÖ\\s]", " ")).
	   flatMap(_.split("\\s+")).filter(!_.isEmpty).toVector
	}
\end{Code}

Skapa objektet \code{PirateSpeech} och lägg till kodrader som läser in alla ord i filen skattkammarön.txt. Skapa sen en klass \code{FrequencyCounter} som för varje ord kan räkna nästkommande ord (sådana ordpar kallas bigram). Varje ord får en egen \code{FrequencyCounter} som i sin tur innehåller en samling, t ex en \code{Map} som listar alla efterföljande ord och deras antal så att vi kan få deras frekvens.

\Subtask Lägg till en funktion \code{add(next:String)} i \code{FrequencyCounter} som ökar frekvensen av det ordet \code{next} eller lägger till det med räknaren satt till 1 om det inte finns. 

\Subtask Skriv ett program som läser in alla ord och räknar ut deras frekvens. Tips: håll reda på orden och deras bigramfrekvensräknare med en \\
\code{[Map[String, FrequencyCounter]]}.

\Subtask Lägg till en funktion \code{getBestGuess()} i \code{FrequencyCounter} som returnerar det mest frekventa efterföljande ordet. 

\Subtask Det tar lång tid att läsa hela boken varje gång programmet körs, det gäller ju att vara kvicktänkt och vi är ju bara intresserade av den mest sannolika gissningen! Spara ordpar \code{word, bestGuess} med den bästa gissningen till varje ord på en fil och läs bara in paren i nästa uppgift. 

\Subtask Skriv ett main-program som läser in ett ord från användaren och gissar vad de ska säga sen. Efterföljs ''James'' oftast av ''Hawkins'' och eller av ''Flint''?


%\Task \emph{Read the map}. 

%\Subtask En underuppgift.

\subsection{Frivilliga extrauppgifter}

\Task Läs in större mängder text, implementera ett tangentbord till Android och bli rik.

%\Subtask En underuppgift.

%\Subtask En underuppgift.
    