%!TEX encoding = UTF-8 Unicode
%!TEX root = ../labs.tex

\Lab{\LabWeekFOUR}
\begin{Goals}
%!TEX encoding = UTF-8 Unicode
%!TEX root = ../labs.tex

\item Kunna förklara hur singelobjekt kan användas som moduler.
\item Kunna förklara hur åtkomst av medlemmar i singelobjekt sker.
\item Kunna skapa kod som reagerar på och förändrar objekts tillstånd.
\item Kunna förklara nyttan med att samla namngivna konstanter i egen modul.
\item Kunna förklara hur import påverkar synlighet av namn.
\item Kunna ge exempel på en situation där man har nytta av namnbyte vid import.
\item Kunna redogöra för skillnaden mellan paket och singelobjekt.
\item Kunna skapa och använda tupler.

\end{Goals}

\begin{Preparations}
\item Gör övning \texttt{\ExeWeekFOUR} och repetera övning \texttt{\ExeWeekTHREE}.
\item Repetera appendix~\ref{appendix:terminal}, ~\ref{appendix:compile}, och ~\ref{appendix:debug}.
\end{Preparations}



\subsection{Bakgrund}


\begin{minipage}{0.48\textwidth}
\begin{figure}[H]
  \centering
  \includegraphics[width=\textwidth]{../img/blockmole-sky-grass.png}
%  \caption{En lundensisk blockmullvad fångad på bild under aktivt grävanade.}
  \label{lab:blockmole:fig:mole}
\end{figure}
\end{minipage}%
%
\hfill\begin{minipage}{0.45\textwidth}
\noindent\textbf{Blockmullvad} (\textit{Talpa laterculus}) är ett fantasidjur i familjen mullvadsdjur.
Den är känd för sitt karaktäristiska kvadratiska utseende.
Den lever mest ensam i sina underjordiska gångar som, till skillnad från den verkliga mullvadens (\emph{Talpa europaea}) gångar, har helt raka väggar.
\end{minipage}



\subsection{Obligatoriska uppgifter}


\Task \emph{Skapa katalog och kodfil.}
Du ska, steg för steg, skapa ett program som låter användaren interagera med en levande blockmullvad. Använd en editor, t.ex. \texttt{atom}, kompilera ditt program i terminalen med \texttt{scalac} och kör med \texttt{scala}.

\Subtask
Skapa en ny fil med namnet \texttt{blockmole.scala} i en ny katalog i din hemkatalog, till exempel \texttt{\textasciitilde/pgk/w04/lab/blockmole.scala}, där \texttt{\textasciitilde} är din hemkatalog.
\begin{REPLnonum}
> mkdir -p ~/pgk/w04/lab
> atom ~/pgk/w04/lab/blockmole.scala
\end{REPLnonum}


\Subtask
Navigera till din nya katalog med \texttt{cd}-kommandot \Eng{change directory} och kontrollera med \texttt{ls}-kommandot \Eng{list} att din nya fil finns där.
\begin{REPLnonum}
> cd ~/pgk/w04/lab/
> ls
\end{REPLnonum}
Om allt går bra ska \texttt{ls}-kommandot skriva ut \texttt{blockmole.scala}.

\Subtask
Gå tillbaka till din texteditor och gör i början av filen \code{blockmole.scala} en paketdeklaration så att koden ingår i paketet \code{blockmole}.

\Subtask
Deklarera sedan ett singelobjekt med namnet \code{Main} och lägg till en \code{main}-funktion i objektet som skriver ut texten: \texttt{"Keep on digging!"}

\Subtask
Kompilera ditt program. När du rätta eventuella fel, ett i taget, och lyckats kompilera helt utan fel, kontrollera med \texttt{ls}-kommandot att några filer som slutar på \texttt{class} har skapats i subkatalogen \code{blockmole}. \Pen Varför hamnade bytekoden i denna katalog?

\Subtask
Kör kommandot \texttt{scala blockmole.Main} för att exekvera ditt program.
Om allt går bra ska texten du angivit skrivas ut i terminalfönstret.

\vspace{1em}\noindent Nu har du skrivit ett program som skriver ut en uppmaning till en mullvad att fortsätta gräva.
Det programmet är inte så användbart, eftersom mullvadar inte kan inte läsa. Nästa steg är därför att skriva ett grafiskt program.%, snarare än ett textbaserat.






\Task \emph{Skapa en grundstruktur för programmet.}
I mindre program fungerar det bra att samla många funktioner i samma singelobjekt, men i stora program blir det lättare att hitta i koden och förstå vad den gör om man har flera moduler med olika ansvar. Ditt program ska ha följande övergripande struktur:

\begin{Code}
package blockmole

object Colors {
  // Samlar olika färger som behövs i övriga objekt
}

object Graphics {
  // Har ett SimpleWindow och procedurer för blockgrafik
}

object Mole {
  // Representerar en blockmullvad som kan gräva:
  def dig(): Unit = println("Här ska det grävas!")
}

object Main {
  def drawWorld(): Unit = println("Ska rita ut underjorden!")

  def main(args: Array[String]): Unit = {
    drawWorld()
    Mole.dig()
  }
}
\end{Code}

Vi lägger i denna laboration alla moduler i samma fil, men om modulerna blir stora och ska återanvändas av flera olika program är det bra att ha dem i olika filer så att de kan kompileras och testas separat.

\Subtask Skapa programskelettet ovan i filen \code{blockmole.scala} och se till att koden kompilerar utan fel och går att köra med utskrifter som förväntat.

\Subtask I singelobjektet \code{Color} ska vi lägga in färger med hjälp av Java-klassen \code{java.awt.Color}. Eftersom vårt singelobjektnamn''krockar'' med namnet på färgklassen i Java-paketet  så byter vi namn på Java-klassen till \code{Jcolor} i importdeklarationen. Lägg in en importdeklaration med namnbytet direkt efter paketdeklarationen. Vi lägger importen så att den syns i hela paketet eftersom flera objekt behöver tillgång till \code{JColor}. Säkerställ att koden fortfarande kompilerar utan fel.

\Subtask Lägg in nedan färger i objektet \code{Colors}:
\begin{Code}
object Color {
  val black  = new JColor(  0,   0,   0)
  val mole   = new JColor( 51,  51,   0)
  val soil   = new JColor(153, 102,  51)
  val tunnel = new JColor(204, 153, 102)
}
\end{Code}


\Subtask Lägg till nedan tre variabler i singelobjektet \code{Graphics}:

\begin{Code}
val windowSize = (30, 50)  // number of blocks width, height
val blockSize  = 10        // number of pixels per block

val win = new SimpleWindow(???, ???, ???)
\end{Code}

\begin{itemize}%[noitemsep]
  \item Gör så att storleken på \code{win} motsvarar blockstorleken gånger fönsterstorleken.
  \item Ge fönstret en lämplig tiltel, t.ex. \code{"Digging Blockmole"}.
  \item Gör en lokal import-deklaration i \code{Graphics} då det bara är detta objekt som behöver tillgång till \code{SimpleWindow}.
  \item När du kompilerar behöver du se till att \code{cslib} finns tillgänglig på \code{classpath} (se övning \texttt{\ExeWeekFOUR}).
  \item Om du glömt ordningen på parametrarna till klassen \code{SimpleWindow} så kolla i dokumentationen\footnote{\url{http://cs.lth.se/pgk/api/}}. Det går tyvärr inte att använda namngivna argument när man använder Java-klasser.
\end{itemize}

\Subtask Lägg till en enkel utritning genom att i proceduren \code{drawWorld} använda \code{Graphics.win}, för att se så att allt fungerar så här långt, till exempel:
\begin{Code}
  def drawWorld(): Unit = Graphics.win.lineTo(100,100)
\end{Code}
Kompilera och kör och säkerställ att allt fungerar som förväntat.


\Task Nu har du gjort ett grafiskt program, men ännu syns ingen mullvad.
Det är dags att skapa koordinatsystemet i blockmullvadens blockvärld.

\Subtask\Pen
Inför redovisningen: säkerställ att du kan förklara vad \code{x}- och \code{y}-parametrarna i \code{SimpleWindow.lineTo} innebär, genom att med papper och penna rita en enkel skiss av ungefär var positionerna $(0,0)$, $(300, 0)$, $(0, 300)$ och $(300, 300)$ ligger i ett fönster som är 300 bildpunkter \Eng{pixels} brett och 500 bildpunkter högt.

\Subtask
Nu ska du skapa ett nytt koordinatsystem för \code{Graphics} som har \emph{stora} bildpunkter.
Vi kallar \code{Graphics} stora bildpunkter för \emph{block} för att lättare skilja dem från de enpixelstora bildpunkterna i \code{SimpleWindow}.

\begin{framed}
\noindent I block-koordinatsystemet för \code{Graphics} gäller följande:

 Om blockstorleken är $b$, så ligger koordinaten $(x, y)$ i \code{Graphics} på koordinaten $(bx, by)$ i \code{SimpleWindow}.

\end{framed}

\noindent Implementera funktionen \code{block} i modulen \code{Graphics} enligt nedan, så att ett block ritas ut. Parametern \code{point} anger blockkoordinaten och parametern \code{color} anger färgen. Fyll i det som saknas.
\begin{Code}
  def block(point: (Int, Int))(color: JColor = Color.black): Unit = {
    win.setLineColor(color)
    val (left, top)     = (point._1 * blockSize, point._2 * blockSize)
    val (right, bottom) = (left + blockSize - 1, top + blockSize - 1)
    for (row <- top to bottom) {
      ???
    }
  }
\end{Code}
Säkerställ att koden kompilerar utan fel.

\Subtask\Pen
Metoden \code{block} ritar ett antal linjer.
Hur många linjer ritas ut?
I vilken ordning ritas linjerna?
Skriv ner dina svar inför redovisningen.

\Subtask
Anropa funktionen \code{Graphics.block} några gånger i \code{Main.drawWorld} så att några block ritas upp i fönstret när programmet körs. Kompilera och kör ditt program och kontrollera att allt fungerar som det ska.




\Task \emph{Rektangelprocedur.}
Nu ska du skriva en funktion för att rita en rektangel. Rektangeln ska ritas med hjälp av funktionen \code{block}.
Sen ska du rita upp mullvadens underjordiska värld med hjälp av denna funktion.

\Subtask
Lägg till en funktion i objektet \code{Graphics} med namnet \code{rectangle} som tar fem parametrar \code{x}, \code{y}, \code{width} och \code{height} av typen \code{Int} och \code{color} av typen \code{Color}.
Parametrarna \code{x} och \code{y} anger \code{Graphics}-koordinaten för rektangelns övre vänstra hörn och \code{width} och \code{height} anger bredden respektive höjden.
Använd följande \code{for}-satser för att rita ut rektangeln.
\begin{Code}
for (yy <- y until (y + height)) {
	for (xx <- x until (x + width)) {
		block(xx, yy, color)
	}
}
\end{Code}

\Subtask\Pen
I vilken ordning ritas blocken ut?

% \Subtask\Pen (Fråga något om skuggning gällande \code{width} och \code{height}.)

\Subtask
Skriv en funktion i objektet \code{Mole} med namnet \code{drawWorld} som ritar ut mullvadens värld, det vill säga en massa jord där den kan gräva sina tunnlar.
\code{Mole.drawWorld} ska inte ha några parametrar och returtypen ska vara \code{Unit} och den ska anropa \code{Graphics.rectangle} för att rita en rektangel med färgen \code{Colors.soil} som precis täcker fönstret.
Eftersom funktionen har många parametrar som lätt kan blandas ihop ska du använda namngivna argument vid anropet.
(Om du har glömt hur man använder namngivna argument kan du titta på övningarna i kapitel~\ref{exe:W03}.)

\Subtask
Anropa \code{Mole.drawWorld} i \code{Mole.main} och testa så att det fungerar.

\Task
I \code{SimpleWindow} finns funktioner för att känna av tangenttryckningar och musklick.
Du ska använda de funktionerna för att styra en liten blockmullvad.

\Subtask
Importera \code{cslib.window.SimpleWindow} i \code{Graphics} och lägg till denna metod:
\begin{Code}
def waitForKey(): Char = {
  w.waitForEvent()
  while (w.getEventType() != SimpleWindow.KEY_EVENT) w.waitForEvent()
  w.getKey
}
\end{Code}
Det finns olika sorters händelser som ett \code{SimpleWindow} kan reagera på, till exempel tangenttryckningar och musklick.
Funktionen som du precis lagt in väntar på en händelse i ditt \code{SimpleWindow} (\code{w.waitForEvent}) ända tills det kommer en tangenttryckning (\code{KEY_EVENT}).
När det kommit en tangenttryckning anropas \code{w.getKey} för att ta reda på vilken bokstav eller vilket tecken det blev, och det resultatet blir också resultatet av \code{waitForKey}, eftersom det ligger sist i blocket.

\Subtask
Lägg till en funktion i objektet \code{Mole} med namnet \code{dig}, utan parametrar och med returtypen \code{Unit}.
Funktionens kropp ska se ut såhär (fast utan \code{???}):
\begin{Code}
{
  var x = Graphics.width / 2
  var y = Graphics.height / 2
  while (true) {
    Graphics.block(x, y, Colors.mole)
    val key = Graphics.waitForKey()
    if (key == 'w') ???
    else if (key == 'a') ???
    else if (key == 's') ???
    else if (key == 'd') ???
  }
}
\end{Code}
Fyll i alla \code{???} så att \code{'w'} styr mullvaden ett steg uppåt, \code{'a'} ett steg åt vänster, \code{'s'} ett steg nedåt och \code{'d'} ett steg åt höger.

\Subtask
Ändra \code{Mole.main} så att den bara innehåller två anrop: ett till \code{drawWorld} och ett till \code{dig}.
Kompilera och kör ditt program för att se om programmet reagerar på w, a, s och d.

\Subtask
Om programmet fungerar kommer det bli många mullvadar som tillsammans bildar en lång mask, och det är ju lite underligt.
Lägg till ett anrop i \code{Mole.dig} som ritar ut en bit tunnel på position $(x, y)$ efter anropet till \code{Graphics.waitForKey} men innan \code{if}-satserna.
Kompilera och kör ditt program för att gräva tunnlar med din blockmullvad.

\subsection{Kontrollfrågor}\Checkpoint

\noindent Repetera teorin för denna vecka och var beredd på att kunna svara på dessa frågor när det blir din tur att redovisa vad du gjort under laborationen:

\begin{enumerate}
\item Till vad används \emph{classpath}?
\item Vad är en \code{jar}-fil?
\item Vad innebär punktnotation?
\item Ge exempel på användning av \code{import} och förklara vad som händer.
\item Vad är fördelen med skuggning och lokala namn?
\item Vi använde flera singelobjekt som olika s.k. \code{moduler} i denna laboration. Vad är fördelen med att att dela upp koden i moduler?
\end{enumerate}

\clearpage

\subsection{Frivilliga extrauppgifter}

\Task
Mullvaden kan för tillfället gräva sig utanför fönstret.
Lägg till några \code{if}-satser i början av \code{while}-satsen som upptäcker om \code{x} eller \code{y} ligger utanför fönstrets kant och flyttar i så fall tillbaka mullvaden precis innanför kanten.

\Task
Mullvadar är inte så intresserade av livet ovanför jord, men det kan vara trevligt att se hur långt ner mullvaden grävt sig.
Lägg till en himmelsfärg och en gräsfärg i objektet \code{Colors} och rita ut himmel och gräs i \code{Mole.drawWorld}.
Justera också det du gjorde i föregående uppgift, så mullvaden håller sig under jord.
(\emph{Tips:} Den andra parametern till \code{Color} reglerar mängden grönt och den tredje parametern reglerar mängden blått.)

\Task
Ändra så att mullvaden kan springa uppe på gräset också, men se till så att ingen tunnel ritas ut där.

\Task
Skriv om loopen i \code{Graphics.waitForKey} så att den använder \code{do while} i stället. Vilken variant tycker du är lättast att förstå?

\Task
Låt mullvaden fortsätta gräva även om man inte trycker ned någon tangent. Tangenttryckning ska ändra riktningen.

\Subtask
Skapa en ny metod \code{Graphics.waitForKeyNonBlocking} som möjliggör tangentbordsavläsning som ej blockerar exekveringen enligt nedan:

\begin{Code}
  def waitForKeyNonBlocking(): Char  = {
    w.waitForEvent(10) //wait max 10 milliseconds
    if (w.getEventType() == SimpleWindow.KEY_EVENT) w.getKey else 0
  }
\end{Code}

\Subtask
Lägg till en ny metod \code{Graphics.delay} som ska göra det möjligt att hindra blockmullvaden från att springa alltför fort:
\begin{Code}
def delay(millis: Int): Unit = SimpleWindow.delay(millis)
\end{Code}


\Subtask
Skapa en ny metod \code{Graphics.keepOnDigging} som från början är en kopia av metoden \code{dig}. Gör följande tillägg/ändringar:
\begin{enumerate}[nolistsep,noitemsep]

\item Lägg till två variabler \code{var dx} och \code{var dy} i början, som ska hålla reda på riktningen som sköldpaddan gräver. Initialisera dem till \code{0} respektive {1}.

\item Lägg in en fördröjning på 200 millisekunder i den oändliga loopen.

\item Kolla efter knapptryckning enligt nedan kodskellett. Fyll i de saknade delarna så att blockmullvaden rör sig ett steg i rätt riktning i varje looprunda.
\begin{Code}
      val key = Graphics.waitForKeyNonBlocking()
      if      (key == 'w') { dy = -1; dx = 0 }
      else if (key == 'a') { ??? }
      else if (key == 's') { ??? }
      else if (key == 'd') { ??? }
      y += ???
      x += ???
\end{Code}

\item Anpassa fördröjningen efter din förmåga att hinna styra blockmullvaden.

\end{enumerate}
