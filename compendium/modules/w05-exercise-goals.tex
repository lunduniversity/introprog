%!TEX encoding = UTF-8 Unicode
%!TEX root = ../compendium2.tex

\item Kunna implementera funktioner som tar argumentsekvenser av godtycklig längd.
\item Kunna tolka enkla sekvensalgoritmer i pseudokod och implementera dem i programkod, t.ex. tillägg i slutet, insättning, borttagning, omvändning, etc., både genom kopiering till ny sekvens och genom förändring på plats i befintlig sekvens.
\item Kunna använda föränderliga och oföränderliga sekvenser.
\item Förstå skillnaden mellan om sekvenser är föränderliga och om innehållet i sekvenser är föränderligt.
\item Kunna välja när det är lämpligt att använda \code{Vector}, \code{Array} och \code{ArrayBuffer}.
\item Känna till att klassen \code{Array} har färdiga metoder för kopiering.
\item Kunna implementera algoritmer som registrerar antalet förekomster av något utfall i en sekvens som indexeras med utfallet.
\item Kunna generera sekvenser av pseudoslumptal med specificerat slumptalsfrö.
\item Kunna implementera sekvensalgoritmer i Java med \jcode{for}-sats och primitiva arrayer.
\item Kunna beskriva skillnaden i syntax mellan arrayer i Scala och Java.
\item Kunna använda klassen \code{java.util.Scanner} i Scala och Java för att läsa in heltalssekvenser från \code{System.in}.
