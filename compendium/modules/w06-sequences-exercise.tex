
%!TEX encoding = UTF-8 Unicode
%!TEX root = ../exercises.tex

\ifPreSolution


\Exercise{\ExeWeekSIX}\label{exe:W06}

\begin{Goals}
\input{modules/w06-sequences-exercise-goals.tex}
\end{Goals}

\begin{Preparations}
\item \StudyTheory{06}
\end{Preparations}

\else

\ExerciseSolution{\ExeWeekSIX}

\fi


\BasicTasks %%%%%%%%%%%



\WHAT{Para ihop begrepp med beskrivning.}

\QUESTBEGIN

\Task \what

\vspace{1em}\noindent Koppla varje begrepp med den (förenklade) beskrivning som passar bäst:

\begin{ConceptConnections}
  element & 1 & & A & algoritm som ordnar element i en viss ordning \\ 
  samling & 2 & & B & algoritm som letar upp element enligt sökkriterium \\ 
  samlingsbibliotek & 3 & & C & algoritm som räknar element med vissa egenskaper \\ 
  sekvens(samling) & 4 & & D & definierar hur element av en viss typ ska ordnas \\ 
  sekvensalgoritm & 5 & & E & noll el. flera element av samma typ i viss ordning \\ 
  ordning & 6 & & F & hur exekveringstiden växer med problemstorleken \\ 
  sortering & 7 & & G & lösning på problem som drar nytta av sekvenssamling \\ 
  söking & 8 & & H & objekt i en datastruktur \\ 
  linjärsökning & 9 & & I & hur minnesåtgången växer med problemstorleken \\ 
  registrering & 10 & & J & sökalgoritm som letar i sekvens tills element hittas \\ 
  tidskomplexitet & 11 & & K & många färdiga samlingar med olika egenskaper \\ 
  minneskomplexitet & 12 & & L & datastruktur med element av samma typ \\ 
\end{ConceptConnections}

\SOLUTION

\TaskSolved \what

\begin{ConceptConnections}
  samlingsbibliotek & 1 & ~~\Large$\leadsto$~~ &  A & många färdiga datastrukturer med olika egenskaper \\ 
  sekvenssamling & 2 & ~~\Large$\leadsto$~~ &  C & noll el. flera element av samma typ i viss ordning \\ 
  sekvensalgoritm & 3 & ~~\Large$\leadsto$~~ &  H & lösning på problem som drar nytta av sekvenser \\ 
  ordning & 4 & ~~\Large$\leadsto$~~ &  D & beskriver hur element av en viss typ ska ordnas \\ 
  sortering & 5 & ~~\Large$\leadsto$~~ &  E & algoritm som ordnar element i en viss ordning \\ 
  söking & 6 & ~~\Large$\leadsto$~~ &  F & algoritm som leta upp element enligt sökkriterium \\ 
  registrering & 7 & ~~\Large$\leadsto$~~ &  B & algoritm som räknar element med vissa egenskaper \\ 
  varargs & 8 & ~~\Large$\leadsto$~~ &  G & variabelt antal argument, asterisk efter parametertyp \\ 
\end{ConceptConnections}

\QUESTEND

\clearpage

\WHAT{Olika sekvenssamlingar.}

\QUESTBEGIN

\Task \what~Koppla varje sekvenssamling med den (förenklade) beskrivning som passar bäst:

\begin{ConceptConnections}
  \code|Vector     | & 1 & & A & oföränderlig, ger snabbt godtyckligt ändrad samling \\ 
  \code|List       | & 2 & & B & förändringsbar, snabb att ändra i början \\ 
  \code|Array      | & 3 & & C & förändringsbar, snabb indexering, kan ändra storlek \\ 
  \code|ArrayBuffer| & 4 & & D & primitiv, förändringsbar, snabb indexering, fix storlek \\ 
  \code|ListBuffer | & 5 & & E & oföränderlig, ger snabbt ny samling ändrad i början \\ 
\end{ConceptConnections}

\SOLUTION

\TaskSolved \what

\begin{ConceptConnections}
  \code|Vector     | & 1 & ~~\Large$\leadsto$~~ &  B & oföränderlig, ger snabbt godtyckligt ändrad samling \\ 
  \code|List       | & 2 & ~~\Large$\leadsto$~~ &  A & oföränderlig, ger snabbt ny samling ändrad i början \\ 
  \code|Array      | & 3 & ~~\Large$\leadsto$~~ &  D & primitiv, förändringsbar, snabb indexering, fix storlek \\ 
  \code|ArrayBuffer| & 4 & ~~\Large$\leadsto$~~ &  E & förändringsbar, snabb indexering, kan ändra storlek \\ 
  \code|ListBuffer | & 5 & ~~\Large$\leadsto$~~ &  C & förändringsbar, snabb att ändra i början \\ 
\end{ConceptConnections}

\QUESTEND




\WHAT{Typer i hierarkin av sekvenssamlingar.}

\QUESTBEGIN

\Task \what~Koppla varje typ i hierarkin av sekvenssamling med den (förenklade) beskrivning som passar bäst:

\begin{ConceptConnections}
  Traversable & 1 & & A & är traverserbar med hjälp av metoden \code|iterator| \\ 
  Iterable & 2 & & B & bastyp för alla sekvenssamlingar, indexposition från 0 \\ 
  Seq & 3 & & C & bastyp för alla samlingar, har metoden \code|foreach| \\ 
\end{ConceptConnections}

\SOLUTION

\TaskSolved \what

\begin{ConceptConnections}
  Traversable & 1 & ~~\Large$\leadsto$~~ &  B & bastyp för alla samlingar, har metoden \code|foreach| \\ 
  Iterable & 2 & ~~\Large$\leadsto$~~ &  C & är traverserbar med hjälp av metoden \code|iterator| \\ 
  Seq & 3 & ~~\Large$\leadsto$~~ &  A & bastyp för alla sekvenssamlingar, indexposition från 0 \\ 
\end{ConceptConnections}

\QUESTEND


\WHAT{Använda sekvenssamlingar.}

\QUESTBEGIN

\Task \what~Antag att nedan variabler finns synliga i aktuell namnrymd:
\begin{Code}
val xs: Vector[Int] = Vector(1, 2, 3)
val x: Int = 0
\end{Code}

\Subtask Koppla varje uttryck till vänster med motsvarande resultat till höger. Om du är osäker på resultatet, läs i snabbreferensen och testa i REPL. \\\emph{Tips: ''colon on the collection side''}.

\begin{ConceptConnections}
  \code|x +: xs         | & 1 & & A & \code|Vector(1, 2, 3)                         | \\ 
  \code|xs +: x         | & 2 & & B & \code|(0 until 3)                             | \\ 
  \code|xs :+ x         | & 3 & & C & \code|1                                       | \\ 
  \code|xs ++ xs        | & 4 & & D & \code|true                                    | \\ 
  \code|xs.indices      | & 5 & & E & \code|2                                       | \\ 
  \code|xs apply 0      | & 6 & & F & \code|Vector(1, 2, 3, 0)                      | \\ 
  \code|xs(3)           | & 7 & & G & \code|error: value +: is not a member of Int  | \\ 
  \code|xs.length       | & 8 & & H & \code|Vector(0, 1, 2, 3)                      | \\ 
  \code|xs.take(4)      | & 9 & & I & \code|3                                       | \\ 
  \code|xs.drop(2)      | & 10 & & J & \code|Vector(2, 2, 3)                         | \\ 
  \code|xs.updated(0, 2)| & 11 & & K & \code|false                                   | \\ 
  \code|xs.tail.head    | & 12 & & L & \code|error: value tail is not a member of Int| \\ 
  \code|xs.head.tail    | & 13 & & M & \code|java.lang.IndexOutOfBoundsException     | \\ 
  \code|xs.isEmpty      | & 14 & & N & \code|Vector(1, 2, 3, 1, 2, 3)                | \\ 
  \code|xs.nonEmpty     | & 15 & & O & \code|Vector(3)                               | \\ 
\end{ConceptConnections}

\Subtask Vid tre tillfällen blir det fel. Varför? Är det kompileringsfel eller exekveringsfel?



\begin{framed}
\noindent\emph{Tips inför fortsättningen:}
Scalas standardbibliotek har många användbara samlingar med enhetlig metoduppsättning. Om du lär dig de viktigaste samlingsmetoderna får du en kraftfull verktygslåda. Läs mer här:

    \begin{itemize}%[nolistsep]
      \item snabbreferensen (enda tentahjälpmedel): \\{\small\url{http://cs.lth.se/pgk/quickref}}
      \item översikt (av Prof. Martin Odersky, uppfinnare av Scala, m.fl.): \\
       {\small\url{http://docs.scala-lang.org/overviews/collections/introduction.html}}
      \item api-dokumentation:\\  {\small\url{https://www.scala-lang.org/api/current/scala/collection/}}
    \end{itemize}
\end{framed}



\SOLUTION

\TaskSolved \what

\SubtaskSolved

\begin{ConceptConnections}
  \code|x +: xs         | & 1 & ~~\Large$\leadsto$~~ &  H & \code|Vector(0, 1, 2, 3)                      | \\ 
  \code|xs +: x         | & 2 & ~~\Large$\leadsto$~~ &  G & \code|error: value +: is not a member of Int  | \\ 
  \code|xs :+ x         | & 3 & ~~\Large$\leadsto$~~ &  F & \code|Vector(1, 2, 3, 0)                      | \\ 
  \code|xs ++ xs        | & 4 & ~~\Large$\leadsto$~~ &  N & \code|Vector(1, 2, 3, 1, 2, 3)                | \\ 
  \code|xs.indices      | & 5 & ~~\Large$\leadsto$~~ &  B & \code|(0 until 3)                             | \\ 
  \code|xs apply 0      | & 6 & ~~\Large$\leadsto$~~ &  C & \code|1                                       | \\ 
  \code|xs(3)           | & 7 & ~~\Large$\leadsto$~~ &  M & \code|java.lang.IndexOutOfBoundsException     | \\ 
  \code|xs.length       | & 8 & ~~\Large$\leadsto$~~ &  I & \code|3                                       | \\ 
  \code|xs.take(4)      | & 9 & ~~\Large$\leadsto$~~ &  A & \code|Vector(1, 2, 3)                         | \\ 
  \code|xs.drop(2)      | & 10 & ~~\Large$\leadsto$~~ &  O & \code|Vector(3)                               | \\ 
  \code|xs.updated(0, 2)| & 11 & ~~\Large$\leadsto$~~ &  J & \code|Vector(2, 2, 3)                         | \\ 
  \code|xs.tail.head    | & 12 & ~~\Large$\leadsto$~~ &  E & \code|2                                       | \\ 
  \code|xs.head.tail    | & 13 & ~~\Large$\leadsto$~~ &  L & \code|error: value tail is not a member of Int| \\ 
  \code|xs.isEmpty      | & 14 & ~~\Large$\leadsto$~~ &  K & \code|false                                   | \\ 
  \code|xs.nonEmpty     | & 15 & ~~\Large$\leadsto$~~ &  D & \code|true                                    | \\ 
\end{ConceptConnections}

\SubtaskSolved

\noindent\renewcommand*{\arraystretch}{1.2}\begin{tabular}{p{5cm} l p{6cm}}

~\\ \emph{fel} & \emph{typ} & \emph{förklaring} \\\hline

\code|error: value +: is| \code|not a member of Int|
& kompileringsfel
& Operatorer som slutar med kolon är högerassociativa. Metodanropet \code|xs +: x| motsvarar med punktnotation \code|x.+:(xs)| och det finns ingen metod med namnet \code|+:| på heltal.\\\hline

\code|IndexOutOfBoundsException|
& körtidsfel & Det finns bara 3 element och index räknas från 0 i sekvenssamlingar.\\\hline

\code|error: value tail is | \code|not a member of Int|
& kompileringsfel
& Metoden \code|head| ger första elementet och heltal saknar sekvenssamlingsmetoden \code|tail|.\\\hline

\end{tabular}


\QUESTEND


\WHAT{Kopiering av sekvenser.}

\QUESTBEGIN

\Task \what~ %\code{map} \code{toArray} \code{copyToArray}
Klassen \code{Mutant} nedan kan användas för att skapa instanser med förändringsbara heltal.\footnote{Om den inbyggda klassen Int, i likhet med \code{Mutant}, knasigt nog  kunnat användas för att skapa förändringsbara instanser hade heltalsmatematiken i Scala omvandlats till ett skrämmande kaos.}

\noindent\begin{minipage}{0.6\textwidth}
\begin{Code}[basicstyle=\ttfamily\large\selectfont]
class Mutant(var int: Int = 0)
\end{Code}
\end{minipage}
\hfill\begin{minipage}{0.36\textwidth}
%https://www.1001freedownloads.com/free-clipart/mutant
\centering\includegraphics[width=3.6cm]{../img/mutant.png}
\captionof{figure}{En instans av klassen Mutant där \code{int} kanske är 5.}
%https://tex.stackexchange.com/questions/55337/how-to-use-figure-inside-a-minipage
\end{minipage}

\vspace{0.5em}\noindent Kör detta i REPL:
\begin{REPL}
scala> val xs = Vector.fill(3)(new Mutant)
scala> val ys = xs.toArray    // kopierar referenserna till ny Array
scala> val zs = xs.map(x => new Mutant(x.int)) // djupkopiera till ny Vector
scala> xs(0).int = 5
\end{REPL}
\Subtask Fyll i tabellen nedan genom att till höger skriva värdet av varje uttryck till vänster. Förklara vad som händer. \emph{Tips:} Metoden \code{eq} jämför alltid referenser (ej innehåll).

\renewcommand{\arraystretch}{2.0}
\vspace{1em}\noindent\begin{tabular}{@{} l | p{5.5cm}}\hline
\code|xs(0)         | & \\\hline
\code|ys(0).int| & \\\hline
\code|zs(0).int| & \\\hline
\code|xs(0) eq ys(0)| & \\\hline
\code|xs(0) eq zs(0)| & \\\hline
\code|(ys.toBuffer :+ new Mutant).apply(0).int| & \\\hline
\end{tabular}

\Subtask Implementera med hjälp av en \code{while}-sats funktionen \code{deepCopy} nedan som gör \emph{djup} kopiering, d.v.s skapar en ny array med nya, innehållskopierade mutanter.
\begin{Code}
def deepCopy(xs: Array[Mutant]): Array[Mutant] = ???
\end{Code}
Använd denna algoritm:

\begin{algorithm}[H]
 \SetKwInOut{Input}{Indata}\SetKwInOut{Output}{Resultat}

 \Input{ ~En mutantarray $xs$}
 \Output{ ~En djup kopia av $xs$}
 $result \leftarrow$ en ny mutantarray med plats för lika många element som i $xs$\\
 $i \leftarrow 0$  \\
 \While{$i$ mindre än antalet element}{
  skapa en kopia av elementet $xs(i)$ och lägg kopian i $result$ på platsen $i$ \\
  öka $i$ med 1
 }
 \Return $result$
\end{algorithm}

\Subtask Testa att din funktion och kolla så att inga läskiga muteringar genom delade referenser går att göra, så som med \code|xs| och \code|ys| i första deluppgiften.

\Subtask Är det vanligt att man, för säkerhets skull, gör djupkopiering av alla element i oföränderliga samlingar som enbart innehåller oföränderliga element?

\SOLUTION

\TaskSolved \what~

\SubtaskSolved

\renewcommand{\arraystretch}{1.5}
\vspace{1em}\noindent\begin{tabular}{@{} p{0.4\textwidth} p{0.6\textwidth}}\hline
\code|xs(0)| & \code|Mutant@5ab7ef8a | varje instans har egen hexkod \\ \hline
\code|ys(0).int               | & \code|5 | eftersom \code|ys| innehåller samma instanser som \code|xs|\\ \hline
\code|zs(0).int               | & \code|0 | eftersom \code|!(xs(0) eq zs(0))| \\ \hline
\code|xs(0) eq ys(0)          | & \code|true |  eftersom samma instanser \\ \hline
\code|xs(0) eq zs(0)          | & \code|false | eftersom olika instanser\\ \hline
\code|(ys.toBuffer :+ |
\code|  new Mutant).apply(0).int| & \code|5 | eftersom kopian av typen \code|ArrayBuffer| har samma instans på första platsen som både \code|ys| och \code|xs|\\ \hline
\end{tabular}

\noindent Observera alltså att kopiering med \code{toArray}, \code{toVector}, \code{toBuffer}, etc. samt kopiering med \code{clone} \emph{inte är djup}, d.v.s det är bara elementreferenserna som förs över till ombandlad eller klonad samling och inga nya element skapas.

\SubtaskSolved
\begin{CodeSmall}
def deepCopy(xs: Array[Mutant]): Array[Mutant] = {
  val result = Array.ofDim[Mutant](xs.length) //fylld med null-referenser
  var i = 0
  while (i < xs.length) {
    result(i) = new Mutant(xs(i).int) //kopia med samma innehåll på samma plats
    i += 1
  }
  result
}
\end{CodeSmall}
Det går också bra att skapa resultatarrayen med \code{new Array[Mutant](xs.length)}.
Du kan också använda \code{size} i stället för \code{length}.

\SubtaskSolved
\begin{REPL}
scala> class Mutant(var int: Int = 0)
defined class Mutant

scala> :pa
def deepCopy(xs: Array[Mutant]): Array[Mutant] = {
  val result = Array.ofDim[Mutant](xs.length)
  var i = 0
  while (i < xs.length) {
    result(i) = new Mutant(xs(i).int)
    i += 1
  }
  result
}
scala> val xs = Array.fill(3)(new Mutant)
xs: Array[Mutant] = Array(Mutant@46a123e4, Mutant@44bc2449, Mutant@3c28e5b6)

scala> val ys = deepCopy(xs)
ys: Array[Mutant] = Array(Mutant@14b8a751, Mutant@7345f97d, Mutant@554566a8)

scala> xs(0).int = 5
xs(0).int: Int = 5

scala> ys(0).int
res0: Int = 0
\end{REPL}

\SubtaskSolved Nej, eftersom elementen inte kan förändras kan man utan problem dela referenser mellan samlingar. Det finns inte någon möjlighet att det kan ske förändringar som påverkar flera samlingar samtidigt.
Dock gör man vanligen (ofta tidsödande) djupkopieringar av samlingar med förändringsbara element för att kunna vara säker på att den ursprungliga samlingen inte förändras.

\QUESTEND



\ifPreSolution
\begin{framed}
\noindent\emph{Tips inför fortsättningen:} Ofta kan du lösa grundläggande delproblem med inbyggda samlingsmetoder ur standardbiblioteket. Till exempel kan ju kopieringen i \code{deepCopy} i föregående uppgift enkelt göras med hjälp av samlingsmetoden \code{map}.

Men det är mycket bra för din förståelse om du kan implementera grundläggande sekvensalgoritmer själv även om det normalt är bättre att använda färdiga, vältestade  metoder. I kommande uppgifter ska du därför göra egna implementationer av några sekvensalgoritmer som redan finns i standardbiblioteket.
\end{framed}
\fi



\WHAT{Uppdatering av sekvenser.}

\QUESTBEGIN

\Task \what~Deklarera dessa variabler i REPL:

\begin{Code}
val xs = (1 to 4).toVector
val buf = xs.toBuffer
\end{Code}

\Subtask Uttrycken till vänster evalueras uppifrån och ned. Para ihop med rätt resultat.

\begin{ConceptConnections}
  \code|{ buf(0) = -1; buf(0) }   | & 1 & & A & \code|-1| \\ 
  \code|{ xs(0) = -1; xs(0) }| & 2 & & B & \code|Vector(1, 2, 3, 4)| \\ 
  \code|buf.update(1, 5)          | & 3 & & C & {\small\code|error: value update is not a member|} \\ 
  \code|xs.updated(0, 5)          | & 4 & & D & \code|(): Unit| \\ 
  \code|{ buf += 5; buf }         | & 5 & & E & \code|Vector(5, 2, 3, 4)| \\ 
  \code|{ xs += 5; xs }         | & 6 & & F & \code|Vector(1, -1, 5)| \\ 
  \code|xs.patch(1,Vector(-1,5),3)| & 7 & & G & \code|ArrayBuffer(1, 5, 3, 4, 5)| \\ 
  \code|xs                        | & 8 & & H & {\small\code|error: value += is not a member|} \\ 
\end{ConceptConnections}
\emph{Tips:} Läs om metoderna i snabbreferensen och undersök i REPL. Exempel:
\begin{REPL}
scala> Vector(1,2,3,4).patch(from = 1, patch = Vector(0,0), replaced = 3)
res0: scala.collection.immutable.Vector[Int] = Vector(1, 0, 0)
\end{REPL}

\Subtask Implementera funktionen \code{insert} nedan med hjälp av sekvenssamlingsmetoden \code{patch}. \emph{Tips:} Ge argumentet \code{0} till parametern \code{replaced}.
\begin{Code}
/** Skapar kopia av xs men med elem insatt på plats pos. */
def insert(xs: Array[Int], elem: Int, pos: Int): Array[Int] = ???
\end{Code}

\Subtask Skriv pseduokod för en algoritm som implementerar \code{insert} med hjälp av \code{while}.

\Subtask Implementera \code{insert} enligt din pseudokod. Testa i REPL och se vad som händer om \code{pos} är negativ? Vad händer om \code{pos} är precis ett steg bortom sista platsen i \code{xs}? Vad händer om \code{pos} är flera steg bortom sista platsen?

\SOLUTION

\TaskSolved \what~

\SubtaskSolved

\begin{ConceptConnections}
  \code|{ buf(0) = -1; buf(0) }   | & 1 & ~~\Large$\leadsto$~~ &  H & \code|-1| \\ 
  \code|{ xs(0) = -1; xs(0) }| & 2 & ~~\Large$\leadsto$~~ &  B & {\small\code|error: value update is not a member|} \\ 
  \code|buf.update(1, 5)          | & 3 & ~~\Large$\leadsto$~~ &  G & \code|(): Unit| \\ 
  \code|xs.updated(0, 5)          | & 4 & ~~\Large$\leadsto$~~ &  E & \code|Vector(5, 2, 3, 4)| \\ 
  \code|{ buf += 5; buf }         | & 5 & ~~\Large$\leadsto$~~ &  D & \code|ArrayBuffer(1, 5, 3, 4, 5)| \\ 
  \code|{ xs += 5; xs }         | & 6 & ~~\Large$\leadsto$~~ &  C & {\small\code|error: value += is not a member|} \\ 
  \code|xs.patch(1,Vector(-1,5),3)| & 7 & ~~\Large$\leadsto$~~ &  F & \code|Vector(1, -1, 5)| \\ 
  \code|xs                        | & 8 & ~~\Large$\leadsto$~~ &  A & \code|Vector(1, 2, 3, 4)| \\ 
\end{ConceptConnections}

\SubtaskSolved

\begin{Code}
def insert(xs: Array[Int], elem: Int, pos: Int): Array[Int] =
  xs.patch(from = pos, patch = Array(elem), replaced = 0)
\end{Code}

\SubtaskSolved Pseudokoden nedan är skriven så att den kompilerar fast den är ofärdig.
\begin{Code}
def insert(xs: Array[Int], elem: Int, pos: Int): Array[Int] = {
  val result = ??? /* ny array med plats för ett element mer än i xs */
  var i = 0
  while(???){/* kopiera elementen före plats pos och öka i */}
  if (i < result.length) /* lägg elem i result på plats i */
  while(???){/* kopiera över resten */}
  result
}
\end{Code}

\SubtaskSolved
\begin{Code}
def insert(xs: Array[Int], elem: Int, pos: Int): Array[Int] = {
  val result = new Array[Int](xs.length + 1)
  var i = 0
  while (i < pos && i < xs.length) { result(i) = xs(i); i += 1}
  if (i < result.length) { result(i) = elem; i += 1 }
  while (i < result.length && i > 0) { result(i) = xs(i - 1); i += 1}
  result
}
\end{Code}
\begin{REPL}
scala> insert(Array(1,2),0,pos = -1)
res2: Array[Int] = Array(0, 1, 2)

scala> insert(Array(1,2),0,pos = 0)
res3: Array[Int] = Array(0, 1, 2)

scala> insert(Array(1,2),0,pos = 1)
res4: Array[Int] = Array(1, 0, 2)

scala> insert(Array(1,2),0,pos = 2)
res5: Array[Int] = Array(1, 2, 0)

scala> insert(Array(1,2),0,pos = 42)
res7: Array[Int] = Array(1, 2, 0)
\end{REPL}

\QUESTEND




\ifPreSolution
\begin{framed}
\noindent\emph{Tips inför fortsättningen:} Det är inte lätt att få rätt på alla specialfall även i små algoritmer så som \code{insert} ovan. Det är därför viktigt att noga tänka igenom sin sekvensalgoritm med avseende på typiska special. Använd denna checklista:
\begin{enumerate}[noitemsep]
  \item Vad händer om sekvensen är tom?
  \item Fungerar det för exakt ett element?
  \item Kan index bli negativt?
  \item Kan index bli mer än längden minus ett?
  \item Kan det bli en oändlig loop, t.ex. p.g.a. saknad loopvariabeluppräkning?
\end{enumerate}
Ibland vill man att vettiga undantag ska kastas vid ogiltig indata eller andra feltillstånd och då är \code{require} eller \code{assert} bra att använda. I andra fall vill man att resultatet t.ex. ska bli en tom sekvenssamling om indata är ogiltigt. Sådana beteenden behöver dokumenteras så att andra som använder dina algoritmer (eller du själv efter att du glömt hur det var) förstår vad som händer i olika fall.


\end{framed}
\fi




\WHAT{Linjärsökning enligt olika sökkriterier.}

\QUESTBEGIN

\Task \what~Linjärsökning innebär att man letar tills man hittar det man söker efter i en sekvens. Detta delproblem återkommer ofta! Vanligen börjar linjärsökning från början och håller på tills man hittar något element som uppfyller kriteriet. Beroende på vad som finns i sekvensen och hur kriteriet ser ut kan det hända att man måste gå igenom alla element utan hitta det som eftersöks.

\Subtask Linjärsökning med inbyggda sekvenssamlingsmetoder.
\begin{Code}
val xs = ((1 to 5).reverse ++ (0 to 5)).toVector
\end{Code}
Deklarera ovan variabel i REPL och para ihop uttrycken nedan med rätt värden. Förklara vad som händer.

\begin{ConceptConnections}
  \code|xs.indexOf(0)        | & 1 & & A & \code|4| \\ 
  \code|xs.indexOf(6)        | & 2 & & B & \code|5| \\ 
  \code|xs.indexWhere(_ < 2) | & 3 & & C & \code|Vector(4, 6)| \\ 
  \code|xs.indexWhere(_ != 5)| & 4 & & D & \code|1| \\ 
  \code|xs.find(_ == 1)      | & 5 & & E & \code|Vector(1, 1)| \\ 
  \code|xs.find(_ == 6)      | & 6 & & F & \code|-1| \\ 
  \code|xs.filter(_ == 1)    | & 7 & & G & \code|Vector(1, 0, 1)| \\ 
  \code|xs.filterNot(_ > 1)  | & 8 & & H & \code|None| \\ 
  \code|xs.zipWithIndex.filter(_._1 == 1).map(_._2)| & 9 & & I & \code|Some(1)| \\ 
\end{ConceptConnections}

\Subtask Implementera linjärsökning i strängvektor med strängpredikat.
\begin{Code}
/** Returns first index where p is true. Returns -1 if not found. */
def indexOf(xs: Vector[String], p: String => Boolean): Int = ???
\end{Code}
Ett strängpredikat \code{p: String => Boolean} är en funktion som tar en sträng som indata och ger ett booleskt värde som resultat. Implementera \code{indexOf} med hjälp av en \code{while}-sats. Du kan t.ex. använda en lokal boolesk variabel \code{found} för att hålla reda på om du har hittat det som eftersöks enligt predikatet.

När element som uppfyller predikatet saknas måste man bestämma vad som ska hända. Kravet på din implementation i detta fall ges av dokumentationskommentaren ovan.

Din funktion ska fungera enligt nedan:
\begin{REPL}
scala> val xs = Vector("hej","på","dej")

scala> indexOf(xs, _.contains('p'))
res0: Int = 1

scala> indexOf(xs, _.contains('q'))
res1: Int = -1

scala> indexOf(Vector(), _.contains('q'))
res2: Int = -1

scala> indexOf(Vector("q"), _.length == 1)
res3: Int = 0
\end{REPL}

\SOLUTION

\TaskSolved \what~

\SubtaskSolved

\begin{ConceptConnections}
  \code|xs.indexOf(0)        | & 1 & ~~\Large$\leadsto$~~ &  B & \code|5| \\ 
  \code|xs.indexOf(6)        | & 2 & ~~\Large$\leadsto$~~ &  A & \code|-1| \\ 
  \code|xs.indexWhere(_ < 2) | & 3 & ~~\Large$\leadsto$~~ &  D & \code|4| \\ 
  \code|xs.indexWhere(_ != 5)| & 4 & ~~\Large$\leadsto$~~ &  F & \code|1| \\ 
  \code|xs.find(_ == 1)      | & 5 & ~~\Large$\leadsto$~~ &  I & \code|Some(1)| \\ 
  \code|xs.find(_ == 6)      | & 6 & ~~\Large$\leadsto$~~ &  C & \code|None| \\ 
  \code|xs.contains(0)       | & 7 & ~~\Large$\leadsto$~~ &  G & \code|true| \\ 
  \code|xs.filter(_ == 1)    | & 8 & ~~\Large$\leadsto$~~ &  E & \code|Vector(1, 1)| \\ 
  \code|xs.filterNot(_ > 1)  | & 9 & ~~\Large$\leadsto$~~ &  H & \code|Vector(1, 0, 1)| \\ 
  \code|xs.zipWithIndex.filter(_._1 == 1).map(_._2)| & 10 & ~~\Large$\leadsto$~~ &  J & \code|Vector(4, 6)| \\ 
\end{ConceptConnections}

\SubtaskSolved Med en boolesk variabel \code{found}:

\begin{Code}
def indexOf(xs: Vector[String], p: String => Boolean): Int = {
  var found = false
  var i = 0
  while (i < xs.length && !found) {
      found = p(xs(i))
      i += 1
  }
  if (found) i - 1 else -1
}
\end{Code}
Eller utan \code{found}:
\begin{Code}
def indexOf(xs: Vector[String], p: String => Boolean): Int = {
  var i = 0
  while (i < xs.length && !p(xs(i))) i += 1
  if (i == xs.length) -1 else i
}
\end{Code}
Eller så kan man börja bakifrån. Lösningen nedan är nog enklare att fatta (?) och definitivt mer koncis, men uppfyller inte kravet att returnera index för \emph{första} förekomsten som det står i uppgiften. Men om vi relaxar kravet till att vår funktion ska returnera \emph{något} index där predikatet gäller så funkar nedan fint:
\begin{Code}
def indexOf(xs: Vector[String], p: String => Boolean): Int = {
  var i = xs.length - 1
  while (i >= 0 && !p(xs(i))) i -= 1
  i
}
\end{Code}
Eller så kan man göra på flera andra sätt. När du ska implementera algoritmer, både på programmeringstentan och i yrkeslivet som systemutvecklare, finns det ofta många olika sätt att lösa uppgiften på som har olika egenskaper, fördelar och nackdelar. Det viktiga är att lösningen fungerar så gott det går enligt kraven, att koden är begriplig för människor och att implementationen inte är så ineffektiv att användarna tröttnar i sin väntan på resultatet...

\QUESTEND




\WHAT{Labbförberedelse: Implementera heltalsregistrering i Array.}

\QUESTBEGIN

\Task \what~Registrering innebär att man räknar antalet förekomster av olika värden. Varje gång ett nytt värde förekommer behöver vi räkna upp en frekvensräknare. Det behövs en räknare för varje värde som ska registreras. Vi ska fortsätta räkna ända tills alla värden är registrerade.

På veckans laboration ska du registrera förekomsten av olika kortkombinationer i kortspelet poker. I denna övning ska du som träning inför laborationen lösa ett liknande registreringsproblem:  frekvensanalys av många tärningskast. Vid tärningsregistrering behövs sex olika räknare. Man kan med fördel då använda en sekvenssamling med plats för sex heltal. Man kan t.ex. låta  plats \code{0} håller reda på antalet ettor, plats \code{1} hålla reda på antalet tvåor, etc.

\Subtask Implementera nedan algoritm enligt pseudokoden:
\begin{Code}
def registreraTärningskast(xs: Seq[Int]): Vector[Int] = {
  val result = ??? /* Array med 6 nollor */
  xs.foreach{ x =>
    require(x >= 1 && x <= 6, "tärninskast ska vara mellan 1 & 6")
    ??? /* räkna förekomsten av x */
  }
  result.toVector
}
\end{Code}

\Subtask Använd funktionen \code{kasta} nedan när du testar din registreringsalgoritm med en sekvenssamling innehållande minst $1000$ tärningskast.
\begin{Code}
def kasta(n: Int) = Vector.fill(n)(util.Random.nextInt(6) + 1)
\end{Code}

\SOLUTION

\TaskSolved \what~

\SubtaskSolved
\begin{Code}
def registreraTärningskast(xs: Seq[Int]): Vector[Int] = {
  val result = Array.fill(6)(0)
  xs.foreach{ x =>
    require(x >= 1 && x <= 6, "tärninskast ska vara mellan 1 & 6")
    result(x - 1) += 1
  }
  result.toVector
}
\end{Code}

\SubtaskSolved
\begin{REPL}
scala> registreraTärningskast(kasta(1000))
res0: Vector[Int] = Vector(171, 163, 166, 152, 184, 164)

scala> registreraTärningskast(kasta(1000))
res1: Vector[Int] = Vector(163, 161, 158, 174, 161, 183)
\end{REPL}

\QUESTEND




\WHAT{Inbyggda metoder för sortering.}

\QUESTBEGIN

\Task \what~Det finns fler olika sätt att ordna sekvenser efter olika kriterier. För  grundtyperna \code{Int}, \code{Double}, \code{String}, etc., finns inbyggda ordningar som gör att sekvenssamlingsmetoden \code{sorted} fungerar utan vidare argument (om du är nöjd med den inbyggda ordningsdefinitionen). Det finns också metoderna \code{sortBy} och \code{sortWith} om du vill ordna en sekvens med element av någon grundtyp efter egna ordningsdefinitioner eller om du har egna klasser i din sekvens.
\begin{Code}
val xs = Vector(1,2,1,3,-1)
val ys = Vector("abra", "ka", "dabra").map(_.reverse)
val zs = Vector('a', 'A', 'b', 'c').sorted

case class Person(förnamn: String, efternamn: String)

val ps = Vector(Person("Kim","Ung"), Person("kamrat", "Clementin"))
\end{Code}
Deklarera ovan i REPL och para ihop uttryck nedan med rätt resultat.
\\\emph{Tips:} Stora bokstäver sorteras före små bokstäver i den inbyggda ordningen för grundtyperna \code{String} och \code{Char}. Dessutom har svenska tecken knasig ordning.\footnote{En kvarleva från föråldrade teckenkodningsstandarder:    \url{https://sv.wikipedia.org/wiki/ASCII}}
\\Läs om sorteringsmetoderna i snabbreferensen och prova i REPL.

\begin{ConceptConnections}
  \code|'a' < 'A'                  | & 1 & & A & \code|3| \\ 
  \code|"AÄÖö" < "AÅÖö"        | & 2 & & B & \code|false| \\ 
  \code|xs.sorted.head             | & 3 & & C & \code|0| \\ 
  \code|xs.sorted.reverse.head     | & 4 & & D & \code|1| \\ 
  \code|ys.sorted.head             | & 5 & & E & \code|-1| \\ 
  \code|zs.indexOf('a')            | & 6 & & F & \code|"ak"| \\ 
  \code|ps.sorted.head.förnamn.take(2)| & 7 & & G & \code|"ka"| \\ 
  \code|ps.sortBy(_.förnamn).apply(1).förnamn.take(2)| & 8 & & H & \code|error: ...| \\ 
  \code|xs.sortWith((x1,x2) => x1 > x2).indexOf(3)| & 9 & & I & \code|true| \\ 
\end{ConceptConnections}
Vi ska senare i kursen implementera egna sorteringsalgoritmer som träning, men i normala fall använder man inbyggda sorteringar som är effektiva och vältestade. Dock är det inte ovanligt att man vill definiera egna ordningar för egna klasser, vilket vi ska undersöka senare i kursen.

\SOLUTION

\TaskSolved \what

\begin{ConceptConnections}
  \code|'a' < 'A'                  | & 1 & ~~\Large$\leadsto$~~ &  C & \code|false| \\ 
  \code|"AÄÖö" < "AÅÖö"        | & 2 & ~~\Large$\leadsto$~~ &  A & \code|true| \\ 
  \code|xs.sorted.head             | & 3 & ~~\Large$\leadsto$~~ &  I & \code|-1| \\ 
  \code|xs.sorted.reverse.head     | & 4 & ~~\Large$\leadsto$~~ &  E & \code|3| \\ 
  \code|ys.sorted.head             | & 5 & ~~\Large$\leadsto$~~ &  G & \code|"ak"| \\ 
  \code|zs.indexOf('a')            | & 6 & ~~\Large$\leadsto$~~ &  B & \code|1| \\ 
  \code|ps.sorted.head.förnamn.take(2)| & 7 & ~~\Large$\leadsto$~~ &  H & \code|error: ...| \\ 
  \code|ps.sortBy(_.förnamn).apply(1).förnamn.take(2)| & 8 & ~~\Large$\leadsto$~~ &  F & \code|"ka"| \\ 
  \code|xs.sortWith((x1,x2) => x1 > x2).indexOf(3)| & 9 & ~~\Large$\leadsto$~~ &  D & \code|0| \\ 
\end{ConceptConnections}
Det blir fel i uttrycket ovan som försöker sortera en sekvens med instanser av \code{Person} direkt med metoden \code{sorted}:
\begin{REPL}
scala> ps.sorted
<console>:13: error: No implicit Ordering defined for Person.
\end{REPL}
Det blir fel eftersom kompilatorn inte hittar någon ordningsdefinition för dina egna klasser. Senare i kursen ska vi se hur vi kan skapa egna ordningar om man vill få \code{sorted} att fungera på sekvenser med instanser av egna klasser, men ofta räcker det fint med \code{sortBy} och \code{sortWith}.
\QUESTEND


\WHAT{Inbyggd metod för blandning.}

\QUESTBEGIN

\Task \what~På veckans laboration ska du implementera en egen blandningsalgoritm och använda den för att blanda en kortlek. Det finns redan en inbygg metod \code{shuffle} i singelobjektet \code{Random} i paketet \code{scala.util}.

\Subtask Sök upp dokumentationen för \code{Random.shuffle} och studera funktionshuvudet. Det står en hel del invecklade saker om \code{CanBuildFrom} etc. Detta smarta krångel, som vi inte går närmare in på i denna kurs, är till för att metoden ska kunna returnera lämplig typ av samling. När du ser ett sådant funktionshuvud kan du anta att metoden fungerar fint med flera olika typer av lämpliga samlingar i Scalas standardbibliotek.

Klicka på \code{shuffle}-dokumentationen så att du ser hela texten. Vad säger dokumentationen om resultatet? Är det blandning på plats eller blandning till ny samling?

\Subtask Prova upprepade blandningar av olika typer av sekvenser med olika typer av element i REPL.

\SOLUTION

\TaskSolved \what~

\SubtaskSolved \code{Random.shuffle} returnerar en ny blandad sekvenssamling av samma typ. Ordningen i den ursprungliga samlingen påverkas inte.

\SubtaskSolved Exempel på användning av \code{random.shuffle}:
\begin{REPL}
scala> import scala.util.Random

scala> val xs = Vector("Sten", "Sax", "Påse")

scala> (1 to 10).foreach(_ => println(Random.shuffle(xs).mkString(" ")))
Sax Påse Sten
Sten Påse Sax
Sten Sax Påse
Sten Sax Påse
Sten Påse Sax
Sten Påse Sax
Sax Sten Påse
Sten Påse Sax
Sax Påse Sten
Sax Påse Sten

scala> scala> (1 to 5).map(_ => Random.shuffle(1 to 6))
res1: IndexedSeq[IndexedSeq[Int]] =
  Vector(Vector(5, 2, 1, 4, 3, 6), Vector(6, 5, 4, 2, 1, 3),
  Vector(3, 1, 4, 6, 5, 2), Vector(3, 2, 6, 5, 1, 4),
  Vector(5, 3, 4, 6, 1, 2))

scala> (1 to 1000).map(_ => Random.shuffle(1 to 6).head).count(_ == 6)
res2: Int = 168
\end{REPL}


\QUESTEND



\clearpage

\ExtraTasks %%%%%%%%%%%%%%%%%%%%%%%%%%%%%%%%%%%%%%%%%%%%%%%%%%%%%%%%%%%%%%%%%%%%



\WHAT{Registrering av booleska värden. Singla slant.}

\QUESTBEGIN

\Task \what~

\Subtask Implementera en funktion som registrerar många slantsinglingar enligt nedan funktionshuvud. Indata är en sekvens av booleska värden där krona kodas som \code{true} och klave kodas som \code{false}. För registreringen ska du använda en lokal \code{Array[Int]}. I resultatet ska antalet utfall av \code{krona} ligga på första platsen i 2-tupeln och på andra platsen ska antalet utfall av \code{klave} ligga.

\begin{Code}
def registerCoinFlips(xs: Seq[Boolean]): (Int, Int) = ???
\end{Code}

\Subtask Skapa en funktion \code{flips(n)} som ger en boolesk \code{Vector} med $n$ stycken slantsinglingar och använd den när du testar din slantsinglingsregistreringsalgoritm.

\SOLUTION

\TaskSolved \what~

\SubtaskSolved
\begin{Code}
def registerCoinFlips(xs: Seq[Boolean]): (Int, Int) = {
  val result = Array.fill(2)(0)
  xs.foreach(x => if (x) result(0) += 1 else result(1) += 1)
  (result(0), result(1))
}
\end{Code}

\SubtaskSolved

\QUESTEND


\WHAT{Kopiering och tillägg på slutet.}

\QUESTBEGIN

\Task \what~
Skapa funktionen \code{copyAppend} som implementerar nedan algoritm, \emph{efter} att du rättat de \textbf{\color{red}{två buggarna}} nedan:

\begin{algorithm}[H]
 \SetKwInOut{Input}{Indata}\SetKwInOut{Output}{Resultat}

 \Input{Heltalsarray $xs$ och heltalet $x$}
 \Output{En ny heltalsarray som som är en kopia av $xs$ men med $x$ tillagt på slutet som extra element.}
 $ys \leftarrow$ en ny array med plats för ett element mer än i $xs$\\
 $i \leftarrow 0$  \\
 \While{$i \leq xs.length$}{
  $ys(i) \leftarrow xs(i)$
 }
lägg $x$ på sista platsen i $ys$
\end{algorithm}

\noindent Granska din kod enligt checklistan i tidigare tipsruta. Testa din funktion för de olika fallen: tom sekvens, sekvens med exakt ett element, sekvens med många element.


\SOLUTION

\TaskSolved \what~

\begin{Code}
def copyAppend(xs: Array[Int], x: Int): Array[Int] = {
  val ys = new Array[Int](xs.length + 1)
  var i = 0
  while(i < xs.length) {
    ys(i) = xs(i)
    i += 1
  }
  ys(xs.length) = x
  ys
}
\end{Code}
De två buggarna i algoritmen finns (1) i villkoret som ska vara strikt mindre än och (2) inne i loopen där uppräkningen av loppvariabeln saknas.

\QUESTEND



% \WHAT{Välja sekvenssamling.}
%
% \QUESTBEGIN
%
% \Task  \what~Vilken sekvenssamling är lämpligast i respektive situation nedan? Välj mellan \code{Vector}, \code{ArrayBuffer} och \code{ListBuffer}.
%
% \Subtask Det asociala mediet ZuckerBok ska lagra statusuppdateringar från sina användare. Dessa lagras i en förändringsbar sekvens där nya poster läggs till först. Indexering mitt i sekvensen är mycket ovanligt eftersom de flesta användarna sällan läser vad andra skriver, utan mest skriver nya inlägg om sig själv.
%
% \Subtask ZuckerBok försöker öka sina intäkter och börjar frenetiskt indexera i kors och tvärs i sekvensen med statusuppdaringar för att söka efter lämpliga spamoffer.
%
% \Subtask ZuckerBok bestämmer sig för att lagra födelsedatum för alla ca $10^7$ medborgare i Sverige i en oföränderlig sekvens för att kunna förmedla specialreklam på födelsedagar.
%
% \SOLUTION
%
% \TaskSolved \what
%
% \SubtaskSolved  \code{ListBuffer} som är snabb på fröändringar i början av sekvensen.
%
% \SubtaskSolved  \code{ArrayBuffer} som är snabb på både storleksförändringar och godtycklig indexering.
%
% \SubtaskSolved  \code{Vector} eftersom ofränderlighet efterfrågas.
%
% \QUESTEND



\WHAT{Kopiera och reversera sekvens.}

\QUESTBEGIN

\Task  \what~  Implementera \code{seqReverseCopy} enligt:

\begin{algorithm}[H]
 \SetKwInOut{Input}{Indata}\SetKwInOut{Output}{Resultat}

 \Input{Heltalsarray $xs$}
 \Output{En ny heltalsarray med elementen i $xs$ i omvänd ordning.}
 $n \leftarrow$ antalet element i $xs$ \\
 $ys \leftarrow$ en ny heltalsarray med plats för $n$ element\\
 $i \leftarrow 0$  \\
 \While{$i < n$}{
  $ys(n - i - 1) \leftarrow xs(i)$ \\
  $i \leftarrow i + 1$
 }
 \Return $ys$
\end{algorithm}

\Subtask Använd en \code{while}-sats på samma sätt som i algoritmen. Prova din implementation i REPL och kolla så att den fungerar i olika fall.

\Subtask Gör en ny implementation som i stället använder en \code{for}-sats som börjar bakifrån. Kör din implementation i REPL och kolla så att den fungerar i olika fall.

\SOLUTION

\TaskSolved \what

\SubtaskSolved  \begin{Code}
def seqReverseCopy(xs: Array[Int]): Array[Int] = {
  val n = xs.length
  val ys = new Array[Int](n)
  var i = 0
  while(i < n) {
    ys(n-i-1) = xs(i)
    i += 1
  }
  ys
}
\end{Code}

\SubtaskSolved  \begin{Code}
def seqReverseCopy(xs: Array[Int]): Array[Int] = {
  val n = xs.length
  val ys = new Array[Int](n)
  for(i <- n - 1 to 0 by -1) ys(n - i - 1) = xs(i)
  ys
}
\end{Code}


\QUESTEND




\WHAT{Kopiera alla utom ett.}

\QUESTBEGIN

\Task  \what~  Implementera kopiering av en array \emph{utom} ett element på en viss angiven plats.
Skriv först pseudokod innan du implementerar:
\begin{Code}
def removeCopy(xs: Array[Int], pos: Int): Array[Int]
\end{Code}

\SOLUTION


\TaskSolved \what

\begin{algorithm}[H]
 \SetKwInOut{Input}{Indata}\SetKwInOut{Output}{Resultat}

 \Input{En sekvens $xs$ av typen \texttt{Array[Int]} och $pos$}
 \Output{En ny sekvens av typen \texttt{Array[Int]} som är en kopia av $xs$ fast med elementet på plats $pos$ borttaget}
 $n \leftarrow$ antalet element $xs$\\
 $ys \leftarrow$ en ny \texttt{Array[Int]} med plats för $n-1$ element \\
 \For{$i \leftarrow 0$ \KwTo $pos - 1$}{
  $ys(i) \leftarrow xs(i)$
 }
 $ys(pos) \leftarrow x$ \\
 \For{$i \leftarrow pos+1$ \KwTo $n - 1$}{
  $ys(i - 1) \leftarrow xs(i)$
 }
 \Return $ys$
\end{algorithm}

\begin{Code}
def removeCopy(xs: Array[Int], pos: Int): Array[Int] = {
  val n = xs.size
  val ys = Array.fill(n - 1)(0)
  for (i <- 0 until pos) ys(i) = xs(i)
  for (i <- pos+1 until n) ys(i - 1) = xs(i)
  ys
}
\end{Code}

\QUESTEND




\WHAT{Borttagning på plats i array.}

\QUESTBEGIN

\Task  \what~  Ibland vill man ta bort ett element på en viss position i en array utan att kopiera element. Ett sätt att göra detta är att flytta alla efterföljande element ett steg mot lägre index och fylla ut sista positionen med ett utfyllnadsvärde, t.ex. $0$.
Skriv först pseudokod. Implementera sedan algoritmen i en funktion med denna signatur:
\begin{Code}
def removeAndPad(xs: Array[Int], pos: Int, pad: Int = 0): Unit
\end{Code}

\SOLUTION

\TaskSolved \what

\begin{algorithm}[H]
 \SetKwInOut{Input}{Indata}\SetKwInOut{Output}{Resultat}

 \Input{En sekvens $xs$ av typen \texttt{Array[Int]}, en position $pos$ och ett utfyllnadsvärde $pad$}
 \Output{En uppdaterad sekvens av $xs$ där elementet på plats $pos$ tagits bort och efterföljande element flyttas ett steg mot lägre index med ett sista elementet som tilldelats värdet av $pad$}
 $n \leftarrow$ antalet element $xs$\\
 \For{$i \leftarrow pos+1$ \KwTo $n - 1$}{
  $xs(i - 1) \leftarrow xs(i)$
 }
 $xs(n - 1) \leftarrow pad$ \\
\end{algorithm}

\begin{Code}
def remove(xs: Array[Int], pos: Int, pad: Int = 0): Unit = {
  val n = xs.size
  for (i <- pos+1 until n) xs(i - 1) = xs(i)
  xs(n-1) = pad
}
\end{Code}

\QUESTEND




\WHAT{Kopiering och insättning.}

\QUESTBEGIN

\Task  \what~

\Subtask Implementera en funktion med detta huvud enligt efterföljande algoritm:
\begin{Code}
def insertCopy(xs: Array[Int], x: Int, pos: Int): Array[Int]
\end{Code}


\begin{algorithm}[H]
 \SetKwInOut{Input}{Indata}\SetKwInOut{Output}{Resultat}

 \Input{En sekvens $xs$ av typen \texttt{Array[Int]} och heltalen $x$ och $pos$}
 \Output{En ny sekvens av typen \texttt{Array[Int]} som är en kopia av $xs$ men där $x$ är infogat på plats $pos$}
 $n \leftarrow$ antalet element $xs$\\
 $ys \leftarrow$ en ny \texttt{Array[Int]} med plats för $n+1$ element \\
 \For{$i \leftarrow 0$ \KwTo $pos - 1$}{
  $ys(i) \leftarrow xs(i)$
 }
 $ys(pos) \leftarrow x$ \\
 \For{$i \leftarrow pos$ \KwTo $n - 1$}{
  $ys(i + 1) \leftarrow xs(i)$
 }
 \Return $ys$
\end{algorithm}


\Subtask Vad måste \code{pos} vara för att det ska fungera med en tom array som argument?

\Subtask Vad händer om din funktion anropas med ett negativt argument för \code{pos}?

\Subtask Vad händer om din funktion anropas med \code{pos} lika med \code{xs.size}?

\Subtask Vad händer om din funktion anropas med \code{pos} större än \code{xs.size}?

\SOLUTION

\TaskSolved \what

\SubtaskSolved  \begin{Code}
def insertCopy(xs: Array[Int], x: Int, pos: Int): Array[Int] = {
  val n = xs.size
  val ys = Array.ofDim[Int](n + 1)
  for (i <- 0 until pos) ys(i) = xs(i)
  ys(pos) = x
  for (i <- pos until n) ys(i + 1) = xs(i)
  ys
}
\end{Code}

\SubtaskSolved  \code{pos} måste vara \code{0}.

\SubtaskSolved  \begin{REPL}
java.lang.ArrayIndexOutOfBoundsException: -1
\end{REPL}

\SubtaskSolved  Elementet \code{x} läggs till på slutet av arrayen, alltså kommer den returnerande arrayen vara större än den som skickades in.

\SubtaskSolved  \begin{REPL}
java.lang.ArrayIndexOutOfBoundsException: 5
\end{REPL}
Man får \code{ArrayIndexOutOfBoundsException} då indexeringen är utanför storleken hos arrayen.

\QUESTEND




\WHAT{Insättning i array.}

\QUESTBEGIN

\Task  \what~  Ett sätt att implementera insättning i en array är att alla elementen efter \code{pos} flyttas fram ett steg och att sista elementet ''försvinner''.
Skriv först algoritmen i pseudokod och implementera sedan enligt:
\begin{Code}
def insertDrop(xs: Array[Int], x: Int, pos: Int): Unit
\end{Code}

\SOLUTION

\TaskSolved \what

\begin{algorithm}[H]
 \SetKwInOut{Input}{Indata}\SetKwInOut{Output}{Resultat}

 \Input{En sekvens $xs$ av typen \texttt{Array[Int]} och heltalen $x$ och $pos$}
 \Output{En uppdaterad sekvens av $xs$ där elementet $x$ har satts in på platsen $pos$ och efterföljande element flyttas ett steg där sista elementet försvinner}
 $n \leftarrow$ antalet element i $xs$\\
 $ys \leftarrow$ en klon av $xs$\\
 $xs(pos) \leftarrow x$\\
 \For{$i \leftarrow pos+1$ \KwTo $n - 1$}{
  $xs(i) \leftarrow ys(i - 1)$
 }
\end{algorithm}

\begin{Code}
def insertDrop(xs: Array[Int], x: Int, pos: Int): Unit = {
  val n = xs.size
  val ys = xs.clone
  xs(pos) = x
  for (i <- pos + 1 until n) xs(i) = ys(i - 1)
}
\end{Code}

\QUESTEND




\clearpage

\AdvancedTasks %%%%%%%%%%%%%%%%%%%%%%%%%%%%%%%%%%%%%%%%%%%%%%%%%%%%%%%%%%%%%%%%%



\WHAT{Repeterade parametrar.}

\QUESTBEGIN

\Task  \what~  Det går att deklarera en funktion som tar en argumentsekvens av godtycklig längd, ä.k. \emph{varargs}. Syntaxen består av en asterisk \code{*} efter typen. Funktion sägs ha repeterade parametrar \Eng{repeated parameters}. I funktionskroppen får man tillgång till argumenten i en sekvenssamling. Argumenten anges godtyckligt många med komma emellan. Exempel:
\begin{Code}
/** Ger en vektor med stränglängder för godtyckligt antal strängar. */
def stringSizes(xs: String*): Vector[Int] = xs.map(_.size).toVector
\end{Code}

\Subtask Deklarera och använd \code{stringSizes} i REPL. Vad händer om du anropar \code{stringSizes} med en tom argumentlista?

\Subtask Det händer ibland att man redan har en sekvenssamling, t.ex. \code{xs}, och vill skicka med varje element som argument till en varargs-funktion. Syntaxen för detta är \code{xs: _* } vilket gör att kompilatorn omvandlar sekvenssamlingen till en argumentsekvens av rätt typ.

Prova denna syntax genom att ge en \code{xs} av typen \code{Vector[String]} som argument till \code{stringSizes}. Fungerar det även om \code{xs} är en sekvens av längden 0?

\SOLUTION

\TaskSolved \what

\SubtaskSolved

\begin{REPL}
scala> def stringSizes(xs: String*): Vector[Int] = xs.map(_.size).toVector
scala> stringSizes("hej")
res83: Vector[Int] = Vector(3)

scala> stringSizes("hej", "på", "dej", "")
res84: Vector[Int] = Vector(3, 2, 3, 0)

scala> stringSizes()
res85: Vector[Int] = Vector()
\end{REPL}

\noindent Anrop med tom argumentlista ger en tom heltalssekvens.

\SubtaskSolved

\begin{REPL}
scala> val xs = Vector("hej","på","dej", "")

scala> stringSizes(xs: _*)
res86: Vector[Int] = Vector(3, 2, 3, 0)

scala> stringSizes(Vector(): _*)
res87: Vector[Int] = Vector()
\end{REPL}
Ja, det funkar fint med tom sekvens.



\QUESTEND



\WHAT{Fibonacci-sekvens med ListBuffer.}

\QUESTBEGIN

\Task  \what~ Samlingen \code{ListBuffer} är en förändrinsbar sekvens som är snabb och minnessnål vid tillägg i början \Eng{prepend}. Undersök vad som händer här:
\begin{REPL}
scala> val xs = scala.collection.mutable.ListBuffer.empty[Int]
scala> xs.prepend(1, 1)
scala> while (xs.head < 100) {xs.prepend(xs.take(2).sum); println(xs)}
scala> xs.reverse.toList
\end{REPL}
Talen i sekvensen som produceras på rad 4 ovan kallas Fibonaccital\footnote{\href{https://sv.wikipedia.org/wiki/Fibonaccital}{sv.wikipedia.org/wiki/Fibonaccital}} och blir snabbt mycket stora.

\Subtask Definera och testa följande funktion. Den ska internt använda förändringsbara \code{ListBuffer} men returnera en sekvens av oföränderliga \code{List}.

\begin{Code}
/** Ger en lista med tal ur Fibonacci-sekvensen 1, 1, 2, 3, 5, 8 ...
  * där det största talet är mindre än max. */
def fib(max: Long): List[Long]  = ???
\end{Code}


\Subtask
Hur lång ska en Fibonacci-sekvens vara för att det sista elementet ska vara så nära \code{Int.MaxValue} som möjligt?

\SOLUTION


\TaskSolved \what


\SubtaskSolved

\begin{Code}
def fib(max: Long): List[Long] = {
  val xs = scala.collection.mutable.ListBuffer.empty[Long]
  xs.prepend(1, 1)
  while (xs.head < max) xs.prepend(xs.take(2).sum)
  xs.reverse.drop(1).toList
}

\end{Code}

\SubtaskSolved

\begin{REPL}
scala> fib(Int.MaxValue).size
res0: Int = 46
\end{REPL}

\QUESTEND



\WHAT{Omvända sekvens på plats.}

\QUESTBEGIN

\Task \what~Implementera nedan algoritm i funktionen \code{reverseChars} och testa så att den fungerar för olika fall i REPL.


\begin{algorithm}[H]
 \SetKwInOut{Input}{Indata}\SetKwInOut{Output}{Resultat}

 \Input{En array $xs$ med tecken}
 \Output{Samma array med tecknen i omvänd ordning}
 $n \leftarrow$ antalet element i $xs$\\
 \For{$i \leftarrow 0$ \KwTo $\frac{n}{2} - 1$}{
  $temp \leftarrow xs(i)$ \\
  $xs(i) \leftarrow xs(n - i - 1)$ \\
  $xs(n - i - 1) \leftarrow temp$ \\
 }
\end{algorithm}

\SOLUTION

\TaskSolved \what~
\begin{Code}
def reverseChars(xs: Array[Char]): Unit = {
  val n = xs.length
  for (i <- 0 to n / 2 - 1) {
    val temp = xs(i)
    xs(i) = xs(n - i - 1)
    xs(n - i - 1) = temp
  }
}
\end{Code}

\QUESTEND






\WHAT{Palindrompredikat.}

\QUESTBEGIN

\Task  \what~ En palindrom\footnote{\url{https://sv.wikipedia.org/wiki/Palindrom}} är ett ord som förblir oförändrat om man läser den baklänges. Exempel på palindromer: kajak, dallassallad.

Ett sätt att implementera ett palindrompredikat visas nedan:
\begin{Code}
def isPalindrome(s: String): Boolean = s == s.reverse
\end{Code}

\Subtask Lösningen kan behöva gå igenom alla tecknen i strängen två gånger och behöver minnesutrymme för dubbla antalet tecken. Varför?

\Subtask Skapa ett palindromtest som går igenom elementen max en gång och som inte behöver extra minnesutrymme för en kopia av strängen. \emph{Lösningsidé:} Jämför parvis första och sista, näst första och näst sista, etc.

\SOLUTION

\TaskSolved \what

\SubtaskSolved Omvändning med \code{reverse} kan kräva genomgång av hela strängen en gång samt minnesutrymme för kopian. Innehållstestet kräver ytterligare en traversering. (Detta är i och för sig inget stort problem eftersom världens längsta palindrom inte är längre än 19 bokstäver.)

\SubtaskSolved

\begin{Code}
def isPalindrome(s: String): Boolean = {
  val n = s.length
  var foundDiff = false
  var i = 0
  while (i < n / 2 && !foundDiff)  {
    foundDiff = s(i) != s(n - i - 1)
    i += 1
  }
  !foundDiff
}
\end{Code}

\QUESTEND




\WHAT{Array och \code{for}-sats i Java.}

\QUESTBEGIN

\Task  \what~Ladda ner programet nedan från kursens github-repo: \href{https://raw.githubusercontent.com/lunduniversity/introprog/master/compendium/examples/DiceReg.java}{\texttt{compendium/examples/DiceReg.java}}


\Subtask
Kompilera med \code{javac DiceReg.java} och kör med \code{java DiceReg 10000 42} och förklara vad som händer.

\javainputlisting{examples/DiceReg.java}

\Subtask Beskriv skillnaderna mellan Scala och Java, vad gäller syntaxen för array och \code{for}-sats. Beskriv några andra skillnader mellan språken som syns i programmet ovan.

\Subtask Ändra i programmet ovan så att loop-variabeln \code{i} skrivs ut i varje runda i varje \code{for}-sats. Kompilera om och kör.

\Subtask Skriv om programmet ovan genom att abstrahera huvudprogrammets delar till de statiska metoderna \code{parseArguments}, \code{registerPips} och \code{printReg} enligt nedan skelett. Spara programmet i filen \code{DiceReg2.java} och kompilera med \texttt{javac DiceReg2.java} i terminalen.

\begin{Code}[language=Java]
// DiceReg2.java
import java.util.Random;

public class DiceReg2 {
    public static int[] diceReg = new int[6];
    private static Random rnd = new Random();

    public static int parseArguments(String[] args) {
        // ???
        return n;
    }

    public static void registerPips(int n){
        // ???
    }

    public static void printReg() {
        // ???
    }

    public static void main(String[] args) {
        int n = parseArguments(args);
        registerPips(n);
        printReg();
    }
}
\end{Code}

\Subtask Starta Scala REPL i samma katalog som filen \texttt{DiceReg2.class} ligger i och kör nedan 7 rader i REPL och förklara vad som händer:
\begin{REPL}
scala> DiceReg2.main(Array("1000","42"))
scala> DiceReg2.diceReg
scala> DiceReg2.registerPips(1000)
scala> DiceReg2.printReg
scala> DiceReg2.registerPips(1000)
scala> DiceReg2.printReg
scala> DiceReg2.rnd
\end{REPL}

\SOLUTION

\TaskSolved \what

\SubtaskSolved Programmet simulerar 10000 tärningskast (med slumptalsfrö 42) och skriver ut förekomsten av respektive tärningskast.

\begin{REPL}
Rolling the dice 10000 times with seed 42
Number of 1's: 1654
Number of 2's: 1715
Number of 3's: 1677
Number of 4's: 1629
Number of 5's: 1643
Number of 6's: 1682
\end{REPL}

\SubtaskSolved  I Java används hakparenteser medan Scala har ''vanliga'' parenteser. En array i scala deklareras så här: \\
 \code{val scalaArray = Array.ofDim[Int](6)} \\
 vilket i java motsvarar: \code{int[] javaArray = new int[6];}

\code{for}-sats i scala skrivs: \code|for(i <- 0 to n) {...}| medan i java skrivs: \code|for (int i = 0; i < n; i++) { ... }|.

I java måste semikolon skrivas efter varje sats och typen måste anges explicit vid varje variabeldeklaration.

I scala behövs inte semikolon (förutom för att separera satser på samma rad) och typer kan ofta härledas i Scala av kompilatorn och behöver inte alltid skrivas explicit.

\SubtaskSolved  Lägg till \code{System.out.println(i);} i for-looparna

\SubtaskSolved  \begin{Code}[language=Java]
// DiceReg2.java
import java.util.Random;
public class DiceReg2{
	public static int[] diceReg = new int[6];
	private static Random rnd = new Random();

	public static int parseArguments(String[] args){
		int n = 100;
		if(args.length > 0) {
			n = Integer.parseInt(args[0]);
		}
		if(args.length > 1) {
			int seed = Integer.parseInt(args[1]);
			rnd.setSeed(seed);
		}
		return n;
	}

	public static void registerPips(int n) {
		for(int i = 0; i<n; i++) {
			int pips = rnd.nextInt(6);
			diceReg[pips]++;
		}
	}

	public static void main(String[] args) {
		int n = parseArguments(args);
		registerPips(n);
		printReg();
	}
}
\end{Code}

\SubtaskSolved

\begin{REPL}
  // Skriver ut förekomsten av 1000 tärningskast med slumptalsfrö 42.
Number of 1's: 165
Number of 2's: 163
Number of 3's: 178
Number of 4's: 183
Number of 5's: 156
Number of 6's: 155

  // Skriver ut diceReg-attributet
res1: Array[Int] = Array(165, 163, 178, 183, 156, 155)

  // Skriver ut diceReg-attributet efter 1000 till kast.
res2: Array[Int] = Array(329, 325, 349, 360, 324, 313)

  // Skriver ut diceReg-attributet efter 1000 till kast.
res3: Array[Int] = Array(498, 484, 531, 513, 485, 489)

  // Det blir kompileringsfel då attributet rnd är privat
<console>:11: error: value rnd is not a member of object DiceReg2
	DiceReg2.rnd
				    ^
\end{REPL}

\QUESTEND





\WHAT{Läsa in tal med \code{java.util.Scanner}.}

\QUESTBEGIN

\Task  \what~  Med \jcode{new Scanner(System.in)} skapas ett objekt som kan läsa in tal från teckensträngar som användaren skriver i terminalfönstret, så som visas i Java-programmet nedan:

\javainputlisting{examples/DiceScanBuggy.java}
Ladda ner programmet   \href{https://raw.githubusercontent.com/lunduniversity/introprog/master/compendium/examples/DiceReg.java}{\texttt{compendium/examples/DiceScanBuggy.java}}
och kompilera och kör med indatasekvensen \texttt{1 2 3 4 -1} och notera hur registreringen sker.

\Subtask Läs dokumentationen av \code{java.util.Scanner}. Vad gör \jcode{hasNextInt()} och \jcode{nextInt()}?


\Subtask Programmet fungerar inte som det ska. Du behöver korrigera 3 saker för att programmet ska göra rätt. Rätta buggarna och spara det rättade programmet som \texttt{DiceScan.java}. Kompilera och testa det rättade programmet.

\SOLUTION

\TaskSolved \what

\SubtaskSolved

\code{hasNextInt()} kollar om det finns ett till tal och returnerar \code{true}/\code{false}. \code{nextInt()} läser nästa tal.
Se \url{https://docs.oracle.com/javase/7/docs/api/java/util/Scanner.html#hasNextInt%28%29} och \url{https://docs.oracle.com/javase/7/docs/api/java/util/Scanner.html#nextInt%28%29 }.

\SubtaskSolved

\begin{Code}[language=Java,numbers=left]
import java.util.Random;
import java.util.Scanner;

public class DiceScanBuggy {
	public static int[] diceReg = new int[6];
	public static Scanner scan = new Scanner(System.in);

	public static void registerPips() {
		System.out.println("Enter pips separated by blanks: ");
		System.out.println("End with -1 and <Enter>.");
		boolean isPips = true;
		while(isPips && scan.hasNextInt()){
			int pips = scan.nextInt();
			if(pips >= 1 && pips <= 6) {
				diceReg[pips-1]++;
			} else {
				isPips = false;
			}
		}
	}

	public static void printReg(){
		for(int i = 1; i<7; i++) {
		System.out.println("Number of " + i + "'s: " + diceReg[i-1]);
		}
	}

	public static void main(String[] args) {
		registerPips();
		printReg();
	}
}
\end{Code}

\QUESTEND
