%!TEX encoding = UTF-8 Unicode
%!TEX root = ../compendium2.tex

\item Kunna läsa och skriva pseudokod för sekvensalgoritmer och implementera sekvensalgoritmer enligt pseudokod.

\item Kunna implementera sekvensalgoritmer, både genom kopiering till ny sekvens och genom förändring på plats i befintlig sekvens.

\item Kunna använda inbyggda metoder för uppdatering av, linjärsökning i, och sortering av sekvenssamlingar.

\item Kunna beskriva skillnaden i användningen av föränderliga och oföränderliga sekvenser, speciellt vid uppdatering.

\item Förstå hur sorteringsordningen är definierad för strängar.

\item Kunna sortera sekvenssamlingar innehållande objekt av grundtyper med hjälp av inbyggda och egendefinierade sorteringsordningar med metoderna \code{sorted}, \code{sortBy} och \code{sortWith}.

\item Kunna implementera linjärsökning enligt olika sökkriterier.


\item Kunna beskriva egenskaperna hos sekvenssamlingarna \code{Vector}, \code{List}, \code{Array}, \code{ArrayBuffer} och \code{ListBuffer}.

\item Förstå bieffekter av uppdatering av delade referenser till föränderliga element.

\item Kunna använda funktioner med repeterade parametrar.

\item Känna till hur man implementerar funktioner med repeterade parametrar.

\item Kunna implementera heltalsregistrering i en heltalsarray.

%\item Kunna beskriva skillnader i syntax mellan arrayer i Scala och Java.

%\item Kunna beskriva skillnader i syntax och semantik mellan enkla for-satser i Scala och Java.


%\item Känna till hur klassen \code{java.util.Scanner} kan användas för att skapa heltalssekvenser ur strängsekvenser.
