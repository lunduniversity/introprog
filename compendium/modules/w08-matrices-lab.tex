%!TEX encoding = UTF-8 Unicode
%!TEX root = ../compendium2.tex

\Lab{\LabWeekEIGHT}

\begin{Goals}
\item Kunna skapa och använda matriser.
\item Kunna iterera över matriser med nästlade for-loopar.
\item Träna på algoritmkonstruktion.
\item Använda en integrerad utvecklingsmiljö (IDE).
\end{Goals}

\begin{Preparations}
\item Gör övning {\tt \ExeWeekEIGHT} i kapitel \ref{chapter:W08}, speciellt övning \ref{exe:matrices:labprep}.

\item Läs igenom hela laborationen och studera den befintliga koden i \TODO \code{workspace}.

\item Läs appendix \ref{appendix:ide} och välj vilken IDE du ska använda (ScalaIDE/Eclipse eller IntelliJ IDEA). Säkerställ att du får igång en av dessa utvecklingsmijöer genom att köra hello-world-exemplet och sedan ladda ner och importera kursens \TODO workspace enligt instruktionerna i appendix \ref{appendix:ide}.
\end{Preparations}

\subsection{Bakgrund}



\subsection{Obligatoriska krav}



\subsection{Valbara krav -- välj minst ett}
