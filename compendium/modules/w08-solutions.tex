%!TEX encoding = UTF-8 Unicode

%!TEX root = ../compendium.tex

\ExerciseSolution{\ExeWeekEIGHT}


\Task

\Subtask Beroende på första bokstaven i din favoritgrönsak får du olika svar såsom \textit{gurka är gott!} vid första bokstaven $g$.\\
Javas \jcode{switch}-sats testar den första bokstaven på favoritgrönsaken genom att stegvis jämföra den med \jcode{case}-uttrycken. Om första bokstaven \jcode{firstChar} matchar bokstaven efter ett \jcode{case} körs koden efter kolonet till \jcode{switch}-satsens slut eller tills ett \jcode{break} avbryter \jcode{switch}-satsen.\\
Matchar inte \jcode{firstChar} något \jcode{case} så finns även \jcode{default}, som körs oavsett vilken första bokstaven är, ett generellt fall.

\Subtask Om \jcode{case 't'} körs kommer både  \textit{tomat är gott!} och \textit{broccoli är gott!} skrivas ut, man säger att koden $"$faller igenom$"$. Utan \jcode{break}-satsen i Java körs koden i efterkommande \jcode{case} tills ett \jcode{break} avbryter exekveringen eller \jcode{switch}-satsen tar slut.
 

\Task 

\Subtask Svaret blir identiskt mot föregående uppgiften i Java.\\
Scalas \code{match}-uttryck fungerar väldigt likt Javas \jcode{switch}. Den jämför stegvis värdet med varje \code{case} för att sedan returnera ett värde tillhörande motsvarande \code{case}.

\Subtask scala.MatchError (of class java.lang.Character). Körtidsfel, uppstår av en viss input under körningen.

\Subtask Scalas \code{match} ersätter kolonet (:) i \jcode{switch} med Scalas högerpil (=>).\\
\code{match} returnerar ett värde till skillnad från \jcode{switch} som inte returnerar något.
\code{match} kan inte $"$falla igenom$"$ så ett \jcode{break} efter varje \jcode{case} är inte nödvändigt.


\Task 
\\
Garden som införts vid \code{case 'g'} slumpar fram ett tal mellan 0 och 1 och om talet inte är större än $0.5$ så blir det ingen matchning med \code{case 'g'} och programmet testar vidare till grenen för default-värden.\\
Gardens krav måste uppfyllas för att det ska matcha som vanligt.


\Task

\Subtask G100true. Vid byte av plats: Gtrue100.\\
\code{match} testar om kompanjonsobjektet \code{Gurka} är av typen \code{Gurka} med två parametervärden. De angivna parametrarna tilldelas namn, \code{vikt} får namnet \code{v} och \code{ärRutten} namnet \code{rutten} som sedan skrivs ut. Byts namnen dessa ges skrivs de ut i den omvända ordningen. 

\Subtask \code{Option[(Int, Boolean)]}

\Subtask \code{Some((100, true))}, en \code{Option} med en tupel av parametrarna från g.

\Subtask \code{ärÄtvärd} testar om \code{Grönsak g} är av typen \code{Gurka(v, rutten)} eller \code{Tomat}. Dessa har sedan garder.\\ \code{Gurka} måste ha \code{vikt} över 100 och \code{ärRutten} vara \code{false} för att \code{case Gurka} ska returnera \code{true}.\\ 
\code{Tomat} måste ha \code{vikt} över 50 och \code{ärRutten} vara \code{false} för att \code{case Tomat} ska returnera \code{true}.\\
Matchas inte \code{Grönsak g} med någon av dessa returneras default-värdet \code{false}.


\Task

\Subtask
\begin{Code}
package vegopoly

trait Grönsak {
	def vikt: Int
	def ärRutten: Boolean
	def ärÄtbar: Boolean
}

case class Gurka(vikt: Int, ärRutten: Boolean) extends
	Grönsak { val ärÄtbar: Boolean = (!ärRutten && vikt > 100)}
case class Tomat(vikt: Int, ärRutten: Boolean) extends
	Grönsak { val ärÄtbar: Boolean = (!ärRutten && vikt > 50)}
case class Broccoli(vikt: Int, ärRutten: Boolean) extends
	Grönsak { val ärÄtbar: Boolean = (!ärRutten && vikt > 50)}

object Main{
	def slumpvikt: Int = (math.random*500 + 100).toInt
	def slumprutten: Boolean = math.random > 0.8
	def slumpgurka: Gurka = Gurka(slumpvikt, slumprutten)
	def slumptomat: Tomat = Tomat(slumpvikt, slumprutten)
	def slumpbroccoli: Broccoli = Broccoli(slumpvikt, slumprutten)
	def slumpgrönsak: Grönsak = if (math.random > 0.2) slumpgurka 
		else{
			if (math.random > 0.2) slumptomat 
				else slumpbroccoli}

	def main(args: Array[String]): Unit = {
		val skörd = Vector.fill(args(0).toInt)(slumpgrönsak)
		val ätvärda = skörd.filter(_.ärÄtbar)
		println("Antal skördade grönsaker: " + skörd.size)
		println("Antal ätvärda grönsaker: " + ätvärda.size)
	}
}
\end{Code}

\Subtask 

\Subtask


\Task





