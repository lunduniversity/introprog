%!TEX encoding = UTF-8 Unicode

%!TEX root = ../compendium2.tex

\item Känna till begrepp:
bastyp,
sypertyp,
subtyp,
körtidstyp,
dynamisk bindning,
polymorfism,
trait,
inmixning,
överskuggad medlem,
anonym klass,
skyddad medlem,
abstrakt medlem,
abstrakt klass,
referenstyp,
värdetyp.

\item Kunna deklarera och använda en arvshierarki i flera nivåer.

\item Känna till synlighetsregler vid arv och nyttan med privata och skyddade medlemmar.

\item Kunna deklarera och använda skyddade medlemmar.

\item Kunna deklarera och använda överskuggade medlemmar.

\item Känna till reglerna som gäller vid överskuggning av olika sorters medlemmar.

\item Kunna deklarera och använda en hierarki av klasser där konstruktorparametrar överförs till superklasser.

\item Kunna deklarera och använda uppräknade värden med case-objekt och gemensam bastyp.

\item Kunna deklarera och känna till nyttan med finala klasser och finala attribut och nyckelordet \code{final}.

%TODO KOLLA PÅ NEDAN MÅL OCH BESTÄM HUR DE SKA IN I ÖVNINGARNA

%\item Känna till hur typtester och typkonvertering under körtid kan göras med metoderna \code{isInstanceOf} och \code{asInstanceOf} och känna till att detta görs bättre med \code{match}.

%\item Kunna deklarera och använda inmixning med flera traits och nyckelordet \code{with}.

%\item Kunna referera till medlem i superklassen med referensen \code{super} och känna till när detta nyckel ord behövs.

%\item Känna till begreppet anonym klass.
