%!TEX encoding = UTF-8 Unicode
%!TEX root = ../compendium2.tex

\Teamlab{\LabWeekNINE}

\begin{Goals}
%!TEX encoding = UTF-8 Unicode
%!TEX root = ../compendium2.tex

\item Kunna använda arv.
\item Kunna göra överskuggning av medlemmar i en supertyp med \code{override}.
\item Kunna referera till medlemmar i superklassen med \code{super} vid överskuggning.
\item Kunna förklara begreppet dynamisk bindning.

\end{Goals}

\begin{Preparations}
\item Gör övning {\tt \ExeWeekNINE} i kapitel \ref{exe:W09}, speciellt uppgift \ref{exe:inheritance:labprep-pair}.
\item Läs dokumentationen för \code{introprog.BlockGame}.
\item Läs igenom hela laborationen och förbered dig inför första gruppmötet.
%!TEX encoding = UTF-8 Unicode
%!TEX root = compendium.tex
\item
Diskutera i din samarbetsgrupp hur ni ska dela upp koden mellan er i flera olika delar, som ni kan arbeta med var för sig. En sådan del kan vara en klass, en trait, ett objekt, ett paket, eller en funktion.
\item
Varje del ska ha en \textbf{huvudansvarig} individ.
\item
Arbetsfördelningen ska vara någorlunda jämnt fördelad mellan gruppmedlemmarna.
\item
Den som är huvudansvarig för en viss del redovisar den delen.
\item 
Ni ska ta fram en gruppgemensam checklista för kodgranskning. Alla ska granska minst en annan gruppmedlems kod enligt checklistan. 
\item
Grupplaborationen görs över \textbf{två veckor} uppdelat på två delredovisningar. Vid första redovisningen ska arbetsupplägget och pågående utveckling redovisas. Vid andra tillfället ska de färdig lösningarna presenteras av respektive huvudansvarig individ.
\item
Vid första redovisningen ska du redogöra för handledaren hur ni delat upp koden och vem som är huvudansvarig för vad och vad ditt ansvar omfattar, samt hur ni jobbar praktiskt med att synkronisera er utveckling.
\item Grupplaborationen är en \textbf{extra stor uppgift} och grupparbetet behöver ledtid för att ni ska hinna koordinera er sinsemellan. Du behöver därför planera för att arbeta med något i grupplabben i stort sett varje dag under de tillgängliga veckorna, och vara redo att bidra i diskussioner.

\item Träffas i din samarbetsgrupp och diskutera ert arbetssätt utifrån följande frågeställningar:
\begin{itemize}
  \item Vilken krav ska ni implementera?
  \item Hur ska ni jobba med gemensamma koddelar?
  \item Hur ska ni dela med er av de koddelar som ni utvecklar var för sig?
\end{itemize}

\end{Preparations}

\subsection{Bakgrund}

Spelet \emph{Snake}\footnote{Även kallat ''masken''. \url{https://sv.wikipedia.org/wiki/Snake}}

\TODO screendumps one+two-player game
\subsection{Obligatoriska funktionella krav}

Följande funktionella krav ska uppfyllas av ert program om ni är sex personer i gruppen. Om ni är färre kan ni välja att skippa krav enligt tabell \ref{lab:snak:table-reqt}.
%\footnote{Om någon student, p.g.a. långvarig sjukdom eller annat giltigt skäl, genomför laborationen själv i efterhand som en individuell laboration ska följande krav implementeras på egen hand: \code{Player}, \code{OnePlayerGame}, \code{Snake}, \code{Apple}.}
\begin{itemize}[nosep, label={$\square$},]
\item \textbf{\texttt{Player}}. Det ska finnas spelare som motsvarar mänskliga användare och som har ett namn och fyra knappar som den kan spela med. Varje spelare har en egen orm som den kan styra med sina knappar.

\item \textbf{\texttt{Snake}}. Det ska finnas ormar. En orm består av ett antal block, där det främsta blocket kallas huvud och resten av blocken kallas svans. Huvudet har en ljusare färg än kroppen. Svansens längd ökar under spelets gång. En orm rör sig i en viss riktning och varje spelare kan ändra riktningen på sin orm med sina knappar, i en av fyra riktningar \code{North}, \code{South}, \code{East} eller \code{West}.

\item \textbf{\texttt{Apple}}. Det ska finnas äpple. Ett äpple består av ett rött block och finns på en slumpvis position. Ett äpple kan ätas av en orm om ormens huvud träffar äpplet. Varje gång ett äpple äts upp av en orm så teleporteras äpplet till en ny position och kan ätas igen.

\item \textbf{\texttt{Banana}}. Det ska finnas bananer. En banan består av tre vertikala gula block och finns på en slumpvis position. En banan kan ätas av en orm om ormens huvud träffar bananen. Varje gång en banan äts upp av en orm så teleporteras bananen till en ny position och kan ätas igen.

\item \textbf{\texttt{OneplayerGame}}. Det ska gå att spela ensam. I varianten med en spelare finns en orm och ett äpple. Varje gång användarens orm lyckas äta ett äpple får användaren poäng. När användaren ätit ett visst antal äpplen är spelet slut och poängen visas. Allteftersom tiden går blir svansen, vid jämna tidsintervall, längre.

\item \textbf{\texttt{TwoplayerGame}}. Det ska gå att spela två och två. I varianten med två spelare finns två ormar. Det finnas också flera äpplen och flera bananer. Om en orm äter en banan blir dess svans längre. Om en orm äter ett äpple får dess spelare poäng. Allteftersom tiden går blir båda ormarnas svansar, vid jämna tidsintervall, längre.

\end{itemize}
\begin{table}[H]
  \centering
  \caption{Krav som minst ska implementeras vid respektive gruppstorlek. \label{lab:snak:table-reqt}}

\begin{tabular}{r | c c c c c}
  Krav                   & 2       & 3       & 4       & 5       & 6 \\ \hline
  \texttt{Player}        & $\surd$ & $\surd$ & $\surd$ & $\surd$ & $\surd$ \\
  \texttt{OnePlayerGame} &         & $\surd$ &         & $\surd$ & $\surd$ \\
  \texttt{TwoPlayerGame} & $\surd$ & $\surd$ & $\surd$ & $\surd$ & $\surd$ \\
  \texttt{Snake}         & $\surd$ & $\surd$ & $\surd$ & $\surd$ & $\surd$ \\
  \texttt{Apple}         & $\surd$ &         & $\surd$ & $\surd$ & $\surd$ \\
  \texttt{Banana}        &         &         & $\surd$ &         & $\surd$ \\
\end{tabular}
\end{table}

\subsection{Obligatoriska design-krav}

\begin{enumerate}[label={$\square$}, leftmargin=*]

\item Snake-spel ska gå att starta med huvudprogrammet nedan. Huvudprogrammet får ändras vid behov i enlighet med minimikrav vad gäller gruppstorlek i tabell \ref{lab:snak:table-reqt}, samt valbara extrakrav i avsnitt \ref{lab:snake:extra-reqts}.
\begin{Code}
package snake

object Main {
  def main(args: Array[String]): Unit = {
    val buttons = Seq("One","Two", "Tournament", "Cancel")
    val selected =
      introprog.Dialog.select("Number of players?", buttons, "Snake")
    selected match {
      case "One" => (new OnePlayerGame).play("Pink")
      case "Two" => (new TwoPlayerGame).play("Purple", "Pink")
      case "Tournament" => ??? // valbart krav
      case _ =>
    }
  }
}
\end{Code}

\item Spelet ska bygga vidare på \code{introprog.BlockWindow} enligt typhierarkin i fig.~\ref{snake:fig:game-hierarchy}.

\begin{figure}[H]
\begin{center}
\newcommand{\TextBox}[1]{\raisebox{0pt}[1em][0.5em]{#1}}
\tikzstyle{umlclass}=[rectangle, draw=black,  thick, anchor=north, text width=3cm, rectangle split, rectangle split parts = 3]
\begin{tikzpicture}[inner sep=0.5em,scale=1.0, every node/.style={transform shape}]

  \node [umlclass, rectangle split parts = 1, xshift=0cm, yshift=4.5cm] (BaseType)  {
              \textit{\textbf{\centerline{\TextBox{\code{BlockGame}}}}}
%              \nodepart[align=left]{second}\code{def x: T} \newline \code{def y: T}
          };


  \node [umlclass, rectangle split parts = 1, xshift=0cm, yshift=3.0cm] (SubType)  {
              \textit{\textbf{\centerline{\TextBox{\code{SnakeGame}}}}}
%              \nodepart[align=left]{second}\code{val x: Int} \newline \code{val y: Int}
          };

\node [umlclass, rectangle split parts = 1, xshift=-3cm, yshift=1.0cm] (SubSubType1)  {
            {\textbf{\centerline{\TextBox{\code{OnePlayerGame}}}}}
%            \nodepart[]{second}\TextBox{\code{val dim: Int}}
        };

\node [umlclass, rectangle split parts = 1, xshift=3cm, yshift=1.0cm] (SubSubType2)  {
            {\textbf{\centerline{\TextBox{\code{TwoPlayerGame}}}}}
%            \nodepart[]{second}\TextBox{\code{val dim: Int}}
        };

\draw[umlarrow] (SubType.north) -- ++(0,0.5) -| (BaseType.south);
\draw[umlarrow] (SubSubType1.north) -- ++(0,0.5) -| (SubType.south);
\draw[umlarrow] (SubSubType2.north) -- ++(0,0.5) -| (SubType.south);

\end{tikzpicture}
\end{center}
\caption{Arvshierarki med klassen \code{introprog.BlockGame} som bastyp.}
\label{snake:fig:game-hierarchy}
\end{figure}


\item Ormar och frukt ska utgå från bastypen \code{Entity} enligt typhierarkin i ~\ref{snake:fig:entity-hierarchy}.

\begin{figure}[H]
\begin{center}
\newcommand{\TextBox}[1]{\raisebox{0pt}[1em][0.5em]{#1}}
\tikzstyle{umlclass}=[rectangle, draw=black,  thick, anchor=north, text width=2.5cm, rectangle split, rectangle split parts = 3]
\begin{tikzpicture}[inner sep=0.5em,scale=1.0, every node/.style={transform shape}]

  \node [umlclass, rectangle split parts = 1, xshift=-1.0cm, yshift=4.5cm] (BaseType)  {
              \textit{\textbf{\centerline{\TextBox{\code{Entity}}}}}
%              \nodepart[align=left]{second}\code{def x: T} \newline \code{def y: T}
          };


  \node [umlclass, rectangle split parts = 1, xshift=-3cm, yshift=2.5cm] (SubType1)  {
              \textit{\textbf{\centerline{\TextBox{\code{MovingEntity}}}}}
%              \nodepart[align=left]{second}\code{val x: Int} \newline \code{val y: Int}
          };

\node [umlclass, rectangle split parts = 1, xshift=-4.5cm, yshift=0.5cm] (SubSubType0)  {
            {\textbf{\centerline{\TextBox{\code{Snake}}}}}
%            \nodepart[]{second}\TextBox{\code{val dim: Int}}
};


\node [umlclass, rectangle split parts = 1, xshift=0.75cm, yshift=2.5cm] (SubType2)  {
            \textit{\textbf{\centerline{\TextBox{\code{Fruit}}}}}
%            \nodepart[]{second}\TextBox{\code{val dim: Int}}
        };

\node [umlclass, rectangle split parts = 1, xshift=-1.0cm, yshift=0.5cm] (SubSubType1)  {
            {\textbf{\centerline{\TextBox{\code{Apple}}}}}
%            \nodepart[]{second}\TextBox{\code{val dim: Int}}
        };

\node [umlclass, rectangle split parts = 1, xshift=2.5cm, yshift=0.5cm] (SubSubType2)  {
            {\textbf{\centerline{\TextBox{\code{Banana}}}}}
%            \nodepart[]{second}\TextBox{\code{val dim: Int}}
        };


\draw[umlarrow] (SubType1.north) -- ++(0,0.5) -| (BaseType.south);
\draw[umlarrow] (SubType2.north) -- ++(0,0.5) -| (BaseType.south);
\draw[umlarrow] (SubSubType1.north) -- ++(0,0.5) -| (SubType2.south);
\draw[umlarrow] (SubSubType2.north) -- ++(0,0.5) -| (SubType2.south);
\draw[umlarrow] (SubSubType0.north) -- ++(0,0.5) -| (SubType1.south);

\end{tikzpicture}
\end{center}
\caption{Arvshierarki med klassen \code{Entity} som bastyp.}
\label{snake:fig:entity-hierarchy}
\end{figure}


\item \code{Entity} representerar en varelse i ett spel och ska se ut så här:
\begin{Code}
trait Entity {
  def draw():   Unit
  def erase():  Unit
  def update(): Unit
  def reset():  Unit
}
\end{Code}
Metoderna \code{draw} resp. \code{erase} anropas vid ritning resp. radering. Metoden \code{reset} återställer ursprungstillståndet. Metoden \code{update} anropas en gång i varje runda i spel-loopen.

\item \code{MovingEntity} representerar en spelvarelse som kan röra sig i en viss hastighet och ska se ut så här:
\begin{Code}
trait MovingEntity extends Entity {
  def move(): Unit

  var movesPerSecond: Double = 20.0

  final def millisBetweenMoves(): Int =
    (1000 / movesPerSecond).round.toInt max 1

  private var _timestampLastMove: Long = System.currentTimeMillis
  final def timestampLastMove = _timestampLastMove

  override final def update(): Unit =
    if (System.currentTimeMillis > _
          timestampLastMove + millisBetweenMoves) {
      _timestampLastMove = System.currentTimeMillis
      move()
    }
}
\end{Code}

\item Det ska i \code{SnakeGame} finnas en typhierarki med \code{State} som bastyp som representerar spelets övergripande tillstånd enligt följande:
\begin{Code}
sealed trait State
case object Starting extends State
case object Playing  extends State
case object GameOver extends State
case object Quitting extends State
\end{Code}

\item Vid varje runda i spelloopen ska följande logik exekveras. Denna kod placeras förslagsvis i \code{gameLoopAction}, se vidare \code{SnakeGame} i avsnitt \ref{lab:snake:tips}.
\begin{Code}
if (state == Playing) {
  entities.foreach(_.erase)
  entities.foreach(_.update)
  entities.foreach(_.draw)
  if (isGameOver) enterGameOverState()
}
\end{Code}

\end{enumerate}



\subsection{Valbara krav -- välj minst ett}\label{lab:snake:extra-reqts}

Du ska implementera minst ett (gärna flera) av dessa krav:
\begin{itemize}[nosep, label={$\square$}]
\item Spelare kan ange sitt namn, t.ex. via en dialog eller genom argument till \code{main}.
\item Om en spelare kör in i sin egen orms svans så är spelet förlorat. %(Detta är vanligt i många Snake-spel.)
\item Om en spelare kör utanför spelplanen så är spelet förlorat. %(Detta är vanligt i många Snake-spel.)
\item \textbf{\code{Points}}. Inför ett poängsystem, där poängen beror på t.ex. längden på svansen, antalet svängar, antal uppätna äpplen, etc.
\item \textbf{\code{Highscore}}. Spelet ska visa en lista med de spelare som fått flest poäng.

\item \textbf{\code{TwoPlayerComp extends Competition}}. Två spelare ska kunna tävla i en bäst-av-$n$-matcher-tävling i en sekvens av \code{TwoPlayerGame.play}, där den som vinner flest matcher blir blir totalvinnare.

\item \textbf{\code{SinglePlayerComp extends Competition}}. Flera spelare ska kunna tävla i en-persons-Snake, där den som får flest poäng av $n$ \code{OnePlayerGame}-spel blir totalvinnare.


\item \textbf{\code{Tournament extends Competition}}. Många spelare ska kunna spela en turnering.\footnote{\url{https://en.wikipedia.org/wiki/Tournament}} Namnen på spelarna läses in från en textfil.
\begin{itemize}[nosep, label={$\square$}]
\item \textbf{\code{KnockOut extends Tournament}}. Det ska gå att spela en utslagsturnering, som avslutas med final efter semi-final, etc., beroende på antal spelare.
\item \textbf{\code{RoundRobin extends Tournament}}. Det ska gå att spela en alla-möter-alla-turnering, där alla möjliga par av spelare möts i slumpvis ordning.
\end{itemize}

\end{itemize}


\subsection{Tips och förslag}\label{lab:snake:tips}

Här följer en skiss på den abstrakta klassen \code{SnakeGame} med de abstrakta metoderna \code{isGameOver} och \code{play} som överskuggas i underklasserna \code{OnePlayerGame} och \code{TwoPlayerGame}:
\begin{CodeSmall}
package snake
import introprog.BlockGame

abstract class SnakeGame(title: String) extends BlockGame(
  title,
  dim = (60, 60),
  blockSize = 15,
  background = Colors.black,
  framesPerSecond = 50,
  messageAreaHeight = 3,
) {
  var entities: Vector[Entity] = Vector.empty
  var players: Vector[Player] = Vector.empty
  var state: State = Starting

  def enterStartingState(): Unit = {
    clearWindow
    drawCenteredText("Press SPACE to start!")
    state = Starting
  }

  def enterPlayingState(): Unit = {
    clearWindow
    entities.foreach(_.reset)
    entities.foreach(_.draw)
    state = Playing
  }

  def enterGameOverState(): Unit =  ???

  def enterQuittingState(): Unit =  ???

  override def handleKey(key: String): Unit = ???

  override def onClose(): Unit = ???

  def isGameOver(): Boolean

  override def gameLoopAction(): Unit =

  def startGameLoop(): Unit = {
    pixelWindow.open // möjliggör omstart även om fönstret stängts
    enterStartingState()
    gameLoop(stopWhen = state == Quitting)
  }

  def play(playerNames: String*): Unit
}
\end{CodeSmall}


\begin{enumerate}[leftmargin=*]
\item
\end{enumerate}
