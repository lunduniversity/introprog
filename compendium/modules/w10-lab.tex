%!TEX encoding = UTF-8 Unicode

%!TEX root = ../compendium.tex

\Lab{\LabWeekTEN}

\begin{Goals}
\item Kunna använda inbyggda sorteringsfunktioner
\item Kunna använda inbyggda sökfunktioner.
\item Känna till hur strängar ordnas.
\item Kunna läsa text i tabellform från fil och webbaddress.
\item Kunna använda registrering (frekvensräkning) för enkla statistikberäkningar.
\item Kunna skriva text i tabellform till fil.
\item Kunna läsa kommandon från startargument.
\item Kunna hantera oren (eng. impure) kod. (Exempelvis inläsning från tangenbordet)
\item ... \TODO mer här
\end{Goals}

\begin{Preparations}
\item \StudyTheory{10}
\item \DoExercise{\ExeWeekTEN}{10}
\item \ReadTheLab
\item Fyll i denna enkät: \url{https://goo.gl/forms/hC6JK2UQXVpbGECc2}  \\
I enkäten ska du svarar på frågan: \textit{Vilket är ditt favoritalternativ?} \\
gällande följande områden och svarsalternativen:
\begin{itemize}[nolistsep,noitemsep]
\item \textbf{lthprogram} (D, W, C, E, F, I, Bio, K, L, M, Bme, Nano, V), 
\item \textbf{os} (Win7, Win10, macOS X, Linux, Android, IOS, ChromeOS), 
\item \textbf{editor} (gedit, vim, emacs, vi, notepad++, sublime text, atom)
\item \textbf{ide} (Eclipse, IntelliJ/AndroidStudio, VisualStudio, xcode), 
\item \textbf{socialnät} (facebook, snapchat, linkedin, instagram, github), 
\item \textbf{webbläsare} (firefox, chrome, safari, edge, vivaldi, opera)
\item \textbf{sorteringsalgoritm} (insättningssortering, urvalssortering)
\item \textbf{språk} (Java, Python, PHP, C\#, Javascript, C++, C, Objective-C, R, Swift, Matlab, Ruby, Visual Basic, VBA, Scala, Perl, lua, Delphi)  \\
Listan enligt \url{http://pypl.github.io/PYPL.html} i Aug 2016
% Alternativet är TIOBE, men den är längre...:
%(Java, C, C++, C\#, Python, PHP, Javascript, Visual Basic, Perl, Pascal, Ruby, Swift, Groovy, R, Matlab, SQL, Go, Dart, Fortram, Lua, Ada, Lisp, Scala, Prolog, Haskell, Erlang, Rust)
\end{itemize}
\end{Preparations}


\subsection{Bakgrund}

I den här veckans laboration ska du utveckla ett program som analyserar svar på enkäter med flervalsfrågor. Indata utgörs av text i form av \textbf{kolumnseparerade värden}, där varje persons svar finns på en egen rad och varje svarsrad innehåller svarsalternativ separerade med en \textbf{kolumnseparator} som till exempel kan vara \code{;} eller \code{\t}. Första raden i textfilen anger kolumnernas namn.

Exempel på indatafil: \footnote{\TODO gör indataexemplet lite längre så analysexempel blir lite roligare och låt indatafilen finnas med som testfil i workspace}
\begin{CodeSmall}[language=, ]
program;OS;editor;IDE;socialnät;webbläsare;sorteringsalgoritm;språk
W;Gedit;Eclipse;Facebook;Firefox;insättningssortering;Java
D;Atom;Intellij;GitHub;Vivaldi;urvalssortering;Scala
E;emacs;Eclipse;Snapchat;Edge;urvalssortering;C
\end{CodeSmall}

Ditt program ska innehålla följande delar:
\begin{itemize}
\item En case-klass för strängmatriser som heter \code{Table} med funktioner för inläsning av tabellformatterad text.
\item En case-class för argumentparsning och kommandoexekvering som heter \code{Command}. 
\item Funktioner för att presentera statistik från enkätdata med hjälp av registrering.
\item \TODO mer här ?
\end{itemize}

\TODO workspace upplägg med tree stats

Här är main. Fixa så den funkar.

\scalainputlisting{../workspace/w10_survey/src/main/scala/stats/Main.scala}

\TODO mer här...
\TODO körs från kommandoraden med argument som sedan tolkas om till Command
\TODO Command: tar ett Table, utför sin grejj och returnerar ett Table. Många Commands tolkas som en kedja, där output från det föregående kommandot är input till nästa.


\subsection{Obligatoriska uppgifter}

\Task Implementera klassen \code{Table} och objektet \code{Table} enligt specifikation:
\TODO footnotes?
\TODO kodkommentarer och scalainputlisting

\begin{ScalaSpec}{Table}
package stats

/**
 * A representation of text data in matrix form.
 *
 * @param matrix   The data. Rows in the outer Vector, columns in the inner.
 * @param headings Column headings.
 * @param sep      The String character that separates the columns.
 */
case class Table(matrix: Vector[Vector[String]], headings: Vector[String], sep: String) {

  /**
   * Returns the width (columns) and height (rows) of the
   * matrix data.
   */
  val dim: (Int, Int) = ???

  /**
   * Returns the values from a single column.
   *
   *  @param c The column index.
   */
  def col(c: Int): Vector[String] = ???

  /** Returns the matrix in text format */
  override lazy val toString: String = ???

  /**
   * Returns a copy of the current table sorted according to
   * the specified column.
   *
   * @param c The column index.
   */
  def sort(c: Int): Table = ???

  /**
   * Returns a copy of the current table filtered according to
   * if the specified column is among the wanted values.
   *
   * @param c      The column index.
   * @param wanted The wanted values.
   */
  def filter(c: Int, wanted: Vector[String]): Table = ???

  /**
   * Returns the frequency registered data from a specified column.
   *
   * @param c The column index.
   */
  def register(c: Int): Vector[(String, Int)] = ???
}

object Table {

  /**
   * Opens and reads column separated text data from either
   * a file or a URL into a Table.
   *
   * @param uri The location of the data.
   * @param sep The separator that distinguishes columns.
   */
  def fromFile(uri: String, sep: String): Table = ???

  /**
   * Writes a text formatted Table to disk.
   *
   * @param path  The location of the file to write.
   * @param table The table to be written.
   */
  def toFile(path: String, table: Table): Unit = ???
}
\end{ScalaSpec}

\Subtask Implementera hela klassen \code{Table} förutom \code{register}. I \code{sort} och \code{filter} ska inbyggda funktioner för sortering och filtrering användas. Table är omuterbar, vilken behändig funktion ger \code{case class} som kan användas för att göra en kopia av Table i \code{sort} och \code{filter}?

\Subtask Flera av funktionerna tar emot kolumnindex som parameter, vilket problem kan uppstå om ett negativt tal eller ett relativt matrisen ett för stort tal tas emot?

\Subtask Implementera \code{register}. Fundera ut en lämplig returtyp och om kolumnrubriken ska vara med i returdatan. Funktionen kommer senare användas för att skapa diagram som presenterar den registrerade datan. Om du fastnar: \footnote{Kolla på \code{groupBy}.}.

\Subtask Implementera objektet \code{Table}. Funktionen \code{fromFile} ska kunna ta emot antingen en webbadress eller en lokal sökväg. Varje rad läses in till en yttre vektor och därefter delas varje rad upp i kolumner:
\begin{CodeSmall}[language=, ]
Vector(Vector(rad1kolumn1,rad1kolumn2,rad1kolumn3),
       Vector(rad2kolumn1,rad2kolumn2,rad2kolumn3),
       Vector(rad3kolumn1,rad3kolumn2,rad3kolumn3))
\end{CodeSmall}
I \code{toFile} kan \code{java.nio}-paketet användas för att skriva till fil.

\Subtask Vilka problem kan uppstå vid inläsning av fil eller URL?

\Subtask Var hanteras problemen från tidigare frågor i programmet just nu?

\Subtask \TODO implementera Table main-funktion med indata utdata exempel



\Task Komplettera klassen \code{Command} enligt specifikation:

\ScalaSpecInputListing{Table}{../workspace/w10_survey/src/main/scala/stats/Table.scala}

\Subtask Implementera \code{parseOne}. Observera att parametertypen är \code{Vector[String]}, precis som i \code{parseAll} men skillnaden är att \code{parseOne} ska matcha och returnera exakt ett kommando. I annat fall kastas undantaget \code{InvalidCommandException}. Till exempel \code{"-sort 4"} funkar men \code{"-sort 4 asdf"} funkar inte. Kommandona som ska kunna tas emot finns i \code{List}.

\Subtask Implementera \code{parseAll}. Använd minustecknet som markerar starten för ett nytt kommando för att dela upp argumenten \code{args} i delar som sedan var för sig skickas med anrop till \code{parseOne}. Tips: \footnote{Kolla på \code{span} och \code{startsWith}.}.

\Subtask Implementera slutligen \code{runAllWith}. 

\Subtask Testa

\subsection{Frivilliga extrauppgifter}
    
\Task En labbuppgiftsbeskrivning.

\Subtask En underuppgift.

\Subtask En underuppgift.