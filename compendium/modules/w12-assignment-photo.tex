%!TEX encoding = UTF-8 Unicode
%!TEX root = ../compendium2.tex

\Assignment{photo}

\subsection{Bakgrund}
Detta projekt innebär att du ska implementera en egen bilbehandlingsapplikation, en mycket förenklad variant av \emph{Phtotoshop} eller \emph{Gimp}. 

En digital bild består av ett rutnät, en s.k. matris \Eng{matrix}, av pixlar, var och en med en viss färg. Om man har många små pixlar bredvid varandra i ett rutnät, så flyter de samman för ögat och betraktaren upplever en bild.

Bilder kan manipuleras genom applicering av olika s.k. \emph{filter}, som förändrar bildens pixlar på ett intressanta sätt. Du ska, utifrån given matematisk teori, implementera olika filter med hjälp av speciella matrisoperationer.


Det finns olika system för hur man färgsätter pixlar. T.ex. så används CMYK-systemet (cyan, magenta, gul, svart) vid blandning av färg som ska tryckas på papper eller annat material. På en dator, däremot, används vanligtvis RGB-systemet, som har de tre grundfärgerna röd, grön och blå. Mättnaden av varje grundfärg anges av ett heltal som vi i fortsättningen förutsätter ligger i intervallet [0, 255]. 0 anger ''ingen färg'' och 255 anger ''maximal färg''. Man kan därmed representera 256 × 256 × 256 = 16 777 216 olika färgnyanser. Man kan också representera gråskalor; det gör man med färger som har samma värde på alla tre 
grundfärgerna: (0, 0, 0) är helt svart, (255, 255, 255) är helt vitt. 

I detta projekt kommer vi skapa matriser av heltal för att beräkna intressanta egenskaper hos en bild, till exempel intensiteten för varje pixel. 
För att spara plats vid bearbetning av stora bilder så använder vi, heltalsmatriser med typen \code{Short}, som använder 16 bitar i minnet, i ståället för \code{Int}, som använder 64 bitar i minnet. 

\subsection{Förberedelser}

I detta projekt har du nytta av följande delar av \href{https://github.com/lunduniversity/introprog-scalalib}{\texttt{introprog-scalalib}} och \code{java.awt}:

\begin{itemize}
\item \code{introprog.Image} för bildhantering.
\item \code{introprog.PixelWindow} och \code{introprog.Dialog} för användarinteraktion.
\item \code{introprog.IO} för filhantering.
\item \code{java.awt.Color} för hantering av pixelfärger.
\end{itemize}
Läs noga dokumentationen för klasserna i introprog här och gör egna experiment i REPL så du förstår hur de kan användas: 
\url{https://cs.lth.se/pgk/api/}\\
Läs om klassen \code{java.awt.Color} här:\\\url{https://docs.oracle.com/en/java/javase/11/docs/api/java.desktop/java/awt/Color.html}
Hämta och studera noga den kod som är given för detta projekt här:\\
\url{https://github.com/lunduniversity/introprog/tree/master/workspace/}

\subsection{Matris med värden av typen \code{Short}}

\Task \textbf{\code{Matrix}}.
I den givna kodfilen \code{Matrix.scala} finns hjälp-funktioner för att skapa och uppdatera matriser med värden av typen \code{Short}, för att spara minne vid stora bilder.  

Gör klart saknade implementationer och testa noga i REPL så att allt fungerar som det ska innan du går vidare. \emph{Tips:} Du har nytta av \code{Array.tabulate}. 
\begin{REPLsmall}
scala> import photo.*

scala> val m = Matrix(3,3)(1,2,3,4,5,6,7,8,9) // en 3x3-matiris med Short-värden
val m: photo.Matrix = Array(Array(1, 4, 7), Array(2, 5, 8), Array(3, 6, 9))

scala> m(0,1)
val res0: Short = 4

scala> m(1,0) = 42

scala> m
val res1: photo.Matrix = Array(Array(1, 4, 7), Array(42, 5, 8), Array(3, 6, 9))

scala> m.row(0)
val res2: Array[Short] = Array(1, 42, 3)
\end{REPLsmall}

\subsection{Användargränssnitt}

När appen startar så visas fönstrrt i fig. \ref{photo:fig:main-window}, som implementeras av givna koden i \code{Main.scala}. Med hjälp av \code{Button.scala} skapas en kolumn med knappar som är klickbara. Studera koden i \code{Main.scala} och \code{Button.scala} så du förstår vad som händer. Ännu öppnas inget ImageEditor-fönster, men det ska du påbörja i näst uppgift.

\begin{figure}[H]
\centering
\includegraphics[width=0.4\textwidth]{../img/w12-assignment-photo/photo-main.png}
\caption{Photo-applikationens startfönster.}
\label{photo:fig:main-window}
\end{figure}

\Task \textbf{\code{ImageEditor}}. Do ska skapa en kodfil \code{ImageEditor.scala} som innehåller en klass med samma namn som implementerar ett bildredigeringsfönster med en kolumn med knappar till vänster och en bild inläst från fil till höger, så som visas i fig. \ref{photo:fig:editor-one-filter}. Följande krav på användargränssnittet ska implementeras:

\begin{itemize}
\item Vid tryck på Exit-knappen ska en varningsfråga ''Ok to Exit without save?'' ges med hjälp av \code{introprog.Dialog.isOK} och användaren ges möjlighet att ångra avslut.
\item Vid tryck på Open-knappen så ska en filöppningsdialog visas med hjälp av \code{introprog.Dialog.file}. Om det i aktuell katalog finns en underkatalog vid namn \code{images} så ska filbläddringen börja där, annars i aktuell katalog. 
\item Efter OK på filöppningen ska en bild öppnas i ett bildredigeringsfönster enligt fig. \ref{photo:fig:editor-one-filter} med knapparna Save, Undo, Close, plus en knapp för varje filter, till vänster om bilden. Fönstrets höjd och bredd ska avpassas så att hela bilden och alla knappar får plats.
\item Huvudfönstret och alla bildredigeringsfönster ska fungera parallellt. Detta kan du åstadkomma genom att händelseloopen i \code{ImageEditor}-klassen körs som argument till metoden \code{runInParallell} enligt nedan: 
\begin{CodeSmall}
  def runInParallell(block: => Unit) = 
    new Thread{ override def run(): Unit = block }.start

  def startEventLoop(): Unit = runInParallell {
    // implementera initialisering och händelseloop här, 
    // i likhet med huvudfönstret 
  }
\end{CodeSmall}
\item TODO: Mer krav här
\end{itemize}

\begin{figure}[H]
\centering
\includegraphics[width=0.8\textwidth]{../img/w12-assignment-photo/photo-duck.png}
\caption{Bildredigeringsfönstret, innan fler filter (utöver identitetsfiltret) implementerats. Varje filter som implementeras ska ha en motsvarande knapp.}
\label{photo:fig:editor-one-filter}
\end{figure}
  


\subsection{Filter}
Du ska skriva ett program där du implementerar olika filter som ska manipulera en given bild på ett flertal olika sätt. 
Filterklasserna ska ärva från en abstrakt \code{ImageFilter}-klass.

Följande specifikation beskriver klassen \code{ImageFilter}:

\begin{ScalaSpec}{abstract class ImageFilter}
	/**
	* Create a filter object with a given name and argument descriptions.
	* @param name
	*            the name of the filter.
	* @param args
	*            optional array of strings with argument descriptions.
	*/
	abstract class ImageFilter(val name: String, val argDescriptions: String*):

		/**
		The number of args this filter needs.
		*/
		def nbrOfArgs: Int

		/**
		* Apply the filter on `img` and return the result as a new 
		* Image using the arguments in `args`.
		* 
		* @param img
		*            the original image.
		* @param args
		*            arguments
		* @return 
		*			 the resulting image.
		*/
		def apply(img: Image, args: Double*): Image

		/**
		* Calculate the intensity in each pixel of `img`.
		* 
		* @param img
		*           the image
		* @return intensitymatrix, values ranging from 0 to 255
		*/
		protected def computeIntensity(img: Image): Array[Array[Short]] = 
		   val intensity: Array[Array[Short]]

		/**
		* Convolute `p[i][j]` with the convolutionkernel `kernel`.
		* 
		* @param p
		*            matrix with numbervalues
		* @param i
		*            current row index
		* @param j
		*            current coloumn index
		* @param kernel
		*            convolutionkernel, a 3x3-matrix
		* @param weight
		*            the sum of the element in `kernel`
		* @return result of the convolution
		*/
		protected def convolve(p: Array[Array[Short]], i: Int, j: Int, 
		kernel: Array[Array[Short]], weight: Int): Short

\end{ScalaSpec}

Utöver filterklasserna ska du även skapa ett program som låter användaren välja en bild som olika filter sedan kan appliceras på.
För att åstadkomma detta ska du implementera klassen \code{ImageEditor}, som hanterar applicering av filter samt ritar bilden med hjälp av \code{PixelWindow}.



\code{ImageEditor}-klassen ska använda sig av en Stack[Image] för att hantera historiken. 
Läs dokumentationen här: \url{https://www.scala-lang.org/api/current/scala/collection/mutable/Stack.html}

\begin{ScalaSpec}{class ImageEditor}
	class ImageEditor(filters: Array[ImageFilter]):

		val history = ???

		/** 
		 * Show which filters are available and let the user choose a filter to apply. 
		 * Draw edited image in PixelWindow.
		 * User can then add more filters, undo, save image or exit.
		 *  
		 *  Example: 
		 *  0. för Blått-filter
		 *	2. för Kontrast-filter
		 * 	3. för Gauss-filter
		 *	4. för Sobel-filter
		 *	a. AVBRYT
		 *	s. SPARA
		 *	z. UNDO
		*/
		def edit(im: Image): Unit = ???

		/**Ask for arguments if required and apply filter with index `i`*/
		private def applyFilter(index: Int) = ???
\end{ScalaSpec}

Studera Scala-koden som är given här: \url{https://github.com/lunduniversity/introprog/tree/master/workspace/w13_photo_proj/}
Till din hjälp får du ett \code{ImageChooser}-objekt som hjälper dig att ladda in en bild. 
I projektet används flera klasser från kursens introprog-bibliotek. Särskilt används \code{Image, PixelWindow} och \code{IO}.
Du hittar dokumentationen till klasserna här: \url{https://cs.lth.se/pgk/api/} och du kan läsa koden här: \url{https://github.com/lunduniversity/introprog-scalalib}

\begin{ScalaSpec}{object ImageChooser}
	object ImageChooser:

		/**
		* Returns a chosen image from the images folder.
		* Prints:	
		*
		*	Välj en av följande bilder genom att mata in en siffra
		*
		*	0. boy.jpg
		*	1. car.jpg
		*	2. duck.jpg
		*	3. jay.jpg
		*	4. moon.jpg
		*	5. shuttle.jpg
		*	Ditt val: 
		*/
		def getImage: Image
\end{ScalaSpec}


\Task \textbf{Blåfilter.} Skriv en klass \code{BlueFilter} som skapar en blå version av bilden. 
Det vill säga skapa ett filter där varje pixel bara innehåller den blå komponenten. 
Testa filtret genom att skapa ett \code{Application}-object som ska innehålla en 
\code{main}-metod (\code{Application} ska användas och utökas i senare uppgifter). 
Använd \code{ImageChooser} för att välja en bild på följande sätt:
\begin{Code}
	val im = ImageChooser.getImage
\end{Code}
\begin{enumerate}
	\item Visa bilden genom att i \code{imageEditor.edit()} använda \code{pixelWindow.drawImage()}. 
	\item Alla \code{Filter}-klasser kan placeras i samma fil \code{Filters.scala}.
	\item Om du har problem med att ärva från den abstrakta \code{ImageFilter}-klassen kan du använda dig av detta exempel som endast kopierar bilden.
\end{enumerate}
\begin{ScalaSpec}{class IdentityFilter}
	class IdentityFilter(name: String) extends ImageFilter(name):
	
		def apply(im: Image, args: Double*): Image = 
			val result = Image.ofDim(im.width, im.height)
			result.update((i, j) => im(i, j))
			result
\end{ScalaSpec}



\Task \textbf{Inverteringsfilter.} Skriv en klass \code{InvertFilter} som inverterar en bild, dvs skapar en ''negativ'' kopia av bilden. Ljusa färger ska alltså bli mörka och mörka färger ska bli ljusa.
Fundera över vad som kan menas med en inverterad eller negativ kopia: de nya RGB-värdena är inte ett dividerat med de gamla värdena (då skulle de nya värdena kunna bli flyttal) och inte de gamla värdena med ombytt tecken (då skulle de nya värdena bli negativa).

\Task \textbf{Gråskalningsfilter.} Skriv en klass \code{GrayScaleFilter} som gör om bilden till en gråskalebild. Använd \code{ImageFilter}s \code{computeIntensity}-metod för att bestämma vilken intensitet varje pixel ska ha. Om intensiteten i en pixel till exempel är 105 så ska ett nytt \code{Color}-objekt med värdena (105, 105, 105) skapas.

\Task \textbf{Krypteringsfilter.} Skriv en klass \code{XORCryptFilter} som krypterar bilden med xor-operatorn ˆ. Denna operator gör binär xor mellan bitarna i ett heltal. Exempelvis ger 8 ˆ 127 värdet 119. Om man gör xor igen med 127, alltså 119 ˆ 127, får man tillbaka värdet 8. Varje pixel krypteras genom att använda xor-operatorn med ursprungsvärdena för rött, grönt och blått tillsammans med slumpmässiga heltalsvärden som genereras av Scalas \code{Random}-klass (tre nya slumptal för varje pixel). Låt användaren ge ett argument som seed för \code{Random}. På så sätt kan du återskapa bilden genom att applicera krypteringsfiltret igen, med samma parametervärde, på den numera krypterade bilden.

\Task \textbf{Gaussfiltrer.} Gaussfiltrering är ett exempel på så kallad faltningsfiltrering. Filtreringen bygger på att man modifierar varje bildpunkt genom att titta på punkten och omgivande punkter.

För detta utnyttjar man en så kallad faltningskärna K som är en liten kvadratisk heltalsmatris. Man placerar K över varje element i intensitetsmatrisen och multiplicerar varje element i K med motsvarande element i intensitetsmatrisen. Man summerar produkterna och dividerar summan med summan av elementen i K för att få det nya värdet på intensiteten i punkten. Divisionen med summan gör man för att de nya intensiteterna ska hamna i rätt intervall.

Exempel:

\begin{minipage}{5cm}
\begin{displaymath}
\mathit{intensity} = \left(
\begin{array}{ccccc}
5 & 4 & 2 & 8 & \ldots \\
4 & 3 & 4 & 9 & \ldots \\
9 & 8 & 7 & 7 & \ldots \\
8 & 6 & 6 & 5 & \ldots \\
\vdots & \vdots & \vdots & \vdots & \ddots
\end{array}
\right)
\end{displaymath}
\end{minipage}\hspace{2cm}
\begin{minipage}{5cm}
\begin{displaymath}
K = \left(
\begin{array}{ccc}
0 & 1 & 0 \\
1 & 4 & 1 \\
0 & 1 & 0
\end{array}
\right)
\end{displaymath}
\end{minipage}

Här är summan av elementen i $K$ $1+1+4+1+1 = 8$. För att räkna ut det nya värdet på intensiteten i punkten med index \code{(1)(1)} (det nuvarande värdet är 3) beräknar man:

\begin{displaymath}
\mathit{newintensity} = \frac{0 \cdot 5 + 1 \cdot 4 + 0 \cdot 2 + 1 \cdot 4 + 4 \cdot 3 + 1 \cdot 4 + 0 \cdot 9 + 1 \cdot 8 + 0 \cdot 7}{8} = \frac{32}{8} = 4
\end{displaymath}


Man fortsätter med att flytta K ett steg åt höger och beräknar på motsvarande sätt ett nytt värde för elementet med index \code{(1)(2)} (där det nuvarande värdet är 4 och det nya värdet blir 5). Därefter gör man på samma sätt för alla element utom för ”ramen” dvs elementen i matrisens ytterkanter.

Skriv en klass \code{GaussFilter} som implementerar denna algoritm. Varje färg ska behandlas separat. Gör på följande sätt:
\begin{enumerate}
	\item Bilda tre \code{short}-matriser och lagra pixlarnas red-, green- och blue-komponenter i matriserna.
	\item Utför faltningen av de tre komponenterna för varje element och updatera result med de uträknade värdena.
	\item Elementen i ramen behandlas inte, men i \code{result} måste också dessa element få värden. Enklast är att flytta över dessa element oförändrade från \code{im} till \code{result}. Man kan också sätta dem till \code{Color.WHITE}, men då kommer den filtrerade bilden att se något mindre ut.
\end{enumerate}

Använd \code{ImageFilter}s \code{convolve}-metod för att utföra faltningen. Metoden behöver en faltningsmatris, \code{kernel}, som input och ska anropas med red-, green- och blue-matrisen. Faltningsmatrisen kan vara ett attribut i klassen och ska ha följande utseende:

\begin{displaymath}
\begin{pmatrix}
  0 & 1 & 0 \\
  1 & 4 & 1 \\
  0 & 1 & 0 \\
\end{pmatrix}
\end{displaymath}

Det kan vara intressant att prova med andra värden än 4 i mitten av faltningsmatrisen. Med värdet 0 får man en större utjämning eftersom man då inte alls tar hänsyn till den aktuella pixelns värde. Låt användaren mata in argument för mittvärdet via terminalen.

Anmärkning: det kan ibland vara svårt att se någon skillnad mellan den filtrerade bilden och originalbilden. Om man vill ha en riktigt suddig bild så måste man använda en större matris som faltningskärna.


\Task  \textbf{Sobelfiltrer.} Sobelfiltrering är, precis som Gaussfiltrering, en typ av faltningsfiltrering. Med Sobelfiltrering får man dock motsatt effekt i jämförelse med Gaussfiltrering, dvs man förstärker konturer i en bild. I princip deriverar man bilden i x- och y-led och sammanställer resultatet.

\begin{figure}[H]
\includegraphics[width=0.9\textwidth]{../img/w12-assignment-photo/derivatabild2.pdf}
\caption { En funktion (heldragen linje) och dess derivata (streckad linje).}
\label{fig:photo:sobelfilter:derivatabild}
\end{figure}

I figur~\ref{fig:photo:sobelfilter:derivatabild} visas en funktion $f$ (heldragen linje) och funktionens derivata $f'$ (streckad linje). Vi ser att där funktionen gör ett ''hopp'' så får derivatan ett stort värde. Om funktionen representerar intensiteten hos pixlarna längs en linje i x-led eller y-led så motsvarar ''hoppen'' en kontur i bilden. Om man sedan bestämmer sig för att pixlar där derivatans värde överstiger ett visst tröskelvärde ska vara svarta och andra pixlar vita så får man en bild med bara konturer.

Nu är ju intensiteten hos pixlarna inte en kontinuerlig funktion som man kan derivera enligt vanliga matematiska regler. Men man kan approximera derivatan, till exempel med följande formel:

\begin{displaymath}
f'(x) \approx \frac{f(x+h) - f(x-h)}{2h}
\end{displaymath}

(Om man här låter $h$ gå mot noll så får man definitionen av derivatan.) Uttryckt i Scala och matrisen \code{intensity} så får man:

\begin{Code}
val derivative = (intensity(i)(j+1) - intensity(i)(j-1)) / 2
\end{Code}

Allt detta kan man uttrycka med hjälp av faltning.

\begin{enumerate}
	\item Beräkna intensitetsmatrisen med metoden \code{computeIntensity}.
	\item Falta varje punkt i intensitetsmatrisen med två kärnor:
$$
X\_SOBEL =
\begin{pmatrix}
  -1 & 0 & 1 \\
  -2 & 0 & 2 \\
  -1 & 0 & 1 \\
\end{pmatrix}
Y\_SOBEL =
\begin{pmatrix}
  -1 & -2 & -1 \\
  0 & 0 & 0 \\
  1 & 2 & 1 \\
\end{pmatrix}
$$
	Använd metoden \code{convolve} med vikten 1. Koefficienterna i matrisen $X\_SOBEL$ uttrycker derivering i x-led, i $Y\_SOBEL$ faltning i y-led. För att förklara varför koefficienterna ibland är 1 och ibland 2 måste man studera den bakomliggande teorin noggrant, men det gör vi inte här.
	\item Om resultaten av faltningen i en punkt betecknas med \code{sx} och \code{sy} så får man en indikator på närvaron av en kontur med \code{math.abs(sx) + math.abs(sy)}. Absolutbelopp behöver man eftersom man har negativa koefficienter i faltningsmatriserna.
	\item  Sätt pixeln till svart om indikatorn är större än tröskelvärdet, till vit annars. Låt tröskelvärdet bestämmas av ett argument som användaren kan ange.
\end{enumerate}

Skriv en klass \code{SobelFilter} som implementerar denna algoritm.

\begin{figure}[H]
\begin{center}
\includegraphics[width=0.6\textwidth]{../img/w12-assignment-photo/sobeljay.png}
\caption { Exempel på en bild där ett Sobelfilter applicerats med ett parametervärde på 150.}
\label{fig:photo:sobelfilter:sobel}
\end{center}
\end{figure}


\Task Implementera \code{ImageEditor} enligt specifikationerna ovan.

\Task Knyt ihop allt i \code{Application}-objektet som du skapade innan. Utskrifterna ska se ut på följande sätt:\newline

{\setlength{\parindent}{0cm}

 Välj en av följande bilder genom att mata in en siffra\newline

0. boy.jpg\newline
1. car.jpg\newline
2. duck.jpg\newline
3. jay.jpg\newline
4. moon.jpg\newline
5. shuttle.jpg\newline
Ditt val: \textbf{1}\newline
Bild car.jpg laddad\newline
\textit{//Bilden visas i ett PixelWindow}\newline

Välj ett alternativ\newline

0. för Blåfilter\newline
1. för Inverterat\newline
2. för Krypterat\newline
3. för Inverterat\newline
4. för Grått\newline
5. för Kontrast\newline
6. för Gauss\newline
7. för Sobel\newline
8. för Intervall\newline
9. för Change\newline
a. AVBRYT\newline
s. SPARA\newline
z. UNDO\newline

Ditt val: \textbf{6}\newline
Till detta filter behöver du ange [1] argument.\newline

Ange argument för Gauss: \newline
Argument 1: [Styrka (0-50) där 0 är max]: \textbf{0}\newline
\textit{//Bilden uppdateras och ritas i samma PixelWindow}\newline

Välj ett alternativ\newline
...
}

Tänk på att användaren kan mata in otillåtna värden. Detta ska hanteras på lämpligt sätt.

\Task Du ska inför redovisningen generera automatisk dokumentation baserat på dokumentationskommentarer enligt instruktioner i Appendix \ref{appendix:doc}. Du ska skriva relevanta dokumentationskommentarer för minst hälften av dina publika metoder. Det är ofta användbart att skriva dokumentationskommentarerna \emph{före} implementationen av metodkroppen.

\subsection{Frivilliga extrauppgifter}

\Task \textbf{Kontrastfilter.} Om man applicerar kontrastfiltrering på en färgbild så kommer bilden att konverteras till en gråskalebild. (Man kan naturligtvis förbättra kontrasten i en färgbild och få en färgbild som resultat. Då behandlar man de tre färgkanalerna var för sig.) Många bilder lider av alltför låg kontrast. Det beror på att bilden inte utnyttjar hela det tillgängliga området 0–255 för intensiteten. Man får en bild med bättre kontrast om man ''töjer ut'' intervallet enligt följande formel (linjär interpolation):

\begin{Code}
val newIntensity = 255 * (intensity - 45) / (225 - 45)
\end{Code}

Som synes kommer en punkt med intensiteten 45 att få den nya intensiteten 0 och en punkt med intensiteten 225 att få den nya intensiteten 255. Mellanliggande punkter sprids ut jämnt över intervallet \code{[0, 255]}. För punkter med en intensitet mindre än 45 sätter man den nya intensiteten till 0, för punkter med en intensitet större än 225 sätter man den nya intensiteten till 255. Vi kallar intervallet där de flesta pixlarna finns för \code{[lowCut, highCut]}. De punkter som har intensitet mindre än \code{lowCut} sätter man till 0, de som har intensitet större än \code{highCut} sätter man till 255. För de övriga punkterna interpolerar man med formeln ovan (45 ersätts med \code{lowCut}, 225 med \code{highCut}).

Det återstår nu att hitta lämpliga värden på \code{lowCut} och \code{highCut}. Detta är inte något som kan göras helt automatiskt, eftersom värdena beror på intensitetsfördelningen hos bildpunkterna. Man börjar med att beräkna bildens intensitetshistogram, dvs hur många punkter i bilden som har intensiteten 0, hur många som har intensiteten 1, . . . , till och med 255.

I de flesta bildbehandlingsprogram kan man sedan titta på histogrammet och interaktivt bestämma värdena på \code{lowCut} och \code{highCut}. Så ska vi dock inte göra här. I stället bestämmer vi oss för ett procenttal \code{cutOff}, som användaren kan ange som argument från terminalen, och som  beräknar \code{lowCut} så att \code{cutOff} procent av punkterna i bilden har en intensitet som är mindre än \code{lowCut} och \code{highCut} så att \code{cutOff} procent av punkterna har en intensitet som är större än \code{highCut}.

Exempel: antag att en bild innehåller 100 000 pixlar och att \code{cutOff} är 1.5. Beräkna bildens intensitetshistogram i en array
\begin{Code}
val histogram = Array[Int](256)
\end{Code}

Beräkna \code{lowCut} så att \code{histogram(0)} + \ldots + \code{histogram(lowCut)} = 0.015 * 100000 (så nära det går att komma, det blir troligen inte exakt likhet). Beräkna \code{highCut} på liknande sätt.

Sammanfattning av algoritmen:
\begin{enumerate}
	\item Beräkna intensiteten hos alla punkterna i bilden, lagra dem i en \code{short}-matris. Använd den färdigskrivna metoden \code{computeIntensity}.
	\item Beräkna bildens intensitetshistogram.
	\item Argument från användaren användas som \code{cutOff}.
	\item Beräkna \code{lowCut} och \code{highCut} enligt ovan.
	\item Beräkna den nya intensiteten för varje pixel enligt interpolationsformeln och lagra de nya pixlarna i \code{result}.
\end{enumerate}
Skriv en klass \code{ContrastFilter} som implementerar algoritmen. I katalogen \emph{images} kan bilden \emph{moon.jpg} vara lämplig att testa, eftersom den har låg kontrast. Anmärkning: om \code{cutOff} sätts = 0 så får man samma resultat av denna filtrering som man får av \code{GrayScaleFilter}. Detta kan man se genom att studera interpolationsformeln.

\Task \textbf{Eget filter.} Skapa ett eget filter som utnyttjar att \code{apply}-metoden tar emot en sekvens av värden. Till exempel så kan du skicka in en array med fem värden där de två första värdena representerar ett intesitetsintervall och de tre sista värdena representerar röd-, grön- och blåkomponenterna till en färg som ska stoppas in där intensiteten hamnar utanför det givna intervallet. Ett annat alternativ kan vara att använda sig av metoder i \code{PixelWindow} för att välja specifika pixlar på originalbilden som sedan kan användas för att manipulera bilden i filtrets \code{apply}-metod. Valet är ditt!
