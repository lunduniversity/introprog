%!TEX encoding = UTF-8 Unicode
%!TEX root = ../exercises.tex

\ifPreSolution

\Exercise{\ExeWeekTHIRTEEN}\label{exe:W13}
% \begin{Goals}
% \item Kunna skriva tentamenslika program med papper, penna och snabbreferens som enda hjälpmedel.
% \item Förbereda projektredovisningen.
% \item Kunna skapa jar-filer.
% \end{Goals}

% \begin{Preparations}
% \item \StudyTheory{13}
% \end{Preparations}

\else

\ExerciseSolution{\ExeWeekTHIRTEEN}

\fi


%\subsection{Förberedelse inför examination}


\WHAT{Gör klart ditt projekt.} %%%%%%%%%%%%%%%%%%%%%%%%%%%%%%%%%%%%%%%%%%%%%%%%%%%%%%%%

\QUESTBEGIN

\Task \what~ % TODO

\SOLUTION

\TaskSolved \what~ ---

\QUESTEND



\WHAT{Gör en extenta.} %%%%%%%%%%%%%%%%%%%%%%%%%%%%%%%%%%%%%%%%%%%%%%%%%%%%%%%%

\QUESTBEGIN

\Task \what~ % TODO

\SOLUTION

\TaskSolved \what~ ---

\QUESTEND




\WHAT{Förbered din projektredovisning.} %%%%%%%%%%%%%%%%%%%%%%%%%%%%%%%%%%%%%%%

\QUESTBEGIN

\Task \what~ % TODO

\SOLUTION

\TaskSolved \what~ --

\QUESTEND



\WHAT{Skapa dokumentation för ditt projekt.} %%%%%%%%%%%%%%%%%%%%%%%%%%%%%%%%%%%%%%%%%%%%%%%%%%%

\QUESTBEGIN

\Task  \what~ Läs mer i Appendix E om hur du skapar dokumentation.
%Läs i appendix om hur du kan skapa dokumentation med scaladoc och javadoc.
%
% \Subtask Kör nedan kommando i terminalen:
%
% \begin{REPL}
% > scaladoc min-kod.scala
% > ls
% > firefox index.html   # eller öppna index.html i valfri webbläsare
% \end{REPL}
%
% Vad händer?

% \Subtask Lägg till några fler metoder i något av objekten i filen \code{paket.scala} och lägg även till några dokumentationskommentarer. Kompilera om och kör. Generera om dokumentationen.
%
% \begin{verbatim}
% //... ändra i filen paket.scala
%
% /** min paketdokumentationskommentar p2 */
% package p2 {
%   /** min paketdokumentationskommentar p21 */
%   package p21 {
%     /** ett hälsningsobjekt */
%     object hello {
%       /** en hälsningsmetod i p2.p21 */
%       def hello = println("Hej paket p2.p21!")
%
%       /** en metod som skriver ut tiden */
%       def date = println(new java.util.Date)
%     }
%   }
% }
%
% \end{verbatim}
%
% \begin{REPL}
% > gedit paket.scala
% > scalac paket.scala
% > jar cvf mittpaket.jar gurka
% > scala -cp mittpaket.jar
% scala> gurka.tomat.banan.p2.p21.hello.date
% scala> :q
% > scaladoc paket.scala
% > firefox index.html
% \end{REPL}

\SOLUTION


\TaskSolved \what --

\QUESTEND




\WHAT{Repetera övningar och laborationer.} %%%%%%%%%%%%%%%%%%%%%%%%%%%%%%%%%%%%

\QUESTBEGIN

\Task \what~

\SOLUTION

\TaskSolved \what~ --

\QUESTEND
