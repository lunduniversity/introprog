%!TEX encoding = UTF-8 Unicode
%!TEX root = ../compendium2.tex

\section{Visual Studio Code med Scala-tillägget Metals}\label{appendix:ide:vscode}

Visual Studio Code\footnote{\href{https://en.wikipedia.org/wiki/Visual\_Studio\_Code}{en.wikipedia.org/wiki/Visual\_Studio\_Code}}, förkortat VS Code eller bara \code{code}, är en gratis utvecklingsmiljö som är mestadels öppen källkod\footnote{Varianten VS Codium \url{https://vscodium.com/} är helt fri från stängd källkod.}. Projektet startades och leds av Microsoft och har en aktiv gemenskap med många utvecklare och många användbara tillägg \Eng{extensions}.

VS Code kallas ofta för ''bara'' en editor, men har genom åren utvecklats till en fullfjädrad IDE med bl.a. inbyggd debugger och stöd för många olika språk via ett omfattande bibliotek av tillägg.%


Du kan aktivera debuggern i VS Code för dina Scala-program genom att klicka på ''debug'' ovanför din \code{main}-metod, förutsatt att du har tillägget Metals installerad i VS Code och en aktiverad och giltig \code{build.sbt}-fil. Läs mer om debugging i Appendix \ref{appendix:debug}.  

Läs mer om hur man använder VS Code här: \\
\url{https://code.visualstudio.com/docs}

Läs mer om hur du använder Scala i VS Code här: \\
\url{https://scalameta.org/metals/docs/editors/vscode}

\subsection{Installera VS Code och Metals}\label{appendix:ide:vscode:install}

VS Code är förinstallerad på LTH:s datorer, men du behöver själv installera Scala-tillägget \textbf{Metals} första gången du kör igång VS Code på LTH:s datorer. Läs om installation av Metals här: \\
\url{https://marketplace.visualstudio.com/items?itemName=scalameta.metals} 

Läs mer om hur du installerar VS Code på din egen dator här: \\\url{https://code.visualstudio.com}

Mer information om installation av verktyg finns på kursens hemsida: \\
\url{https://cs.lth.se/pgk/verktyg}

