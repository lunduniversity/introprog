%!TEX encoding = UTF-8 Unicode
%!TEX root = ../compendium.tex

\chapter{Integrerad utvecklingsmiljö}\label{appendix:ide}

\section{Vad är en integrerad utvecklingsmiljö?}

En integrerad utvecklingsmiljö \Eng{integrated development environment, IDE} samlar ett flertal verktyg, inklusive en avancerad \textbf{editor} (se appendix \ref{appendix:edit}), för att skapa, köra och testa program. Det finns flera utvecklingsmiljöer att välja mellan, som kan användas för både Scala och Java.

En IDE ger stöd för \textbf{kodkomplettering} \Eng{code completion} där tillgängliga metoder visas i en lista och resten av ett namn kan fyllas efter att du skrivit de första bokstäverna i namnet. En IDE kan hjälpa dig med formattering och även skapa skelettkod utifrån \textbf{kodmallar} \Eng{code templates}. Med \textbf{felindikering} \Eng{error highlighting} får du understrykning av vissa fel direkt i koden och ibland kan du även få hjälp med förslag på åtgärder för att rätta till enkla fel. Funktioner för \textbf{avlusning} \Eng{debugging} hjälper dig att felsöka medan du kör din kod. Med funktioner för \textbf{omstrukturering} \Eng{refactoring} av kod får du hjälp av editorn i samarbete med kompilatorn att göra omfattande strukturförändringar i många kodfiler samtidigt, t.ex. namnbyten med hänsyn taget till språkets synlighetsregler.  

Alla dessa avancerade funktioner kan öka produktiviteten avsevärt, men samtidigt tar de tid att lära sig och en IDE kan kräva mycket datorkraft och viss väntetid jämfört med en vanlig, fristående editor. I början kan all funktionalitet upplevas som överväldigande och det kan vara svårt att hitta i alla menyer och inställningar. Ska man bara skriva ett litet, enkelt program, eller göra några mindre ändringar, är det många som föredrar en fristående, snabbstartad kodeditor före en fullfjädrad, tungrodd IDE. Å andra sidan kan en IDE med kodkomplettering vara till stor hjälp när man ska lära sig ett nytt api och experimentera med en okänd kodmassa.

I kursen använder vi flera utvecklingsmiljöer. På första labben använder vi Kojo (avsnitt \ref{appendix:ide:kojo}) som är en IDE speciellt anpassad på nybörjare. I laborationerna senare i kursen kan du välja att använda någon av de professionella utvecklingsmiljöerna Eclipse (avsnitt \ref{appendix:ide:eclipse}) eller IntelliJ (avsnitt \ref{appendix:ide:intellij}). Om du inte vet vilken du ska välja, börja med att prova Eclipse.


\newpage

\section{Kojo}\label{appendix:ide:kojo}

Kojo är en integrerad utvecklingsmiljö för Scala som är speciellt anpassad för nybörjare i programmering. Kojo används i LTH:s Science Center Vattenhallen för utbildning av grundskolelärare i programmering och vid skolbesök och annan besöksverksamhet, i vilken lärare och studenter vid LTH arbetar som handledare. 

Kursens första laboration genomförs med hjälp av Kojo, men Kojo kan med fördel användas som komplement till Scala REPL och annan IDE under hela kursens gång. Medan Scala REPL lämpar sig för korta kodsnuttar, och en fullfjädrad, professionell IDE har funktioner för att hantera riktigt stora programmeringsprojekt, passar Kojo bra för mellanstora program. I Kojo finns även lätttillgängliga bibliotek som gör tröskeln lägre att programmera rörlig grafik och enkla spel.   


\begin{figure}[H]
\centering
\includegraphics[width=\textwidth]{../img/kojo/kojo.png}
\caption{Den integrerade utvecklingsmiljön Kojo, anpassad för nybörjare.}
\label{fig:appendix:ide:kojo}
\end{figure} 

\subsection{Installera Kojo}

Kojo är förinstallerat på LTH:s datorer och körs igång med kommandot \texttt{kojo}. För instruktioner om hur du installerar Kojo på din egen dator se här:\\
\href{http://www.lth.se/programmera/installera/}{lth.se/programmera/installera}

Kojo kräver att \texttt{java} finns på din dator. Eftersom du behöver tillgång till JDK i kursen, är det lika bra att installera hela JDK direkt (och inte bara JRE, så som beskrivs å länken ovan); se vidare hur du gör detta i avsnitt \ref{appendix:compile:install-jdk}. 
%\href{http://www.kogics.net/kojo-download}{www.kogics.net/kojo-download}


\subsection{Använda Kojo}

När du startar Kojo första gången, välj ''Svenska'' i språkmenyn och starta om Kojo. Därefter fungerar grafikfunktionerna på svenska enligt tabell \ref{table:kojo:functions}. När du startat om Kojo inställt på Svenska ser programmet ut ungeför som i figur \ref{fig:appendix:ide:kojo} på sidan \pageref{fig:appendix:ide:kojo}.


Det finns ett antal användbara kortkommando som du hittar i menyerna i Kojo. Undersök speciellt Ctrl+Alt+Mellanslag som ger autokomplettering baserat på det du börjat skriva.


{\small\renewcommand{\arraystretch}{1.45}
\begin{longtable}{@{}p{0.42\textwidth} p{0.55\textwidth}}

\caption{Några av sköldpaddans funktioner. Se även \href{http://lth.se/programmera}{lth.se/programmera}}\label{table:kojo:functions}\\

\emph{Svenska/Engelska} & \emph{Vad händer?}  \\ \hline
%!TEX encoding = UTF-8 Unicode
%!TEX root = ../compendium2.tex

\chapter{Kojo}\label{appendix:kojo}

\section{Vad är Kojo?}

Kojo%
\footnote{\href{https://en.wikipedia.org/wiki/Kojo_(programming_language)}{en.wikipedia.org/wiki/Kojo\_(programming\_language)}}
 är en integrerad utvecklingsmiljö för Scala som är speciellt anpassad för programmeringsundervisning i grundskolan. Kojo används i LTH:s Science Center Vattenhallen för utbildning av grundskolelärare i programmering och vid skolbesök och annan besöksverksamhet, i vilken lärare och studenter vid LTH arbetar som handledare. 
 
 Kojo är öppen källkod och utvecklingsgemenskapen leds av Lalit Pant från Indien. I Kojo finns även lättillgängliga bibliotek som gör tröskeln lägre att programmera rörlig grafik och enkla spel.

Under kursens första laboration använder vi grafikbiblioteket i Kojo för att illustrera grundläggande begrepp, så som sekvens, alternativ, repetition och abstraktion.  


\begin{figure}[H]
\centering
\includegraphics[width=0.8\textwidth]{../img/kojo/kojo.png}
\caption{Den nybörjarvänliga utvecklingsmiljön Kojo för Scala på svenska.}
\label{fig:appendix:ide:kojo}
\end{figure}

\section{Använda grafikbiblioteket i Kojo}\label{appendix:ide:kojo:install}

Kojo bygger på den beprövade pedagogiska idén med sköldpaddsgrafik \Eng{turtle graphics}\footnote{\url{https://en.wikipedia.org/wiki/Turtle_graphics}}, där du skriver program som styr en sköldpadda med en penna under magen. När sköldpaddan rör sig bildas ett streck av valfri färg på skärmen. Beroende på hur du bestämmer att sköldpaddan ska röra sig och vilken färg du bestämmer att pennan ska ha, kan du skapa olika intressanta bilder och samtidigt lära dig om programmeringens grunder.

Under kursens första laboration ska du använda grafikbiblioteket i Kojo tillsammans med editorn VS \code{code} och \code{scala-cli} i terminalen (se appendix \ref{appendix:terminal} och \ref{appendix:compile}). Ladda ner filen \texttt{kojo.scala} från \url{https://cs.lth.se/pgk/kojolib} och spara i en ny katalog med hjälp av din webbläsare, eller via dessa kommandon:

\begin{REPLnonum}
> mkdir w01-kojo
> cd w01-kojo
> curl -o kojolib.scala -sL https://cs.lth.se/pgk/kojolib
\end{REPLnonum}

Nu kan du starta Scala REPL och rita med Kojo så här:

\begin{REPLnonum}
> scala-cli repl .
Welcome to Scala 3.1.2 (17.0.2, Java OpenJDK 64-Bit Server VM).
Type in expressions for evaluation. Or try :help.
                                                                                                                               
scala> fram; höger; fram; vänster

\end{REPLnonum}

Du kan starta VS \code{code} i aktuellt bibliotek så här:
\begin{REPLnonum}
> code .
\end{REPLnonum}

Skriv nedan progam i VS \code{code} och spara det i samma katalog som den tidigare nedladdade filen, under ett nytt valfritt filnamn, t.ex. \code{rita.scala}:

\begin{Code}
@main def rita = fram; höger; fram; vänster
\end{Code}

Kör ditt fristående program med:
\begin{REPLnonum}
> scala-cli run .
\end{REPLnonum}

Du ska nu få upp ett fönster som heter Kojo Canvas med en sköldpadda som ritat två streck. När du stänger fönstret så avslutas programmet. Prova fler sköldpaddsfunktioner enligt tabell \ref{table:kojo:functions}.

I stället för att ladda ned filen \code{kojolib.scala} så kan du placera dess innehåll på lämpligt ställe i ditt program enligt nedan. Observera att raden som börjar med \code{//> using lib} ska vara en enda lång rad utan radbrytningar.%\code{export} gör Kojos kommandon tillgängliga utan prefix:
\lstinputlisting[breaklines=true,basicstyle=\ttfamily\fontsize{9}{11}\selectfont]{../workspace/w01_kojo/kojo.scala}

\noindent Scala-koden för den svenska paddans api finns här: \\
%\href{https://github.com/litan/kojo/blob/master/src/main/scala/net/kogics/kojo/lite/i18n/svInit.scala}{github.com/litan/kojo/blob/master/src/main/scala/net/kogics/kojo/lite/i18n/svInit.scala} \\
\href{https://github.com/litan/kojo-lib/blob/main/src/main/scala/net/kogics/kojo/i18n/Swedish.scala}{github.com/litan/kojo-lib/blob/main/src/main/scala/net/kogics/kojo/i18n/Swedish.scala}


%Kojo kräver (numera) \emph{inte} att \texttt{java} finns på din dator utan kommer med en egen JVM. 
%Eftersom du behöver tillgång till JDK i kursen, är det lika bra att installera hela JDK direkt (och inte bara JRE, så som beskrivs å länken ovan); se vidare hur du gör detta i avsnitt \ref{appendix:compile:install-jdk}.
%\href{http://www.kogics.net/kojo-download}{www.kogics.net/kojo-download}



\section{Kojo Desktop}

Kojo finns som fristående skrivbordsapplikation, kallad Kojo Desktop. Kojo Desktop innehåller en egen editor med syntaxfärgning för Scala, men fungerar ännu så länge bara för Scala 2. En av de synligaste skillnaderna mellan Scala 2 och Scala 3 är att klammerparenteser vid flerradiga funktioner är nödvändiga i Scala 2, medan Scala 3 har valfria klammerparenteser. Så om du använder Kojo Desktop behöver du komma ihåg att omgärda sekvenser av rader som hör ihop med \code|{| och \code|}|. 

Kojo Desktop är förinstallerad på LTH:s datorer och körs igång med terminalkommandot \texttt{kojo} eller via applikationsmenyn.  För instruktioner om hur du installerar Kojo Desktop på din egen dator se här: \href{http://www.lth.se/programmera/installera/}{lth.se/programmera/installera}

När du startar Kojo första gången, välj ''Svenska'' i språkmenyn och starta om Kojo. Därefter fungerar grafikfunktionerna på svenska enligt tabell \ref{table:kojo:functions} på sidan \pageref{table:kojo:functions}. När du startat om Kojo inställt på svenska ser programmet ut ungefär som i figur \ref{fig:appendix:ide:kojo} på sidan \pageref{fig:appendix:ide:kojo}.

Det finns ett antal användbara kortkommando som du hittar i menyerna i Kojo Desktop. Undersök speciellt Ctrl+Alt+Mellanslag som ger autokomplettering baserat på det du börjat skriva.

\section{Kojo i Webbläsaren}

En begränsad variant av Kojo finns tillgänglig för programmering direkt i din webbläsare här: \url{http://kojo.lu.se/}

När du trycker på play-knappen så kompileras din kod på en server till Javascript via ScalaJS och därefter körs Javascript-koden i din webbläsare. 
Kojo på webben är också ännu så länge begränsad till Scala 2 och kräver att du omgärdar sekvenser av rader som hör ihop med \code|{| och \code|}|.


\section{Mer om Kojo}

I detta dokument finns en enkel introduktion till Kojo: \\ ''Introduction to Kojo'' \url{http://www.kogics.net/kojo-ebooks#intro}

\noindent I tabell \ref{table:kojo:functions}, som fortsätter på efterföljande sidor, finns ett urval av kommando i Kojo på svenska och engelska.

{\small\renewcommand{\arraystretch}{1.4}
\begin{longtable}{@{}p{0.42\textwidth} p{0.55\textwidth}}

\caption{Ett urval av funktioner i Kojo. Se även \href{http://lth.se/programmera}{lth.se/programmera}}\label{table:kojo:functions}\\

\emph{Svenska/Engelska} & \emph{Vad händer?}  \\ \hline
\code|sudda| \newline \code|clear| & Ritfönstret suddas \\
\code|fram| \newline \code|forward()| & Paddan går framåt 25 steg. \\
\code|fram(100)| \newline \code|forward(100)| & Paddan går framåt 100 steg. \\
\code|höger| \newline \code|right(90)| & Paddan vrider sig 90 grader åt höger. \\
\code|höger(45)| \newline \code|right(45)| & Paddan vrider sig 45 grader åt höger. \\
\code|vänster| \newline \code|left(90)| & Paddan vrider sig 90 grader åt vänster. \\
\code|vänster(45)| \newline \code|left(45)| & Paddan vrider sig 45 grader åt vänster. \\
\code|hoppa| \newline \code|hop| & Paddan hoppar 25 steg utan att rita. \\
\code|hoppa(100)| \newline \code|hop(100)| & Paddan hoppar 100 steg utan att rita. \\
\code|hoppaTill(100, 200)| \newline \code|jumpTo(100, 200)| & Paddan hoppar till läget (100, 200) utan att rita. \\
\code|gåTill(100, 200)| \newline \code|moveTo(100, 200)| & Paddan vrider sig och går till läget (100, 200). \\
\code|hem| \newline \code|home| & Paddan går tillbaka till utgångsläget (0, 0). \\
\code|öster| \newline \code|setHeading(0)| & Paddan vrider sig så att nosen pekar åt höger. \\
\code|väster| \newline \code|setHeading(180)| & Paddan vrider sig så att nosen pekar åt vänster. \\
\code|norr| \newline \code|setHeading(90)| & Paddan vrider sig så att nosen pekar uppåt. \\
\code|söder| \newline \code|setHeading(-90)  | & Paddan vrider sig så att nosen pekar neråt. \\
\code|mot(100,200)| \newline \code|towards(100, 200)| & Paddan vrider sig så att nosen pekar mot läget (100, 200) \\
\code|sättVinkel(90)| \newline \code|setHeading(90)| & Paddan vrider nosen till vinkeln 90 grader. \\
\code|vinkel| \newline \code|heading| & Ger vinkelvärdet dit paddans nos pekar. \\
\code|sakta(5000)| \newline \code|setAnimationDelay(5000) | & Gör så att paddan ritar jättesakta. \\
\code|suddaUtdata| \newline \code|clearOutput| & Utdatafönstret suddas. \\
\code|utdata("hej")| \newline \code|println("hej")| & Skriver texten \texttt{hej} i utdatafönstret. \\
\code|val t = indata("Skriv")| \newline \code|val t = readln("Skriv:")| & Väntar på inmatning efter ledtexten \texttt{Skriv} och sparar den inmatade texten i t.  \\
\code|textstorlek(100)| \newline \code|setPenFontSize(100)| & Paddan skriver med jättestor text nästa gång du gör skriv. \\
\code|båge(100, 90)| \newline \code|arc(100, 90)| & Paddan ritar en båge med radie 100 och vinkel 90. \\
\code|cirkel(100)| \newline \code|circle(radie)| & Paddan ritar en cirkel med radie 100. \\
\code|synlig| \newline \code|visible| & Paddan blir synlig. \\
\code|osynlig| \newline \code|invisible| & Paddan blir osynlig. \\
\code|läge.x| \newline \code|position.x| & Ger paddans x-läge \\
\code|läge.y| \newline \code|position.y| & Ger paddans y-läge \\
\code|pennaNer| \newline \code|penDown| & Sätter ner paddans penna så att den ritar när den går. \\
\code|pennaUpp| \newline \code|penUp| & Lyfter upp paddans penna så att den INTE ritar när den går. \\
\code|pennanÄrNere| \newline \code|penIsDown| & Kollar om pennan är nere eller inte. \\
\code|färg(rosa)| \newline \code|setPenColor(pink)| & Sätter pennans färg till rosa. \\
\code|fyll(lila)| \newline \code|setFillColor(purple)| & Sätter ifyllnadsfärgen till lila. \\
\code|fyll(genomskinlig)| \newline \code|setFillColor(noColor)| & Gör så att paddan inte fyller i något när den ritar. \\
\code|bredd(20)| \newline \code|setPenThickness(20)| & Gör så att pennan får bredden 20. \\
\code|sparaStil| \newline \code|saveStyle| & Sparar pennans färg, bredd och fyllfärg. \\
\code|laddaStil| \newline \code|restoreStyle| & Laddar tidigare sparad färg, bredd och fyllfärg. \\
\code|sparaLägeRiktning| \newline \code|savePosHe| & Sparar pennans läge och riktning \\
\code|laddaLägeRiktning| \newline \code|restorePosHe| & Laddar tidigare sparad riktning och läge \\
\code|siktePå| \newline \code|beamsOn| & Sätter på siktet. \\
\code|sikteAv| \newline \code|beamsOff| & Stänger av siktet. \\
\code|bakgrund(svart)| \newline \code|setBackground(black)| & Bakgrundsfärgen blir svart. \\
\code|bakgrund2(grön,gul)| \newline \code|setBackgroundV(green, yellow)| & Bakgrund med övergång från grönt till gult. \\
\code|upprepa(4){fram; höger}| \newline \code|repeat(4){forward; right}| & Paddan går fram och svänger höger 4 gånger. \\
\code|avrunda(3.99)| & Avrundar 3.99 till 4.0 \\
\code|slumptal(100)| & Ger ett slumptal mellan 0 och 99. \\
\code|slumptalMedDecimaler(100)| & Ger ett slumptal mellan 0 och 99.99999999 \\
\code|systemtid| & Ger nuvarande systemklocka i sekunder. \\
\code|räknaTill(5000)| & Kollar hur lång tid det tar för din dator att räkna till 5000. \\


\end{longtable}
}%end small


\hline
\end{longtable}
}%end small

\noindent Scala-koden för den svenska paddans api finns här: \\
\href{https://bitbucket.org/lalit_pant/kojo/src/tip/src/main/scala/net/kogics/kojo/lite/i18n/svInit.scala}{bitbucket.org/lalit\_pant/kojo/src/tip/src/main/scala/net/kogics/\\kojo/lite/i18n/svInit.scala}




\newpage

\section{Eclipse och ScalaIDE}\label{appendix:ide:eclipse}

Eclipse 
\footnote{\href{https://en.wikipedia.org/wiki/Eclipse_(software)}{en.wikipedia.org/wiki/Eclipse\_(software)}}
är en IDE som stödjer många olika programmeringsspråk. Eclipse är skriven i Java och bygger vidare på ett öppenkällkodsprojekt som initierades av IBM. Eclipse är ett fritt och öppet projekt som numera kontrolleras av en oberoende stiftelse.

Till Eclipse finns en insticksmodul \Eng{plug-in} som kallas ScalaIDE och erbjuder stöd för Scala med tillhörande standardbibliotek.

Eclipse är en omfattande och avancerad programmeringsmiljö med många funktioner och inställningar. Det finns även en omfattande uppsättning insticksmoduler och tilläggsprogram som underlättar utveckling av t.ex. webbprogram, databaser och mycket annat. 

I detta avsnitt ges länkar till installation samt tips om hur du kommer igång med att använda Eclipse och ScalaIDE. Det går ganska snabbt att lära sig grunderna, men det kräven en viss ansträngning att lära sig de mer avancerade funktionerna. Det finns omfattande resurser på nätet som hjälper dig vidare. 


\subsection{Installera Eclipse Mars och ScalaIDE}\label{appendix:ide:eclipse:install}

Eclipse med ScalaIDE är förinstallerat på LTH:s datorer och startas med kommandot \texttt{scalaide} i ett terminalfönster.
ScalaIDE fungerar med Eclipse-versionerna Luna och Mars (men i skrivande stund ännu \textit{inte} för den allra senaste versionen kallad Neon). 

Installationen görs enligt följande:

\begin{enumerate}
\item Kontrollera enligt avsnitt \ref{appendix:compile:check-jdk} att du har \texttt{java} installerat och installera vid behov JDK enligt avsnitt \ref{appendix:compile:install-jdk}.

\item Installera Eclipse version \textbf{Mars}, varianten för \textbf{Java Developers} som återfinns på denna sida: \\ \url{https://www.eclipse.org/downloads/packages/release/Mars/2} \\ som är den andra varianten i listan (alltså inte Java EE). Följ dessa steg:
\begin{enumerate}
\item Klicka på den \textbf{64-bit}-variant som passar ditt operativsystem.
\item Filen som laddas ner heter något som liknar (beroende på OS): \\ \texttt{eclipse-java-mars-2-win32-x86\_64.zip} 
\\ Det tar ett tag att ladda ner filen som är på ca 170MB. Om du klickar på ''select a mirror'' kan du välja en svensk sajt för att ladda ner snabbare. 

\item Dubbelklicka på filen för att packa upp den, vilket kan ta många minuter. Du får, när upppackningen är klar, ett bibliotek med en fungerande Eclipse-installation som du kan placera var du vill. Kör du Windows läggs den förslagsvis här:\\ 
\code|C:\Program Files (x86)\eclipse\eclipse-java-mars-2-win32-x86_64|

\item för Ubuntu Linux finns kompletterande installationsanvisningar här, som ger dig ikon i dash m.m.: 
\\ \url{http://askubuntu.com/questions/26632/how-to-install-eclipse}
\end{enumerate}

\item Installera Scala IDE inifrån Eclipse enligt nedan steg:
\begin{enumerate}
\item Starta Eclipse, t.ex. genom att köra igång den exekverbara filen som ligger i underbiblioteket \texttt{eclipse}, i Windows heter den \texttt{eclipse.exe} medan den exekverbara filen i Linux heter \texttt{eclipse} utan filändelse.

\item Välj någon plats för workspace (kvittar vilken just nu, kan ändras senare).

\item Klicka på menyn \textit{Help} $\rightarrow$ \textit{Install new software}.

\item Klicka på \textit{Add}-knappen till höger och skriv: \\ \textit{''ScalaIDE for Scala 2.11''} i \textit{Name}-fältet och ange denna adress i \textit{Location}-fältet: \\
  {\small\mbox{\url{http://download.scala-ide.org/sdk/lithium/e44/scala211/stable/site}}} \\
  och klicka \textit{OK}.
  
\item Du får nu upp en lista med alternativ. Kryssa för alternativet\\ \textit{Scala IDE for Eclipse} \\ och klicka \textit{Next} och sedan \textit{Next} och acceptera licensvillkoren och klicka \textit{Finnish}.

\item Låt installationen ta sin tid och starta sedan om Eclipse när installationen är färdig. 

\item När Eclipse är igång igen visas en dialog som föreslår att du ska köra \textit{Setup Diagnostics}. Gör detta och klicka för \textit{Use recommended default settings}. Följ även instruktionerna om hur du ändrar i filen \textit{eclipse.ini} för att ge Eclipse mer minne, vilket gör att Scala IDE fungerar smidigare.  \TODO Ge hjälp om vad sätta här...

\item Kompletterande information finns här, inklusive en video som visar installationsproceduren och hur man kommer igång med ett ''hello world''-program: \\ \url{http://scala-ide.org/download/current.html}


\end{enumerate}





\end{enumerate}

\subsection{Använda Eclipse och ScalaIDE}\label{appendix:ide:eclipse:use}

Ett grundläggande koncept i Eclipse är \textbf{workspace}, som betecknar ett bibliotek i ditt filsystem som utgör ett arbetsområde där du lagrar ett eller flera \textbf{projekt}. Eclipse genererar en specifik katalogstruktur i ditt workspace och lagra en hel mängd kataloger och filer som Eclipse använder när du editerar, kompilerar och kör dina programmeringsprojekt. Även de inställningar du gjort som påverkar hur Eclipse ser ut och fungerar lagras i ditt workspace.

\subsubsection{Ladda ner och importera projekt från kursens workspace}

TODO: skriv mer här

\begin{itemize}
\item Ladda ner kursens workspace här: \url{http://cs.lth.se/pgk/workspace}
\item Packa upp filen på lämpligt ställe.
\item Starta Eclipse med ScalaIDE-plugin (för installation se \ref{appendix:ide:eclipse:install}).
\item Bläddra till biblioteket du nyss packade upp, ungefär som i \ref{fig:eclipse:ide:open}
\begin{figure}[H]
\centering
\includegraphics[width=0.7\textwidth]{../img/pirates/selectws.png}
\caption { \emph{Öppna workspace:} Bläddra fram till kursens workspace och klicka {\bf OK. }}
\label{fig:eclipse:ide:open}
\end{figure}

\item Gå vidare från startskärmen genom att välja {\bf Workspace}, se Fig.\ref{fig:eclipse:ide:selectws}.
\begin{figure}[H]
\centering
\includegraphics[width=0.7\textwidth]{../img/pirates/selectws2.png} \\

\caption {Välj {\bf workspace}.}
\label{fig:eclipse:ide:selectws}
\end{figure}

\item Uppe till höger ser du vilken \emph{vy} du har. I Eclipse kommer vi växla mellan Scala, Java och debug-vyer. Dessa läggs till via listan som nås genom den fönsterliknande ikonen enligt Fig.~\ref{fig:eclipse:ide:changeview}. Du ska ha {\bf Scala} igång.

\begin{figure}[H]
\centering
\includegraphics[width=0.7\textwidth]{../img/pirates/selectscala.png} 

\caption {Lägg till vyer från listan med installerade plugin.}
\label{fig:eclipse:ide:changeview}
\end{figure}

\item Högerklicka i {\bf Project Explorer} och välj {\bf New} -> {\bf Scala Project}, se Fig.~\ref{fig:eclipse:ide:createproject}. Importera existerande project genom att genom att avmarkera \emph{Use default location} och bläddra till katalogen för respektive laboration och sen {\bf Finish}, se exempel i Fig.~\ref{fig:eclipse:ide:import} (namnet sätts automatiskt). Du kan också skapa nya projekt genom att ange ett projektnamn direkt och sen klicka {\bf Finish} enligt Fig.~\ref{fig:eclipse:ide:newproject}.

\begin{figure}[H]
\centering
\includegraphics[width=0.7\textwidth]{../img/pirates/createproject.png} 

\caption {Välj att {\bf Scala Project} i menyerna.}
\label{fig:eclipse:ide:createproject}
\end{figure}
\begin{figure}[H]
\centering
\includegraphics[width=0.5\textwidth]{../img/pirates/importproject.png} 

\caption {Importera existerande projekt genom att ange sökvägen.}
\label{fig:eclipse:ide:import}
\end{figure}

\begin{figure}[H]
\centering
\includegraphics[width=0.5\textwidth]{../img/pirates/nameproject.png} 

\caption {Skapa ett nytt projekt genom att ange namn.}
\label{fig:eclipse:ide:newproject}
\end{figure}


\item Du skapar nya klasser och objekt på liknande sätt, genom att högerklicka och välja {\bf New}, se exempel i Fig.~\ref{fig:eclipse:ide:createobject}


\begin{figure}[H]
\centering
\includegraphics[width=0.7\textwidth]{../img/pirates/createobject.png} 
\caption {Skapa ett nytt objekt via menyerna.}
\label{fig:eclipse:ide:createobject}
\end{figure}

\item Skriv ett main-program och exekvera det genom att markera klassen i {\bf Project Explorer} och sen klicka på den gröna pilen. Utskrifter kommer till konsolen längst ner. 

\begin{figure}[H]
\centering
\includegraphics[width=0.7\textwidth]{../img/pirates/exekvera.png} 
\caption {Exekvera med den gröna pilen.}
\label{fig:eclipse:ide:exec}
\end{figure}
\end{itemize}

\newpage

\subsection{IntelliJ IDEA}\label{appendix:ide:intellij}

IntelliJ IDEA%
\footnote{\href{https://en.wikipedia.org/wiki/IntelliJ_IDEA}{en.wikipedia.org/wiki/IntelliJ\_IDEA}}
 är en IDE som stödjer många olika programmeringsspråk. IntelliJ är skriven i Java och utvecklas av det tjeckiska företaget JetBrains. 

IntelliJ IDEA finns i två varianter: en gratis gemenskapsvariant med öppenkällkodslicens \Eng{Community edition}, samt en betalvariant med stängd källkod och support-tjänster.


Till IntelliJ IDEA finns en insticksmodul \Eng{plug-in} för Scala och dess tillhörande standardbibliotek.

IntelliJ IDEA är en omfattande och avancerad programmeringsmiljö med många funktioner och inställningar. Det finns även en omfattande uppsättning insticksmoduler och tilläggsprogram som underlättar utveckling av t.ex. webbprogram, databaser och mycket annat. 

I detta avsnitt ges länkar till installation samt tips om hur du kommer igång med att använda IntelliJ IDEA med Scala. Det går ganska snabbt att lära sig grunderna, men det kräven en viss ansträngning att lära sig de mer avancerade funktionerna. Det finns omfattande resurser på nätet som hjälper dig vidare. 



\subsection{Installera IntelliJ med Scala}\label{appendix:ide:intellij:install}

IntelliJ med Scala-plug-in är förinstallerat på LTH:s datorer och startas med kommandot \texttt{intellij} i ett terminalfönster.

\subsection{Använda IntelliJ}\label{appendix:ide:intellij:use}

