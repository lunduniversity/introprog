%!TEX encoding = UTF-8 Unicode
%!TEX root = ../compendium2.tex

\chapter{Integrerad utvecklingsmiljö}\label{appendix:ide}

\section{Vad är en integrerad utvecklingsmiljö?}

En integrerad utvecklingsmiljö \Eng{integrated development environment, IDE} samlar ett flertal verktyg för att skapa, köra och testa program, inklusive en avancerad \textbf{editor} och speciella felsökningsverktyg. % (se appendix \ref{appendix:compile})
 Det finns flera utvecklingsmiljöer att välja mellan, som kan användas för både Scala och Java.

En IDE ger stöd för \textbf{kodkomplettering} \Eng{code completion} där tillgängliga metoder visas i en lista och resten av ett namn kan fyllas i automatiskt efter att du skrivit de första bokstäverna i namnet. En IDE kan hjälpa dig med formatering och även skapa skelettkod utifrån \textbf{kodmallar} \Eng{code templates}. Med \textbf{felindikering} \Eng{error highlighting} får du understrykning av vissa fel direkt i koden och ibland kan du även få hjälp med förslag på åtgärder för att rätta till enkla fel. Funktioner för \textbf{avlusning} \Eng{debugging} hjälper dig att felsöka medan du kör din kod. Med funktioner för \textbf{omstrukturering} \Eng{refactoring} av kod får du hjälp av editorn i samarbete med kompilatorn att göra omfattande strukturförändringar i många kodfiler samtidigt, t.ex. namnbyten med hänsyn taget till språkets synlighetsregler.

Alla dessa avancerade funktioner kan öka produktiviteten avsevärt, men samtidigt tar de tid att lära sig och en IDE kan kräva mycket datorkraft och viss väntetid jämfört med en vanlig, fristående editor. I början kan all funktionalitet upplevas som överväldigande och det kan vara svårt att hitta i alla menyer och inställningar. Det är därför många som föredrar en fristående, snabbstartad kodeditor före en fullfjädrad, tungrodd IDE, speciellt om det rör ett mindre program. Å andra sidan kan en IDE med kodkomplettering vara till stor hjälp när man ska lära sig ett nytt api och experimentera med en okänd kodmassa. Med tiden har hanliga editorer, så som VS Code, fått allt fler  funktioner som tidigiare bara fanns i en fullfjädrad IDE\footnote{Se t.ex. LSP: \url{https://en.wikipedia.org/wiki/Language_Server_Protocol}}, och den praktiska skillnaden allt mindre mellan en ''vanlig'' editor och en IDE blir allt mindre.

I kursen använder vi flera utvecklingsmiljöer. På första labben använder vi Kojo (se appendix \ref{appendix:kojo}) som är en IDE speciellt anpassad på nybörjare. Sedan använder vi en editor t.ex. VS Code, gärna i kombination med byggverktyget \code{sbt}. Du kan under andra halvan av kursen välja att gå över från VS Code till att använda en (annan) IDE, men det går utmärkt att fortsätta med VS Code som numera har en bra debugger för Scala genom tillägget Metals. Om du vill använda en IDE i stället för VS Code så rekommenderas IntelliJ IDEA med Scala-plugin. 
Om du redan lärt dig Eclipse på djupet och verkligen vill fortsätta med denna IDE, välj då ScalaIDE -- dock har denna IDE inte hängt med i den tekniska utvecklingen och ligger kvar på Scala 2.12. 

%!TEX encoding = UTF-8 Unicode
%!TEX root = ../compendium2.tex

\section{Visual Studio Code med tillägget Scala Metals}\label{appendix:ide:vscode}

Visual Studio Code\footnote{\href{https://en.wikipedia.org/wiki/Visual\_Studio\_Code}{en.wikipedia.org/wiki/Visual\_Studio\_Code}}, förkortat VS Code eller bara \code{code}, är en gratis utvecklingsmiljö som är mestadels öppen källkod\footnote{Varianten VS Codium \url{https://vscodium.com/} är helt fri från stängd källkod och telemetri.}. Projektet startades och leds av Microsoft och har en aktiv gemenskap med många utvecklare och många användbara tillägg \Eng{extensions}.

VS Code kallas ofta för ''bara'' en editor, men har genom åren utvecklats till en fullfjädrad IDE med bl.a. inbyggd debugger och stöd för många olika språk via ett omfattande bibliotek av tillägg.%

\begin{itemize}
\item 
Läs mer om hur man använder VS Code här: \\
\url{https://code.visualstudio.com/docs}

\item
Läs mer om hur du använder Scala i VS Code här: \\
\url{https://scalameta.org/metals/docs/editors/vscode}

\end{itemize}

Det finns många användbara kortkommandon som gör dig snabbare och snabbare när du kodar, allteftersom du lär dig nya kortkommandon. Ett bra tips är att du lär dig minst ett nytt kortkommando om dagen och efter ett tag kan du riktigt många. Här finns en sammanfattning av de viktigaste kortkommandona för VS Code för Linux:\\
\url{https://code.visualstudio.com/shortcuts/keyboard-shortcuts-linux.pdf}\\
Byt ut \code{linux} mot \code{windows} eller \code{macos} i adressen ovan för motsvarande plattform.

\subsection{Installera VS Code och Metals}\label{appendix:ide:vscode:install}

VS Code är förinstallerad på LTH:s datorer, men du behöver själv installera Scala-tillägget \textbf{Metals} första gången du kör igång VS Code på LTH:s datorer. Läs om installation av Metals här: \\
\url{https://marketplace.visualstudio.com/items?itemName=scalameta.metals} 

Läs mer om hur du installerar VS Code på din egen dator här: \\\url{https://code.visualstudio.com}

Mer information om installation av verktyg finns på kursens hemsida: \\
\url{https://lunduniversity.github.io/pgk/#verktyg}

\begin{figure}
\centering
\includegraphics[width=1.0\textwidth]{../img/vscode-run}
\caption{Kör program genom att klicka på \textsf{run} ovanför huvudprogrammet. \label{appendix-ide:vscode-run}}
\end{figure}

\subsection{Köra program i VS Code}

Det finns olika sätt att köra igång huvudprogrammet i ett projekt i VS Code:

\begin{enumerate}
  \item Använd \texttt{scala run .} i ett separat terminalfönster. Läs mer om \texttt{scala-cli} i Appendix \ref{appendix:compile:scala-cli}.
  \item Kör igång \texttt{sbt} i ett separat terminalfönster och kör kommandot \texttt{run} inifrån \texttt{sbt}. Detta kräver att du har en giltig \texttt{build.sbt}, se Appendix \ref{appendix:build}.
  \item Köra igång program inifrån VS Code. Detta kräver att du öppnat katalogen med din kod med File-menyns ''Open Folder'', eller genom att du startar VS Code med \texttt{code .} överst i din projektkatalog (du ser att detta är gjort om nedre meddelandefältet är blått i stället för lila). Du behöver \textit{innan} du startar VS Code en första gång köra \texttt{scala setup-ide .} (se Appendix \ref{appendix:compile:scala-cli}), eller skapa en giltig \code{build.sbt}-fil (se Appendix \ref{appendix:build}) som du importerar i VS Code när frågan dyker upp i nedre högra hörnet.
  \item Kombinera Scala CLI eller \code{sbt} och VS Code. Kör detta kommando i ett separat terminalfönster: \texttt{scala compile . -w}~~ där \texttt{-w} betyder \emph{watch} och gör så att ändringar bevakas. Om du istället använder \texttt{sbt} kör \code{sbt ~compile} i ett separat terminalfönster (notera tilde-tecknet som gör att ändringar bevakas). Vid ändringsbevakning kommer kompileringsfel visas där varje gång du sparar en ändring i VS Code med Ctrl+S. När alla kompileringsfel är åtgärdade och du är redo att testköra så klickar du på \textsf{run}.
\end{enumerate}

Du ser att VS Code är beredd att köra igång ditt program genom att det (efter ett tag) kommer upp en extra rad ovanför ditt huvudprogram med texten \textsf{run|debug} och då kan du klicka på \textsf{run} för att köra ditt program. Utdata från körningen visas i en flik under koden. Observera att det kan ta lite tid för VS Code att förbereda allt som behövs för att kunna köra ditt program. Håll koll på om VS Code håller på med dessa förberedelser i det blåa meddelandefältet längst ned till höger. När allt är klart efter att du startat VS Code står det ''Index complete!'' bredvid en raketsymbol i meddelandefältet.

Om något krånglar och du inte får fram \textsf{run|debug} ovanför din \code{@main}-funktion, trots du har startat VS code enligt ovan, så prova att under Metals-fliken (ikonen med det stiliserade M:et i det gröna verktygsfältet) klicka på någon av ''Restart build server'' eller ''Import build'' (den senare tar längre tid men börjar om helt) och vänta tills det står ''Index Complete!'' i det blå meddelandefältet och då ska \textsf{run|debug} synas ovanför din \code{@main}-funktion.

\begin{figure}
\centering
\includegraphics[width=1.0\textwidth]{../img/vscode-debug}
\caption{Debuggern i VS Code. Nederkanten är orange när debuggern kör. \label{appendix-ide:vscode-debug}}
\end{figure}

\subsection{Använda debuggern i VS Code}

Innan du börjar använda debuggern, läs först om allmän felhantering i Appendix \ref{appendix:debug}.

Du kan aktivera debuggern i VS Code för dina Scala-program genom att klicka på ''debug'' ovanför din \code{main}-metod, förutsatt att du har tillägget Metals installerad i VS Code. Du behöver även köra \texttt{scala setup-ide .} en första gång (se Appendix \ref{appendix:compile:scala-cli}), eller ha en giltig \code{build.sbt}-fil (se Appendix \ref{appendix:build}) som du importerar i VS Code när frågan dyker upp i nedre högra hörnet. 

Figur \ref{appendix-ide:vscode-debug} på sidan \pageref{appendix-ide:vscode-debug} visar hur det kan se ut när debuggern i VS Code är aktiverad. När debuggern är igång får det nedersta meddelandefältet en orange färg (istället för blå). Till vänster om radnummerkolumnen kan du klicka för att aktivera och avaktivera brytpunkter. Aktiverade brytpunkter visas som en röd prick i marginalen till vänster. Den ihåliga gula pilen i marginalen pekar på den rad som kommer att exekveras härnäst. Notera panelen med olika knappar i överkanten av editorfönstret. Med dessa knappar kan du styra exekveringen enligt följande (lär dig gärna kortkommandona så blir du snabbare):
\begin{itemize}
  \item \textbf{Fortsätt}. Den blåa play-knappen kör vidare till nästa brytpunkt eller tills programmet är klart om brytpunkt ej påträffas. Kortkommando ''Continue'': F5.
  \item \textbf{Stega över}.Den blåa böjda framåtpilen kör en rad i taget \emph{utan} att hoppa in i funktioner.  Kortkommando ''Step Over'': F10.
  \item \textbf{Stega in}. Den blåa nedåtpilen kör vidare en rad i taget och hoppar in i funktioner om raden innehåller funktionsanrop. Kortkommando ''Step Into'': F11.
  \item \textbf{Stega ut}. Den blåa uppåtpilen kör klar innevarande funktion. Kortkommando ''Step Out'': Shift+F11.
  \item \textbf{Kör igen}. Den gröna återstartsikonen kör om ditt program. Kortkommando ''Restart'': Ctrl+Shift+F5.
  \item \textbf{Avbryt}. Den röda stoppknappen avbryter denna debuggingsession. Kom ihåg att avbryta innan du startar en ny debuggingsession, annars kan det lätt bli förvirrande med många samtidigt pågående körningar. Kortkommando ''Stop'': Shift+F5. 
\end{itemize}

Figur \ref{appendix-ide:vscode-trace} på sidan \pageref{appendix-ide:vscode-trace} visar hur VS Code presenterar anropsstacken och värdet på de variabler som syns där exekveringen befinner sig för tillfället. Du får fram detta genom att klicka på ikonen med en lus och en playknapp i det vertikala, gröna verktygsfältet längst till vänster. I en blå ring står en etta om du har startat en debuggingsession. Om det står en tvåa eller mer så har du flera sessioner igång och då kan det vara klokt att avsluta alla utom en, så att inte förvirring uppstår om vilken session som är den aktuella. 

\begin{figure}
\centering
\includegraphics[width=1.0\textwidth]{../img/vscode-trace}
\caption{Anropsstack och variabler i VS Code.\label{appendix-ide:vscode-trace}}
\end{figure}

Mycket av konsten i debugging handlar om att undersöka variablers värde under exekveringen för att ta reda på om din hypotes om vad som händer under exekvering verkligen stämmer, eller om något egentligen inte fungerar så som du antar. Detta kan du med fördel göra genom att placera brytpunkter på relevanta ställen. Även vid användning av en debugger kan du ha stor nytta av att göra \code{println} av intressanta uttryck för att i detalj undersöka vad som egentligen händer. Läs mer om debugging i Appendix \ref{appendix:debug}.
 

%!TEX encoding = UTF-8 Unicode
%!TEX root = ../compendium2.tex

\section{JetBrains IntelliJ IDEA med Scala-plugin}\label{appendix:ide:intellij}

IntelliJ IDEA%
\footnote{\href{https://en.wikipedia.org/wiki/IntelliJ_IDEA}{en.wikipedia.org/wiki/IntelliJ\_IDEA}}
 är en professionell IDE som stödjer många olika programmeringsspråk. IntelliJ är skriven i Java och utvecklas av det tjeckiska företaget JetBrains.

IntelliJ IDEA finns i två varianter: en gratis gemenskapsvariant med öppenkällkodslicens \Eng{Community edition}, samt en betalvariant med sluten källkod och support-tjänster.

\begin{figure}
\centering
\includegraphics[width=1.0\textwidth]{../img/intellij/idea-hello}
\caption{Den integrerade utvecklingsmiljön Intellij IDEA.\label{appendix-ide:intellij-hello}}
\end{figure}

IntelliJ IDEA är en omfattande och avancerad programmeringsmiljö med många funktioner och inställningar. Det finns även en omfattande uppsättning insticksmoduler och tilläggsprogram som underlättar utveckling av t.ex. mobilappar, webbprogram, databaser och mycket annat.

Till IntelliJ IDEA finns en insticksmodul \Eng{plug-in} som stöd för Scala med tillhörande standardbibliotek och byggverktyget \code{sbt}, med mera. Scala-insticksmodulen kan inkluderas genom att välja Scala i en av de dialoger som visas vid första körningen, enligt instruktioner nedan.

I detta avsnitt ges länkar till installation samt tips om hur du kommer igång med att använda IntelliJ IDEA med Scala. Det går ganska snabbt att lära sig grunderna, men det kräven en viss ansträngning att lära sig de mer avancerade funktionerna. Det finns omfattande resurser på nätet som hjälper dig vidare.

Google tillkännagav 2013 att företaget övergår från Eclipse till IntelliJ som den officiellt understödda utvecklingsmiljön för Android och 2014 lanserades utvecklingsmiljön Android Studio%
\footnote {\href{https://en.wikipedia.org/wiki/Android_Studio}{en.wikipedia.org/wiki/Android\_Studio}}
 som bygger vidare på IntelliJ.

\subsection{Installera IntelliJ IDEA}\label{appendix:ide:intellij:install}

IntelliJ med Scala-plug-in är förinstallerat på LTH:s datorer och startas med kommandot \texttt{idea} i ett terminalfönster.

Du kan installera IntelliJ på din egen dator genom att följa instruktionerna för ditt operativsystem (Windows/macOS/Linux) här: \\
\url{https://www.jetbrains.com/help/idea/run-for-the-first-time.html}


Du behöver Scala-plugin som du kan välja under installationen av IntelliJ, men det går också att installera plugin för Scala i efterhand, se vidare här:\\
\url{https://www.jetbrains.com/help/idea/discover-intellij-idea-for-scala.html} 



% \newpage

% %!TEX encoding = UTF-8 Unicode
%!TEX root = ../compendium2.tex

\section{ScalaIDE och Eclipse}\label{appendix:ide:eclipse}

\begin{oframed}
  \noindent \textbf{OBS!} Eclipse-stödet för Scala har på senare tid inte hängt med i den tekniska utvecklingen och Eclipse med ScalaIDE \ScalaIDEVersion~ligger kvar på gamla Scala 2.12. Om du inte av speciella skäl (t.ex. att du redan lärt dig Eclipse på djupet) vill använda Eclipse, så rekommenderas i stället VS Code eller IntelliJ.  
  \end{oframed}
  

Eclipse%
\footnote{\href{https://en.wikipedia.org/wiki/Eclipse_(software)}{en.wikipedia.org/wiki/Eclipse\_(software)}}
är en professionell IDE som stödjer många olika programmeringsspråk. Eclipse är skriven i Java och bygger vidare på ett utvecklingsprojekt som initierades av IBM. Eclipse är ett fritt och öppet projekt som numera kontrolleras av en oberoende stiftelse.

Till Eclipse finns en insticksmodul \Eng{plug-in} som kallas ScalaIDE och erbjuder stöd för Scala med tillhörande standardbibliotek.

Eclipse är en omfattande och avancerad programmeringsmiljö med många funktioner och inställningar. Det finns även en omfattande uppsättning insticksmoduler och tilläggsprogram som underlättar utveckling av t.ex. webbprogram, databaser och mycket annat.

I detta avsnitt ges länkar till installation samt tips om hur du kommer igång med att använda Eclipse och ScalaIDE. 

%Det går ganska snabbt att lära sig grunderna, men det kräven en viss ansträngning att lära sig de mer avancerade funktionerna. Det finns omfattande resurser på nätet som hjälper dig vidare.


\subsection{Installera ScalaIDE}\label{appendix:ide:eclipse:install}

Eclipse med ScalaIDE är förinstallerat på LTH:s datorer och startas med kommandot \texttt{scalaide} i ett terminalfönster.
För att installera ScalaIDE på din egen dator, följ nedan instruktioner:

\begin{enumerate}
\item Kontrollera enligt avsnitt \ref{appendix:compile:check-jdk} att du har \texttt{java} installerat och installera vid behov JDK enligt avsnitt \ref{appendix:compile:install-jdk}.

\item Installera senaste version av ScalaIDE (i skrivande stund \ScalaVersion för Eclipse Oxygene och Scala 2.12.3) från denna sida:  \url{http://scala-ide.org/download/sdk.html} \\
Följ dessa steg:
\begin{enumerate}
\item Klicka på den variant som passar ditt operativsystem.
\item Filen som laddas ner heter något som liknar (beroende på OS): \\ till exempel  \emph{Windows 64-bit}
\\ Det kan ta ett tag att ladda ner filen som är på ca 280MB.

\item Dubbelklicka på filen för att packa upp den, vilket kan ta många minuter. Du får, när up	packningen är klar, en katalog med namnet \code{eclipse} med en fungerande Eclipse-installation som du kan placera var du vill.
\end{enumerate}

\item Kör igång ScalaIDE första gången för att se så det fungerar genom att dubbelklicka på den exekverbara filen som ligger i underkatalogen \texttt{eclipse}. I Windows heter den \texttt{eclipse.exe} medan den exekverbara filen i Linux heter \texttt{eclipse} utan filändelse.

% \item För Ubuntu Linux finns kompletterande installationsanvisningar här, som beskriver hur man kan skapa en ikon i app-menyn m.m.:
% \\ \url{http://askubuntu.com/questions/26632/how-to-install-eclipse}
% \\ Instruktionerna för Linux å länken ovan är nedan anpassade för version 4.6.1 av ScalaIDE som en serie kommandon att köra i terminalen. Adressen är hämtad från \url{http://scala-ide.org/download/sdk.html} genom att högerklicka på nedladdningsknappen och välja kopiering av adressen. Denna adress står efter \code{wget} nedan (anpassa adressen vid behov till nyare version om sådan kommit sedan detta skrevs). Klistra in ett kommando i taget och kontrollera att det inte blir några fellmeddelande. Nedladdningen med \texttt{wget} kan ta ett tag.
% \begin{verbatim}
% $ cd ~/Downloads
% $ wget http://downloads.typesafe.com/scalaide-pack/\
% 4.6.1-vfinal-neon-212-20170609/scala-SDK-4.6.1-vfinal-\
% 2.12-linux.gtk.x86_64.tar.gz
% $ tar -zxvf scala-SDK-4.6.1-vfinal-2.12-linux.gtk.x86_64.tar.gz
% $ sudo mv ~/Downloads/eclipse /opt/scalaide46  # ändra 46 vid behov
% $ cat > ~/Desktop/scalaide.desktop <<EOF
% [Desktop Entry]
% Name=ScalaIDE
% Type=Application
% Exec=env UBUNTU_MENUPROXY=0 scalaide
% Terminal=false
% Icon=scalaide
% Comment=Integrated Development Environment
% NoDisplay=false
% Categories=Development;IDE;
% Name[en]=ScalaIDE
% EOF
% $ chmod +x ~/Desktop/scalaide.desktop
% $ sudo ln -s /opt/scalaide46/eclipse /usr/local/bin/scalaide
% $ sudo desktop-file-install ~/Desktop/scalaide.desktop
% $ sudo cp /opt/scalaide46/icon.xpm /usr/share/pixmaps/scalaide.xpm
% \end{verbatim}
\end{enumerate}


\subsection{Använda ScalaIDE och Eclipse}\label{appendix:ide:eclipse:use}

Ett grundläggande koncept i Eclipse är \textbf{workspace}. Ett workspace utgör ett arbetsområde kopplat till en katalog i ditt filsystem där du kan arbeta med ett eller flera \textbf{projekt}. Ett projekt innehåller i sin tur dina källkodsfiler och klassfiler etc. i en specifik katalogstruktur som Eclipse skapar när du editerar, kompilerar och kör dina projekt.

\subsubsection{Starta och välja workspace}\label{subsubsection:start:eclipse}

När du startar Eclipse måste du välja vilket workspace du vill använda innan du kommer vidare. När du kör igång Eclipse första gången, klicka OK enligt det förslag som ges. Du kan senare växla workspace genom menyn \MenuArrow{File}\Menu{Switch Workspace}. Om katalogen du anger inte redan finns, kommer den att skapas och initieras med de filer Eclipse behöver.

% I figur \ref{fig:appendix:eclipse:welcome} visas välkomstfliken i Eclipse med sina länkar till funktionsöversikt och olika handledningar. Stäng välkomstfliken genom att klicka på flikens kryss eller på ikonen \textit{Workbench}. Då kommer du vidare till den normala arbetsytan i Eclipse. Du kan få tillbaka välkomstfliken igen via menyn \MenuArrow{Help}\Menu{Welcome}.
%
% \begin{figure}[H]
% \centering
% \includegraphics[width=1.0\textwidth]{../img/eclipse/eclipse-welcome.png}
% \caption{Välkomstfliken för Eclipse, som nås via menyn \MenuArrow{Help}\Menu{Welcome}. Gå vidare genom att klicka på \textit{Workbench}.}
% \label{fig:appendix:eclipse:welcome}
% \end{figure}

\subsubsection{Välja perspektiv och visa olika vyer}

Eclipse-fönstret kan innehålla många underfönster i olika flikar, så kallade \textbf{views} eller vyer, som kan arrangeras på olika vis efter hur du vill ha dem. Vilka vyer som syns och hur de placeras beror på vilket s.k. \textbf{perspective} som är aktivt.  Figur \ref{fig:appendix:eclipse:open-perspective} visar arbetsytan med olika vyer i Scala-perspektivet.

Stäng vyn \textit{Outline} om du vill ha mer plats till de övriga vyerna för paketnavigering, editering och utdata. Du kan öppna stängda vyer igen genom menyn \MenuArrow{Window}\Menu{Show View}.
Du kan även återställa perspektivet om din vy blivit konstig med \MenuArrow{Window}\MenuArrow{Perspective}\MenuArrow{Reset Perspective...}

Om du vill anpassa arbetsytan för Java kan du byta perspektiv med klick på \includegraphics[scale=0.75]{../img/eclipse/eclipse-perspective-button.png} eller genom menyn \MenuArrow{Window}\MenuArrow{Perspective}\MenuArrow{Open Perspective}\MenuArrow{Other...}\Menu{Java}.

\begin{figure}
\centering
\includegraphics[width=1.0\textwidth]{../img/eclipse/eclipse-scala-perspective.png}
\caption{Arbetsytan i Eclipse. Du kan växla mellan Scala-perspektivet och andra perspektiv, t.ex. Debug-perspektivet eller Java-perspektivet genom att klicka på knappen \Menu{Open Perspective}.}
\label{fig:appendix:eclipse:open-perspective}
\end{figure}

\subsubsection{Hello World}\label{subsubsection:eclipse:hello-world}

Efter att du öppnat ScalaIDE i ett tomt workspace med Scala-perspektivet enligt föregående avsnitt, kan du skapa ditt första projekt med ett \textit{''Hello World''}-program enligt stegen nedan.

\begin{enumerate}
\item Högerklicka i \Menu{Package Explorer} och välj \MenuArrow{New}\Menu{Scala Project}, varefter en dialogruta visas.

\item Fyll i namnet \texttt{hello} i fältet \Menu{Project Name} och klicka \Button{Finish}. Vänta tills skapandet av projektet är klart enligt notifieringar i fönstrets nederkant.

\item Högerklicka igen i \Menu{Package Explorer} och välj \MenuArrow{New}\Menu{Scala Object}, varefter en ny dialogruta visas.

\item Fyll i namnet \texttt{hi} i fältet \Menu{Name} och klicka \Button{Finish}.

\item Du får nu i editorvyn ett kodskelett med \code{object hi}.

\item Börja skriv \code{main} som visas i figur \ref{fig:appendix:eclipse:complete-main} och tryck Ctrl+Mellanslag för att aktivera kodkomplettering \Eng{code completion}. Då får du upp en lista med många alternativ. Välj alternativet \texttt{main - main} varefter ett kodskelett med en main-metod klistras in automatiskt i din kod.

\begin{figure}
\centering
\includegraphics[width=1.0\textwidth]{../img/eclipse/eclipse-complete-main.png}
\caption{Aktivera kodkomplettering med Ctrl+Mellanslag efter ordet \code{main}.}
\label{fig:appendix:eclipse:complete-main}
\end{figure}

\item Fyll i lämplig utskriftstext i ett \code{println}-anrop så att din \code{main}-metod blir så som visas i editorfliken i figur \ref{fig:appendix:eclipse:hello-world}.

\begin{figure}[H]
\centering
\includegraphics[width=1.0\textwidth]{../img/eclipse/eclipse-hello-world.png}
\caption{Skriv klart \code{main}-metoden och kör ditt program med play-knappen.}
\label{fig:appendix:eclipse:hello-world}
\end{figure}

\item Kör ditt program genom att klicka på den lilla ner-pilen som finns \emph{bredvid} den gröna play-knappen, som muspekaren i figur \ref{fig:appendix:eclipse:hello-world} pekar på. Då vecklas en meny ut där du kan välja \MenuArrow{Run As...}\Menu{Scala Application}. Eller så kan du höger-klicka på filen \texttt{hi.scala} och välja \MenuArrow{Run As...}\Menu{Scala Application}. Detta behöver du bara göra första gången och då skapas en s.k. \textit{Run Configuration}. Du kan sedan trycka Ctrl+F11 för att köra igång din app enligt senaste \textit{Run Configuration} eller klicka på den gröna play-knappen.

\item Du kan öppna din \textit{Run Configuration} genom menyn  \MenuArrow{Run}\Menu{Run Configurations...} och där ställa in många saker, bland annat vilka argument som ska ges till \code{main}-metoden under fliken \textit{Arguments} i textrutan \textit{Program Arguments}.

\end{enumerate}



\subsection{Anpassa ScalaIDE och Eclipse}\label{subsection:appendix:ide:eclipse:tweaks}

\newcommand\EclipsePrefs{\MenuArrow{Window}\MenuArrow{Preferences}}
\newcommand\EclipsePrefsGeneral{\EclipsePrefs\MenuArrow{General}}


%Förutom maxminneshöjningen i filen \texttt{eclipse.ini}, som finns i installationskatalogen för Eclipse, till minst \texttt{-Xmx1G}
Det kan vara bra att göra några ytterligare anpassningar av Eclipse och ScalaIDE enligt nedan. Du hittar inställningarna i menyn \EclipsePrefs ... uppe till höger i Eclipse-fönstret.

\begin{enumerate}
% \item \EclipsePrefsGeneral
% \\ Markera \FramedCheckmark{Show Heap Status} så får du se minnesanvändningen i en liten ruta i nederdelen av fönstret, vilket hjälper dig att upptäcka om minnesbegränsningen i filen \texttt{eclipse.ini} är en flaskhals vid stora projekt och många öppna fönster. Klicka sedan \Button{Apply} längst ner.

% \item \label{item:scala-perspective} \EclipsePrefsGeneral\MenuArrow{Editors}\MenuArrow{Perspectives}
% \\ Markera \textit{Scala} i listan med perspektiv och klicka på knappen
%  \\ \Button{Make default} till höger och sedan på knappen \Button{Apply} längst ner.

\item \EclipsePrefsGeneral\MenuArrow{Editors}\MenuArrow{TextEditors}
\\ Markera \FramedCheckmark{Insert spaces for tabs} så att du slipper specialtecken som kan tolkas olika av olika editorer. Klicka sedan \Button{Apply} längst ner.

\item \EclipsePrefsGeneral\MenuArrow{Editors}\MenuArrow{TextEditors}
\\ \MenuArrow{Spelling} Avmarkera \FramedUnchecked{Enable spell checking} för att slippa att svenska namn och svenska kommentarer markeras som felstavade. Om du senare jobbar med ett projekt helt på engelska, kan du med fördel markera denna kryssruta igen. Klicka sedan \Button{Apply} längst ner.

\item \EclipsePrefsGeneral\MenuArrow{Editors}\MenuArrow{Webbrowser}
\\ Markera \FramedCheckmark{Use external web browser} för att köra din vanliga webbläsare när du klickar på länkar. Klicka sedan \Button{Apply} längst ner.

\item  \EclipsePrefs\MenuArrow{Scala}\MenuArrow{Compiler}
\\ I fliken \textbf{Standard} markera dessa kryssrutor för att få extra varningar: \\
\begin{tabular}{l @{}l @{}l}
\textit{deprecation} & \FramedCheckmark{} & varnar vid användning av föråldrad kod som snart utgår \\
\textit{feature}     & \FramedCheckmark{} & påminner om import vid användning av avancerad kod  \\
\textit{unchecked}   & \FramedCheckmark{} & ger tips vid speciella problem med generiska typer \\
\end{tabular}\\
och klicka sedan på knappen \Button{Apply} längst ner.

\item \EclipsePrefs\MenuArrow{Java}\MenuArrow{Compiler}\MenuArrow{Errors/Warnings}
\\ Veckla ut listan \textbf{Potential programming problems} och sätt \textbf{Resource leak} till alternativet \textbf{Ignore}, så slipper du varningar vid användning \jcode{Scanner} i Java. Klicka sedan \Button{Apply} längst ner.

\end{enumerate}

\noindent Ovan anpassningar är rekommenderade men inte nödvändiga och du kan gärna välja att göra andra anpassningar som passar just dig. Skriv då gärna ner vilken inställning du ändrat, så att du hittar tillbaka om du ångrar dig.

%Du hittar tips om fler inställningar för att anpassa ScalaIDE här: \\
%\url{http://scala-ide.org/docs/current-user-doc/advancedsetup}






\subsubsection{Ladda ner kursens workspace och importera i Eclipse}\label{subsubsection:download--import-workspace}

Det finns en zip-fil med ett workspace med projekt för flera av kursens laborationer som du kan ladda ner och importera i Eclipse. Följ stegen nedan.

\begin{enumerate}
\item Ladda ner kursens workspace här: \url{https://cs.lth.se/pgk/ws}

\item Packa upp filen på lämpligt ställe, t.ex. i katalogen \texttt{eclipse/pgk/workspace/}

\item Starta Eclipse med ScalaIDE-plugin (se startinstruktioner på sidan \pageref{subsubsection:start:eclipse}).

\item Växla workspace till biblioteket du nyss packade upp, ungefär som i figur \ref{fig:eclipse:ide:open} och klicka \Button{OK}.

\begin{figure}[H]
\centering
\includegraphics[width=0.8\textwidth]{../img/eclipse/eclipse-select-workspace.png}
\caption {Öppna kursens workspace genom att bläddra till katalogen där du packade upp filen som du laddat ned från: \url{https://cs.lth.se/pgk/ws} \\Om du redan har ett annat workspace öppet, man du växla workspace med menyn \MenuArrow{File}\MenuArrow{Switch Workspace}\Menu{Other...}}
\label{fig:eclipse:ide:open}
\end{figure}

\item
%Stäng välkomstfliken för att komma vidare till workbench (se figur \ref{fig:appendix:eclipse:welcome} på sidan \pageref{fig:appendix:eclipse:welcome}).
Det ska nu se ut ungefär som i figur~\ref{fig:appendix:eclipse:open-perspective} på sidan \pageref{fig:appendix:eclipse:open-perspective}. Det syns ännu inget i \textit{Package Explorer} då vi ännu inte importerat något projekt.

\item Innan du går vidare, säkerställ att du har Scala-perspektivet aktiverat. Du kan växla till Scala-perspektivet genom att trycka på \includegraphics[scale=0.75]{../img/eclipse/eclipse-perspective-button.png} eller genom menyn \MenuArrow{Window}\MenuArrow{Perspective}\MenuArrow{Open Perspective}\MenuArrow{Other...}\Menu{Scala}.
%Du kan anpassa inställningarna så att Scala blir \textit{default perspective}, se steg \ref{item:scala-perspective} i avsnitt \ref{subsection:appendix:ide:eclipse:tweaks} på sidan \pageref{subsection:appendix:ide:eclipse:tweaks}.


\item Högerklicka i \textit{Package Explorer} och välj \Menu{Import...}, se Fig.~\ref{fig:eclipse:import}, eller välj menyn \MenuArrow{File}\Menu{Import...}.

\begin{figure}[h]
\centering
\includegraphics[width=0.9\textwidth]{../img/eclipse/eclipse-import.png}
\caption {Välj \Menu{Import}-menyn för att importera existerande projekt.}
\label{fig:eclipse:import}
\end{figure}

\item Nu öppnas \Menu{Import}-dialogen som visas i figur \ref{fig:eclipse:import-existing}. Öppna mappen \Menu{General}, markera \textbf{Existing Projects into Workspace} och klicka \Button{Next}.



\begin{figure}[h]
\centering
\includegraphics[width=0.6\textwidth]{../img/eclipse/eclipse-import-existing.png}
\caption {Välj att importera existerande projekt under \Menu{General}.}
\label{fig:eclipse:import-existing}
\end{figure}


\begin{figure}[h]
\centering
\includegraphics[width=0.85\textwidth]{../img/eclipse/eclipse-import-projects.png}
\caption {Välj \FramedCheckmark{Select Root Directory} (anpassa sökvägen till katalogen där lagt den uppackade katalogen med kursens workspace) och klicka \Button{Browse} och därefter \Button{Ok}. Då kommer katalogerna upp i listan som bilden ovan visar. Alla kataloger ska vara förvalda. Avsluta med att klicka \Button{Finish}}
\label{fig:eclipse:import-projects}
\end{figure}

\item Nu kommer ytterligare ett dialogfönster som visas i figure \ref{fig:eclipse:import-projects}. Följ instruktionerna i bildtexten.

% \item Följ ''Hello World''-instruktionerna på sidan \pageref{subsubsection:eclipse:hello-world} och skapa programmet som visas i figure \ref{fig:eclipse:pirates-hi}, genom att veckla ut projektet \textbf{w04\_pirates}, markera och högerklicka på paketet \textbf{priates}, och välja \MenuArrow{New}\Menu{Scala Object}.

\item Efter att du klickat \Button{Finish} sätter ScalaIDE igång att bygga workspace i bakgrunden. Detta kan ta ett tag. Du kan följa bygget nere till höger i fönstret:
\includegraphics[width=0.75\textwidth]{../img/eclipse/scalaide-import-progress.png}

\item När bygget är klart kan du köra huvudprogrammet i laboration \texttt{w08\_life} genom att högerklicka på filen \texttt{src/main/scala/life/Main.scala} och välja \MenuArrow{Run As}\Menu{Scala Application}.

%\item Om du får problem, fråga någon som känner till Eclipse om hjälp. Det finns även mycket hjälp på nätet, se till exempel: \\ \href{http://stackoverflow.com/questions/8522149/eclipse-not-recognizing-scala-code}{stackoverflow.com/questions/8522149/eclipse-not-recognizing-scala-code}

% \begin{figure}[H]
% \centering
% \includegraphics[width=1.0\textwidth]{../img/eclipse/eclipse-pirates-hello.png}
% \caption {Skapa ett \MenuArrow{New}\Menu{Scala Object} med kod enligt bilden.}
% \label{fig:eclipse:pirates-hi}
% \end{figure}


\end{enumerate}

%
% \subsection{Använda debuggern i ScalaIDE/Eclipse}
%
% !!! Läs först appendix \ref{appendix:debug}
%
% \subsubsection{Sätta brytpunkter i Eclipse}\TODO
% \subsubsection{Stegad exekvering i Eclipse}\TODO
% \subsubsection{Inspektera variabler i Eclipse}\TODO

