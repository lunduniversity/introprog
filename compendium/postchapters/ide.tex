%!TEX encoding = UTF-8 Unicode
%!TEX root = ../compendium1.tex

\chapter{Integrerad utvecklingsmiljö}\label{appendix:ide}

\section{Vad är en integrerad utvecklingsmiljö?}

En integrerad utvecklingsmiljö \Eng{integrated development environment, IDE} samlar ett flertal verktyg, inklusive en avancerad \textbf{editor} (se appendix \ref{appendix:edit}), för att skapa, köra och testa program. Det finns flera utvecklingsmiljöer att välja mellan, som kan användas för både Scala och Java.

En IDE ger stöd för \textbf{kodkomplettering} \Eng{code completion} där tillgängliga metoder visas i en lista och resten av ett namn kan fyllas i automatiskt efter att du skrivit de första bokstäverna i namnet. En IDE kan hjälpa dig med formatering och även skapa skelettkod utifrån \textbf{kodmallar} \Eng{code templates}. Med \textbf{felindikering} \Eng{error highlighting} får du understrykning av vissa fel direkt i koden och ibland kan du även få hjälp med förslag på åtgärder för att rätta till enkla fel. Funktioner för \textbf{avlusning} \Eng{debugging} hjälper dig att felsöka medan du kör din kod. Med funktioner för \textbf{omstrukturering} \Eng{refactoring} av kod får du hjälp av editorn i samarbete med kompilatorn att göra omfattande strukturförändringar i många kodfiler samtidigt, t.ex. namnbyten med hänsyn taget till språkets synlighetsregler.  

Alla dessa avancerade funktioner kan öka produktiviteten avsevärt, men samtidigt tar de tid att lära sig och en IDE kan kräva mycket datorkraft och viss väntetid jämfört med en vanlig, fristående editor. I början kan all funktionalitet upplevas som överväldigande och det kan vara svårt att hitta i alla menyer och inställningar. Ska man bara skriva ett litet, enkelt program, eller göra några mindre ändringar, är det många som föredrar en fristående, snabbstartad kodeditor före en fullfjädrad, tungrodd IDE. Å andra sidan kan en IDE med kodkomplettering vara till stor hjälp när man ska lära sig ett nytt api och experimentera med en okänd kodmassa.

I kursen använder vi flera utvecklingsmiljöer. På första labben använder vi Kojo (avsnitt \ref{appendix:ide:kojo}) som är en IDE speciellt anpassad på nybörjare. I laborationerna senare i kursen kan du välja att använda någon av de professionella utvecklingsmiljöerna Eclipse (avsnitt \ref{appendix:ide:eclipse}) eller IntelliJ (avsnitt \ref{appendix:ide:intellij}). Om du inte vet vilken du ska välja, börja med att prova Eclipse.


\newpage

\section{Kojo}\label{appendix:ide:kojo}

Kojo%
\footnote{\href{https://en.wikipedia.org/wiki/Kojo_(programming_language)}{en.wikipedia.org/wiki/Kojo\_(programming\_language)}}
 är en integrerad utvecklingsmiljö för Scala som är speciellt anpassad för nybörjare i programmering. Kojo används i LTH:s Science Center Vattenhallen för utbildning av grundskolelärare i programmering och vid skolbesök och annan besöksverksamhet, i vilken lärare och studenter vid LTH arbetar som handledare. Kojo är fri öppenkällkod och utvecklingsgemenskapen leds av Lalit Pant från Indien.

Kursens första laboration genomförs med hjälp av Kojo, men Kojo kan med fördel användas som komplement till Scala REPL och annan IDE under hela kursens gång. Medan Scala REPL lämpar sig för korta kodsnuttar, och en fullfjädrad, professionell IDE har funktioner för att hantera riktigt stora programmeringsprojekt, passar Kojo bra för mellanstora program. I Kojo finns även lättillgängliga bibliotek som gör tröskeln lägre att programmera rörlig grafik och enkla spel.   


\begin{figure}[H]
\centering
\includegraphics[width=0.8\textwidth]{../img/kojo/kojo.png}
\caption{Den nybörjarvänliga utvecklingsmiljön Kojo för Scala på svenska.}
\label{fig:appendix:ide:kojo}
\end{figure} 

\subsection{Installera Kojo}\label{appendix:ide:kojo:install}

Kojo är förinstallerat på LTH:s datorer och körs igång med kommandot \texttt{kojo}. För instruktioner om hur du installerar Kojo på din egen dator se här:\\
\href{http://www.lth.se/programmera/installera/}{lth.se/programmera/installera}

Kojo kräver att \texttt{java} finns på din dator. Eftersom du behöver tillgång till JDK i kursen, är det lika bra att installera hela JDK direkt (och inte bara JRE, så som beskrivs å länken ovan); se vidare hur du gör detta i avsnitt \ref{appendix:compile:install-jdk}. 
%\href{http://www.kogics.net/kojo-download}{www.kogics.net/kojo-download}


\subsection{Använda Kojo}

När du startar Kojo första gången, välj ''Svenska'' i språkmenyn och starta om Kojo. Därefter fungerar grafikfunktionerna på svenska enligt tabell \ref{table:kojo:functions}. När du startat om Kojo inställt på svenska ser programmet ut ungefär som i figur \ref{fig:appendix:ide:kojo} på sidan \pageref{fig:appendix:ide:kojo}.


Det finns ett antal användbara kortkommando som du hittar i menyerna i Kojo. Undersök speciellt Ctrl+Alt+Mellanslag som ger autokomplettering baserat på det du börjat skriva.


{\small\renewcommand{\arraystretch}{1.45}
\begin{longtable}{@{}p{0.42\textwidth} p{0.55\textwidth}}

\caption{Några av sköldpaddans funktioner. Se även \href{http://lth.se/programmera}{lth.se/programmera}}\label{table:kojo:functions}\\

\emph{Svenska/Engelska} & \emph{Vad händer?}  \\ \hline
%!TEX encoding = UTF-8 Unicode
%!TEX root = ../compendium2.tex

\chapter{Kojo}\label{appendix:kojo}

\section{Vad är Kojo?}

Kojo%
\footnote{\href{https://en.wikipedia.org/wiki/Kojo_(programming_language)}{en.wikipedia.org/wiki/Kojo\_(programming\_language)}}
 är en integrerad utvecklingsmiljö för Scala som är speciellt anpassad för programmeringsundervisning i grundskolan. Kojo används i LTH:s Science Center Vattenhallen för utbildning av grundskolelärare i programmering och vid skolbesök och annan besöksverksamhet, i vilken lärare och studenter vid LTH arbetar som handledare. 
 
 Kojo är öppen källkod och utvecklingsgemenskapen leds av Lalit Pant från Indien. I Kojo finns även lättillgängliga bibliotek som gör tröskeln lägre att programmera rörlig grafik och enkla spel.

Under kursens första laboration använder vi grafikbiblioteket i Kojo för att illustrera grundläggande begrepp, så som sekvens, alternativ, repetition och abstraktion.  


\begin{figure}[H]
\centering
\includegraphics[width=0.8\textwidth]{../img/kojo/kojo.png}
\caption{Den nybörjarvänliga utvecklingsmiljön Kojo för Scala på svenska.}
\label{fig:appendix:ide:kojo}
\end{figure}

\section{Använda grafikbiblioteket i Kojo}\label{appendix:ide:kojo:install}

Kojo bygger på den beprövade pedagogiska idén med sköldpaddsgrafik \Eng{turtle graphics}\footnote{\url{https://en.wikipedia.org/wiki/Turtle_graphics}}, där du skriver program som styr en sköldpadda med en penna under magen. När sköldpaddan rör sig bildas ett streck av valfri färg på skärmen. Beroende på hur du bestämmer att sköldpaddan ska röra sig och vilken färg du bestämmer att pennan ska ha, kan du skapa olika intressanta bilder och samtidigt lära dig om programmeringens grunder.

Under kursens första laboration ska du använda grafikbiblioteket i Kojo tillsammans med editorn VS \code{code} och \code{scala-cli} i terminalen (se appendix \ref{appendix:terminal} och \ref{appendix:compile}). Ladda ner filen \texttt{kojo.scala} från \url{https://cs.lth.se/pgk/kojolib} och spara i en ny katalog med hjälp av din webbläsare, eller via dessa kommandon:

\begin{REPLnonum}
> mkdir w01-kojo
> cd w01-kojo
> curl -o kojolib.scala -sL https://cs.lth.se/pgk/kojolib
\end{REPLnonum}

Nu kan du starta Scala REPL och rita med Kojo så här:

\begin{REPLnonum}
> scala-cli repl .
Welcome to Scala 3.1.2 (17.0.2, Java OpenJDK 64-Bit Server VM).
Type in expressions for evaluation. Or try :help.
                                                                                                                               
scala> fram; höger; fram; vänster

\end{REPLnonum}

Du kan starta VS \code{code} i aktuellt bibliotek så här:
\begin{REPLnonum}
> code .
\end{REPLnonum}

Skriv nedan progam i VS \code{code} och spara det i samma katalog som den tidigare nedladdade filen, under ett nytt valfritt filnamn, t.ex. \code{rita.scala}:

\begin{Code}
@main def rita = fram; höger; fram; vänster
\end{Code}

Kör ditt fristående program med:
\begin{REPLnonum}
> scala-cli run .
\end{REPLnonum}

Du ska nu få upp ett fönster som heter Kojo Canvas med en sköldpadda som ritat två streck. När du stänger fönstret så avslutas programmet. Prova fler sköldpaddsfunktioner enligt tabell \ref{table:kojo:functions}.

I stället för att ladda ned filen \code{kojolib.scala} så kan du placera dess innehåll på lämpligt ställe i ditt program enligt nedan. Observera att raden som börjar med \code{//> using lib} ska vara en enda lång rad utan radbrytningar.%\code{export} gör Kojos kommandon tillgängliga utan prefix:
\lstinputlisting[breaklines=true,basicstyle=\ttfamily\fontsize{9}{11}\selectfont]{../workspace/w01_kojo/kojo.scala}

\noindent Scala-koden för den svenska paddans api finns här: \\
%\href{https://github.com/litan/kojo/blob/master/src/main/scala/net/kogics/kojo/lite/i18n/svInit.scala}{github.com/litan/kojo/blob/master/src/main/scala/net/kogics/kojo/lite/i18n/svInit.scala} \\
\href{https://github.com/litan/kojo-lib/blob/main/src/main/scala/net/kogics/kojo/i18n/Swedish.scala}{github.com/litan/kojo-lib/blob/main/src/main/scala/net/kogics/kojo/i18n/Swedish.scala}


%Kojo kräver (numera) \emph{inte} att \texttt{java} finns på din dator utan kommer med en egen JVM. 
%Eftersom du behöver tillgång till JDK i kursen, är det lika bra att installera hela JDK direkt (och inte bara JRE, så som beskrivs å länken ovan); se vidare hur du gör detta i avsnitt \ref{appendix:compile:install-jdk}.
%\href{http://www.kogics.net/kojo-download}{www.kogics.net/kojo-download}



\section{Kojo Desktop}

Kojo finns som fristående skrivbordsapplikation, kallad Kojo Desktop. Kojo Desktop innehåller en egen editor med syntaxfärgning för Scala, men fungerar ännu så länge bara för Scala 2. En av de synligaste skillnaderna mellan Scala 2 och Scala 3 är att klammerparenteser vid flerradiga funktioner är nödvändiga i Scala 2, medan Scala 3 har valfria klammerparenteser. Så om du använder Kojo Desktop behöver du komma ihåg att omgärda sekvenser av rader som hör ihop med \code|{| och \code|}|. 

Kojo Desktop är förinstallerad på LTH:s datorer och körs igång med terminalkommandot \texttt{kojo} eller via applikationsmenyn.  För instruktioner om hur du installerar Kojo Desktop på din egen dator se här: \href{http://www.lth.se/programmera/installera/}{lth.se/programmera/installera}

När du startar Kojo första gången, välj ''Svenska'' i språkmenyn och starta om Kojo. Därefter fungerar grafikfunktionerna på svenska enligt tabell \ref{table:kojo:functions} på sidan \pageref{table:kojo:functions}. När du startat om Kojo inställt på svenska ser programmet ut ungefär som i figur \ref{fig:appendix:ide:kojo} på sidan \pageref{fig:appendix:ide:kojo}.

Det finns ett antal användbara kortkommando som du hittar i menyerna i Kojo Desktop. Undersök speciellt Ctrl+Alt+Mellanslag som ger autokomplettering baserat på det du börjat skriva.

\section{Kojo i Webbläsaren}

En begränsad variant av Kojo finns tillgänglig för programmering direkt i din webbläsare här: \url{http://kojo.lu.se/}

När du trycker på play-knappen så kompileras din kod på en server till Javascript via ScalaJS och därefter körs Javascript-koden i din webbläsare. 
Kojo på webben är också ännu så länge begränsad till Scala 2 och kräver att du omgärdar sekvenser av rader som hör ihop med \code|{| och \code|}|.


\section{Mer om Kojo}

I detta dokument finns en enkel introduktion till Kojo: \\ ''Introduction to Kojo'' \url{http://www.kogics.net/kojo-ebooks#intro}

\noindent I tabell \ref{table:kojo:functions}, som fortsätter på efterföljande sidor, finns ett urval av kommando i Kojo på svenska och engelska.

{\small\renewcommand{\arraystretch}{1.4}
\begin{longtable}{@{}p{0.42\textwidth} p{0.55\textwidth}}

\caption{Ett urval av funktioner i Kojo. Se även \href{http://lth.se/programmera}{lth.se/programmera}}\label{table:kojo:functions}\\

\emph{Svenska/Engelska} & \emph{Vad händer?}  \\ \hline
\code|sudda| \newline \code|clear| & Ritfönstret suddas \\
\code|fram| \newline \code|forward()| & Paddan går framåt 25 steg. \\
\code|fram(100)| \newline \code|forward(100)| & Paddan går framåt 100 steg. \\
\code|höger| \newline \code|right(90)| & Paddan vrider sig 90 grader åt höger. \\
\code|höger(45)| \newline \code|right(45)| & Paddan vrider sig 45 grader åt höger. \\
\code|vänster| \newline \code|left(90)| & Paddan vrider sig 90 grader åt vänster. \\
\code|vänster(45)| \newline \code|left(45)| & Paddan vrider sig 45 grader åt vänster. \\
\code|hoppa| \newline \code|hop| & Paddan hoppar 25 steg utan att rita. \\
\code|hoppa(100)| \newline \code|hop(100)| & Paddan hoppar 100 steg utan att rita. \\
\code|hoppaTill(100, 200)| \newline \code|jumpTo(100, 200)| & Paddan hoppar till läget (100, 200) utan att rita. \\
\code|gåTill(100, 200)| \newline \code|moveTo(100, 200)| & Paddan vrider sig och går till läget (100, 200). \\
\code|hem| \newline \code|home| & Paddan går tillbaka till utgångsläget (0, 0). \\
\code|öster| \newline \code|setHeading(0)| & Paddan vrider sig så att nosen pekar åt höger. \\
\code|väster| \newline \code|setHeading(180)| & Paddan vrider sig så att nosen pekar åt vänster. \\
\code|norr| \newline \code|setHeading(90)| & Paddan vrider sig så att nosen pekar uppåt. \\
\code|söder| \newline \code|setHeading(-90)  | & Paddan vrider sig så att nosen pekar neråt. \\
\code|mot(100,200)| \newline \code|towards(100, 200)| & Paddan vrider sig så att nosen pekar mot läget (100, 200) \\
\code|sättVinkel(90)| \newline \code|setHeading(90)| & Paddan vrider nosen till vinkeln 90 grader. \\
\code|vinkel| \newline \code|heading| & Ger vinkelvärdet dit paddans nos pekar. \\
\code|sakta(5000)| \newline \code|setAnimationDelay(5000) | & Gör så att paddan ritar jättesakta. \\
\code|suddaUtdata| \newline \code|clearOutput| & Utdatafönstret suddas. \\
\code|utdata("hej")| \newline \code|println("hej")| & Skriver texten \texttt{hej} i utdatafönstret. \\
\code|val t = indata("Skriv")| \newline \code|val t = readln("Skriv:")| & Väntar på inmatning efter ledtexten \texttt{Skriv} och sparar den inmatade texten i t.  \\
\code|textstorlek(100)| \newline \code|setPenFontSize(100)| & Paddan skriver med jättestor text nästa gång du gör skriv. \\
\code|båge(100, 90)| \newline \code|arc(100, 90)| & Paddan ritar en båge med radie 100 och vinkel 90. \\
\code|cirkel(100)| \newline \code|circle(radie)| & Paddan ritar en cirkel med radie 100. \\
\code|synlig| \newline \code|visible| & Paddan blir synlig. \\
\code|osynlig| \newline \code|invisible| & Paddan blir osynlig. \\
\code|läge.x| \newline \code|position.x| & Ger paddans x-läge \\
\code|läge.y| \newline \code|position.y| & Ger paddans y-läge \\
\code|pennaNer| \newline \code|penDown| & Sätter ner paddans penna så att den ritar när den går. \\
\code|pennaUpp| \newline \code|penUp| & Lyfter upp paddans penna så att den INTE ritar när den går. \\
\code|pennanÄrNere| \newline \code|penIsDown| & Kollar om pennan är nere eller inte. \\
\code|färg(rosa)| \newline \code|setPenColor(pink)| & Sätter pennans färg till rosa. \\
\code|fyll(lila)| \newline \code|setFillColor(purple)| & Sätter ifyllnadsfärgen till lila. \\
\code|fyll(genomskinlig)| \newline \code|setFillColor(noColor)| & Gör så att paddan inte fyller i något när den ritar. \\
\code|bredd(20)| \newline \code|setPenThickness(20)| & Gör så att pennan får bredden 20. \\
\code|sparaStil| \newline \code|saveStyle| & Sparar pennans färg, bredd och fyllfärg. \\
\code|laddaStil| \newline \code|restoreStyle| & Laddar tidigare sparad färg, bredd och fyllfärg. \\
\code|sparaLägeRiktning| \newline \code|savePosHe| & Sparar pennans läge och riktning \\
\code|laddaLägeRiktning| \newline \code|restorePosHe| & Laddar tidigare sparad riktning och läge \\
\code|siktePå| \newline \code|beamsOn| & Sätter på siktet. \\
\code|sikteAv| \newline \code|beamsOff| & Stänger av siktet. \\
\code|bakgrund(svart)| \newline \code|setBackground(black)| & Bakgrundsfärgen blir svart. \\
\code|bakgrund2(grön,gul)| \newline \code|setBackgroundV(green, yellow)| & Bakgrund med övergång från grönt till gult. \\
\code|upprepa(4){fram; höger}| \newline \code|repeat(4){forward; right}| & Paddan går fram och svänger höger 4 gånger. \\
\code|avrunda(3.99)| & Avrundar 3.99 till 4.0 \\
\code|slumptal(100)| & Ger ett slumptal mellan 0 och 99. \\
\code|slumptalMedDecimaler(100)| & Ger ett slumptal mellan 0 och 99.99999999 \\
\code|systemtid| & Ger nuvarande systemklocka i sekunder. \\
\code|räknaTill(5000)| & Kollar hur lång tid det tar för din dator att räkna till 5000. \\


\end{longtable}
}%end small


\hline
\end{longtable}
}%end small

\noindent Scala-koden för den svenska paddans api finns här: \\
\href{https://bitbucket.org/lalit_pant/kojo/src/tip/src/main/scala/net/kogics/kojo/lite/i18n/svInit.scala}{bitbucket.org/lalit\_pant/kojo/src/tip/src/main/scala/net/kogics/\\kojo/lite/i18n/svInit.scala}




\newpage

\section{Eclipse och ScalaIDE}\label{appendix:ide:eclipse}

Eclipse%
\footnote{\href{https://en.wikipedia.org/wiki/Eclipse_(software)}{en.wikipedia.org/wiki/Eclipse\_(software)}}
är en professionell IDE som stödjer många olika programmeringsspråk. Eclipse är skriven i Java och bygger vidare på ett utvecklingsprojekt som initierades av IBM. Eclipse är ett fritt och öppet projekt som numera kontrolleras av en oberoende stiftelse.

Till Eclipse finns en insticksmodul \Eng{plug-in} som kallas ScalaIDE och erbjuder stöd för Scala med tillhörande standardbibliotek.

Eclipse är en omfattande och avancerad programmeringsmiljö med många funktioner och inställningar. Det finns även en omfattande uppsättning insticksmoduler och tilläggsprogram som underlättar utveckling av t.ex. webbprogram, databaser och mycket annat. 

I detta avsnitt ges länkar till installation samt tips om hur du kommer igång med att använda Eclipse och ScalaIDE. Det går ganska snabbt att lära sig grunderna, men det kräven en viss ansträngning att lära sig de mer avancerade funktionerna. Det finns omfattande resurser på nätet som hjälper dig vidare. 


\subsection{Installera Eclipse Mars och ScalaIDE}\label{appendix:ide:eclipse:install}

Eclipse med ScalaIDE är förinstallerat på LTH:s datorer och startas med kommandot \texttt{scalaide} i ett terminalfönster.

ScalaIDE fungerar med Eclipse-versionerna \textit{Luna} och \textit{Mars} (men i skrivande stund fungerar ScalaIDE ännu \textit{inte} med den allra senaste versionen kallad \textit{Neon}). 

För att installera ScalaIDE på din egen dator, följ nedan instruktioner: 

\begin{enumerate}
\item Kontrollera enligt avsnitt \ref{appendix:compile:check-jdk} att du har \texttt{java} installerat och installera vid behov JDK enligt avsnitt \ref{appendix:compile:install-jdk}.

\item Installera Eclipse version \textbf{Mars}, varianten för \textbf{Java Developers} som återfinns på denna sida: \\ \url{https://www.eclipse.org/downloads/packages/release/Mars/2} \\ som är den \textit{andra} varianten i listan (alltså inte Java EE). Följ dessa steg:
\begin{enumerate}
\item Klicka på den \textbf{64-bit}-variant som passar ditt operativsystem.
\item Filen som laddas ner heter något som liknar (beroende på OS): \\ \texttt{eclipse-java-mars-2-win32-x86\_64.zip} 
\\ Det kan ta lång tid att ladda ner filen som är på ca 170MB. Om du klickar på \textit{''select a mirror''} kan du välja en svensk sajt för att ladda ner snabbare. 

\item Dubbelklicka på filen för att packa upp den, vilket kan ta många minuter. Du får, när up	packningen är klar, ett bibliotek med en fungerande Eclipse-installation som du kan placera var du vill. Kör du Windows, lägg den förslagsvis här:\\ 
\code|C:\eclipse\eclipse-java-mars-2-win32-x86_64|

\item för Ubuntu Linux finns kompletterande installationsanvisningar här, som ger dig en ikon i app-menyn m.m.: 
\\ \url{http://askubuntu.com/questions/26632/how-to-install-eclipse}
\end{enumerate}

\item Installera Scala IDE inifrån%
\footnote{Det finns på ScalaIDE-hemsidan möjlighet att ladda ner en Eclipse-variant med färdiginstallerad ScalaIDE-plugin, men då får du i skrivande stund den gamla versionen Eclipse \textit{Luna}, varför du rekommenderas att, enligt instruktionerna här, själv installera ScalaIDE inifrån Eclipse \textit{Mars}, som är den senaste Eclipse-versionen för vilken ScalaIDE fungerar.}
 Eclipse enligt nedan steg:
\begin{enumerate}
\item Starta Eclipse, t.ex. genom att köra igång den exekverbara filen som ligger i underbiblioteket \texttt{eclipse}, i Windows heter den \texttt{eclipse.exe} medan den exekverbara filen i Linux heter \texttt{eclipse} utan filändelse.

\item Välj i frågerutan som dyker upp, någon plats för \textit{workspace} (kvittar vilken just nu, kan ändras senare).

\item Klicka på menyn \textit{Help} $\rightarrow$ \textit{Install new software}.

\item Klicka på \textit{Add}-knappen till höger och skriv: \\ \textit{''ScalaIDE for Scala 2.11''} i \textit{Name}-fältet och ange denna adress i \textit{Location}-fältet: \\
  {\small\mbox{\url{http://download.scala-ide.org/sdk/lithium/e44/scala211/stable/site}}} \\
  och klicka \textit{OK}.
  
\item Du får nu upp en lista med alternativ. Kryssa för alternativet
\\ {\frame{\checkmark}}~~\textit{Scala IDE for Eclipse} \\ och klicka \textit{Next} och sedan \textit{Next} igen och acceptera licensvillkoren och klicka \textit{Finish}.

\item Låt installationen ta sin tid och starta sedan om Eclipse när installationen är färdig. 

\item När Eclipse är igång igen visas en dialog som föreslår att du ska köra \textit{Setup Diagnostics}. Gör detta och välj \textit{Use recommended default settings}. Ändra även i filen \textbf{eclipse.ini} för höja den övre minnesgränsen. Det gör du genom att ändra på den rad i filen som börjar med \texttt{-Xmx}. Hur mycket du ska tillåta som max beror på hur mycket minne du har, men ge minst 1 gigabyte för smidig körning, genom att skriva så här på relevant rad i filen \textbf{eclipse.ini}: \\
\texttt{-Xmx1G } \\


\item Kompletterande information finns här, inklusive en video som visar installationsproceduren och hur man kommer igång med ett ''hello world''-program: \\ \url{http://scala-ide.org/download/current.html}


\end{enumerate}


\end{enumerate}

\noindent I nästa avsnitt beskrivs några rekommenderade anpassningar som du kan göra bland de omfattande inställningsmöjligheterna för Eclipse.

\newpage

\subsection{Anpassa Eclipse och ScalaIDE}\label{subsection:appendix:ide:eclipse:tweaks}

\newcommand\EclipsePrefs{\MenuArrow{Window}\MenuArrow{Preferences}}
\newcommand\EclipsePrefsGeneral{\EclipsePrefs\MenuArrow{General}}


Förutom maxminneshöjningen i filen \texttt{eclipse.ini}, som finns i installationskatalogen för Eclipse, till minst \texttt{-Xmx1G} (se föregående avsnitt), är det bra att göra några ytterligare anpassningar av Eclipse och ScalaIDE för att få en snabbare och smidigare utvecklingsmiljö. Du hittar inställningarna i menyn \EclipsePrefs ... uppe till höger i Eclipse-fönstret.



\begin{enumerate}
\item \EclipsePrefsGeneral 
\\ Markera \FramedCheckmark{Show Heap Status} så får du se minnesanvändningen i en liten ruta i nederdelen av fönstret, vilket hjälper dig att upptäcka om minnesbegränsningen i filen \texttt{eclipse.ini} är en flaskhals vid stora projekt och många öppna fönster. Klicka sedan \Button{Apply} längst ner.

\item \label{item:scala-perspective} \EclipsePrefsGeneral\MenuArrow{Editors}\MenuArrow{Perspectives}  
\\ Markera \textit{Scala} i listan med perspektiv och klicka på knappen 
 \\ \Button{Make default} till höger och sedan på knappen \Button{Apply} längst ner.

\item \EclipsePrefsGeneral\MenuArrow{Editors}\MenuArrow{TextEditors}
\\ Markera \FramedCheckmark{Insert spaces for tabs} så att du slipper specialtecken som kan tolkas olika av olika editorer. Klicka sedan \Button{Apply} längst ner.

\item \EclipsePrefsGeneral\MenuArrow{Editors}\MenuArrow{TextEditors}
\\ \MenuArrow{Spelling} Avmarkera \FramedUnchecked{Enable spell checking} för att slippa att svenska namn och svenska kommentarer markeras som felstavade. Om du senare jobbar med ett projekt helt på engelska, kan du med fördel markera denna kryssruta igen. Klicka sedan \Button{Apply} längst ner.

\item \EclipsePrefsGeneral\MenuArrow{Editors}\MenuArrow{Webbrowser}
\\ Markera \FramedCheckmark{Use external web browser} för att köra din vanliga webbläsare när du klickar på länkar. Klicka sedan \Button{Apply} längst ner.
  
\item  \EclipsePrefs\MenuArrow{Scala}\MenuArrow{Compiler}
\\ I fliken \textbf{Standard} markera dessa kryssrutor för att få extra varningar: \\
\begin{tabular}{l @{}l @{}l}
\textit{deprecation} & \FramedCheckmark{} & varnar vid användning av föråldrad kod som snart utgår \\
\textit{feature}     & \FramedCheckmark{} & påminner om import vid användning av avancerad kod  \\
\textit{unchecked}   & \FramedCheckmark{} & ger tips vid speciella problem med generiska typer \\
\end{tabular}\\
och klicka sedan på knappen \Button{Apply} längst ner.

\item \EclipsePrefs\MenuArrow{Java}\MenuArrow{Compiler}\MenuArrow{Errors/Warnings}
\\ Veckla ut listan \textbf{Potential programming problems} och sätt \textbf{Resource leak} till alternativet \textbf{Ignore}, så slipper du varningar vid användning \jcode{Scanner} i Java. Klicka sedan \Button{Apply} längst ner.

\end{enumerate}

\noindent Ovan anpassningar är rekommenderade men inte nödvändiga och du kan gärna välja att göra andra anpassningar som passar just dig. Skriv då gärna ner vilken inställning du ändrat, så att du hittar tillbaka om du ångrar dig. 

Du hittar tips om fler inställningar för att anpassa ScalaIDE här: \\
\url{http://scala-ide.org/docs/current-user-doc/advancedsetup}



\subsection{Använda Eclipse och ScalaIDE}\label{appendix:ide:eclipse:use}

Ett grundläggande koncept i Eclipse är \textbf{workspace}. Ett workspace utgör ett arbetsområde kopplat till en katalog i ditt filsystem där du kan arbeta med ett eller flera \textbf{projekt}. Ett projekt innehåller i sin tur dina källkodsfiler och klassfiler etc. i en specifik katalogstruktur som Eclipse skapar när du editerar, kompilerar och kör dina projekt. 

\subsubsection{Starta och välja workspace}\label{subsubsection:start:eclipse}

När du startar Eclipse måste du välja vilket workspace du vill använda innan du kommer vidare. När du kör igång Eclipse första gången, klicka OK enligt det förslag som ges. Du kan senare växla workspace genom menyn \MenuArrow{File}\Menu{Switch Workspace}. Om katalogen du anger inte redan finns, kommer den att skapas och initieras med de filer Eclipse behöver.

I figur \ref{fig:appendix:eclipse:welcome} visas välkomstfliken i Eclipse med sina länkar till funktionsöversikt och olika handledningar. Stäng välkomstfliken genom att klicka på flikens kryss eller på ikonen \textit{Workbench}. Då kommer du vidare till den normala arbetsytan i Eclipse. Du kan få tillbaka välkomstfliken igen via menyn \MenuArrow{Help}\Menu{Welcome}. 

\begin{figure}[H]
\centering
\includegraphics[width=1.0\textwidth]{../img/eclipse/eclipse-welcome.png}
\caption{Välkomstfliken för Eclipse, som nås via menyn \MenuArrow{Help}\Menu{Welcome}. Gå vidare genom att klicka på \textit{Workbench}.}
\label{fig:appendix:eclipse:welcome}
\end{figure}

\subsubsection{Välja perspektiv och visa olika vyer}

Eclipse-fönstret kan innehålla många underfönster i olika flikar, så kallade \textbf{views} eller vyer, som kan arrangeras på olika vis efter hur du vill ha dem. Vilka vyer som syns och hur de placeras beror på vilket s.k. \textbf{perspective} som är aktivt.  Figur \ref{fig:appendix:eclipse:open-perspective} visar arbetsytan med olika vyer i Java-perspektivet. 

Du kan byta till Scala-perspektivet genom att trycka på \includegraphics[scale=0.75]{../img/eclipse/eclipse-perspective-button.png} eller genom menyn \MenuArrow{Window}\MenuArrow{Perspective}\MenuArrow{Open Perspective}\MenuArrow{Other...}\Menu{Scala}.
Du kan anpassa inställningarna så att Scala blir \textit{default perspective}, se steg \ref{item:scala-perspective} i avsnitt \ref{subsection:appendix:ide:eclipse:tweaks} på sidan \pageref{subsection:appendix:ide:eclipse:tweaks}.

Stäng vyerna \textit{Task List} och \textit{Outline} om du vill ha mer plats till de övriga vyerna för paketnavigering, editering och utdata. Du kan öppna stängda vyer igen genom menyn \MenuArrow{Window}\Menu{Show View}. 

\begin{figure}
\centering
\includegraphics[width=1.0\textwidth]{../img/eclipse/eclipse-open-perspective.png}
\caption{Arbetsytan i Eclipse. Du kan växla mellan Scala- och Java-perspektivet genom att klicka på perspektivvalsknappen.}
\label{fig:appendix:eclipse:open-perspective}
\end{figure}

\subsubsection{Hello World}\label{subsubsection:eclipse:hello-world}

Efter att du öppnat Eclipse med ScalaIDE i ett tomt workspace och valt Scala-perspektivet enligt föregående avsnitt, kan du skapa ditt första projekt med ett \textit{''Hello World''}-program enligt stegen nedan.

\begin{enumerate}
\item Högerklicka i \Menu{Package Explorer} och välj \MenuArrow{New}\Menu{Scala Project}, varefter en dialogruta visas. 

\item Fyll i namnet \texttt{hello} i fältet \Menu{Project Name} och klicka \Button{Finish}.

\item Högerklicka igen i \Menu{Package Explorer} och välj \MenuArrow{New}\Menu{Scala Object}, varefter en ny dialogruta visas. 

\item Fyll i namnet \texttt{hi} i fältet \Menu{Name} och klicka \Button{Finish}.

\item Du får nu i editorvyn ett kodskelett med \code{object hi}.

\item Börja skriv \code{main} som visas i figur \ref{fig:appendix:eclipse:complete-main} och tryck Ctrl+Mellanslag för att aktivera kodkomplettering \Eng{code completion}. Då får du upp en lista med alternativ. Välj det översta alternativet \texttt{main} varefter ett kodskelett med en main-metod klistras in automatiskt i din kod.

\begin{figure}
\centering
\includegraphics[width=1.0\textwidth]{../img/eclipse/eclipse-complete-main.png}
\caption{Aktivera kodkomplettering med Ctrl+Mellanslag efter ordet \code{main}.}
\label{fig:appendix:eclipse:complete-main}
\end{figure}

\item Fyll i lämplig utskriftstext i ett \code{println}-anrop så att din \code{main}-metod blir så som visas i editorfliken i figur \ref{fig:appendix:eclipse:hello-world}.

\begin{figure}[H]
\centering
\includegraphics[width=1.0\textwidth]{../img/eclipse/eclipse-hello-world.png}
\caption{Skriv klart \code{main}-metoden och kör ditt program med play-knappen.}
\label{fig:appendix:eclipse:hello-world}
\end{figure}

\item Kör ditt program genom att trycka på den gröna play-knappen, som muspekaren i figur \ref{fig:appendix:eclipse:hello-world} pekar på. Du kan också trycka F11 för att köra igång din app, efter att du vid första körningen i dialogen \textit{Select Preferred Launcher} markerat  \FramedCheckmark{Use configuration specific settings} och valt alternativet \textit{Scala Application (new debugger) Launcher}. 

\end{enumerate}




\subsubsection{Ladda ner kursens workspace och importera i Eclipse}\label{subsubsection:download--import-workspace}

Det finns en zip-fil med ett workspace med projekt för flera av kursens laborationer som du kan ladda ner och importera i Eclipse. Följ stegen nedan.

\begin{enumerate}
\item Ladda ner kursens workspace här: \url{http://cs.lth.se/pgk/ws}

\item Packa upp filen på lämpligt ställe.

\item Starta Eclipse med ScalaIDE-plugin (se startinstruktioner på sidan \pageref{subsubsection:start:eclipse}). 

\item Växla workspace till biblioteket du nyss packade upp, ungefär som i figur \ref{fig:eclipse:ide:open} och klicka \Button{OK}.

\begin{figure}[H]
\centering
\includegraphics[width=1.0\textwidth]{../img/eclipse/eclipse-select-workspace.png}
\caption {Öppna kursens workspace genom att bläddra till biblioteket där du packade upp filen som du laddat ned från: \url{http://cs.lth.se/pgk/ws} }
\label{fig:eclipse:ide:open}
\end{figure}

\item Stäng välkomstfliken för att komma vidare till workbench (se figur \ref{fig:appendix:eclipse:welcome} på sidan \pageref{fig:appendix:eclipse:welcome}). Det ser då ut ungefär som i figur~\ref{fig:appendix:eclipse:open-perspective} på sidan \pageref{fig:appendix:eclipse:open-perspective}. Det syns ännu inget i \textit{Package Explorer} då vi ännu inte importerat något projekt. 

\item Innan du går vidare, säkerställ att du har Scala-perspektivet aktiverast. Du kan växla till Scala-perspektivet genom att trycka på \includegraphics[scale=0.75]{../img/eclipse/eclipse-perspective-button.png} eller genom menyn \MenuArrow{Window}\MenuArrow{Perspective}\MenuArrow{Open Perspective}\MenuArrow{Other...}\Menu{Scala}.
Du kan anpassa inställningarna så att Scala blir \textit{default perspective}, se steg \ref{item:scala-perspective} i avsnitt \ref{subsection:appendix:ide:eclipse:tweaks} på sidan \pageref{subsection:appendix:ide:eclipse:tweaks}.


\item Högerklicka i \textit{Package Explorer} och välj \Menu{Import...}, se Fig.~\ref{fig:eclipse:import}, eller välj menyn \MenuArrow{File}\Menu{Import...}. 

\begin{figure}[H]
\centering
\includegraphics[width=1.0\textwidth]{../img/eclipse/eclipse-import.png} 
\caption {Välj \Menu{Import}-menyn för att importera existerande projekt.}
\label{fig:eclipse:import}
\end{figure}

\item Nu öppnas \Menu{Import}-dialogen som visas i figur \ref{fig:eclipse:import-existing}. Öppna mappen \Menu{General}, markera \textbf{Existing Projects into Workspace} och klicka \Button{Next}.



\begin{figure}[H]
\centering
\includegraphics[width=0.75\textwidth]{../img/eclipse/eclipse-import-existing.png} 
\caption {Välj att importera existerande projekt under \Menu{General}.}
\label{fig:eclipse:import-existing}
\end{figure}


\begin{figure}[H]
\centering
\includegraphics[width=1.0\textwidth]{../img/eclipse/eclipse-import-projects.png} 
\caption {Välj \FramedCheckmark{Select Root Directory} och klicka \Button{Browse}.}
\label{fig:eclipse:import-projects}
\end{figure}


\item Nu kommer ytterligare ett dialogfönster som visas i figure \ref{fig:eclipse:import-projects}. Med \FramedCheckmark{Select Root Directory} markerad kan du klicka \Button{Browse} för att ange workspace-mappen i ännu en dialog där du bara ska trycka \Button{Ok} utan att välja underbibliotek till workspace. När det är klart ska det se ut som i figur \ref{fig:eclipse:import-projects} där alla Eclipse-projekt \FramedCheckmark{cslib}, \FramedCheckmark{w04\_pirates}, etc. är markerade. Klicka sedan \Button{Finish}.

\item Följ ''Hello World''-instruktionerna på sidan \pageref{subsubsection:eclipse:hello-world} och skapa programmet som visas i figure \ref{fig:eclipse:pirates-hi}, genom att veckla ut projektet \textbf{w04\_pirates}, markera och högerklicka på paketet \textbf{priates}, och välja \MenuArrow{New}\Menu{Scala Object}.

\item Om du får problem, fråga någon som känner till Eclipse om hjälp. Det finns även mycket hjälp på nätet, se till exempel: \\ \href{http://stackoverflow.com/questions/8522149/eclipse-not-recognizing-scala-code}{stackoverflow.com/questions/8522149/eclipse-not-recognizing-scala-code}

\begin{figure}[H]
\centering
\includegraphics[width=1.0\textwidth]{../img/eclipse/eclipse-pirates-hello.png} 
\caption {Skapa ett \MenuArrow{New}\Menu{Scala Object} med kod enligt bilden.}
\label{fig:eclipse:pirates-hi}
\end{figure}


\end{enumerate}



\newpage

\section{IntelliJ IDEA}\label{appendix:ide:intellij}

IntelliJ IDEA%
\footnote{\href{https://en.wikipedia.org/wiki/IntelliJ_IDEA}{en.wikipedia.org/wiki/IntelliJ\_IDEA}}
 är en professionell IDE som stödjer många olika programmeringsspråk. IntelliJ är skriven i Java och utvecklas av det tjeckiska företaget JetBrains. 

IntelliJ IDEA finns i två varianter: en gratis gemenskapsvariant med öppenkällkodslicens \Eng{Community edition}, samt en betalvariant med sluten källkod och support-tjänster.


Till IntelliJ IDEA finns en insticksmodul \Eng{plug-in} som erbjuder stöd för Scala med tillhörande standardbibliotek.

IntelliJ IDEA är en omfattande och avancerad programmeringsmiljö med många funktioner och inställningar. Det finns även en omfattande uppsättning insticksmoduler och tilläggsprogram som underlättar utveckling av t.ex. mobilappar, webbprogram, databaser och mycket annat. 

I detta avsnitt ges länkar till installation samt tips om hur du kommer igång med att använda IntelliJ IDEA med Scala. Det går ganska snabbt att lära sig grunderna, men det kräven en viss ansträngning att lära sig de mer avancerade funktionerna. Det finns omfattande resurser på nätet som hjälper dig vidare. 

Google tillkännagav 2013 att företaget övergår från Eclipse till IntelliJ som den officiellt understödda utvecklingsmiljön för Android och 2014 lanserades utvecklingsmiljön Android Studio%
\footnote {\href{https://en.wikipedia.org/wiki/Android_Studio}{en.wikipedia.org/wiki/Android\_Studio}}
 som bygger vidare på IntelliJ. 

\subsection{Installera IntelliJ med Scala-plugin}\label{appendix:ide:intellij:install}

IntelliJ med Scala-plug-in är förinstallerat på LTH:s datorer och startas med kommandot \texttt{idea} i ett terminalfönster.

\begin{itemize}
\item För Ubuntu Linux finns ett färdigt paket som du kan installera med dessa kommandon i terminalen: 
\begin{REPLnonum}
sudo add-apt-repository ppa:mmk2410/intellij-idea-community
sudo apt-get update
\end{REPLnonum}
Mer information om denna ppa finns här:\\ \url{https://launchpad.net/~mmk2410/+archive/ubuntu/intellij-idea-community}\item För Windows och Mac: ladda ner och kör installationsfil för ditt operativsystem för den öppna varianten kallad \textbf{Community} här: \\
\url{https://www.jetbrains.com/idea/download/} \\
Följ instruktionerna som ges av installationsprogrammet.
\end{itemize}

\subsection{Anpassa IntelliJ}\label{appendix:ide:intellij:tweak}
Första gången du kör igång IntelliJ får du ett antal frågor om vilka anpassningar du vill göra. Följ instruktionerna steg för steg enligt nedan.
\begin{enumerate}
\item \textbf{UI Theme}. Denna dialog gäller utseende på gränssnittet. Det tema som kallas \textit{Dracula} är en populär variant med nedtonade färger anpassade för att vara skonsamma mot ögonen. Klicka \Button{Next} när du valt tema.

\item \textbf{Default plugins}. Denna dialog gäller inställningar av befintliga insticksmoduler. Dessa inställningar fungerar bra som de är. Klicka \Button{Next}.

\item \textbf{Featured plugins}. I rutan för \textbf{Plugin for Scala language support} Klicka \Button{Install} och låt installationen av Scala fullbordas. 

\item Klicka därefter \Button{Start using IntelliJ IDEA}.

\item I välkomstfönstrets nedre hörn, välj \MenuArrow{Configure}\Menu{Settings} och överväg om du vill göra följande lämpliga men ej nödvändiga inställningar. 
\begin{enumerate}
\item I fliken \MenuArrow{Editor}\Menu{General} markera \FramedCheckmark{Change font size (Zoom) with Ctrl+Mouse Wheel} för att lätt kunna ändra textstorlek i editorn. Klicka \Button{Apply} nere till höger.

\item I fliken \MenuArrow{Editor}\Menu{Inspections} och välj \Menu{Spelling} i högra listan. Avmarkera \FramedUnchecked{Typo} för att undvika att svenska ord blir markerade som felstavade. Klicka \Button{Apply} nere till höger.

\item I fliken \MenuArrow{Editor}\Menu{File and Code Templates} och under fliken \Menu{Files} i högra listan: för varje Scala-filtyp (Scala Class, Scala Trait, Scala Object, ...) ta bort de initiala raderna i mallen som börjar med \code{#} för att slippa onödiga kommentarer i koden när du skapar nya filer. Klicka \Button{Apply} nere till höger.
 
\end{enumerate} 
Du kan också göra ovan och liknande anpassningar senare genom menyn \MenuArrow{File}\Menu{Settings...}
\end{enumerate} 

\subsection{Använda IntelliJ}\label{appendix:ide:intellij:use}

\subsubsection{Skapa ett nytt projekt}

När du startar IntelliJ IDEA utan förvalt projekt visas välkomstskärmen i figur \ref{fig:idea:welcome}. Klicka på \Menu{Create New Project}, varefter dialogen i figur \ref{fig:idea:new-project} visas. Följ stegen enligt nedan.

\begin{figure}[H]
\centering
\includegraphics[width=0.8\textwidth]{../img/intellij/idea-welcome.png} 
\caption{Välkomstfönstret för IntelliJ IDEA.}
\label{fig:idea:welcome}
\end{figure}

\begin{enumerate}
\item I dialogen \textbf{New Project} ska du ge projektet ett namn och välja körmiljö för ditt projekt. Ge projektet namnet \texttt{hello} enligt figur \ref{fig:idea:new-project}. 

\item Välj sedan att skapa ett Scala-projekt genom att markera \textbf{Scala} enligt figur \ref{fig:idea:new-scala-project} och klicka \Button{Next}.

\begin{figure}[H]
\centering
\includegraphics[width=0.65\textwidth]{../img/intellij/idea-new-scala-project.png} 
\caption{Välj att skapa ett Scala-projekt.}
\label{fig:idea:new-scala-project}
\end{figure}

\item Välj sedan \textbf{Project SDK} genom att klicka på \Button{New...}, välja \textit{JDK} och sedan veckla ut fliken \textit{JVM} och  välja \code{Oracle_jdk8} och klicka \Button{OK} enligt  figur \ref{fig:idea:new-project}.

\item Välj sedan \textbf{Scala SDK} genom att klicka på \Button{Create...}, markera raden med \textit{System 2.11.8} och klicka \Button{OK} enligt  figur \ref{fig:idea:new-project}.

\item Avsluta med att klicka \Button{Finish} i dialogen \textbf{New Project}.

\begin{figure}[H]
\centering
\includegraphics[width=1.0\textwidth]{../img/intellij/idea-new-project.png} 

\begin{minipage}{0.27\textwidth}
\includegraphics[width=1.0\textwidth]{../img/intellij/idea-project-sdk-jvm.png} 
\end{minipage}
\begin{minipage}{0.35\textwidth}
\includegraphics[width=1.0\textwidth]{../img/intellij/idea-project-sdk-home.png} 
\end{minipage}
\begin{minipage}{0.35\textwidth}
\includegraphics[width=1.0\textwidth]{../img/intellij/idea-scala-sdk.png} 
\end{minipage}%

\caption{Namnge ditt projekt och ställ in körmiljön för JVM och Scala genom att klicka på \Button{New...} och \Button{Create...}}.
\label{fig:idea:new-project}
\end{figure}

\item Du får nu ett projektfönster som liknar det i figur \ref{fig:idea:project-hello} på sidan \pageref{fig:idea:project-hello}.


\begin{figure}
\centering
\includegraphics[width=1.0\textwidth]{../img/intellij/idea-new-scala-class.png} 

\begin{minipage}{0.35\textwidth}
\includegraphics[width=1.0\textwidth]{../img/intellij/idea-scala-object.png} 
\end{minipage}
\begin{minipage}{0.60\textwidth}
\includegraphics[width=1.0\textwidth]{../img/intellij/idea-complete-main.png} 
\end{minipage}%

\caption{Högerklicka på mappen \textbf{src} och välj \MenuArrow{New}\Menu{Scala Class} och skapa ett nytt Scala-objekt med main-metod. Aktivera kodkomplettering i editorn efter ordet main med TAB.}
\label{fig:idea:project-hello}
\end{figure}


\item Veckla ut ditt projekt och högerklicka på \texttt{src} och välj \MenuArrow{New}\Menu{Scala Class}.Välj sedan \textbf{Object} i dialogen \textbf{Create New Scala Class} och klicka \Button{OK}, enligt figur \ref{fig:idea:project-hello}.

\item Du får nu upp ett editorfönster med koden för objektet \code{hello}. Skriv ordet \code{main} inuti objektet och tryck TAB för att aktivera kodkomplettering. En mall för main-metoden klistras då in i objektet. 

\item Skriv kod så att det ser ut som i editorfönstret i figur \ref{fig:idea:hello-world} på sidan \pageref{fig:idea:hello-world}.

\item Kör igång ditt program genom att klicka på play-knappen eller genom att trycka Shift+F10. Om play-knappen är initialt är grå i stället för grön, välj menyn \MenuArrow{Run}\Menu{Run...}. 

\end{enumerate}


\noindent Mer information om hur du använder Scala-plugin för IntelliJ finns här:\\
\href{https://confluence.jetbrains.com/display/SCA/Scala+Plugin+for+IntelliJ+IDEA}{confluence.jetbrains.com/display/SCA/Scala+Plugin+for+IntelliJ+IDEA}

\begin{figure}
\centering
\includegraphics[width=1.0\textwidth]{../img/intellij/idea-hello.png} 
\caption{Kör ditt program med play-knappen eller \MenuArrow{Run}\Menu{Run...}.}
\label{fig:idea:hello-world}
\end{figure}


\subsubsection{Ladda ner kursens workspace och importera i IntelliJ IDEA}

Det finns en zip-fil med ett workspace med projekt för flera av kursens laborationer som du kan ladda ner och importera i Eclipse. Följ stegen nedan.

\begin{enumerate}
\item Ladda ner kursens workspace här: \url{http://cs.lth.se/pgk/ws}

\item Packa upp filen på lämpligt ställe.

\item Starta IntelliJ. Om du redan har ett projekt igång välj menyn \MenuArrow{File}\Menu{Close project} så kommer du tillbaka till välkomstfönstret. Välj \textbf{Import Project} så som visas i figure \ref{fig:idea:import1-project}.

\begin{figure}[H]
\centering
\includegraphics[width=1.0\textwidth]{../img/intellij/idea-import1-project.png} 
\caption{Välj \Menu{Import Project} efter att du stängt ev. öppna projekt.}
\label{fig:idea:import1-project}
\end{figure}

\item Bläddra dit du packat upp workspace och markera denna folder i likhet med figur \ref{fig:idea:import23-select} och klicka \Button{OK}. I den efterföljande dialogen välj \textbf{Eclipse} och klicka \Button{Next}.

\item Klicka \Button{Next} igen i dialogen som liknar figur \ref{fig:idea:import4-directory} på sidan \pageref{fig:idea:import4-directory}, där mappen du valt är förvald.

\item Klicka \Button{Next} igen enligt figur \ref{fig:idea:import5-select-projects} på sidan  \pageref{fig:idea:import5-select-projects}. Alla tillgängliga Eclipse-projekt ska vara markerade.

\item Klicka \Button{Finish} enligt figur \ref{fig:idea:import5-select-projects} med förifylld text oförändrad.

\item Bläddra fram filen \texttt{PirateSpeech.scala} och öppna den med ett dubbelklick. Klicka på länken \textbf{Setup Scala SDK} uppe till höger enligt figur \ref{fig:idea:import78-setup-scala-sdk} på sidan \pageref{fig:idea:import78-setup-scala-sdk}. I efterföljande dialog kontrollera att \texttt{scala-sdk-2.11.8} är förvalt och klicka \Button{OK}.

\item Lägg till testutskrift enligt rad 7 i figur \ref{fig:idea:import9-run} på sidan \pageref{fig:idea:import9-run}. Testkör genom att välja menyn \MenuArrow{Run}\Menu{Run..} eller tryck Alt+Shift+F10 och sedan välja \code{PirateSpeech}. Kontrollera att utskriften i utskriftsfönstret ser ut som förväntat.
\end{enumerate}

\noindent Om du får problem på vägen, be någon med erfarenhet av IntelliJ om hjälp.


{\vfill
\begin{figure}[H]
\centering
\includegraphics[width=1.0\textwidth]{../img/intellij/idea-import2-select.png} 

{\hfill\includegraphics[width=0.4\textwidth]{../img/intellij/idea-import3-eclipse.png}} 

\caption{Markera den upp-packade workspace-mappen från zip-filen som du laddat ner från: \url{http://cs.lth.se/pgk/ws} och välj \textbf{Eclipse}-import.}
\label{fig:idea:import23-select}
\end{figure}
}

\begin{figure}
\centering
\includegraphics[width=0.85\textwidth]{../img/intellij/idea-import4-directory.png} 
\caption{Klicka \Button{Next} med förvalda alternativ oförändrade.}
\label{fig:idea:import4-directory}
\end{figure}

\begin{figure}
\centering
\includegraphics[width=0.85\textwidth]{../img/intellij/idea-import5-select-projects.png} 
\caption{Klicka \Button{Next} med alla projekt markerade.}
\label{fig:idea:import5-select-projects}
\end{figure}

\begin{figure}
\centering
\includegraphics[width=0.8\textwidth]{../img/intellij/idea-import6-select-SDK.png} 
\caption{Klicka \Button{Finish} med förifyllda fält oförändrade.}
\label{fig:idea:import6-select-SDK}
\end{figure}

\begin{figure}
\centering
\includegraphics[width=1.0\textwidth]{../img/intellij/idea-import7-setup-scala-sdk.png} 

\vspace{1em}{\hfill\includegraphics[width=0.6\textwidth]{../img/intellij/idea-import8-add-scala-support.png}} 
\caption{Bläddra fram PirateSpeech.scala i projektet \code{w04_pirates} och klicka på länken \textbf{Setup Scala SDK} och klicka \Button{OK} i efterföljande dialog.}
\label{fig:idea:import78-setup-scala-sdk}
\end{figure}

\begin{figure}
\centering
\includegraphics[width=1.0\textwidth]{../img/intellij/idea-import9-run.png} 
\caption{Lägg till utskriften i bilden ovan på rad7. Testkör genom att välja menyn \MenuArrow{Run}\Menu{Run..} (eller trycka Alt+Shift+F10) och sedan välja \code{PirateSpeech}. Observera utskriften i utskriftsfönstret.}
\label{fig:idea:import9-run}
\end{figure}




