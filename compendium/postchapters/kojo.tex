%!TEX encoding = UTF-8 Unicode
%!TEX root = ../compendium1.tex

\chapter{Kojo}\label{appendix:kojo}

\section{Vad är Kojo?}

Kojo%
\footnote{\href{https://en.wikipedia.org/wiki/Kojo_(programming_language)}{en.wikipedia.org/wiki/Kojo\_(programming\_language)}}
 är en integrerad utvecklingsmiljö för Scala som är speciellt anpassad för nybörjare i programmering. Kojo används i LTH:s Science Center Vattenhallen för utbildning av grundskolelärare i programmering och vid skolbesök och annan besöksverksamhet, i vilken lärare och studenter vid LTH arbetar som handledare. Kojo är fri öppenkällkod och utvecklingsgemenskapen leds av Lalit Pant från Indien.

Kursens första laboration genomförs med hjälp av Kojo, men Kojo kan med fördel användas som komplement till Scala REPL och annan IDE under hela kursens gång. Medan Scala REPL lämpar sig för korta kodsnuttar, och en fullfjädrad, professionell IDE har funktioner för att hantera riktigt stora programmeringsprojekt, passar Kojo bra för mellanstora program. I Kojo finns även lättillgängliga bibliotek som gör tröskeln lägre att programmera rörlig grafik och enkla spel.   


\begin{figure}[H]
\centering
\includegraphics[width=0.8\textwidth]{../img/kojo/kojo.png}
\caption{Den nybörjarvänliga utvecklingsmiljön Kojo för Scala på svenska.}
\label{fig:appendix:ide:kojo}
\end{figure} 

\subsection{Installera Kojo}\label{appendix:ide:kojo:install}

Kojo är förinstallerat på LTH:s datorer och körs igång med kommandot \texttt{kojo}. För instruktioner om hur du installerar Kojo på din egen dator se här:\\
\href{http://www.lth.se/programmera/installera/}{lth.se/programmera/installera}

Kojo kräver att \texttt{java} finns på din dator. Eftersom du behöver tillgång till JDK i kursen, är det lika bra att installera hela JDK direkt (och inte bara JRE, så som beskrivs å länken ovan); se vidare hur du gör detta i avsnitt \ref{appendix:compile:install-jdk}. 
%\href{http://www.kogics.net/kojo-download}{www.kogics.net/kojo-download}


\subsection{Använda Kojo}

När du startar Kojo första gången, välj ''Svenska'' i språkmenyn och starta om Kojo. Därefter fungerar grafikfunktionerna på svenska enligt tabell \ref{table:kojo:functions}. När du startat om Kojo inställt på svenska ser programmet ut ungefär som i figur \ref{fig:appendix:ide:kojo} på sidan \pageref{fig:appendix:ide:kojo}.


Det finns ett antal användbara kortkommando som du hittar i menyerna i Kojo. Undersök speciellt Ctrl+Alt+Mellanslag som ger autokomplettering baserat på det du börjat skriva.


{\small\renewcommand{\arraystretch}{1.45}
\begin{longtable}{@{}p{0.42\textwidth} p{0.55\textwidth}}

\caption{Några av sköldpaddans funktioner. Se även \href{http://lth.se/programmera}{lth.se/programmera}}\label{table:kojo:functions}\\

\emph{Svenska/Engelska} & \emph{Vad händer?}  \\ \hline
\code|sudda| \newline \code|clear| & Ritfönstret suddas \\
\code|fram| \newline \code|forward()| & Paddan går framåt 25 steg. \\
\code|fram(100)| \newline \code|forward(100)| & Paddan går framåt 100 steg. \\
\code|höger| \newline \code|right(90)| & Paddan vrider sig 90 grader åt höger. \\
\code|höger(45)| \newline \code|right(45)| & Paddan vrider sig 45 grader åt höger. \\
\code|vänster| \newline \code|left(90)| & Paddan vrider sig 90 grader åt vänster. \\
\code|vänster(45)| \newline \code|left(45)| & Paddan vrider sig 45 grader åt vänster. \\
\code|hoppa| \newline \code|hop| & Paddan hoppar 25 steg utan att rita. \\
\code|hoppa(100)| \newline \code|hop(100)| & Paddan hoppar 100 steg utan att rita. \\
\code|hoppaTill(100, 200)| \newline \code|jumpTo(100, 200)| & Paddan hoppar till läget (100, 200) utan att rita. \\
\code|gåTill(100, 200)| \newline \code|moveTo(100, 200)| & Paddan vrider sig och går till läget (100, 200). \\
\code|hem| \newline \code|home| & Paddan går tillbaka till utgångsläget (0, 0). \\
\code|öster| \newline \code|setHeading(0)| & Paddan vrider sig så att nosen pekar åt höger. \\
\code|väster| \newline \code|setHeading(180)| & Paddan vrider sig så att nosen pekar åt vänster. \\
\code|norr| \newline \code|setHeading(90)| & Paddan vrider sig så att nosen pekar uppåt. \\
\code|söder| \newline \code|setHeading(-90)  | & Paddan vrider sig så att nosen pekar neråt. \\
\code|mot(100,200)| \newline \code|towards(100, 200)| & Paddan vrider sig så att nosen pekar mot läget (100, 200) \\
\code|sättVinkel(90)| \newline \code|setHeading(90)| & Paddan vrider nosen till vinkeln 90 grader. \\
\code|vinkel| \newline \code|heading| & Ger vinkelvärdet dit paddans nos pekar. \\
\code|sakta(5000)| \newline \code|setAnimationDelay(5000) | & Gör så att paddan ritar jättesakta. \\
\code|suddaUtdata| \newline \code|clearOutput| & Utdatafönstret suddas. \\
\code|utdata("hej")| \newline \code|println("hej")| & Skriver texten \texttt{hej} i utdatafönstret. \\
\code|val t = indata("Skriv")| \newline \code|val t = readln("Skriv:")| & Väntar på inmatning efter ledtexten \texttt{Skriv} och sparar den inmatade texten i t.  \\
\code|textstorlek(100)| \newline \code|setPenFontSize(100)| & Paddan skriver med jättestor text nästa gång du gör skriv. \\
\code|båge(100, 90)| \newline \code|arc(100, 90)| & Paddan ritar en båge med radie 100 och vinkel 90. \\
\code|cirkel(100)| \newline \code|circle(radie)| & Paddan ritar en cirkel med radie 100. \\
\code|synlig| \newline \code|visible| & Paddan blir synlig. \\
\code|osynlig| \newline \code|invisible| & Paddan blir osynlig. \\
\code|läge.x| \newline \code|position.x| & Ger paddans x-läge \\
\code|läge.y| \newline \code|position.y| & Ger paddans y-läge \\
\code|pennaNer| \newline \code|penDown| & Sätter ner paddans penna så att den ritar när den går. \\
\code|pennaUpp| \newline \code|penUp| & Lyfter upp paddans penna så att den INTE ritar när den går. \\
\code|pennanÄrNere| \newline \code|penIsDown| & Kollar om pennan är nere eller inte. \\
\code|färg(rosa)| \newline \code|setPenColor(pink)| & Sätter pennans färg till rosa. \\
\code|fyll(lila)| \newline \code|setFillColor(purple)| & Sätter ifyllnadsfärgen till lila. \\
\code|fyll(genomskinlig)| \newline \code|setFillColor(noColor)| & Gör så att paddan inte fyller i något när den ritar. \\
\code|bredd(20)| \newline \code|setPenThickness(20)| & Gör så att pennan får bredden 20. \\
\code|sparaStil| \newline \code|saveStyle| & Sparar pennans färg, bredd och fyllfärg. \\
\code|laddaStil| \newline \code|restoreStyle| & Laddar tidigare sparad färg, bredd och fyllfärg. \\
\code|sparaLägeRiktning| \newline \code|savePosHe| & Sparar pennans läge och riktning \\
\code|laddaLägeRiktning| \newline \code|restorePosHe| & Laddar tidigare sparad riktning och läge \\
\code|siktePå| \newline \code|beamsOn| & Sätter på siktet. \\
\code|sikteAv| \newline \code|beamsOff| & Stänger av siktet. \\
\code|bakgrund(svart)| \newline \code|setBackground(black)| & Bakgrundsfärgen blir svart. \\
\code|bakgrund2(grön,gul)| \newline \code|setBackgroundV(green, yellow)| & Bakgrund med övergång från grönt till gult. \\
\code|upprepa(4){fram; höger}| \newline \code|repeat(4){forward; right}| & Paddan går fram och svänger höger 4 gånger. \\
\code|avrunda(3.99)| & Avrundar 3.99 till 4.0 \\
\code|slumptal(100)| & Ger ett slumptal mellan 0 och 99. \\
\code|slumptalMedDecimaler(100)| & Ger ett slumptal mellan 0 och 99.99999999 \\
\code|systemtid| & Ger nuvarande systemklocka i sekunder. \\
\code|räknaTill(5000)| & Kollar hur lång tid det tar för din dator att räkna till 5000. \\



\hline
\end{longtable}
}%end small

\noindent Scala-koden för den svenska paddans api finns här: \\
\href{https://bitbucket.org/lalit_pant/kojo/src/tip/src/main/scala/net/kogics/kojo/lite/i18n/svInit.scala}{bitbucket.org/lalit\_pant/kojo/src/tip/src/main/scala/net/kogics/\\kojo/lite/i18n/svInit.scala}









