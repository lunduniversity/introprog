%!TEX encoding = UTF-8 Unicode
%!TEX root = ../compendium.tex


\chapter{Versionshantering och kodlagring}

\section{Vad är versionshantering?}

\textbf{Versionshantering}\footnote{\href{https://en.wikipedia.org/wiki/Version_control}{en.wikipedia.org/wiki/Version\_control}} \Eng{version control eller revision control} av mjukvara innebär att hålla koll olika versioner av koden i ett utvecklingsprojekt allteftersom koden ändras. Versionshantering är en deldisciplin inom \textbf{konfigurationshantering} \Eng{software configuration managament} som inbegriper allt i processen för att identifiera, besluta, genomföra och följa upp ändringar.

En viktig del av versionshantering är att \textit{lagra} olika versioner av koden allt eftersom den utvecklas, så att tidigare versioner kan \textit{återskapas} vid behov. Ett bra verktygsstöd och en väldefinierad arbetsprocess för versionshanteringen, som alla i utvecklingsprojektet följer, möjliggör att flera utvecklare kan \textit{arbeta parallellt} med att sammanfoga \Eng{merge} varandras tillägg och ändringar i den gemensamma kodbasen utan att det blir kaos och förvirring.

God versionshantering är helt avgörande för utvecklarnas produktivitet, speciellt för stora projekt med många utvecklare som jobbar parallellt mot en omfattande kodbas med många olika interna och externa komponenter. 
Men även ett litet projekt med en enda utvecklare kan ha god nytta av ett versionshanteringsverktyg och ett disciplinerat förfarande för att namge versioner, t.ex. för att kunna återskapa tidigare versioner av projektets olika kodfiler när en ändring visar sig mindre lyckad.   

Det finns flera olika modeller för hur kodlagringen sker:
\begin{itemize}
\item \textbf{lokal}; alla utvecklare jobbar i samma, lokala filsystem där alla olika versioner lagras.
\item \textbf{centraliserad}; ett repositorium (förk. repo), alltså en databas med koden, finns centralt på en server som alla jobbar mot med hjällp av en versionshanteringsklient.
\item \textbf{distribuerad}; alla utvecklare har sitt eget lokala repo och varje utvecklare initierar enskilt när ändringar ska delas mellan olika repo. 
\end{itemize}


\section{Versionshanteringsverktyget Git}

Det finns många olika versionshanteringsverktyg\footnote{\href{https://en.wikipedia.org/wiki/List_of_version_control_software}{https://en.wikipedia.org/wiki/List\_of\_version\_control\_software}}
 som använder olika modeller för kodlagring; lokal, centraliserad, distribuerad eller kombinationer därav. 
På senare tid har verktyget \textbf{Git}\footnote{\href{https://en.wikipedia.org/wiki/Git_(software)}{https://en.wikipedia.org/wiki/Git\_(software)}} fått en stark ställning, speciellt i öppenkällkodsvärlden. Git utvecklades ursprungligen av Linus Torvalds för att versionshantera Linuxkärnan, men har växt till ett omfattande öppenkällkodsprojekt med stor spridning och många användare och bidragsgivare. 

Git möjliggör \textbf{distribuerad} versionshantering där var och en kan jobba snabbt och smidigt i sitt lokala repo, utan att behöva vänta på att en server ska synkronisera ett centralt repo över nätverket. Ändringar delas mellan repo på begäran ev enskilda utvecklare. 

Varje ny version av koden lagras som en avgränsad mängd ändringar sedan förra versionen, en s.k. \textbf{commit}%
\footnote{På svenska kan t.ex. ''inlämning'' användas, men låneordet commit är redan etablerat.}%
, och hanteras internt av Git i en lokal databas i katalogen \code{.git} som ligger överst i din projektkatalog. Genom olika kommandon i terminalen, eller via en klient med ett grafiskt användargränssnitt, kan din kod överföras till och från den lokala koddatabasen, alternativt delas med andra repon via nätet. 

Det finns en välskriven bok kallad \textit{''Pro Git''} som förklarar Git på djupet och är tillgänglig fritt här: 
\url{https://git-scm.com/book/en/v2}.
Läs kapitel 1 och 2 så får du en bra grund att stå på. 

Dessa termer är bra att kunna utantill innan du kör igång med Git:
\newcommand{\TermItem}[3]{\item \textbf{#1} (\textit{substantiv}: #2, \textit{verb}: #3).}
\begin{itemize}
\item \textbf{repo} (\textit{substantiv}: ett repositorium, \textit{eng. a repository}) En koddatabas med ändringshistorik. 
\TermItem{commit}{inlämning}{lämna in} 
  En avgränsad mängd ändringar sedan förra versionen lämnas in i det lokala repot.
\TermItem{push}{en leverans}{att leverera, att trycka upp} En eller flera inlämningar trycks upp till ett annat repo.
\TermItem{pull}{en hämtning}{att hämta, att dra ner} En eller flera inlämningar dras ner från ett annat repo.
\TermItem{merge}{en ihopslagning}{att sammanfoga} En eller flera inlämningar slås samman till en ny inlämning. 
\item \textbf{merge conflict} (\textit{substantiv}: en sammanfogningskonflikt, \textit{eng. a merge conflict}) Ändringar i samma kodfil som inte enkelt kan sammanfogas på ett entydigt sätt.
\TermItem{pull request}{en hämtningsbegäran, förk. PR}{att begära en hämtning} Utvecklare A ber en annan utvecklare B att hämta en eller flera inlämningar från A:s repo och sammanfoga med B:s repo.

\end{itemize}

\subsection{Installera git}

Git finns förinstallerat på LTH:s Linuxdatorer. Du kan kolla om Git redan finns på din maskin genom att skriva \code{git help} i terminalen. 

Det finns bra instruktioner om hur du installerar Git på din egen maskin här: \url{https://git-scm.com/book/en/v2/Getting-Started-Installing-Git}

Om du vill ha en Git-klient med grafiskt användargränssnitt finns det många att välja på, se här:  \url{https://git-scm.com/downloads/guis} 

Om du inte vet vilken du ska välja, prova GitGraken som är gratis, fri, öppen och finns för alla plattformar och kan laddas ner här: \\ \url{https://www.gitkraken.com/}

\subsection{Använda git}

\subsubsection{Anpassa Git}

Innan du börjar använda git, konfigurera ditt användarnamn och din email med nedan terminalkommando, där du anger ditt användarnamn i stället för \code{fornamnefternamn} och din mejladress i stället för \code{mejladr@plats.se}:
\begin{REPLnonum}
$ git config --global user.name fornamnefternamn
$ git config --global user.email mejladr@plats.se
\end{REPLnonum}
Det är bra att välja \textit{ett} användarnamn, för \textit{alla} repo, även kodlagringsplatser på nätet; förslagsvis \code{fornamnefternamn} utan svenska tecken,  så att du blir lätt att känna igen, speciellt om du jobbar med öppen källkod där ditt namn kommer associerat med alla de kodbidrag du gör under ditt yrkesliv.

Läs mer om hur du gör andra inställningar här, t.ex. hur du anger vilken editor som git startar när du ska skriva commit-beskrivningar: \\ \url{https://git-scm.com/book/en/v2/Getting-Started-First-Time-Git-Setup}
  
  
\subsubsection{Några vanliga kommandon}


  
\section{Kodlagringsplastser på nätet}

\begin{itemize}

\TermItem{fork}{en förgrening av ett helt repo}{att förgrena ett repo, ''forka''} En kopia av ett annat repo som utvecklas separat. Gör det möjligt för dig att införa ändringar i en kodbas, även om du inte har rättigheter att leverera till (''pusha till'') originalet. 
\item \textbf{upstream} (\textit{preposition}: uppströms, \textit{substantiv}: uppströmsrepo) Ett uppströmsrepo utgör orginal till ett förgrenat repo (fork). 
\begin{itemize}
\item Här beskrivs hur du länkar en förgrening uppströms: \\ 
{\small\url{https://help.github.com/articles/configuring-a-remote-for-a-fork/}}

\item Här beskrivs hur du synkar en förgrening uppströms:\\
{\small\url{https://help.github.com/articles/syncing-a-fork/}}

\end{itemize}

\end{itemize}



\subsection{GitLab}


\subsection{GitHub}

\subsubsection{Installera klienten för GitHub}

\subsubsection{Använda GitHub}


\subsection{BitBucket}

\subsubsection{Installera SourceTree}

\subsubsection{Använda BitBucket och SourceTree}
