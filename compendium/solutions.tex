%!TEX encoding = UTF-8 Unicode
\documentclass[a4paper]{compendium}

\externaldocument{lectures}
\externaldocument{labs}

\usepackage[swedish]{babel}

\setlength{\columnsep}{16mm}

\newcommand{\LibVersion}{1.1.5} % latest version of introlib at https://github.com/lunduniversity/introprog-scalalib
\newcommand{\LibJar}{\texttt{introprog\_3-\LibVersion.jar}}
\newcommand{\JDKApiUrl}{\url{https://docs.oracle.com/en/java/javase/11/docs/api/}}
\newcommand{\CurrentYear}{2021}
\newcommand{\VMName}{vm2020} %TODO: update vm
\newcommand{\VMPassword}{pgkBytMig\CurrentYear}
\newcommand{\VirtualBoxVersion}{6.1} %https://www.virtualbox.org/wiki/Downloads
\newcommand{\UbuntuVersion}{20.04}
\newcommand{\ScalaVersion}{3.0.1} %https://www.scala-lang.org/
\newcommand{\SbtVersion}{1.5.3} %https://eed3si9n.com/category/tags/sbt
\newcommand{\JDKVersion}{11} %https://adoptopenjdk.net/
\newcommand{\KojoVersion}{2.9.10} %https://www.kogics.net/kojo-download
\newcommand{\VSCodeVersion}{1.41} %https://code.visualstudio.com/updates
\newcommand{\MetalsVersion}{v1.10.6} %https://marketplace.visualstudio.com/items?itemName=scalameta.metals
\newcommand{\WindowsVersion}{10}
\newcommand{\ScalaIDEVersion}{4.7.0} %%DEPRECATED




\title{
{\vspace{-3.0cm}\bf\sffamily\Huge\selectfont  Introduktion till programmering med Scala}
\\ \vspace{1em}%\hspace*{1.5cm}\inputgraphics[width=0.6\textwidth]{../img/gurka} \\
{\sffamily Lösningar till övningar}\\\vspace{2cm}
%\includegraphics[height=4cm]{../img/scala-logo.png}
%\includegraphics[height=4cm]{../img/java-logo.png}
\includegraphics[height=12cm]{cover/gurka.jpg}
}

%\author{Redaktör: Björn Regnell}
\date{\raggedbottom%%%
\vspace{-2em}\begin{minipage}{1.0\textwidth}\centering
EDAA45, Lp1-2, HT \CurrentYear\\
Datavetenskap, LTH\\
Lunds Universitet\\
~\\
Kompileringsdatum: \today \\
\url{https://lunduniversity.github.io/pgk}
\end{minipage}
}

\usepackage{pgffor}  %% http://stackoverflow.com/questions/2561791/iteration-in-latex
                     %  allows:  \foreach \n in {1,...,4}{ do something with \n }

\usepackage{framed}  %  allows:   \begin{framed}\end{framed}
%\newenvironment{Slide}[2][]
%  {\begin{framed}\setlist{noitemsep}\section*{#2}}
%  {\end{framed}}


\usepackage[most]{tcolorbox}
\newenvironment{Slide}[2][]
  {\vspace{0.5em}\begin{tcolorbox}[%breakable,
                                   enhanced]\setlist{noitemsep}\SlideHeading{#2}}
  {\end{tcolorbox}\vspace{0.5em}}

\newcommand{\Subsection}[1]{} %ignore slide sections
\newcommand{\SlideOnly}[1]{} %ignore slide font size

\newif\ifkompendium  % to allow conditional text in slides only showing up in compendium
\kompendiumtrue      % in slides: \kompendiumfalse


%!TEX encoding = UTF-8 Unicode
\newcommand{\ExeWeekONE}{expressions}
\newcommand{\LabWeekONE}{kojo}

\newcommand{\ExeWeekTWO}{programs}
\newcommand{\LabWeekTWO}{--}

\newcommand{\ExeWeekTHREE}{functions}
\newcommand{\LabWeekTHREE}{irritext}

\newcommand{\ExeWeekFOUR}{objects}
\newcommand{\LabWeekFOUR}{blockmole}

\newcommand{\ExeWeekFIVE}{classes}
\newcommand{\LabWeekFIVE}{turtlegraphics}

\newcommand{\ExeWeekSIX}{sequences}
\newcommand{\LabWeekSIX}{shuffle}

\newcommand{\ExeWeekSEVEN}{sets-maps}
\newcommand{\LabWeekSEVEN}{words}

\newcommand{\ExeWeekEIGHT}{matrices}
\newcommand{\LabWeekEIGHT}{maze}

\newcommand{\ExeWeekNINE}{inheritance}
\newcommand{\LabWeekNINE}{turtlerace-team}

\newcommand{\ExeWeekTEN}{patterns}
\newcommand{\LabWeekTEN}{chords-team}

\newcommand{\ExeWeekELEVEN}{scala-java}
\newcommand{\LabWeekELEVEN}{lthopoly-team}

\newcommand{\ExeWeekTWELVE}{sorting}
\newcommand{\LabWeekTWELVE}{survey}

\newcommand{\ExeWeekTHIRTEEN}{--}
\newcommand{\LabWeekTHIRTEEN}{Projekt}

\newcommand{\ExeWeekFOURTEEN}{threads}
\newcommand{\LabWeekFOURTEEN}{--}


%Make section top heading not chapter:
\makeatletter
\renewcommand\thechapter{}
\renewcommand\thesection{\@arabic\c@section.}
\renewcommand\thesubsection{\@arabic\c@section.\@arabic\c@subsection}
\makeatother
\setcounter{tocdepth}{1}


\newif\ifPreSolution  % to allow tasks and solutions in same file
\PreSolutionfalse      % in non-solutions: \PreSolutiontrue

\let\QUESTBEGIN\ifPreSolution  % to mark formatting and numbering of exercises
\let\SOLUTION\else  % to mark solutions in the same file as questions
\let\QUESTEND\fi    % to mark end of exercise



\begin{document}
\maketitle
\mainmatter
\tableofcontents

%\part{Första läsperiodens övningar}

%!TEX encoding = UTF-8 Unicode
%!TEX root = ../exercises.tex

\ifPreSolution
\Exercise{\ExeWeekONE}\label{exe:W01}

\begin{Goals}
%!TEX encoding = UTF-8 Unicode

\item Förstå vad som händer när satser exekveras och uttryck evalueras.
\item Förstå sekvens, alternativ och repetition.
\item Känna till litteralerna för enkla värden, deras typer och omfång.
\item Kunna deklarera och använda variabler och tilldelning, samt kunna rita bilder av minnessituationen då variablers värden förändras.
\item Förstå skillnaden mellan olika numeriska typer, kunna omvandla mellan dessa och vara medveten om noggrannhetsproblem som kan uppstå.
\item Förstå booleska uttryck och värdena \code{true} och \code{false}, samt kunna förenkla booleska uttryck.
\item Förstå skillnaden mellan heltalsdivision och flyttalsdivision, samt användning av rest vid heltalsdivision.
\item Förstå precedensregler och användning av parenteser i uttryck.
\item Kunna använda \code{if}-satser och \code{if}-uttryck.
\item Kunna använda \code{for}-satser och \code{while}-satser.
\item Kunna använda \code{math.random()} för att generera slumptal i olika intervaller.
\item Kunna beskriva skillnader och likheter mellan en procedur och en funktion.

\end{Goals}

\begin{Preparations}
\item \StudyTheory{01}
\item Du behöver en dator med Scala och Kojo installerad, se appendix~\ref{appendix:compile} och  \ref{appendix:kojo}.
\end{Preparations}

\else

\ExerciseSolution{\ExeWeekONE}

\fi  %%% END \ifPreSolution


\BasicTasks
%%%% TODO Strukturera övningen annorlunda: atomer, sammansatta uttryck, funktiner, kojo ??}


\def\what{\emph{Para ihop begrepp med beskrivning.}}

\QUESTBEGIN

\Task \what

\vspace{1em}\noindent Koppla varje begrepp med den (förenklade) beskrivning som passar bäst: 

\begin{ConceptConnections}
  litteral & 1 & & A & att översätta kod till exekverbar form \\ 
  sträng & 2 & & B & anger ett specifikt datavärde \\ 
  sats & 3 & & C & decimaltal med begränsad noggrannhet \\ 
  uttryck & 4 & & D & bra då antalet repetitioner ej är bestämt i förväg \\ 
  funktion & 5 & & E & vid anrop sker (sido)effekt; returvärdet är tomt \\ 
  procedur & 6 & & F & sker innan exekveringen startat \\ 
  exekveringsfel & 7 & & G & bra då antalet repetitioner är bestämt i förväg \\ 
  kompileringsfel & 8 & & H & för att ändra en variabels värde \\ 
  abstrahera & 9 & & I & sker medan programmet kör \\ 
  kompilera & 10 & & J & beskriver vad data kan användas till \\ 
  typ & 11 & & K & antingen sann eller falsk \\ 
  for-sats & 12 & & L & vid anrop beräknas ett returvärde \\ 
  while-sats & 13 & & M & en kodrad som gör något; kan särskiljas med semikolon \\ 
  tilldelning & 14 & & N & kombinerar värden och funktioner till ett nytt värde \\ 
  flyttal & 15 & & O & en sekvens av tecken \\ 
  boolesk & 16 & & P & att införa nya begrepp som förenklar kodningen \\ 
\end{ConceptConnections}

\SOLUTION

\TaskSolved \what

\begin{ConceptConnections}
  litteral & 1 & ~~\Large$\leadsto$~~ &  C & anger ett specifikt datavärde \\ 
  sträng & 2 & ~~\Large$\leadsto$~~ &  G & en sekvens av tecken \\ 
  sats & 3 & ~~\Large$\leadsto$~~ &  K & en kodrad som gör något; kan särskiljas med semikolon \\ 
  uttryck & 4 & ~~\Large$\leadsto$~~ &  N & kombinerar värden och funktioner till ett nytt värde \\ 
  funktion & 5 & ~~\Large$\leadsto$~~ &  L & vid anrop beräknas ett returvärde \\ 
  procedur & 6 & ~~\Large$\leadsto$~~ &  H & vid anrop sker (sido)effekt; returvärdet är tomt \\ 
  exekveringsfel & 7 & ~~\Large$\leadsto$~~ &  P & kan inträffa medan programmet kör \\ 
  kompileringsfel & 8 & ~~\Large$\leadsto$~~ &  A & kan inträffa innan exekveringen startat \\ 
  abstrahera & 9 & ~~\Large$\leadsto$~~ &  F & att införa nya begrepp som förenklar kodningen \\ 
  kompilera & 10 & ~~\Large$\leadsto$~~ &  B & att översätta kod till exekverbar form \\ 
  typ & 11 & ~~\Large$\leadsto$~~ &  D & beskriver vad data kan användas till \\ 
  for-sats & 12 & ~~\Large$\leadsto$~~ &  O & bra då antalet repetitioner är bestämt i förväg \\ 
  while-sats & 13 & ~~\Large$\leadsto$~~ &  E & bra då antalet repetitioner ej är bestämt i förväg \\ 
  tilldelning & 14 & ~~\Large$\leadsto$~~ &  I & för att ändra en variabels värde \\ 
  flyttal & 15 & ~~\Large$\leadsto$~~ &  M & decimaltal med begränsad noggrannhet \\ 
  boolesk & 16 & ~~\Large$\leadsto$~~ &  J & antingen sann eller falsk \\ 
\end{ConceptConnections}

\QUESTEND






\def\what{\emph{Utskrift i Scala REPL.}}

\QUESTBEGIN

\Task \what 

\vspace{1em}\noindent Starta Scala REPL \Eng{Read-Evaluate-Print-Loop}.

\begin{REPLnonum}
$ scala
Welcome to Scala version 2.11.8 (Java HotSpot(TM) 64-Bit Server VM, Java 1.8).
Type in expressions to have them evaluated.
Type :help for more information.

scala> 
\end{REPLnonum}

\Subtask Skriv efter prompten \code{scala>} en sats som skriver ut en valfri (bruklig/knasig) hälsningsfras, genom anrop av proceduren \code{println} med något strängargument. Tryck på \textit{Enter} så att satsen kompileras och exekveras. 

\Subtask Skriv samma sats igen (eller tryck pil-upp) men ''glöm bort'' att skriva högerparentesen efter argumentet innan du trycker på \textit{Enter}. Vad händer?

\begin{framed}
\noindent\emph{Tips inför fortsättningen:} Det finns många användbara kortkommandon och andra trix för att jobba snabbt i REPL. Be gärna någon som kan dessa trix att visa dig hur man kan jobba snabbare. Läs appendix \ref{appendix:compile:REPL} och prova sedan att kopiera och klistra in text. Använd piltangenterna för att bläddra i historiken och Ctrl+A för att komma till början av raden, Ctrl+K för att radera resten av raden, etc.
\end{framed}



\SOLUTION 
\TaskSolved \what

\SubtaskSolved Till exempel:
\begin{REPLnonum}
scala> println("hejsan svejsan")
\end{REPLnonum}

\SubtaskSolved Om högerparentes fattas får man fortsätta skriva på nästa rad. Detta indikeras med vertikalstreck i början av varje ny rad:
\begin{REPLnonum}
scala> println("hejsan svejsan"
     | + "!" 
     | )
hejsan svejsan!
\end{REPLnonum}

\QUESTEND



\def\what{\emph{Konkatenering av strängar.}}

\QUESTBEGIN

\Task \what

\Subtask Skriv ett uttryck som konkatenerar två strängar, t.ex. \code{"gurk"} och \code{"burk"}, med hjälp av operatorn \code{+} och studera resultatet. Vad har uttrycket för värde och typ? Vilken siffra står efter ordet \code{res} i variabeln som lagrar resultatet?

\Subtask Använd resultatet från konkateneringen, t.ex. \code{res0} (byt ev. ut \code{0}:an mot siffran efter \code{res} i utskriften från förra evalueringen), och skriv ett uttryck med hjälp av operatorn \code{*} som upprepar resultatet från förra deluppgiften 42 gånger. 


\SOLUTION

\TaskSolved \what

\SubtaskSolved 
\begin{REPLnonum}
scala> "gurk" + "burk"
res1: String = gurkburk
\end{REPLnonum}
värde: \code{"gurkburk"}, typ:  \code{String}

\SubtaskSolved
\begin{REPLnonum}
scala> res1 * 42
res2: String = gurkatomatgurkatomatgurkatomatgurkatomatgurkatomatgurkatomatgurkatomatgurkatomatgurkatomatgurkatomatgurkatomatgurkatomatgurkatomatgurkatomatgurkatomatgurkatomatgurkatomatgurkatomatgurkatomatgurkatomatgurkatomatgurkatomatgurkatomatgurkatomatgurkatomatgurkatomatgurkatomatgurkatomatgurkatomatgurkatomatgurkatomatgurkatomatgurkatomatgurkatomatgurkatomatgurkatomatgurkatomatgurkatomatgurkatomatgurkatomatgurkatomatgurkatomat
\end{REPLnonum}

\QUESTEND




\def\what{\emph{När upptäcks felet?}}

\QUESTBEGIN

\Task \what 

\Subtask Vad har uttrycket \code{ "hej" * 3 } för typ och värde? Testa i REPL.

\Subtask Byt ut 3:an ovan mot ett så pass stort heltal så att minnet blir fullt. Hur börjar felmeddelandet? Är detta ett körtidsfel eller ett kompileringsfel?

\Subtask Välj ett värde på argumentet efter operatorn \code{*} så att ett typfel genereras. Hur börjar felmeddelandet? Är detta ett körtidsfel eller ett kompileringsfel?

\begin{framed}
\noindent\emph{Tips inför fortsättningen:} Gör gärna fel när du kodar så lär du dig mer! Träna på att tolka olika felmeddelanden och fråga någon om hjälp om du inte förstår. Kompilatorns utskrifter kan vara till stor hjälp, men är ibland kryptiska. Om du kör fast och inte kommer vidare själv så be om hjälp, \emph{men be om tips snarare än färdiga lösningar} så att du behåller initiativet själv och tar kontroll över nästa steg i ditt lärande.
\end{framed}


\SOLUTION

\TaskSolved \what

\SubtaskSolved Typ: \code{String}, värde: \code{"hejhejhej"}

\SubtaskSolved Körtiddsfel:
\begin{REPLnonum}
scala> "hej" * Int.MaxValue
java.lang.OutOfMemoryError: Java heap space
\end{REPLnonum}

\SubtaskSolved Kompileringsfel: (indikeras av texten \code{<console> ... error:})
\begin{REPLnonum}
scala> "hej" * true
<console>:12: error: type mismatch;
 found   : Boolean(true)
 required: Int
       "hej" * true
\end{REPLnonum}


\QUESTEND




\def\what{\emph{Litteraler och typer.}}

\QUESTBEGIN

\Task \what

\Subtask Ta hjälp av REPL-kommadot \verb+:type+ (kan förkortas \code{:t}) vid behov för att para ihop nedan litteraler med rätt typ. 

\begin{ConceptConnections}[0.35\textwidth]
  \code|1    | & 1 & & A & \code|Char   | \\ 
  \code|1L   | & 2 & & B & \code|Double | \\ 
  \code|1.0  | & 3 & & C & \code|Boolean| \\ 
  \code|1D   | & 4 & & D & \code|Int    | \\ 
  \code|1F   | & 5 & & E & \code|Boolean| \\ 
  \code|'1'  | & 6 & & F & \code|Double | \\ 
  \code|"1"| & 7 & & G & \code|Long   | \\ 
  \code|true | & 8 & & H & \code|Float  | \\ 
  \code|false| & 9 & & I & \code|String | \\ 
  \code|()   | & 10 & & J & \code|Unit   | \\ 
%\Connect{\code|1      |}  {\code|Int    |}
%\Connect{\code|1L     |}  {\code|Long   |}
%\Connect{\code|1.0    |}  {\code|Double |}
%\Connect{\code|1D     |}  {\code|Double |}
%\Connect{\code|1F     |}  {\code|Float  |}
%\Connect{\code|'1'    |}  {\code|Char   |}
%\Connect{\code|\"1\"  |}  {\code|String |}
%\Connect{\code|true   |}  {\code|Boolean|} 
%\Connect{\code|false  |}  {\code|Boolean|} 
%\Connect{\code|()     |}  {\code|Unit   |} 
\end{ConceptConnections}

\Subtask Vad händer om du adderar 1 till det största möjliga värdet av typen \code{Int}? 
\\\emph{Tips:} se snabbreferensen \footnote{\url{http://cs.lth.se/pgk/quickref/}} under rubriken ''The Scala type system'' avsnitt ''Methods on numbers''.

\Subtask Vad är skillnaden mellan typerna \code{Long} och \code{Int}?

\Subtask Vad är skillnaden mellan typerna \code{Double} och \code{Float}?


\SOLUTION

\TaskSolved \what

\SubtaskSolved 

\begin{ConceptConnections}
  \code|1    | & 1 & ~~\Large$\leadsto$~~ &  D & \code|Int    | \\ 
  \code|1L   | & 2 & ~~\Large$\leadsto$~~ &  J & \code|Long   | \\ 
  \code|1.0  | & 3 & ~~\Large$\leadsto$~~ &  G & \code|Double | \\ 
  \code|1D   | & 4 & ~~\Large$\leadsto$~~ &  C & \code|Double | \\ 
  \code|1F   | & 5 & ~~\Large$\leadsto$~~ &  B & \code|Float  | \\ 
  \code|'1'  | & 6 & ~~\Large$\leadsto$~~ &  H & \code|Char   | \\ 
  \code|"1"| & 7 & ~~\Large$\leadsto$~~ &  A & \code|String | \\ 
  \code|true | & 8 & ~~\Large$\leadsto$~~ &  E & \code|Boolean| \\ 
  \code|false| & 9 & ~~\Large$\leadsto$~~ &  F & \code|Boolean| \\ 
  \code|()   | & 10 & ~~\Large$\leadsto$~~ &  I & \code|Unit   | \\ 
%\ConnectSolved{\code|1      |}  {\code|Int    |}
%\ConnectSolved{\code|1L     |}  {\code|Long   |}
%\ConnectSolved{\code|1.0    |}  {\code|Double |}
%\ConnectSolved{\code|1D     |}  {\code|Double |}
%\ConnectSolved{\code|1F     |}  {\code|Float  |}
%\ConnectSolved{\code|'1'    |}  {\code|Char   |}
%\ConnectSolved{\code|\"1\"  |}  {\code|String |}
%\ConnectSolved{\code|true   |}  {\code|Boolean|} 
%\ConnectSolved{\code|false  |}  {\code|Boolean|} 
\end{ConceptConnections}

\SubtaskSolved Värdet går över gränsen för vad som får plats i ett 32 bitars heltal och ''börjar om'' på det minsta möjliga heltalet \code{Int.MinValue}
\begin{REPL}
scala> Int.MaxValue + 1
res3: Int = -2147483648

scala> Int.MinValue
res4: Int = -2147483648
\end{REPL}

\SubtaskSolved Båda är heltal men \code{Long} kan representera större tal än \code{Int}.

\SubtaskSolved Båda är flyttal men \code{Double} har dubbel precision och kan representera större tal med fler decimaler.



\QUESTEND





\def\what{\emph{Matematiska funktioner. Scaladoc.}}

\QUESTBEGIN

\Task \what

\Subtask Antag att du har ett schackbräde med 64 rutor. Tänk dig att du börjar med ett enda riskorn på första rutan och sedan lägger dubbelt så många riskorn i en ny hög för varje efterföljande ruta: 1, 2, 4, 8, ...  etc. Hur många riskorn\footnote{\url{https://en.wikipedia.org/wiki/Wheat_and_chessboard_problem}} blir det då i den sextiofjärde rishögen?

\emph{Tips:} Du ska beräkna $2^{64} - 1$. Om du skriver \code{math.} i REPL och trycker TAB får du se inbyggda matematiska funktioner i Scalas standardbibliotek:
\begin{REPL}
scala> math.    // Tryck TAB direkt efter punkten och betrakta listan
\end{REPL}
Använd funktionen \code{math.pow} och lämpliga argument. Om du skriver \code{math.pow} och trycker TAB \emph{två gånger} får du se funktionshuvudet med parameterlistan. 

Om du surfar till \url{http://www.scala-lang.org/api/current/} och skriver \code{math} i sökrutan och sedan, efter att du klickat på \textbf{\textsf{\small scala.math}}, skriver \textbf{\textsf{\small pow}} i rutan längre ner, så filtreras sidan och du hittar dokumentationen av \code{ def pow } som du kan klicka på och läsa mer om.   

\Subtask Definiera funktionen \code{omkrets} nedan i REPL. Går det bra att utelämna returtyp-annoteringen? Varför? Finns det anledning att ha den kvar?
\begin{Code}
def omkrets(radie: Double): Double = 2 * math.Pi * radie
\end{Code}

\Subtask Jordens (genomsnittliga) diameter (vid ekvatorn) är ca $12 750$ $km$. Anropa funktionen \code{omkrets} ovan för att beräkna hur många kilometer per dag man ungefär måste färdas om man vill åka jorden runt på 80 dagar. 

\SOLUTION

\TaskSolved \what

\SubtaskSolved Ja, returtyp-annoteringen \code{: Double} kan utelämnas. 

\begin{itemize}
\item Varför kan returtyp utelämnas?\\Eftersom kompilatorns typhärledning kan härleda returtypen. 
\item Varför kan man vilja utelämna den?\\Det blir kortare att skriva utan. 
\item Anledningar att ange returtyp: 
\begin{itemize}
\item  Med explicit returtyp får du hjälp av kompilatorn att redan under kompileringen kontrollera att uttrycket till höger om likhetstecknet har den typ som förväntas. 

\item Genom att du anger returtypen explicit får de som enbart läser metodhuvudet (och inte implementationen)
 tydligt se vad som returneras.
\end{itemize}
\end{itemize}	


\SubtaskSolved Beräkning av $2^{64} - 1$ med \code{math.pow} enligt nedan ger ungefär $1.8 \cdot 10^{19}$
\begin{REPL}
scala> math.pow(2, 64) - 1
res0: Double = 1.8446744073709552E19
\end{REPL}


\SubtaskSolved Ca $500$ $km$.
\begin{REPL}
scala> omkrets(12750 / 2) / 80
res0: Double = 500.6913291658733
\end{REPL}

\QUESTEND




\def\what{\emph{Förändringsbara variabler och tilldelning.}}

\QUESTBEGIN

\Task \what~Rita en \emph{ny} bild av datorns minne efter \emph{varje} exekverad rad 1--6 nedan. Varje bild ska visa alla variabler som finns i minnet och deras variabelnamn, typ och värde.

\begin{REPL}[numbers=left, numberstyle=\color{black}\ttfamily\scriptsize\selectfont]
scala> var a = 13
scala> var b = a + 1
scala> var c = (a + b) * 2.0
scala> b = 0
scala> a = 0
scala> c = c + 1
\end{REPL}
Efter första raden ser minnessituationen ut så här:

\MEM{a}{Int}{13}

\SOLUTION

\TaskSolved \what

\begin{tabular}{l l l}
\MEM{{\it Efter rad1:~~~~} a}{Int}{13}\\
\MEM{{\it Efter rad2:~~~~} a}{Int}{13} & \MEM{b}{Int}{14}\\
\MEM{{\it Efter rad3:~~~~} a}{Int}{13} & \MEM{b}{Int}{14} & \MEM{c}{Double}{54.0}\\
\MEM{{\it Efter rad4:~~~~} a}{Int}{13} & \MEM{b}{Int}{0} & \MEM{c}{Double}{54.0}\\
\MEM{{\it Efter rad5:~~~~} a}{Int}{0} & \MEM{b}{Int}{0} & \MEM{c}{Double}{54.0}\\
\MEM{{\it Efter rad6:~~~~} a}{Int}{0} & \MEM{b}{Int}{0} & \MEM{c}{Double}{55.0}\\
\end{tabular}

\QUESTEND


\def\what{\emph{Slumptal med \code{math.random}.}}

\QUESTBEGIN

\Task\label{exercise:expressions:roll} \what

\Subtask Vad ger funktionen \code{math.random} för resultatvärde? Vilken typ? Vad är största och minsta möjliga värde?
\\\emph{Tips:} Se scaladoc här: \Scaladoc och prova i REPL.

\Subtask Deklarera den parameterlösa funktionen \code{def roll: Int = ???} som ska representera ett tärningskast och ge ett slumpmässigt heltal mellan 1 och 6. Testa funktionen genom att anropa den många gånger. \\\emph{Tips:} Använd \code{math.random} och multiplicera och addera med lämpliga heltal. Omge beräkningen med parenteser och avsluta med \code{.toInt} för att avkorta decimaler och omvandla typen från \code{Double} till \code{Int}.

\SOLUTION

\TaskSolved \what

\SubtaskSolved Ur dokumentationen:
\begin{Code}
/** Returns a Double value with a positive sign, 
 *  greater than or equal to 0.0 and less than 1.0.
 */
def random(): Double
\end{Code}


\SubtaskSolved 
\begin{REPL}
scala> def roll: Int = (math.random * 6 + 1).toInt

scala> roll
res0: Int = 4

scala> roll
res1: Int = 1
\end{REPL}

\QUESTEND




\def\what{\emph{Repetition med \code{for}, \code{foreach} och \code{while}.}}

\QUESTBEGIN

\Task \what

\Subtask Så här kan en \code{for}-sats ser ut: 
\begin{Code}
for (i <- 1 to 10) print(i + ", ")
\end{Code}
Använd en \code{for}-sats för att skriva ut resultatet av 100 tärningskast med funktionen \code{roll} från uppgift \ref{exercise:expressions:roll}. 

\Subtask Så här kan en \code{foreach}-sats ser ut: 
\begin{Code}
(1 to 10).foreach { i => print(i + ",") }
\end{Code}
Använd en \code{foreach}-sats för att skriva ut resultatet av 100 tärningskast med funktionen \code{roll} från uppgift \ref{exercise:expressions:roll}. 

\Subtask Så här kan en \code{while}-sats ser ut: 
\begin{Code}
var i = 1
while (i <= 10) { print(i + ","); i = i + 1 }
\end{Code}
Använd en \code{while}-sats för att skriva ut resultatet av 100 tärningskast med funktionen \code{roll} från uppgift \ref{exercise:expressions:roll}. Vad händer om du glömmer \code{i = i + 1} ?


\SOLUTION

\TaskSolved \what

\SubtaskSolved \TODO

\QUESTEND


\def\what{\emph{Alternativ med \code{if}-sats och \code{if}-uttryck.}}

\QUESTBEGIN

\Task \what

\Subtask Så här kan en \code{if}-sats se ut (notera dubbla likhetstecken):
\begin{Code}
if (roll == 3) println("TRE") else println("INTE TRE") 
\end{Code}
Testa ovan i REPL. Skriv sedan en \code{for}-sats som kastar 100 tärningar och skriver ut strängen \code{"GRATTIS!"} om det blir en sexa, annars en ledsen smiley: \code{":("} 

\Subtask Så här kan ett \code{if}-uttryck se ut:
\begin{Code}
if (roll < 6) 0 else 1 
\end{Code}
Testa ovan i REPL. Skriv sedan en \code{while}-sats som kastar 100 tärningar och räknar antalet sexor. 

\SOLUTION

\TaskSolved \what

\SubtaskSolved \TODO

\QUESTEND



\def\what{\emph{Sekvens, sats och procedur.}}

\QUESTBEGIN

\Task \what

\Subtask Vad gör dessa satser? 
\begin{REPLnonum}
scala> def p = { print("san"); print("!"); println("hej")}
scala> p;p;p;p
\end{REPLnonum}

\Subtask
Använd pil-upp för att få tillbaka raden du skrev med definitionen av proceduren \code{p}. Byt plats på strängarna i utskriftsanropen i proceduren \code{p} så att utskriften blir: 
\begin{REPLnonum}
hejsan!
hejsan!
hejsan!
hejsan!
\end{REPLnonum}

\Subtask Hur tolkar kompilatorn klammerparenteser och semikolon?

\SOLUTION

\TaskSolved \what

\SubtaskSolved 
Satserna skapar denna utskrift:
\begin{REPLnonum}
san!hej
san!hej
san!hej
san!hej
\end{REPLnonum}

\SubtaskSolved 
\begin{REPLnonum}
scala> def p = { print("hej"); print("san"); println("!")}
scala> p;p;p;p
\end{REPLnonum}

\SubtaskSolved 
\begin{itemize}
\item Klammerparenteser används för att gruppera flera satser. Klammerparenteser behövs om man vill definiera en funktion som består av mer än en sats.  

\item Semikolon särskiljer flera satser. Semikolon behövs om man vill skriva många satser på samma rad.


\end{itemize}

\QUESTEND




\def\what{\emph{Heltalsdivision.}}

\QUESTBEGIN

\Task \what~Vilket värde och vilken typ hör till vilket uttryck?  Är du osäker på svaret, testa i REPL.

\begin{ConceptConnections}[0.3\textwidth]
  \code| 4 / 42      | & 1 & & A & \code|true : Boolean  | \\ 
  \code| 42.0 / 2    | & 2 & & B & \code|    2: Int      | \\ 
  \code| 42 / 4      | & 3 & & C & \code| 10.5: Double   | \\ 
  \code| 42 % 4      | & 4 & & D & \code|   10: Int      | \\ 
  \code| 4 % 42      | & 5 & & E & \code|    0: Int      | \\ 
  \code| 40 % 4 == 0 | & 6 & & F & \code|false: Boolean  | \\ 
  \code| 42 % 4 == 0 | & 7 & & G & \code|    4: Int      | \\ 
\end{ConceptConnections}

\SOLUTION

\TaskSolved \what

\begin{ConceptConnections}[0.3\textwidth]
  \code| 4 / 42      | & 1 & ~~\Large$\leadsto$~~ &  A & \code|    0: Int      | \\ 
  \code| 42.0 / 2    | & 2 & ~~\Large$\leadsto$~~ &  G & \code| 10.5: Double   | \\ 
  \code| 42 / 4      | & 3 & ~~\Large$\leadsto$~~ &  E & \code|   10: Int      | \\ 
  \code| 42 % 4      | & 4 & ~~\Large$\leadsto$~~ &  C & \code|    2: Int      | \\ 
  \code| 4 % 42      | & 5 & ~~\Large$\leadsto$~~ &  F & \code|    4: Int      | \\ 
  \code| 40 % 4 == 0 | & 6 & ~~\Large$\leadsto$~~ &  D & \code|true : Boolean  | \\ 
  \code| 42 % 4 == 0 | & 7 & ~~\Large$\leadsto$~~ &  B & \code|false: Boolean  | \\ 
\end{ConceptConnections}

\QUESTEND





\def\what{\emph{Booleska värden.}}

\QUESTBEGIN

\Task \what~Vilket värde har dessa uttryck?  % Uppgift 13

\Subtask \code{true && true}

\Subtask \code{false && true}

\Subtask \code{true || true}

\Subtask \code{false || true}

\Subtask \code{false || false}

\Subtask \code{true == true}

\Subtask \code{true != false}

\Subtask \code{true > false}

\Subtask \code{true && (1 / 0 > 1)}

\Subtask \code{false && (1 / 0 > 1)}

\SOLUTION

\TaskSolved \what

\SubtaskSolved \code{true}

\SubtaskSolved \code{false}

\SubtaskSolved \code{false}

\SubtaskSolved \code{true}

\SubtaskSolved \code{true}

\SubtaskSolved \code{false}

\SubtaskSolved \code{true}

\SubtaskSolved \code{true}

\SubtaskSolved Undantag kastas: \code{java.lang.ArithmeticException: / by zero}

\SubtaskSolved \code{false}

\QUESTEND





\def\what{\emph{Booleska variabler.}}

\QUESTBEGIN

\Task \what~Vad skrivs ut på rad 2 och 4 nedan?

\begin{REPL}
scala> var monster = false
scala> if (monster) println("akta dig!!!")
scala> monster = true
scala> if (monster) println("akta dig!!!")
\end{REPL}

\SOLUTION

\TaskSolved \what

\begin{itemize}
\item[Rad 2:] Ingenting skrivs ut.
\item[Rad 4:] \code{akta dig!!!}
\end{itemize}


\QUESTEND






\def\what{\emph{Turtle graphics med Kojo.}}

\QUESTBEGIN

\Task \what~På veckans laboration ska du använda Kojo för att verifiera att du kan använda sekvens, alternativ, repetition och abstraktion. Med Kojo kan du rita färgglada figurer med hjälp av ett lättanvänt Scala-bibliotek för \emph{turtle graphics}\footnote{\url{https://en.wikipedia.org/wiki/Turtle_graphics}}. 

Starta Kojo (se appendix \ref{appendix:kojo}). Om du inte redan har svenska menyer: välj svenska i språkmenyn och starta om Kojo.  Skriv in nedan program och tryck på den \emph{gröna} play-knappen. Notera kopplingen mellan satssekvensen och vad som händer i ritfönstret.

\begin{Code}
sudda

fram; höger
fram; vänster
färg(grön)
fram
\end{Code}
\noindent


\Subtask Vad händer om du \emph{inte} börjar programmet med \code{sudda} och kör samma program upprepade gånger? Varför är det bra att börja programmet med \code{sudda}?

\Subtask Skriv kod som ritar en kvadrat enligt bilden nedan.
\vspace{1em}\\\includegraphics[width=0.47\textwidth]{../img/kojo/kvadrat}

\noindent Prova gärna olika sätt att skriva din kod \emph{utan} att resultatet ändras: skriv satser i sekvens på flera rader eller satser i sekvens på samma rad med semikolon emellan; använd blanktecken och blanka rader i koden. Hur vill du gruppera dina satser så att de är lätta för en människa att läsa?
%Prova att ändra på \emph{ordningen} mellan satserna och studera hur resultatet påverkas. Använd den \emph{gula} play-knappen  (programspårning) för att studera exekveringen i detalj. Vad händer du klickar på satser i ditt program och på rutor i programspårningen?


\Subtask Rita en trappa enligt bilden nedan.

\includegraphics[width=0.3\textwidth]{../img/kojo/stairs}

\Subtask Rita valfri bild på valfri bakgrund med hjälp av några av procedurerna i tabellen nedan. Du kan till exempel rita en rosa triangel med lila konturer mot svart bakgrund. % \ref{lab:kojo:kojo-procedures}. 
Försök att underlätta läsbarheten av din kod med hjälp av lämpliga radbrytningar och gruppering av satser. 


\begin{table}[H]
\begin{longtable}{l l}\small
\code|fram(100)| & Paddan går framåt 100 steg (25 om argument saknas).\\
\code|färg(rosa)| & Sätter pennans färg till rosa. \\
\code|fyll(lila)| & Sätter ifyllnadsfärgen till lila. \\
\code|fyll(genomskinlig)| & Gör så att paddan \emph{inte} fyller i något när den ritar. \\
\code|bredd(20)| & Gör så att pennan får bredden 20. \\
\code|bakgrund(svart)| & Bakgrundsfärgen blir svart. \\
\code|bakgrund2(grön,gul)| & Bakgrund med övergång från grönt till gult. \\
\code|pennaNer|  & Sätter ner paddans penna så att den ritar när den går. \\
\code|pennaUpp|  & Sänker paddans penna så att den \emph{inte} ritar när den går. \\
\code|höger(45)|   & Paddan vrider sig 45 grader åt höger. \\
\code|vänster(45)| & Paddan vrider sig 45 grader åt vänster. \\
\code|hoppa|       & Paddan hoppar 25 steg utan att rita. \\
\code|hoppa(100)|  & Paddan hoppar 100 steg utan att rita. \\
\code|hoppaTill(100, 200)| & Paddan hoppar till läget (100, 200) utan att rita. \\
\code|gåTill(100, 200)|    & Paddan vrider sig och går till läget (100, 200). \\
\code|öster|   & Paddan vrider sig så att nosen pekar åt höger. \\
\code|väster|  & Paddan vrider sig så att nosen pekar åt vänster. \\
\code|norr|    & Paddan vrider sig så att nosen pekar uppåt. \\
\code|söder|   & Paddan vrider sig så att nosen pekar neråt. \\
\code|mot(100,200)|   & Paddan vrider sig så att nosen pekar mot läget (100, 200) \\
\code|sättVinkel(90)| & Paddan vrider nosen till vinkeln 90 grader. \\
\end{longtable}
%\label{lab:kojo:kojo-procedures}
%\caption{Några användbara procedurer i Kojo.}
\end{table}

\begin{framed}
\noindent\emph{Tips inför fortsättningen:} Ha gärna både REPL och Kojo igång samtidigt. Då kan du undersöka hur olika kodkonstruktioner fungerar i REPL, medan du stegvis skapar allt större program i editorn i Kojo. Detta sätt att jobba har du nytta av under resten av kursen, både om du använder en texteditor och kompilerar i terminalen, och om du använder en professionell integrerad utvecklingsmiljö. Oavsett vilka andra verktyg du kör är det användbart att ha REPL igång i ett eget fönster som hjälp i den kreativa processen, medan du jagar buggar och medan du lär dig nya koncept. Så fort du undrar hur något fungerar i Scala: fram med REPL och testa!
\end{framed}


\SOLUTION

\TaskSolved \what
 
\SubtaskSolved Genom att börja din Kojo-program med \code{sudda} så startar du exekveringen i samma utgångsläge: en tom rityta \Eng{canvas} där paddan pekar uppåt, pennan är nere och pennans färg är röd.  Då blir det lättare att resonera om vad programmet gör från början till slut, jämfört med om exekveringen beror på resultatet av tidigare exekveringar.


\SubtaskSolved
\begin{Code}
sudda

fram; vänster
fram; vänster
fram; vänster
fram; vänster
\end{Code}


\SubtaskSolved
\begin{Code}
sudda

fram; vänster
fram; höger

fram; vänster
fram; höger

fram; vänster
fram; höger

fram; vänster
\end{Code}


\QUESTEND









\clearpage

\ExtraTasks %%%%%%%%%%%%%%%%%% EXTRAUPPGIFTER



\def\what{\emph{Typ och värde.}}

\QUESTBEGIN

\Task \what~Vilket värde och vilken typ hör till vilket uttryck?  Är du osäker på svaret, testa i REPL.

\begin{ConceptConnections}[0.3\textwidth]
  \code|1.0 + 18          | & 1 & & A & \code|" ": String   | \\ 
  \code|(41 + 1).toDouble | & 2 & & B & \code|19.0: Double    | \\ 
  \code|1.042e42 + 1      | & 3 & & C & \code|57: Int         | \\ 
  \code|12E6.toLong       | & 4 & & D & \code|42.0: Double    | \\ 
  \code|32.toChar.toString| & 5 & & E & \code|48: Int         | \\ 
  \code|'A'.toInt         | & 6 & & F & \code|0: Int          | \\ 
  \code|0.toInt           | & 7 & & G & \code|1.042E42: Double| \\ 
  \code|'0'.toInt         | & 8 & & H & \code|'*': Char       | \\ 
  \code|'9'.toInt         | & 9 & & I & \code|12000000: Long  | \\ 
  \code|'A' + '0'         | & 10 & & J & \code|65: Int         | \\ 
  \code|('A' + '0').toChar| & 11 & & K & \code|'q': Char       | \\ 
  \code|"*!%#".charAt(0)| & 12 & & L & \code|113: Int        | \\ 
\end{ConceptConnections}

\SOLUTION

\TaskSolved \what

\begin{ConceptConnections}
  \code|1.0 + 18          | & 1 & ~~\Large$\leadsto$~~ &  B & \code|19.0: Double    | \\ 
  \code|(41 + 1).toDouble | & 2 & ~~\Large$\leadsto$~~ &  D & \code|42.0: Double    | \\ 
  \code|1.042e42 + 1      | & 3 & ~~\Large$\leadsto$~~ &  G & \code|1.042E42: Double| \\ 
  \code|12E6.toLong       | & 4 & ~~\Large$\leadsto$~~ &  I & \code|12000000: Long  | \\ 
  \code|32.toChar.toString| & 5 & ~~\Large$\leadsto$~~ &  A & \code|" ": String   | \\ 
  \code|'A'.toInt         | & 6 & ~~\Large$\leadsto$~~ &  J & \code|65: Int         | \\ 
  \code|0.toInt           | & 7 & ~~\Large$\leadsto$~~ &  F & \code|0: Int          | \\ 
  \code|'0'.toInt         | & 8 & ~~\Large$\leadsto$~~ &  E & \code|48: Int         | \\ 
  \code|'9'.toInt         | & 9 & ~~\Large$\leadsto$~~ &  C & \code|57: Int         | \\ 
  \code|'A' + '0'         | & 10 & ~~\Large$\leadsto$~~ &  L & \code|113: Int        | \\ 
  \code|('A' + '0').toChar| & 11 & ~~\Large$\leadsto$~~ &  K & \code|'q': Char       | \\ 
  \code|"*!%#".charAt(0)| & 12 & ~~\Large$\leadsto$~~ &  H & \code|'*': Char       | \\ 
\end{ConceptConnections}

%\Subtask \code{1.0 + 18}
%
%\Subtask \code{(41 + 1).toDouble}
%
%\Subtask \code{1.042e42 + 1}
%
%\Subtask \code{12E6.toLong}
%
%\Subtask \code{"gurk" + 'a'}
%
%\Subtask \code{32.toChar.toString}
%
%\Subtask \code{'A'.toInt}
%
%\Subtask \linebreak[0] \code{'0'.toInt}
%
%\Subtask \code{'0'.toInt}
%
%\Subtask \code{'9'.toInt}
%
%\Subtask \code{'A' + '0'}
%
%\Subtask \code{('A' + '0').toChar}
%
%\Subtask \code{"*!%#".charAt(0)}
%%%%%%%%%%%%%%%%%%%%%%%%%%%%%%%%%%%%%%%%%%%%%%%%
%\SubtaskSolved \code{Double, 19}
%
%\SubtaskSolved \code{Double, 42}
%
%\SubtaskSolved \code{Double, 1.042E42}
%
%\SubtaskSolved \code{Long, 12000000}
%
%\SubtaskSolved \code{String, gurka}
%
%\SubtaskSolved \code{String, " "}
%
%\SubtaskSolved \code{Int, 65}
%
%\SubtaskSolved \code{Int, 48}
%
%\SubtaskSolved \code{Int,49}
%
%\SubtaskSolved \code{Int,57}
%
%\SubtaskSolved \code{Int, 113}
%
%\SubtaskSolved \code{Char, 'q'}
%
%\SubtaskSolved \code{Char, '*'}


\QUESTEND




\def\what{\emph{Satser och uttryck.}}

\QUESTBEGIN

\Task \what

\Subtask Vad är det för skillnad på en sats och ett uttryck?

\Subtask Ge exempel på satser som inte är uttryck?

\Subtask Förklara vad som händer för varje evaluerad rad:
\begin{REPL}
scala> def värdeSaknas = ()
scala> värdeSaknas
scala> värdeSaknas.toString
scala> println(värdeSaknas)
scala> println(println("hej"))
\end{REPL}

\Subtask Vilken typ har literalen \code{()}?

\Subtask Vilken returtyp har \code{println}?

\SOLUTION

\TaskSolved \what

\SubtaskSolved  Ett utryck kan evalueras och resulterar då i ett användbart värde. En sats \emph{gör} något (t.ex. skriver ut något), men resulterat inte i något användbart värde.

\SubtaskSolved \code{println()}

\SubtaskSolved 

 Värdesaknas innehåller Unit

 Skriver ut \code{Unit}

 Skriver ut \code{"()"}

 Skriver ut \code{"()"}

 Skriver först ut hej med det innersta anropet och sen \code{()} med det yttre anropet

\SubtaskSolved  \code{Unit}

\SubtaskSolved  \code{Unit}

\QUESTEND



\def\what{\emph{Procedur med parameter.} \TODO}

\QUESTBEGIN

\Task \what~En procedur är en funktion som orsakar en effekt, till exempel en utskrift eller en variabeltilldelning, men som inte returnerar något intressant resultatvärde. \footnote{I Scala är procedurer funktioner som returnerar det \emph{tomma värdet}, vilket skrivs \code{()} och är av typen \code{Unit}. I Java och flera andra språk finns inget tomt värde och man har en specialsyntax för procedurer som använder nyckelordet \code{void}. }

\Subtask Deklarera en förändringsbar variabel \code{highscore} som initieras till 0.

\Subtask Deklarera en procedur \code{updateHighscore} som tar en parameter \code{points} och tilldelar \code{highscore} \TODO ...


\SOLUTION

\TaskSolved \what

\SubtaskSolved 

\QUESTEND





\def\what{\emph{\code{if}\textit{-sats}.}}

\QUESTBEGIN

\Task \what~För varje rad nedan, beskriv vad som skrivs ut.  % Uppgift 18
\begin{REPL}
scala> if (!true) println("sant") else println("falskt")
scala> if (!false) println("sant") else println("falskt")
scala> def singlaSlant = if (math.random > 0.5) "krona" else "klave"
scala> for (i <- 1 to 5) print(s"$i:$singlaSlant ")
\end{REPL}

\SOLUTION

\TaskSolved \what

\begin{enumerate}
\item Utskrift: \code{falskt}
\item Utskrift: \code{sant}
\item Inget skrivs ut, funktionen deklareras men körs ej.
\item Utskrift: code{1:krona 2:klave 3:krona 4:krona 5:klave }
\end{enumerate}

\QUESTEND





\def\what{\emph{\code{if}\textit{-uttryck}.}}

\QUESTBEGIN

\Task  Deklarera följande variabler med nedan initialvärden:  

\begin{REPLnonum}
scala> var grönsak = "gurka"
scala> var frukt = "banan"
\end{REPLnonum}

Ange för varje rad nedan vad uttrycket har för värde och typ:
\begin{REPLnonum}
scala> if (grönsak == "tomat") "gott" else "inte gott" 
scala> if (frukt == "banan") "gott" else "inte gott" 
scala> if (true) grönsak else 42 
scala> if (false) grönsak else 42 
\end{REPLnonum}

\SOLUTION


\TaskSolved \what~Notera typen \code{Any} på de sista två uttrycken.

\begin{REPLnonum}
scala> if (grönsak == "tomat") "gott" else "inte gott"
res0: String = inte gott

scala> if (frukt == "banan") "gott" else "inte gott"
res1: String = gott

scala> if (true) grönsak else 42
res2: Any = gurka

scala> if (false) grönsak else 42
res3: Any = 42
\end{REPLnonum}


\QUESTEND






\def\what{\emph{QUESTTEMPLATE}}

\QUESTBEGIN

\Task \what

\Subtask

\SOLUTION

\TaskSolved \what

\SubtaskSolved 

\QUESTEND




\clearpage

\AdvancedTasks   %%%%%%%%%%%%%%%%%%% FÖRDJUPNINGSUPPGIFTER




\def\what{\emph{Stränginterpolatorn \code{s}.}}

\QUESTBEGIN

\Task \what~Med ett \code{s} framför en strängliteral får man hjälp av kompilatorn att, på ett typsäkert sätt, infoga variabelvärden i en sträng. 
Variablernas namn ska föregås med ett dollartecken, t.ex. \code{s"Hej $namn"}.  
Om man vill evaluera ett uttryck placeras detta inom klammer direkt efter dollartecknet, t.ex.
\code/s"Dubbla längden: ${namn.size * 2}"/  

\Subtask Vad skrivs ut nedan?
\begin{REPL}
scala> val f = "Kim"
scala> val e = "Finkodare"
scala> println(s"Namnet '$f $e' har ${f.size + e.size} bokstäver.")
\end{REPL}

\Subtask Skapa följande utskrifter med hjälp av stränginterpolatorn \code{s} och variablerna \code{f} och \code{e} i föregående deluppgift.
\begin{REPL}
Kim har 3 bokstäver.
Finkodare har 9 bokstäver.
\end{REPL}

\SOLUTION

\TaskSolved \what

\SubtaskSolved 
\begin{REPLnonum}
Namnet 'Kim Finkodare' har 12 bokstäver.
\end{REPLnonum}

\SubtaskSolved 
\begin{REPLnonum}
println(s"$f har  ${f.size} bokstäver.")
println(s"$e har  ${e.size} bokstäver.")
\end{REPLnonum}

\QUESTEND






\def\what{\emph{Flyttalsaritmetik}}

\QUESTBEGIN

\Task \what

\Subtask Vilket är det minsta positiva värdet av typen \code{Double}?

\Subtask Vad är värdet av detta uttryck? Varför blir det så?
\begin{REPL}
scala> Double.MaxValue + Double.MinPositiveValue == Double.MaxValue
\end{REPL}

\SOLUTION

\TaskSolved \what

\SubtaskSolved 

\begin{REPL}
scala> Double.MinPositiveValue
res0: Double = 4.9E-324
\end{REPL}

\SubtaskSolved 

\begin{REPL}
scala> Double.MaxValue + Double.MinPositiveValue == Double.MaxValue
res2: Boolean = true
\end{REPL}

\QUESTEND




\def\what{\emph{Stora tal.}}

\QUESTBEGIN

\Task \what~Om vi vill beräkna $2^{64} -1$ som ett exakt heltal\footnote{\url{https://en.wikipedia.org/wiki/Wheat_and_chessboard_problem}} blir det större än \code{Int.MaxValue}, så vi kan tyvärr inte använda snabba \code{Int}. Till vår räddning: \code{BigInt} 

\Subtask Läs om \code{BigInt} och \code{BigDecimal} på \Scaladoc \\ Notera vad de kan användas till. 

\Subtask Du skapar ett \code{BigInt}-heltal med \code{BigInt(2)} och kan anropa funktionen \code{pow} på en \code{BigInt} med punktnotation. Beräkna $2^{64} -1$ som ett exakt heltal.

\Subtask Vilka nackdelar finns med \code{BigInt} och \code{BigDecimal}?

\SOLUTION

\TaskSolved \what

\SubtaskSolved \code{BigInt} kan användas i stället för \code{Int} vid mycket stora heltal. \code{BigDecimal} kan användas i stället för \code{Double} vid mycket stora decimaltal.

\SubtaskSolved 
\begin{REPL}
scala> BigInt(2).pow(64)
res0: scala.math.BigInt = 18446744073709551616
\end{REPL}

\SubtaskSolved Beräkningar går mycket långsammare och de är lite krångligare att använda.

\QUESTEND





\def\what{\emph{Precedensregler}}

\QUESTBEGIN

\Task \what~Evalueringsordningen kan styras med parenteser. Vilket värde och vilken typ har följande uttryck? 

\Subtask \code{23 + 2 * 2 + (23 + 2) * 2}

\Subtask \code{(-(2 - 42)) / (1 + 1 + 1)}

\Subtask \code{(-(2 - 42)) / (-1)/(1 + 1 + 1)}

\SOLUTION

\TaskSolved \what

\SubtaskSolved \code{77:  Int}

\SubtaskSolved \code{13: Int}

\SubtaskSolved \code{-13: Int}

\QUESTEND






\def\what{\emph{QUESTTEMPLATE}}

\QUESTBEGIN

\Task \what

\Subtask

\SOLUTION

\TaskSolved \what

\SubtaskSolved 

\QUESTEND




\subsection{TODO}

\TODO{SAKERNA NEDAN SKA FLYTTAS/UPPDATERAS/TAS BORT???} 
%%%%%%%%%%%%%%%%%%%%%%%%%%%%%%%%%%%%%%%%%%%%%%%%
%%%%%%%%%%%%%%%%%%%%%%%%%%%%%%%%%%%%%%%%%%%%%%%%
%%%%%%%%%%%%%%%%%%%%%%%%%%%%%%%%%%%%%%%%%%%%%%%%
%%%%%%%%%%%%%%%%%%%%%%%%%%%%%%%%%%%%%%%%%%%%%%%%
%%%%%%%%%%%%%%%%%%%%%%%%%%%%%%%%%%%%%%%%%%%%%%%%
%%%%%%%%%%%%%%%%%%%%%%%%%%%%%%%%%%%%%%%%%%%%%%%%
%%%%%%%%%%%%%%%%%%%%%%%%%%%%%%%%%%%%%%%%%%%%%%%%
%%%%%%%%%%%%%%%%%%%%%%%%%%%%%%%%%%%%%%%%%%%%%%%%
%%%%%%%%%%%%%%%%%%%%%%%%%%%%%%%%%%%%%%%%%%%%%%%%


\ifPreSolution  %%% TODO remove \fi at end of file and break sultions into pieces





\Task Klassen \code{java.lang.Math} och paketobjektet \code{scala.math}. % Uppgift 11
Genom att trycka på tab tagenten kan man se vad som finns i olika paket.

\begin{REPL}
scala> java.    //tryck TAB efter punkten
applet   awt   beans   io   lang   math   net   nio   rmi   security   sql

scala>
\end{REPL}

\Subtask Undersök genom att trycka på Tab-tangenten, vilka funktioner som finns i \code{Math} och \code{math}. Vad heter konstanten $\pi$ i java.lang.Math respektive scala.math?

\begin{REPL}
scala> java.lang.Math.    //tryck TAB efter punkten
scala> scala.math.        //tryck TAB efter punkten
\end{REPL}

\Subtask Undersök dokumentationen för klassen \code{java.lang.Math} här: \\ \url{https://docs.oracle.com/javase/8/docs/api/java/lang/Math.html} \\
Vad gör \code{java.lang.Math.hypot}?

\Subtask Undersök dokumentationen för paketobjektet \code{scala.math} här: \\
\url{http://www.scala-lang.org/api/current/#scala.math.package} \\
Ge exempel på någon funktion i \code{java.lang.Math} som inte finns i \code{scala.math}.

%\TaskSection{Noggrannhet och undantag i aritmetiska uttryck}

\Task Vad händer här? Notera undantag \Eng{exceptions} och noggrannhetsproblem. % Uppgift 12

\Subtask \code{Int.MaxValue} + 1

\Subtask \code{1 / 0}

\Subtask \code{1E8 + 1E-8}

\Subtask \code{1E9 + 1E-9}

\Subtask \code{math.pow(math.hypot(3,6), 2)}

\Subtask \code{1.0 / 0}

\Subtask \code{(1.0 / 0).toInt}

\Subtask \code{math.sqrt(-1)}

\Subtask \code{math.sqrt(Double.NaN)}

\Subtask \code{throw new Exception("PANG!!!")}





\Task \textit{Deklarationer: \code{var}, \code{val}, \code{def}}. Evaluera varje rad nedan i tur och ordning i Scala REPL.  % Uppgift 15
\begin{REPL}[numbers=left, numberstyle=\color{black}\ttfamily\scriptsize\selectfont]
scala> var x = 30
scala> x + 1
scala> x
scala> x = x + 1
scala> x
scala> x == x + 1
scala> val y = 20
scala> y = y + 1
scala> var z = {println("gurka"); 10}
scala> def w = {println("gurka"); 10}
scala> z
scala> z
scala> z = z + 1
scala> w
scala> w
scala> w = w + 1
\end{REPL}

\Subtask För varje rad ovan: förklara för vad som händer.

\Subtask Vilka rader ger kompileringsfel och i så fall vilket och varför?

\Subtask\Pen Vad är det för skillnad på \code{var}, \code{val} och \code{def}?

\Subtask\Pen Tilldela variabeln \code{val even } värdet av ett uttryck som med modulo-operatorn \code
och olikhetsoperatorn \code{!=} testar om ett tal \code{n} är udda.


\Task\Pen \emph{Tilldelningsoperatorer.} Man kan förkorta en tilldelningssats som förändrar en variabel, t.ex. \code{x = x + 1}, genom att använda så kallade tilldelningsoperatorer och skriva \code{x += 1} som betyder samma sak. Rita en ny bild av datorns minne efter varje evaluerad rad nedan. Bilderna ska visa variablers namn, typ och värde.  % Uppgift 16

\begin{REPL}
scala> var a = 40
scala> var b = a + 40
scala> a += 10
scala> b -= 10
scala> a *= 2
scala> b /= 2
\end{REPL}



\Task \emph{Stränginterpolatorn \code{s}.} Man behöver ofta skapa strängar som innehåller variabelvärden. Med ett \code{s} framför en strängliteral får man hjälp av kompilatorn att, på ett typsäkert sätt, infoga variabelvärden i en sträng. Variablernas namn ska föregås med ett dollartecken, t.ex. \code{s"Hej $namn"}.  Om man vill evaluera ett uttryck placeras detta inom klammer direkt efter dollartecknet, t.ex.
\code/s"Dubbla längden: ${namn.size * 2}"/  % Uppgift 17

\begin{REPL}
scala> val f = "Kim"
scala> val e = "Finkodare"
scala> val tot = f.size + e.size
scala> println(s"Namnet '$f $e' har $tot bokstäver.")
scala> println(s"Efternamnet '$e' har ${e.size} bokstäver.")
\end{REPL}

\Subtask Vad skrivs ut ovan?

\Subtask Skapa följande utskrifter med hjälp av stränginterpolatorn \code{s} och lämpliga variabler.
\begin{REPL}
Namnet 'Kim' har 3 bokstäver.
Namnet 'Finkodare' har 9 bokstäver.
\end{REPL}



\Task \code{if}\textit{-sats}.För varje rad nedan; förklara vad som händer.  % Uppgift 18
\begin{REPL}
scala> if (true) println("sant") else println("falskt")
scala> if (false) println("sant") else println("falskt")
scala> if (!true) println("sant") else println("falskt")
scala> if (!false) println("sant") else println("falskt")
scala> def singlaSlant =
scala> 	 if (math.random > 0.5) print(" krona") else print(" klave")
scala> singlaSlant; singlaSlant; singlaSlant
\end{REPL}


\Task \code{if}\textit{-uttryck}. Deklarera följande variabler med nedan initialvärden:  % Uppgift 19

\begin{REPLnonum}
scala> var grönsak = "gurka"
scala> var frukt = "banan"
\end{REPLnonum}

Vad har följande uttryck för värden och typ?

\Subtask \code{if (grönsak == "tomat") "gott" else "inte gott" }

\Subtask \code{if (frukt == "banan") "gott" else "inte gott" }

\Subtask \code{if (frukt.size == grönsak.size) "lika stora" else "olika stora" }

\Subtask \code{if (true) grönsak else frukt }

\Subtask \code{if (false) grönsak else frukt }


\Task \code{for}\textit{-sats}.  Med bakåtpilen \texttt{<-} kan man i en \code{for}-sats ange vilka värden som ska gås igenom i sekvens. Vid varje runda i loopen får en lokal variabel ett nytt värde i sekvensen. % Uppgift 20

\Subtask Vad ger nedan \code{for}-satser för utskrift?

\begin{REPL}
scala> for (i <- 1 to 10) print(i + ", ")
scala> for (i <- 1 until 10) print(i + ", ")
scala> for (i <- 1 to 5) print((i * 2) + ", ")
scala> for (i <- 1 to 92 by 10) print(i + ", ")
scala> for (i <- 10 to 1 by -1) print(i + ", ")
\end{REPL}

\Subtask Skriv en \code{for}-sats som ger följande utskrift:
\begin{REPLnonum}
A1, A4, A7, A10, A13, A16, A19, A22, A25, A28, A31, A34, A37, A40, A43,
\end{REPLnonum}

\Task Repetition med metoden \code{foreach}. Efter framåtpilen \texttt{=>} (se nedan) anges vad som ska hända för varje element som gås igenom sekventiellt. Vid varje runda i loopen får en lokal variabel ett nytt värde i sekvensen.   % Uppgift 21

\Subtask Vad ger nedan satser för utskrifter?

\begin{REPL}
scala> (9 to 19).foreach{i => print(i + ", ")}
scala> (1 until 20).foreach{i => print(i + ", ")}
scala> (0 to 33 by 3).foreach{i => print(i + ", ")}
\end{REPL}

\Subtask Använd \code{foreach} och skriv ut följande:
\begin{REPLnonum}
B33, B30, B27, B24, B21, B18, B15, B12, B9, B6, B3, B0,
\end{REPLnonum}

\Task \code{while}\textit{-sats}. En sats eller ett block med satser upprepas så länge ett villkor är sant.  % Uppgift 22

\Subtask Vad ger nedan satser för utskrifter?
\begin{REPL}
scala> var i = 0
scala> while (i < 10) { println(i); i = i + 1 }
scala> var j = 0; while (j <= 10) { println(j); j = j + 2 }; println(j)
\end{REPL}

\Subtask Skriv en \code{while}-sats som ger följande utskrift. Använd en variabel \code{k} som initialiseras till 1.
\begin{REPLnonum}
A1, A4, A7, A10, A13, A16, A19, A22, A25, A28, A31, A34, A37, A40, A43,
\end{REPLnonum}

\Subtask\Pen Vilken av \code{for}, \code{while} och \code{foreach} är kortast att skriva om man vill repetera mer än en sats 100 gånger? Vilken tycker du är lättast att läsa?

\Task \textit{Slumptal}. Undersök vad dokumentationen säger om funktionen \code{scala.math.random}:\\  % Uppgift 23
\url{http://www.scala-lang.org/api/current/#scala.math.package}

\Subtask\Pen Vilken typ har värdet som returneras av funktionen \code{random}?

\Subtask\Pen Vilket är det minsta respektive största värde som kan returneras?

\Subtask\Pen Är \code{random} en \textit{äkta} funktion \Eng{pure function} i matematisk mening?

\Subtask Anropa funktionen \code{math.random} upprepade gånger och notera vad som händer. Använd pil-upp-tangenten.
\begin{REPLnonum}
scala> math.random
\end{REPLnonum}


\Subtask Vad händer? Använd \textit{pil-upp} och kör nedan \code{for}-sats flera gånger. Förklara vad som sker.

\begin{REPLnonum}
scala> for (i <- 1 to 20) println((math.random * 3 + 1).toInt)
\end{REPLnonum}

\Subtask Skriv en \code{for}-sats som skriver ut 100 slumpmässiga heltal från 0 till och med 9 på var sin rad.

\begin{REPLnonum}
scala> for (i <- 1 to 100) println(???)
\end{REPLnonum}

\Subtask Skriv en \code{for}-sats som skriver ut 100 slumpmässiga heltal från 1 till och med 6 på samma rad.

\begin{REPLnonum}
scala> for (i <- 1 to 100) print(???)
\end{REPLnonum}


\Subtask Använd \textit{pil-upp} och kör nedan \code{while}-sats flera gånger. Förklara vad som sker.

\begin{REPLnonum}
scala> while (math.random > 0.2) println("gurka")
\end{REPLnonum}

\Subtask Ändra i \code{while}-satsen ovan så att sannolikheten ökar att riktigt många strängar ska skrivas ut.

\Subtask Förklara vad som händer nedan.
\begin{REPL}
scala> var slumptal = math.random
scala> while (slumptal > 0.2) { println(slumptal); slumptal = math.random }
\end{REPL}

\Task\Pen \textit{Logik och De Morgans Lagar}. Förenkla följande uttryck. Antag att \code{poäng} och \code{highscore} är heltalsvariabler medan \code{klar} är av typen \code{Boolean}.
  % Uppgift 24

\Subtask \code{poäng > 100 && poäng > 1000}

\Subtask \code{poäng > 100 || poäng > 1000}

\Subtask \code{!(poäng > highscore)}

\Subtask \code{!(poäng > 0 && poäng < highscore) }

\Subtask \code{!(poäng < 0 || poäng > highscore) }

\Subtask \code{klar == true}

\Subtask \code{klar == false}


\clearpage

\ExtraTasks

\Task \textit{Slumptal}.

\Subtask Ersätt \code{???} nedan med literaler så att \code{tärning} returnerar ett slumpmässigt heltal mellan 1 och 6.
\begin{REPLnonum}
scala> def tärning = (math.random * ??? + ???).toInt
\end{REPLnonum}

\Subtask Ersätt \code{???} med literaler så att \code{rnd} blir ett decimaltal med max en decimal mellan 0.0 och 1.0.
\begin{REPLnonum}
scala> def rnd = math.round(math.random * ???) / ???
\end{REPLnonum}

\Subtask Vad blir det för skillnad om \code{math.round} ersätts med \code{math.floor} ovan? (Se dokumentationen av \code{java.lang.Math.round} och \code{java.lang.Math.floor}.)

\Task Undersök vad som finns i paketet \code{scala.math} genom att studera dess dokumentation: \href{http://www.scala-lang.org/api/current/#scala.math.package}{www.scala-lang.org/api/current/\#scala.math.package} och gör några matematiska beräkningar i REPL som använder olika funktioner i \code{math}-paketet.

\Task\Pen Antag att du byter plats mellan satsen efter villkoret och satsen efter \code{else} i \code{if}-satsen nedan. Hur kan du ändra i villkoret så att det ändå skrivs ut samma sak som före bytet?
\begin{Code}
if (x == 42) println("the meaning of it all") else println(":(")
\end{Code}

\Task\Pen Rita en ny bild av datorns minne efter varje evaluerad rad nedan. Bilderna ska visa variablers namn, typ och värde.
\begin{REPL}
scala> var x = 42
scala> var y = x + 1
scala> x += -1
scala> y -= 1
\end{REPL}

\Task Skapa med hjälp av \code{while} några olika oändliga loopar som skriver ut olika saker vid varje loop-runda.

\Task Hitta på några egna övningar för att träna mer på De Morgans lagar.



\clearpage

\AdvancedTasks

\Task Läs om moduloräkning här \href{https://en.wikipedia.org/wiki/Modulo\_operation}{en.wikipedia.org/wiki/Modulo\_operation} och undersök hur det blir med olika tecken (positivt resp. negativt) på divisor och dividend.



\Task Läs om identifierare i Scala och speciellt \emph{literal identifiers} här: \url{http://www.artima.com/pins1ed/functional-objects.html#6.10}.

\Subtask Förklara vad som händer nedan:
\begin{REPLnonum}
scala> val `konstig val` = 42
scala> println(`konstig val`)
\end{REPLnonum}

\Subtask Scala och Java har olika uppsättningar med reserverade ord. På vilket sätt kan ''backticks'' vara använbart med anledning av detta?


\Task Sök upp dokumentationen för \code{java.lang.Integer}.

\Subtask Undersök i REPL hur metoderna \code{toBinaryString} och \code{toHexString} fungerar.

\Subtask Vad betyder literalen \code{0x2a}?

\Task Typannoteringar skapas genom att i ett uttryck placera ett kolon följt av en typ, vid behov  omslutet av en parentes. Skapa ett större uttryck med typannoteringar och försök få kompilatorn att kontrollera typen på intressanta ställen. Märk att typannoteringar också ibland kan användas för att konvertera mellan numeriska typer.


\Task Förklara vad som händer nedan:
\begin{REPL}
scala> var i = 42
scala> i += 1
scala> i *= 2
scala> i /= 3
\end{REPL}


\Task Läs om BigInt och BigDecimal här: \href{http://alvinalexander.com/scala/how-to-use-large-integer-decimal-numbers-in-scala-bigint-bigdecimal}{alvinalexander.com/scala/how-to-use-large-integer-decimal-numbers-in-scala-bigint-bigdecimal} och prova att skapa riktigt stora tal med hjälp av metoden \code{pow} på BigInt och tal med riktigt många decimaler med BigDecimal dess metod \code{pow}.

\Task Sök upp dokumentationtionen för \code{java.lang.Math.multiplyExact} och läs om vad den metoden gör.

\Subtask Vad händer här?
\begin{REPLnonum}
scala> Math.multiplyExact(2, 42)
scala> Math.multiplyExact(Int.MaxValue, Int.MaxValue)
\end{REPLnonum}

\Subtask\Pen Varför kan man vilja använda \code{java.lang.Math.multiplyExact} i stället för ''vanlig'' multiplikation?



\Subtask\Pen Sök med Ctrl+F i webbläsaren och efter förekomster av texten \textit{''overflow''} i javadoc för klassen \code{java.lang.Math} i JDK 8. Vad är ''overflow''? Vilka metoder finns i \code{java.lang.Math} som hjälper dig att upptäcka om det blir overflow?

\Task Använda Scala REPL för att undersöka konstanterna nedan. Vilket av dessa värden är negativt? Vad kan man ha för praktisk nytta av dessa värden i ett program som gör flyttalsberäkningar?

\Subtask \code{java.lang.Double.MIN_VALUE}

\Subtask \code{scala.Double.MinValue}

\Subtask \code{scala.Double.MinPositiveValue}

\Task För typerna \code{Byte}, \code{Short}, \code{Char}, \code{Int}, \code{Long}, \code{Float}, \code{Double}: Undersök hur många bitar som behövs för att representera varje typs omfång? \\*
\textit{Tips:} Några användbara uttryck: \\*
 \code{Integer.toBinaryString(Int.MaxValue + 1).size} \\*
 \code{Integer.toBinaryString((math.pow(2,16) - 1).toInt).size} \\*
 \code{1 + math.log(Long.MaxValue)/math.log(2)}
Se även språkspecifikationen för Scala, kapitlet om heltalsliteraler: \\
\url{http://www.scala-lang.org/files/archive/spec/2.11/01-lexical-syntax.html#integer-literals}

\Subtask Undersök källkoden för paketobjektet \code{scala.math} här: \\
\url{https://github.com/scala/scala/blob/v2.11.7/src/library/scala/math/package.scala} \\
Hur många olika överlagrade varianter av funktionen \code{abs} finns det och för vilka parametertyper är den definierad?

\Task Läs mer om stränginterpolatorer här:\\ \href{http://docs.scala-lang.org/overviews/core/string-interpolation.html}{docs.scala-lang.org/overviews/core/string-interpolation.html} \\ Hur kan du använda \code{f}-interpolatorn för att göra följande utskrift i REPL? Byt ut \code{???} mot lämpliga tecken.
\begin{REPLnonum}
scala> val g: Double = 1 / 3.0
scala> val s: String = f"Gurkan är ??? meter lång"
scala> println(s)
Gurkan är 0.333 meter lång
\end{REPLnonum}

\fi %%% TODO fix solutions





%!TEX encoding = UTF-8 Unicode
%!TEX root = ../exercises.tex

\ifPreSolution

\Exercise{\ExeWeekTWO}\label{exe:W02}
\begin{Goals}
%!TEX encoding = UTF-8 Unicode
%!TEX root = ../exercises.tex

\item Kunna skapa, kompilera och köra en enkel applikation i terminalen.
\item Kunna skapa samlingarna Range, Array och Vector med heltal och strängar.
\item Kunna indexera i en indexerbar samling, t.ex. Array och Vector.
\item Kunna anropa operationerna size, mkString, sum, min, max på samlingar som innehåller heltal.
\item Känna till skillnader och likheter mellan samlingarna Range, Array och Vector.
\item Förstå skillnaden mellan en while-sats och ett for-uttryck.
\item Kunna skapa samlingar med heltalsvärden som resultat av enkla for-uttryck.
\item Förstå skillnaden mellan en algoritm i pseudo-kod och dess implementation.
\item Kunna implementera algoritmerna SUM, MIN, MAX med en indexerbar samling och en while-sats.

\end{Goals}

\begin{Preparations}
\item \StudyTheory{02}
\item Bekanta dig med grundläggande terminalkommandon, se appendix~\ref{appendix:terminal}.
\item Bekanta dig med den editor du vill använda, se appendix~\ref{appendix:compile}.
\end{Preparations}

\else

\ExerciseSolution{\ExeWeekTWO}

\fi


% terminalkommando
% scalac -> hello world; scala som script; javac
% paket, import, jar, main,


\BasicTasksNoLab %%%%%%%%%%%%%%%%




\WHAT{Para ihop begrepp med beskrivning.}

\QUESTBEGIN

\Task \what

\vspace{1em}\noindent Koppla varje begrepp med den (förenklade) beskrivning som passar bäst: 

\begin{ConceptConnections}
  kompilerad & 1 & & A & där exekveringen av kompilerad app startar \\ 
  skript & 2 & & B & en samling som representerar ett intervall av heltal \\ 
  objekt & 3 & & C & maskinkod sparad och kan köras igen utan kompilering \\ 
  main & 4 & & D & en oföränderlig, indexerbar sekvenssamling \\ 
  programargument & 5 & & E & applicerar en funktion på varje element i en samling \\ 
  datastruktur & 6 & & F & stegvis beskrivning av en lösning på ett problem \\ 
  samling & 7 & & G & maskinkod sparas ej utan skapas vid varje körning \\ 
  sekvenssamling & 8 & & H & samlar variabler och funktioner \\ 
  Array & 9 & & I & överförs via parametern args i main \\ 
  Vector & 10 & & J & en specifik realisering av en algoritm \\ 
  Range & 11 & & K & används i for-uttryck för att skapa ny samling \\ 
  yield & 12 & & L & en förändringsbar, indexerbar sekvenssamling \\ 
  map & 13 & & M & datastruktur med element av samma typ \\ 
  algoritm & 14 & & N & många olika element i en helhet; elementvis åtkomst \\ 
  implementation & 15 & & O & datastruktur med element i en viss ordning \\ 
\end{ConceptConnections}

\SOLUTION

\TaskSolved \what

\begin{ConceptConnections}
  kompilerad & 1 & ~~\Large$\leadsto$~~ &  C & maskinkod sparad och kan köras igen utan kompilering \\ 
  skript & 2 & ~~\Large$\leadsto$~~ &  G & maskinkod sparas ej utan skapas vid varje körning \\ 
  objekt & 3 & ~~\Large$\leadsto$~~ &  H & samlar variabler och funktioner \\ 
  main & 4 & ~~\Large$\leadsto$~~ &  A & där exekveringen av kompilerad app startar \\ 
  programargument & 5 & ~~\Large$\leadsto$~~ &  I & överförs via parametern args i main \\ 
  datastruktur & 6 & ~~\Large$\leadsto$~~ &  N & många olika element i en helhet; elementvis åtkomst \\ 
  samling & 7 & ~~\Large$\leadsto$~~ &  M & datastruktur med element av samma typ \\ 
  sekvenssamling & 8 & ~~\Large$\leadsto$~~ &  O & datastruktur med element i en viss ordning \\ 
  Array & 9 & ~~\Large$\leadsto$~~ &  L & en förändringsbar, indexerbar sekvenssamling \\ 
  Vector & 10 & ~~\Large$\leadsto$~~ &  D & en oföränderlig, indexerbar sekvenssamling \\ 
  Range & 11 & ~~\Large$\leadsto$~~ &  B & en samling som representerar ett intervall av heltal \\ 
  yield & 12 & ~~\Large$\leadsto$~~ &  K & används i for-uttryck för att skapa ny samling \\ 
  map & 13 & ~~\Large$\leadsto$~~ &  E & applicerar en funktion på varje element i en samling \\ 
  algoritm & 14 & ~~\Large$\leadsto$~~ &  F & stegvis beskrivning av en lösning på ett problem \\ 
  implementation & 15 & ~~\Large$\leadsto$~~ &  J & en specifik realisering av en algoritm \\ 
\end{ConceptConnections}

\QUESTEND






%%%%%%%%%%%%%%%%%%% SKA FIXAS:




\WHAT{Datastrukturen \code+Range+.}

\QUESTBEGIN

\Task  \what~Evaluera nedan uttryck i Scala REPL. Vad har respektive uttryck för värde och typ?

\Subtask \code{Range(1, 10)}

\Subtask \code{Range(1, 10).inclusive}

\Subtask \code{Range(0, 50, 5)}

\Subtask \code{Range(0, 50, 5).size}

\Subtask \code{Range(0, 50, 5).inclusive}

\Subtask \code{Range(0, 50, 5).inclusive.size}

\Subtask \code{0.until(10)}

\Subtask \code{0 until (10)}

\Subtask \code{0 until 10}

\Subtask \code{0.to(10)}

\Subtask \code{0 to 10}

\Subtask \code{0.until(50).by(5)}

\Subtask \code{0 to 50 by 5}

\Subtask \code{(0 to 50 by 5).size}

\Subtask \code{(1 to 1000).sum}


\SOLUTION


\TaskSolved \what
 

\SubtaskSolved  värde: \code{Range(1,2,3,4,5,6,7,8,9)}

typ: \code{scala.collection.immutable.Range}

\SubtaskSolved  värde: \code{Range(1,2,3,4,5,6,7,8,9,10)}

typ: \code{scala.collection.immutable.Range}

\SubtaskSolved  värde: \code{Range(0,5,10,15,20,25,30,35,40,45)}

 typ: \code{scala.collection.immutable.Range}

\SubtaskSolved  värde: \code{10}, typ: \code{Int}

\SubtaskSolved  värde: \code{Range(0,5,10,15,20,25,30,35,40,45,50)}

typ: \code{scala.collection.immutable.Range}

\SubtaskSolved  värde: \code{11}, typ: \code{Int}

\SubtaskSolved  värde: \code{Range(0,1,2,3,4,5,6,7,8,9)}

typ: \code{scala.collection.immutable.Range}

\SubtaskSolved  värde: \code{Range(0,1,2,3,4,5,6,7,8,9)}

typ: \code{scala.collection.immutable.Range}

\SubtaskSolved  värde: \code{Range(0,1,2,3,4,5,6,7,8,9)}

typ: \code{scala.collection.immutable.Range}

\SubtaskSolved  värde: \code{Range(0,1,2,3,4,5,6,7,8,9,10)}

typ: \code{scala.collection.immutable.Range.Inclusive}

\SubtaskSolved  värde: \code{Range(0,1,2,3,4,5,6,7,8,9,10)}

typ: \code{scala.collection.immutable.Range.Inclusive}

\SubtaskSolved  värde: \code{Range(0,5,10,15,20,25,30,35,40,45)}

typ: \code{scala.collection.immutable.Range}

\SubtaskSolved  värde: \code{Range(0,5,10,15,20,25,30,35,40,45,50)}

typ: \code{scala.collection.immutable.Range}

\SubtaskSolved  värde: \code{11}, typ: \code{Int}

\SubtaskSolved  värde: \code{500500}, typ: \code{Int}




\QUESTEND




%%<AUTOEXTRACTED by mergesolu>%%      %Uppgift 2




\WHAT{Datastrukturen \code+Array+.}

\QUESTBEGIN

\Task \label{task:array} \what~   Kör nedan kodrader i Scala REPL. Beskriv vad som händer.

\Subtask \code{val xs = Array("hej","på","dej", "!")}

\Subtask \code{xs(0)}

\Subtask \code{xs(3)}

\Subtask \code{xs(4)}

\Subtask \code{xs(1) + " " + xs(2)}

\Subtask \code{xs.mkString}

\Subtask \code{xs.mkString(" ")}

\Subtask \code{xs.mkString("(", ",", ")")}

\Subtask \code{xs.mkString("Array(", ", ", ")")}

\Subtask \code{xs(0) = 42}

\Subtask \code{xs(0) = "42"; println(xs(0))}

\Subtask \code{val ys = Array(42, 7, 3, 8)}

\Subtask \code{ys.sum}

\Subtask \code{ys.min}

\Subtask \code{ys.max}

\Subtask \code{val zs = Array.fill(10)(42)}

\Subtask \code{zs.sum}

\Subtask\Pen Datastrukturen \code{Range} håller reda på start- och slutvärde, samt stegstorleken för en uppräkning, men alla talen i uppräkningen genereras inte förrän så behövs. En \code{Int} tar 4 bytes i minnet. Ungefär hur mycket plats i minnet tar de objekt som variablerna \code{r} respektive \code{a} refererar till nedan?
\begin{REPL}
scala> val r = (1 to Int.MaxValue by 2)
scala> val a = r.toArray
\end{REPL}
\emph{Tips:} Använd uttrycket \code{ BigInt(Int.MaxValue) * 2 } i dina beräkningar.

\SOLUTION


\TaskSolved \what
 

\SubtaskSolved  Ett objekt av typen \code{Array[String]} skapas med värdet 

\code{Array(hej, på, dej, !)} och med namnet \code{xs}.

\SubtaskSolved  Returnerar en sträng med värdet \code{hej}.

\SubtaskSolved  Returnerar en sträng med värdet \code{!}.

\SubtaskSolved  Ett exception genereras. Skriver ut:

\code{java.lang.ArrayIndexOutOfBoundsException: 4}

\SubtaskSolved  Returnerar en sträng med värdet \code{på dej}.

\SubtaskSolved  Returnerar en sträng med värdet \code{hejpådej!}.

\SubtaskSolved  Returnerar en sträng med värdet \code{hej på dej !}.

\SubtaskSolved  Returnerar en sträng med värdet \code{(hej,på,dej,!)}.

\SubtaskSolved  Returnerar en sträng med värdet \code{Array(hej,på,dej,!)}.

\SubtaskSolved  Ett fel uppstår av typen \code{type mismatch}. Konsollen talar om för oss vad den fick, dvs värdet \code{42} av typen \code{Int}. Den talar även om för oss vad den ville ha, dvs något värde av typen \code{String}. Till sist skriver den ut vår kodrad och pekar ut felet.

\SubtaskSolved  Det första elementet i \code{xs} ändras till värdet \code{42}. Därefter skrivs det första värdet i \code{xs} ut.

\SubtaskSolved  Ett objekt av typen \code{Array[Int]} skapas med värdet \code{Array(42, 7, 3, 8)} och med namnet \code{ys}.

\SubtaskSolved  Returnerar summan av elementen i \code{ys}. Resultatet är \code{60}.

\SubtaskSolved  Returnerar det minsta värdet i \code{ys}. Resultatet är \code{3}.

\SubtaskSolved  Returnerar det största värdet i \code{ys}. Resultatet är \code{42}.

\SubtaskSolved  Ett nytt värde av typen \code{Array[Int]} skapas med \code{10} stycken element, alla med värdet \code{42}.

\SubtaskSolved  Returnerar summan av elementen i \code{zs}. Resultatet blir 420 (42 multiplicerat med 10).

\SubtaskSolved  \code{r} tar upp 12 bytes. \code{a} tar upp ca 4 miljarder bytes.



\QUESTEND




%%<AUTOEXTRACTED by mergesolu>%%      %Uppgift 3




\WHAT{Datastrukturen \code+Vector+.}

\QUESTBEGIN

\Task  \what~  Kör nedan kodrader i Scala REPL. Beskriv vad som händer.

\Subtask \code{val words = Vector("hej","på","dej", "!")}

\Subtask \code{words(0)}

\Subtask \code{words(3)}

\Subtask \code{words.mkString}

\Subtask \code{words.mkString(" ")}

\Subtask \code{words.mkString("(", ",", ")")}

\Subtask \code{words.mkString("Ord(", ", ", ")")}

\Subtask \code{words(0) = "42"}

\Subtask \code{val numbers = Vector(42, 7, 3, 8)}

\Subtask \code{numbers.sum}

\Subtask \code{numbers.min}

\Subtask \code{numbers.max}

\Subtask \code{val moreNumbers = Vector.fill(10000)(42)}

\Subtask \code{moreNumbers.sum}

\Subtask\Pen Jämför med uppgift \ref{task:array}. Vad kan man göra med en \code{Array} som man inte kan göra med en \code{Vector}?

\SOLUTION


\TaskSolved \what
 

\SubtaskSolved  Ett objekt av typen \code{scala.collection.immutable.Vector[String]} initieras med värdet \code{Vector(hej, på dej, !)}.

\SubtaskSolved  Returnerar det nollte elementet i \code{words}, dvs strängen \code{hej}.

\SubtaskSolved  Returnerar det tredje elementet i \code{words}, dvs strängen \code{!}.

\SubtaskSolved  Omvandlar vektorn till en Sträng.

\SubtaskSolved  Samma som ovan, fast den här gången används mellanrum för att seperera elementen.

\SubtaskSolved  Samma som ovan, fast den här gången sepereras elementen av kommatecken istället för mellanrum och dessutom börjar och slutar den resulterande strängen med parenteser.

\SubtaskSolved  Samma som ovan, fast med ordet \code{Ord} tillagt i början av den resulterande strängen.

\SubtaskSolved  Ett fel uppstår. Typen \code{Vector} är immutable. Dess element kan alltså inte bytas ut.

\SubtaskSolved  En ny \code{Vector[Int]} skapas med värdet \code{Vector(42, 7, 3, 8)}. 

\SubtaskSolved  Returnerar summan av vektorn \code{numbers}.

\SubtaskSolved  Returnerar vektorns minsta element.

\SubtaskSolved  Returnerar vektorns största element. 

\SubtaskSolved  En ny vektor skapas innehållandes tiotusen 42or.

\SubtaskSolved  Returnerar summan av vektorns element.

\SubtaskSolved  Byta ut element.



\QUESTEND




%%<AUTOEXTRACTED by mergesolu>%%      %Uppgift 4




\WHAT{\code+for+-uttryck}

\QUESTBEGIN

\Task  \what~ . Evaluera nedan uttryck i Scala REPL. Vad har respektive uttryck för värde och typ?

\Subtask \code{for (i <- Range(1,10)) yield i}

\Subtask \code{for (i <- 1 until 10) yield i}

\Subtask \code{for (i <- 1 until 10) yield i + 1}

\Subtask \code{for (i <- Range(1,10).inclusive) yield i}

\Subtask \code{for (i <- 1 to 10) yield i}

\Subtask \code{for (i <- 1 to 10) yield i + 1}

\Subtask \code{(for (i <- 1 to 10) yield i + 1).sum}

\Subtask \code{for (x <- 0.0 to 2 * math.Pi by math.Pi/4) yield math.sin(x)}


\SOLUTION


\TaskSolved \what
 

\SubtaskSolved  typ: \code{scala.collection.immutable.IndexedSeq[Int]}

värde: \code{Vector(1, 2, 3, 4, 5, 6, 7, 8, 9)}

\SubtaskSolved  typ: \code{scala.collection.immutable.IndexedSeq[Int]}

värde: \code{Vector(1, 2, 3, 4, 5, 6, 7, 8, 9)}

\SubtaskSolved  typ: \code{scala.collection.immutable.IndexedSeq[Int]}

värde: \code{Vector(2, 3, 4, 5, 6, 7, 8, 9, 10)}

\SubtaskSolved  typ: \code{scala.collection.immutable.IndexedSeq[Int]}

värde: \code{Vector(1, 2, 3, 4, 5, 6, 7, 8, 9, 10)}

\SubtaskSolved  typ: \code{scala.collection.immutable.IndexedSeq[Int]}

värde: \code{Vector(1, 2, 3, 4, 5, 6, 7, 8, 9, 10)}

\SubtaskSolved  typ: \code{scala.collection.immutable.IndexedSeq[Int]}

värde: \code{Vector(2, 3, 4, 5, 6, 7, 8, 9, 10, 11)}

\SubtaskSolved  typ: \code{Int}, värde: \code{Vector(65)}

\SubtaskSolved  typ: \code{scala.collection.immutable.IndexedSeq[Int]}

värde: \code{Vector(0.0, 0.707, 1.0, 0.707, 0.0, -0.707, -1.0, -0.707)}



\QUESTEND




%%<AUTOEXTRACTED by mergesolu>%%      %Uppgift 5




\WHAT{Metoden \code+map+ på en samling.}

\QUESTBEGIN

\Task  \what~  Evaluera nedan uttryck i Scala REPL. Vad har respektive uttryck för värde och typ?

\Subtask \code{Range(0,10).map(i => i + 1)}

\Subtask \code{(0 until 10).map(i => i + 1)}

\Subtask \code{(1 to 10).map(i => i * 2)}

\Subtask \code{(1 to 10).map(_ * 2)}

\Subtask \code{Vector.fill(10000)(42).map(_ + 43)}

\SOLUTION


\TaskSolved \what
 

\SubtaskSolved  typ: \code{scala.collection.immutable.IndexedSeq[Int]}

värde: \code{Vector(1, 2, 3, 4, 5, 6, 7, 8, 9, 10)}

\SubtaskSolved  typ: \code{scala.collection.immutable.IndexedSeq[Int]}

värde: \code{Vector(1, 2, 3, 4, 5, 6, 7, 8, 9, 10)}

\SubtaskSolved  typ: \code{scala.collection.immutable.IndexedSeq[Int]}

värde: \code{Vector(2, 4, 6, 8, 10, 12, 14, 16, 18, 20)}

\SubtaskSolved  typ: \code{scala.collection.immutable.IndexedSeq[Int]}

värde: \code{Vector(2, 4, 6, 8, 10, 12, 14, 16, 18, 20)}

\SubtaskSolved  typ: \code{scala.collection.immutable.Vector[Int]}

värde: En vector av tiotusen 85or (85 = 42 + 43).



\QUESTEND




%%<AUTOEXTRACTED by mergesolu>%%      %Uppgift 6




\WHAT{Metoden \code+foreach+ på en samling.}

\QUESTBEGIN

\Task  \what~  Kör nedan satser i Scala REPL. Vad händer?

\Subtask \code{Range(0,10).foreach(i => println(i))}

\Subtask \code{(0 until 10).foreach(i => println(i))}

\Subtask \code|(1 to 10).foreach{i => print("hej"); println(i * 2)}|

\Subtask \code{(1 to 10).foreach(println)}

\Subtask \code{Vector.fill(10000)(math.random).foreach(r => }\\
           \code{      if (r > 0.99) print("pling!"))}


\SOLUTION


\TaskSolved \what
 

\SubtaskSolved  En \code{Range} skapas och dess element skrivs ut ett och ett.

\SubtaskSolved  Samma sak händer.

\SubtaskSolved  De tio första jämna talen (noll ej inräknat) skrivs ut med ett "hej" framför.

\SubtaskSolved  Talen 1 till 10 skrivs ut.

\SubtaskSolved  Tiotusen slumptal mellan 0 och 1 genereras. Varje gång ett tal är större än 0.99 kommer det ett pling.



\QUESTEND




%%<AUTOEXTRACTED by mergesolu>%%      %Uppgift 7




\WHAT{Algoritm: SWAP.}

\QUESTBEGIN

\Task  \what~ 

\Subtask Skriv med \emph{pseudo-kod} algoritmen SWAP. Beskriv på vanlig svenska, steg för steg, hur en variabel $temp$ används för mellanlagring vid värdebytet:

\emph{Indata:} två heltalsvariabler $x$ och $y$

\emph{???}

\emph{Utdata:} variablerna $x$ och $y$ vars värden har bytt plats.

\Subtask Implementerar algoritmen SWAP. Ersätt \code{???} nedan med satser separerade av semikolon:

\begin{REPL}
scala> var (x, y) = (42, 43)
scala> ???
scala> println("x är " + x + ", y är " + y)
x är 43, y är 42
\end{REPL}



\SOLUTION


\TaskSolved \what
 

\SubtaskSolved  Pseudokoden kan se ut såhär:

Skapa heltalsvariabel temp. 
Flytta värdet från x till temp. 
Flytta värdet från y till x. 
Flytta värdet från temp till y.

\SubtaskSolved 
\begin{REPLnonum}
scala> var (x, y) = (42, 43)
x: Int = 42
y: Int = 43
scala> var temp = x; x = y; y = temp;
temp: Int = 42
x: Int = 43
y: Int = 42
scala> println("x är " + x + ", y är " + y)
x är 43, y är 42
\end{REPLnonum}



\QUESTEND




%%<AUTOEXTRACTED by mergesolu>%%      %Uppgift 8




\WHAT{Skript.}

\QUESTBEGIN

\Task  \what~  Skapa en fil med namn \texttt{hello-script.scala} med hjälp av en editor som innehåller denna enda rad:
\begin{Code}
println("hej skript")
\end{Code}
Spara filen och kör kommandot \code{scala hello-script.scala} i terminalen:
\begin{REPLnonum}
> scala hello-script.scala
\end{REPLnonum}

\Subtask Vad händer?

\Subtask Ändra i filen så att högerparentesen saknas. Spara och kör skriptfilen igen. Vad händer?

\Subtask Lägg till en sats sist i skriptet som skriver ut summan av de ett tusen stycken heltalen från och med 2 till och med 1001, så som visas nedan.
\begin{REPL}
> scala hello-script.scala
hej skript
501500
\end{REPL}

\Subtask Ändra i hello-script.scala genom att införa \code{val n = args(0).toInt} och använd \code{n} som övre gräns för summeringen av de n första heltalen.
\begin{REPL}
> scala hello-script.scala 5001
hej skript
12507501
\end{REPL}

\Subtask Vad blir det för felmeddelande om du glömmer ge programmet ett argument?


\SOLUTION


\TaskSolved \what
 

\SubtaskSolved  Skriver ut "hej skript".

\SubtaskSolved  Ett felmeddelande skrivs ut.

\SubtaskSolved  Lägg till raden:
\code{println((2 to 1001).sum)} 
eller motsvarande.

\SubtaskSolved  Filen ska se ut ungefär såhär: \\
\begin{Code} 
val n = args(0).toInt 
println("hej skript") 
println((1 to n).sum)
\end{Code}

\SubtaskSolved  \code{java.lang.ArrayIndexOutOfBoundsException: 0}



\QUESTEND




%%<AUTOEXTRACTED by mergesolu>%%      %Uppgift 9




\WHAT{Applikation med \code+main+-metod.}

\QUESTBEGIN

\Task  \what~  Skapa med hjälp av en editor en fil med namn \texttt{hello-app.scala}.
\begin{REPLnonum}
> gedit hello-app.scala
\end{REPLnonum}
Skriv dessa rader i filen:


\scalainputlisting{examples/hello-app.scala}

\Subtask Kompilera med \code{scalac hello-app.scala} och kör koden med \code{scala Hello}.
\begin{REPLnonum}
> scalac hello-app.scala
> ls
> scala Hello
\end{REPLnonum}
Vad heter filerna som kompilatorn skapar?

\Subtask Ändra i din kod så att kompilatorn ger följande felmeddelande: \\
\texttt{Missing closing brace}

\Subtask\Pen Varför behövs \code{main}-metoden?

\Subtask\Pen Vilket alternativ går snabbast att köra igång, ett skript eller en kompilerad applikation? Varför? Vilket alternativ kör snabbast när väl exekveringen är igång?


\SOLUTION


\TaskSolved \what
 

\SubtaskSolved  Hello.class och Hello\$.class

\SubtaskSolved  Ta bort en av krullparenteserna i slutet.

\SubtaskSolved  I ett skript behöver man inte skriva någon main-metod. Kompilatorn lägger till en automatiskt precis när koden ska köras. I en applikation behöver man däremot det. För att göra en applikation definierar vi ett objekt som vi i det här fallet kallar för \code{Hello}. Från början gör inte objekt någonting. De bara finns. För att objekt ska kunna göra något behövs det metoder. I vanliga fall utförs inte metoder förrän en annan metod "ropar" på metoden. main-metoden ropas dock automatiskt när en applikation startas. Annars hade ju ingenting hänt, eftersom alla metoderna väntar på att någon annan metod ska börja. \\
\SubtaskSolved  Första gången man ska köra en applikation måste den först kompileras innan den exekveras. Skript kompileras automatiskt samtidigt som de exekveras, vilket totalt sett görs på kortare tid. Därför tar det längre tid att starta en applikation första gången än att starta ett skript första gånge. När en applikation väl har kompileras och kan exekveras, går det dock mycket fortare. Fördelen med applikationer är att de kan exekveras flera gånger utan att kompileras om.



\QUESTEND




%%<AUTOEXTRACTED by mergesolu>%%      %Uppgift 10




\WHAT{Java-applikation.}

\QUESTBEGIN

\Task \label{task:java} \what~   Skapa med hjälp av en editor en fil med namn \texttt{Hi.java}.
\begin{REPLnonum}
> gedit Hi.java
\end{REPLnonum}
Skriv dessa rader i filen:

\javainputlisting{examples/Hi.java}

\noindent Kompilera med \code{javac Hi.java} och kör koden med \code{java Hi}.
\begin{REPLnonum}
> javac Hi.java
> ls
> java Hi
\end{REPLnonum}

\Subtask\Pen Vad heter filen som kompilatorn skapat?

\Subtask\Pen Jämför signaturen för Java-programmets main-metod med signaturen för Scala-programmets main-metod. De betyder samma sak men syntaxen är olika. Beskriv skillnader och likheter i syntaxen.

\Subtask\Pen Vad blir det för felmeddelande om källkodsfilen och klassnamnet inte överensstämmer i ett Java-program?


\SOLUTION


\TaskSolved \what
 

\SubtaskSolved  Hi.class

\SubtaskSolved  I javas syntax börjar man med orden \code{public static}. I scala uteblir dessa. I scala är alla metoder automatiskt publika om inget annat används. Därför behövs aldrig ordet \code{public} i scala. I scala finns det tekniskt sett inga statiska metoder. Men i praktiken fungerar vanliga metoder i ett scala-objekt på ungefär samma sätt som statiska metoder i en java-klass. I scala används ordet \code{def} varje gång en funktion ska definieras. I java slipper man det. I java skriver man returtypen (\code{void}) innan parametrarna. I scala kommer istället metodens returtyp (\code{Unit}) i slutet. Javas \code{void} motsvarar scalas \code{Unit}. I scalas syntax kommer parameterns namn (\code{args}) före parameterns typ (\code{Array[String]}), separerat med ett kolon. I java kommer typen (\code{String[]}) först och sen kommer namnet (\code{args}). \code{String[]} i java betyder ungefär samma sak som \code{Array[String]} i scala.

\SubtaskSolved  -



\QUESTEND




%%<AUTOEXTRACTED by mergesolu>%%      %Uppgift 11




\WHAT{Algoritm: SUMBUG}

\QUESTBEGIN

\Task  \what~ . Nedan återfinns pseudo-koden för SUMBUG.

\begin{algorithm}[H]
 \SetKwInOut{Input}{Indata}\SetKwInOut{Output}{Resultat}

 \Input{heltalet $n$}
 \Output{utskrift av summan av de första $n$ heltalen }
 $sum \leftarrow 0$ \\
 $i \leftarrow 1$  \\
 \While{$i \leq n$}{
  $sum \leftarrow sum + 1$
 }
 skriv ut $sum$
\end{algorithm}

\Subtask\Pen Kör algoritmen steg för steg med penna och papper, där du skriver upp hur värdena för respektive variabel ändras. Det finns två buggar i algoritmen. Vilka? Rätta buggarna och test igen genom att ''köra'' algoritmen med penna på papper och kontrollera så att algoritmen fungerar för $n=0$, $n=1$, och $n=5$. Vad händer om $n=-1$?

\Subtask Skapa med hjälp av en editor filen \code{sumn.scala}. Implementera algoritmen SUM enligt den rättade pseudokoden och placera implementationen i en main-metod i ett objekt med namnet \code{sumn}. Du kan skapa indata \code{n} till algoritmen med denna deklaration i början av din main-metod: \\ \code{val n = args(0).toInt} \\ Vad ger applikationen för utskrift om du kör den med argumentet 8888?

\begin{REPLnonum}
> scalac sumn.scala
> scala sumn 8888
\end{REPLnonum}

\Subtask Kontrollera att din implementation räknar rätt genom att jämföra svaret med detta uttrycks värde, evaluerat i Scala REPL:
\begin{REPLnonum}
scala> (1 to 8888).sum
\end{REPLnonum}

\Subtask Implementera algoritmen SUM enligt pseudokoden ovan, men nu i Java. Skapa filen \code{SumN.java} och använd koden från uppgift \ref{task:java} som mall för att deklarera den publika klassen \code{SumN} med en main-metod. Några tips om Java-syntax och standarfunktioner i Java:

\begin{itemize}[noitemsep, nolistsep]
\item Alla satser i Java måste avslutas med semikolon.
\item Heltalsvariabler deklareras med nyckelordet \lstinline[language=Java]{int} (litet i).
\item Typnamnet ska stå \emph{före} namnet på variabeln. Exempel: \\ \lstinline[language=Java]{int sum = 0;}
\item Indexering i en array görs i Java med hakparenteser: \code{args[0]}
\item I stället för Scala-uttrycket \code{args(0).toInt}, använd Java-uttrycket: \\ \code{Integer.parseInt(args[0])}
\item \code{while}-satser i Scala och Java har samma syntax.
\item Utskrift i Java görs med \code{System.out.println}
\end{itemize}


\SOLUTION


\TaskSolved \what
 

\SubtaskSolved  Bugg: Eftersom \code{i} inte ökar, fastnar programmet i en oändlig loop. Fix: Lägg till en sats i slutet av while-blocket som ökar värdet på i med 1.
Bugg: Eftersom man bara ökar summan med 1 varje gång, kommer resultatet att bli summan av n stycken 1or, inte de n första heltalen. Fix: Ändra så att summan ökar med \code{i} varje gång, istället för 1.
För -1, blir resultatet 0. Förklaring: i börjar på 1 och är alltså aldrig mindre än n som ju är -1. while-blocket genomförs alltså noll gånger, och efter att \code{sum} får sitt ursprungsvärde förändras den aldrig.
\SubtaskSolved  39502716
\SubtaskSolved  -
\SubtaskSolved  Såhär kan implementationen se ut:
\begin{Code}
public class SumN {
  public static void main(String[] args) {
    int n = Integer.parseInt(args[0]);
    int sum = 0;
    int i = 1;
    while(i <= n){
      sum = sum + i;
      i = i + 1;
      }
    }
    System.out.println(sum);
}
\end{Code}



\QUESTEND




%%<AUTOEXTRACTED by mergesolu>%%      %Uppgift 12




\WHAT{Algoritm: MAXBUG}

\QUESTBEGIN

\Task  \what~ . Nedan återfinns pseudo-koden för MAXBUG.

\begin{algorithm}[H]
 \SetKwInOut{Input}{Indata}\SetKwInOut{Output}{Resultat}

 \Input{Array $args$ med strängar som alla innehåller heltal}
 \Output{utskrift av största heltalet }
 $max \leftarrow$ det minsta heltalet som kan uppkomma  \\
 $n \leftarrow $ antalet heltal \\
 $i \leftarrow 0$ \\
 \While{$i < n$}{
   $x \leftarrow args(i).toInt$ \\
   \If{( x > $max$)}{$max \leftarrow x$}
  % $i \leftarrow i + 1$
 }
 skriv ut $max$
\end{algorithm}

\Subtask\Pen Kör med penna och papper. Det finns en bugg i algoritmen ovan. Vilken? Rätta buggen.

\Subtask Implementera algoritmen MAX (utan bugg) som en Scala-applikation. Tips:
\begin{itemize}[noitemsep, nolistsep]
\item Det minsta \code{Int}-värdet som någonsin kan uppkomma: \code{Int.MinValue}
\item Antalet element i $args$ ges av: \code{args.size}
\end{itemize}

\begin{REPL}
> gedit maxn.scala
> scalac maxn.scala
> scala maxn 7 42 1 -5 9
42
\end{REPL}

\Subtask\Pen \label{subtask:arg0} Skriv om algoritmen så att variabeln $max$ initialiseras med det första talet i sekvensen.

\Subtask Implementera den nya algoritmvarianten från uppgift \ref{subtask:arg0} och prova programmet. Vad händer om $args$ är tom?

\SOLUTION


\TaskSolved \what
 

\SubtaskSolved  Bugg: i ökar aldrig. Programmet fastnar i en oändlig loop. Fix: Lägg till en sats som ökar i med 1, i slutet av while-blocket.

\SubtaskSolved  Så här kan implementationen se ut:
\begin{Code}
object Max {
  def main(args: Array[String]): Unit = {
    var max = Int.MinValue
    val n = args.size
    var i = 0
    while(i < n) {
      val x = args(i).toInt
      if(x > max) {
        max = x
      }
      i = i + 1
    }
    println(max)
  }
}
\end{Code}
\SubtaskSolved  Raden där max initieras ändras till \code{var max = args(0).toInt} 

\SubtaskSolved  \code{java.lang.ArrayIndexOutOfBoundsException: 0}



\QUESTEND




%%<AUTOEXTRACTED by mergesolu>%%      %Uppgift 13




\WHAT{Block, namnsynlighet, namnöverskuggning}

\QUESTBEGIN

\Task  \what~ . Kör nedan kod i Scala REPL eller i Kojo. Vad händer nedan? Varför?

\Subtask \code|val a = {1 + 1; 2 + 2; 3 + 3; 4 + 4}; println(a)|

\Subtask \code|val b = {1; 2; 3; {val b = 4; b + b; b + 1}}; println(b)|

\Subtask \code|{val a = 42; println(a)}|

\Subtask \code|{val a = 42}; println(a)|

\Subtask \code|{val a = 42; {val a = 43; println(a)}; println(a)}|

\Subtask \code|{var a = 42; {a = a + 1}; var a = 43}|

\Subtask \code|{var a = 42; {a = a + b; var b = 43}; println(a)}|

\Subtask \code|{var a = 42; {var b = 43; a = a + b}; println(a)}|

\Subtask \code|{var a = 42; {a = a + b; def b = 43}; println(a)}|

\Subtask \code|{object a{var b=42;object a{var a=43}};println(a.b+a.a.a)}|

\Subtask

\begin{Code}
{
  object a {
    var b = 42
    object a {
      var a = 43
    }
  }
  println(a.b + a.a.a)
}
\end{Code}

\Subtask Vad är fördelen med att namn deklarerade inne i ett block är lokala i stället för globala?


\SOLUTION


\TaskSolved \what


\SubtaskSolved  Skriver ut talet 8. \code{a} får värdet \code{4 + 4} eftersom detta är den sista satsen i blocket. Man får också tre stycken varningar. Detta beror på att det förekommer tre satser i blocket som inte gör någon skillnad.

\SubtaskSolved  Skriver ut talet 5. De tre första satserna i det yttre blocket ignoreras. \code{b} får värdet som returneras av det yttre blocket. Det yttre blocket returnerar värdet som returneras i den sista satsen i blocket, som i sin tur är ett block. I det inre blocket skapas en ny \code{val} som också får namnet \code{b}. Notera att detta alltså inte är samma värde, även om det har samma namn. Den andra satsen räknar summan av \code{b} med sig själv. Eftersom vi nu befinner oss i det block där det andra \code{b}et precis har definieras så är det detta \code{b} som används och summan blir alltså åtta. Detta är dock helt irrelevant eftersom resultatet inte sparas någonstans. I den sista satsen blir resultatet 5 (eftersom \code{b} är fyra och vi adderar ett). Detta resultatet returneras från det innre blocket och vidare ur det yttre blocket.

\SubtaskSolved  Skriver ut talet 42. Blockets satser exekveras i ordning. 

\SubtaskSolved  Skriver inte ut 42. I blocket skapas ett \code{val} med namnet \code{a} och värdet \code{42}. Detta värde finns inte utanför blocket och kommer därför inte att skrivas ut. Om du däremot definierat \code{a} som något annat tidigare så kommer istället det värdet att skrivas ut.

\SubtaskSolved  Skriver först ut \code{43} och sedan \code{42}. Förklaring:

\code{a} initieras med värdet \code{42}. Ett nytt värde som också har namnet \code{a} initieras med värdet \code{43}. Eftersom detta sker innanför ett nytt block, befinner vi oss i ett annat "namespace" och det gör alltså inget att vi använder samma namn. \code{a} skrivs ut. Eftersom vi befinner oss i det inre blocket är det \code{43} som skrivs ut, inte \code{42}. Scala kollar först efter värden som heter \code{a} i det inre "namespacet". Det är först i andra hand som den skulle upptäcka att det finns ett \code{a} i det yttre blocket. Till sist körs den sista satsen i det yttre blocket. Då skrivs \code{a} ut. Eftersom vi nu befinner oss i det yttre blocket, vet inte ens scala om att det andra \code{a}:et existerar. Resultatet av den här utskriften blir alltså \code{42}.

\SubtaskSolved  Ett fel uppstår. Variabeln \code{a} initieras två gånger i samma namespace. Förklaring till felet:

I det yttre blockets första sats initieras variablen \code{a} med värdet \code{42}. I det yttre blockets tredje sats försöker vi definiera en ny variabel med samma namn. I och med att vi befinner oss i samma namespace, krockar namnen.

Förklaring till vad som händer i sats två:

I det inre blocket har vi inte definierat någon variabel \code{a}. Till en början hittar alltså inte scala något sådant. Då letar scala vidare i det namespace som finns utanför det inre blocket och hittar variabeln som vi definierade i det yttre blockets första sats. Denna variabel får sitt värde förändrat.

\SubtaskSolved  Fel. Framåtreferens. Förklaring:

Det är inte tillåtet att referera till variabler som initieras senare i koden.

\SubtaskSolved  Skriver ut \code{85}. Förklaring:

I och med att vi den här gången initierade variabeln \code{b} och gav den ett värde innan vi använder oss av den, slipper vi problemet ovan.

\SubtaskSolved  Skriver ut \code{85}. Förklaring:

Det är tillåtet att referera till funktioner som definieras senare i koden.

\SubtaskSolved  Skriver ut \code{85}. Förklaring:

\code{a.b} refererar till variabeln \code{b} som ingår i objektet \code{a}.
\code{a.a.a} refererar till variabeln \code{a}, som ingår i ett objekt som heter \code{a} som i sin tur befinner sig i ett annat objekt som också heter \code{a}.

\SubtaskSolved  Skriver ut \code{85}. Förklaring:

Koden är identisk med förra deluppgiften förutom att ny rad används istället för semikolon.

\SubtaskSolved  I stora projekt med mycket kod, kan det vara svårt att hitta unika namn till alla sina variabler. Då är det en fördel om man kan hålla sina variabler i begränsade namespaces, så att de bara är tillgängliga precis när de behöver användas. 



\QUESTEND




%%<AUTOEXTRACTED by mergesolu>%%      %Uppgift 14??? NUMMER I KOMMENTAR STÄMMER EJ MED GENERERAT NUMMER




\WHAT{Paket, \code{import} och klassfilstrukturer.}

\QUESTBEGIN

\Task \label{task:package} \what~   Med Java-8-plattformen kommer 4240 färdiga klasser, som är organiserade i 217 olika paket.\footnote{Se Stackoverflow: \href{http://stackoverflow.com/questions/3112882/how-many-classes-are-there-in-java-standard-edition}{how-many-classes-are-there-in-java-standard-edition}}

\Subtask Vilka paket finns i paketet javax som börjar på s?

\begin{REPLnonum}
scala> javax.s   //tryck på TAB-tangenten
\end{REPLnonum}

\Subtask Kör raderna nedan i REPL. Beskriv vad som händer för varje rad.
\begin{REPL}[numbers=left, numberstyle=\color{black}\ttfamily\scriptsize\selectfont]
scala> import javax.swing.JOptionPane
scala> def msg(s: String) = JOptionPane.showMessageDialog(null, s)
scala> msg("Hej på dej!")
scala> def input(msg: String) = JOptionPane.showInputDialog(null, msg)
scala> input("Vad heter du?")
scala> import JOptionPane.{showOptionDialog => optDlg}
scala> def inputOption(msg: String, opt: Array[Object]) =
         optDlg(null, msg, "Option", 0, 0, null, opt, opt(0))
scala> inputOption("Vad väljer du?", Array("Sten", "Sax", "Påse"))
\end{REPL}

\Subtask\Pen Vad hade du behövt ändra på efterföljande rader om import-satsen på rad 1 ovan ej hade gjorts?

\Subtask Skapa med en editor filen paket.scala och kompilera. Rita en bild av hur katalogstrukturen ser ut.

\begin{Code}
package gurka.tomat.banan

package p1 {
  package p11 {
    object hello {
      def hello = println("Hej paket p1.p11!")
    }
  }
  package p12 {
    object hello {
      def hello = println("Hej paket p1.p12!")
    }
  }
}

package p2 {
  package p21 {
    object hello {
      def hello = println("Hej paket p2.p21!")
    }
  }
}

object Main {
  def main(args: Array[String]): Unit = {
    import p1._
    p11.hello.hello
    p12.hello.hello
    import p2.{p21 => apelsin}
    apelsin.hello.hello
  }
}
\end{Code}

\begin{REPL}
> gedit paket.scala
> scalac paket.scala
> scala gurka.tomat.banan.Main
> ls -R
\end{REPL}

\SOLUTION


\TaskSolved \what
 

\SubtaskSolved  \code{script   security   smartcardio   sound   sql   swing}

\SubtaskSolved  Radernas funktion i ordning:

1. Importerar JOptionPane från javax.swing

2. Definierar en metod som tar en sträng och öppnar en dialogruta med strängen.

3. Testar funktionen med argumentet "Hej på dej!". En dialogruta öppnas med texten "Hej på dej!".

4. Definierar en metod som tar emot en sträng som argument och öppnar en input-dialogruta med strängen.

5. Testar funktionen med argumentet "Vad heter du?". En dialogruta öppnas med texten "Vad heter du?". I ett fält kan man fylla i sitt namn. Funktionen returnerar namnet.

6. Importerar showOptionDialog från JOptionPane under namnet optDlg.

7. Definierar en metod som tar emot en sträng och en Array som argument och öppnar en flervalsdialog. Strängen ska innehålla frågan som flervalsdialogen visar upp. Arrayn ska innehålla alternativen som användaren ska välja mellan.

8.Testar funktionen med argumenten \code{"Vad väljer du?"} och \\ \code{Array("Sten, "Sax", "Påse")}. En dialogruta kommer upp och man får möjlighet att välja sten sax eller påse. Funktionen returnerar valet som man gör.

\SubtaskSolved  På alla ställen där \code{JOptionPane} förekommer, hade man istället fått skriva \code{javax.swing.JOptionPane}.

\SubtaskSolved  -



\QUESTEND




%%<AUTOEXTRACTED by mergesolu>%%      %Uppgift 15




\WHAT{Skapa \code{jar}-filer och använda classpath}

\QUESTBEGIN

\Task  \what~ 

\Subtask Skriv kommandot \code{jar} i terminalen och undersök vad som finns för optioner. Se speciellt ''Example 1.'' i hjälputskriften. Vilket kommando ska du använda för att packa ihop flera filer i en enda jar-fil?

\Subtask Som en fortsättning på uppgift \ref{task:package}, packa ihop biblioteket \code{gurka} i en jar-fil med nedan kommando, samt kör igång REPL med jar-filen på classpath.

\begin{REPL}
> jar cvf mittpaket.jar gurka
> scala -cp mittpaket.jar
scala> gurka.tomat.banan.Main.main(Array())
\end{REPL}


\SOLUTION


\TaskSolved \what
 

\SubtaskSolved  jar cvf [namn på skapad fil] [namn på input-filer]

\SubtaskSolved  -



\QUESTEND




%%<AUTOEXTRACTED by mergesolu>%%      %Uppgift 16




\WHAT{Skapa dokumentation med \code{scaladoc}-kommandot}

\QUESTBEGIN

\Task  \what~ 

\Subtask Som en fortsättning på uppgift \ref{task:package}, kör nedan kommando i terminalen:

\begin{REPL}
> scaladoc paket.scala
> ls
> firefox index.html   # eller öppna index.html i valfri webbläsare
\end{REPL}

Vad händer?

\Subtask Lägg till några fler metoder i något av objekten i filen \code{paket.scala} och lägg även till några dokumentationskommentarer. Kompilera om och kör. Generera om dokumentationen.

\begin{verbatim}
//... ändra i filen paket.scala

/** min paketdokumentationskommentar p2 */
package p2 {
  /** min paketdokumentationskommentar p21 */
  package p21 {
    /** ett hälsningsobjekt */
    object hello {
      /** en hälsningsmetod i p2.p21 */
      def hello = println("Hej paket p2.p21!")

      /** en metod som skriver ut tiden */
      def date = println(new java.util.Date)
    }
  }
}

\end{verbatim}

\begin{REPL}
> gedit paket.scala
> scalac paket.scala
> jar cvf mittpaket.jar gurka
> scala -cp mittpaket.jar
scala> gurka.tomat.banan.p2.p21.hello.date
scala> :q
> scaladoc paket.scala
> firefox index.html
\end{REPL}

\newpage

\ExtraTasks %%%%%%%%%%%%%%%%%%%

\SOLUTION


\TaskSolved \what
 

\SubtaskSolved  -

\SubtaskSolved  -
\QUESTEND






\WHAT{NEEDS A TOPIC DESCRIPTION}

\QUESTBEGIN

\Task \label{task:minindex} \what~  Implementera algoritmen MININDEX som söker index för minsta heltalet i en sekvens. Pseudokod för algoritmen MININDEX:

\begin{algorithm}[H]
 \SetKwInOut{Input}{Indata}\SetKwInOut{Output}{Utdata}

 \Input{Sekvens $xs$ med $n$ st heltal.}
 \Output{Index för det minsta talet eller $-1$ om $xs$ är tom.  }
 $minPos \leftarrow 0 $\\
 $i \leftarrow 1$ \\
 \While{$i < n$}{
   \If{xs(i) < $xs(minPos)$}{$minPos \leftarrow i$}
   $i \leftarrow i + 1$
 }
 \eIf{$n > 0$}{\Return{$minPos$}}{\Return{$-1$}}
\end{algorithm}

\Subtask Prova algoritmen med penna och papper på sekvensen $(1, 2, -1, 4)$ och rita minnessituationen efter varje runda i loopen. Vad blir skillnaden i exekveringsförloppet om loopvariablen $i$  initialiserats till $0$ i stället för $1$?

\Subtask Implementera algoritmen MININDEX i Scala i en funktion med denna signatur:
\begin{Code}
def indexOfMin(xs: Array[Int]): Int = ???
\end{Code}
Testa för olika fall: tom sekvens; sekvens med endast ett tal; lång sekvens med det minsta talet först, någonstans mitt i, samt sist.

\begin{Code}
// kod till facit
def indexOfMin(xs: Array[Int]): Int = {
  var minPos = 0
  var i = 1
  while (i < xs.size) {
    if (xs(i) < xs(minPos)) minPos = i
    i += 1
  }
  if (xs.size > 0) minPos else -1
}


\end{Code}

\newpage

\AdvancedTasks %%%%%%%%%%%%%%%%%


\SOLUTION


\QUESTEND






\WHAT{NEEDS A TOPIC DESCRIPTION}

\QUESTBEGIN

\Task  \what~ Läs om krullparenteser och vanliga parenteser på stack overflow: \\ \href{http://stackoverflow.com/questions/4386127/what-is-the-formal-difference-in-scala-between-braces-and-parentheses-and-when}{stackoverflow.com/questions/4386127/what-is-the-formal-difference-in-scala-between-braces-and-parentheses-and-when} och prova själv i REPL hur du kan blanda dessa olika slags parenteser på olika vis.

\SOLUTION


\QUESTEND






\WHAT{Tips:}

\QUESTBEGIN

\Task  \what~ Gör jämförande studier av Scalas api-dokumentation för \code{ArrayBuffer}, \code{Array} och \code{Vector}. Ge exempel på metoder som finns på objekt av typen \code{Array} och \code{ArrayBuffer} men inte på objekt av typen \code{Vector}.  Kolla efter metoder som returnerar \code{Unit}. Prova några muterande metoder på \code{Array} och \code{ArrayBuffer} i REPL.

\SOLUTION


\QUESTEND






\WHAT{Tips:}

\QUESTBEGIN

\Task  \what~ Bygg vidare på koden nedan och gör ett Sten-Sax-Påse-spel\footnote{\href{https://sv.wikipedia.org/wiki/Sten,\_sax,\_p\%C3\%A5se}{sv.wikipedia.org/wiki/Sten,\_sax,\_p\%C3\%A5se}} som även meddelar vem som vinner. Koden fungerar att köra som den är, men funktionen \code{winnerMsg} är ej klar.  Du kan använda modulo-räkning med \code{%}-operatorn för att avgöra vem som vinner.

\begin{Code}[basicstyle=\ttfamily\footnotesize\selectfont]]
object Rock {
  import javax.swing.JOptionPane
  import JOptionPane.{showOptionDialog => optDlg}

  def inputOption(msg: String, opt: Vector[String]) =
    optDlg(null, msg, "Option", 0, 0, null, opt.toArray[Object], opt(0))

  def msg(s: String) = JOptionPane.showMessageDialog(null, s)

  val opt =  Vector("Sten", "Sax", "Påse")

  def userChoice = inputOption("Vad väljer du?", opt)

  def computerChoice = (math.random * 3).toInt

  def winnerMsg(user: Int, computer: Int) = "??? vann!"

  def main(args: Array[String]): Unit = {
    var keepPlaying = true
    while (keepPlaying) {
      val u = userChoice
      val c = computerChoice
      msg("Du valde " + opt(u) + "\n" +
          "Datorn valde " + opt(c) + "\n" +
          winnerMsg(u, c))
      if (u != c) keepPlaying = false
    }
  }
}
\end{Code}\SOLUTION


\QUESTEND


%!TEX encoding = UTF-8 Unicode
%!TEX root = ../compendium1.tex

\ifPreSolution

\Exercise{\ExeWeekTHREE}\label{exe:W03}
\begin{Goals}
%!TEX encoding = UTF-8 Unicode
%!TEX root = ../exercises.tex

\item Kunna skapa och använda funktioner med en eller flera parametrar, default-argument, namngivna argument, och uppdelad parameterlista.
\item Kunna använda funktioner som äkta värden.
\item Kunna skapa och använda anonyma funktioner (ä.k. lambda-funktioner).
\item Kunna applicera en funktion på element i en samling.
\item Förstå skillnader och likheter mellan en funktion och en procedur.
\item Förstå vad ett block och en lokal variabel är.
\item Kunna skapa och använda lokala funktioner och förklara nyttan med dessa.
\item Förstå skillnader och likheter mellan värdeanrop och namnanrop.
\item Kunna skapa en enkel kontrollstruktur med fördröjd evaluering av ett block.
\item Förstå skillnaden mellan äkta funktioner och funktioner med sidoeffekter.
%\item Kunna skapa och använda variabler med fördröjd initialisering och förstå när de är användbara.
\item Kunna förklara hur nästlade funktionsanrop sker med   aktiveringsposter.
\item Känna till rekursion och kunna förklara hur rekursiva funktioner fungerar.
\item Känna till att det går att partiellt applicera argument på funktioner med uppdelad parameterlista för att skapa s.k. stegade funktioner (ä.k. curry-funktioner).

%\item Känna till svansrekursion och att svansrekursiva funktioner kan optimeras till loopar.

\end{Goals}

\begin{Preparations}
\item \StudyTheory{03}
\end{Preparations}

\BasicTasks %%%%%%%%%%%%%%%%

\else

\ExerciseSolution{\ExeWeekTHREE}

\fi





\WHAT{Para ihop begrepp med beskrivning.}

\QUESTBEGIN

\Task \what~Koppla varje begrepp med den (förenklade) beskrivning som passar bäst:

\begin{ConceptConnections}
  funktionshuvud & 1 & & A & har parameterlista och eventuellt en returtyp \\ 
  funktionskropp & 2 & & B & beskriver namn och typ på parametrar \\ 
  parameterlista & 3 & & C & argumentet evalueras innan anrop \\ 
  block & 4 & & D & en funktion som anropar sig själv \\ 
  namngivna argument & 5 & & E & gör att argument kan utelämnas \\ 
  defaultargument & 6 & & F & koden som exekveras vid funktionsanrop \\ 
  värdeanrop & 7 & & G & gör att en funktion kan flera resultatvärden \\ 
  namnanrop & 8 & & H & gör att argument kan ges i valfri ordning \\ 
  tupel & 9 & & I & fördröjd evaluering av argument \\ 
  tupelreturtyp & 10 & & J & kan ha lokala namn; sista raden ger värdet \\ 
  äkta funktion & 11 & & K & funktion utan namn; kallas även lambda \\ 
  predikat & 12 & & L & ger alltid samma resultat om samma argument \\ 
  slumptalsfrö & 13 & & M & lista med bestämt antal (heterogena) värden \\ 
  anonym funktion & 14 & & N & ger återupprepningsbar sekvens av pseudoslumptal \\ 
  rekursiv funktion & 15 & & O & en funktion som ger ett booleskt värde \\ 
\end{ConceptConnections}

\SOLUTION

\TaskSolved \what

\begin{ConceptConnections}
  funktionshuvud & 1 & ~~\Large$\leadsto$~~ &  K & har parameterlista och eventuellt returtyp \\ 
  funktionskropp & 2 & ~~\Large$\leadsto$~~ &  M & koden som exekveras vid funktionsanrop \\ 
  parameterlista & 3 & ~~\Large$\leadsto$~~ &  I & beskriver namn och typ på parametrar \\ 
  parameter & 4 & ~~\Large$\leadsto$~~ &  N & namn i funktionshuvud; binds till argument \\ 
  argument & 5 & ~~\Large$\leadsto$~~ &  E & uttryck som är invärde vid funktionsanrop \\ 
  block & 6 & ~~\Large$\leadsto$~~ &  G & kan ha lokala namn; sista raden ger värdet \\ 
  namngivna argument & 7 & ~~\Large$\leadsto$~~ &  H & gör att argument kan ges i valfri ordning \\ 
  default-argument & 8 & ~~\Large$\leadsto$~~ &  L & gör att argument kan utelämnas \\ 
  värdeanrop & 9 & ~~\Large$\leadsto$~~ &  B & argumentet evalueras innan anrop \\ 
  namnanrop & 10 & ~~\Large$\leadsto$~~ &  A & fördröjd evaluering av argument \\ 
  tupel & 11 & ~~\Large$\leadsto$~~ &  J & lista med bestämt antal (heterogena) värden \\ 
  tupelreturtyp & 12 & ~~\Large$\leadsto$~~ &  D & gör att en funktion kan flera resultatvärden \\ 
  anonym funktion & 13 & ~~\Large$\leadsto$~~ &  F & funktion utan namn; kallas även lambda \\ 
  rekursiv funktion & 14 & ~~\Large$\leadsto$~~ &  C & en funktion som anropar sig själv \\ 
\end{ConceptConnections}

\QUESTEND





\WHAT{Definiera och anropa funktioner.}

\QUESTBEGIN

\Task \label{task:funcall} \what~
En funktion med en parameter definieras med följande syntax i Scala:
\vspace{0.5em} \\
\texttt{\code{def} \textit{namn}(\textit{parameter}: \textit{Typ} = \textit{defaultArgument}): \textit{Returtyp} = \textit{returvärde}}

% En funktion med två parametrar definieras med följande syntax i Scala: \vspace{0.5em} \\  \texttt{\code{def} \textit{namn}(\textit{parameter1}: \textit{Typ1}, \textit{parameter2}: \textit{Typ2}): \textit{Returtyp} = \textit{returvärde}}

\Subtask Definiera funktionen \code{öka} som har en heltalsparameter \code{x} och vars returvärde är argumentet plus 1. Defaultargument ska vara 1. Ange returtypen explicit.

\Subtask Vad har uttrycket \code{öka(öka(öka(öka())))} för värde?

\Subtask Definiera funktionen \code{minska} som har en heltalsparameter \code{x} och vars returvärde är argumentet minus 1. Defaultargument ska vara 1. Ange returtypen explicit.

\Subtask Vad är värdet av uttrycket \code{öka(minska(öka(öka(minska(minska())))))}

\Subtask Vad är det för skillnad mellan parameter och argument?

\SOLUTION

\TaskSolved \what

\SubtaskSolved
\begin{Code}
def öka(x: Int = 1): Int = x + 1
\end{Code}

\SubtaskSolved  \code{5}

\SubtaskSolved
\begin{Code}
def minska(x: Int = 1): Int = x - 1
\end{Code}

\SubtaskSolved  \code{1}

\SubtaskSolved
\begin{itemize}
  \item \emph{Kort, förenklad förklaring:} Parametern i funktionshuvudet är ett lokalt namn på indata som kan användas i funktionskroppen, medan argumentet är själva värdet på parametern som skickas med vid anrop.
  \item \emph{Längre, mer exakt förklaring:} En \textbf{parameter} är en deklaration av en oföränderlig variabel i ett funktionshuvud vars namn finns tillgängligt lokalt i funktionskroppen. Vid anrop \emph{binds} parameternamnet till ett specifikt argument. Ett \textbf{argument} är ett uttryck som  appliceras på en funktion vid anrop. Normalt evalueras argumentet innan anropet sker, men om parametertypen föregås av \code{=>} fördröjs evalueringen av argumentet och sker i stället \emph{varje gång} parameternamnet förekommer i funktionskroppen.
\end{itemize}

\QUESTEND



\WHAT{Implementera funktion på olika sätt.}

\QUESTBEGIN

\Task \label{task:funcsumfirst} \what~
Skapa en funktion som kan summera de första \code{n} positiva heltalen.

\Subtask Skriv först funktionshuvudet med \code{???} som funktionskropp. Ge funktionen ett bra namn. Ange returtyp. Kontrollera att din funktion kompilerar utan kompileringsfel innan du går vidare.

\Subtask Implementera funktionen med hjälp av ett intervall och metoden \code{sum}. Testa så att funktionen fungerar. Vad händer om du ger ett negativt argument?

\Subtask Implementera funktionen med hjälp av \code{while}-\code{do}. Vad händer om du ger ett negativt argument?

\SOLUTION

\TaskSolved \what

\SubtaskSolved
\begin{Code}
def sumFirst(n: Int): Int = ???
\end{Code}

\SubtaskSolved
\begin{Code}
def sumFirst(n: Int): Int = (1 to n).sum
\end{Code}
\begin{REPL}
scala> sumFirst(-1)
val res0: Int = 0
\end{REPL}

\SubtaskSolved
\begin{Code}
def sumFirst(n: Int): Int = 
  var result = 0
  var i = 1
  while i <= n do 
    result += i
    i += 1
  end while
  result
end sumFirst
\end{Code}
\begin{REPL}
scala> sumFirst(-1)
val res1: Int = 0
\end{REPL}

\QUESTEND




\WHAT{Textspelet AliensOnEarth.}

\QUESTBEGIN

\Task  \what~Ladda ner spelet nedan \footnote{
\url{https://raw.githubusercontent.com/lunduniversity/introprog/master/compendium/examples/AliensOnEarth.scala}} och studera koden.

\scalainputlisting[basicstyle=\ttfamily\fontsize{10}{12}\selectfont,numbers=left]{examples/AliensOnEarth.scala}

% def randomDistribution(weights: Vector[Int]): Int = {
%   require(weights.size > 0)
%   require(weights.forall(_ >= 0))
%
%   val probabilities = for (w <- weights) yield w / weights.sum.toDouble
%   val rnd = math.random()
%   var i = 0
%   var sum = probabilities(i)
%   while (i < probabilities.size - 1 && rnd > sum) {
%     i += 1
%     sum += probabilities(i)
%   }
%   i
% }

\Subtask Medan du läser koden, försök lista ut vilket som är bästa strategin för att få så mycket poäng som möjligt. Kompilera och kör spelet i terminalen med ditt favoritnamn som argument. Vilket av de tre objekten på planeten jorden har störst sannolikhet att vara bästa alternativet?

\Subtask Para ihop kodsnuttarna nedan med bästa beskrivningen.\footnote{Gör så gott du kan även om allt inte är solklart. Vissa saker kommer vi att gå igenom i detalj först under senare kursmoduler.}

\begin{ConceptConnections}
  \code|options.indices| & 1 & & A & fångar undantag för att förhindra krasch \\ 
  \code|"1X2".toLowercase| & 2 & & B & gör om en sträng till små bokstäver \\ 
  \code|Random.nextInt(n)| & 3 & & C & slumptal i intervallet \code|0 until n| \\ 
  \code|try { } catch { }| & 4 & & D & sträng som kan sträcka sig över flera kodrader \\ 
  \code|""" ... """| & 5 & & E & heltalssekvens med alla index i en sekvens \\ 
  \code|s.stripMargin| & 6 & & F & tar bort marginal till och med vertikalstreck \\ 
  \code|e.printStackTrace| & 7 & & G & skriver ut information om ett undantag \\ 
\end{ConceptConnections}

\noindent\emph{Tips:} Med hjälp av REPL kan du ta reda på hur olika delar fungerar, t.ex.:

\begin{REPL}
scala> val xs = Vector("p", "w", "a")
scala> xs.indices
scala> xs.indices.foreach(i => println(i))
scala> xs.indexOf("w")
scala> xs.indexOf("gurka")
scala> Vector("hej", "hejsan", "hej").indexOf("hej")
scala> try 1 / 0 catch case e: Exception => println(e)
\end{REPL}
%Kolla även dokumentationen för \code{nextInt}, \code{readLine}, m.fl genom att söka här: \\ \url{http://www.scala-lang.org/api/current/index.html}


%\begin{framed}
\noindent\emph{Tips inför fortsättningen:}

\begin{itemize}[nolistsep]
  \item När jag hittade på \code{AliensOnEarth} började jag med ett mycket litet program med en enkel \code{main}-funktion som bara skrev ut något kul. Sedan byggde jag vidare på programmet steg för steg och kompilerade och testade efter varje liten ändring.

  \item När jag kodar har jag REPL igång i ett eget terminalfönster och min kodeditor i ett annat fönster. I ett tredje fönster har jag en terminal med kompilering i \textit{watch mode}, se appendix \ref{appendix:build-scala-cli-watch-mode}. Fråga en handledare om hur du kan arbeta effektivt med stegvisa experimentering i REPL för att bygga upp ett allt större program i små steg.

  \item Detta arbetssätt tar ett tag att komma in i, men är ett bra sätt att uppfinna allt större och bättre program. Ett stort program byggs lättast i små steg och felsökning blir mycket lättare om man bara gör små tillägg åt gången.

  \item Du får också det mycket lättare att förstå ditt program om du delar upp koden i många korta funktioner med bra namn. Du kan sedan lättare hitta på mer avancerade funktioner genom att återanvända befintliga.

  \item Under veckans laboration ska du utveckla ditt eget textspel. Då har du nytta av att återanvända funktionerna för indata och slumpdragning från exempelprogrammet \code{AliensOnEarth}.
\end{itemize}

%\end{framed}


\SOLUTION

\TaskSolved \what~

\SubtaskSolved \code{"penguin"} är bästa alternativ med sannolikheten $\frac{1}{2} + \frac{1}{2}\cdot\frac{1}{3} = \frac{2}{3}$

\SubtaskSolved

\begin{ConceptConnections}
    \code|options.indices| & 1 & ~~\Large$\leadsto$~~ &  F & heltalssekvens med alla index i en sekvens \\ 
  \code|"1X2".toLowercase| & 2 & ~~\Large$\leadsto$~~ &  C & gör om en sträng till små bokstäver \\ 
  \code|Random.nextInt(n)| & 3 & ~~\Large$\leadsto$~~ &  D & slumptal i intervallet \code|0 until n| \\ 
  \code|try { } catch { }| & 4 & ~~\Large$\leadsto$~~ &  B & fångar undantag för att förhindra krasch \\ 
  \code|""" ... """| & 5 & ~~\Large$\leadsto$~~ &  G & sträng som kan sträcka sig över flera kodrader \\ 
  \code|s.stripMargin| & 6 & ~~\Large$\leadsto$~~ &  A & tar bort marginal till och med vertikalstreck \\ 
  \code|e.printStackTrace| & 7 & ~~\Large$\leadsto$~~ &  E & skriver ut information om ett undantag \\ 
\end{ConceptConnections}

\QUESTEND



\WHAT{Äkta funktioner.}

\QUESTBEGIN

\Task  \what~  En äkta funktion%
\footnote{Äkta funktioner uppfyller per definition  \textit{referentiell transparens} \Eng{referential transparency} som du kan läsa mer om här:  \href{https://simple.wikipedia.org/wiki/Referential_transparency}{simple.wikipedia.org/wiki/Referential\_transparency}}
\Eng{pure function} ger alltid samma resultat med samma argument (så som vi är vana vid inom matematiken) och har inga externt observerbara sidoeffekter (till exempel utskrifter).

Vilka funktioner nedan är äkta funktioner?
\begin{Code}
var x = 0
val y = x

def inc(i: Int) = i + 1

def nöff(i: Int) = 
  x = x + i
  "nöff " * x
end nöff

def addX(i: Int) = x + i

def addY(i: Int) = y + i

def isPalindrome(s: String) = s == s.reverse

def rnd(min: Int, max: Int) = math.random() * max + min
\end{Code}


\noindent\emph{Tips:} Skriv av och testa funktionerna i REPL en och en, så att du förstår exakt vad som händer.

\SOLUTION

\TaskSolved \what

\begin{itemize}
  \item Funktionerna  \code{inc}, \code{addY} och \code{isPalindrome} är äkta. Notera att \code{y}-variablen initialiseras till \code{0} och kan sedan inte ändras eftersom den är deklarerad med nyckelordet \code{val}.
\end{itemize}

\QUESTEND


\WHAT{Applicera funktion på varje element i en samling. Funktion som argument.}

\QUESTBEGIN

\Task  \what~

\noindent Deklarera funktionen \code{öka} och variabeln \code{xs} enligt nedan i REPL:
\begin{REPL}
scala> def öka(x: Int) = x + 1
scala> val xs = Vector(3, 4, 5)
\end{REPL}
\noindent Para ihop nedan uttryck till vänster med det uttryck till höger som har samma värde. Om du undrar något, testa uttrycken och olika varianter av dem i REPL.

\begin{ConceptConnections}
  \code|for (i <- 1 to 3) yield öka(i)| & 1 & & A & \code|Vector(5, 6, 7)| \\ 
  \code|Vector(2, 3, 4).map(i => öka(i))| & 2 & & B & \code|Vector(4, 5, 6)| \\ 
  \code|xs.map(öka)| & 3 & & C & \code|Vector(2, 3, 4)| \\ 
  \code|xs.map(öka).map(öka)| & 4 & & D & \code|()| \\ 
  \code|xs.foreach(öka)| & 5 & & E & \code|xs| \\ 
\end{ConceptConnections}

\SOLUTION

\TaskSolved \what

\begin{ConceptConnections}
    \code|for (i <- 1 to 3) yield öka(i)| & 1 & ~~\Large$\leadsto$~~ &  D & \code|Vector(2, 3, 4)| \\ 
  \code|Vector(2, 3, 4).map(i => öka(i))| & 2 & ~~\Large$\leadsto$~~ &  C & \code|xs| \\ 
  \code|xs.map(öka)| & 3 & ~~\Large$\leadsto$~~ &  E & \code|Vector(4, 5, 6)| \\ 
  \code|xs.map(öka).map(öka)| & 4 & ~~\Large$\leadsto$~~ &  A & \code|Vector(5, 6, 7)| \\ 
  \code|xs.foreach(öka)| & 5 & ~~\Large$\leadsto$~~ &  B & \code|()| \\ 
\end{ConceptConnections}

\QUESTEND




\WHAT{Anonyma funktioner.}

\QUESTBEGIN

\Task  \what~  Vi har flera gånger sett syntaxen \code{i => i + 1}, till exempel i en loop \code{(1 to 10).map(i => i + 1)} där funktionen \code{i => i + 1} appliceras på alla heltal från 1 till och med 10 och resultatet blir en ny sekvenssamling.

Syntaxen \code{(i: Int) => i + 1} är en litteral för att skapa ett \emph{funktionsvärde} (kallas även \emph{anonym funktion} eller \emph{lambda-uttryck}). Syntaxen liknar den för funktionsdeklarationer, men nyckelordet \code{def} saknas i funktionshuvudet och i stället för likhetstecken används \code{=>} för att avskilja parameterlistan från funktionskroppen.
Om kompilatorn kan härleda typen ur sammanhanget kan kortformen \code{i => i + 1} användas.

Det finns ett \emph{ännu} kortare sätt att skriva en anonym funktion \emph{om} typen kan härledas \emph{och} den bara använder sin parameter \emph{en enda gång}; då går funktionslitteraler att skriva med s.k. \emph{platshållarsyntax} som använder understreck, till exempel \code{ _ + 1} och som automatiskt expanderas av kompilatorn till \code{ngtnamn => ngtnamn + 1} (namnet på parametern spelar ingen roll; kompilatorn väljer något eget, internt namn).

Para ihop uttryck till vänster med uttryck till höger som har samma värde:

\begin{ConceptConnections}
\input{generated/quiz-w03-lambda-taskrows-generated.tex}
\end{ConceptConnections}

\noindent
Funktionslitteraler kallas \textit{anonyma funktioner}, eftersom de inte har något namn, till skillnad från t.ex. \code{def öka(i: Int): Int = i + 1}, som ju heter \code{öka}. Ett annat vanligt namn är \textit{lambda-uttryck} efter det datalogiska matematikverktyget \href{https://sv.wikipedia.org/wiki/Lambdakalkyl}{lambdakalkyl}.

\SOLUTION

\TaskSolved \what

\begin{ConceptConnections}
    \code|(0 to 2).map(i => i + 1)           | & 1 & ~~\Large$\leadsto$~~ &  B & \code|(2 to 4).map(i => i - 1)| \\ 
  \code|(1 to 3).map(_ + 1)                | & 2 & ~~\Large$\leadsto$~~ &  D & \code|Vector(2, 3, 4)         | \\ 
  \code|(2 to 4).map(math.pow(2, _))       | & 3 & ~~\Large$\leadsto$~~ &  A & \code|Vector(4.0, 8.0, 16.0)  | \\ 
  \code|(3 to 5).map(math.pow(_, 2))       | & 4 & ~~\Large$\leadsto$~~ &  C & \code|Vector(9.0, 16.0, 25.0) | \\ 
  \code|(4 to 6).map(_.toDouble).map(_ / 2)| & 5 & ~~\Large$\leadsto$~~ &  E & \code|Vector(2.0, 2.5, 3.0)   | \\ 
\end{ConceptConnections}

\QUESTEND




\WHAT{Skapa din egen kontrollstruktur med hjälp av namnanrop.}\label{func:upprepa}

\QUESTBEGIN

\Task  \what~Namnanrop skrivs med en raket efter kolon före parametertypen och innebär att argumentet evalueras på plats varje gång.

\Subtask Använd namnanrop i kombination med en uppdelad parameterlista och skapa din egen kontrollstruktur enligt nedan.\footnote{Det är så loopen \code{upprepa} i Kojo är definierad.}
\begin{Code}
def upprepa(n: Int)(block: => Unit): Unit =
  var i = 0
  while i < n do 
    ???
  end while
\end{Code}

\Subtask
Testa din kontrollstruktur i REPL. Låt upprepa 100 gånger att ett slumptal mellan 1 och 6 dras och sedan skrivs ut. Prova även att använda färre klammerparenteser med hjälp av kolon.

\Subtask
Varför behövs namnanrop här?

\SOLUTION

\TaskSolved \what

\SubtaskSolved
\begin{Code}
def upprepa(n: Int)(block: => Unit): Unit =
  var i = 0
  while i < n do
    block
    i += 1
  end while
\end{Code}

\SubtaskSolved
\begin{Code}
upprepa(100):
  val tärningskast = (math.random() * 6 + 1).toInt
  print(s"\$tärningskast ")
\end{Code}

\SubtaskSolved Om parametern \code{block} inte vore deklarerad med namnanrop så hade argumentet evaluerats en gång innan anropet och sedan hade det blivit samma resultat vid varje iteration. Med namnanrop kan block innehålla kod som t.ex. uppdaterar en variabel som vi vill ska ske vid varje iteration. Namn-anrop liknar att koden för argumentet ''klistras in'' på varje plats i funktionskroppen där parameternamnet förekommer. 

\QUESTEND



\WHAT{Lär dig läsa en stack trace.}

\QUESTBEGIN

\Task  \what~  Skriv ett program i filen \texttt{fel.scala} som orsakar ett \emph{körtidsfel} och kör igång det i terminalen med \code{scala-cli run fel.scala}. Studera den stack trace som skrivs ut. Vad innehåller en \code{stack trace}? Diskutera med handledare hur du kan ha nytta av en stack trace när du felsöker.

\SOLUTION

\TaskSolved \what En stack trace innehåller följande information:
\begin{enumerate}
  \item ett felmeddelande
  \item namn på alla funktioner som anropats vid tiden för körtidsfelet, enligt alla aktiveringsposter som ligger på anropsstacken 
  \item aktuell namnrymnd för varje funktionen, alltså paket/singelobjekt
  \item namnet på kodfilen för varje funktion
  \item radnummer i varje funktion 
  \item den funktion som kommer först är den funktion där felet inträffade
  \item eventuellt kan felet inträffa i standardbibliotekets funktioner och då är din egen funktion tidigare i anropskedjan
\end{enumerate}

Exempel på en stack trace:
\begin{REPLnonum}
> cat fel.scala 
@main def run = 
  println("Hej Scala!" + Vector().head)
> scala-cli run fel.scala
Compiling project (Scala 3.3.0, JVM)
Compiled project (Scala 3.3.0, JVM)
Exception in thread "main" java.util.NoSuchElementException: empty.head
	at scala.collection.immutable.Vector.head(Vector.scala:279)
	at fel$package$.run(fel.scala:2)
	at run.main(fel.scala:1)
>
\end{REPLnonum}

\QUESTEND


\ExtraTasks %%%%%%%%%%%%%%%%%%%%%%%%%%%%%%%%%%%%%%%%%%%%%%%%%%%%%%%%%%



\WHAT{Funktion med flera parametrar.}

\QUESTBEGIN

\Task  \what~  

\Subtask Definiera i REPL två funktioner \code{sum} och \code{diff} med två heltalsparametrar som returnerar summan respektive differensen av argumenten:
\begin{Code}
def sum(x: Int, y: Int): Int = ???

def diff(x: Int, y: Int): Int = ???
\end{Code}
Vad har nedan uttryck för värden? Förklara vad som händer.

\Subtask \code{diff(0, 100)}

\Subtask \code{diff(100, sum(42, 43))}

\Subtask \code{sum(sum(42, 43), diff(100, sum(0, 0)))}

\Subtask \code{sum(diff(Byte.MaxValue, Byte.MinValue), 1)}

\SOLUTION

\TaskSolved \what

\SubtaskSolved
\begin{Code}
  def sum(x: Int, y: Int): Int = x + y
  
  def diff(x: Int, y: Int): Int = x - y
\end{Code}
  

\SubtaskSolved  Det blir \code{-100} efter som \code{0 - 100 == -100} 

\SubtaskSolved  Det blir \code{15} eftersom det nästlade anropet motsvarar \\\code{diff(100, 42 + 43) == (100 - 85)}

\SubtaskSolved  Det blir \code{185} eftersom det nästlade anropet motsvarar \\\code{sum(42 + 43, 100 - 0) == (85 + 100)}

\SubtaskSolved  Det blir \code{256} eftersom \code{Byte.MaxValue == 127} och \  code{Byte.MinValue == -128} och \code{sum(127 + 128, 1) == 256}

\QUESTEND



\WHAT{Medelvärde.}

\QUESTBEGIN

\Task  \what~ Skriv och testa en funktion \code{avg} som räknar ut medelvärdet mellan två heltal och returnerar en \code{Double}.

\SOLUTION

\TaskSolved \what

\begin{Code}
def avg(x: Int, y: Int): Double = (x + y) / 2.0
\end{Code}

\QUESTEND




\WHAT{Funktionsanrop med namngivna argument.}

\QUESTBEGIN

\Task  \what~
\begin{REPL}
scala> def skrivNamn(efternamn: String, förnamn: String) =
         println(s"Namn: $efternamn, $förnamn")
scala> skrivNamn(förnamn = "Stina", efternamn = "Triangelsson")
scala> skrivNamn(efternamn = "Oval", "Viktor")

\end{REPL}

\Subtask Vad skrivs ut efter rad 3 resp. rad 4 ovan?

\Subtask Nämn tre fördelar med namngivna argument.

\SOLUTION

\TaskSolved \what~

\SubtaskSolved
\begin{REPL}
Namn: Triangelsson, Stina
Namn: Oval, Viktor
\end{REPL}

\SubtaskSolved
\begin{itemize}
  \item Anroparen kan själv välja ordning.
  \item Koden blir lättare att begripa om parameternamnen är självbeskrivande.
  \item Hjälper till att förhindra buggar som beror på förväxlade parametrar.
\end{itemize}

\QUESTEND



\WHAT{Funktion som äkta värde.}

\QUESTBEGIN

\Task  \what~  Funktioner är \emph{äkta värden} i Scala%\footnote{I likhet med t.ex. Javascript, men till skillnad från t.ex. Java.}
. Det betyder att variabler kan ha funktioner som värden och funktionsvärden kan vara argument till funktioner som har funktionsparametrar. Funktioner som tar funktioner som argument kallas \emph{högre ordningens funktioner}.

En funktion som har en heltalsparameter och ett heltalsresultat är av funktionstypen \code{Int => Int} (uttalas \emph{int-till-int}) och värdet av funktionen utgör ett objekt som har en metod som heter \code{apply} med motsvarande funktionstyp.

\Subtask \label{subtask:funcval} Deklarera nedan funktioner och variabler i REPL. Para sedan ihop nedan uttryck till vänster med det uttryck till höger som skapar samma utskrift. Om du undrar något, testa uttrycken och olika varianter av dem i REPL.

\begin{REPL}
scala> def hälsa(): Unit = println("Hej!")
scala> def fleraAnrop(antal: Int, f: () => Unit): Unit =
         for _ <- 1 to antal do f()
scala> val f1 = () => hälsa()
scala> var f2 = (s: String) => println(s)
scala> val f3 = () => f2("Thunk")
\end{REPL}

\begin{ConceptConnections}
  \code| fleraAnrop(1, hälsa) | & 1 & & A & \code| f2("Hej!\nHej!")| \\ 
  \code| fleraAnrop(3, hälsa) | & 2 & & B & \code| fleraAnrop(3, f1)  | \\ 
  \code| fleraAnrop(2, f1)    | & 3 & & C & \code| f3()               | \\ 
  \code| fleraAnrop(1, f3)    | & 4 & & D & \code| f2("Hej!")       | \\ 
\end{ConceptConnections}


\Subtask Vilka typer har variablerna \code{f1}, \code{f2} och \code{f3}?

\Subtask Funkar detta? Varför? \code{f2 = f1}

\Subtask Funkar detta? Varför? \code{val f4 = fleraAnrop}

\Subtask Funkar detta? Varför? \code{val f4 = hälsa}

\Subtask Funkar detta? Varför? \code{val f4: () => Unit = hälsa}

\SOLUTION

\TaskSolved \what

\SubtaskSolved

\begin{ConceptConnections}
    \code| fleraAnrop(1, hälsa) | & 1 & ~~\Large$\leadsto$~~ &  D & \code| f2("Hej!")       | \\ 
  \code| fleraAnrop(3, hälsa) | & 2 & ~~\Large$\leadsto$~~ &  B & \code| fleraAnrop(3, f1)  | \\ 
  \code| fleraAnrop(2, f1)    | & 3 & ~~\Large$\leadsto$~~ &  A & \code| f2("Hej!\nHej!")| \\ 
  \code| fleraAnrop(1, f3)    | & 4 & ~~\Large$\leadsto$~~ &  C & \code| f3()               | \\ 
\end{ConceptConnections}

\SubtaskSolved \code{f1} och \code{f3} är av typen \code{() => Unit} och \code{f2} av typen \code{String => Unit}.

\SubtaskSolved  Nej. \code{f1} och \code{f2} är av två olika funktionstyper.

\SubtaskSolved  Ja, det går fint.

\SubtaskSolved  Nej. När funktionen inte har någon parameter behöver kompilatorn mer information för att vara säker på att det är ett funktionsvärde du vill ha.

\SubtaskSolved Ja! Nu med typinformationen på plats är kompilatorn säker på vad du vill göra.

\QUESTEND



\WHAT{Bortkastade resultatvärden och returtypen \code{Unit}.}

\QUESTBEGIN

\Task  \what~ Undersök nedan kod i REPL och förklara vad som händer.

\Subtask
\begin{REPL}
scala> def tom = println("")
scala> println(tom)
\end{REPL}

\Subtask
\begin{REPL}
scala> def bortkastad: Unit = 1 + 1
scala> println(bortkastad)
\end{REPL}

\Subtask
\begin{REPL}
scala> def bortkastad2 = { val x = 1 + 1 }
scala> println(bortkastad2)
\end{REPL}

\Subtask Varför är det bra att explicit ange \code{Unit} som returtyp för procedurer?

\SOLUTION

\TaskSolved \what

\SubtaskSolved Procedurer returnerar tomma värdet och \code{println} är en procedur. När tomma värdet skrivs ut visas \code{()}.

\SubtaskSolved Procedurer returnerar tomma värdet. Om du anger returtyp \code{Unit} explicit, har du bättre chans att kompilatorn kan ge varning då uträkningar kommer att kastas bort. En varning avbryter inte exekveringen, utan är ett sätt för kompilatorn att ge dig tips om saker som kan behöva fixas till i din kod.

\SubtaskSolved I Scala är variabeldeklaration, precis som en tilldelningssats, och inte ett uttryck och saknar värde.

\SubtaskSolved  Koden blir lättare att läsa och kompilatorn får bättre möjlighet att hjälpa till med varningar om resultatvärden riskerar att bli bortkastade.

\QUESTEND


\WHAT{Namnanrop.}

\QUESTBEGIN

\Task  \what~

Deklarera denna procedur i REPL:
\begin{Code}
def görDettaTvåGånger(b: => Unit): Unit = { b; b }
\end{Code}

Anropa \code{görDettaTvåGånger} med ett block som parameter. Blocket ska innehålla en utskriftssats. Förklara vad som händer.

\SOLUTION

\TaskSolved \what

Blocket är ett uttryck som har värdet \code{(): Unit}. Evalueringen av blocket sker där namnet \code{b} förekommer i procedurkroppen, vilket är två gånger.
\begin{REPL}
scala> görDettaTvåGånger { println("goddag") }
goddag
goddag
\end{REPL}

\QUESTEND




\clearpage

\AdvancedTasks %%%%%%%%%%%%%%%%%%%%%%%%%%%%%%%%%%%%%%%%%%%%%%%%%%%%%%%%%%%




\WHAT{Föränderlighet av parametrar.}

\QUESTBEGIN

\Task \what~Vad tror du om detta: Är en parameter förändringsbar i funktionskroppen ...

\Subtask ... i Scala?  (Ja/Nej)

\Subtask ... i Java?  (Ja/Nej)

\Subtask ... i Python?  (Ja/Nej)


\SOLUTION

\TaskSolved \what~

\Subtask Nej, i Scala är parametern oföränderlig och det blir kompileringsfel om man försöker tilldela den ett nytt värde i funktionskroppen.

\Subtask \Subtask Ja det går utmärkt i både Java och Python att ändra värdet på parametern i funktionskroppen med tilldelning, men koden riskerar att bli förvirrande.\\
\url{https://stackoverflow.com/questions/2970984}

\QUESTEND



\WHAT{Värdeanrop och namnanrop.}

\QUESTBEGIN

\Task  \what~Normalt sker i Scala (och i Java) s.k. \emph{värdeanrop} vid anrop av funktioner, vilket innebär att argumentuttrycket evalueras \emph{före} bindningen till parameternamnet sker.

Man kan också i Scala (men inte i Java) med syntaxen \code{=>} framför parametertypen deklarera att \emph{namnanrop} ska ske, vilket innebär att evalueringen av argumentuttrycket \emph{fördröjs} och sker \emph{varje gång} namnet används i metodkroppen.

Deklarera nedan funktioner i REPL.

\begin{Code}
def snark: Int = { print("snark "); Thread.sleep(1000); 42 }
def callByValue(x: Int):   Int = x + x
def callByName(x: => Int): Int = x + x
lazy val zzz = snark
\end{Code}

\noindent Förklara vad som händer när nedan uttryck evalueras.

\Subtask \code{snark + snark}

\Subtask \code{callByValue(snark)}

\Subtask \code{callByName(snark)}

\Subtask \code{callByName(zzz)}

\SOLUTION

\TaskSolved \what

\SubtaskSolved Vid varje anrop av \code{snark} sker en utskrift och en fördröjnig innan $42$ returneras. \\\code{42 + 42 == 84} vilket blir värdet av uttrycket.
\begin{REPL}
scala> snark + snark
snark snark val res1: Int = 84
\end{REPL}

\SubtaskSolved Uttrycket \code{snark} evalueras direkt vid anropet och parametern \code{x} binds till värdet $42$ och i funktionskroppen beräknas $42+42$. Utskriften sker bara en gång.
\begin{REPL}
callByValue(snark)
snark val res2: Int = 84
\end{REPL}

\SubtaskSolved Evalueringen av uttrycket \code{snark} fördröjs tills varje förekomst av parametern \code{x} i funktionskroppen. Utskriften sker två gånger.
\begin{REPL}
callByName(snark)
snark snark val res3: Int = 84
\end{REPL}

\SubtaskSolved Evalueringen av uttrycket \code{zzz} fördröjs tills varje förekomst av parametern \code{x} i funktionskroppen. Utskriften sker en gång eftersom \code{val}-variabler tilldelas sitt värde en gång för alla vid den fördröjda initialiseringen.
\begin{REPL}
callByName(zzz)
snark val res4: Int = 84
\end{REPL}

\QUESTEND



\WHAT{Skapa egen kontrollstruktur för iteration med loop-variabel.}

\QUESTBEGIN

\Task  \what~

\Subtask Fördelen med \code{upprepa} i uppgift \ref{func:upprepa} är att den är koncis och lättanvänd. Men den är inte lika lätt att använda om man behöver tillgång till en loopvariabel. Implementera därför nedan kontrollstruktur.

\begin{Code}
def repeat(n: Int)(p: Int => Unit): Unit = 
  var i = 0
  while i < n do
    ??? 
\end{Code}

\Subtask Använd \code{repeat} för att 100 gånger skriva ut loopvariabeln och ett slumpdecimaltal mellan 0 och 1.


\SOLUTION

\TaskSolved \what

\SubtaskSolved
\begin{Code}
def repeat(n: Int)(p: Int => Unit): Unit = 
  var i = 0
  while i < n do
    p(i)
    i += 1
  end while
end repeat
\end{Code}

\SubtaskSolved

\begin{Code}
repeat(100){ i =>
  print("i ")
  println(math.random())
}
\end{Code}
Du kan använda färre klammerparenteser med hjälp av kolon:
\begin{Code}
repeat(100): i =>
  print("i ")
  println(math.random())
\end{Code}

\QUESTEND






\WHAT{Uppdelad parameterlista och stegade funktioner.}

\QUESTBEGIN

\Task \what~Man kan dela upp parametrarna till en funktion i flera parameterlistor. Funktionen \code{add1} nedan har en parameterlista med två parametrar medan \code{add2} har två parameterlistor med en parameter vardera:
\begin{Code}
  def add1(a: Int, b: Int) = a + b
  def add2(a: Int)(b: Int) = a + b
\end{Code}

\Subtask  När man anropar funktionen \code{add2} ska argumenten skrivas inom två olika parentespar. Hur kan du använda \code{add2} för att räkna ut \code{1 + 1}?

\Subtask En fördel med uppdelade parameterlistor är att man kan skapa s.k. \emph{stegade funktioner}\footnote{Kallas även Curry-funktioner efter matematikern och logikern Haskell Brooks Curry.} där argumenten är partiellt applicerade. Prova det stegade funktionsvärdet \code{singLa} nedan. Vad skrivs ut på efter raderna 3 och 5?

\begin{REPL}
scala> def repeat(s: String)(n: Int): String = s * n
scala> val song = repeat("doremi ")(3)
scala> println(song)
scala> val singLa = repeat("la")
scala> println(singLa(7))
\end{REPL}

\SOLUTION

\TaskSolved \what

\SubtaskSolved
\begin{REPL}
scala> def add2(a: Int)(b: Int) = a + b
def add2(a: Int)(b: Int): Int

scala> add2(1)(1)
val res0: Int = 2
\end{REPL}

\SubtaskSolved
\begin{itemize}

\item Rad 3:
\begin{REPLnonum}
doremi doremi doremi 
\end{REPLnonum}

\item Rad 5:
\begin{REPLnonum}
lalalalalalala
\end{REPLnonum}

\end{itemize}


\QUESTEND




\WHAT{Rekursion.}

\QUESTBEGIN

\Task\Uberkurs  \what~  En rekursiv funktion anropar sig själv.

\Subtask Förklara vad som händer nedan.

\begin{REPL}
scala> def countdown(x: Int): Unit = 
         if x > 0 then {println(x); countdown(x - 1)}
scala> countdown(10)
scala> countdown(-1)
scala> def finalCountdown(x: Byte): Unit =
         {println(x); Thread.sleep(100); finalCountdown((x-1).toByte); 1 / x}
scala> finalCountdown(Byte.MaxValue)
\end{REPL}

\Subtask Vad händer om du gör satsen som riskerar division med noll \emph{före} det rekursiva anropet i funktionen \code{finalCountdown} ovan?

\Subtask Förklara vad som händer nedan. Varför tar sista raden längre tid än näst sista raden?
\begin{REPL}
scala> def signum(a: Int): Int = if a >= 0 then 1 else -1
scala> def add(x: Int, y: Int): Int =
         if y == 0 then x else add(x + 1, y - signum(y))
scala> add(100, 100)
scala> add(Int.MaxValue, 0)
scala> add(0, Int.MaxValue)
\end{REPL}

\SOLUTION

\TaskSolved \what

\SubtaskSolved
\code{countdown} skriver ut x och gör ett rekursivt anrop med \code{x - 1} som argument, men bara om basvillkoret \code{x > 0} är uppfyllt. Resultatet blir en ändlig  repetition.
\code{finalCountdown} anropar sig själv rekursivt men saknar ett basvillkor som kan avbryta rekursionen, vilket genererar en oändlig repetition. Vid -128 blir det \emph{overflow} eftersom bitarna inte räcker till för större negativa tal och räkningen börjar om på 127. (Om minskar fördröjningen till \code{Thread.sleep(1)} blir det ganska snabbt \emph{stack overflow})

\SubtaskSolved
Eftersom vi hade \code{1/x} \emph{efter} det rekursiva anropet i föregående deluppgift, så kom vi aldrig till denna (potentiellt ödesdigra) beräkning, utan lade bara aktiveringsposter på hög på stacken vid varje anrop. Om vi placerar \code{1/x} \emph{före} det rekursiva anropet, så når vi detta uttryck direkt och det kastas ett undantag p.g.a. division med noll.

\SubtaskSolved
Den sista raden leder till många fler rekursiva anrop, så som basvillkoret och det rekursiva anropet är konstruerade. Lägg gärna in en \code{println}-sats före det rekursiva anropet och undersök i detalj vad som sker.

\QUESTEND



\WHAT{Undersök svansrekursion genom att kasta undantag.}

\QUESTBEGIN

\Task\Uberkurs  \what~  Förklara vad som händer. Kan du hitta bevis för att kompilatorn kan optimera rekursionen till en vanlig loop?

\begin{REPL}
scala> def explode = throw Exception("BANG!!!")
scala> explode
scala> def countdown(n: Int): Unit =
         if n == 0 then explode else countdown(n-1)
scala> countdown(10)
scala> countdown(10000)
scala> def countdown2(n: Int): Unit =
         if n == 0 then explode else {countdown2(n-1); print("no tailrec")}
scala> countdown2(10)
scala> countdown2(10000)
\end{REPL}

\SOLUTION

\TaskSolved \what~\code{countdown} är svansrekursiv eftersom det rekursiva anropet står \emph{sist} och kan då optimeras till en \code{while}-loop av kompilatorn. Det går fint att köra ända till det exploderar, även med 10000 anrop, och i felmeddelandet finns det endast ett anrop till \code{countdown}.

\code{countdown2} är inte svansrekursiv eftersom den har ett uttryck \code{efter} det rekursiva anropet. I felutskriften syns alla rekursiva anrop till \code{countdown2} innan basvillkoret inträffade. Vid \code{countdown2(10000)} uppfylls inte basvillkoret innan det blir \code{StackOverflowError}.

\QUESTEND



\WHAT{\code{@tailrec}-annotering.}

\QUESTBEGIN

\Task\Uberkurs  \what~  Du kan be kompilatorn att ge felmeddelande om den inte kan optimera koden till en motsvarande while-loop. Detta kan användas i de fall man vill vara helt säker på att kompilatorn kan optimera koden och det inte kan finnas risk för en överfull stack \Eng{stack overflow} på grund av för djup anropsnästling.

Prova nedan rader i REPL och förklara vad som händer.
\begin{REPL}
scala> def countNoTailrec(n: Long): Unit =
         if n <= 0L then println("Klar! " + n) else {countNoTailrec(n-1L); ()}
scala> countNoTailrec(1000L)
scala> countNoTailrec(100000L)
scala> import scala.annotation.tailrec
scala> @tailrec def countNoTailrec(n: Long): Unit =
         if n <= 0L then println("Klar! " + n) else {countNoTailrec(n-1L); ()}
scala> @tailrec def countTailrec(n: Long): Unit =
         if n <= 0L then println("Klar! " + n) else countTailrec(n-1L)
scala> countTailrec(1000L)
scala> countTailrec(100000L)
scala> countTailrec(Int.MaxValue.toLong * 2L)
\end{REPL}

\SOLUTION

\TaskSolved \what~Första gången \code{countNoTailrec(100000L)} anropas blir det \code{StackOverflowError}. Med annoteringen \code{@tailrec} får vi ett kompileringsfel eftersom kompilatorn inte kan optimera en icke svansrekursiv funktion. Om funktionen skrivs om kan kompilatorn optimera funktionen så att rekursionen byts ut mot en \code{while}-loop och vi kan köra så länge vi orkar utan att stacken flödar över. Och himla snabbt går det!!

\QUESTEND

%!TEX encoding = UTF-8 Unicode
%!TEX root = ../compendium2.tex

\Exercise{\ExeWeekFOUR}\label{exe:W04}
\begin{Goals}
%!TEX encoding = UTF-8 Unicode
%!TEX root = ../compendium2.tex

\item Kunna skapa och använda objekt som moduler.
\item Känna till att funktioner är objekt med en \code{apply}-metod.
\item Förstå begreppen synlighet, \code{private}, \code{import}, namnrymd och skuggning.
\item \TODO{FLER MÅL OM OBJEKT HÄR}

%\item Känna till svansrekursion och att svansrekursiva funktioner kan optimeras till loopar.

\end{Goals}

\begin{Preparations}
\item \StudyTheory{04}
\end{Preparations}

\BasicTasks %%%%%%%%%%%%%%%%

\TODO{ÖVNINGAR OM OBJEKT}


%!TEX encoding = UTF-8 Unicode
%!TEX root = ../exercises.tex

\ifPreSolution

\Exercise{\ExeWeekFIVE}\label{exe:W05}

\begin{Goals}
%!TEX encoding = UTF-8 Unicode

%!TEX root = ../compendium2.tex

\item Kunna deklarera klasser med klassparametrar.
\item Kunna skapa objekt med \code{new} och konstruktorargument.
\item Förstå innebörden av referensvariabler och värdet \code{null}.
\item Förstå innebörden av begreppen instans och referenslikhet.
\item Kunna använda nyckelordet \code{private} för att styra synlighet i primärkonstruktor.
\item Förstå i vilka sammanhang man kan ha nytta av en privat konstruktor.
\item Kunna implementera en klass utifrån en specifikation.
\item Förstå skillnaden mellan referenslikhet och strukturlikhet.
\item Känna till hur case-klasser hanterar likhet.
\item Förstå nyttan med att möjliggöra framtida förändring av attributrepresentation.
\item Känna till begreppen getters och setters.
\item Känna till accessregler för kompanjonsobjekt.
\item Känna till skillnaden mellan \code{==} och \code{eq}, samt \code{!=} versus \code{ne}.

\end{Goals}

\begin{Preparations}
\item \StudyTheory{05}
\end{Preparations}

\else


\ExerciseSolution{\ExeWeekFIVE}


\fi


\BasicTasks %%%%%%%%%%%


\WHAT{Para ihop begrepp med beskrivning.}

\QUESTBEGIN

\Task \what

\vspace{1em}\noindent Koppla varje begrepp med den (förenklade) beskrivning som passar bäst:

\begin{ConceptConnections}
  klass & 1 & & A & ett värde som ej refererar till någon instans \\ 
  instans & 2 & & B & nyckelord vid direkt instansiering av klass \\ 
  konstruktor & 3 & & C & ser privata medlemmar i klass med samma namn \\ 
  klassparameter & 4 & & D & binds till argument som ges vid konstruktion \\ 
  referenslikhet & 5 & & E & indirekt åtkomst av attributvärde \\ 
  innehållslikhet & 6 & & F & slipper skriva new; automatisk innehållslikhet \\ 
  case-klass & 7 & & G & instanser anses lika om de har samma tillstånd \\ 
  getter & 8 & & H & indirekt tilldelning av attributvärde \\ 
  setter & 9 & & I & hjälpfunktion för indirekt konstruktonsanrop \\ 
  kompanjonsobjekt & 10 & & J & upplaga av ett objekt med eget tillståndsminne \\ 
  fabriksmetod & 11 & & K & skapar instans, allokerar plats för tillståndsminne \\ 
  \code|null| & 12 & & L & en mall för att skapa flera instanser av samma typ \\ 
  \code|new| & 13 & & M & instanser anses olika även om tillstånden är lika \\ 
\end{ConceptConnections}

\SOLUTION

\TaskSolved \what

\begin{ConceptConnections}
  klass & 1 & ~~\Large$\leadsto$~~ &  I & en mall för att skapa flera instanser av samma typ \\ 
  instans & 2 & ~~\Large$\leadsto$~~ &  F & upplaga av ett objekt med eget tillståndsminne \\ 
  konstruktor & 3 & ~~\Large$\leadsto$~~ &  E & skapar instans, allokerar plats för tillståndsminne \\ 
  klassparameter & 4 & ~~\Large$\leadsto$~~ &  M & binds till argument som ges vid konstruktion \\ 
  referenslikhet & 5 & ~~\Large$\leadsto$~~ &  L & instanser anses olika även om tillstånden är lika \\ 
  innehållslikhet & 6 & ~~\Large$\leadsto$~~ &  J & instanser anses lika om de har samma tillstånd \\ 
  case-klass & 7 & ~~\Large$\leadsto$~~ &  D & slipper skriva new; automatisk innehållslikhet \\ 
  getter & 8 & ~~\Large$\leadsto$~~ &  A & indirekt åtkomst av attributvärde \\ 
  setter & 9 & ~~\Large$\leadsto$~~ &  B & indirekt tilldelning av attributvärde \\ 
  kompanjonsobjekt & 10 & ~~\Large$\leadsto$~~ &  H & ser privata medlemmar i klass med samma namn \\ 
  fabriksmetod & 11 & ~~\Large$\leadsto$~~ &  G & hjälpfunktion för indirekt konstruktonsanrop \\ 
  \code|null| & 12 & ~~\Large$\leadsto$~~ &  K & ett värde som ej refererar till någon instans \\ 
  \code|new| & 13 & ~~\Large$\leadsto$~~ &  C & nyckelord vid direkt instansiering av klass \\ 
\end{ConceptConnections}

\QUESTEND


\WHAT{Klass och instans.}

\QUESTBEGIN

\Task \what~Du har i övning \texttt{\ExeWeekFOUR}~sett hur singelobjekt i en egen namnrymd  kan samla funktioner (metoder) och ha tillstånd (attribut). Men singelobjekt finns bara i en upplaga.

Vill du kunna skapa många objekt av samma typ behöver du en \emph{klass}. En objektupplaga som skapats ur en klass kallas en \emph{instans} av klassen. Varje instans har sitt eget tillstånd.

Deklarera singelobjektet och klassen nedan i REPL.

\begin{Code}
object Singelpunkt { var x = 1; var y = 2 }
class  Punkt       { var x = 3; var y = 2 }
\end{Code}

\Subtask  Antag att uttrycken till vänster evalueras uppifrån och ned. Vilket resultat till höger hör ihop med respektive uttryck? Prova i REPL om du är osäker.\footnote{Strängen efter \code{@}-tecknet är en hexadecimal representation av det heltal som tillordnas varje objekt för att systemet ska kunna särskilja olika instanser. \url{https://stackoverflow.com/questions/4712139}}


\begin{ConceptConnections}
  \code|Singelpunkt.x               | & 1 & & A & \code|2| \\ 
  \code|Punkt.x                     | & 2 & & B & \verb|p2: Punkt = Punkt@51ab04bd| \\ 
  \code|val p  = new Singelpunkt    | & 3 & & C & \verb|p1: Punkt = Punkt@27a1a53c| \\ 
  \code|val p1 = new Punkt          | & 4 & & D & \verb|error: not found: type| \\ 
  \code|val p2 = new Punkt          | & 5 & & E & \code|java.lang.NullPointerException| \\ 
  \code|{ p1.x = 1; p2.x }          | & 6 & & F & \code|1| \\ 
  \code|(new Punkt).y               | & 7 & & G & \code|3| \\ 
  \code|{ val p: Punkt = null; p.x }| & 8 & & H & \code|error: not found: value| \\ 
\end{ConceptConnections}

\Subtask Vid tre tillfällen blir det fel. Varför? Är det kompileringsfel eller exekveringsfel?

\SOLUTION

\TaskSolved \what

\SubtaskSolved

\begin{ConceptConnections}
  \code|Singelpunkt.x               | & 1 & ~~\Large$\leadsto$~~ &  E & \code|1| \\ 
  \code|Punkt.x                     | & 2 & ~~\Large$\leadsto$~~ &  A & \code|error: not found: value| \\ 
  \code|val p  = new Singelpunkt    | & 3 & ~~\Large$\leadsto$~~ &  F & \verb|error: not found: type| \\ 
  \code|val p1 = new Punkt          | & 4 & ~~\Large$\leadsto$~~ &  D & \code|p1: Punkt = Punkt@27a1a53c| \\ 
  \code|val p2 = new Punkt          | & 5 & ~~\Large$\leadsto$~~ &  C & \code|p2: Punkt = Punkt@51ab04bd| \\ 
  \code|{ p1.x = 1; p2.x }          | & 6 & ~~\Large$\leadsto$~~ &  G & \code|3| \\ 
  \code|(new Punkt).y               | & 7 & ~~\Large$\leadsto$~~ &  B & \code|2| \\ 
  \code|{ val p: Punkt = null; p.x }| & 8 & ~~\Large$\leadsto$~~ &  H & \code|java.lang.NullPointerException| \\ 
\end{ConceptConnections}

\SubtaskSolved

\noindent\begin{tabular}{l l p{5cm}}

~\\ \emph{fel} & \emph{typ} & \emph{förklaring} \\\hline

\code|error: not found: value|
& kompileringsfel & det finns ingen instans med namnet \code|Punkt|\\

\verb|error: not found: type|
& kompileringsfel & det finns ingen klass som heter \code|Singelpunkt|\\

\code|NullPointerException|
& körtidsfel & det går inte att referera attribut i en instans som inte finns\\

\end{tabular}

\QUESTEND



\WHAT{Klassparametrar.}

\QUESTBEGIN

\Task \what~Klassen punkt i föregående uppgift är inte så smidig att använda eftersom man först \emph{efter} instansiering kan ge attributen \code{x} och \code{y} de koordinatvärden man önskar och detta måste ske med explicita tilldelningssatser.

Detta problem kan du lösa med \emph{klassparametrar} som låter dig initialisera attributen med konstruktionsargument och på så sätt ange ett initialtillstånd direkt i samband med instansiering.

Deklarera klassen nedan i REPL.

\begin{Code}
class Point(var x: Int, var y: Int)
\end{Code}


\Subtask  Antag att uttrycken till vänster evalueras uppifrån och ned. Vilket resultat till höger hör ihop med respektive uttryck? Prova i REPL om du är osäker.

\begin{ConceptConnections}
  \code|val p1 = Point(1, 2)        | & 1 & & A & \code|1| \\ 
  \code|val p2 = new Point          | & 2 & & B & \verb|p2: Point = Point@218cf600| \\ 
  \code|val p1 = new Point(1, 2)    | & 3 & & C & \code|error: not found: value| \\ 
  \code|val p2 = new Point(3, 4)    | & 4 & & D & \code|error: too many arguments| \\ 
  \code|p2.x - p1.x                 | & 5 & & E & \code|error: not enough arguments| \\ 
  \code|(new Point(0, 1)).y         | & 6 & & F & \code|2| \\ 
  \code|new Point(0, 1, 2)          | & 7 & & G & \verb|p1: Point = Point@30ef773e| \\ 
\end{ConceptConnections}

\Subtask Vid tre tillfällen blir det fel. Varför? Är det kompileringsfel eller exekveringsfel?

\SOLUTION

\TaskSolved \what

\SubtaskSolved

\begin{ConceptConnections}
  \code|val p1 = Point(1, 2)        | & 1 & ~~\Large$\leadsto$~~ &  C & \verb|p1: Point = Point@30ef773e| \\
  \code|val p2 = new Point          | & 2 & ~~\Large$\leadsto$~~ &  B & \verb|missing argument for parameter| \\
  \code|val p2 = new Point(3, 4)    | & 3 & ~~\Large$\leadsto$~~ &  D & \verb|p2: Point = Point@218cf600| \\
  \code|p2.x - p1.x                 | & 4 & ~~\Large$\leadsto$~~ &  F & \code|2| \\
  \code|(new Point(0, 1)).y         | & 5 & ~~\Large$\leadsto$~~ &  A & \code|1| \\
  \code|new Point(0, 1, 2)          | & 6 & ~~\Large$\leadsto$~~ &  E & \verb|too many arguments for constructor|

\end{ConceptConnections}

\SubtaskSolved

\noindent\begin{tabular}{l l p{5cm}}

  ~\\ \emph{fel} & \emph{typ} & \emph{förklaring} \\\hline

  \code|error: not found: value|
  & kompileringsfel & det finns ingen instans med namnet \code|Point|\\

  \verb|error: not enough arguments|
  & kompileringsfel  & du måste ge argument vid konstruktion av klassen \code|Point| \\

  \code|error: too many arguments|
  & kompileringsfel & antalet argument stämmer ej överens med antalet klassparametrar\\

\end{tabular}

\QUESTEND



\WHAT{Oföränderlig klass med defaultargument.}

\QUESTBEGIN

\Task \what~Det du tidigare lärt dig om parametrar och argument är tillämpligt även på klassparametrar, t.ex. defaultargument och namngivna argument. Man kan dessutom framför klassparametrar använda synlighetsmodifieraren \code{private} och nyckelorden \code{var} och \code{val}.

Om inget anges framför en klassparameter är det \code{private val} som gäller\footnote{För case-klasser, som vi ska se snart, är det i stället \code{val} som gäller (alltså inte \code{private}).}.

Deklarera nedan klass i REPL.

\begin{Code}
class Point3D(val x: Int = 0, val y: Int = 0, z: Int = 0)
\end{Code}

\Subtask Antag att uttrycken till vänster evalueras uppifrån och ned. Vilket resultat till höger hör ihop med respektive uttryck? Prova i REPL om du är osäker.

\begin{ConceptConnections}
  \code|val p1 = Point3D()          | & 1 & & A & \code|false| \\ 
  \code|val p2 = Point3D(y = 1)     | & 2 & & B & \code|Reassignment to val| \\ 
  \code|Point3D(z = 2).z            | & 3 & & C & \verb|p1: Point3D = Point3D@2eb37eee| \\ 
  \code|p2.y = 0                    | & 4 & & D & \code|true| \\ 
  \code|p2.y == 0                   | & 5 & & E & \code|value cannot be accessed| \\ 
  \code|p1.x == Point3D().x         | & 6 & & F & \verb|p2: Point3D = Point3D@65a9e8d7| \\ 
\end{ConceptConnections}

\Subtask Vad är problemet med ovan klass om man vill använda den för att representera punkter i 3 dimensioner?

\SOLUTION

\TaskSolved \what~

\SubtaskSolved

\begin{ConceptConnections}
  \code|val p1 = new Point3D        | & 1 & ~~\Large$\leadsto$~~ &  A & \verb|p1: Point3D = Point3D@2eb37eee| \\ 
  \code|val p2 = new Point3D(y = 1) | & 2 & ~~\Large$\leadsto$~~ &  B & \verb|p2: Point3D = Point3D@65a9e8d7| \\ 
  \code|(new Point3D(z = 2)).z      | & 3 & ~~\Large$\leadsto$~~ &  C & \code|error: not found: value| \\ 
  \code|p2.y = 0                    | & 4 & ~~\Large$\leadsto$~~ &  D & \code|error: reassignment to val| \\ 
  \code|p2.y == 0                   | & 5 & ~~\Large$\leadsto$~~ &  F & \code|false| \\ 
  \code|p1.x == (new Point3D).x     | & 6 & ~~\Large$\leadsto$~~ &  E & \code|true| \\ 
\end{ConceptConnections}

\SubtaskSolved Problemet är att så som klassen \code{Point3D} är deklarerad går det inte att avläsa \code{z}-koordinaten efter att en instans konstruerats. Det vore bättre om även \code{z}-attributet är \code{val}.

\QUESTEND



\WHAT{Case-klass.}

\QUESTBEGIN

\Task \what~\TODO

\begin{Code}
case class Pt(x: Int = 0, y: Int = 0) {
  def moved(dx: Int = 0, dy: Int = 0): Pt = Pt(x + dx, y + dy)
}

class MutablePt(private var p: (Int, Int) = (0, 0)) {
  def x: Int = p._1
  def y: Int = p._2
  def move(dx: Int = 0, dy: Int = 0) = { p = (x + dx, y+ dy); this }
}
\end{Code}

\Subtask Vilken returtyp kommer kompilatorn härleda för funktionen MutablePt.move?

\Subtask Implementera en fabriksmetod \code{apply} i ett kompanjonsobjekt till klassen \code{MutablePt} som gör att du inte behöver skriva \code{new} när du skapar instanser.

\SOLUTION

\TaskSolved \what~\TODO


\QUESTEND


\WHAT{Skapa en punktklass att använda på veckans laboration.}

\QUESTBEGIN

\Task \what~
Du ska som förberedelse till laborationen skapa den oföränderliga case-klassen \code{Point} som ska beskriva en koordinat i ett kartesiskt koordinatsystem\footnote{\url{https://sv.wikipedia.org/wiki/Kartesiskt_koordinatsystem}}. Skapa kod med hjälp av en editor, t.ex. \code{atom}, i filen  \code{Point.scala} enligt följande riktlinjer:
\begin{enumerate}[noitemsep]
\item \code{Point} ska ligga i paketet \code{graphics}.

\item \code{Point} ska ha följande två publika, oföränderliga klassparametrar:
\begin{itemize}[nolistsep, noitemsep]
\item \code{x: Double} för x-koordinaten.
\item \code{y: Double} för y-koordinaten.
\end{itemize}

\item \code{Point} ska ha följande publika medlemmar (två oföränderliga attribut och två metoder):
\begin{itemize}[nolistsep, noitemsep]
\item \code{val r: Double} ska ge motsvarande polära kordinatens%
\footnote{\url{https://sv.wikipedia.org/wiki/Pol\%C3\%A4ra\_koordinater}}
 avstånd till origo.
\item \code{val theta: Double} ska ge polära kordinatens vinkel i radianer.
\item \code{def negY: Point} ska ge en ny punkt med y-koordinaten negerad.
\item \code{def +(p: Point): Point} ska ge en ny punkt vars koordinat är summan av x- respektive y-kordinaterna för denna instans och punkten \code{p}.
\end{itemize}

\item \code{Point} ska ha ett kompanjonsobjekt med en metod som konstruerar en punkt från polära koortdinater. Metoden ska ha detta huvud: \\\code{def polar(r: Double, theta: Double): Point}

\end{enumerate}

\noindent Tips vid implementation och senare användning:
\begin{itemize}
\item Du har nytta av metoderna \code{math.hypot(x, y)} och \code{math.atan2(y, x)} vid omvandling till polära koordinater.

\item Du har nytta av metoderna \code{math.cos(x)} och \code{math.sin(y)} vid omvandling från polära koordinater.

\item Attributet \code{negY} kommer att underlätta för dig när du i metoden \code{draw} i klassen \code{Turtle} ska omvandla en punkt till fönsterkoordinater där y-axeln är omvänd jämfört med kartesiska koordinater.

\item Notera att klassens attribut är av typen \code{Double} och inte \code{Int}, trots att vi senare ska använda punkten för att beskriva en diskret pixelposition. Anledningen till detta är att det kan uppstå avrundningsfel vid numeriska beräkningar. Detta blir särskilt märkbart vid upprepad räkning med små värden, t.ex. när man ritar en approximerad cirkel med många linjesegment.
\end{itemize}

\SOLUTION

\TaskSolved \what~\TODO

\QUESTEND



\AdvancedTasks %%%%%%%%%%%%%%%%%%%%%%%%%%%%%%%%%%%%%%%%%%%%%%%%%%%%%%%%%%%%%%%%%

\WHAT{Ändra attributrepresentation. Privat konstruktor}

\QUESTBEGIN

\Task \what~Kim Kodkunnig skapade för länge sedan denna punktklass som används på många ställen i befintlig kod:

\begin{Code}
class Point private (val x: Int, val y: Int)
object Point {
  def apply(x: Int = 0, y: Int = 0): Point = new Point(x, y)
  def origo = apply()
}
\end{Code}

\Subtask Vad händer om du försöker instansiera Kim Kodkunnigs klass i din egen kod direkt med nyckelordet \code{new}?

\Subtask Varför använder Kim Kodkunnig ett kompanjonsobjekt med en fabriksmetod? Vilka accessregler gäller mellan ett kompanjonsobjekt och klassen med samma namn?

\Subtask Hjälp Kim Kodkunnig att ändra attributrepresentationen så att det oföränderliga tillståndet utgörs av en 2-tupel \code{val p: (Int, Int)} i stället. Befintlig kod ska inte behöva ändras och klassen \code{Point} ska bete sig från ''utsidan'' precis som innan.

\SOLUTION

\TaskSolved \what~

\SubtaskSolved Det blir kompileringsfel eftersom konstruktorn är privat.
\begin{REPL}
scala> :paste

class Point private (val x: Int, val y: Int)
object Point {
  def apply(x: Int = 0, y: Int = 0): Point = new Point(x, y)
  def origo = apply()
}

scala> new Point(0,0)
<console>:14: error: constructor Point in class Point cannot be accessed
\end{REPL}

\SubtaskSolved
\begin{itemize}
  \item Genom att ha en privat konstruktor och bara göra indirekt instansiering via fakriksmetod är det möjligt att ändra attributrepresentation i framtiden utan att befintlig kod behöver ändras.

  \item Med en \code{apply}-metod i kompansjonsobjektet kan man instansiera genom att skriva \code{Point(1, 2)} utan new.

  \item Accessreglerna för kompanjonsobjekt är sådana att kompanjoner ser varandras privata delar.
\end{itemize}

\SubtaskSolved

\begin{Code}
class Point private (private val p: (Int, Int)) {
  def x: Int = p._1
  def y: Int = p._2
}
object Point {
  def apply(x: Int = 0, y: Int = 0): Point = new Point(x, y)
  def origo = apply()
}
\end{Code}

\QUESTEND


\subsection{\TODO värdera nedan uppgifter}


\WHAT{Instansiering med \code{new} och värdet \code{null}.}

\QUESTBEGIN

\Task  \what~  Man skapar instanser av klasser med \code{new}. Då anropas konstruktorn och plats reserveras i datorns minne för objektet. Variabler av referenstyp som inte refererar till något objekt har värdet \code{null}.

\Subtask Vad händer nedan? Vilka rader ger felmeddelande och i så fall hur lyder felmeddelandet?

\begin{REPL}
scala> class Gurka(val vikt: Int)
scala> var g: Gurka = null
scala> g.vikt
scala> g = new Gurka(42)
scala> g.vikt
scala> g = null
scala> g.vikt
\end{REPL}

\Subtask\Pen Rita minnessituationen efter raderna 2, 4, 6.

\SOLUTION


\TaskSolved \what


\SubtaskSolved  Rad 3 och 7 ger båda felmeddelandet "java.lang.NullPointerException". Detta eftersom \code{g} i båda fallen inte innehåller en referens till en \code{Gurka} utan pekar på inget -- "null".

\SubtaskSolved  \includegraphics[scale=0.6]{../img/w06-solutions/1b}


\QUESTEND




%%<AUTOEXTRACTED by mergesolu>%%      % uppgift 1




\WHAT{Klasser och instanser.}

\QUESTBEGIN

\Task  \what~

\Subtask Vad händer nedan?
\begin{REPL}
scala> :pa
class Arm(val ärTillVänster: Boolean)
class Ben(val ärTillVänster: Boolean)
class Huvud(val harHår: Boolean)
class Rymdvarelse {
  var arm1 = new Arm(true)
  var arm2 = new Arm(false)
  var ben1 = new Ben(true)
  var ben2 = new Ben(false)
  var huvud1 = new Huvud(false)
  var huvud2 = new Huvud(true)
  def ärSkallig = !huvud1.harHår && !huvud2.harHår
}
scala> val alien = new Rymdvarelse
scala> alien.ärSkallig
scala> val predator = new Rymdvarelse
scala> predator.ärSkallig
scala> predator.huvud2 = alien.huvud1
scala> predator.ärSkallig
\end{REPL}

\Subtask\Pen Rita minnessituationen efter rad 18.

\Subtask\Pen Vad händer så småningom med det ursprungliga huvud2-objektet i predator efter tilldelningen på rad 18? Går det att referera till detta objekt på något sätt?


\SOLUTION


\TaskSolved \what


\SubtaskSolved  Vi skapar två rymdvarelser, \code{alien} och \code{predator}, med två ben, två armar samt två huvuden (där det ena är skalligt och det andra har hår) vardera. Efter det är varken \code{alien} eller \code{predator} skallig eftersom båda har ett huvud med hår. Sen låter man referensen till \code{predator}s huvud med hår referera till aliens huvud utan hår. Nu är predator helt skallig.

\SubtaskSolved   \includegraphics[scale=0.7]{../img/w06-solutions/2b}

\SubtaskSolved  Eftersom det inte längre finns någon referens som pekar på det objektet kommer Garbage Collector ta hand om det och kommer förr eller senare skrivas över av något annat som behöver sparas. Nej, det går inte att komma åt.

% uppgift 3

\QUESTEND




%%<AUTOEXTRACTED by mergesolu>%%      % uppgift 2




\WHAT{Synlighet i primärkonstruktorer.}

\QUESTBEGIN

\Task  \what~  Undersök nedan vad nyckelorden \code{val} och \code{private} får för konsekvenser. Förklara vad som händer. Vilka rader ger vilka felmeddelanden?

\begin{REPL}
scala> class Gurka1(vikt: Int)
scala> new Gurka1(42).vikt
scala> class Gurka2(val vikt: Int)
scala> new Gurka2(42).vikt
scala> class Gurka3(private val vikt: Int)
scala> new Gurka3(42).vikt
scala> class Gurka4(private val vikt: Int, kompis: Gurka4){
         def kompisVikt = kompis.vikt
       }
scala> val ingenGurka: Gurka4 = null
scala> new Gurka4(42, ingenGurka).kompisVikt
scala> new Gurka4(42, new Gurka4(84, null)).kompisVikt
scala> class Gurka5(private[this] val vikt: Int, kompis: Gurka5){
         def kompisVikt = kompis.vikt
       }
scala> class Gurka6 private (vikt: Int)
scala> new Gurka6(42)
scala> :pa
class Gurka7 private (var vikt: Int)
object Gurka7 {
  def apply(vikt: Int) = {
    require(vikt >= 0, s"negativ vikt: $vikt")
    new Gurka7(vikt)
  }
}
scala> new Gurka7(-42)
scala> Gurka7(-42)
scala> val g = Gurka7(42)
scala> g.vikt
scala> g.vikt = -1
scala> g.vikt
\end{REPL}


\SOLUTION


\TaskSolved \what
 Rad 2:
\begin{REPL}
	error: value vikt is not a member of Gurka1
\end{REPL}
Detta eftersom om man varken väljer att skriva \code{val} eller \code{var} skapar inte scala någon getter eller setter (metoder för att läsa/ändra en variabel) och därför ser det ut som att vikt inte finns för kompilatorn.

Rad 4: Denna rad skapar inte en error eftersom om man skriver \code{val} innan variabeln skapas en getter automatiskt och man kan därför komma åt \code{vikt}.

Rad 6:
\begin{REPL}
	error: value vikt in class Gurka3 cannot be accessed in Gurka3
\end{REPL}
I detta fallet skapas en \code{getter} men eftersom accessnivån sätts till \code{private} vet kompilatorn att man inte får komma åt variabeln utifrån.

Rad 11:
\begin{REPL}
	java.lang.NullPointerException
\end{REPL}
Detta eftersom \code{kompis} är \code{ingenGurka} som inte pekar på något objekt och när man då försöker komma åt ett attribut från den kommer det inte funka.

Rad 12: Kommer inte generera en error eftersom när man kallar \code{kompisVikt} (som är \code{public}) försöker den komma åt \code{Gurka4(84, null).vikt}. \code{vikt} är \code{private val} vilket innebär att det har en getter och eftersom huvudobjektet också är av typen \code{Gurka4} är accessnivån tillräckligt hög.

Rad 13:
\begin{REPL}
	error: value vikt is not a member of Gurka5
\end{REPL}
När man sätter ett attribut till \code{private[this]} tillåts inte ens objekt av samma typ att komma åt variabeln och därför får man en error som säger att den inte finns.

Rad 17:
\begin{REPL}
	error: constructor Gurka6 in class Gurka6 cannot be accessed in object
\end{REPL}
Eftersom man satt klassparametrarna till \code{private} kan man inte komma åt konstruktorn och därför får man en error.

Rad 26:
\begin{REPL}
	error: constructor Gurka7 in class Gurka7 cannot be accessed in object
\end{REPL}
Samma anledning som på rad 17.

Rad 27:
\begin{REPL}
	java.lang.IllegalArgumentException: requirement failed: negativ vikt: -42
\end{REPL}
Kompanjonsobjektet har en requirement på att \code{vikt >= 0} vilket innebär att om det inte stämmer kommer man få en error av typen \code{IllegalArgumentException}.

Rad 30: Anledningen till att man kan sätta vikten till något negativt är att checken om det är negativt endast görs när man skapar \code{Gurka7} vilket innebär att i efterhand kan man ändra den till vilket värde som helst (av typen \code{Int}).


\QUESTEND






\WHAT{Egendefinierad setter kombinerat med privat konstruktor.}

\QUESTBEGIN

\Task  \what~

\Subtask Förklara vad som händer nedan. Vilka rader ger vilka felmeddelanden?
\begin{REPL}
scala> :pa
class Gurka8 private (private var _vikt: Int) {
  def vikt = _vikt
  def vikt_=(v: Int): Unit = {
    require(v >= 0, s"negativ vikt: $v")
    _vikt = v
  }
}

object Gurka8 {
  def apply(vikt: Int) = {
    require(vikt >= 0, s"negativ vikt: $vikt")
    new Gurka8(vikt)
  }
}
scala> val g = Gurka8(-42)
scala> val g = Gurka8(42)
scala> g.vikt
scala> g.vikt = 0
scala> g.vikt = -1
scala> g.vikt += 42
scala> g.vikt -= 1000
\end{REPL}

\Subtask\Pen Vad är fördelen med möjligheten att skapa egendefinierade setters?

\SOLUTION


\TaskSolved \what


\SubtaskSolved  Rad 16:
\begin{REPL}
	java.lang.IllegalArgumentException: requirement failed: negativ vikt: -42
\end{REPL}
Kompanjonsobjektet har en requirement på att \code{vikt >= 0} vilket innebär att om det inte stämmer kommer man få en error.

Rad 20:
\begin{REPL}
	java.lang.IllegalArgumentException: requirement failed: negativ vikt: -1
\end{REPL}
Eftersom settern har implementerat ett krav på att vikten måste vara större eller lika med 0 får man en error när man försöker sätta den till -1.

Rad 22:
\begin{REPL}
	java.lang.IllegalArgumentException: requirement failed: negativ vikt: -958
\end{REPL}
Eftersom 42-1000 är mindre än noll får man en error.

\SubtaskSolved  Man kan sätta egna mer specifika krav på vad som får göras med värdena så man har större koll på att inget oväntat händer.

% uppgift 5

\QUESTEND




%%<AUTOEXTRACTED by mergesolu>%%      % uppgift 4




\WHAT{En oföränderlig kvadrat med alternativ fabriksmetod.}

\QUESTBEGIN

\Task \label{task:Square} \what~

\Subtask Implementera klassen \code{Square} enligt nedan specifikation. Gör  implementationen i en kodeditor, så som \code{gedit}, och klistra in klassen i Scala REPL efter kommandot \code{:pa} (förkortning av \code{:paste}). På så sätt blir \code{object Square} ett kompanjonsobjekt till \code{class Square}.

\begin{ScalaSpec}{Square}
/** A class representing a square object with position and side. */
class Square(val x: Int, val y: Int, val side: Int) {
  /** The area of this Square */
  val area: Int = ???

  /** Creates a new Square moved to position (x + dx, y + dy) */
  def move(dx: Int, dy: Int): Square = ???

  /** Tests if this Square has equal size as that Square */
  def isEqualSizeAs(that: Square): Boolean = ???

  /** Multiplies the side with factor and rounded to nearest integer */
  def scale(factor: Double): Square = ???

  /** A string representation of this Square */
  override def toString: String = ???
}

object Square {
  /** A square placed in origin with size 1 */
  val unit: Square = ???

  /** Constructs a new Square object at (x, y) with size side */
  def apply(x: Int, y: Int, side: Int): Square = ???

  /** Constructs a new Square object at (0, 0) with side 1 */
  def apply(): Square = ???
}
\end{ScalaSpec}

\Subtask Testa din kvadrat enligt nedan. Förklara vad som händer.

\begin{REPL}
scala> val (s1, s2) = (Square(), Square(1, 10, 1))
scala> val s3 = s1.move(1,-5)
scala> s1 isEqualSizeAs s3
scala> s2 isEqualSizeAs s1
scala> s1 isEqualSizeAs Square.unit
scala> s2.scale(math.Pi) isEqualSizeAs s2
scala> s2.scale(math.Pi) isEqualSizeAs s2.scale(math.Pi)
\end{REPL}

\SOLUTION


\TaskSolved \what
 \begin{CodeSmall}
	class Square(val x: Int, val y: Int, val side: Int) {
		val area: Int = side*side

		def move(dx: Int, dy: Int): Square = new Square(x + dx, y + dy, side)

		def isEqualSizeAs(that: Square): Boolean = this.side == that.side

		def scale(factor: Double): Square = new Square(x, y, (side*factor).toInt)

		override def toString: String = s"Square(x: $x, y: $y, side: $side)"
	}

	object Square {
		val unit: Square = new Square(0, 0, 1)

		def apply(x: Int, y: Int, side: Int): Square = new Square(x, y, side)

		def apply(): Square = new Square(0, 0, 1)
	}
\end{CodeSmall}

Eftersom \code{s1}, \code{s2}, \code{s3} och \code{Square.unit} alla har en sida med längden 1 så kommer rad 3-5 returnera \code{true}. Rad 6 kommer returnera \code{false} eftersom \code{s2.scale(math.Pi)} sida är $\pi$ och \code{s2} fortfarande har sidan 1. Rad 7 kommer däremot returnera \code{true} då båda har sidan $\pi$.


\QUESTEND






\WHAT{Referenslikhet versus strukturlikhet.}

\QUESTBEGIN

\Task  \what~  Metoden \code{==} på case-klasser ger \textbf{strukturlikhet} (även kallad innehållslikhet) så att \emph{innehållet} i klassens klassparametrar jämförs om de har lika värde, medan för vanliga klasser ger metoden \code{==} \textbf{referenslikhet} där olika objekt är olika även om de har samma innehåll (om man inte överskuggar metoden \code{equals} som anropas av \code{==} vilket vi ska titta närmare på i kapitel \ref{chapter:W08}).

\begin{REPL}
scala> class GurkaRef(val vikt: Int)
scala> case class GurkaStrukt(val vikt: Int)
scala> val a = new GurkaRef(42)
scala> val b = new GurkaRef(42)
scala> val c = new GurkaStrukt(42)
scala> val d = new GurkaStrukt(42)
scala> a == b
scala> c == d
\end{REPL}

\Subtask Förklara vad som händer ovan.

\Subtask Istället för \code{==}, prova metoden \code{eq} på objekten ovan. Metoden \code{eq} ger alltid referenslikhet (även om byter ut metoden \code{equals}).

\SOLUTION


\TaskSolved \what


\SubtaskSolved  Variablerna \code{a} och \code{b} är båda objekt av en vanlig klass vilket kommer innebära att de jämförs med referenslikhet och eftersom de inte är samma objekt retunerar \code{==} \code{false}. \code{c} och \code{d} är däremot objekt av en case klass så de jämförs med strukturlikhet och eftersom de har samma vikt returnerar \code{==} \code{true}.

\SubtaskSolved  Både \code{a eq b} och \code{c eq d} ska returnera \code{false} eftersom de alla är olika objekt och det är referenslikhetsom gäller.


\QUESTEND




%%<AUTOEXTRACTED by mergesolu>%%      % uppgift 6




\WHAT{Klassen \code{Point} med case-klass.}

\QUESTBEGIN

\Task \label{task:Point} \what~

\Subtask Implementera klassen \code{Point} som en oföränderlig case-klass med heltalsattributen \code{x} och \code{y}.

\Subtask Lägg till metoden \code{distanceTo(that: Point): Double} som räknar ut avståndet till en annan punkt med hjälp av \code{math.hypot}.

\Subtask Lägg till metoden \code{distanceTo(x: Int, y: Int): Double} som räknar ut avståndet till koordinaterna x och y med hjälpa av metoden i föregående deluppgift.

\Subtask Lägg till metoden \code{move(dx: Int, dy: Int): Point} som skapar en ny punkt på translaterad position enligt delta-koordinaterna \code{dx} och {dy}.

\Subtask Lägg till ett kompanjonsobjekt med medlemmen \code{val origin} som ger en punkt i origo.

\Subtask Undersök metoderna \code{==}, \code{!=}, \code{eq} och \code{ne} och förklara vad som händer nedan:
\begin{REPL}
scala> Point(1, 2) == Point(1, 3)
scala> Point(1, 2) != Point(1, 3)
scala> Point(1, 2) == Point(1, 2)
scala> Point(1, 2) != Point(1, 2)
scala> Point.origin.move(1, 1) == Point.origin.move(1, 1)
scala> Point.origin.move(1, 1).move(1, 1) != Point(2, 2)
scala> Point(0, 0) eq Point(0, 0)
scala> Point(0, 0) ne Point(0, 0)
scala> Point.origin eq Point.origin
scala> Point.origin ne Point.origin
scala> val p1 = Point(0, 0)
scala> val p2 = p1
scala> p1 eq p2
\end{REPL}

\Subtask Vad ger \code{Point.origin eq Point.origin} för resultat om \code{origin} istället  implementeras som \code{def origin: Point = Point(0, 0)}

\Subtask\Pen Vad är det för skillnad på strukturlikhet och referenslikhet?

\SOLUTION


\TaskSolved \what


\SubtaskSolved  se e) för komplett lösning

\SubtaskSolved  se e) för komplett lösning

\SubtaskSolved  se e) för komplett lösning

\SubtaskSolved  se e) för komplett lösning

\SubtaskSolved  \begin{CodeSmall}
case class Point(x: Int, y: Int) {

	def distanceTo(that: Point): Double = math.hypot(that.x - x, that.y -y)

	def distanceTo(x: Int, y: Int): Double = distanceTo(Point(x, y))

	def move(dxdy: (Int, Int)): Point = Point(dxdy._1 + x, dxdy._2 + y)
}

object Point {
	//val origin: Point = new Point(0, 0)
	def origin: Point = Point(0, 0)
}
\end{CodeSmall}

\SubtaskSolved  \code{==} och \code{!=} kollar strukturlikhet så om två objekt innehåller samma värden kommer \code{==} returnera \code{true} och \code{!=} \code{false} och vise versa. \code{eq} och \code{ne} kollar referenslikhet så om två variabler pekar på samma objekt kommer \code{eq} returnera \code{true} och \code{ne} \code{false} och vise versa.

\SubtaskSolved  \code{false}. Detta eftersom om origin implementeras som en metod som returnerar en ny \code{Point} varje gång den kallas kommer \code{Point.origin} inte peka på samma objekt varje gång metoden kallas (\code{eq} är referenslikhet).

\SubtaskSolved  Sturkturlikhet bryr sig endast om innehållet i objekten och jämför det. Det kvittar alltså om det är samma objekt eller två olika så länge de innehåller samma värden. Referenslikhet kollar endast på om det är samma objekt variablerna pekar på och struntar fullständigt i om de innehåller samma värden.

% uppgift 8

\QUESTEND




%%<AUTOEXTRACTED by mergesolu>%%      % uppgift 7




\WHAT{NEEDS A TOPIC DESCRIPTION}

\QUESTBEGIN

\Task \label{task:PointSquare} \what~ Ändra representationen av positionen i klassen \code{Square} från deluppgift \ref{task:Square} till att vara en \code{Point} från deluppgift \ref{task:Point}.


\SOLUTION


\TaskSolved \what
 \begin{CodeSmall}
class Square(val p: Point, val side: Int) {
	val area: Int = side*side

	def move(dx: Int, dy: Int): Square = new Square(p.move(dx, dy), side)

	def isEqualSizeAs(that: Square): Boolean = this.side == that.side

	def scale(factor: Double): Square = new Square(p, (side*factor).toInt)

	override def toString: String = s"Square(p: $p, side: $side)"
}

object Square {
	val unit: Square = new Square(new Point(0, 0), 1)

	def apply(x: Int, y: Int, side: Int): Square =
		new Square(new Point(x, y), side)

	def apply(): Square = new Square(new Point(0, 0), 1)
}
\end{CodeSmall}



\QUESTEND




%%<AUTOEXTRACTED by mergesolu>%%      % uppgift 9




\WHAT{Case-klassen \code{Point} med 2-tupel.}

\QUESTBEGIN

\Task \label{task:PointTuple} \what~   I ett utvecklingsprojekt vill man ändra representationen av positionen i den gamla klassen  \\ \code{case class Point(x: Int, y: Int)} så att positionen istället i den uppdaterade klassen representeras av en 2-tupel. Man kan då vid konstruktion utnyttja att n-tupler som parameter även kan skrivas som en parameterlista med n argument, varför både \code{Point(1,2)} och \code{Point((1,2))} fungerar fint. Samtidigt vill man att befintlig kod som fortfarande använder \code{x} och \code{y} ska fungera utan ändringar.  Implementera den nya \code{Point} enligt specifikationen nedan.
\begin{ScalaSpec}{Point}
/** A 2-dimensional immutable position p in an integer coordinate system */
case class Point(p:(Int, Int)) {
  /** The x-axis position of this Point */
  val x: Int = ???

  /** The y-axis position of this Point */
  val y: Int = ???

  /** The distance to another Point that */
  def distanceTo(that: Point): Double = ???

  /** The distance to another 2-tuple that representing (x, y). */
  def distanceTo(that: (Int, Int)): Double = ???

  /** A new Point that is moved (dx, dy) */
  def move(dxdy: (Int, Int)): Point = ???
}

object Point {
  /** A Point object at position (0, 0) */
  val origin: Point = ???
}
\end{ScalaSpec}

\SOLUTION


\TaskSolved \what
  \begin{CodeSmall}
case class Point(p:(Int,Int)) {
	val x: Int = p._1

	val y: Int = p._2

	def distanceTo(that: Point): Double = math.hypot(that.x - x, that.y -y)

	def distanceTo(that: (Int, Int)): Double = distanceTo(Point(that))
	def move(dx: Int, dy: Int): Point = Point(x + dx, y + dy)
}

object Point {
	val origin: Point = new Point(0, 0)
}
\end{CodeSmall}



\QUESTEND




%%<AUTOEXTRACTED by mergesolu>%%      % uppgift 10




\WHAT{NEEDS A TOPIC DESCRIPTION}

\QUESTBEGIN

\Task  \what~\Pen Vad behöver du ändra i klassen \code{Square} från uppgift \ref{task:PointSquare} för att den ska fungera med en \code{Point} med 2-tupel från uppgift \ref{task:PointTuple}?

\SOLUTION


\TaskSolved \what
 Inget! Eftersom både \code{Point(1,2)} och \code{Point((1,2))} är okej sätt att komma åt den nya klassen så kommer det se likadant utifrån och därför behöver man inte ändra något i \code{Square}.


\QUESTEND






\WHAT{Objekt med föränderligt tillstånd \Eng{mutable state}.}

\QUESTBEGIN

\Task  \what~  Du ska implementera en modell av en hoppande groda som uppfyller följande krav:
\begin{enumerate}[nolistsep, noitemsep]
\item Varje grodobjekt ska hålla reda på var den är.
\item Varje grodobjekt ska hålla reda på hur långt grodan hoppat totalt.
\item Varje grodobjekt ska kunna beräkna hur långt det är mellan grodans nuvarande position och utgångsläget.
\item Alla grodor börjar sitt hoppande i origo.
\item En groda kan hoppa enligt två metoder:
  \begin{itemize} [nolistsep, noitemsep]
  \item relativ förflyttning enligt parametrarna \code{dx} och \code{dy},
  \item slumpmässig förflyttning $[1, 10]$ i x-led och $[1, 10]$ i y-led.
  \end{itemize}
\end{enumerate}

\Subtask Implementera klassen \code{Frog} enligt nedan specifikation och ovan krav. \\  \emph{Tips:}
  \begin{itemize} [nolistsep, noitemsep]
  \item Om namnet man vill ge ett privat föränderligt attribut ''krockar'' med ett metodnamn, är det vanligt att man börjar attributets namn med understreck, t.ex. \code{private var _x } för att på så sätt undkomma namnkonflikten.
  \item Inför en metod i taget och klistra in den nya grodan i REPL efter varje utvidgning och testa.
  \end{itemize}

\begin{ScalaSpec}{Frog}
class Frog private (initX: Int = 0, initY: Int = 0) {
  def jump(dx: Int, dy: Int): Unit = ???
  def x: Int = ???
  def y: Int = ???
  def randomJump: Unit = ???
  def distanceToStart: Double = ???
  def distanceJumped: Double = ???
  def distanceTo(that: Frog): Double = ???
}
object Frog {
  def spawn(): Frog = ???
}
\end{ScalaSpec}

\Subtask Skriv ett testhuvudprogram som kontrollerar så att alla krav är uppfyllda och att alla metoder fungerar som de ska.

\Subtask\Pen Vad kallas en metod som enbart returnerar värdet av ett privat attribut?

\Subtask\Pen Hur kan man från en metods signatur få en ledtråd om att ett objekt har föränderligt tillstånd \Eng{mutable state}?

\Subtask Inför setters för attributen som håller reda på x- och y-postitionen. Förändringar av positionen i x- eller y-led ska räknas som ett hopp och alltså registreras i det attribut som håller reda på det ackumulerade hoppavståndet.

\Subtask Simulera ett massivt grodhoppande med krockdetektering genom att skapa 100 grodor som till att börja med är placerade på x-axeln med avståndet $8$ längdenheter mellan sig. Låt grodorna i en \code{while}-sats hoppa slumpmässigt tills någon groda befinner sig närmare än $0.5$ längdenheter som är definitionen på att de har krockat. Räkna hur många looprundor som behövs innan något grodpar krockar och skriv ut antalet. \\ \emph{Tips:} Börja med pseudokod på papper. Använd en grodvektor.

\clearpage

\ExtraTasks %%%%%%%%%%%%%%%%%%%

\SOLUTION


\TaskSolved \what


\SubtaskSolved  \begin{CodeSmall}
class Frog private (initX: Int = 0, initY: Int = 0) {
	private var _x: Int = initX
	private var _y: Int = initY
	private var _distanceJumped: Double = 0

	def jump(dx: Int, dy: Int): Unit = {
		_x += dx
		_y += dy
		_distanceJumped += Math.hypot(dx, dy)
	}

	def x: Int = _x
	def y: Int = _y

	def randomJump: Unit = {
		val r = scala.util.Random
		val xtmp = r.nextInt(10)+1
		val ytmp = r.nextInt(10)+1
		_x += xtmp
		_y += ytmp
		_distanceJumped += Math.hypot(xtmp, ytmp)
	}

	def distanceToStart: Double = Math.hypot(_x,_y)
	def distanceJumped: Double = _distanceJumped
	def distanceTo(that: Frog): Double = Math.hypot(_x - that.x, _y - that.y)
}

object Frog {
	def spawn(): Frog = new Frog()
}
\end{CodeSmall}

\SubtaskSolved  \begin{CodeSmall}
val f1 = Frog.spawn()
//test requirement 1 and 4
assert(f1.x == 0 && f1.y == 0, "Either x or y isn't 0")

f1.jump(4,3)
//test requirement 1 and 5
assert(f1.x == 4 && f1.y == 3, "Either x isn't 4 or y isn't 3")

f1.jump(4,3)
//test requirement 2
var text = "distanceJumped is " + f1.distanceJumped + ". Should be 10"
assert(f1.distanceJumped == 10, text)

f1.jump(-4,-3)
//test requirement 3
text = "distanceToStart is " + f1.distanceJumped + ". Should be 5"
assert(f1.distanceToStart == 5, text)

var f2 = Frog.spawn()
for (x <- 1 to 1000) {
	f2.randomJump
	//test requirement 5
	text = "Either x or y isn't in [1,10]. x:" + f2.x + ", y: " + f2.y
	assert(f2.x > 0 && f2.x <= 10 && f2.y > 0 && f2.y <= 10, text)
	f2 = Frog.spawn()
}

val f3 = Frog.spawn()
f3.jump(1,1)
val f4 = Frog.spawn()
f4.jump(4,5)
// Test distanceT()
text = "distanceTo is " + f3.distanceTo(f4) + ". Should be 5"
assert(f3.distanceTo(f4) == 5, text)
\end{CodeSmall}

\SubtaskSolved  Getter

\SubtaskSolved  Om metoden har parametrar och retur-typen \code{Unit}. Det betyder troligen att parametrarna ändrar något istället för att skapa något nytt.

\SubtaskSolved  \begin{CodeSmall}
class Frog private (initX: Int = 0, initY: Int = 0) {
	private var _x: Int = initX
	private var _y: Int = initY
	private var _distanceJumped: Double = 0

	def jump(dx: Int, dy: Int): Unit = {
		_x += dx
		_y += dy
		_distanceJumped += Math.hypot(dx, dy)
	}

	def x: Int = _x
	def y: Int = _y

	def x_= (newX: Int): Unit = {
		_distanceJumped += Math.abs(_x - newX)
		_x = newX
	}
	def y_= (newY: Int): Unit = {
		_distanceJumped += Math.abs(_y - newY)
		_y = newY
	}

	def randomJump: Unit = {
		val r = scala.util.Random
		val xtmp = r.nextInt(10)+1
		val ytmp = r.nextInt(10)+1
		_x += xtmp
		_y += ytmp
		_distanceJumped += Math.hypot(xtmp, ytmp)
	}

	def distanceToStart: Double = Math.hypot(_x,_y)
	def distanceJumped: Double = _distanceJumped
	def distanceTo(that: Frog): Double = Math.hypot(_x - that.x, _y - that.y)
}

object Frog {
	def spawn(): Frog = new Frog()
}
\end{CodeSmall}

\SubtaskSolved  \begin{CodeSmall}
var noCollision = true
var counter = 0
val numberOfFrogs = 100
val distanceBetweenFrogs = 8
val frogArray = Array.fill(numberOfFrogs){Frog.spawn()}
(0 until numberOfFrogs).foreach(i => frogArray(i).x(i*distanceBetweenFrogs))
while (noCollision) {
	frogArray.foreach(frog => frog.randomJump)
	for (frog <- frogArray) {
		for (frog2 <- frogArray) {
			if (frog != frog2 && frog.distanceTo(frog2) < 0.5) {
				noCollision = false
			}
		}
	}
	counter += 1
}
print(counter)
\end{CodeSmall}


\clearpage

\ExtraTasks %%%%%%%%%%%%


\QUESTEND




%%<AUTOEXTRACTED by mergesolu>%%      % uppgift 11




\WHAT{En kvadratklass med föränderligt tillstånd \Eng{mutable state}.}

\QUESTBEGIN

\Task  \what~  Webbshoppen UberSquare säljer flyttbara kvadrater. I affärsmodellen ingår att ta betalt per förflyttning. Du ska hjälpa UberSquare med att utveckla en enkel systemprototyp.

\Subtask Implementera \code{Square} enligt nedan specifikation, under uppfyllandet av följande krav:

\begin{enumerate}[nolistsep, noitemsep]
\item Till skillnad från uppgift \ref{task:Square} ska du nu göra en kvadrat med föränderligt tillstånd \Eng{mutable state}. I stället för att vid förflyttning returnera ett nytt kvadratobjekt, returneras \code{Unit} i samband med att privata attribut uppdateras.
\item Du ska införa funktionalitet som räknar antalet förflyttningar som gjorts för varje kvadrat som skapats och även räkna ut det totala antalet förflyttningar som någonsin gjorts.
\item Varje gång förflyttning sker adderas en kostnad till den ackumulerade kostnaden för respektive kvadrat. Kostnaden för varje förflyttning är avståndet till ursprungsläget multiplicerat med storleken på kvadraten.
\end{enumerate}

\begin{ScalaSpec}{Square}
/** A mutable and expensive Square. */
class Square private (val initX: Int, val initY: Int, val initSide: Int) {

  private var nMoves = 0;
  private var sumCost = 0.0;
  private var _x = initX;
  private var _y = initY;
  private var _side = initSide;

  private def addCost: Unit = {
   sumCost += ???
  }

  /** The current position on the x axis */
  def x: Int = ???

  /** The current position on the y axis */
  def y: Int = ???

  /** The size of this Square */
  def side = ???

  /** Scales the side of this square and rounds it to nearest integer */
  def scale(factor: Double): Unit = ???

  /** Moves this square to position (x + xd, y + dy) */
  def move(dx: Int, dy: Int): Unit = ???

  /** Moves this square to position (x, y) */
  def moveTo(x: Int, y: Int): Unit = ???

  /** The accumulated cost of this Square */
  def cost: Double = ???

  /** Reset the cost of this Square */
  def pay: Unit = ???

  /** A string representation of this Square */
  override def toString: String =
    s"Square[($x, $y), side: $side, #moves: $nMoves times, cost: $sumCost]"
}

object Square {
  private var created = Vector[Square]()

  /** Constructs a new Square object at (x, y) with size side */
  def apply(x: Int, y: Int, side: Int): Square = {
    require(side >= 0, s"side must be positive: $side")
    ???
  }

  /** Constructs a new Square object at (0, 0) with side 1 */
  def apply(): Square = apply(0, 0, 1)

  /** The total number of moves that have been made for all squares. */
  def totalNumberOfMoves: Int = ???

  /** The total cost of all squares. */
  def totalCost: Double = ???
}
\end{ScalaSpec}

\Subtask Testa din kvadratprototyp i REPL enligt nedan:
\begin{REPL}
scala> val xs = Vector.fill(10)(Square())
scala> xs.foreach(_.move(2,3))
scala> xs.foreach(_.scale(2.9))
scala> val (m, c) = (Square.totalNumberOfMoves, Square.totalCost)
m: Int = 10
c: Double = 36.055512754639885
\end{REPL}


\clearpage

\AdvancedTasks %%%%%%%%%%%%%%%%%


\SOLUTION


\TaskSolved \what


\vspace{1em} %tweak pagination

\begin{CodeSmall}
/** A mutable and expensive Square. */
class Square private (val initX: Int, val initY: Int, val initSide: Int) {

  private var nMoves = 0;
  private var sumCost = 0.0;
  private var _x = initX;
  private var _y = initY;
  private var _side = initSide;

  private def addCost: Unit = {
   sumCost += math.hypot(x - initX, y - initY) * side
  }

  /** The current position on the x axis */
  def x: Int = _x

  /** The current position on the y axis */
  def y: Int = _y

  /** The size of the side */
  def side = _side

  /** Scales the size of this square and rounds it to nearest integer */
  def scale(factor: Double): Unit = { _side = (_side * factor).round.toInt }

  /** Moves this square to position (x + xd, y + dy) */
  def move(dx: Int, dy: Int): Unit = {
    _x += dx; _y += dy;
    nMoves += 1
    addCost
  }

  /** Moves this square to position (x, y) */
  def moveTo(x: Int, y: Int): Unit = {
    _x = x; _y = y;
    nMoves += 1
    addCost
  }

  /** The accumulated cost of this Square */
  def cost: Double = sumCost

  /** Reset the cost of this Square */
  def pay: Unit = {sumCost = 0}

  /** A string representation of this Square */
  override def toString: String =
    s"Square[($x, $y), side: $side, #moves: $nMoves times, cost: $sumCost]"
}

object Square {
  private var created = Vector[Square]()

  /** Constructs a new Square object at (x, y) with size side */
  def apply(x: Int, y: Int, side: Int): Square = {
    require(side >= 0, s"side must be positive: $side")
    val sq = (new Square(x, y, side))
    created :+= sq
    sq
  }

  /** Constructs a new Square object at (0, 0) with side 1 */
  def apply(): Square = apply(0, 0, 1)

  /** The total number of moves that have been made for all squares. */
  def totalNumberOfMoves: Int = created.map(_.nMoves).sum

  /** The total cost of all squares. */
  def totalCost: Double = created.map(_.cost).sum
}
\end{CodeSmall}

\AdvancedTasks %%%%%%%%%

\TODO
\QUESTEND






\WHAT{Hjälpkonstruktor.}

\QUESTBEGIN

\Task \label{task:aux-constructor} \what~   I uppgift \ref{task:Square} erbjöds ett alternativt sätt att skapa \code{Square} med en extra fabriksmetod med namnet \code{apply} i kompanjonsobjektet. Ett annat sätt att göras detta på, som i Scala är mindre vanligt (men i Java är desto vanligare), är att definiera flera konstruktorer innuti klassen. I Scala kallas en sådan extra konstruktor för \textbf{hjälpkonstruktor} \Eng{auxiliary constructor}.

En hjälpkonstruktor skapar man i Scala genom att definiera en metod som har det speciella namnet \code{this}, alltså en deklaration \code{def this(...) = ...} Hjälponstruktorer måste börja med att anropa en annan konstruktor, antingen den primära konstruktorn eller en tidigare definierad  hjälpkonstruktor.

\Subtask Läs mer om hjälpkonstruktorer här: \\ \href{http://www.artima.com/pins1ed/functional-objects.html#6.7}{www.artima.com/pins1ed/functional-objects.html\#6.7}

\Subtask Hitta på en egen uppgift med hjälpkonstruktorer, baserat på någon av klasserna i tidigare övningar.


%\Task \TODO \\ \code{class Rational private (numerator: BigInt, denominator: BigInt)} \\
%Inspirerat av Rational i pins1ed med GCD\SOLUTION


\QUESTEND


%!TEX encoding = UTF-8 Unicode
%!TEX root = ../exercises.tex

\ifPreSolution



\Exercise{\ExeWeekSIX}\label{exe:W06}

\begin{Goals}
\item Kunna skapa och använda \code{match}-uttryck med konstanta värden, garder och mönstermatchning med case-klasser.
\item Kunna skapa och använda case-objekt för matchningar på uppräknade värden.
\item Kunna hantera saknade värden med hjälp av typen \code{Option} och mönstermatchning på \code{Some} och \code{None}.
\item Kunna fånga undantag med \code{scala.util.Try}.
\item Känna till \code{try}, \code{catch} och \code{throw}.
%\item Känna till \jcode{switch}-satser i Java.
\item Känna till nyckelordet \code{sealed} och förstå nyttan med förseglade typer.
%\item Känna till relationen mellan \code{hashCode} och \code{equals}.
%\item Kunna skapa partiella funktioner med case-uttryck.
%\item Känna till betydelsen av små och stora begynnelsebokstäver i case-grenar i en matchning, samt förstå hur namn binds till värden in en case-gren.
%\item Kunna använda \code{flatMap} tillsammans med \code{Option} och \code{Try}.
%\item Känna till skillnaderna mellan \code{try}-\code{catch} i Scala och java.
%\item Känna till att metoden \code{unapply} används vid mönstermatchning.
%\item Kunna implementera \code{equals} med hjälp av en \code{match}-sats, som fungerar för finala klasser utan arv.
%\item Känna till \code{null}.
\end{Goals}

\begin{Preparations}
\item \StudyTheory{06}
\end{Preparations}

\BasicTasks %%%%%%%%%%%%%%%%

\else



\ExerciseSolution{\ExeWeekSIX}

\BasicTasks %%%%%%%%%%%

\fi




\WHAT{Matcha på konstanta värden.}

\QUESTBEGIN

\Task \label{task:vegomatch} \what~   % I Scala finns ingen \jcode{switch}-sats. I stället har Scala ett \code{match}-uttryck som är mer kraftfullt. Dock saknar Scala nyckelordet \jcode{break} och Scalas \code{match}-uttryck kan inte ''falla igenom'' som skedde i uppgift \ref{task:switch}\ref{subtask:break}.

\Subtask \label{subtask:vegomatch} Skriv nedan program med en kodeditor och spara i filen \texttt{Match.scala}. Kompilera och kör och och ge som argument din favoritgrönsak. Vad händer? Förklara hur ett \code{match}-uttryck fungerar.

\scalainputlisting[numbers=left,basicstyle=\ttfamily\fontsize{11}{12}\selectfont]{examples/Match.scala}

\Subtask Vad blir det för felmeddelande om du tar bort case-grenen för defaultvärden och indata väljs så att inga case-grenar matchar? Är det ett exekveringsfel eller ett kompileringsfel?

% \Subtask Beskriv några skillnader i syntax och semantik mellan Javas flervalssats \jcode{switch} och Scalas flervalsuttryck \code{match}.



\SOLUTION


\TaskSolved \what


\SubtaskSolved  %Svaret blir identiskt mot föregående uppgiften i Java.\\
Scalas \code{match}-uttryck jämför stegvis värdet med varje \code{case} för att sedan returnera ett värde tillhörande motsvarande \code{case}.

\SubtaskSolved  \begin{REPL}
scala.MatchError 
\end{REPL}
Exekveringsfel, uppstår av en viss input under körningen.

% \SubtaskSolved  Scalas \code{match} ersätter kolonet (:) i \jcode{switch} med Scalas högerpil (=>).\\
% \code{match} returnerar ett värde till skillnad från \jcode{switch} som inte returnerar något.\\
% \code{match} kan inte $"$falla igenom$"$ så ett \jcode{break} efter varje \jcode{case} är inte nödvändigt.\\
% Till skillnad från \jcode{switch}-satsen kastar \code{match} ett \code{MatchError} om ingen matchning skulle ske.



\QUESTEND






\WHAT{Gard i case-grenar.}

\QUESTBEGIN

\Task  \what~  Med hjälp en gard \Eng{guard} i en case-gren kan man begränsa med ett villkor om grenen ska väljas.

Utgå från koden i uppgift \ref{task:vegomatch}\ref{subtask:vegomatch} och byt ut case-grenen för \code{'g'}-matchning till nedan variant med en gard med nyckelordet \code{if} (notera att det inte behövs parenteser runt villkoret):
\begin{Code}
    case 'g' if math.random() > 0.5 => "gurka är gott ibland..."
\end{Code}
Kompilera om och kör programmet upprepade gånger med olika indata tills alla grenar i \code{match}-uttrycket har exekverats. Förklara vad som händer.

\SOLUTION


\TaskSolved \what

Garden som införts vid \code{case 'g'} slumpar fram ett tal mellan 0 och 1 och om talet inte är större än $0.5$ så blir det ingen matchning med \code{case 'g'} och programmet testar vidare tills default-caset.\\
Gardens krav måste uppfyllas för att det ska matcha som vanligt.



\QUESTEND






\WHAT{Mönstermatcha på attributen i case-klasser.}

\QUESTBEGIN

%\Task \label{task:match-caseclass} \what~   Scalas \code{match}-uttryck är extra kraftfulla om de används tillsammans med \code{case}-klasser: då kan attribut extraheras automatiskt och bindas till lokala variabler direkt i case-grenen som nedan exempel visar (notera att \code{v} och \code{rutten} inte behöver deklareras explicit). Detta kallas för \textbf{mönstermatchning}.

\Task \label{task:match-caseclass} \what~   Scalas \code{match}-uttryck är extra kraftfulla om de används tillsammans med \code{case}-klasser: då kan attribut extraheras automatiskt och bindas till lokala variabler direkt i case-grenen som nedan exempel visar (notera att \code{v} och \code{rutten} inte behöver deklareras explicit). Detta kallas för \textbf{mönstermatchning}. 
Vad skrivs ut nedan? Varför? Prova att byta namn på \code{v} och \code{rutten}.
%\Subtask \label{subtask:autobinding-match} Vad skrivs ut nedan? Varför? Prova att byta namn på \code{v} och \code{rutten}.
\begin{REPL}
scala> case class Gurka(vikt: Int, ärRutten: Boolean)
scala> val g = Gurka(100, true)
scala> g match { case Gurka(v,rutten) => println("G" + v + rutten) }
\end{REPL}

%\TODO %Tab två gånger fungerar inte i scala3-repl, issue #536
%\Subtask Skriv sedan nedan i REPL och tryck TAB två gånger efter punkten. Vad har \code{unapply}-metoden för resultattyp?
%\begin{REPL}
%scala> Gurka.unapply   // Tryck TAB två gånger
%\end{REPL}
%\begin{Background}
%Case-klasser får av kompilatorn automatiskt ett kompanjonsobjekt \Eng{companion object}, i detta fallet \code{object Gurka}. Det objektet får av kompilatorn automatiskt en \code{unapply}-metod. Det är \code{unapply} som anropas ''under huven'' när case-klassernas attribut extraheras vid mönstermatchning, men detta sker alltså automatiskt och man behöver inte explicit nyttja \code{unapply} om man inte själv vill implementera s.k. extraherare \Eng{extractors}; om du är nyfiken på detta, se fördjupningsuppgift \ref{task:extractor}.
%\end{Background}

%\Subtask Anropa \code{unapply}-metoden enligt nedan. Vad blir resultatet?
%\begin{REPL}
%scala> Gurka.unapply(g)
%\end{REPL}
%Vi ska i senare uppgifter undersöka hur typerna \code{Option} och \code{Some} fungerar och hur man kan ha nytta av dessa i andra sammanhang.

% \Subtask Spara programmet nedan i filen \texttt{vegomatch.scala} och kompilera och kör med \code{scala vegomatch.Main 1000} i terminalen. Förklara hur predikatet \code{ärÄtvärd} fungerar.
% \scalainputlisting[numbers=left,basicstyle=\ttfamily\fontsize{11}{12}\selectfont]{examples/vegomatch.scala}
%

\SOLUTION


\TaskSolved \what \\
G100true. Vid byte av plats: Gtrue100.\\
\code{match} testar om kompanjonsobjektet \code{Gurka} är av typen \code{Gurka} med två parametervärden. De angivna parametrarna tilldelas namn, \code{vikt} får namnet \code{v} och \code{ärRutten} namnet \code{rutten} och skrivs sedan ut. Byts namnen dessa ges skrivs de ut i den omvända ordningen.

%\TODO % TAB+TAB fungerar inte i scala3-repl så svaret till uppgiften är felaktig
%\SubtaskSolved  \code{Option[(Int, Boolean)]}

%\SubtaskSolved	\code{Gurka(100, true)}

% \SubtaskSolved  \code{ärÄtvärd} testar om \code{Grönsak g} är av typen \code{Gurka(v, rutten)} eller \code{Tomat}. Dessa har sedan garder.\\ \code{Gurka} måste ha \code{vikt} över 100 och \code{ärRutten} vara \code{false} för att \code{case Gurka} ska returnera \code{true}.\\
% \code{Tomat} måste ha \code{vikt} över 50 och \code{ärRutten} vara \code{false} för att \code{case Tomat} ska returnera \code{true}.\\
% Matchas inte \code{Grönsak g} med någon av dessa returneras default-värdet \code{false}.



\QUESTEND







\WHAT{Matcha på case-objekt och nyttan med \code{sealed}.}

\QUESTBEGIN

\Task	\label{task:match-sealedtrait} \what~	Skriv nedan kodrader i en REPL en för en. Notera nyckelordet \code{sealed} som används för att försegla en typ. En \textbf{förseglad typ} måste ha alla sina subtyper i en och samma kodfil.
\begin{REPL}
scala> sealed trait Färg
scala> case object Spader extends Färg
\end{REPL}
\Subtask Hur lyder felmeddelandet och varför sker det? Är det ett kompileringsfel eller ett körtidsfel?

\Subtask  \label{subtask:match-sealedtrait-caseobject}
Skapa nu nedan kod i en editor och klistra in i REPL.
\begin{Code}
object Kortlek:
  sealed trait Färg
  object Färg:
      val values = Vector(Spader, Hjärter, Ruter, Klöver)
  case object Spader extends Färg
  case object Hjärter extends Färg
  case object Ruter extends Färg
  case object Klöver extends Färg
\end{Code}

\Subtask \label{subtask:match-sealedtrait-function}
Skapa en funktion \code{def parafärg(f: Färg): Färg} i en editor, som med hjälp av ett match-uttryck returnerar parallellfärgen till en färg. Parallellfärgen till \code{Hjärter} är \code{Ruter} och vice versa, medan parallellfärgen till \code{Klöver} är \code{Spader} och vice versa. Klistra in funktionen i REPL. Passa även på att skriva en \code{import}-sats för det yttre objektet \textbf{Kortlek}, så medlemmarna av objektet kan nås enkelt.
\begin{REPL}
scala> parafärg(Spader)
scala> val xs = Vector.fill(5)(Färg.values((math.random() * 4).toInt))
scala> xs.map(parafärg)
\end{REPL}

\Subtask \label{subtask:match-forgetcase}
Vi ska nu undersöka vad som händer om man glömmer en av case-grenarna i matchningen i \code{parafärg}. ''Glöm'' alltså avsiktligt en av case-grenarna och klistra in den nya \code{parafärg} med den ofullständiga matchningen. Hur lyder varningen? Kommer varningen vid körtid eller vid kompilering?

\Subtask Anropa \code{parafärg} med den ''glömda'' färgen. Hur lyder felmeddelandet? Är det ett kompileringsfel eller ett körtidsfel?

\Subtask Förklara vad nyckelordet \code{sealed} innebär och vilken nytta man kan ha av att \textbf{försegla} en supertyp.


\SOLUTION


\TaskSolved \what

\SubtaskSolved
\begin{REPL}
Cannot extend sealed trait Färg in a different source file
\end{REPL}
Felmeddelandet fås av att REPL:en behandlar varje inmatning individuellt och tillåter därför inte att subtypen \code{Spader} ärver från \Eng{extends} supertypen \code{Färg} eftersom denna var förseglad \Eng{sealed}. Mer om detta senare i kursen...

\SubtaskSolved
-

\SubtaskSolved
Förusatt att \code{import Kortlek._} har skrivits...
\begin{Code}
def parafärg(f: Färg): Färg = f match
  case Spader  => Klöver
  case Hjärter => Ruter
  case Ruter   => Hjärter
  case Klöver  => Spader
\end{Code}

\SubtaskSolved
\begin{REPL}
<console>:17: warning: match may not be exhaustive.
It would fail on the following input: Ruter
\end{REPL}
Varningen kommer redan vid kompilering.

\SubtaskSolved
\begin{REPL}
scala.MatchError: Ruter (of class Ruter)
  at .parafärg(<console>:17)
\end{REPL}
Detta är ett körtidsfel.

\SubtaskSolved  Om en klass är \code{sealed} innebär det att om ett element ska matchas och är en subtyp av denna klass så ger Scala varning redan vid kompilering om det finns en risk för ett \code{MatchError}, alltså om \code{match}-uttrycket inte är uttömmande och det finns fall som inte täcks av ett \code{case}.\\
En förseglad supertyp innebär att programmeraren redan vid kompileringstid får en varning om ett fall inte täcks och i sånt fall vilket av undertyperna, liksom annan hjälp av kompilatorn. Detta kräver dock att alla subtyperna delar samma fil som den förseglade klassen.



\QUESTEND


\WHAT{Mönstermatcha enumeration.}

\QUESTBEGIN
%\TODO %Se gärna över denna frågan samt facit.
\Task	\what~ Vi ska nu undersöka och jämföra skillnad mellan nyckelorden \code{enum} och \code{sealed trait}. Skriv nedan kod i en REPL.
\begin{Code}
enum Färg:
  case Spader, Hjärter, Ruter, Klöver
\end{Code}

\Subtask Skapa med hjälp av en editor igen en funktion \code{def parafärg(f: Färg): Färg}, nästintill likadan som den som vi skapade i deluppgift \ref{task:match-sealedtrait}\ref{subtask:match-sealedtrait-function}. Funktionen ska återigen utnyttja match-uttryck för att returnera paralellfärgen till argumentet som ges. Tänk på att denna gången är \code{Färg} inget \code{sealed trait}, utan istället en enumeration (\code{enum}). Klistra in funktionen i REPL.
\begin{REPL}
scala> parafärg(Färg.Ruter)
scala> val xs = Vector.fill(5)(Färg.values((math.random() * 4).toInt))
scala> xs.map(parafärg)
\end{REPL}


\Subtask
Fundera på skillnader och likheter mellan att utnyttja \code{sealed trait} ihop med \code{case}-objekt gentemot att använda sig av \code{enum} vid mönstermatchning.


\SOLUTION


\TaskSolved \what
\SubtaskSolved
\begin{Code}
def parafärg(f: Färg): Färg = f match
  case Färg.Spader  => Färg.Klöver
  case Färg.Hjärter => Färg.Ruter
  case Färg.Ruter   => Färg.Hjärter
  case Färg.Klöver  => Färg.Spader
\end{Code}
Likt uppgift \ref{task:match-sealedtrait}\ref{subtask:match-sealedtrait-function} så kan även här en \code{import}-sats skrivas för att nå medlemmarna i \code{Färg} utan punktnotation.
Det är dock inte alltid fördelaktigt att importera medlemmar till den globala namnrymden, då det kan förekomma namnkrockar. Anta ett exempel där vi jobbar på ett program med grafiskt användargränssnitt där vi har en färg \code{Red} definerad.
Anta också att vi nu till vårt program vill importera ytterligare en röd färg för kulörerna hjärter och ruter, denna också namngiven \code{Red}. I detta scenario hade det uppstått en namnkrock då \code{Red} redan är definerad så importeringen hade ej kunnat ske.

\SubtaskSolved
Vid mönstermatchning så fungerar \code{sealed trait} ihop med \code{case}-objekt i praktiken likadant som att använda sig av \code{enum}.
Vi såg att i deluppgift \ref{task:match-sealedtrait}\ref{subtask:match-forgetcase} så varnade REPL redan vid kompilering att denna matchning inte var uttömmande \Eng{exhaustive}. Detta gäller även vid användning av \code{enum}.

\QUESTEND



\WHAT{Betydelsen av små och stora begynnelsebokstäver vid matchning.}

\QUESTBEGIN

\Task  \what~  För att åstadkomma att namn kan bindas till variabler vid matchning utan att de behöver deklareras i förväg (som vi såg i uppgift \ref{task:match-caseclass}) så har identifierare med liten begynnelsebokstav fått speciell betydelse: den tolkas av kompilatorn som att du vill att en variabel  binds till ett värde vid matchningen. En identifierare med stor begynnelsebokstav tolkas däremot som ett konstant värde (t.ex. ett case-objekt eller ett case-klass-mönster).

\Subtask \emph{En case-gren som fångar allt}. En case-gren med en identifierare med liten begynnelsebokstav som saknar gard kommer att matcha allt. Prova nedan i REPL, men försök lista ut i förväg vad som kommer att hända. Vad händer?
\begin{REPL}
scala> val x = "urka"
scala> x match
         case str if str.startsWith("g") => println("kanske gurka")
         case vadsomhelst => println("ej gurka: " + vadsomhelst)
scala> val g = "gurka"
scala> g match
         case str if str.startsWith("g") => println("kanske gurka")
         case vadsomhelst => println("ej gurka: " + vadsomhelst)
\end{REPL}

\Subtask \emph{Fallgrop med små begynnelsbokstäver.} Innan du provar nedan i REPL, försök gissa vad som kommer att hända. Vad händer? Hur lyder varningarna och vad innebär de?
\begin{REPL}
scala> val any: Any = "varken tomat eller gurka"
scala> case object Gurka
scala> case object tomat
scala> any match
         case Gurka => println("gurka")
         case tomat => println("tomat")
         case _ => println("allt annat")
\end{REPL}

\Subtask \emph{Använd backticks för att tvinga fram match på konstant värde.} Det finns en utväg om man inte vill att kompilatorn ska skapa en ny lokal variabel: använd specialtecknet \emph{backtick}, som skrivs \`{} och kräver speciella tangentbordstryck.\footnote{Fråga någon om du inte hittar hur man gör backtick \`{} på ditt tangentbord.}  Gör om föregående uppgift men omgärda nu identifieraren \code{tomat} i tomat-case-grenen med backticks, så här: \code{  case `tomat` => ...}



\SOLUTION


\TaskSolved \what


\SubtaskSolved  Både \code{str} och \code{vadsomhelst} matchar med inputen, oavsett vad denna är på grund av att de har en liten begynnelsebokstav.\\
 \code{str} har dock en gard att strängen måste börja med $g$ vilket gör så endast \code{val g = "gurka"} matchar med denna. \code{val x = "urka"} plockas dock upp av \code{vadsomhelst} som är utan gard.

\SubtaskSolved
\begin{REPL}
<console>:16: warning: patterns after a variable pattern cannot match (SLS 8.1
.1)
\end{REPL}
och
\begin{REPL}
<console>:17: warning: unreachable code due to variable patter 'tomat' on line
16
\end{REPL}
Trots att en klass \code{tomat} existerar så tolkar Scalas \code{match} den som en \code{case}-gren som fångar allt på grund av en liten begynnelsebokstav. Detta gör så alla objekt som inte är av typen \code{Gurka} kommer ge utskriften \textit{tomat} och att sista caset inte kan nås.

\SubtaskSolved
\begin{Code}
case `tomat` => println("tomat")
\end{Code}



\QUESTEND





\WHAT{Matcha på innehåll i en Vector.}

\QUESTBEGIN

\Task \what ~ Kör nedan i REPL. Vad skrivs ut? Förklara vad som händer.
\begin{REPL}
scala> val xss = Vector(Vector("hej"),Vector("på", "dej"),Vector("4","x","2"))
scala> xss.map( _ match
  case Vector() => "tom"
  case Vector(a) => a.reverse
  case Vector(_, b) => b.reverse
  case Seq(a, "x", b) => a + b
  case _ => "ANNARS DETTA"
  ).foreach(println)
\end{REPL}


\SOLUTION

\TaskSolved \what

\begin{REPL}
jeh
jed
42
\end{REPL}
För varje element i \code{xss} görs en matching som resulterar i en sträng. Vad som händer i varje gren förklaras nedan.
\begin{enumerate}
  \item Första match-grenen aktiveras aldrig eftersom \code{xss} ej innehåller någon tom vektor.
  \item Andra grenen passar med \code{Vector("hej")} och variablen \code{a} binds till \code{"hej"}.
  \item Tredje grenen matchar \code{Vector("på", "dej")} där första värdet binds inte till någon variabel eftersom understreck finns på motsvarande plats, medan andra värdet binds till \code{b}.
  \item Fjärde grenen matchar en sekvens med tre värden där mittenvärdet är \code{"x"}. Den sista grenen aktiveras inte i detta exempel men hade matchat allt som inte fångas av tidigare grenar.
\end{enumerate}

\QUESTEND




\WHAT{Använda \code{Option} och matcha på värden som kanske saknas.}

\QUESTBEGIN

\Task  \what~  Man behöver ofta skriva kod för att hantera värden som eventuellt saknas, t.ex. saknade telefonnummer i en persondatabas. Denna situation är så pass vanlig att många språk har speciellt stöd för saknande värden.

I Java\footnote{Scala har också \code{null} men det behövs bara vid samverkan med Java-kod.} används värdet \code{null} för att indikera att en referens saknar värde. Man får då komma ihåg att testa om värdet saknas varje gång sådana värden ska behandlas, t.ex. med \code+if (ref != null) { ...} else { ... }+. Ett annat vanligt trick är att låta \code{-1} indikera saknade positiva heltal, till exempel saknade index, som får behandlas med \code+if (i != -1) { ...} else { ... }+.

I Scala finns en speciell typ \code{Option} som möjliggör smidig och typsäker hantering av saknade värden. Om ett kanske saknat värde packas in i en \code{Option} \Eng{wrapped in an Option}, finns det i en speciell slags samling som bara kan innehålla \emph{inget} eller \emph{något} värde, och alltså har antingen storleken \code{0} eller \code{1}.

\Subtask Förklara vad som händer nedan.
\begin{REPL}
scala> var kanske: Option[Int] = None
scala> kanske.size
scala> kanske = Some(42)
scala> kanske.size
scala> kanske.isEmpty
scala> kanske.isDefined
scala> def ökaOmFinns(opt: Option[Int]): Option[Int] = opt match
         case Some(i) => Some(i + 1)
         case None    => None
scala> val annanKanske = ökaOmFinns(kanske)
scala> def öka(i: Int) = i + 1
scala> val merKanske = kanske.map(öka)
\end{REPL}

\Subtask Mönstermatchingen ovan är minst lika knölig som en \code{if}-sats, men tack vare att en \code{Option} är en slags (liten) samling finns det smidigare sätt. Förklara vad som händer nedan.
\begin{REPL}
val meningen = Some(42)
val ejMeningen = Option.empty[Int]
meningen.map(_ + 1)
ejMeningen.map(_ + 1)
ejMeningen.map(_ + 1).orElse(Some("saknas")).foreach(println)
meningen.map(_ + 1).orElse(Some("saknas")).foreach(println)
\end{REPL}

\Subtask \emph{Samlingsmetoder som ger en \code{Option}.} Förklara för varje rad nedan vad som händer. En av raderna ger ett felmeddelande; vilken rad och vilket felmeddelande?
\begin{REPL}
val xs = (42 to 84 by 5).toVector
val e = Vector.empty[Int]
xs.headOption
xs.headOption.get
xs.headOption.getOrElse(0)
xs.headOption.orElse(Some(0))
e.headOption
e.headOption.get
e.headOption.getOrElse(0)
e.headOption.orElse(Some(0))
Vector(xs, e, e, e)
Vector(xs, e, e, e).map(_.lastOption)
Vector(xs, e, e, e).map(_.lastOption).flatten
xs.lift(0)
xs.lift(1000)
e.lift(1000).getOrElse(0)
xs.find(_ > 50)
xs.find(_ < 42)
e.find(_ > 42).foreach(_ => println("HITTAT!"))
\end{REPL}

\Subtask Vilka är fördelerna med \code{Option} jämfört med \code{null} eller \code{-1} om man i sin kod glömmer hantera saknade värden?

\SOLUTION


\TaskSolved \what


\SubtaskSolved  \begin{enumerate}
\item \code{var kanske} blir en \code{Option} som håller \code{Int} men är utan något värde, kallas då \code{None}.
\item Eftersom \code{var kanske} är utan värde är storleken av den 0.
\item \code{var kanske} tilldelas värdet 42 som förvaras i en \code{Some} som visar att värde finns.
\item Eftersom \code{var kanske} nu innehåller ett värde är storleken 1.
\item Eftersom \code{var kanske} innehåller ett värde är den inte tom.
\item Eftersom \code{var kanske} innehåller ett värde är den definierad.
\item \code{def ökaOmFinns} matchar en \code{Option[Int]} med dess olika fall.\\
Finns ett värde, alltså \code{opt: Option[Int]} är en \code{Some}, så returneras en \code{Some} med ursprungliga värdet plus 1.\\
Finns inget värde, alltså \code{opt: Option[Int]} är en \code{None}, så returneras en \code{None}.
\item -
\item -
\item -
\item \code{def ökaOmFinns} appliceras på \code{kanske} och returnerar en \code{Some} med värdet hos \code{kanske} plus 1, alltså 43.
\item \code{def öka} tar emot värdet av en \code{Int} och returnerar värdet av denna plus 1.
\item \code{map} applicerar \code{def öka} till det enda elementen i \code{kanske}, 42. Denna funktion returnerar en \code{Some} med värdet 43 som tilldelas \code{merKanske}.
\end{enumerate}

\SubtaskSolved  \begin{enumerate}
\item \code{val meningen} blir en \code{Some} med värdet 42.
\item \code{val ejMeningen} blir en \code{Option[Int]} utan något värde, en \code{None}.
\item \code{map(_ + 1)} appliceras på \code{meningen} och ökar det existerande värdet med 1 till 43.
\item \code{map(_ + 1)} appliceras på \code{ejMening} men eftersom inget värde existerar fortsätter denna vara \code{None}.
\item \code{map(_ + 1)} appliceras ännu en gång på \code{ejMening} men denna gång inkluderas metoden \code{orElse}. Om ett värde inte existerar hos en \code{Option}, alltså är av typen \code{None}, så utförs koden i \code{orElse}-metoden som i detta fall skriver ut \textit{saknas} för värdet som saknas.
\item Samma anrop från föregående rad utförs denna gång på \code{meningen} och eftersom ett värde finns utförs endast första biten som ökar detta värde med 1.
\end{enumerate}
Denna metod kan användas i stället för \code{match}-versionen i föregående exempel i och med dennas simplare form. En \code{Option} innehåller ju antingen ett värde eller inte så ett längre \code{match}-uttryck är inte nödvändigt.

\SubtaskSolved \begin{enumerate}
\item En vektor \code{xs} skapas med var femte tal från 42 till 82.
\item En tom \code{Int}-vektor \code{e} skapas.
\item \code{headOption} tar ut första värdet av vektorn \code{xs} och returnerar den sparad i en \code{Option}, \code{Some(42)}.
\item Första värdet i vektorn \code{xs} sparas i en \code{Option} och hämtas sedan av \code{get}-metoden, 42.
\item Som i föregående rad men denna gång används \code{getOrElse} som om den \code{Option} som returneras saknar ett värde, alltså är av typen \code{None}, returnerar 0 istället.\\
 Eftersom \code{xs} har minst ett värde så är den \code{Option} som returneras inte \code{None} och ger samma värde som i föregående, 42.
\item Som föregående rad fast istället för att returnera 0 om värde saknas så returneras en \code{Option[Int]} med 0 som värde.
\item \code{headOption} försöker ta ut första värdet av vektorn \code{e} men eftersom denna saknar värden returneras en \code{None}.
\item \begin{REPL}
java.util.NoSuchElementException: None.get
\end{REPL}
Liksom föregående rad returnerar \code{headOption} på den tomma vektorn \code{e} en \code{None}. När  \code{get}-metoden försöker hämta ett värde från en \code{None} som saknar värde ger detta upphov till ett körtidsfel.
\item Liksom i föregående returneras \code{None}  av \code{headOption} men eftersom \code{getOrElse}-metoden används på denna \code{None} returneras 0 istället.
\item Liksom föregående används \code{getOrElse}-metoden på den \code{None} som returneras. Denna gång returneras dock en \code{Option[Int]} som håller värdet 0.
\item En vektor innehållandes elementen \code{xs}-vektorn och 3 \code{e}-vektorer skapas.
\item \code{map} använder metoden \code{lastOption} på varje delvektor från vektorn på föregående rad. Detta sammanställer de sista elementen från varje delvektor i en ny vektor. Eftersom vektor \code{e} är tom returneras \code{None} som element från denna.
\item Samma sker som i föregående rad men \code{flatten}-metoden appliceras på slutgiltiga vektorn som rensar vektorn på \code{None} och lämnar endast faktiska värden.
\item \code{lift}-metoden hämtar det eventuella värdet på plats 0  i \code{xs} och returnerar den i en \code{Option} som blir \code{Some(42)}.
\item \code{lift}-metoden försöker hämta elementet på plats 1000 i \code{xs}, eftersom detta inte existerar returneras \code{None}.
\item  Samma sker som i föregående fast applicerat på vektorn \code{e}. Sedan appliceras \code{getOrElse(0)} som, eftersom \code{lift}-metoden returnerar \code{None}, i sin tur returnerar 0.
\item \code{find}-metoden anropas på \code{xs}-vektorn. Den letar upp första talet över 50 och returnerar detta värde i en \code{Option[Int]}, alltså \code{Some(52)}.
\item \code{find}-metoden anropas på \code{xs}-vektorn. Den letar upp första värdet under 42 men eftersom inget värde existerar under 42 i \code{xs} returneras \code{None} istället.
\item \code{find}-metoden anropas på \code{e}-vektorn och skriver ut \textit{HITTAT!} om ett element under 42 hittas. Eftersom \code{e}-vektorn är tom returneras \code{None} vilket \code{foreach} inte räknar som element och därav inte utförs på.
\end{enumerate}

\SubtaskSolved  Användning av -1 som returvärde vid fel eller avsaknad på värde kan ge upphov till körtidsfel som är svåra att upptäcka. \jcode{null} kan i sin tur orsaka kraschar om det skulle bli fel under körningen. \code{Option} har inte samma problem som dessa, används ett \code{getOrElse}-uttryck eller dylikt så kraschar inte heller programmet.\\
Dessutom behöver inte en funktion som returnerar en \code{Option} samma dokumentation av returvärdena. Istället för att skriva kommentarer till koden på vilka värden som kan returneras och vad dessa betyder så syns det direkt i koden.\\
Slutgiltligen är \code{Option} mer typsäkert än \code{null}. När du returnerar en \code{Option} så specificeras typen av det värde som den kommer innehålla, om den innehåller något, vilket underlättar att förstå och begränsar vad den kan returnera.



\QUESTEND






\WHAT{Kasta undantag.}

\QUESTBEGIN

\Task  \what~  Om man vill signalera att ett fel eller en onormal situtation uppstått så kan man \textbf{kasta} \Eng{throw} ett \textbf{undantag} \Eng{exception}. Då avbryts programmet direkt med ett felmeddelande, om man inte väljer att \textbf{fånga} \Eng{catch} undantaget.
\Subtask Vad händer nedan?
\begin{REPL}
scala> throw new Exception("PANG!")
scala> java.lang.   // Tryck TAB efter punkten
scala> throw new IllegalArgumentException("fel fel fel")
scala> val carola = 
         try 
           throw new Exception("stormvind!")
           42
         catch 
           case e: Throwable => 
             println("Fångad av en " + e)
             -1
\end{REPL}
\Subtask Nämn ett par undantag som finns i paketet \code{java.lang} som du kan gissa vad de innebär och i vilka situationer de kastas.

\Subtask Vilken typ har variabeln \code{carola} ovan? Vad hade typen blivit om catch-grenen hade returnerat en sträng i stället?

\SOLUTION


\TaskSolved \what


\SubtaskSolved  \begin{enumerate}
\item Ett \code{Exception} kastas med felmeddelandet \textit{PANG!}.
\item Flera olika typer av \code{Exception} visas.
\item En typ av \code{Exception}, \code{IllegalArgumentException}, kastas med felmeddelandet \textit{fel fel fel}.
\item Ett undantag med felmeddelandet \code{stormvind!} kastas och fångas av \code{catch}-uttrycket. Ett \code{match}-uttryck undersöker undantaget och skriver ut meddelandet, samt returnerar -1.
\end{enumerate}

\SubtaskSolved  Exempelvis: \\
\code{OutOfMemoryError}, om programmet får slut på minne.\\
\code{IndexOutOfBoundsException}, om en vektorposition som är större än vad som finns hos vektorn försöker nås.\\
\code{NullPointerException}, om en metod eller dylikt försöker användas hos ett objekt som inte finns och därav är en nullreferens.

\SubtaskSolved  om både try-grenen och catch-grenen har samma typ, här \code{Int}, så härleder kompilatorn samma typ för hela uttrycket. 
Skulle \code{catch}-grenen returnera ett värde av en helt annan typ istället, t.ex. \code{String}, så blir den mest precisa typen som kompilatorn kan härleda för hela uttrycket \code{Matchable}, som är en direkt subtyp till den mest generella typen \code{Any}.



\QUESTEND










\WHAT{Fånga undantantag med \code{scala.util.Try}.}

\QUESTBEGIN

\Task  \what~  I paketet \code{scala.util} finns typen \code{Try} med stort T som är som en slags samling som kan innehålla antingen ett ''lyckat'' eller ''misslyckat'' värde. Om beräkningen av värdet lyckades och inga undantag kastas blir värdet inkapslat i en \code{Success}, annars blir undantaget inkapslat i en \code{Failure}. Man kan extrahera värdet, respektive undantaget, med mönstermatchning, men det är oftast smidigare att använda samlingsmetoderna \code{map} och \code{foreach}, i likhet med hur \code{Option} används. Det finns även en smidig metod \code{recover} på objekt av typen \code{Try} där man kan skicka med kod som körs om det uppstår en undantagssituation.

\Subtask Förklara vad som händer nedan.
\begin{REPL}
scala> def pang = throw new Exception("PANG!")
scala> import scala.util.{Try, Success, Failure}
scala> Try{pang}
scala> Try{pang}.recover{case e: Throwable =>   "desarmerad bomb: " + e}
scala> Try{"tyst"}.recover{case e: Throwable => "desarmerad bomb: " + e}
scala> def kanskePang = if math.random() > 0.5 then "tyst" else pang
scala> def kanskeOk = Try{kanskePang}
scala> val xs = Vector.fill(100)(kanskeOk)
scala> xs(13) match
         case Success(x) => ":)"
         case Failure(e) => ":( " + e
scala> xs(13).isSuccess
scala> xs(13).isFailure
scala> xs.count(_.isFailure)
scala> xs.find(_.isFailure)
scala> val badOpt = xs.find(_.isFailure)
scala> val goodOpt = xs.find(_.isSuccess)
scala> badOpt
scala> badOpt.get
scala> badOpt.get.get
scala> badOpt.map(_.getOrElse("bomben desarmerad!")).get
scala> goodOpt.map(_.getOrElse("bomben desarmerad!")).get
scala> xs.map(_.getOrElse("bomben desarmerad!")).foreach(println)
scala> xs.map(_.toOption)
scala> xs.map(_.toOption).flatten
scala> xs.map(_.toOption).flatten.size
\end{REPL}


\Subtask Vad har funktionen \code{pang} för returtyp?

\Subtask Varför får funktionen \code{kanskePang} den härledda returtypen \code{String}?

\SOLUTION


\TaskSolved \what


\SubtaskSolved  \begin{enumerate}
\item \code{def pang} skapas som kastar ett \code{Exception} med felmeddelandet \textit{PANG!}.
\item Scalas verktyg \code{Try}, \code{Success} och \code{Failure} importeras.
\item \code{def pang} anropas i \code{Try} som fångar undantaget och kapslar in den i en \code{Failure}.
\item Metoden \code{recover} matchar undantaget i \code{Failure} från föregående rad med ett \code{case} och gör om föredetta \code{Failure} till \code{Success} vid matchning, liknande \code{catch}.
\item Strängen \textit{tyst} körs i föregående test men eftersom inget undantag kastas blir den inkapslad i en \code{Success} och \code{recover} behöver inte göra något. Den tar endast hand om undantag.
\item \code{def kanskePang} skapas som har lika stor chans att returnera strängen \textit{tyst} såsom anropa \code{def pang}.
\item \code{def kanskeOk} skapas som testar \code{def kanskePang} med \code{Try}.
\item En vektor \code{xs} fylls med resultaten, \code{Success} och \code{Failure}, från 100 körningar av \code{kanskeOk}.
\item Elementet på plats 13 i vektor \code{xs} matchas med något av 2 \code{case}. Om det är en \code{Success} skrivs \textit{:)} ut, om en \code{Failure} skrivs \textit{:(} plus felmeddelandet ut.
\item -
\item -
\item Metoden \code{isSuccess} testar om elementet på plats 13 i \code{xs} är en \code{Success} och returnerar \code{true} om så är fallet.
\item Metoden \code{isFailure} testar om elementet på plats 13 i \code{xs} är en \code{Failure} och returnerar \code{true} om så är fallet.
\item Metoden \code{count} räknar med hjälp av \code{isFailure} hur många av elementen i \code{xs} som är \code{Failure} och returnerar detta tal.
\item Metoden \code{find} letar upp med hjälp av \code{isFailure} ett element i \code{xs} som är \code{Failure} och returnerar denna i en \code{Option}.
\item \code{badOpt} tilldelas den första \code{Failure} som hittas i \code{xs}.
\item \code{goodOpt} tilldelas den första \code{Success} som hittas i \code{xs}.
\item Resultatet badOpt skrivs ut, \code{Option[scala.util.Try[String]] =}\\
\code{Some(Failure(java.lang.Exception: PANG!))}
\item Metoden \code{get} hämtar från \code{badOpt} den \code{Failure} som förvaras i en \code{Option}.
\item Metoden \code{get} anropas ännu en gång på resultatet från föregående rad, alltså en \code{Failure}, som hämtar undantaget från denna och som då i sin tur kastas.
\item Metoden \code{getOrElse} anropas på den \code{Failure} som finns i \code{badOpt}. Eftersom detta är en \code{Exception} utförs \code{orElse}-biten istället för att undantaget försöker hämtas. Då returneras strängen \textit{bomben desarmerad!}.
\item Metoden \code{getOrElse} anropas på den \code{Success} som finns i \code{goodOpt}. Eftersom detta är en \code{Success} med en normal sträng sparad i sig returneras denna sträng, \textit{tyst}.
\item Metoden från föregående används denna gång på alla element i \code{xs} där resultatet skrivs ut för varje.
\item Metoden \code{toOption} appliceras på alla \code{Success} och \code{Failure} i \code{xs}. De med ett exception, alltså \code{Failure}, blir en \code{None} medan de med värden i \code{Success} ger en \code{Some} med strängen \textit{tyst} i sig.
\item Metoden \code{flatten} appliceras på vektorn fylld med \code{Option} från föregående rad för att ta bort alla \code{None}-element.
\item Metoden \code{size} används på slutgiltiga listan från föregående rad för att räkna ut hur många \code{Some} som resultatet innehåller. Den har alltså beräknat antalet element i \code{xs} som var av typen \code{Success} med hjälp av \code{Option}-typen.
\end{enumerate}

\SubtaskSolved  \code{pang} har returtypen \code{Nothing}, en specialtyp inom Scala som inte är kopplad till \code{Any}, och som inte går att returnera.

\SubtaskSolved  Typen \code{Nothing} är en subtyp av varenda typ i Scalas hierarki. Detta innebär att den även är en subtyp av \code{String} vilket implicerar att \code{String} inkluderar både strängar och \code{Nothing} och därav blir returtypen.


\QUESTEND




% \WHAT{Laborationsförberedelse.}

% \QUESTBEGIN

% \Task  \what~ \label{task:labprep-patterns-tabular} På veckans laboration ska du hantera data som finns i tabeller med celler som kan bestå av decimaltal eller strängar. Studera den givna koden som du ska utgå ifrån; uppgifterna nedan berör \code{Cell.scala} och \code{Table.scala} här:
% \url{https://github.com/lunduniversity/introprog/tree/master/workspace-old/w13_tabular/src/main/scala/tabular}

% Bastypen \code{Cell} i koden nedan har två subtyper \code{Str} och \code{Num}.

% \begin{CodeSmall}
% sealed trait Cell { def value: String }
% case class Str(value: String) extends Cell
% case class Num(num: BigDecimal) extends Cell { def value = num.toString }
% \end{CodeSmall}
% \code{BigDecimal} används för att representera decimaltal med bättre precision än vanliga flyttal av typen \code{Double}.

% \Subtask Studera dokumentationen för \code{BigDecimal}: \url{https://www.scala-lang.org/api}\\
% Vad gör fabriksmetoden \code{def apply(x: String): BigDecimal} (se kompanjonsobj.).


% \Subtask Vad är fördelen med att \code{Cell} är förseglad?

% \Subtask Kör igång REPL med koden för \code{Cell}-hierarkin tillgänglig på classpath, t.ex. med \code{sbt console}. Vad ger koden nedan för resultat? Ange värde och typ för varje rad.

% \begin{REPL}
% scala> val xs = Seq[Cell](Str("!"), Num(BigDecimal("100000000.000000001")))
% scala> val ys = xs.map(_ match { case Num(n) => Some(n) case _ => None })
% scala> val b = ys.flatten.headOption.getOrElse(BigDecimal(0))
% \end{REPL}

% \Subtask Lägg till ett kompanjonsobjekt enligt nedan. Gör klart den saknade implementationen. Använd \code{Try} och matcha på \code{Success} och \code{Failure}. Testa så att alla metoder i kompanjonsobjektet fungerar.

% \Subtask Gör om implementation så att du i stället använder \code{Try} och \code{getOrElse}. Testa så att det fungerar som innan. Vilken implementation är smidigast?
% \begin{CodeSmall}
% object Cell {
%   import scala.util.{Try, Success, Failure}

%   /** Ger en Num om BigDecimal(s) lyckas annars en Str. */
%   def apply(s: String): Cell =  ???

%   def apply(i: Int): Num = Num(BigDecimal(i))

%   def empty: Str = Str("")

%   def zero: Num = Num(BigDecimal(0))
% }
% \end{CodeSmall}

% \Subtask I given kod och nedan finns en nästan färdig klass för tabelldatahantering. Implementera de saknade delarna enligt beskrivning i dokumentationskommentarerna. Testa så att dina implementationer fungerar och försök förstå hur övriga delar av \code{Table} fungerar.

% \scalainputlisting[numbers=left,basicstyle=\ttfamily\fontsize{9}{11.5}\selectfont]{../workspace-old/w13_tabular/src/main/scala/tabular/Table.scala}

% \noindent Tips vid färdigställande av \code{Table}:
% \begin{itemize}[leftmargin=*]
%   \item Nyckel-värde-tabeller har en metod \code{withDefaultValue} som är smidig om man vill undvika undantag vid uppslagning med nyckel som inte finns och det i stället för undantag är möjligt/lämpligt att erbjuda ett vettigt defaultvärde.
%   \item Metoderna \code{getOrElse} och \code{toOption} på en \code{Try} är smidiga när man vill ge resultat som beror av om det är \code{Success} eller \code{Failure} utan att man behöver göra en \code{match}.
% \item Skiss på implementation av \code{load} i kompanjonsobjektet:
% \begin{CodeSmall}
% def load(fileOrUrl: String, separator: Char): Table = {
%   val source = fileOrUrl match {
%     case /* använd gard och startsWith*/ => scala.io.Source.fromURL(url)
%     case path  => scala.io.Source.fromFile(path)
%   }
%   val lines = try source.getLines.toVector finally source.close
%   val rows = ??? // kör split(separator).toVector på alla rader i lines
%   Table(rows.head, rows.tail.map(_.map(Cell.apply)), separator)
% }
% \end{CodeSmall}
% En webbadress börjar med \code{http}.
% Med \code{try sats1 finally sats2} så kan man garantera att \code{sats2} alltid görs även om \code{sats1} ger undantag. Detta används typiskt för att frigöra resurser som annars förblir allokerade vid undantag. I koden ovan används det för att undvika att filer inte stängs även om något går fel under läsningen.
% \end{itemize}
% \SOLUTION


% \TaskSolved \what

% \SubtaskSolved ''Translates the decimal String representation of a BigDecimal into a BigDecimal.''

% \SubtaskSolved Eftersom \code{Cell} är förseglad med \code{sealed} så kan inga andra subtyper finnas och vi behöver inte kolla efter andra subtyper när vi matchar. Kompilatorn varnar också om vi glömmer matcha på någon av subtyperna.

% \SubtaskSolved
% \begin{REPL}
% scala> val xs = Seq[Cell](Str("!"), Num(BigDecimal("100000000.000000001")))
% xs: Seq[Cell] = List(Str(!), Num(100000000.000000001))

% scala> val ys = xs.map(_ match { case Num(n) => Some(n) case _ => None })
% ys: Seq[Option[BigDecimal]] = List(None, Some(100000000.000000001))

% scala> val b = ys.flatten.headOption.getOrElse(BigDecimal(0))
% b: BigDecimal = 100000000.000000001
% \end{REPL}

% \SubtaskSolved
% \begin{Code}
%   def apply(s: String): Cell = Try(BigDecimal(s)) match {
%     case Success(num) => Num(num)
%     case Failure(_)   => Str(s)
%   }
% \end{Code}

% \SubtaskSolved
% \begin{Code}
%   def apply(s: String): Cell = Try(Num(BigDecimal(s))).getOrElse(Str(s))
% \end{Code}

% \SubtaskSolved \emph{Lämnas som egen laborationsförberedelse.}

% \QUESTEND


\AdvancedTasks %%%%%%%%%%%%%%%%%%%



\WHAT{Använda matchning eller dynamisk bindning?}

\QUESTBEGIN

\Task  \what~ Man kan åstadkomma urskiljningen av de ätbara grönsakerna i uppgift \ref{task:match-caseclass} med dynamisk bindning i stället för \code{match}.

\Subtask Gör en ny variant av ditt program enligt nedan riktlinjer och spara den modifierade koden i filen \texttt{vegopoly.scala} och kompilera och kör.
\begin{itemize}[noitemsep]
\item Ta bort predikatet \code{ärÄtvärd} i objektet \code{Main} och inför i stället en abstrakt metod \code{def ärÄtbar: Boolean} i traiten \code{Grönsak}.
\item Inför konkreta \code{val}-medlemmar i respektive grönsak som definierar ätbarheten.
\item Ändra i huvudprogrammet i enlighet med ovan ändringar så att \code{ärÄtvärd} anropas som en metod på de skördade grönsaksobjekten när de ätvärda ska filtreras ut.
\end{itemize}

\Subtask Lägg till en ny grönsak \code{case class Broccoli} och definiera dess ätbarhet. Ändra i slump-funktionerna så att broccoli blir ovanligare än gurka.

\Subtask Jämför lösningen med \code{match} i uppgift \ref{task:match-caseclass} och lösningen ovan med polymorfism. Vilka är för- och nackdelarna med respektive lösning? Diskutera två olika situationer på ett hypotetiskt företag som utvecklar mjukvara för jordbrukssektorn: 1) att uppsättningen grönsaker inte ändras särskilt ofta medan definitionerna av ätbarhet ändras väldigt ofta och 2) att uppsättningen grönsaker ändras väldigt ofta men att ätbarhetsdefinitionerna inte ändras särskilt ofta.



\SOLUTION


\TaskSolved \what


\SubtaskSolved
\begin{Code}
package vegopoly

trait Grönsak:
	def vikt: Int
	def ärRutten: Boolean
	def ärÄtbar: Boolean

case class Gurka(vikt: Int, ärRutten: Boolean) extends Grönsak:
  val ärÄtbar: Boolean = (!ärRutten && vikt > 100)

case class Tomat(vikt: Int, ärRutten: Boolean) extends Grönsak:
  val ärÄtbar: Boolean = (!ärRutten && vikt > 50)

object Main:
	def slumpvikt: Int = (math.random()*500 + 100).toInt

	def slumprutten: Boolean = math.random() > 0.8

	def slumpgurka: Gurka = Gurka(slumpvikt, slumprutten)

	def slumptomat: Tomat = Tomat(slumpvikt, slumprutten)

	def slumpgrönsak: Grönsak =
    if math.random() > 0.2 then slumpgurka else slumptomat

	def main(args: Array[String]): Unit = 
		val skörd = Vector.fill(args(0).toInt)(slumpgrönsak)
		val ätvärda = skörd.filter(_.ärÄtbar)
		println("Antal skördade grönsaker: " + skörd.size)
		println("Antal ätvärda grönsaker: " + ätvärda.size)
\end{Code}

\SubtaskSolved
Följande \code{case class} läggs till:
\begin{Code}
case class Broccoli(vikt: Int, ärRutten: Boolean) extends Grönsak:
  val ärÄtbar: Boolean = (!ärRutten && vikt > 80)
\end{Code}
~\\
Därefter läggs följande till i \code{object Main} innan \code{def slumpgrönsak}:

\begin{Code}
def slumpbroccoli: Broccoli = Broccoli(slumpvikt, slumprutten)
\end{Code}
~\\
Slutligen ändras \code{def slumpgrönsak} till följande:

\begin{Code}
def slumpgrönsak: Grönsak =     // välj t.ex. denna fördelning:
  val rnd = math.random()
  if rnd > 0.5 then slumpgurka      // 50% sannolikhet för gurka
  else if rnd > 0.2 then slumptomat // 30% sannolikhet för tomat
  else slumpbroccoli             // 20% sannolikhet för broccoli

\end{Code}

\SubtaskSolved  Fördelarna med \code{match}-versionen, och mönstermatchning i sig, är att det är väldigt lätt att göra ändringar på hur matchningen sker. Detta innebär att det skulle vara väldigt lätt att ändra definitionen för ätbarheten. Skulle dock dessa inte ändras ofta utan snarare grönsaksutbudet så kan det polyformistiska alternativet vara att föredra. Detta eftersom det skulle implementeras och ändras lättare än mönstermatchningen vid byte av grönsaker.



\QUESTEND





\WHAT{Metoden \code{equals}.}

\QUESTBEGIN

\Task  \what~   Om man överskuggar den befintliga metoden \code{equals} så kommer metoden \code{==} att fungera annorlunda. Man kan då själv åstadkomma innehållslikhet i stället för referenslikhet. Vi börjar att studera den befintliga equals med referenslikhet.

\Subtask \label{subtask:refequals} Vad händer nedan? Undersök parametertyp och returvärdestyp för  \code{equals}.
\begin{REPL}
scala> class Gurka(val vikt: Int, val ärÄtbar: Boolean)
scala> val g1 = new Gurka(42, true)
scala> val g2 = g1
scala> val g3 = new Gurka(42, true)
scala> g1 == g2
scala> g1 == g3
scala> g1.equals  // tryck ENTER för att se funktionstyp
\end{REPL}

\Subtask Rita minnessituationen efter rad 4.

\Subtask \emph{Överskugga metoderna \code{equals} och \code{hashCode}.}

\begin{Background}
Det visar sig förvånande komplicerat att implementera innehållslikhet med metoden \code{equals} så att den ger bra resultat under alla speciella omständigheter. Till exempel måste man även överskugga en metod vid namn \code{hashCode} om man överskuggar \code{equals}, eftersom dessa båda används gemensamt av effektivitetsskäl för att skapa den interna lagringen av objekten i vissa samlingar. Om man missar det kan objekt bli ''osynliga'' i \code{hashCode}-baserade samlingar -- men mer om detta i senare kurser. Om objekten ingår i en öppen arvshierarki blir det också mer komplicerat; det är enklare om man har att göra med finala klasser. Dessutom krävs speciella hänsyn om klassen har en typparameter.
\end{Background}

\noindent Definera klassen nedan i REPL med överskuggade \code{equals} och \code{hashCode}; den ärver inte något och är final.

\begin{Code}
// fungerar fint om klassen är final och inte ärver något
final class Gurka(val vikt: Int, val ärÄtbar: Boolean):
  override def equals(other: Any): Boolean = other match
    case that: Gurka => vikt == that.vikt && ärÄtbar == that.ärÄtbar
    case _ => false
  override def hashCode: Int = (vikt, ärÄtbar).## //förklaras sen
\end{Code}
\Subtask Vad händer nu nedan, där \code{Gurka} nu har en överskuggad \code{equals} med innehållslikhet?
\begin{REPL}
scala> val g1 = new Gurka(42, true)
scala> val g2 = g1
scala> val g3 = new Gurka(42, true)
scala> g1 == g2
scala> g1 == g3
\end{REPL}
\Subtask Hur märker man ovan att den överskuggade \code{equals} medför att \code{==} nu ger innehållslikhet? Jämför med deluppgift \ref{subtask:refequals}.

I uppgift \ref{task:equals:Complex} får du prova på att följa det fullständiga receptet i 8 steg för att överskugga en \code{equals} enligt konstens alla regler. I efterföljande kurs kommer mer träning i att hantera innehållslikhet och hash-koder. I Scala får man ett objekts hash-kod med metoden \code{##}.%
\footnote{Om du är nyfiken på hash-koder, läs mer här:
\href{https://en.wikipedia.org/wiki/Hash_function}
{en.wikipedia.org/wiki/Hash\_function}
}


\SOLUTION


\TaskSolved \what


\SubtaskSolved  \begin{enumerate}
\item En klass \code{Gurka} skapas med parametrarna \code{vikt} av typen \code{Int} och ärÄtbar av typen \code{Boolean}.
\item \code{g1} tilldelas en instans av \code{Gurka}-klassen med \code{vikt = 42} och \code{ärÄtbar = true}.
\item \code{g2} tilldelas samma \code{Gurka}-objekt som g1.
\item \code{g3} tilldelas en ny instans av \code{Gurka}-klassen med motsvarande parametrar som g1.
\item \code{==}(\code{equals})-metoden jämför g1 med g2 och returnerar \code{true}.
\item \code{==}(\code{equals})-metoden jämför g1 med g3 och returnerar \code{false}.
\item \code{def equals(x\$1: Any): Boolean}
\end{enumerate}
Som kan ses ovan är elementet som jämförs i \code{equals} av typen \code{Any}. Eftersom programmet inte känner till klassen så används \code{Any.equals} vid jämförelsen. Till skillnad från de primitiva datatyperna som vid jämförelse med \code{equals} jämför innehållslikhet, så jämförs referenslikheten hos klasser om inget annat är specificerat. \code{g1} och \code{g2} refererar till samma objekt medan \code{g3} pekar på ett eget sådant vilket innebär att \code{g1} och \code{g3} inte har referenslikhet.

\SubtaskSolved  \\
\vspace{1em}
\tikzstyle{mybox} = [draw=red, fill=blue!20, very thick,
    rectangle, rounded corners, inner sep=10pt, inner ysep=20pt]
\begin{tikzpicture}[
	font=\large\sffamily,
	varname/.style={node distance=0.2cm},
	varbox/.style={draw, node distance=0.2cm},
	objcloud/.style={cloud, cloud puffs=15.7, cloud ignores aspect, align=center, draw},
]

\node [varname] (g1var) {\texttt{g1}};
\node [varbox, right = of g1var] (g1ref) {\phantom{abc}};
\filldraw[black] (g1ref) circle (3pt) node[] (g1dot) {};
\node [objcloud, right = of g1ref, yshift=1.3cm, scale =0.8] (g1obj) {
	\texttt{\textbf{Gurka}} \\~\\ \texttt{vikt} \framebox{42} ~ \texttt{ärÄtvärd} \framebox{true}
};
\draw [arrow] (g1dot) -- (g1obj);

\node [varname, below = of g1var] (g2var) {\texttt{g2}};
\node [varbox, right = of g2var] (g2ref) {\phantom{abc}};
\filldraw[black] (g2ref) circle (3pt) node[] (g2dot) {};
\node [objcloud, right = of g2ref, yshift=-1.3cm, scale =0.8] (g2obj) {
	\texttt{\textbf{Gurka}} \\~\\ \texttt{vikt} \framebox{42} ~ \texttt{ärÄtvärd} \framebox{true}
};
\draw [arrow] (g2dot) -- (g1obj);
\node [varname, below = of g2var] (g3var) {\texttt{g3}};
\node [varbox, right = of g3var] (g3ref) {\phantom{abc}};
\filldraw[black] (g3ref) circle (3pt) node[] (g3dot) {};
\draw [arrow] (g3dot) -- (g2obj);

\end{tikzpicture}

\SubtaskSolved  -

\SubtaskSolved  I de första 3 raderna sker samma som i deluppgift \textit{a}. När nu dessa jämförelser görs mellan \code{Gurka}-objekten så överskuggas \code{Any.equals} av den \code{equals} som är specificerad för just \code{Gurka}. Eftersom båda objekten \code{g1} jämförs med också är av typen \code{Gurka} så matchar den med \code{case that: Gurka}. Denna i sin tur jämför vikterna hos de båda gurkorna och returnerar en \code{Boolean} huruvida de är lika eller inte, vilket de i båda fallen är.

\SubtaskSolved  I deluppgift a gav \code{g1 == g3 false} trots innehållslikhet. Efter skuggningen ger dock detta uttryck \code{true} vilket påvisar jämförelse av innehållslikhet.



\QUESTEND






\WHAT{Polynom.}

\QUESTBEGIN

\Task \label{task:polynomial} \what~   Med hjälp av koden nedan, kan man göra följande:
\begin{REPL}
scala> import polynomial.*

scala> Const(1) * x
res0: polynomial.Term = x

scala> (x*5)^2
res1: polynomial.Prod = 25x^2

scala> Poly(x*(-5), y^4, (z^2)*3)
res2: polynomial.Poly = -5x + y^4 + 3z^2

\end{REPL}

\Subtask Förklara vad som händer ovan genom att studera koden nedan\footnote{Koden finns även här:\\ \href{https://github.com/lunduniversity/introprog/tree/master/compendium/examples/polynomial}{github.com/lunduniversity/introprog/tree/master/compendium/examples/polynomial}}.

\scalainputlisting[numbers=left,basicstyle=\ttfamily\fontsize{10.5}{13}\selectfont]{examples/polynomial/polynomial.scala}

\Subtask Bygg vidare på \code{object polynomial} och implementera addition mellan olika termer.


\SOLUTION


\TaskSolved \what


\SubtaskSolved \TODO

\SubtaskSolved \TODO



\QUESTEND






\WHAT{\code{Option} som en samling.}

\QUESTBEGIN

\Task  \what~Studera dokumentationen för \code{Option} här och se om du känner igen några av metoderna som också finns på samlingen \code{Vector}:\\ \href{http://www.scala-lang.org/api/current/scala/Option.html}{www.scala-lang.org/api/current/scala/Option.html}
\\Förklara hur metoden \code{contains} på en \code{Option} fungerar med hjälp av dokumentationens exempel.



\SOLUTION


\TaskSolved \what 

Exempel på metoder som finns både för \code{Vector} och \{Option}:
\code{foreach}, \code{filter}, \code{fold} etc.

Contains returnerar en \code{Boolean} som visar om den har ett värde eller ej.


\QUESTEND






\WHAT{Fånga undantag med \code{catch} i Java och Scala.}

\QUESTBEGIN

\Task  \what~ Gör motsvarande program i Scala som visas i uppgift \ref{task:javatry}, men utnyttja att Scalas \code{try}-\code{catch} är ett uttryck. Kompilera och kör och testa så att de ur användarens synvinkel fungerar precis på samma sätt. Notera de viktigaste skillnaderna mellan de båda programmen.


\SOLUTION


\TaskSolved \what \TODO


\QUESTEND



\WHAT{Polynom, fortsättning: reducering.}

\QUESTBEGIN

\Task  \what~ Bygg vidare på \code{object polynomial} i uppgift \ref{task:polynomial} på sidan \pageref{task:polynomial} och implementera metoden \code{def reduce: Poly} i case-klassen \code{Poly} som förenklar polynom om flera \code{Prod}-termer kan adderas.

\SOLUTION


\TaskSolved \what



\QUESTEND




% \WHAT{Hash-koder.}

% \QUESTBEGIN

% \Task  \what~ Läs om hash-funktioner här: \href{https://en.wikipedia.org/wiki/Hash_function}{en.wikipedia.org/wiki/Hash_function} \\
% Vad ger metoden \code{##} i scala.Any för resultat? Läs dokumentationen här: \\ \href{http://www.scala-lang.org/api/current/scala/Any.html}{www.scala-lang.org/api/current/scala/Any.html}

% \SOLUTION

% \TaskSolved \what I Scala får man ett objekts hash-kod med metoden \code{##}.

% \QUESTEND






\WHAT{Typsäker innehållstest med metoden \code{===}.}

\QUESTBEGIN

\Task  \what~  Metoderna \code{equals} och \code{==} tillåter jämförelse med vad som helst. Ibland vill man ha en typsäker innehållsjämförelse som bara tillåter jämförelse av objekt av en mer specifik typ och ger kompileringsfel annars. Man brukar då definiera en metod \code{===} som har en parameter \code{that} som har en så specifik typ som önskas. Inför nedan abstrakta metod \code{===} i traiten \code{polynomial.Term} i uppgift \ref{task:polynomial} på sidan \pageref{task:polynomial} och överskugga den sedan i alla subklasser till Term. Testa så att du får kompileringsfel om du försöker jämföra en \code{Term} med något helt annat, t.ex. en \code{String} eller \code{Vector}.
\begin{Code}
  def ===(that: Term): Boolean
\end{Code}


\SOLUTION


\TaskSolved \what



\QUESTEND






\WHAT{Överskugga \code{equals} med innehållslikhet även för icke-finala klasser.}

\QUESTBEGIN

\Task \label{task:equals:Complex} \what~   Nedan visas delar av klassen \code{Complex} som representerar ett komplext tal med realdel och imaginärdel. I stället för att, som man ofta gör i Scala, använda en case-klass och en \code{equals}-metod som automatiskt ger innehållslikhet, ska du träna på att implementera en egen \code{equals}.
\begin{Code}
class Complex(val re: Double, val im: Double):
  def abs: Double = math.hypot(re, im)
  override def toString = s"Complex($re, $im)"
  def canEqual(other: Any): Boolean = ???
  override def hashCode: Int  = ???
  override def equals(other: Any): Boolean = ???

case object Complex:
  def apply(re: Double, im: Double): Complex = new Complex(re, im)
\end{Code}
Följ detta \textbf{recept}\footnote{Detta recept bygger på \url{http://www.artima.com/pins1ed/object-equality.html}} i 8 steg för att överskugga \code{equals} med innehållslikhet som fungerar även för klasser som inte är \code{final}:

\begin{enumerate}[leftmargin=*]
\item Inför denna metod: \code{ def canEqual(other: Any): Boolean}\\Observera att typen på parametern ska vara \code{Any}. Om detta görs i en subklass till en klass som redan implementerat \code{canEqual}, behövs även \code{override}.

\item Metoden \code{canEqual} ska ge \code{true} om \code{other} är av samma typ som \code{this}, alltså till exempel: \\
\code{def canEqual(other: Any): Boolean = other.isInstanceOf[Complex]}

\item Inför metoden \code{equals} och var noga med att parametern har typen \code{Any}: \\ \code{override def equals(other: Any): Boolean}

\item Implementera metoden \code{equals} med ett match-uttryck som börjar så här: \\
%\code|other match { ... } |
\code|other match |

\item Match-uttrycket ska ha två grenar. Den första grenen ska ha ett typat mönster för den klass som ska jämföras: \\ \code{  case that: Complex =>}

\item Om du implementerar \code{equals} i den klass som inför \code{canEqual}, börja uttrycket med: \\ \code{(that canEqual this) &&} \\
och skapa därefter en fortsättning som baseras på innehållet i klassen, till exempel: \code{this.re == that.re && this.im == that.im} \\
Om du överskuggar en \textit{annan} equals än den standard-equals som finns i \code{AnyRef}, vill du förmodligen börja det logiska uttrycket med att anropa superklassens equals-metod:
 \code{super.equals(that) && } men du får fundera noga på vad likhet av underklasser egentligen ska innebära i ditt speciella fall.

\item Den andra grenen i matchningen ska vara:
\code{case _ => false}

\item Överskugga \code{hashCode}, till exempel genom att göra en tupel av innehållet i klassen och anropa metoden \code{##} på tupeln så får du i en bra hashcode: \\
\code{override def hashCode: Int  = (re, im).## }

\end{enumerate}


\SOLUTION


\TaskSolved \what



\QUESTEND






\WHAT{Överskugga equals vid arv.}

\QUESTBEGIN

\Task  \what~ Bygg vidare på exemplet nedan och överskugga equals vid arv, genom att följa receptet i uppgift \ref{task:equals:Complex}.
\begin{Code}
trait Number:
  override def equals(other: Any): Boolean = ???

class Complex(re: Double, im: Double) extends Number:
  override def equals(other: Any): Boolean = ???

class Rational(numerator: Int, denominator: Int) extends Number:
  override def equals(other: Any): Boolean = ???
\end{Code}


\SOLUTION


\TaskSolved \what



\QUESTEND






\WHAT{Speciella matchningar.}

\QUESTBEGIN

\Task  \what~ Läs om användning av speciella matchningar här: \\
\href{https://dotty.epfl.ch/docs/reference/changed-features/vararg-splices.html}{dotty.epfl.ch/docs/reference/changed-features/vararg-splices.html}

\Subtask Prova variabelbinding med \texttt{@} i en matchning i REPL.

\Subtask Prova sekvensmönster med \texttt{\_} och \texttt{\_*} i en matching i REPL.

\SOLUTION


\TaskSolved \what \TODO



\QUESTEND






\WHAT{Extraktorer.}

\QUESTBEGIN

\Task \label{task:extractor} \what~  Läs mer om extraktorer här: \\ \href{https://dotty.epfl.ch/docs/reference/changed-features/pattern-matching.html}{dotty.epfl.ch/docs/reference/changed-features/pattern-matching.html} \\
Skapa ditt eget extraktor-objekt för http-addresser som i t.ex.: \\
\texttt{http://my.host.domain/path/to/this} \\ extraherar \texttt{my.host.domain} och \texttt{path/to/this} med metoden \texttt{unapply} och testa i en matchning.

%\Task \TODO \emph{flatten och flatMap med Option och Try}
%Ska detta vara ordinarie uppgift eller fördjupning???


%\Task \TODO \emph{partiella funktioner och metoderna collect och collectFirst på samlingar}
%Ska detta vara ordinarie uppgift eller fördjupning???

\SOLUTION


\TaskSolved \what \TODO



\QUESTEND




\WHAT{Polynom, fortsättning: polynomdivision.}

\QUESTBEGIN

\Task  \what~ Implementera polynomdivision på lämpligt sätt genom att bygga vidare på  \code{object polynomial} i  uppgift \ref{task:polynomial} på sidan \pageref{task:polynomial}.  \\ Läs mer om polynomdivision här: \href{https://sv.wikipedia.org/wiki/Polynomdivision}{sv.wikipedia.org/wiki/Polynomdivision}

\SOLUTION


\TaskSolved \what \TODO

\QUESTEND


%!TEX encoding = UTF-8 Unicode
%!TEX root = ../exercises.tex

\ifPreSolution


\Exercise{\ExeWeekSEVEN}\label{exe:W07}

\begin{Goals}
%!TEX encoding = UTF-8 Unicode
%!TEX root = ../compendium2.tex

\item Kunna läsa och skriva pseudokod för sekvensalgoritmer och implementera sekvensalgoritmer enligt pseudokod.

\item Kunna implementera sekvensalgoritmer, både genom kopiering till ny sekvens och genom förändring på plats i befintlig sekvens.

\item Kunna använda inbyggda metoder för uppdatering av, linjärsökning i, och sortering av sekvenssamlingar.

\item Kunna beskriva skillnaden i användningen av föränderliga och oföränderliga sekvenser, speciellt vid uppdatering.

\item Förstå hur sorteringsordningen är definierad för strängar.

\item Kunna sortera sekvenssamlingar innehållande objekt av grundtyper med hjälp av inbyggda och egendefinierade sorteringsordningar med metoderna \code{sorted}, \code{sortBy} och \code{sortWith}.

\item Kunna implementera linjärsökning enligt olika sökkriterier.


\item Kunna beskriva egenskaperna hos sekvenssamlingarna \code{Vector}, \code{List}, \code{Array}, \code{ArrayBuffer} och \code{ListBuffer}.

\item Förstå bieffekter av uppdatering av delade referenser till föränderliga element.

\item Kunna använda funktioner med repeterade parametrar.

\item Känna till hur man implementerar funktioner med repeterade parametrar.

\item Kunna implementera heltalsregistrering i en heltalsarray.

%\item Kunna beskriva skillnader i syntax mellan arrayer i Scala och Java.

%\item Kunna beskriva skillnader i syntax och semantik mellan enkla for-satser i Scala och Java.


%\item Känna till hur klassen \code{java.util.Scanner} kan användas för att skapa heltalssekvenser ur strängsekvenser.

\end{Goals}

\begin{Preparations}
\item \StudyTheory{07}
\end{Preparations}

\else

\ExerciseSolution{\ExeWeekSEVEN}

\fi


\BasicTasks %%%%%%%%%%%



\WHAT{Para ihop begrepp med beskrivning.}

\QUESTBEGIN

\Task \what

\vspace{1em}\noindent Koppla varje begrepp med den (förenklade) beskrivning som passar bäst:

\begin{ConceptConnections}
  mängd & 1 & & A & egenskapen att finnas kvar efter programmets avslut \\ 
  nyckel-värde-tabell & 2 & & B & unika identifierare, associerade med ett enda värde \\ 
  nyckelmängd & 3 & & C & unika element, kan snabbt se om element finns \\ 
  persistens & 4 & & D & koda objekt till avkodningsbar sekvens av symboler \\ 
  serialisera & 5 & & E & för att snabbt hitta tillhörande värde \\ 
  de-serialisera & 6 & & F & avkoda symbolsekvens och återskapa objekt i minnet \\ 
\end{ConceptConnections}

\SOLUTION

\TaskSolved \what

\begin{ConceptConnections}
  mängd & 1 & ~~\Large$\leadsto$~~ &  C & unika element, kan snabbt se om element finns \\ 
  nyckel-värde-tabell & 2 & ~~\Large$\leadsto$~~ &  E & för att snabbt hitta tillhörande värde \\ 
  nyckelmängd & 3 & ~~\Large$\leadsto$~~ &  B & unika identifierare, associerade med ett enda värde \\ 
  persistens & 4 & ~~\Large$\leadsto$~~ &  A & egenskapen att finnas kvar efter programmets avslut \\ 
  serialisera & 5 & ~~\Large$\leadsto$~~ &  D & koda objekt till avkodningsbar sekvens av symboler \\ 
  de-serialisera & 6 & ~~\Large$\leadsto$~~ &  F & avkoda symbolsekvens och återskapa objekt i minnet \\ 
\end{ConceptConnections}

\QUESTEND



\WHAT{Olika sekvenssamlingar.}

\QUESTBEGIN

\Task \what~Koppla varje sekvenssamling med den (förenklade) beskrivning som passar bäst:

\begin{ConceptConnections}
\input{generated/quiz-w07-seq-collections-taskrows-generated.tex}
\end{ConceptConnections}

\SOLUTION

\TaskSolved \what

\begin{ConceptConnections}
\input{generated/quiz-w07-seq-collections-solurows-generated.tex}
\end{ConceptConnections}

\QUESTEND



% This task has been removed because it didn't make much sense anymore after the removal of Traversable in Scala 2.13. https://github.com/lunduniversity/introprog/issues/497
%
%\WHAT{Typer i hierarkin av sekvenssamlingar.}
%
%\QUESTBEGIN
%
%\Task \what~Koppla varje typ i hierarkin av sekvenssamling %med den (förenklade) beskrivning som passar bäst:
%
%\begin{ConceptConnections}
%\input{generated/quiz-w07-abstract-collections-taskrows-generated.tex}
%\end{ConceptConnections}
%
%\SOLUTION
%
%\TaskSolved \what
%
%\begin{ConceptConnections}
%\input{generated/quiz-w07-abstract-collections-solurows-generated.tex}
%\end{ConceptConnections}
%
%\QUESTEND


\WHAT{Använda sekvenssamlingar.}

\QUESTBEGIN

\Task \what~Antag att nedan variabler finns synliga i aktuell namnrymd:
\begin{Code}
val xs: Vector[Int] = Vector(1, 2, 3)
val x: Int = 0
\end{Code}

\Subtask Koppla varje uttryck till vänster med motsvarande resultat till höger. Om du är osäker på resultatet, läs i snabbreferensen och testa i REPL. \\\emph{Tips: ''colon on the collection side''}.

\begin{ConceptConnections}
  \code|x +: xs         | & 1 & & A & \code|true                                    | \\ 
  \code|xs +: x         | & 2 & & B & \code|Vector(2, 2, 3)                         | \\ 
  \code|xs :+ x         | & 3 & & C & \code|1                                       | \\ 
  \code|xs ++ xs        | & 4 & & D & \code|value tail is not a member of Int       | \\ 
  \code|xs.indices      | & 5 & & E & \code|Range 0 until 3                         | \\ 
  \code|xs apply 0      | & 6 & & F & \code|Vector(1, 2, 3)                         | \\ 
  \code|xs(3)           | & 7 & & G & \code|Vector(0, 1, 2, 3)                      | \\ 
  \code|xs.length       | & 8 & & H & \code|false                                   | \\ 
  \code|xs.take(4)      | & 9 & & I & \code|java.lang.IndexOutOfBoundsException     | \\ 
  \code|xs.drop(2)      | & 10 & & J & \code|Vector(1, 2, 3, 0)                      | \\ 
  \code|xs.updated(0, 2)| & 11 & & K & \code|Vector(3)                               | \\ 
  \code|xs.tail.head    | & 12 & & L & \code|value +: is not a member of Int         | \\ 
  \code|xs.head.tail    | & 13 & & M & \code|Vector(1, 2, 3, 1, 2, 3)                | \\ 
  \code|xs.isEmpty      | & 14 & & N & \code|2                                       | \\ 
  \code|xs.nonEmpty     | & 15 & & O & \code|3                                       | \\ 
\end{ConceptConnections}

\Subtask Vid tre tillfällen blir det fel. Varför? Är det kompileringsfel eller exekveringsfel?

\begin{framed}
\noindent\emph{Tips inför fortsättningen:}
Scalas standardbibliotek har många användbara samlingar med enhetlig metoduppsättning. Om du lär dig de viktigaste samlingsmetoderna får du en kraftfull verktygslåda. Läs mer här:

    \begin{itemize}%[nolistsep]
      \item snabbreferensen (enda tentahjälpmedel): \\{\small\url{http://cs.lth.se/pgk/quickref}}
      \item översikt (av Prof. Martin Odersky, uppfinnare av Scala, m.fl.): \\
       {\small\url{http://docs.scala-lang.org/overviews/collections/introduction.html}}
      \item api-dokumentation:\\  {\small\url{https://www.scala-lang.org/api/current/scala/collection/}}
    \end{itemize}
\end{framed}

\SOLUTION

\TaskSolved \what

\SubtaskSolved

\begin{ConceptConnections}
  \code|x +: xs         | & 1 & ~~\Large$\leadsto$~~ &  G & \code|Vector(0, 1, 2, 3)                      | \\ 
  \code|xs +: x         | & 2 & ~~\Large$\leadsto$~~ &  L & \code|value +: is not a member of Int         | \\ 
  \code|xs :+ x         | & 3 & ~~\Large$\leadsto$~~ &  J & \code|Vector(1, 2, 3, 0)                      | \\ 
  \code|xs ++ xs        | & 4 & ~~\Large$\leadsto$~~ &  M & \code|Vector(1, 2, 3, 1, 2, 3)                | \\ 
  \code|xs.indices      | & 5 & ~~\Large$\leadsto$~~ &  E & \code|Range 0 until 3                         | \\ 
  \code|xs apply 0      | & 6 & ~~\Large$\leadsto$~~ &  C & \code|1                                       | \\ 
  \code|xs(3)           | & 7 & ~~\Large$\leadsto$~~ &  I & \code|java.lang.IndexOutOfBoundsException     | \\ 
  \code|xs.length       | & 8 & ~~\Large$\leadsto$~~ &  O & \code|3                                       | \\ 
  \code|xs.take(4)      | & 9 & ~~\Large$\leadsto$~~ &  F & \code|Vector(1, 2, 3)                         | \\ 
  \code|xs.drop(2)      | & 10 & ~~\Large$\leadsto$~~ &  K & \code|Vector(3)                               | \\ 
  \code|xs.updated(0, 2)| & 11 & ~~\Large$\leadsto$~~ &  B & \code|Vector(2, 2, 3)                         | \\ 
  \code|xs.tail.head    | & 12 & ~~\Large$\leadsto$~~ &  N & \code|2                                       | \\ 
  \code|xs.head.tail    | & 13 & ~~\Large$\leadsto$~~ &  D & \code|value tail is not a member of Int       | \\ 
  \code|xs.isEmpty      | & 14 & ~~\Large$\leadsto$~~ &  H & \code|false                                   | \\ 
  \code|xs.nonEmpty     | & 15 & ~~\Large$\leadsto$~~ &  A & \code|true                                    | \\ 
\end{ConceptConnections}

\SubtaskSolved

\noindent\renewcommand*{\arraystretch}{1.2}\begin{tabular}{p{5cm} l p{6cm}}

~\\ \emph{fel} & \emph{typ} & \emph{förklaring} \\\hline

\code|value +: is not| \code|a member of Int|
& kompileringsfel
& Operatorer som slutar med kolon är högerassociativa. Metodanropet \code|xs +: x| motsvarar med punktnotation \code|x.+:(xs)| och det finns ingen metod med namnet \code|+:| på heltal.\\\hline

\code|IndexOutOfBoundsException|
& körtidsfel & Det finns bara 3 element och index räknas från 0 i sekvenssamlingar.\\\hline

\code|value tail is not| \code|a member of Int|
& kompileringsfel
& Metoden \code|head| ger första elementet och heltal saknar sekvenssamlingsmetoden \code|tail|.\\\hline

\end{tabular}


\QUESTEND


\WHAT{Kopiering av sekvenser.}

\QUESTBEGIN

\Task \what~ %\code{map} \code{toArray} \code{copyToArray}
Klassen \code{Mutant} nedan kan användas för att skapa förändringsbara instanser med heltal.\footnote{Om den inbyggda grundtypen Int, i likhet med \code{Mutant}, knasigt nog  kunnat användas för att skapa förändringsbara instanser hade heltalsmatematiken i Scala omvandlats till ett skrämmande kaos.
%\\Lär mer om fem här: \url{https://www.youtube.com/watch?v=dpdOUEe9mm4}
}

\noindent\begin{minipage}{0.6\textwidth}
\begin{Code}[basicstyle=\ttfamily\large\selectfont]
class Mutant(var int: Int = 0)
\end{Code}
\end{minipage}
\hfill\begin{minipage}{0.38\textwidth}
%https://www.1001freedownloads.com/free-clipart/mutant
\centering\includegraphics[width=3.4cm]{../img/mutant.png}
\captionof{figure}{En instans av klassen Mutant där \code{int} kanske är 5.}
%https://tex.stackexchange.com/questions/55337/how-to-use-figure-inside-a-minipage
\end{minipage}

\vspace{1em}\noindent Kör nedan i REPL efter studier av detta:  \url{https://youtu.be/dpdOUEe9mm4}
\begin{REPL}
scala> val fem = new Mutant(5)
scala> val xs = Vector(fem, fem, fem)
scala> val ys = xs.toArray    // kopierar referenserna till ny Array
scala> val zs = xs.map(x => new Mutant(x.int)) // djupkopierar till ny Vector
scala> xs(0).int = (new Mutant).int
\end{REPL}
\Subtask Fyll i tabellen nedan genom att till höger skriva värdet av varje uttryck till vänster. Förklara vad som händer. \emph{Tips:} Metoden \code{eq} jämför alltid referenser (ej innehåll).

\renewcommand{\arraystretch}{2.0}
\vspace{1em}\noindent\begin{tabular}{@{} l | p{5.5cm}}\hline
\code|xs(0)         | & \\\hline
\code|ys(0).int| & \\\hline
\code|zs(0).int| & \\\hline
\code|xs(0) eq ys(0)| & \\\hline
\code|xs(0) eq zs(0)| & \\\hline
\code|(ys.toBuffer :+ new Mutant).apply(0).int| & \\\hline
\end{tabular}

\Subtask Implementera med hjälp av en \code{while}-sats funktionen \code{deepCopy} nedan som gör \emph{djup} kopiering, d.v.s skapar en ny array med nya, innehållskopierade mutanter.
\begin{Code}
def deepCopy(xs: Array[Mutant]): Array[Mutant] = ???
\end{Code}
Använd denna algoritm:

\begin{algorithm}[H]
 \SetKwInOut{Input}{Indata}\SetKwInOut{Output}{Resultat}

 \Input{ ~En mutantarray $xs$}
 \Output{ ~En djup kopia av $xs$}
 $result \leftarrow$ en ny mutantarray med plats för lika många element som i $xs$\\
 $i \leftarrow 0$  \\
 \While{$i$ mindre än antalet element}{
  skapa en kopia av elementet $xs(i)$ och lägg kopian i $result$ på platsen $i$ \\
  öka $i$ med 1
 }
 \Return $result$
\end{algorithm}

\Subtask Testa att din funktion och kolla så att inga läskiga muteringar genom delade referenser går att göra, så som med \code|xs| och \code|ys| i första deluppgiften.

\Subtask Är det vanligt att man, för säkerhets skull, gör djupkopiering av alla element i oföränderliga samlingar som enbart innehåller oföränderliga element?

\SOLUTION

\TaskSolved \what~

\SubtaskSolved

\renewcommand{\arraystretch}{1.5}
\vspace{1em}\noindent\begin{tabular}{@{} p{0.4\textwidth} p{0.6\textwidth}}\hline
\code|xs(0)| & \code|rs$line5$Mutant@66d766b9 | nya instanser får nya hexkoder \\ \hline 
\code|ys(0).int               | & \code|0 | eftersom \code|ys| innehåller samma instans som \code|xs|\\ \hline
\code|zs(0).int               | & \code|5 | eftersom \code|!(xs(0) eq zs(0))| \\ \hline
\code|xs(0) eq ys(0)          | & \code|true |  eftersom samma instans \\ \hline
\code|xs(0) eq zs(0)          | & \code|false | eftersom olika instanser\\ \hline
\code|(ys.toBuffer :+ |
\code|  new Mutant).apply(0).int| & \code|0 | eftersom den ej djupkopierade kopian av typen \code|ArrayBuffer| refererar samma instans på första platsen som både \code|ys| och \code|xs| och \code|x(0).int| blev noll i en tilldelning på rad 5 i REPL-körningen\\ \hline
\end{tabular}

\vspace{0.5em}\noindent Observera alltså att kopiering med \code{toArray}, \code{toVector}, \code{toBuffer}, etc. \emph{inte är djup}, d.v.s. det är bara instansreferenserna som kopieras och inte själva instanserna.


\SubtaskSolved
\begin{CodeSmall}
def deepCopy(xs: Array[Mutant]): Array[Mutant] =
  val result = Array.ofDim[Mutant](xs.length) //fylld med null-referenser
  var i = 0
  while i < xs.length do
    result(i) = new Mutant(xs(i).int) //kopia med samma innehåll på samma plats
    i += 1
  result
\end{CodeSmall}
Det går också bra att skapa resultatarrayen med \code{new Array[Mutant](xs.length)}.
Du kan också använda \code{size} i stället för \code{length}.

\SubtaskSolved
\begin{REPL}
scala> class Mutant(var int: Int = 0)
// defined class Mutant

scala> def deepCopy(xs: Array[Mutant]): Array[Mutant] =
     |   val result = Array.ofDim[Mutant](xs.length)
     |   var i = 0
     |   while i < xs.length do
     |     result(i) = new Mutant(xs(i).int)
     |     i += 1
     |   result

scala> val xs = Array.fill(3)(new Mutant)
xs: Array[Mutant] = Array(rs$line$2$Mutant@46a123e4, rs$line$2$Mutant@44bc2449,
rs$line2$Mutant@3c28e5b6)

scala> val ys = deepCopy(xs)
ys: Array[Mutant] = Array(rs$line$2$Mutant@14b8a751, rs$line2$Mutant@7345f97d,
rs$line$2$Mutant@554566a8)

scala> xs(0).int = 5

scala> ys(0).int
val res0: Int = 0
\end{REPL}

\SubtaskSolved Nej, eftersom elementen inte kan förändras kan man utan problem dela referenser mellan samlingar. Det finns inte någon möjlighet att det kan ske förändringar som påverkar flera samlingar samtidigt.
Dock gör man vanligen (ofta tidsödande) djupkopieringar av samlingar med förändringsbara element för att kunna vara säker på att den ursprungliga samlingen inte förändras.

\QUESTEND



\ifPreSolution
\begin{framed}
\noindent\emph{Tips inför fortsättningen:} Ofta kan du lösa grundläggande delproblem med inbyggda samlingsmetoder ur standardbiblioteket. Till exempel kan ju kopieringen i \code{deepCopy} i föregående uppgift enkelt göras med hjälp av samlingsmetoden \code{map}.

Men det är mycket bra för din förståelse om du kan implementera grundläggande sekvensalgoritmer själv även om det normalt är bättre att använda färdiga, vältestade  metoder. I kommande uppgifter ska du därför göra egna implementationer av några sekvensalgoritmer som redan finns i standardbiblioteket.
\end{framed}
\fi



\WHAT{Uppdatering av sekvenser.}

\QUESTBEGIN

\Task \what~Deklarera dessa variabler i REPL:

\begin{Code}
val xs = (1 to 4).toVector
val buf = xs.toBuffer
\end{Code}

\Subtask Uttrycken till vänster evalueras uppifrån och ned. Para ihop med rätt resultat.

\begin{ConceptConnections}
  \code|{ buf(0) = -1; buf(0) }   | & 1 & & A & {\small\code|value update is not a member|} \\ 
  \code|{ xs(0) = -1; xs(0) }| & 2 & & B & \code|Vector(5, 2, 3, 4)| \\ 
  \code|buf.update(1, 5)          | & 3 & & C & \code|ArrayBuffer(-1, 5, 3, 4, 5)| \\ 
  \code|xs.updated(0, 5)          | & 4 & & D & \code|-1| \\ 
  \code|buf += 5                 | & 5 & & E & \code|Vector(1, -1, 5)| \\ 
  \code|xs += 5                  | & 6 & & F & \code|(): Unit| \\ 
  \code|xs.patch(1,Vector(-1,5),3)| & 7 & & G & {\small\code|value += is not a member|} \\ 
  \code|xs                        | & 8 & & H & \code|Vector(1, 2, 3, 4)|
\end{ConceptConnections}

\smallskip
\emph{Tips:} Läs om metoderna i snabbreferensen och undersök i REPL. Exempel:
\begin{REPL}
scala> Vector(1,2,3,4).patch(from = 1, other = Vector(0,0), replaced = 3)
val res0: Vector[Int] = Vector(1, 0, 0)
\end{REPL}

\Subtask Implementera funktionen \code{insert} nedan med hjälp av sekvenssamlingsmetoden \code{patch}. \emph{Tips:} Ge argumentet \code{0} till parametern \code{replaced}.
\begin{Code}
/** Skapar kopia av xs men med elem insatt på plats pos. */
def insert(xs: Array[Int], elem: Int, pos: Int): Array[Int] = ???
\end{Code}

\Subtask Skriv pseduokod för en algoritm som implementerar \code{insert} med hjälp av \code{while}.

\Subtask Implementera \code{insert} enligt din pseudokod. Testa i REPL och se vad som händer om \code{pos} är negativ? Vad händer om \code{pos} är precis ett steg bortom sista platsen i \code{xs}? Vad händer om \code{pos} är flera steg bortom sista platsen?

\SOLUTION

\TaskSolved \what~

\SubtaskSolved

\begin{ConceptConnections}
  \code|{ buf(0) = -1; buf(0) }   | & 1 & ~~\Large$\leadsto$~~ &  D & \code|-1| \\ 
  \code|{ xs(0) = -1; xs(0) }| & 2 & ~~\Large$\leadsto$~~ &  A & {\small\code|value update is not a member|} \\ 
  \code|buf.update(1, 5)          | & 3 & ~~\Large$\leadsto$~~ &  F & \code|(): Unit| \\ 
  \code|xs.updated(0, 5)          | & 4 & ~~\Large$\leadsto$~~ &  B & \code|Vector(5, 2, 3, 4)| \\ 
  \code|buf += 5                | & 5 & ~~\Large$\leadsto$~~ &  C & \code|ArrayBuffer(-1, 5, 3, 4, 5)| \\ 
  \code|xs += 5                 | & 6 & ~~\Large$\leadsto$~~ &  G & {\small\code|value += is not a member|} \\ 
  \code|xs.patch(1,Vector(-1,5),3)| & 7 & ~~\Large$\leadsto$~~ &  E & \code|Vector(1, -1, 5)| \\ 
  \code|xs                        | & 8 & ~~\Large$\leadsto$~~ &  H & \code|Vector(1, 2, 3, 4)| 
\end{ConceptConnections}

\SubtaskSolved

\begin{Code}
def insert(xs: Array[Int], elem: Int, pos: Int): Array[Int] =
  xs.patch(from = pos, other = Array(elem), replaced = 0)
\end{Code}

\SubtaskSolved Pseudokoden nedan är skriven så att den kompilerar fast den är ofärdig.
\begin{Code}
def insert(xs: Array[Int], elem: Int, pos: Int): Array[Int] = 
  val result = ??? /* ny array med plats för ett element mer än i xs */
  var i = 0
  while(???){/* kopiera elementen före plats pos och öka i */}
  if i < result.length then /* lägg elem i result på plats i */
  while(???){/* kopiera över resten */}
  result

\end{Code}

\SubtaskSolved
\begin{Code}
def insert(xs: Array[Int], elem: Int, pos: Int): Array[Int] = 
  val result = new Array[Int](xs.length + 1)
  var i = 0
  while i < pos && i < xs.length do  { result(i) = xs(i); i += 1}
  if i < result.length then { result(i) = elem; i += 1 }
  while i < result.length do { result(i) = xs(i - 1); i += 1}
  result

\end{Code}
\begin{REPL}
scala> insert(Array(1, 2), 0, pos = -1)
val res2: Array[Int] = Array(0, 1, 2)

scala> insert(Array(1, 2), 0, pos = 0)
val res3: Array[Int] = Array(0, 1, 2)

scala> insert(Array(1, 2), 0, pos = 1)
val res4: Array[Int] = Array(1, 0, 2)

scala> insert(Array(1, 2), 0, pos = 2)
val res5: Array[Int] = Array(1, 2, 0)

scala> insert(Array(1, 2), 0, pos = 42)
val res7: Array[Int] = Array(1, 2, 0)
\end{REPL}

\QUESTEND




\ifPreSolution
\begin{framed}
\noindent\emph{Tips inför fortsättningen:} Det är inte lätt att få rätt på alla specialfall även i små algoritmer så som \code{insert} ovan. Det är därför viktigt att noga tänka igenom sin sekvensalgoritm med avseende på olika specialfall. Använd denna checklista:
\begin{enumerate}[noitemsep]
  \item Vad händer om sekvensen är tom?
  \item Fungerar det för exakt ett element?
  \item Kan index bli negativt?
  \item Kan index bli mer än längden minus ett?
  \item Kan det bli en oändlig loop, t.ex. p.g.a. saknad loopvariabeluppräkning?
\end{enumerate}
Ibland vill man att vettiga undantag ska kastas vid ogiltig indata eller andra feltillstånd och då är \code{require} eller \code{assert} bra att använda. I andra fall vill man att resultatet t.ex. ska bli en tom sekvenssamling om indata är ogiltigt. Sådana beteenden behöver dokumenteras så att andra som använder dina algoritmer (eller du själv efter att du glömt hur det var) förstår vad som händer i olika fall.


\end{framed}
\fi

\WHAT{Jämföra strängar i Scala.}

\QUESTBEGIN

\Task \label{task:string-order-operators} \what~  I Scala kan strängar jämföras med operatorerna \code{==}, \code{!=}, \code{<}, \code{<=}, \code{>}, \code{>=},  där likhet/olikhet avgörs av om alla tecken i strängen är lika eller inte, medan större/mindre avgörs av sorteringsordningen i enlighet med varje teckens Unicode-värde.\footnote{Överkurs: Alla tecken i en \code{java.lang.String} representeras enligt UTF-16-standarden (\href{https://en.wikipedia.org/wiki/UTF-16}{https://en.wikipedia.org/wiki/UTF-16}), vilket innebär att varje Unicode-kodpunkt \Eng{code point} lagras som antingen ett eller två 16-bitars heltal. Strängjämförelse i Scala och Java jämför egentligen inte varje tecken, utan varje 16-bitars heltal. Denna skillnad har ingen betydelse när en sträng bara innehåller tecken som tar upp ett 16-bitars heltal var, och praktiskt nog är nästan alla tecken som används vardagligen av den typen. De flesta tecken som kräver två 16-bitars heltal är sällsynta kinesiska tecken, sällsynta symboler, tecken från utdöda språk och emoji. Vi kommer att bortse från sådana tecken i den här kursen.}

\Subtask Vad ger följande jämförelser för värde?
\begin{REPL}
scala> 'a' < 'b'
scala> "aaa" < "aaaa"
scala> "aaa" < "bbb"
scala> "AAA" < "aaa"
scala> "ÄÄÄ" < "ÖÖÖ"
scala> "ÅÅÅ" < "ÄÄÄ"
\end{REPL}
Tyvärr så följer ordningen av ÄÅÖ inte svenska regler, men det ignorerar vi i fortsättningen för enkelhets skull; om du är intresserad av hur man kan fixa  detta, gör uppgift \ref{task:swedish-letter-ordering}.

\Subtask\Pen Vilken av strängarna $s1$ och $s2$ kommer först (d.v.s. är ''mindre'') om $s1$ utgör början av $s2$ och $s2$ innehåller fler tecken än $s1$?


\SOLUTION


\TaskSolved \what


\SubtaskSolved
\begin{REPL}
true
true
true
true
true
false
\end{REPL}

\SubtaskSolved
\emph{s1} kommer först.


\QUESTEND




\WHAT{Linjärsökning enligt olika sökkriterier.}

\QUESTBEGIN

\Task \what~Linjärsökning innebär att man letar tills man hittar det man söker efter i en sekvens. Detta delproblem återkommer ofta! Vanligen börjar linjärsökning från början och håller på tills man hittar något element som uppfyller kriteriet. Beroende på vad som finns i sekvensen och hur kriteriet ser ut kan det hända att man måste gå igenom alla element utan att hitta det som söks.

\Subtask Linjärsökning med inbyggda sekvenssamlingsmetoder.
\begin{Code}
val xs = ((1 to 5).reverse ++ (0 to 5)).toVector
\end{Code}
Deklarera ovan variabel i REPL och para ihop uttrycken nedan med rätt värden. Förklara vad som händer.

\begin{ConceptConnections}
\input{generated/quiz-w07-seq-find-taskrows-generated.tex}
\end{ConceptConnections}

\Subtask Implementera linjärsökning i strängvektor med strängpredikat.
\begin{Code}
/** Returns first index where p is true. Returns -1 if not found. */
def indexOf(xs: Vector[String], p: String => Boolean): Int = ???
\end{Code}
Ett strängpredikat \code{p: String => Boolean} är en funktion som tar en sträng som indata och ger ett booleskt värde som resultat. Implementera \code{indexOf} med hjälp av en \code{while}-sats. Du kan t.ex. använda en lokal boolesk variabel \code{found} för att hålla reda på om du har hittat det som eftersöks enligt predikatet.

När element som uppfyller predikatet saknas måste man bestämma vad som ska hända. Kravet på din implementation i detta fall ges av dokumentationskommentaren ovan.

Din funktion ska fungera enligt nedan:
\begin{REPL}
scala> val xs = Vector("hej", "på", "dej")
val xs: Vector[String] = Vector(hej, på, dej)

scala> indexOf(xs, _.contains('p'))
val res0: Int = 1

scala> indexOf(xs, _.contains('q'))
val res1: Int = -1

scala> indexOf(Vector(), _.contains('q'))
val res2: Int = -1

scala> indexOf(Vector("q"), _.length == 1)
val res3: Int = 0
\end{REPL}

\SOLUTION

\TaskSolved \what~

\SubtaskSolved

\begin{ConceptConnections}
\input{generated/quiz-w07-seq-find-solurows-generated.tex}
\end{ConceptConnections}

\SubtaskSolved Med en boolesk variabel \code{found}:

\begin{Code}
def indexOf(xs: Vector[String], p: String => Boolean): Int = 
  var found = false
  var i = 0
  while i < xs.length && !found do
      found = p(xs(i))
      i += 1
  if found then i - 1 else -1
\end{Code}
Eller utan \code{found}:
\begin{Code}
def indexOf(xs: Vector[String], p: String => Boolean): Int = 
  var i = 0
  while i < xs.length && !p(xs(i)) do i += 1
  if i == xs.length then -1 else i
\end{Code}
Eller så kanske man vill börja bakifrån; lösningen nedan är nog enklare att fatta (?) och definitivt mer koncis, men uppfyller \emph{inte} kravet att returnera index för \emph{första} förekomsten som det står i uppgiften. Men om sammanhanget tillåter att vi returnerar \emph{något} index för vilket predikatet gäller, eller om man faktiskt har kravet att leta bakifrån, så funkar detta:
\begin{Code}
def indexOf(xs: Vector[String], p: String => Boolean): Int = 
  var i = xs.length - 1
  while i >= 0 && !p(xs(i)) do i -= 1
  i
\end{Code}
Eller så kan man göra på flera andra sätt. När du ska implementera algoritmer, både på programmeringstentan och i yrkeslivet som systemutvecklare, finns det ofta många olika sätt att lösa uppgiften på som har olika egenskaper, fördelar och nackdelar. Det viktiga är att lösningen fungerar så gott det går enligt kraven, att koden är begriplig för människor och att implementationen inte är så ineffektiv att användarna tröttnar i sin väntan på resultatet...

\QUESTEND




\WHAT{Labbförberedelse: Implementera heltalsregistrering i Array.}

\QUESTBEGIN

\Task \what~Registrering innebär att man räknar antalet förekomster av olika värden. Varje gång ett nytt värde förekommer behöver vi räkna upp en frekvensräknare. Det behövs en räknare för varje värde som ska registreras. Vi ska fortsätta räkna ända tills alla värden är registrerade.

På veckans laboration ska du registrera förekomsten av olika kortkombinationer i kortspelet poker. I denna övning ska du som träning inför laborationen lösa ett liknande registreringsproblem:  frekvensanalys av många tärningskast. Vid tärningsregistrering behövs sex olika räknare. Man kan med fördel då använda en sekvenssamling med plats för sex heltal. Man kan t.ex. låta  plats \code{0} håller reda på antalet ettor, plats \code{1} hålla reda på antalet tvåor, etc.

\Subtask Implementera nedan algoritm enligt pseudokoden:
\begin{Code}
def registreraTärningskast(xs: Seq[Int]): Vector[Int] = 
  val result = ??? /* Array med 6 nollor */
  xs.foreach{ x =>
    require(x >= 1 && x <= 6, "tärningskast ska vara mellan 1 & 6")
    ??? /* räkna förekomsten av x */
  }
  result.toVector
\end{Code}

\Subtask Använd funktionen \code{kasta} nedan när du testar din registreringsalgoritm med en sekvenssamling innehållande minst $1000$ tärningskast.
\begin{Code}
def kasta(n: Int) = Vector.fill(n)(util.Random.nextInt(6) + 1)
\end{Code}

\SOLUTION

\TaskSolved \what~

\SubtaskSolved
\begin{Code}
def registreraTärningskast(xs: Seq[Int]): Vector[Int] = 
  val result = Array.fill(6)(0)
  xs.foreach{ x =>
    require(x >= 1 && x <= 6, "tärningskast ska vara mellan 1 & 6")
    result(x - 1) += 1
  }
  result.toVector
\end{Code}

\SubtaskSolved
\begin{REPL}
scala> registreraTärningskast(kasta(1000))
val res0: Vector[Int] = Vector(171, 163, 166, 152, 184, 164)

scala> registreraTärningskast(kasta(1000))
val res1: Vector[Int] = Vector(163, 161, 158, 174, 161, 183)
\end{REPL}

\QUESTEND




\WHAT{Inbyggda metoder för sortering.}

\QUESTBEGIN

\Task \what~Det finns fler olika sätt att ordna sekvenser efter olika kriterier. För  grundtyperna \code{Int}, \code{Double}, \code{String}, etc., finns inbyggda ordningar som gör att sekvenssamlingsmetoden \code{sorted} fungerar utan vidare argument (om du är nöjd med den inbyggda ordningsdefinitionen). Det finns också metoderna \code{sortBy} och \code{sortWith} om du vill ordna en sekvens med element av någon grundtyp efter egna ordningsdefinitioner eller om du har egna klasser i din sekvens.
\begin{Code}
val xs = Vector(1, 2, 1, 3, -1)
val ys = Vector("abra", "ka", "dabra").map(_.reverse)
val zs = Vector('a', 'A', 'b', 'c').sorted

case class Person(förnamn: String, efternamn: String)

val ps = Vector(Person("Kim", "Ung"), Person("kamrat", "Clementin"))
\end{Code}
Deklarera ovan i REPL och para ihop uttryck nedan med rätt resultat.
\\\emph{Tips:} Stora bokstäver sorteras före små bokstäver i den inbyggda ordningen för grundtyperna \code{String} och \code{Char}. Dessutom har svenska tecken knasig ordning.\footnote{Ordningen kommer ursprungligen från föråldrade teckenkodningsstandarder:    \url{https://sv.wikipedia.org/wiki/ASCII}}
\\Läs om sorteringsmetoderna i snabbreferensen och prova i REPL.

\begin{ConceptConnections}
  \code|'a' < 'A'                  | & 1 & & A & \code|"ka"| \\ 
  \code|"AÄÖö" < "AÅÖö"        | & 2 & & B & \code|1| \\ 
  \code|xs.sorted.head             | & 3 & & C & \code|-1| \\ 
  \code|xs.sorted.reverse.head     | & 4 & & D & \code|error: ...| \\ 
  \code|ys.sorted.head             | & 5 & & E & \code|false| \\ 
  \code|zs.indexOf('a')            | & 6 & & F & \code|0| \\ 
  \code|ps.sorted.head.förnamn.take(2)| & 7 & & G & \code|3| \\ 
  \code|ps.sortBy(_.förnamn).apply(1).förnamn.take(2)| & 8 & & H & \code|true| \\ 
  \code|xs.sortWith((x1, x2) => x1 > x2).indexOf(3)| & 9 & & I & \code|"ak"| 
\end{ConceptConnections}
Vi ska senare i kursen implementera egna sorteringsalgoritmer som träning, men i normala fall använder man inbyggda sorteringar som är effektiva och vältestade. Dock är det inte ovanligt att man vill definiera egna ordningar för egna klasser, vilket vi ska undersöka senare i kursen.

\SOLUTION

\TaskSolved \what

\begin{ConceptConnections}
  \code|'a' < 'A'                  | & 1 & ~~\Large$\leadsto$~~ &  E & \code|false| \\ 
  \code|"AÄÖö" < "AÅÖö"        | & 2 & ~~\Large$\leadsto$~~ &  H & \code|true| \\ 
  \code|xs.sorted.head             | & 3 & ~~\Large$\leadsto$~~ &  C & \code|-1| \\ 
  \code|xs.sorted.reverse.head     | & 4 & ~~\Large$\leadsto$~~ &  G & \code|3| \\ 
  \code|ys.sorted.head             | & 5 & ~~\Large$\leadsto$~~ &  I & \code|"ak"| \\ 
  \code|zs.indexOf('a')            | & 6 & ~~\Large$\leadsto$~~ &  B & \code|1| \\ 
  \code|ps.sorted.head.förnamn.take(2)| & 7 & ~~\Large$\leadsto$~~ &  D & \code|error: ...| \\ 
  \code|ps.sortBy(_.förnamn).apply(1).förnamn.take(2)| & 8 & ~~\Large$\leadsto$~~ &  A & \code|"ka"| \\ 
  \code|xs.sortWith((x1, x2) => x1 > x2).indexOf(3)| & 9 & ~~\Large$\leadsto$~~ &  F & \code|0| 
\end{ConceptConnections}
Det blir fel i uttrycket ovan som försöker sortera en sekvens med instanser av \code{Person} direkt med metoden \code{sorted}:
\begin{REPL}
scala> ps.sorted
No implicit Ordering defined for Person.
\end{REPL}
Det blir fel eftersom kompilatorn inte hittar någon ordningsdefinition för dina egna klasser. Senare i kursen ska vi se hur vi kan skapa egna ordningar om man vill få \code{sorted} att fungera på sekvenser med instanser av egna klasser, men ofta räcker det fint med \code{sortBy} och \code{sortWith}.
\QUESTEND


\WHAT{Inbyggd metod för blandning.}

\QUESTBEGIN
\Task \what~På veckans laboration ska du implementera en egen blandningsalgoritm och använda den för att blanda en kortlek. Det finns redan en inbygg metod \code{shuffle} i singelobjektet \code{Random} i paketet \code{scala.util}.

\Subtask Sök upp dokumentationen för \code{Random.shuffle} och studera funktionshuvudet. Det står en hel del invecklade saker om \code{CanBuildFrom} etc. Detta smarta krångel, som vi inte går närmare in på i denna kurs, är till för att metoden ska kunna returnera lämplig typ av samling. När du ser ett sådant funktionshuvud kan du anta att metoden fungerar fint med flera olika typer av lämpliga samlingar i Scalas standardbibliotek.

Klicka på \code{shuffle}-dokumentationen så att du ser hela texten. Vad säger dokumentationen om resultatet? Är det blandning på plats eller blandning till ny samling?

\Subtask Prova upprepade blandningar av olika typer av sekvenser med olika typer av element i REPL.

\SOLUTION

\TaskSolved \what~

\SubtaskSolved \code{Random.shuffle} returnerar en ny blandad sekvenssamling av samma typ. Ordningen i den ursprungliga samlingen påverkas inte.

\SubtaskSolved Exempel på användning av \code{random.shuffle}:
\begin{REPL}
scala> import scala.util.Random

scala> val xs = Vector("Sten", "Sax", "Påse")
val xs: Vector[String] = Vector(Sten, Sax, Påse)

scala> (1 to 10).foreach(_ => println(Random.shuffle(xs).mkString(" ")))
Sax Påse Sten
Sten Påse Sax
Sten Sax Påse
Sten Sax Påse
Sten Påse Sax
Sten Påse Sax
Sax Sten Påse
Sten Påse Sax
Sax Påse Sten
Sax Påse Sten

scala> (1 to 5).map(_ => Random.shuffle(1 to 6))
val res1: IndexedSeq[IndexedSeq[Int]] =
  Vector(Vector(5, 2, 1, 4, 3, 6), Vector(6, 5, 4, 2, 1, 3),
  Vector(3, 1, 4, 6, 5, 2), Vector(3, 2, 6, 5, 1, 4),
  Vector(5, 3, 4, 6, 1, 2))

scala> (1 to 1000).map(_ => Random.shuffle(1 to 6).head).count(_ == 6)
val res2: Int = 168
\end{REPL}

\QUESTEND



\WHAT{Repeterade parametrar.}

\QUESTBEGIN

\Task  \what~  Det går att deklarera en funktion som tar en argumentsekvens av godtycklig längd, ä.k. \emph{varargs}. Syntaxen består av en asterisk \code{*} efter typen. Funktion sägs då ha repeterade parametrar \Eng{repeated parameters}. I funktionskroppen får man tillgång till argumenten i en sekvenssamling. Argumenten anges godtyckligt många med komma emellan. Exempel:
\begin{Code}
/** Ger en vektor med stränglängder för godtyckligt antal strängar. */
def stringSizes(xs: String*): Vector[Int] = xs.map(_.size).toVector
\end{Code}

\Subtask Deklarera och använd \code{stringSizes} i REPL. Vad händer om du anropar \code{stringSizes} med en tom argumentlista?

\Subtask Det händer ibland att man redan har en sekvenssamling, t.ex. \code{xs}, och vill skicka med varje element som argument till en varargs-funktion. Syntaxen för detta är \code{xs: _* } vilket gör att kompilatorn omvandlar sekvenssamlingen till en argumentsekvens av rätt typ.

Prova denna syntax genom att ge en \code{xs} av typen \code{Vector[String]} som argument till \code{stringSizes}. Fungerar det även om \code{xs} är en sekvens av längden 0?

\SOLUTION

\TaskSolved \what

\SubtaskSolved

\begin{REPL}
scala> def stringSizes(xs: String*): Vector[Int] = xs.map(_.size).toVector
def stringSizes(xs: String*): Vector[Int]

scala> stringSizes("hej")
val res0: Vector[Int] = Vector(3)

scala> stringSizes("hej", "på", "dej", "")
val res1: Vector[Int] = Vector(3, 2, 3, 0)

scala> stringSizes()
val res2: Vector[Int] = Vector()
\end{REPL}

\noindent Anrop med tom argumentlista ger en tom heltalssekvens.

\SubtaskSolved

\begin{REPL}
scala> val xs = Vector("hej","på","dej", "")
val xs: Vector[String] = Vector(hej, på, dej, "")

scala> stringSizes(xs: _*)
val res0: Vector[Int] = Vector(3, 2, 3, 0)

scala> stringSizes(Vector(): _*)
val res1: Vector[Int] = Vector()
\end{REPL}
Ja, det funkar fint med tom sekvens.

\QUESTEND



\clearpage

\ExtraTasks %%%%%%%%%%%%%%%%%%%%%%%%%%%%%%%%%%%%%%%%%%%%%%%%%%%%%%%%%%%%%%%%%%%%



\WHAT{Registrering av booleska värden. Singla slant.}

\QUESTBEGIN

\Task \what~

\Subtask Implementera en funktion som registrerar många slantsinglingar enligt nedan funktionshuvud. Indata är en sekvens av booleska värden där krona kodas som \code{true} och klave kodas som \code{false}. För registreringen ska du använda en lokal \code{Array[Int]}. I resultatet ska antalet utfall av \code{krona} ligga på första platsen i 2-tupeln och på andra platsen ska antalet utfall av \code{klave} ligga.

\begin{Code}
def registerCoinFlips(xs: Seq[Boolean]): (Int, Int) = ???
\end{Code}

\Subtask Skapa en funktion \code{flips(n)} som ger en boolesk \code{Vector} med $n$ stycken slantsinglingar och använd den när du testar din slantsinglingsregistreringsalgoritm.

\SOLUTION

\TaskSolved \what~

\SubtaskSolved
\begin{Code}
def registerCoinFlips(xs: Seq[Boolean]): (Int, Int) = 
  val result = Array.fill(2)(0)
  xs.foreach(x => if (x) result(0) += 1 else result(1) += 1)
  (result(0), result(1))
\end{Code}

\SubtaskSolved

\QUESTEND


\WHAT{Kopiering och tillägg på slutet.}

\QUESTBEGIN

\Task \what~
Skapa funktionen \code{copyAppend} som implementerar nedan algoritm, \emph{efter} att du rättat de \textbf{\color{red}{två buggarna}} nedan:

\begin{algorithm}[H]
 \SetKwInOut{Input}{Indata}\SetKwInOut{Output}{Resultat}

 \Input{Heltalsarray $xs$ och heltalet $x$}
 \Output{En ny heltalsarray som som är en kopia av $xs$ men med $x$ tillagt på slutet som extra element.}
 $ys \leftarrow$ en ny array med plats för ett element mer än i $xs$\\
 $i \leftarrow 0$  \\
 \While{$i \leq xs.length$}{
  $ys(i) \leftarrow xs(i)$
 }
lägg $x$ på sista platsen i $ys$
\end{algorithm}

\noindent Granska din kod enligt checklistan i tidigare tipsruta. Testa din funktion för de olika fallen: tom sekvens, sekvens med exakt ett element, sekvens med många element.


\SOLUTION

\TaskSolved \what~

\begin{Code}
def copyAppend(xs: Array[Int], x: Int): Array[Int] = 
  val ys = new Array[Int](xs.length + 1)
  var i = 0
  while i < xs.length do
    ys(i) = xs(i)
    i += 1
  ys(xs.length) = x
  ys
\end{Code}
De två buggarna i algoritmen finns (1) i villkoret som ska vara strikt mindre än och (2) inne i loopen där uppräkningen av loppvariabeln saknas.

\QUESTEND



% \WHAT{Välja sekvenssamling.}
%
% \QUESTBEGIN
%
% \Task  \what~Vilken sekvenssamling är lämpligast i respektive situation nedan? Välj mellan \code{Vector}, \code{ArrayBuffer} och \code{ListBuffer}.
%
% \Subtask Det asociala mediet ZuckerBok ska lagra statusuppdateringar från sina användare. Dessa lagras i en förändringsbar sekvens där nya poster läggs till först. Indexering mitt i sekvensen är mycket ovanligt eftersom de flesta användarna sällan läser vad andra skriver, utan mest skriver nya inlägg om sig själv.
%
% \Subtask ZuckerBok försöker öka sina intäkter och börjar frenetiskt indexera i kors och tvärs i sekvensen med statusuppdaringar för att söka efter lämpliga spamoffer.
%
% \Subtask ZuckerBok bestämmer sig för att lagra födelsedatum för alla ca $10^7$ medborgare i Sverige i en oföränderlig sekvens för att kunna förmedla specialreklam på födelsedagar.
%
% \SOLUTION
%
% \TaskSolved \what
%
% \SubtaskSolved  \code{ListBuffer} som är snabb på fröändringar i början av sekvensen.
%
% \SubtaskSolved  \code{ArrayBuffer} som är snabb på både storleksförändringar och godtycklig indexering.
%
% \SubtaskSolved  \code{Vector} eftersom ofränderlighet efterfrågas.
%
% \QUESTEND



\WHAT{Kopiera och reversera sekvens.}

\QUESTBEGIN

\Task  \what~  Implementera \code{seqReverseCopy} enligt:

\begin{algorithm}[H]
 \SetKwInOut{Input}{Indata}\SetKwInOut{Output}{Resultat}

 \Input{Heltalsarray $xs$}
 \Output{En ny heltalsarray med elementen i $xs$ i omvänd ordning.}
 $n \leftarrow$ antalet element i $xs$ \\
 $ys \leftarrow$ en ny heltalsarray med plats för $n$ element\\
 $i \leftarrow 0$  \\
 \While{$i < n$}{
  $ys(n - i - 1) \leftarrow xs(i)$ \\
  $i \leftarrow i + 1$
 }
 \Return $ys$
\end{algorithm}

\Subtask Använd en \code{while}-sats på samma sätt som i algoritmen. Prova din implementation i REPL och kolla så att den fungerar i olika fall.

\Subtask Gör en ny implementation som i stället använder en \code{for}-sats som börjar bakifrån. Kör din implementation i REPL och kolla så att den fungerar i olika fall.

\SOLUTION

\TaskSolved \what

\SubtaskSolved  \begin{Code}
def seqReverseCopy(xs: Array[Int]): Array[Int] =
  val n = xs.length
  val ys = new Array[Int](n)
  var i = 0
  while i < n do
    ys(n - i - 1) = xs(i)
    i += 1
  ys
\end{Code}

\SubtaskSolved  \begin{Code}
def seqReverseCopy(xs: Array[Int]): Array[Int] = 
  val n = xs.length
  val ys = new Array[Int](n)
  for i <- (n - 1) to 0 by -1 do
    ys(n - i - 1) = xs(i)
  ys
\end{Code}


\QUESTEND




\WHAT{Kopiera alla utom ett.}

\QUESTBEGIN

\Task  \what~  Implementera kopiering av en array \emph{utom} ett element på en viss angiven plats.
Skriv först pseudokod innan du implementerar:
\begin{Code}
def removeCopy(xs: Array[Int], pos: Int): Array[Int]
\end{Code}

\SOLUTION


\TaskSolved \what

\begin{algorithm}[H]
 \SetKwInOut{Input}{Indata}\SetKwInOut{Output}{Resultat}

 \Input{En sekvens $xs$ av typen \texttt{Array[Int]} och $pos$}
 \Output{En ny sekvens av typen \texttt{Array[Int]} som är en kopia av $xs$ fast med elementet på plats $pos$ borttaget}
 $n \leftarrow$ antalet element $xs$\\
 $ys \leftarrow$ en ny \texttt{Array[Int]} med plats för $n-1$ element \\
 \For{$i \leftarrow 0$ \KwTo $pos - 1$}{
  $ys(i) \leftarrow xs(i)$
 }
 $ys(pos) \leftarrow x$ \\
 \For{$i \leftarrow pos+1$ \KwTo $n - 1$}{
  $ys(i - 1) \leftarrow xs(i)$
 }
 \Return $ys$
\end{algorithm}

\begin{Code}
def removeCopy(xs: Array[Int], pos: Int): Array[Int] =
  val n = xs.size
  val ys = Array.fill(n - 1)(0)
  for i <- 0 until pos do
    ys(i) = xs(i)
  for i <- (pos + 1) until n do
    ys(i - 1) = xs(i)
  ys
\end{Code}

\QUESTEND




\WHAT{Borttagning på plats i array.}

\QUESTBEGIN

\Task  \what~  Ibland vill man ta bort ett element på en viss position i en array utan att kopiera alla element, utom ett, till en ny samling. Ett sätt att göra detta är att flytta alla efterföljande element ett steg mot lägre index och fylla ut sista positionen med ett utfyllnadsvärde, t.ex. $0$.
Skriv först pseudokod för en sådan algoritm. Implementera sedan algoritmen i en funktion med denna signatur:
\begin{Code}
def removeAndPad(xs: Array[Int], pos: Int, pad: Int = 0): Unit
\end{Code}

\SOLUTION

\TaskSolved \what

\begin{algorithm}[H]
 \SetKwInOut{Input}{Indata}\SetKwInOut{Output}{Resultat}

 \Input{En sekvens $xs$ av typen \texttt{Array[Int]}, en position $pos$ och ett utfyllnadsvärde $pad$}
 \Output{En uppdaterad sekvens av $xs$ där elementet på plats $pos$ tagits bort och efterföljande element flyttas ett steg mot lägre index med ett sista elementet som tilldelats värdet av $pad$}
 $n \leftarrow$ antalet element $xs$\\
 \For{$i \leftarrow pos+1$ \KwTo $n - 1$}{
  $xs(i - 1) \leftarrow xs(i)$
 }
 $xs(n - 1) \leftarrow pad$ \\
\end{algorithm}

\begin{Code}
def remove(xs: Array[Int], pos: Int, pad: Int = 0): Unit =
  val n = xs.size
  for i <- (pos + 1) until n do
    xs(i - 1) = xs(i)
  xs(n - 1) = pad
\end{Code}

\QUESTEND




\WHAT{Kopiering och insättning.}

\QUESTBEGIN

\Task  \what~

\Subtask Implementera en funktion med detta huvud enligt efterföljande algoritm:
\begin{Code}
def insertCopy(xs: Array[Int], x: Int, pos: Int): Array[Int]
\end{Code}


\begin{algorithm}[H]
 \SetKwInOut{Input}{Indata}\SetKwInOut{Output}{Resultat}

 \Input{En sekvens $xs$ av typen \texttt{Array[Int]} och heltalen $x$ och $pos$}
 \Output{En ny sekvens av typen \texttt{Array[Int]} som är en kopia av $xs$ men där $x$ är infogat på plats $pos$}
 $n \leftarrow$ antalet element $xs$\\
 $ys \leftarrow$ en ny \texttt{Array[Int]} med plats för $n+1$ element \\
 \For{$i \leftarrow 0$ \KwTo $pos - 1$}{
  $ys(i) \leftarrow xs(i)$
 }
 $ys(pos) \leftarrow x$ \\
 \For{$i \leftarrow pos$ \KwTo $n - 1$}{
  $ys(i + 1) \leftarrow xs(i)$
 }
 \Return $ys$
\end{algorithm}


\Subtask Vad måste \code{pos} vara för att det ska fungera med en tom array som argument?

\Subtask Vad händer om din funktion anropas med ett negativt argument för \code{pos}?

\Subtask Vad händer om din funktion anropas med \code{pos} lika med \code{xs.size}?

\Subtask Vad händer om din funktion anropas med \code{pos} större än \code{xs.size}?

\SOLUTION

\TaskSolved \what

\SubtaskSolved  \begin{Code}
def insertCopy(xs: Array[Int], x: Int, pos: Int): Array[Int] =
  val n = xs.size
  val ys = Array.ofDim[Int](n + 1)
  for i <- 0 until pos do
    ys(i) = xs(i)
  ys(pos) = x
  for i <- pos until n do
    ys(i + 1) = xs(i)
  ys
\end{Code}

\SubtaskSolved  \code{pos} måste vara \code{0}.

\SubtaskSolved  \begin{REPL}
java.lang.ArrayIndexOutOfBoundsException: -1
\end{REPL}

\SubtaskSolved  Elementet \code{x} läggs till på slutet av arrayen, alltså kommer den returnerande arrayen vara större än den som skickades in.

\SubtaskSolved  \begin{REPL}
java.lang.ArrayIndexOutOfBoundsException: 5
\end{REPL}
Man får \code{ArrayIndexOutOfBoundsException} då indexeringen är utanför storleken hos arrayen.

\QUESTEND




\WHAT{Insättning på plats i array.}

\QUESTBEGIN

\Task  \what~  Ett sätt att implementera insättning i en array, utan att kopiera alla element till en ny array med en plats extra, är att alla elementen efter \code{pos} flyttas fram ett steg till högre index, så att plats bereds för det nya elementet. Med denna lösning får det sista elementet ''försvinna'' genom brutal överskrivning eftersom arrayer inte kan ändra storlek.

Skriv först en sådan algoritm i pseudokod och implementera den sedan i en procedur med detta huvud:
\begin{Code}
def insertDropLast(xs: Array[Int], x: Int, pos: Int): Unit
\end{Code}

\SOLUTION

\TaskSolved \what

\begin{algorithm}[H]
 \SetKwInOut{Input}{Indata}\SetKwInOut{Output}{Resultat}

 \Input{En sekvens $xs$ av typen \texttt{Array[Int]} och heltalen $x$ och $pos$}
 \Output{En uppdaterad sekvens av $xs$ där elementet $x$ har satts in på platsen $pos$ och efterföljande element flyttas ett steg där sista elementet försvinner}
 $n \leftarrow$ antalet element i $xs$\\
 $ys \leftarrow$ en klon av $xs$\\
 $xs(pos) \leftarrow x$\\
 \For{$i \leftarrow pos+1$ \KwTo $n - 1$}{
  $xs(i) \leftarrow ys(i - 1)$
 }
\end{algorithm}

\begin{Code}
def insertDropLast(xs: Array[Int], x: Int, pos: Int): Unit =
  val n = xs.size
  val ys = xs.clone
  xs(pos) = x
  for i <- pos + 1 until n do
    xs(i) = ys(i - 1)
\end{Code}

\QUESTEND


\WHAT{Fler inbyggda metoder för linjärsökning.}

\QUESTBEGIN

\Task \what~

\Subtask Läs i snabbreferensen om metoderna \code{lastIndexOf}, \code{indexOfSlice}, \code{segmentLength} och \code{maxBy} och beskriv vad var och en kan användas till.

\Subtask Testa metoderna i REPL.

\SOLUTION

\TaskSolved \what~

\SubtaskSolved

\begin{itemize}[noitemsep]
  \item \code{lastIndexOf} är bra om man vill leta bakifrån i stället för framifrån; utan denna hade man annars då behövt använda \code{xs.reverse.indexOf(e)}
  \item \code{indexOfSlice(ys)} letar efter index där en hel sekvens \code{ys} börjar, till skillnad från \code{indexOf(e)} som bara letar efter ett enstaka element.
  \item \code{segmentLength(p, i)} ger längden på den längsta sammanhängande sekvens där alla element uppfyller predikatet \code{p} och sökningen efter en sådan sekvens börjar på plats \code{i}
  \item \code{xs.maxBy(f)} kör först funktionen \code{f} på alla element i \code{xs} och letar sedan upp det största värdet; motsvarande \code{minBy(f)} ger minimum av \code{f(e)} över alla element \code{e} i \code{xs}
\end{itemize}

\SubtaskSolved --

\QUESTEND



\clearpage

\AdvancedTasks %%%%%%%%%%%%%%%%%%%%%%%%%%%%%%%%%%%%%%%%%%%%%%%%%%%%%%%%%%%%%%%%%

\WHAT{Fixa svensk sorteringsordning av ÄÅÖ.}

\QUESTBEGIN

\Task \label{task:swedish-letter-ordering} \what~   Svenska bokstäver kommer i, för svenskar, konstig ordning om man inte vidtar speciella åtgärder. Med hjälp av klassen \code{java.text.Collator} kan man få en \code{Comparator} för strängar som följer lokala regler för en massa språk på planeten jorden.

\Subtask Verifiera att sorteringsordningen blir rätt i REPL enligt nedan.

\begin{REPL}
scala> val fel = Vector("ö","å","ä","z").sorted
scala> val svColl = java.text.Collator.getInstance(new java.util.Locale("sv"))
scala> val svOrd = Ordering.comparatorToOrdering(svColl)
scala> val rätt = Vector("ö","å","ä","z").sorted(svOrd)
\end{REPL}
\Subtask Använd metoden ovan för att skriva ett program som skriver ut raderna i en textfil i korrekt svensk sorteringsordning. Programmet ska kunna köras med kommandot:\\\texttt{scala sorted -sv textfil.txt}

\Subtask Läs mer här: \\
\noindent{\href{http://stackoverflow.com/questions/24860138/sort-list-of-string-with-localization-in-scala}{\small stackoverflow.com/questions/24860138/sort-list-of-string-with-localization-in-scala}}



\SOLUTION


\TaskSolved \what



\QUESTEND



\WHAT{Fibonacci-sekvens med ListBuffer.}

\QUESTBEGIN

\Task  \what~ Samlingen \code{ListBuffer} är en förändringsbar sekvens som är snabb och minnessnål vid tillägg i början \Eng{prepend}. Undersök vad som händer här:
\begin{REPL}
scala> val xs = scala.collection.mutable.ListBuffer.empty[Int]
scala> xs.prependAll(Vector(1, 1))
scala> while xs.head < 100 do {xs.prepend(xs.take(2).sum); println(xs)}
scala> xs.reverse.toList
\end{REPL}
Talen i sekvensen som produceras på rad 4 ovan kallas Fibonacci-tal  \footnote{\href{https://sv.wikipedia.org/wiki/Fibonaccital}{sv.wikipedia.org/wiki/Fibonaccital}} och blir snabbt mycket stora.

\Subtask Definera och testa följande funktion. Den ska internt använda förändringsbara \code{ListBuffer} men returnera en sekvens av oföränderliga \code{List}.

\begin{Code}
/** Ger en lista med tal ur Fibonacci-sekvensen 1, 1, 2, 3, 5, 8 ...
  * där det största talet är mindre än max. */
def fib(max: Long): List[Long] = ???
\end{Code}


\Subtask
Hur lång ska en Fibonacci-sekvens vara för att det sista elementet ska vara så nära \code{Int.MaxValue} som möjligt?


\Subtask Implementera \code{fibBig} som använder \code{BigInt} i stället för \code{Long} och låt din dator få använda sitt stora minne medan planeten värms upp en aning.

\SOLUTION

\TaskSolved \what


\SubtaskSolved

\begin{Code}
def fib(max: Long): List[Long] = 
  val xs = scala.collection.mutable.ListBuffer.empty[Long]
  xs.prependAll(Vector(1, 1))
  while xs.head < max do xs.prepend(xs.take(2).sum)
  xs.reverse.drop(1).toList
\end{Code}

\SubtaskSolved

\begin{REPL}
scala> fib(Int.MaxValue).size
val res0: Int = 46
\end{REPL}

\SubtaskSolved

\begin{Code}
def fibBig(max: BigInt): List[BigInt] =
  val xs = scala.collection.mutable.ListBuffer.empty[BigInt]
  xs.prependAll(Vector(BigInt(1), BigInt(1)))
  while xs.head < max do xs.prepend(xs.take(2).sum)
  xs.reverse.drop(1).toList
\end{Code}

\begin{REPL}
scala> fibBig(Long.MaxValue).size
val res0: Int = 92

scala> fibBig(BigInt(Long.MaxValue).pow(64)).size
val res1: Int = 5809

scala> fibBig(BigInt(Long.MaxValue).pow(128)).last
val res2: BigInt = 466572805528355449194553611102863153950720005186045547177525242118545194247268198196024304108711020686545660707513547993668927474420737702772726410095432646683782038269206733583562623144723659044965174192994997081915291671203135284448809948278794870130243195729759407652514927641622448506112336858244040087748168546825439555497978038066584506772917257705338472345660520902622305735366348690501583267086607109594118454398543160294999638070938386822164561738531661786873174424857409631803971069795886028284195109247953151499404937810249349132907101567724032186422592145774126660328936577771749713614176045435526886758975994177511201005911748503347657112775964769397750819976041389533451539207673441658345632507479241970993525868183091563469584756527454807108...

scala> fibBig(BigInt(Long.MaxValue).pow(128)).last.toString.size
val res3: Int = 2428

scala> fibBig(BigInt(Long.MaxValue).pow(256)).last.toString.size
val res4: Int = 4856

scala> fibBig(BigInt(Long.MaxValue).pow(1024)).last.toString.size
java.lang.OutOfMemoryError: Java heap space

\end{REPL}

\QUESTEND



\WHAT{Omvända sekvens på plats.}

\QUESTBEGIN

\Task \what~Implementera nedan algoritm i funktionen \code{reverseChars} och testa så att den fungerar för olika fall i REPL.


\begin{algorithm}[H]
 \SetKwInOut{Input}{Indata}\SetKwInOut{Output}{Resultat}

 \Input{En array $xs$ med tecken}
 \Output{Samma array med tecknen i omvänd ordning}
 $n \leftarrow$ antalet element i $xs$\\
 \For{$i \leftarrow 0$ \KwTo $\frac{n}{2} - 1$}{
  $temp \leftarrow xs(i)$ \\
  $xs(i) \leftarrow xs(n - i - 1)$ \\
  $xs(n - i - 1) \leftarrow temp$ \\
 }
\end{algorithm}

\SOLUTION

\TaskSolved \what~
\begin{Code}
def reverseChars(xs: Array[Char]): Unit =
  val n = xs.length
  for i <- 0 to (n/2 - 1) do
    val temp = xs(i)
    xs(i) = xs(n - i - 1)
    xs(n - i - 1) = temp
\end{Code}

\QUESTEND



\WHAT{Palindrompredikat.}

\QUESTBEGIN

\Task  \what~ En palindrom\footnote{\url{https://sv.wikipedia.org/wiki/Palindrom}} är ett ord som förblir oförändrat om man läser det baklänges. Exempel på palindromer: kajak, dallassallad.

Ett sätt att implementera ett palindrompredikat visas nedan:
\begin{Code}
def isPalindrome(s: String): Boolean = s == s.reverse
\end{Code}

\Subtask Implementationen ovan kan innebära att alla tecken i strängen gås igenom två gånger och behöver minnesutrymme för dubbla antalet tecken. Varför?

\Subtask Skapa ett palindromtest som går igenom elementen max en gång och som inte behöver extra minnesutrymme för en kopia av strängen. \emph{Lösningsidé:} Jämför parvis första och sista, näst första och näst sista, etc.

\SOLUTION

\TaskSolved \what

\SubtaskSolved Omvändning med \code{reverse} kan kräva genomgång av hela strängen en gång samt minnesutrymme för kopian. Innehållstestet kräver ytterligare en genomgång. (Detta är i och för sig inget stort problem eftersom världens längsta palindrom inte är längre än 19 bokstäver och är ett obskyrt finskt ord som inte ofta yttras i dagligt tal. Vilket?)

\SubtaskSolved

\begin{Code}
def isPalindrome(s: String): Boolean =
  val n = s.length
  var foundDiff = false
  var i = 0
  while i < n/2 && !foundDiff do
    foundDiff = s(i) != s(n - i - 1)
    i += 1
  !foundDiff
\end{Code}

\QUESTEND



\WHAT{Fler användbara sekvenssamlingsmetoder.}

\QUESTBEGIN

\Task \what~Sök på webben och läs om dessa metoder och testa dem i REPL:
\begin{itemize}[noitemsep]
  \item \code{xs.tabulate(n)(f)}
  \item \code{xs.forall(p)}
  \item \code{xs.exists(p)}
  \item \code{xs.count(p)}
  \item \code{xs.zipWithIndex}
\end{itemize}

\SOLUTION

\TaskSolved \what~
\begin{REPL}
scala> val xs = Vector.tabulate(10)(i => math.pow(2, i).toInt)
xs: Vector[Int] = Vector(1, 2, 4, 8, 16, 32, 64, 128, 256, 512)

scala> xs.forall(_ < 1024)
val res0: Boolean = true

scala> xs.exists(_ == 3)
val res1: Boolean = false

scala> xs.count(_ > 64)
val res2: Int = 3

scala> xs.zipWithIndex.take(5)
val res3: Vector[(Int, Int)] = Vector((1,0), (2,1), (4,2), (8,3), (16,4))
\end{REPL}
\QUESTEND






\WHAT{Arrays don't behave, but \code{ArraySeq}s do!}

\QUESTBEGIN

\Task \what~Även om \code{Array} är primitiv så finns smart krångel ''under huven'' i Scalas samlingsbibliotek för att arrayer ska bete sig nästan som ''riktiga'' samlingar. Därmed behöver man inte ägna sig åt olika typer av specialhantering, t.ex. s.k. boxning, wrapperklasser och typomvandlingar \Eng{type casting}, vilket man ofta behöver kämpa med som Java-programmerare.

Dock finns fortfarande begränsningar och anomalier vad gäller till exempel likhetstest. Om du vill att en array ska bete sig som andra samlingar kan du enkelt ''wrappa'' den med metoden \code{toSeq} som vid anrop på arrayer ger en \code{ArraySeq}. Denna beter sig som en helt vanlig oföränderlig sekvenssamling utan att offra snabbheten hos en primitiv array.
\begin{Code}
val as = Array(1,2,3)
val xs = as.toSeq
\end{Code}
\Subtask Hur fungerar likhetstest mellan två \code{ArraySeq}s? Vad har \code{xs} ovan för typ? Går det att uppdatera en wrappad array?

\Subtask Vilken typ av argumentsekvens får du tillgång till i kroppen för en funktion med repeterande parametrar?

\Subtask\Uberkurs Läs här:
\url{http://docs.scala-lang.org/overviews/collections/arrays.html}
och ge ett exempel på vad mer man inte kan göra med en array, förutom innehållslikhetstest.



\SOLUTION

\TaskSolved \what~

\SubtaskSolved \code{xs} erbjuder innehållslikhet och har typen \code{Seq[Int]} med den underliggande typen \code{ArraySeq[Int]}. Det går inte att göra tilldelning av element i en \code{ArraySeq} eftersom metoden \code{update} saknas, och den är oföränderlig. Den uppdateras därför inte när den urspringliga arrayen uppdateras.

\begin{REPL}
scala> val as1 = Array(1,2,3)
val as1: Array[Int] = Array(1, 2, 3)

scala> val as2 = Array(1,2,3)
val as2: Array[Int] = Array(1, 2, 3)


scala> val (xs1, xs2) = (as1.toSeq, as2.toSeq)
val xs1: Seq[Int] = ArraySeq(1, 2, 3)
val xs2: Seq[Int] = ArraySeq(1, 2, 3)

scala> as1 == as2
val res0: Boolean = false

scala> xs1 == xs2
val res1: Boolean = true

scala> as1(0) = 42

scala> xs1
val res2: Seq[Int] = ArraySeq(1, 2, 3)

scala> xs1(0) = 42
value update is not a member of Seq[Int]
\end{REPL}

\SubtaskSolved Vid repeterade parametrar får man en \code{ArraySeq}.

\begin{REPL}
scala> def f(xs: Int*) = xs
def f(xs: Int*): Seq[Int]

scala> println(f(1,2,3))
ArraySeq(1, 2, 3)
\end{REPL}


\SubtaskSolved Det går inte att ha en generisk array som funktionsresultat utan att bifoga kontextgränsen \code{ClassTag} i typparametern för att kompilatorn ska kunna generera kod för den typkonvertering som krävs under runtime av JVM. Se exempel här:\\
\url{http://docs.scala-lang.org/overviews/collections/arrays.html}


\QUESTEND




\WHAT{List eller Vector?}

\QUESTBEGIN

\Task\Uberkurs  \what~ Jämför tidskomplexitet mellan List och Vector vid hantering i början och i slutet, baserat på efterföljande REPL-session i din egen körmiljö.  Körningen nedan gjordes på en AMD Ryzen 7 5800X (16) @ 3.800GHz under Arch Linux 5.12.8-arch1-1 med Scala 3.0.1 och openjdk 11.0.11, men du ska använda det du har på din dator.

Hur snabbt går nedan på din dator? När är List snabbast och när är Vector snabbast? Hur stor är skillnaderna i prestanda?
\footnote{Denna typ av mätningar lär du dig mer om i LTH-kursen ''Utvärdering av programvarusystem'', som ges i slutet av årskurs 1 för Datateknikstudenter.}
%sudo lshw -class processor


\begin{CodeSmall}
> head -5 /proc/cpuinfo
processor    : 0
vendor_id    : AuthenticAMD
cpu family    : 25
model        : 33
model name    : AMD Ryzen 7 5800X 8-Core Processor

scala> def time(n: Int)(block: => Unit): Double =                  
     |   def now = System.nanoTime
     |   var timestamp = now
     |   var sum = 0L
     |   var i = 0
     |   while i < n do
     |     block
     |     sum = sum + (now - timestamp)
     |     timestamp = now
     |     i = i + 1
     |   val average = sum.toDouble / n
     |   println("Average time: " + average + " ns")
     |   average


// Exiting paste mode, now interpreting.

time: (n: Int)(block: => Unit)Double


scala> val n = 100000
scala> val l = List.fill(n)(math.random())
scala> val v = Vector.fill(n)(math.random())

scala> (for i <- 1 to 20 yield time(n){l.take(10)}).min
average time: 97.66952 ns
average time: 91.90033 ns
average time: 79.88311 ns
average time: 69.5963 ns
average time: 69.69892 ns
average time: 69.8033 ns
average time: 69.7705 ns
average time: 69.68491 ns
average time: 69.54222 ns
average time: 69.66051 ns
average time: 69.73661 ns
average time: 69.54112 ns
average time: 69.69141 ns
average time: 69.46341 ns
average time: 69.4098 ns
average time: 61.34162 ns
average time: 41.1333 ns
average time: 40.97051 ns
average time: 40.9075 ns
average time: 41.12321 ns
val res0: Double = 40.9075

scala> (for i <- 1 to 20 yield time(n){v.take(10)}).min
average time: 84.56978 ns
average time: 75.20167 ns
average time: 57.16529 ns
average time: 34.84469 ns
average time: 34.38478 ns
average time: 34.77709 ns
average time: 34.77179 ns
average time: 35.0506 ns
average time: 34.7967 ns
average time: 35.04258 ns
average time: 34.82559 ns
average time: 36.3673 ns
average time: 34.91029 ns
average time: 34.87239 ns
average time: 34.51958 ns
average time: 34.83949 ns
average time: 34.56169 ns
average time: 34.80719 ns
average time: 34.84459 ns
average time: 34.89468 ns
val res1: Double = 34.38478

scala> (for i <- 1 to 20 yield time(1000){l.takeRight(10)}).min
average time: 131365.106 ns
average time: 118632.787 ns
average time: 118440.066 ns
average time: 118687.567 ns
average time: 118428.487 ns
average time: 118871.686 ns
average time: 118964.797 ns
average time: 119030.236 ns
average time: 119262.534 ns
average time: 119228.344 ns
average time: 119226.494 ns
average time: 119310.933 ns
average time: 119352.854 ns
average time: 119121.913 ns
average time: 119133.664 ns
average time: 119015.193 ns
average time: 119276.674 ns
average time: 119224.882 ns
average time: 119301.771 ns
average time: 119444.401 ns
val res2: Double = 118428.487

scala> (for i <- 1 to 20 yield time(1000){v.takeRight(10)}).min
average time: 805.989 ns
average time: 365.219 ns
average time: 225.49 ns
average time: 125.92 ns
average time: 124.98 ns
average time: 130.689 ns
average time: 139.86 ns
average time: 128.29 ns
average time: 132.59 ns
average time: 125.729 ns
average time: 125.46 ns
average time: 130.59 ns
average time: 122.03 ns
average time: 121.9 ns
average time: 119.69 ns
average time: 120.48 ns
average time: 125.239 ns
average time: 126.09 ns
average time: 125.92 ns
average time: 125.91 ns
val res3: Double = 119.69

\end{CodeSmall}

\noindent Varför går olika rundor i for-loopen olika snabbt även om varje runda gör samma sak?

\SOLUTION

\TaskSolved
Sekvenssamlingen \code{List} är nästan dubbelt så snabb vid bearbetning i början men ungefär 1000 gånger långsammare vid bearbetning i slutet av en sekvens med 100000 element.


Olika körningar går olika snabbt på JVM bl.a. p.g.a optimeringar som sker när JVM-en ''värms upp'' och den så kallade Just-In-Time-kompileringen gör sitt mäktiga jobb. Det går ibland plötsligt väsentligt långsammare när skräpsamlaren tvingas göra tidsödande storstädning av minnet.

\QUESTEND






\WHAT{Tidskomplexitet för olika samlingar i Scalas standardbibliotek.}

\QUESTBEGIN

\Task\Uberkurs  \what~\\
Studera skillnader i tidskomplexitet mellan olika samlingar här: \\ \href{http://docs.scala-lang.org/overviews/collections/performance-characteristics.html}{docs.scala-lang.org/overviews/collections/performance-characteristics.html} \\
Läs även kritiken av förenklingar i ovan beskrivning här:\\
\href{http://www.lihaoyi.com/post/ScalaVectoroperationsarentEffectivelyConstanttime.html}{www.lihaoyi.com/post/ScalaVectoroperationsarentEffectivelyConstanttime.html}
\\
Läs denna grundliga empirisk genomgång av prestanda i Scalas samlingsbibliotek:\\
\href{http://www.lihaoyi.com/post/BenchmarkingScalaCollections.html}{www.lihaoyi.com/post/BenchmarkingScalaCollections.html}
\\Du får lära dig mer om hur man resonerar kring komplexitet i kommande kurser.


\SOLUTION

\TaskSolved --

\QUESTEND


%\part{Andra läsperiodens övningar}


%!TEX encoding = UTF-8 Unicode
%!TEX root = ../exercises.tex

\ifPreSolution

\Exercise{\ExeWeekEIGHT}\label{exe:W08}

\begin{Goals}
\item Kunna skapa och använda matriser med nästlade strukturer av \code{Vector}.
\item Kunna iterera över elementen i en matris med nästlade \code{for}-satser och \code{for}-\code{yield}-uttryck, samt nästlad applicering av \code{map} respektive \code{foreach}.
\item Kunna skapa och använda funktioner som tar matriser som parametrar.
\item Kunna skapa en enkel generisk klass och enkla generiska funktioner med hjälp av en typparameter.
\item Kunna beskriva skillnader och likheter mellan Scala och Java vad gäller indexering och iterering i matriser implementerade med nästlade arrayer.
%\item Kunna skapa och använda matriser med hjälp inbyggda arrayer i Java.
%\item Kunna använda nästlade \code{for}-satser i Java för att iterera över elementen i en matris.
\end{Goals}

\begin{Preparations}
\item \StudyTheory{08}
\end{Preparations}

\BasicTasks

\else

\ExerciseSolution{\ExeWeekEIGHT}

\BasicTasks

\fi



\WHAT{Para ihop begrepp med beskrivning.}

\QUESTBEGIN

\Task \what

\vspace{1em}\noindent Koppla varje begrepp med den (förenklade) beskrivning som passar bäst:

\begin{ConceptConnections}
  matris & 1 & & A & konkret typ, binds till typparameter vid kompilering \\ 
  generisk & 2 & & B & indexerbar datastruktur i två dimensioner \\ 
  typargument & 3 & & C & har abstrakt typparameter, typen är generell \\ 
  typhärledning & 4 & & D & kompilatorn beräknar typ ur sammanhanget \\ 
\end{ConceptConnections}

\SOLUTION

\TaskSolved \what

\begin{ConceptConnections}
  matris & 1 & ~~\Large$\leadsto$~~ &  A & indexerbar datastruktur i två dimensioner \\ 
  radvektor & 2 & ~~\Large$\leadsto$~~ &  F & matris av dimension $1\times{}m$ med $m$ horisontella värden \\ 
  kolumnvektor & 3 & ~~\Large$\leadsto$~~ &  G & matris av dimension $m\times{}1$ med $m$ vertikala värden \\ 
  kolonn & 4 & ~~\Large$\leadsto$~~ &  C & annat ord för kolumn \\ 
  generisk & 5 & ~~\Large$\leadsto$~~ &  B & har abstrakt typparameter, typen är generell \\ 
  typargument & 6 & ~~\Large$\leadsto$~~ &  D & konkret typ, binds till typparameter vid kompilering \\ 
  typhärledning & 7 & ~~\Large$\leadsto$~~ &  E & kompilatorn beräknar typ ur sammanhanget \\ 
\end{ConceptConnections}

\QUESTEND




\WHAT{Skapa matriser med hjälp av nästlade samlingar.}

\QUESTBEGIN

\Task  \what~  Man kan i ett datorprogram, med hjälp av samlingar som innehåller samlingar, skapa nästlade strukturer som kan indexeras i två dimensioner och på så sätt representera en  \textbf{matris}.\footnote{\href{https://sv.wikipedia.org/wiki/Matris}{sv.wikipedia.org/wiki/Matris}}

\Subtask Rita minnessituationen efter tilldelningen på rad 1 nedan. Vad har \code{m} för typ och värde? Vad har \code{m} för dimensioner? Hur sker indexeringen i ett datorprogram jämfört med i matematiken?

\begin{REPL}
scala> val m = Vector((1 to 5).toVector, (3 to 7).toVector)
scala> m.apply(0).apply(1)
scala> m(1)
scala> m(1)(4)
\end{REPL}

\Subtask Vad ger uttrycken på raderna 2, 3 och 4 ovan för värden och typ?

\Subtask Man kan i ett datorprogram mycket väl skapa tvådimensionella, nästlade strukturer där raderna \emph{inte} innehåller samma antal element. Det blir då ingen äkta matris i strikt matematisk mening, men man kallar ofta ändå en sådan struktur för en ''matris''. Vilken typ har variablerna \code{m2}, \code{m3}, \code{m4} och \code{m5} nedan?

\begin{REPL}
scala> val m2 = Vector(Vector(1,2,3),Vector(4,5),Vector(42))
scala> val m3 = Vector(Vector(1,2), Vector(1.0, 2.0, 3.0))
scala> val m4 = m3(1) +: Vector("a") +: m3
scala> val m5 = Vector.fill(42){ m2(1).map(e => (e * math.random()).toInt) }
\end{REPL}

\Subtask Vilken av variablerna \code{m2}, \code{m3}, \code{m4} och \code{m5} ovan representerar en äkta matris i matematisk mening? Vilken är dess dimensioner?

\SOLUTION

\TaskSolved \what

\SubtaskSolved   \includegraphics{../img/w09-solutions/1a} \\
Typ: \code{Vector[Vector[Int]]}\\
Värde: \code{Vector(Vector(1, 2, 3, 4, 5), Vector(3, 4, 5, 6, 7))} \\
Dimensioner: $2 \times 5$\\
Inom matematiken sker indexering enligt konvention med 1 som lägsta index. I scala är lägsta index 0, man använder s.k. 0-indexering. \footnote{Detta är inte fallet i alla programmeringsspråk, vilket du kan läsa mer om på \url{https://en.wikipedia.org/wiki/Array\_data\_type\#Index\_origin}}

\SubtaskSolved
\begin{REPL}
scala> val m = Vector((1 to 5).toVector, (3 to 7).toVector)
m: Vector[Vector[Int]] = Vector(Vector(1, 2, 3, 4, 5), Vector(3, 4, 5, 6, 7))

scala> m.apply(0).apply(1)
res4: Int = 2

scala> m(1)
res5: Vector[Int] = Vector(3, 4, 5, 6, 7)

scala> m(1)(4)
res6: Int = 7
\end{REPL}

\SubtaskSolved  \\
m2: \code{Vector[Vector[Int]]}\\
m3: \code{Vector[Vector[Int | Double]]}\\
m4: \code{Vector[Vector[Int | Double | String]]}\\
m5: \code{Vector[Vector[Int]]}

\SubtaskSolved  m5, $42 \times 2$

\QUESTEND





\WHAT{Skapa och iterera över matriser.}

\QUESTBEGIN

\Task  \label{matrices:task:yatzy} \what~  Du ska skapa matriser där varje rad representerar 5 kast med en tärning i spelet Yatzy.\footnote{\href{https://sv.wikipedia.org/wiki/Yatzy}{sv.wikipedia.org/wiki/Yatzy}}


\Subtask Definiera i REPL en funktion \code{def throwDie: Int = ???} som returnerar ett slumptal mellan 1 och 6.

\Subtask Skapa nedan heltalsmatris i REPL. Vilken dimension får matrisen?
\begin{REPL}
scala> val ds1 = for (i <- 1 to 1000) yield 
            for (j <- 1 to 5) yield throwDie
          
\end{REPL}

\Subtask Man kan också använda nedan varianter för att skapa en heltalsmatris. Vilken av varianterna \code{ds1} ... \code{ds6} tycker du är lättast att läsa och förstå? Prova respektive variant i REPL och ange vilken typ på \code{ds1} ... \code{ds6} som härleds av kompilatorn.
\begin{REPL}
val ds2 = (1 to 1000).map(i => (1 to 5).map(j => throwDie))
val ds3 = (1 to 1000).map(i => Vector.fill(5)(throwDie))
val ds4 = for (i <- 1 to 1000) yield Vector.fill(5)(throwDie)
val ds5 = Vector.fill(1000)(Vector.fill(5)(throwDie))
val ds6 = Vector.fill(1000, 5)(throwDie)
\end{REPL}


\Subtask Definiera en funktion \\ \code{def roll(n: Int): Vector[Int] = ???}\\ som ger en heltalsvektor med $n$ stycken slumpvisa tärningskast. Kasten ska vara sorterade i växande ordning; använd för detta ändamål samlingsmetoden \code{sorted}.


\Subtask \label{matrices:subtask:isyatzyforall} Definera i REPL en funktion \code{isYatzy(xs: Vector[Int]): Boolean = ???} som testar om alla elementen i en heltalsvektor är samma. Använd samlingsmetoden \code{forall}.


\Subtask Skapa en funktion  \\ \code{def diceMatrix(m: Int, n: Int): Vector[Vector[Int]] = ???} \\ som med hjälp av funktionen \code{roll} skapar en matris med \code{m} st vektorer med vardera \code{n} slumpvisa tärningskast.


\Subtask \label{matrices:subtask:diceMatrixToString} Skapa en funktion som returnerar en utskriftsvänlig sträng \\ \code{def diceMatrixToString(xss: Vector[Vector[Int]]): String = ???} \\med hjälp av \code{map} och \code{mkString}, som fungerar enligt nedan.
\begin{REPL}
scala> val dm2s = diceMatrixToString(diceMatrix(4, 5))
val dm2s: String = 1 4 4 6 6
1 1 2 6 6
2 4 4 5 6
1 1 5 6 6

scala> println(dm2s)
1 4 4 6 6
1 1 2 6 6
2 4 4 5 6
1 1 5 6 6
\end{REPL}



\Subtask Implementera funktionen \\ \code{def filterYatzy(xss: Vector[Vector[Int]]): Vector[Vector[Int]]} \\ som filtrerar fram alla yatzy-rader i matrisen \code{xss} enligt nedan. Använd din funktion \code{isYatzy} och samlingsmetoden \code{filter}.
\begin{REPL}
scala> println(diceMatrixToString(filterYatzy(diceMatrix(10000, 5))))
4 4 4 4 4
6 6 6 6 6
4 4 4 4 4
6 6 6 6 6
4 4 4 4 4
4 4 4 4 4
2 2 2 2 2
\end{REPL}



\Subtask Implementera funktionen \\
\code{def yatzyPips(xss: Vector[Vector[Int]]): Vector[Int] = ???}\\
som ska ge en vektor med de tärningsvärden som gav yatzy, för kasten i matrisen \code{xss} enligt nedan. Använd din funktion \code{filterYatzy}.
\begin{REPL}
scala> val dm = Vector(Vector(1,2,3,4,5),Vector(4,4,4,4,4),Vector(3,3,3,3,3))
scala> yatzyPips(dm)
val res42: Vector[Int] = Vector(4, 3)
\end{REPL}

\SOLUTION

\TaskSolved \what

\SubtaskSolved
\begin{Code}
def throwDie: Int = (math.random() * 6).toInt + 1
\end{Code}
Eller:
\begin{Code}
def throwDie: Int = scala.util.Random.nextInt(6) + 1
\end{Code}

\SubtaskSolved  Matrisdimension i matematisk notation: $1000 \times 5$, vilket motsvarar en matris med 1000 rader och 5 kolumner.

\SubtaskSolved
\begin{Code}
ds1: IndexedSeq[IndexedSeq[Int]]
ds2: IndexedSeq[IndexedSeq[Int]]
ds3: IndexedSeq[Vector[Int]]
ds4: IndexedSeq[Vector[Int]]
ds5: Vector[Vector[Int]]
ds6: Vector[Vector[Int]]
\end{Code}
\code{IndexedSeq} och \code{Vector} ovan finns i paketet \code{scala.collection.immutable}

\SubtaskSolved  \begin{Code}
def roll(n: Int) = Vector.fill(n)(throwDie).sorted
\end{Code}

\SubtaskSolved  \begin{Code}
def isYatzy(xs: Vector[Int]): Boolean = xs.forall(_ == xs(0))
\end{Code}



%2.g)
\SubtaskSolved  \begin{Code}
def diceMatrix(m: Int, n: Int): Vector[Vector[Int]] =
  Vector.fill(m)(roll(n))
\end{Code}

\SubtaskSolved  \begin{Code}
def diceMatrixToString(xss: Vector[Vector[Int]]): String =
  xss.map(_.mkString(" ")).mkString("\n")
\end{Code}


%2.j)
\SubtaskSolved
\begin{Code}
def filterYatzy(xss: Vector[Vector[Int]]): Vector[Vector[Int]] =
  xss.filter(isYatzy)
\end{Code}



%2.m)
\SubtaskSolved  \begin{Code}
def yatzyPips(xss: Vector[Vector[Int]]): Vector[Int] =
  filterYatzy(xss).map(_.head)
\end{Code}

\QUESTEND








\WHAT{En oföränderlig, generisk matris-klass till veckans laboration \hyperref[section:lab:\LabWeekEIGHT]{\texttt{\LabWeekEIGHT}}.}

\QUESTBEGIN

\Task\label{exe:matrices:labprep}  \what~Under veckans laboration ska du simulera en enkel form av ''liv'' som består av celler i ett rutnät. För detta ändamål har vi nytta av en matris-klass som du ska implementera steg för steg i denna övning.
Skapa case-klassen nedan med en editor i filen \code{Matrix.scala}. Testa din lösning med hjälp av valfri \hyperref[appendix:ide]{IDE}, t.ex. \code{scalaide} eller \code{idea}.
\begin{Code}
case class Matrix(data: Vector[Vector[String]]){
  def apply(row: Int, col: Int): String = data(row)(col)
}
object Matrix {
  def fill(dim: (Int, Int))(value: String): Matrix =
    Matrix(Vector.fill(dim._1, dim._2)(value))
}
\end{Code}

\begin{REPLnonum}
scala> val m = Matrix.fill(3,4)("hej")
scala> val e = m(2, 2)
\end{REPLnonum}

\Subtask Vad får \code{m} ovan för typ?

\Subtask Vad får \code{e} ovan för typ?

\Subtask På hur många ställen måste du ändra i \code{Matrix} ovan för att den i stället ska representera en matris av heltal?

\Subtask Du ska nu med hjälp av en \textbf{typparameter} göra \code{Matrix} \textbf{generisk} \Eng{generic}, så att den blir en mer användbar matrisklass som kan innehålla element av vilken typ som helst. Genomför följande ändringar i \code{Matrix.scala}:

\begin{itemize}[noitemsep, nolistsep]
  \item Lägg till en typparameter \code{T} inom klammerparenteser efter namnet \code{Matrix} på alla ställen där det förekommer \emph{utom} efter namnet på kompanjonsobjektet\footnote{Singelobjekt kan inte ha typparametrar, men deras medlemmar kan.}.
  \item Byt ut \code{String} mot \code{T} på alla ställen där \code{String} förekommer.
  \item Lägg till en typparameter \code{T} inom klammerparenteser efter \code{def fill}.
\end{itemize}
Testa din generiska klass i REPL genom att skapa en boolesk matris:
\begin{REPLnonum}
scala> val bm = Matrix.fill(3,4)(false)
scala> val be = bm(0, 0)
\end{REPLnonum}

\Subtask Vad får \code{bm} ovan för typ?

\Subtask Vad får \code{be} ovan för typ?

\Subtask Lägg en kodrad i början av klasskroppen som med hjälp av \code{require} garanterar att alla rader i matrisen är lika långa.

\Subtask Lägg till en medlem \code{val dim: (Int, Int)} i klasskroppen efter \code{require}-satsen som ger ett par (alltså en 2-tupel) med antalet rader resp. kolumner i matrisen.

\Subtask Lägg till en metod \code{def updated(row: Int, col: Int)(value: T): Matrix[T]} som ger en ny matris där element på platsen \code{(row, col)} har uppdaterats till \code{value}.

\Subtask Lägg till en metod \code{def foreachIndex(f: (Int, Int) => Unit): Unit} som för varje index i \code{data} applicerar funktionen \code{f}.

\Subtask Lägg till en metod \code{override def toString} som så att en instans av \code{Matrix} visas enligt följande:
\begin{REPLnonum}
scala> val dm = Matrix.fill(3,4)(42.0)
val dm: Matrix[Double] =
Matrix of dim (3,4):
42.0 42.0 42.0 42.0
42.0 42.0 42.0 42.0
42.0 42.0 42.0 42.0
\end{REPLnonum}


\SOLUTION


\TaskSolved \what

\SubtaskSolved Typen på \code{m} blir \code{Matrix}.

\SubtaskSolved Typen på \code{e} blir \code{String}.

\SubtaskSolved Man behöver ändra på 3 ställen från \code{String} till \code{Int}.

\SubtaskSolved Generisk matris \code{Matrix[T]} för element av godtycklig typ \code{T}:

\begin{CodeSmall}
case class Matrix[T](data: Vector[Vector[T]]):
  def apply(row: Int, col: Int): T = data(row)(col)

object Matrix:
  def fill[T](dim: (Int, Int))(value: T): Matrix[T] =
    Matrix[T](Vector.fill(dim._1, dim._2)(value))
\end{CodeSmall}

\SubtaskSolved Tack vare kompilatorns typinferens så får \code{bm} typen \code{Matrix[Boolean]}.

\SubtaskSolved Typen på \code{be} blir \code{Boolean}.

\noindent \SubtaskSolved \SubtaskSolved \SubtaskSolved \SubtaskSolved \SubtaskSolved är alla implementerade i koden nedan: \vspace{-0.5em}
\begin{CodeSmall}
case class Matrix[T](data: Vector[Vector[T]]):
  require(data.forall(row => row.length == data(0).length))

  val dim: (Int, Int) = (data.length, data(0).length)

  def apply(row: Int, col: Int): T = data(row)(col)

  def updated(row: Int, col: Int)(value: T): Matrix[T] =
    Matrix(data.updated(row, data(row).updated(col, value)))

  def foreachIndex(f: (Int, Int) => Unit): Unit =
    for r <- data.indices; c <- data(r).indices do f(r, c)

  override def toString =
    s"""Matrix of dim $dim:\n${ data.map(_.mkString(" ")).mkString("\n") }"""

object Matrix:
  def fill[T](dim: (Int, Int))(value: T): Matrix[T] =
    Matrix[T](Vector.fill(dim._1, dim._2)(value))

\end{CodeSmall}

\QUESTEND


\clearpage

\ExtraTasks %%%%%%%%%%%%%%%%%%%%%%%%%%%%%%%%%%%%%%%%%%%%%%%%%


\WHAT{Imperativa matrisalgoritmer.}

\QUESTBEGIN

\Task  \what~Imperativa angreppssätt är nödvändiga att kunna när du stöter på samlingar och/eller språk som saknar funktionella metoder och/eller funktionsprogrammeringsmöjligheter. Genom att studera imperativa lösningar till de ofta mer koncisa funktionella lösningarna, får du träning i att skapa algoritmer som använder förändring genom tilldelning vid iterering.

\Subtask Implementera \code{isYatzy} från uppgift \ref{matrices:task:yatzy}\ref{matrices:subtask:isyatzyforall} igen, men nu med ett imperativt angreppssätt som använder en \code{while}-sats i stället för funktionella \code{forall}. Ta hjälp av en variabel \code{i} som håller reda på index och en variabel \code{foundDiff} som håller reda på om ett avvikande värde upptäcks. Funktionen kräver ca 9 rader, så det kan vara lämpligt att öppna en editor att skriva i medan du klurar ut lösningen. Börja med att skriva pseudokod, gärna med penna på papper. Prova genom att klistra in i REPL.

\Subtask En imperativ implementation av \code{diceMatrixToString} från uppgift \ref{matrices:task:yatzy}\ref{matrices:subtask:diceMatrixToString} med hjälp av förändringsbara  \code{StringBuilder}\footnote{\url{https://www.scala-lang.org/api/2.12.9/scala/collection/mutable/StringBuilder.html}} visas nedan. Förklara hur nedan kod fungerar. Vad händer om \code{xss} är tom? Vad händer om \code{xss} bara innehåller tomma vektorer? Nämn en fördel och en nackdel med att använda \code{val sb: StringBuilder} och \code{append}, jämfört med en vanlig, oföränderlig \code{var s: String} och \code{+} för tillägg i slutet.
\begin{Code}
def diceMatrixToString(xss: Vector[Vector[Int]]): String = 
  val sb = new StringBuilder()
  for(m <- xss.indices) do
    for(n <- xss(m).indices) do
      sb.append(xss(m)(n).toString)
      if n < xss(m).size - 1 then sb.append(" ")
      else if m < xss.size - 1 then sb.append("\n")
    end for
  end for
  sb.toString
\end{Code}

\Subtask Gör som träning en imperativ implementation av \code{filterYatzy} med en \code{for}-\code{do}-sats (alltså utan att använda \code{filter}, och utan att använda \code{yield}).


\Subtask Förklara hur nedan funktionella implementation av \code{filterYatzy} med \code{for}-\code{yield}-uttryck fungerar. Tycker du din imperativa lösning är lättare eller svårare att läsa och förstå jämfört nedan funktionella lösning?
\begin{CodeSmall}
def filterYatzy(xss: Vector[Vector[Int]]): Vector[Vector[Int]] = 
  (for i <- xss.indices if isYatzy(xss(i)) yield xss(i)).toVector
\end{CodeSmall}


\SOLUTION

\TaskSolved \what

\SubtaskSolved  \begin{Code}
def isYatzy(xs: Vector[Int]): Boolean = 
  var foundDiff = false
  var i = 0
  while (i < xs.size && !foundDiff) do
    foundDiff = xs(i) != xs(0)
    i += 1
  end while
  !foundDiff
\end{Code}


\SubtaskSolved  Funktionen går igenom varje matrisrad, där den i sin tur går igenom
varje element på raden och lägger till i \code{StringBuilder}-objektet. Om det inte är
det sista elementet på raden läggs även ett blanktecken till, annars läggs ett
nyradstecken till. Undantaget är sista raden, där inget nyradstecken läggs till.
Slutligen konverteras \code{StringBuilder}-objektet till en \code{String} som
returneras.


Är \code{xss} tom blir \code{xss.indices} en tom \code{Range} och den yttre \code{for}-loopen hoppas över och en tom sträng returneras.
Är alla rader tomma hoppas i stället de inre \code{for}-looparna över, med samma resultat.

\emph{Fördel:} \code{StringBuilder} är snabbare vid tillägg på slutet vid stora strängar (men här kommer det inte märkas eftersom strängen är så liten).

\emph{Nackdel:} StringBuilder-koden uppfattas av många som svårare att läsa.

\SubtaskSolved
\begin{Code}
def filterYatzy(xss: Vector[Vector[Int]]): Vector[Vector[Int]] = 
  var result: Vector[Vector[Int]] = Vector()
  for i <- xss.indices if isYatzy(xss(i)) do result = result :+ xss(i)
  result
\end{Code}

\SubtaskSolved  Varje looprunda ger en vektor \code{xss(i)} om filtervillkoret är uppfyllt och resultatet av \code{for}-uttrycket blir en vektor med vektorer som är yatzyslag.

\QUESTEND



\WHAT{Strängtabell med kolumnrubriker.}

\QUESTBEGIN

\Task  \what~  %Denna övning utgör en början på laboration \hyperref[section:lab:survey]{\texttt{survey}} i avsnitt \ref{section:lab:survey} på sidan \pageref{section:lab:survey}.

\Subtask Implementera case-klassen \code{Table} enligt specifikationen nedan. Du kan förutsätta att alla rader har lika många kolumner som antalet element i \code{headings}, samt att alla rubrikerna i \code{headings} är unika. Parametern \code{sep} anger det tecken som används för att separera kolumner. Detta förutsätts också gälla för indatafiler som läses in med \code{fromFile}.

\emph{Tips:}
\begin{itemize}%[nolistsep,noitemsep]
\item Värdet \code{indexOfHeading} kan skapas med hjälp av metoden \code{zipWithIndex} som fungerar på alla sekvenssamlingar, samt metoden \code{toMap} som fungerar på sekvenser av 2-tupler. Undersök först hur metoderna fungerar i REPL och sök upp deras dokumentation.
\item Skapa en indatafil som du kan använda för att testa att \code{Table} fungerar.
\end{itemize}


\begin{CodeSmall}
case class Table(
  data: Vector[Vector[String]],
  headings: Vector[String],
  sep: Char
):
  /** A 2-tuple with (number of rows, number of columns) in data */
  val dim: (Int, Int) = ???

  /** The element in row r and column c of data, counting from 0 */
  def apply(r: Int, c: Int): String = ???

  /** The row-vector r in data, counting from 0 */
  def row(r: Int): Vector[String]= ???

  /** The column-vector c in data, counting from 0 */
  def col(c: Int): Vector[String] = ???

  /** A map from heading to index counting from 0 */
  lazy val indexOfHeading: Map[String, Int] = ???

  /** The column-vector with heading h in data */
  def col(h: String): Vector[String] = ???

  /** A vector with the distinct, sorted values of col with heading h */
  def values(h: String): Vector[String] = ???

  /** Headings and data with columns separated by sep */
  override lazy val toString: String = ???

object Table:
  /** Creates a new Table from fileName with columns split by sep */
  def fromFile(fileName: String, sep: Char = ';'): Table = ???
\end{CodeSmall}

\Subtask Skapa med hjälp av \code{Table} ett program som kan köras från terminalen med \texttt{scala run infile.csv ';'} som ger en utskrift av antalet förekomster av olika värden i respektive kolumn (alltså en variant av registrering).



\SOLUTION

\TaskSolved \what

\SubtaskSolved  \begin{CodeSmall}
case class Table(
  data: Vector[Vector[String]],
  headings: Vector[String],
  sep: Char
):

  val dim: (Int, Int) = (data.size, headings.size)

  def apply(r: Int, c: Int): String = data(r)(c)

  def row(r: Int): Vector[String]= data(r)

  def col(c: Int): Vector[String] = data.map(r => r(c))

  lazy val indexOfHeading: Map[String, Int] = headings.zipWithIndex.toMap

  def col(h: String): Vector[String] = col(indexOfHeading(h))

  def values(h: String): Vector[String] = col(h).distinct.sorted

  override def toString: String =
    val s = sep.toString
    headings.mkString(s) + "\n" +data.map(_.mkString(s)).mkString("\n")

object Table:
  def fromFile(fileName: String, sep: Char = ';'): Table = 
    val lines = scala.io.Source.fromFile(fileName).getLines.toVector
    val matrix= lines.map(_.split(sep).toVector)
    new Table(matrix.tail, matrix.head, sep)
\end{CodeSmall}

\SubtaskSolved  \begin{CodeSmall}
@main 
def run(fileName: String, separator: String): Unit = 
  require(separator.length == 1, "separator ska vara exakt ett tecken")
  val t = Table.fromFile(fileName, separator.head)
  val counts: Vector[Vector[String]] =
    (0 until t.dim._2)
      .map(i => t.values(t.headings(i))
      .map(x => s"$x: ${t.col(i).count(_ == x)}"))
      .toVector
  for (i <- 0 until t.dim._2) do
    println(s"\nColumn: ${i + 1}, ${t.headings(i)}:")
    for (j <- 0 until counts(i).length) do
      println(counts(i)(j))
\end{CodeSmall}

\QUESTEND




\WHAT{Skapa ett yatzy-spel för användning i terminalen.}

\QUESTBEGIN

\Task  \what~%
% \Subtask Skapa en yatzy-matris enligt nedan specifikation. Läs om hur de olika predikaten för att kolla olika giltiga kombinationer i Yatzy ska fungera här: \href{https://en.wikipedia.org/wiki/Yahtzee}{en.wikipedia.org/wiki/Yahtzee}. Bygg ett huvudprogram som testar dina funktioner. Kompilera och testa i terminalen allteftersom du lägger till nya funktioner.
%
% \begin{CodeSmall}
% /** En skiss på en klass som kan användas till ett förenklat yatzy-spel */
% case class YatzyRows(val rows: Vector[Vector[Int]]) {
%   /** A new YatzyRows with a new row of 5 dice rolls appended to rows  */
%   def roll: YatzyRows = ???
%
%   /** A new YatzyRows with some indices of the last row re-rolled  */
%   def reroll(indices: Vector[Int]): YatzyRows = ???
% }
%
% object YatzyRows {
%   def isYatzy(xs: Vector[Int]): Boolean = ???
%   def isThreeOfAKind(xs: Vector[Int]): Boolean = ???
%   def isFourOfAKind(xs: Vector[Int]): Boolean = ???
%   def isFullHouse(xs: Vector[Int]): Boolean = ???
%   def isSmallStraight(xs: Vector[Int]): Boolean = ???
%   def isLargeStraight(xs: Vector[Int]): Boolean = ???
% }
% \end{CodeSmall}
%
%
% \Subtask Använd \code{YatzyRows} för att med hjälp av många tärningskast beräkna sannolikheter för några olika giltiga kombinationer. Använd, om du vill, möjligheten som reglerna ger att slå om tärningar i två ytterliggare kast, där de tärningar som slås om väljs slumpmässigt.
%
%\Subtask
Bygg ett förenklat yatzy-spel i terminalen där användaren kan bestämma vilka tärningar som ska slås om. Börja med något riktigt enkelt och bygg sedan vidare på ditt spel genom att införa fler och fler funktioner.

\SOLUTION


\TaskSolved \what
     %starts with: \emph{Skapa ett yatzy-spel för %%%

 --

% \SubtaskSolved   \begin{CodeSmall}
% /** En skiss på en klass som kan användas till ett förenklat yatzy-spel */
% case class YatzyRows(val rows: Vector[Vector[Int]]) {
%
%   private def throwDie: Int = (math.random() * 6).toInt + 1
%
%   /** A new YatzyRows with a new row of 5 dice rolls appended to rows */
%   def roll: YatzyRows = new YatzyRows(rows :+ Vector.fill(5)(throwDie))
%
%   /** A new YatzyRow with some indices of the last row re-rolled */
%   def reroll(indices: Vector[Int]): YatzyRows =
%     new YatzyRows(rows :+ rows(rows.length - 1).zipWithIndex.map {
%       case (x, i) => if (indices.contains(i)) throwDie else x
%     })
% }
% object YatzyRows {
%
%   def isYatzy(xs: Vector[Int]): Boolean = xs.forall(_ == xs(0))
%
%   def isThreeOfAKind(xs: Vector[Int]): Boolean =
%     xs.exists(x => xs.count(_ == x) >= 3)
%
%   def isFourOfAKind(xs: Vector[Int]): Boolean =
%     xs.exists(x => xs.count(_ == x) >= 4)
%
%   def isFullHouse(xs: Vector[Int]): Boolean =
%     xs.exists(x => xs.count(_ == x) == 3) &&
%     xs.exists(x => xs.count(_ == x) == 2)
%
%   def isSmallStraight(xs: Vector[Int]): Boolean =
%     xs.forall(x => xs.count(_ == x) == 1) && !xs.exists(_ == 6)
%
%   def isLargeStraight(xs: Vector[Int]): Boolean =
%     xs.forall(x => xs.count(_ == x) == 1) && !xs.exists(_ == 1)
% }
%
% \end{CodeSmall}
% Observera att fem stycken 2:or uppfyller kraven för Yatzy, men även för triss och fyrtal.
%
% \SubtaskSolved   Slumpen gör att utfallet inte kommer stämma exakt överens med teorin, men för ett stort antal kast bör resultaten hamna ganska nära. De teoretiska sannolikheterna (utan omkast) finns i \ref{yatzyProb}.
% \begin{table}[h]
% \centering
% \caption{Sannolikhet för olika Yatzy-resultat}
% \label{yatzyProb}
% \begin{tabular}{ll}
% Yatzy&  $0,077\%$  \\
% $\geq3$ av samma& $21\%$\\
% $\geq4$ av samma& $2,0\%$\\
% Kåk& $3,9\%$\\
% Liten stege& $1,5\%$\\
% Stor stege& $1,5\%$
% \end{tabular}
% \end{table}
%
% Kodexempel:
% \begin{CodeSmall}
% import YatzyRows._
%
% object YatzyStats extends App {
%   val n = 1000000.0
%   var yr = YatzyRows(Vector(Vector[Int]()))
%   for (i <- 1 to n.toInt) yr = yr.roll
%   println(s"Yatzy: ${yr.rows.count(isYatzy(_)) / n * 100}%")
%   println(s"Three of a kind: ${yr.rows.count(isThreeOfAKind(_)) / n * 100}%")
%   println(s"Four of a kind: ${yr.rows.count(isFourOfAKind(_)) / n * 100}%")
%   println(s"Full house: ${yr.rows.count(isFullHouse(_)) / n * 100}%")
%   println(s"Small straight: ${yr.rows.count(isSmallStraight(_)) / n * 100}%")
%   println(s"Large straight: ${yr.rows.count(isLargeStraight(_)) / n * 100}%")
% }
% \end{CodeSmall}
%
% \SubtaskSolved  --

\QUESTEND






\clearpage

\AdvancedTasks %%%%%%%%%%%%%%%%%


\WHAT{Generiska funktioner.}

\QUESTBEGIN

\Task  \what~  En generisk funktion har (minst) en typparameter inom klammerparenteser efter namnet, till exempel \code{[T]}. Denna typ förekommer sedan som typ på (någon av) parametrarna i parameterlistan. Kompilatorn härleder en konkret typ vid kompileringstid och ersätter typparametern med denna konkreta typ. På så sätt kan en funktion fungera för många olika typer.

\Subtask Förklara för varje rad nedan vad som händer.

\begin{REPL}
scala> def tnirp[T](x: T): Unit = println(x.toString.reverse)
scala> tnirp(42)
scala> tnirp("hej")
scala> case class Gurka(vikt: Int)
scala> tnirp(Gurka(42))
scala> tnirp[String](42)
scala> tnirp[Double](42)
\end{REPL}

\Subtask Man kan kombinera generiska funktioner med funktioner som tar funktioner som parametrar. Det är så \code{map} och \code{foreach} är implementerade. Förklara för varje rad nedan vad som händer.

\begin{REPL}
scala> def compose[A, B, C](f: A => B, g: B => C)(x: A): C = g(f(x))
scala> def inc(x: Int): Int = x + 1
scala> def half(x: Int): Double = x / 2.0
scala> compose(inc, half)(42)
scala> compose(half, inc)(42)
\end{REPL}

\Subtask Hur lyder felmeddelandet på sista raden ovan? Ändra \code{inc} och/eller \code{half} så att typerna passar.

\SOLUTION

\TaskSolved \what
     %starts with: \emph{Generiska funkioner.} En %%%

%4.a)
\SubtaskSolved   \begin{enumerate}
\item --
\item Strängrepresentationen av \code{42} spegelvänds
\item \code{"hej"} spegelvänds - \code{toString} av en sträng ger en likadan sträng
\item --
\item Gurk-objektets strängrepresentation spegelvänds
\item Funktionens typparameter matchar inte parameterns typ: \code{42} är ingen sträng
\item Implicit typkonvertering till \code{Double} sker för att stämma överens med typparametern, vilket ger en strängrepresentation med decimal
\end{enumerate}

%4.b)
\SubtaskSolved   \begin{enumerate}
\item En funktion definieras så att den tar emot två andra funktioner som argument, sätter ihop dem, och matar in ett tredje argument till den den sammansatta funktionen.
\item En funktion som inkrementerar ett heltal med 1 definieras.
\item En funktion som halverar ett flyttal definieras.
\item \code{42} matas in i \code{inc()} och resultatet (\code{43}) matas vidare till \code{half()}. Inuti \code{half()} sker implicit typkonvertering till \code{Double} då talet divideras med ett flyttal (\code{2.0}) och resultatet blir \code{43.0 / 2.0}, alltså \code{21.5}.
\item Resultatet från \code{half()} är av typ \code{Double}, medan \code{inc()} tar emot ett argument av typ \code{Int}. Då flyttal generellt inte kan konverteras till heltal utan informationsförlust sker ingen implicit konvertering, istället sker ett kompileringsfel.
\end{enumerate}

%4.c)
\SubtaskSolved  \begin{Code}
def inc(x: Double): Double = x + 1.0
\end{Code}
Nu ges kompileringsfel på rad 4 istället, vilket kan lösas med följande ändring:
\begin{Code}
def half(x: Double): Double = x / 2.0
\end{Code}

\QUESTEND




\WHAT{Generiska klasser.}

\QUESTBEGIN

\Task  \what~  Även klasser kan vara generiska. En generisk klass har (minst) en typparameter inom klammerparenteser efter klassens namn.

\Subtask Testa nedan generiska klass \code{Cell[T]} i REPL. Skapa instanser av klassen \code{Cell[T]} där typparametern \code{T} binds till olika konkreta typer och förklara vad som händer.

\begin{REPL}
scala> class Cell[T](var value: T):
         override def toString = "Cell(" + value + ")"
       
scala> new Cell(42)
scala> new Cell("hej")
scala> new Cell(new Cell(math.Pi))
scala> new Cell[String](42)
scala> new Cell[Double](42)
\end{REPL}

\Subtask Lägg till metoden \code{def concat[U](that: Cell[U]):Cell[String]} i klassen \code{Cell} som konkatenerar strängrepresentationerna av de båda cellvärdena.

\begin{REPL}
scala> val a = new Cell("hej")
scala> val b = new Cell(42)
scala> a concat b
\end{REPL}

\Subtask Vilken sorts celler kan du konkatenera om du tar bort typparameternamnet \code{U} i \code{concat} samtidigt som du använder \code{Cell[T]} som typ på värdeparametern \code{that}? Vad ger det för konsekvenser för celler av annan typ än \code{Cell[String]}?

\SOLUTION

\TaskSolved \what

%5.a)
\SubtaskSolved  --

%5.b)
\SubtaskSolved  \begin{Code}
class Cell[T](var value: T):
  override def toString = "Cell(" + value + ")"
  def concat[U](that: Cell[U]): Cell[String] = 
    Cell(s"$value${that.value}")
\end{Code}

%5.c)
\SubtaskSolved   Endast celler med samma typparameter kan nu konkateneras. Eftersom \code{concat()} returnerar ett objekt av typ \code{Cell[String]} kan ett ojämnt antal celler med någon annan typparameter än \code{String} alltså inte längre konkateneras. Är antalet jämnt går det att konkatenera dem parvis och sedan konkatenera de returnerade \code{Cell[String]}-objekten, men det är något omständigt.

\QUESTEND

\WHAT{Implementera fler generiska metoder i \code{Matrix[T]}.}

\QUESTBEGIN

\Task \what~ Bygg vidare på uppgift \ref{exe:matrices:labprep} och implementera nedan specifikation. Skapa egna tester som kontrollerar att alla metoder fungerar som förväntat.

\begin{ScalaSpec}{Matrix[T]}
/** En oföränderlig, generisk Matris-klass. */
case class Matrix[T](data: Vector[Vector[T]]):
  require(???)  // garantera att alla rader har lika många kolumner

  /** Ger ett par med antal rader och kolumner. */
  val dim: (Int, Int) = ???

  /** Ger elementet på plats (row, col). */
  def apply(row: Int, col: Int): T = ???

  /** Ger en ny matris där elementet på plats (row, col) har värdet value. */
  def updated(row: Int, col: Int)(value: T): Matrix[T] =  ???

  /** Applicerar f på alla element. */
  def foreach(f: T => Unit): Unit = ???

  /** Applicerar f på alla index. */
  def foreachIndex(f: (Int, Int) => Unit): Unit = ???

  /** Ger en ny matris med resultaten av elementvis applicering av f. */
  def map[U](f: T => U): Matrix[U] = ???

  /** Ger en ny matris med resultaten av applicering av f på varje index. */
  def mapIndex[U](f: (Int, Int) => U): Matrix[U] = ???

  /** Ger en utskriftsvänlig strängrepresentation av matrisen. */
  override def toString = ???

object Matrix:
  /** Ger en matris med dimension dim där alla element har värdet value. */
  def fill[T](dim: (Int, Int))(value: T): Matrix[T] = ???
\end{ScalaSpec}

\SOLUTION


\TaskSolved \what

\begin{CodeSmall}
case class Matrix[T](data: Vector[Vector[T]]):
  require(data.forall(row => row.size == data(0).size))

  val dim: (Int, Int) = (data.length, data(0).length)

  def apply(row: Int, col: Int): T = data(row)(col)

  def updated(row: Int, col: Int)(value: T): Matrix[T] =
    Matrix(data.updated(row, data(row).updated(col, value)))

  def foreach(f: T => Unit): Unit = data.foreach(_.foreach(f))

  def foreachIndex(f: (Int, Int) => Unit): Unit =
    for r <- data.indices; c <- data(r).indices do f(r, c)

  def map[U](f: T => U): Matrix[U] = Matrix(data.map(_.map(f)))

  def mapIndex[U](f: (Int, Int) => U): Matrix[U] =
    var result = Matrix.fill(dim)(f(0,0))
    for 
      r <- data.indices
      c <- data(r).indices 
    do
      result = result.updated(r, c)(f(r, c))
    end for
    result

  override def toString =
    s"""Matrix of dim $dim:\n${ data.map(_.mkString(" ")).mkString("\n") }"""

object Matrix:
  def fill[T](dim: (Int, Int))(value: T): Matrix[T] =
    Matrix[T](Vector.fill(dim._1, dim._2)(value))
\end{CodeSmall}


\QUESTEND





% \WHAT{Skapa en generisk, oföränderlig matrisklass.}
%
% \QUESTBEGIN
%
% \Task \label{task:generic-matrix} \what~   Med hjälp av en typparameter kan vi skapa en matrisklass som kan innehålla vilka element som helst. Implementera nedan specifikation. Testa din matrisklass i REPL för olika typer av element.
%
% \begin{ScalaSpec}{Matrix[T]}
% case class Matrix[T](data: Vector[Vector[T]]){
%
%   def foreachRowCol(f: (Int, Int, T) => Unit): Unit =
%     for (r <- 0 until data.size) {
%       for (c <- 0 until data(r).size) {
%         f(r, c, data(r)(c))
%       }
%     }
%
%   def map[U](f: T => U): Matrix[U] = Matrix(data.map(_.map(f)))
%
%   /** The element at row r and column c */
%   def apply(r: Int, c: Int): T = ???
%
%   /** Gives Some[T](element) at row r and column c
%    *  if r and c are within index bounds, else None */
%   def get(r: Int, c: Int): Option[T] = ???
%
%   /** The row vector of row r */
%   def row(r: Int): Vector[T] = ???
%
%   /** The column vector of column c */
%   def col(c: Int): Vector[T] = ???
%
%   /** A new Matrix with element at row r and col c updated */
%   def updated(r: Int, c: Int, value: T): Matrix[T] = ???
% }
% object Matrix {
%   def fill[T](rowSize: Int, colSize: Int)(init: T): Matrix[T] =
%     new Matrix(Vector.fill(rowSize)(Vector.fill(colSize)(init)))
% }
% \end{ScalaSpec}
%
% \SOLUTION
%
%
% \TaskSolved \what
%      %%%TODO number  8 %%%starts with: \label{task:generic-matrix} \em%%%
%
% \SubtaskSolved  -- %%%TODO in task 8 %%%
%
%
%
% \QUESTEND
%

% \clearpage
%
% \WHAT{Skapa en Sprite-editor.}
%
% \QUESTBEGIN
%
% \Task  \what~ Använd matrisklassen från uppgift \ref{task:generic-matrix} för att göra en SpriteEditor med JColorChoser enligt nedan skiss.
%
% \begin{Code}
% object ColorChooser {
%   import java.awt.Color
%   import javax.swing.JColorChooser
%
%   var title = "Pick Color"
%   private val chooser = new JColorChooser(Color.BLACK)
%   private val dialog = JColorChooser.
%     createDialog(null, title, true, jcs, null, null)
%
%   def getColor(initColor: Color = Color.BLACK): Color = {
%     chooser.setColor(initColor)
%     dialog.setVisible(true)
%     chooser.getColor
%   }
% }
%
% class Sprite(// en bild med många lager av pixlar i olika färger
%   val id: String,
%   val size: (Int, Int),
%   val pixels: Matrix[Int],   // färg i colors, -1 betyder genomskinlig
%   var scale: Int,            // uppskalning av storlek i pixlar
%   var colors: Vector[Color], // tillgängliga färger
%   var pos: (Int, Int, Int)   // (row, col, layer)
% ){
%   def row = pos._1
%   def col = pos._2
%   def layer = pos._3
% }
%
% class SpriteEditor(
%     rows: Int = 64, cols: Int = 64,
%     scale: Int = 16, nColors: Int = 16) {
%   private val w = new SimpleWindow(???)
%   def edit: Unit = ???
% }
%
% \end{Code}
%
%
%
% \SOLUTION
%
%
% \TaskSolved \what
%      %%%TODO number  9 %%%starts with: \TODO \emph{Klasser för täta oc%%%
%
% \SubtaskSolved  -- %%%TODO in task 9 %%%
%
% \SubtaskSolved  -- %%%TODO in task 9 %%%
%
% \SubtaskSolved  -- %%%TODO in task 9 %%%
%
% \SubtaskSolved  -- %%%TODO in task 9 %%%
%
% \SubtaskSolved  -- %%%TODO in task 9 %%%
%
% \SubtaskSolved  -- %%%TODO in task 9 %%%
%
%
%
% \QUESTEND




% \WHAT{Klasser för täta och glesa matematiska matriser med flyttal.}
%
% \QUESTBEGIN
%
% \Task  \what~   Läs om matrisräkning här: \href{https://sv.wikipedia.org/wiki/Matris}{sv.wikipedia.org/wiki/Matris}
%
% \Subtask Skapa en oföränderlig klass \code{DenseMatrix} för matematiska matriser med dubbelprecisionsflyttal. \code{DenseMatrix} ska internt lagra elementen i en privat \emph{endimensionell} array av flyttal av typen \code{Array[Double]}.
%
% Klassen ska inte vara en case-klass. Det ska gå att skapa matriser med uttryck så som  \code{DenseMatrix.ofDim(3,7)(1.0,42,3.2,1.0,2.2,3)} tack vare ett kompanjonsobjekt med lämplig fabriksmetod som anropar den privata konstruktorn.  Om antalet element är för litet i förhållande till den angivna dimensionen så fyll på med nollor.
%
% \Subtask Överskugga metoderna equals och hashcode och ge \code{DenseMatrix} innehållslikhet i stället för referenslikhet.
%
% \Subtask Implementera egna innehålllikhetsmetoder med namnet \code{===} på \code{DenseMatrix} som är typsäker, d.v.s. bara tillåter innehållsjämförelse mellan täta matriser.
%
% \Subtask Läs om glesa matriser här: \href{https://sv.wikipedia.org/wiki/Gles_matris}{https://sv.wikipedia.org/wiki/Gles\_matris} och implementera \code{SparseMatrix} med ett privat attribut av typen \\ \code{mutable.Map[(Int, Int), Double]} som bara lagrar index som inte är noll.
%
% \Subtask Skapa ett \code{trait Matrix} som både \code{DenseMatrix} och \code{SparseMatrix} ärver, med lämpliga abstrakta och konkreta medlemmar. Implementera addition, subtraktion och multiplikation av täta och glesa matriser.
%
% %\Task \emph{Matriser med \jcode{ArrayList} i Java.} Om man i Java inte vet antalet element i matrisen från början kan man använda en lista av typen \jcode{ArrayList}, där varje element i sin tur innehåller en lista av typen\jcode{ArrayList}. Javas \jcode{ArrayList} är en generisk samling som motsvaras av Scalas \code{ArrayBuffer}. Generiska samlingar i Java kan endast innehålla referenstyper; vill man ha en primitiv typ, t.ex. \jcode{int}, behöver man packa in denna i en s.k. wrapper-klass, t.ex.  klassen \jcode{Integer}. Det finns en wrapper-klass för varje primitiv typ i Java. Matristypen för en heltalstyp i Java skrivs \jcode{ArrayList<ArrayList<Integer>>} där alltså \code{<T>} motsvarar Scalas hakparenteser \code{[T]} för typparametern T.
% %
% %
%
% \SOLUTION
%
% \TaskSolved \what
%      %%%TODO number  10 %%%starts with: \emph{Matriser med \jcode{Array%%%
%
% \SubtaskSolved  -- %%%TODO in task 10 %%%
% \QUESTEND

%!TEX encoding = UTF-8 Unicode
%!TEX root = ../exercises.tex

\ifPreSolution


\Exercise{\ExeWeekNINE}\label{exe:W09}

\begin{Goals}
%!TEX encoding = UTF-8 Unicode
%!TEX root = ../compendium2.tex

%\item Kunna skapa och använda tupler, som variabelvärden, parametrar och returvärden.

%\item Förstå skillnaden mellan ett objekt och en klass och kunna förklara betydelsen av begreppet instans.

%\item Kunna skapa och använda attribut som medlemmar i objekt och klasser och som som klassparametrar.

%\item Kunna beskriva den praktiska nyttan med att ett attribut är privat.

%\item Kunna byta ut implementationen av metoden \code{toString}.

%\item Kunna skapa och använda en objektfabrik med metoden \code{apply}.

%\item Kunna skapa och använda en enkel case-klass.

%\item Kunna använda operatornotation och förklara relationen till punktnotation.

%\item Förstå konsekvensen av uppdatering av föränderlig data i samband med multipla referenser.

%\item Kunna förklara den principiella skillnaderna mellan olika typer av samlingar.
\item Kunna skapa och använda tupler som parametrar och returvärden.
\item Känna till och kunna använda grundläggande metoder på samlingar.
\item Kunna skapa och använda både oföränderliga och föränderliga mängder.
\item Förstå skillnader och likheter mellan en mängd och en sekvens.
\item Kunna beskriva hur algoritmen linjärsökning fungerar.
\item Kunna skapa och använda både oföränderliga och föränderliga nyckel-värde-tabeller.
\item Kunna använda nyckel-värde-tabeller för att implementera registrering.
\item Förstå likheter och skillnader mellan en nyckel-värde-tabell och en sekvens.
\item Kunna spara och läsa data till/från textfiler på disk.
 
\end{Goals}

\begin{Preparations}
\item \StudyTheory{09}
\end{Preparations}

\else

\ExerciseSolution{\ExeWeekNINE}

\fi



\BasicTasks %%%%%%%%%%%%%%%%




\WHAT{Para ihop begrepp med beskrivning.}

\QUESTBEGIN

\Task \what

\vspace{1em}\noindent Koppla varje begrepp med den (förenklade) beskrivning som passar bäst:

\begin{ConceptConnections}
  mängd & 1 & & A & leta i sekvens tills sökkriteriet är uppfyllt \\ 
  nyckel-värde-tabell & 2 & & B & avkoda symbolsekvens och återskapa objekt i minnet \\ 
  mappning & 3 & & C & en unik identifierare \\ 
  nyckel & 4 & & D & egenskapen att finnas kvar efter programmets avslut \\ 
  persistens & 5 & & E & koda objekt till avkodningsbar sekvens av symboler \\ 
  serialisera & 6 & & F & oordnad samling av mappningar med unika nycklar \\ 
  de-serialisera & 7 & & G & \code|nyckel -> värde| \\ 
  linjärsöka & 8 & & H & oordnad samling med unika element \\ 
\end{ConceptConnections}

\SOLUTION

\TaskSolved \what

\begin{ConceptConnections}
  mängd & 1 & ~~\Large$\leadsto$~~ &  H & oordnad samling med unika element \\ 
  nyckel-värde-tabell & 2 & ~~\Large$\leadsto$~~ &  F & oordnad samling av mappningar med unika nycklar \\ 
  mappning & 3 & ~~\Large$\leadsto$~~ &  G & \code|nyckel -> värde| \\ 
  nyckel & 4 & ~~\Large$\leadsto$~~ &  C & en unik identifierare \\ 
  persistens & 5 & ~~\Large$\leadsto$~~ &  D & egenskapen att finnas kvar efter programmets avslut \\ 
  serialisera & 6 & ~~\Large$\leadsto$~~ &  E & koda objekt till avkodningsbar sekvens av symboler \\ 
  de-serialisera & 7 & ~~\Large$\leadsto$~~ &  B & avkoda symbolsekvens och återskapa objekt i minnet \\ 
  linjärsöka & 8 & ~~\Large$\leadsto$~~ &  A & leta i sekvens tills sökkriteriet är uppfyllt \\ 
\end{ConceptConnections}

\QUESTEND



\WHAT{Vad är en mängd?}
\QUESTBEGIN

\Task \what~ Förklara vad som händer nedan. Varför hamnar elementen i en ''konstig'' ordning? Varför ''försvinner'' det element?

\begin{REPL}
scala> val xs = Vector(1,2,3,1,2,3,4,5,7).toSet
xs: scala.collection.immutable.Set[Int] = Set(5, 1, 2, 7, 3, 4)
scala> xs.foreach(print)
512734
\end{REPL}

\SOLUTION

\TaskSolved \what~En mängd är en samling som snabbt kan ge svaret på frågan om ett visst element ingår i samlingen eller ej. Elementen i en mängd är unika. Tilläg av redan existerande element ignoreras. En mängd är inte en  sekvens, eftersom traversering med t.ex. \code{map} eller \code{foreach} inte (nödvändigtvis) sker i den ordning som elementen gavs när mängden konstruerades eller uppdaterades.

\QUESTEND


\WHAT{Använda mängder.}

\QUESTBEGIN

\Task \what

\vspace{1em}\noindent Para ihop varje uttryck till vänster med ett uttryck till höger som har samma värde:

\begin{ConceptConnections}
\input{generated/quiz-w09-setops-taskrows-generated.tex}
\end{ConceptConnections}

\SOLUTION

\TaskSolved \what

\begin{ConceptConnections}
  \code|Set(1, 2) ++ Set(1, 2)          | & 1 & ~~\Large$\leadsto$~~ &  I & \code|Set(1, 2)     | \\ 
  \code|(1 to 3).toSet                  | & 2 & ~~\Large$\leadsto$~~ &  G & \code|Set(1) + 2 + 3| \\ 
  \code|Vector.fill(3)(1).toSet         | & 3 & ~~\Large$\leadsto$~~ &  F & \code|Set(1, 2) - 2 | \\ 
  \code|Set(1, 2, 3) diff Set(1, 2)     | & 4 & ~~\Large$\leadsto$~~ &  B & \code|Set(3)        | \\ 
  \code|(1 to 7).toSet.apply(8)         | & 5 & ~~\Large$\leadsto$~~ &  H & \code|false         | \\ 
  \code|Set(1, 2, 3).sorted             | & 6 & ~~\Large$\leadsto$~~ &  D & \code|error: ...    | \\ 
  \code|Set(2,4) subsetOf (1 to 7).toSet| & 7 & ~~\Large$\leadsto$~~ &  E & \code|true          | \\ 
  \code|Set(1, -1, 2, -2).map(_.abs).sum| & 8 & ~~\Large$\leadsto$~~ &  A & \code|3             | \\ 
  \code|Set(1, 1, 1, 1, 1, 5).sum       | & 9 & ~~\Large$\leadsto$~~ &  C & \code|6             | \\ 
\end{ConceptConnections}

\QUESTEND


\WHAT{Räkna unika ord med hjälp av en mängd.}

\QUESTBEGIN

\Task \what~På veckans laboration ska vi göra automatisk språkbehandling av långa texter som vi delar upp i ord. Med metoden \code{s.split(' ').toVector} kan du dela upp en sträng \code{s} i en sekvens av ord, där \code{s} blivit uppdelad i många strängar vid varje blanktecken och alla blanktecken är borttagna.

\Subtask Använd metoderna \code{split} och \code{toSet} för skapa ett uttryck som beräknar hur många unika ord det finns i strängen \code{hej} nedan:
\begin{REPLnonum}
scala> val hej = "hej hej hemskt mycket hej"
\end{REPLnonum}

\Subtask Mängder är snabba på att kolla om ett element finns i mängden men du kan inte förvänta dig att elementen finns i någon viss ordning. Det finns en sekvenssamlingsmetod som skapar en sekvens med unika element ur en sekvens och behåller den ursprungliga ordningen. Vad heter metoden? \\\emph{Tips:} Leta i snabbreferensen eller sök på nätet. Metoden fungerar på alla samlingar som är av typen \code{Seq} och har ett namn som börjar med bokstäverna \code{di}.

\SOLUTION

\TaskSolved \what~

\SubtaskSolved
\begin{REPL}
scala> val hej = "hej hej hemskt mycket hej"
scala> val n = hej.split(' ').toSet.size
n: Int = 3
\end{REPL}

\SubtaskSolved Metoden \code{distinct} returnerar en sekvens med unika element och bibehållen ursprunglig ordning.

\QUESTEND




\WHAT{Skapa 2-tupler med metoden \code{->} som kan uttalas ''mappas till''.}

\QUESTBEGIN

\Task \what~Vi har tidigare sett hur två olika värden kan samlas i en 2-tupel, till exempel \code{(0, true)}. Par kan även skapas med hjälp av metoden \code{->} enligt nedan. Testa detta i REPL:
\begin{REPL}
scala> ("Skåne", "Lund")          // ett strängpar med vanlig 2-tupel
scala> "Skåne" -> "Lund"           // operatornotation med ->
scala> "Skåne".->("Lund")         // punktnotation med -> (inte alls vanligt)
\end{REPL}
Metoden \code{->} fungerar med alla typer och är en fabriksmetod för par. Metodnamnet liknar en högerpil och illustrerar en mappning från första till andra värdet.

\Subtask Fungerar det på par skapade med \code{->} att använda metoderna \code{_1} och \code{_2}?


\Subtask Deklarera en variabel \code{val huvudstad: Vector[(String, String)]} som innehåller mappningar mellan geografiska områden och deras huvudstäder enligt tabellen nedan.

\begin{table}[H]
  \renewcommand{\arraystretch}{1.2}
  \begin{tabular}{|l|l|}\hline
  Sverige & Stockholm \\\hline
  Danmark & Köpenhamn \\\hline
  Grönland & Nuuk \\\hline
  Skåne & Lund \\\hline
  \end{tabular}
\end{table}

\Subtask Skriv ett uttryck som plockar fram \code{"Lund"} ur \code{huvudstad}.

\SOLUTION


\TaskSolved \what

\SubtaskSolved Ja, fabriksmetoden returnerar ett helt vanligt par:
\begin{REPLnonum}
scala> val härBorJag = "Skåne" -> "Lund"
val härBorJag: (String, String) = (Skåne,Lund)

scala> härBorJag._1
val res0: String = Skåne

scala> härBorJag._2
val res1: String = Lund
\end{REPLnonum}


\SubtaskSolved

\begin{Code}
val huvudstad = Vector(
  "Sverige"  -> "Stockholm",
  "Danmark"  -> "Köpenhamn",
  "Grönland" -> "Nuuk",
  "Skåne"    -> "Lund"
)
\end{Code}

\SubtaskSolved
\begin{REPL}
scala> huvudstad(3)._2
val res2: String = Lund
\end{REPL}

\QUESTEND



\WHAT{Linjärsöka efter nyckel i sekvens av mappningar.}

\QUESTBEGIN

\Task \what~

\Subtask Implementera funktionen \code{lookupIndex} nedan med hjälp av samlingsmetoden \code{indexWhere} så att linjärsökning sker efter index för ett par i sekvensen där \code{key} finns på första platsen i paret.

\begin{Code}
def lookupIndex(xs: Vector[(String, String)])(key: String): Int = ???
\end{Code}

\Subtask Testa din funktion i REPL genom att slå upp index för Skånes huvudstad i sekvensen \code{huvudstad} från föregående uppgift.

\SOLUTION

\TaskSolved \what~

\SubtaskSolved
\begin{Code}
def lookupIndex(xs: Vector[(String, String)])(key: String): Int =
  xs.indexWhere(_._1 == key)
\end{Code}

\SubtaskSolved
\begin{REPL}
scala> val i = lookupIndex(huvudstad)("Skåne")
val i: Int = 3

scala> huvudstad(i)._2
val res2: String = Lund
\end{REPL}

\noindent Eller med funktioner som återanvändbara dellösningar:
\begin{REPL}
scala> val indexOf = lookupIndex(huvudstad) _

scala> def capital(key: String) = huvudstad(indexOf(key))._2

scala> capital("Skåne")
val res3: String = Lund

scala> capital("Sverige")
val res4: String = Stockholm
\end{REPL}

\QUESTEND



\WHAT{Nyckel-värde-tabell.}

\QUESTBEGIN

\Task \what~En nyckel-värde-tabell är en smart datastruktur som gör att du kan slå upp det värde som en nyckel mappar till \emph{utan} att linjärsökning behöver ske. Värdet plockas fram direkt på en konstant tid, d.v.s. tiden att slå upp ett värde beror \emph{inte} på antalet element i samlingen, utan sker med mycket liten fördröjning.

I Scala heter nyckelvärdetabeller \code{Map} med stort M och är praktiska att använda i många olika sammanhang. \code{Map} finns i både en oföränderlig och en förändringsbar variant. Det går med metoder på formen \code{toXXX} lätt att omvandla mellan en \code{Map} och en sekvens av par av typen \code{XXX[(Nyckeltyp, Värdetyp)]}.

\Subtask Deklarera mappen \code{telnr} nedan i REPL och använd \code{apply} för att ta reda på telefonnumret till Fröken Ur.

\Subtask Vad har \code{telnr} för typ?

\Subtask Vad har \code{telnr.toVector} för typ?

\begin{Code}
val telnr = Map(
  "Anna"     -> 46462229812L,
  "Björn"     -> 46462229009L,
  "Sandra"    -> 46462220368L,
  "Fröken Ur" -> 4690510L,
)
\end{Code}
En uppsättning \code{Map}-instanser, vid behov nästlade, kan med fördel användas för att bygga upp en i-minnet-databas där inbyggda samlingsmetoder, t.ex. \code{map}, \code{filter}, och \code{for}-\code{yield}-uttryck, ger flexibla och effektiva sökmöjligheter. På veckans laboration ska du göra detta.

Samlingen \code{Map} är en generalisering av en sekvens, där man kan ''indexera'', inte bara med ett heltal, utan med vilken typ av värde som helst, t.ex. en sträng. Datastrukturen \code{Map} kallas också \emph{associativ array}\footnote{\href{https://en.wikipedia.org/wiki/Associative_array}{https://en.wikipedia.org/wiki/Associative\_array}} och är implementerad som en s.k. \emph{hashtabell}\footnote{\href{https://en.wikipedia.org/wiki/Hash_table}{https://en.wikipedia.org/wiki/Hash\_table}}, men du får vänta till fördjupningskursen innan vi går igenom hur en sådan datastruktur implementeras.

\SOLUTION

\TaskSolved \what~

\begin{REPL}
scala> telnr("Fröken Ur")
val res0: Long = 464690510

scala> :type telnr
Map[String,Long]

scala> :type telnr.toVector
Vector[(String, Long)]
\end{REPL}

\QUESTEND



\WHAT{Använda nyckel-värdetabell.}

\QUESTBEGIN

\Task \what~

\Subtask Skapa nedan variabler i REPL.
\begin{Code}
val follow = for i <- 2 to 16 by 2 yield (i, i + 1)
val xs = follow.toMap
val ys = xs.toVector
\end{Code}
Hamnar mappningarna i \code{ys} i samma ordning som \code{follow}? Varför?

\Subtask Med \code{xs} och \code{ys} deklarerade i REPL enligt ovan, para ihop yttryck till vänster med rätt resultat till höger. Om du är osäker på de sammansatta uttrycken, prova enklare uttryck i REPL och undersök värde och typ hos delresultat.

\begin{ConceptConnections}
  \code|xs(2) + xs(4)                 | & 1 & & A & \code|1                     | \\ 
  \code|ys(2) + ys(4)                 | & 2 & & B & \code|-9                    | \\ 
  \code|ys(0)                         | & 3 & & C & \code|8                     | \\ 
  \code|xs(0)                         | & 4 & & D & \code|7                     | \\ 
  \code|(xs + (0 -> 1)).apply(0)      | & 5 & & E & \code|NoSuchElementException| \\ 
  \code|xs.keySet.apply(2)            | & 6 & & F & \code|(10, 11)              | \\ 
  \code|xs.mapValues(v => -v).apply(8)| & 7 & & G & \code|false                 | \\ 
  \code|xs isDefinedAt 0              | & 8 & & H & \verb|error: type mismatch  | \\ 
  \code|xs.getOrElse(0, 7)            | & 9 & & I & \code|(16, 17)              | \\ 
  \code|xs.maxBy(_._2)                | & 10 & & J & \code|true                  | \\ 
\end{ConceptConnections}

\SOLUTION

\TaskSolved \what


\SubtaskSolved Nej nyckel-värde-paren lagras i någon speciell ordning som bestäms av en intern, smart lagringsprincip enligt en s.k. hashfunktion\footnote{\url{https://sv.wikipedia.org/wiki/Hashfunktion}}, för att åstadkomma snabba uppslagningar av värden från nycklar och vilket normalt inte sammanfaller med ordningen i den sekvens som de skapades ur.

\SubtaskSolved

\begin{ConceptConnections}
    \code|xs(2) + xs(4)                 | & 1 & ~~\Large$\leadsto$~~ &  A & \code|8                     | \\ 
  \code|ys(0)                         | & 2 & ~~\Large$\leadsto$~~ &  C & \code|(10, 11)              | \\ 
  \code|xs(0)                         | & 3 & ~~\Large$\leadsto$~~ &  I & \code|NoSuchElementException| \\ 
  \code|(xs + (0 -> 1)).apply(0)      | & 4 & ~~\Large$\leadsto$~~ &  D & \code|1                     | \\ 
  \code|xs.keySet.apply(2)            | & 5 & ~~\Large$\leadsto$~~ &  G & \code|true                  | \\ 
  \code|xs isDefinedAt 0              | & 6 & ~~\Large$\leadsto$~~ &  H & \code|false                 | \\ 
  \code|xs.getOrElse(0, 7)            | & 7 & ~~\Large$\leadsto$~~ &  B & \code|7                     | \\ 
  \code|xs.maxBy(_._2)                | & 8 & ~~\Large$\leadsto$~~ &  E & \code|(16, 17)              | \\ 
  \code|xs.map(p => p._1 -> -p._2)(8) | & 9 & ~~\Large$\leadsto$~~ &  F & \code|-9                    | \\ 
\end{ConceptConnections}

%%% BELOW IS SOLVED IN SCALA 3 AND the err msg is better! :)
% \noindent \emph{Fördjupning}:  Felmeddelandet som rad 2 ovan orsakar är lurigt:

% \begin{REPL}
% scala> ys(2)
% val res22: (Int, Int) = (6,7)

% scala> ys(4)
% val res23: (Int, Int) = (12,13)

% scala> ys(2) + ys(4)
% <console>:13: error: type mismatch;
%  found   : (Int, Int)
%  required: String
%        ys(2) + ys(4)

% \end{REPL}
% Det går som förväntat inte att addera två tupler, men varför säger kompilatorn att en sträng krävs?!? Detta beror på att, i enlighet med hur det fungerar i Java, valde Scala-språkets konstruktörer att låta strängsammanfogning fungera med alla möjliga typer vilket gör att kompilatorn inte ger upp när metoden \code{+} inte finns för tupler, utan i stället gör ett misslyckat försök med strängsammanfogning.

% Det mest olyckliga med detta är inte att felmeddelanden ibland blir missvisande, utan att det i vissa situationer inte ens \emph{blir} något felmeddelande, trots att man av rent misstag råkat strängkonkatenera i stället för t.ex. lägga till ett element i en mängd eller en mappning i en tabell. Detta typosäkra beteendet av strängsammanfogning har kritiserats, men det är inte okontroversiellt att ändra detta nu när så många utvecklare skrivit så mycket Scala-kod som bygger på strängars förmåga att kunna lägga till vad som helst på slutet. Situationen i Scala är dock inte hopplös efter introduktionen av stränginterpolering i Scala 2.10, som möjliggör infogande av värden i strängar på ett typsäkert sätt.
\QUESTEND





\WHAT{Registrering i förändringsbar nyckel-värde-tabell.}

\QUESTBEGIN

\Task \what~I denna uppgift ska du implementera en hjälpklass för registrering i en frekvenstabell som du sedan ska använda på veckans laboration. Klassen ska heta  \code{FreqMapBuilder} som efter upprepade anrop av metoden \code{add(s: String): Unit} kan skapa frekvenstabeller av typen \code{Map[String, Int]}, där nyckel-värde-paren i tabellen anger antalet förekomster av en viss sträng. Du ska utgå från koden nedan.

Klassen använder en förändringsbar tabell internt. Efter att man har lagt till många strängar kan man med metoden \code{toMap} få en oföränderlig tabell för  uppslagning av frekvenser för specifika strängar. Läs i snabbreferensen om vilka extra metoder för uppdatering som erbjuds av \code{mutable.Map[K, V]}.

\begin{Code}
class FreqMapBuilder:
  private val register = collection.mutable.Map.empty[String, Int]
  def toMap: Map[String, Int] = register.toMap
  def add(s: String): Unit = ???

object FreqMapBuilder:
  def apply(xs: String*): FreqMapBuilder = ???
\end{Code}

\noindent Implementera och testa \code{FreqMapBuilder}. \emph{Tips:} Du kan t.ex. använda metoderna \code{+=} och \code{getOrElse}.

\SOLUTION

\TaskSolved \what~
\begin{Code}
class FreqMapBuilder:
  private val register = scala.collection.mutable.Map.empty[String,Int]
  def toMap: Map[String, Int] = register.toMap
  def add(s: String): Unit =
    register += (s -> (register.getOrElse(s, 0) + 1))

object FreqMapBuilder:
  def apply(xs: String*): FreqMapBuilder = 
    val result = new FreqMapBuilder
    xs.foreach(result.add)
    result
\end{Code}

\QUESTEND



\WHAT{Metoden \code{sliding}.}

\QUESTBEGIN

\Task  \what~  I veckans laboration kommer du att ha nytta av metoden \code{sliding}, som ger en iterator för speciella delsekvenser av en sekvens, vilka kan liknas vid ''utsikten'' i ett ''glidande fönster''.

\Subtask Kör nedan i REPL och beskriv vad som händer.

\begin{REPL}
scala> val xs = Vector("fem", "gurkor", "är", "fler", "än", "fyra", "tomater")
scala> xs.sliding(2).toVector
scala> xs.sliding(3).toVector
scala> xs.sliding(10).toVector
\end{REPL}

\Subtask Använd \code{xs.sliding(2)} och omvandla varje element i resultatet till ett par. Gör sedan om sekvensen av par till en nyckel-värde-tabell. Vad kan tabellen användas till?

\SOLUTION

\TaskSolved \what

\SubtaskSolved
\begin{REPL}
scala> val xs = Vector("fem", "gurkor", "är", "fler", "än", "fyra", "tomater")
val xs: Vector[String] =
  Vector(fem, gurkor, är, fler, än, fyra, tomater)

scala> xs.sliding(2).toVector
val res9: Vector[Vector[String]] =
  Vector(Vector(fem, gurkor), Vector(gurkor, är), Vector(är, fler), Vector(fler, än), Vector(än, fyra), Vector(fyra, tomater))

scala> xs.sliding(3).toVector
val res10: Vector[Vector[String]] =
  Vector(Vector(fem, gurkor, är), Vector(gurkor, är, fler), Vector(är, fler, än), Vector(fler, än, fyra), Vector(än, fyra, tomater))

scala> xs.sliding(10).toVector
val res11: Vector[Vector[String]] =
  Vector(Vector(fem, gurkor, är, fler, än, fyra, tomater))

\end{REPL}
\code{xs.sliding(n).toVector} skapar en sekvens som innehåller sekvenser av längden \code{n} som bildas genom att ta varje element och dess \code{n - 1} efterföljande element.

\SubtaskSolved
\begin{REPL}
scala> xs.sliding(2).map(ys => ys(0) -> ys(1)).toMap
val res0: Map[String,String] =
  Map(är -> fler,
      än -> fyra,
      fyra -> tomater,
      gurkor -> är,
      fem -> gurkor,
      fler -> än
  )
\end{REPL}
Man kan använda tabellen till att slå upp vilket som är efterföljande ord. Det fungerar eftersom alla ord är unika. Om det funnits flera likadana ord med olika efterföljande ord så hade vi behövt skapa en tabell med nycklar som mappar till en samling som registrerar efterföljande ord. Detta ska vi göra på veckans laboration.

\QUESTEND




\WHAT{Läsa text från fil och webbservrar.}

\QUESTBEGIN

\Task \what~På laborationen ska du bygga upp tabeller från data i textformat. Då har du nytta av att kunna läsa text från filer och från webben. Testa detta i REPL:
\begin{REPL}
scala> val url = "https://fileadmin.cs.lth.se/pgk/europa.txt"
scala> val xs = io.Source.fromURL(url, "UTF-8").getLines.toVector
scala> val data = xs.map(_.split(';').toVector)
scala> data.head
scala> data.foreach(println)
\end{REPL}

\Subtask Skapa dessa tabeller ur sekvensen \code{data}:
\begin{Code}
val populationOf: Map[String, Int]    = ???  // länders invånarantal
val sizeOf:       Map[String, Int]    = ???  // länders yta i km^2
val capitalOf:    Map[String, String] = ???  // länders huvudstäder
\end{Code}
Testa tabellerna i REPL.

\Subtask Spara ner data i en textfil \code{europa.txt}. Läsa in data från filen med metoden \code{Source.fromFile(filnamn, teckenkodning)} på liknande sätt som med  \code{fromURL} ovan. Om du kör i en Linux-terminal kan du enkelt ladda ner en fil så här:
\begin{REPLnonum}
> wget https://fileadmin.cs.lth.se/pgk/europa.txt
\end{REPLnonum}
Skriv ut alla raderna i \code{europa.txt} med hjälp av \code{Source.fromFile} i REPL.

\SOLUTION

\TaskSolved \what~

\SubtaskSolved
\begin{CodeSmall}
val populationOf = data.tail.map(v => v(0) -> v(1).toInt).toMap
val sizeOf       = data.tail.map(v => v(0) -> v(2).toInt).toMap
val capitalOf    = data.tail.map(v => v(0) -> v(3)).toMap
\end{CodeSmall}

\begin{REPL}
scala> capitalOf("Sverige")
res2: String = Stockholm

scala> populationOf("Sverige")
res3: Int = 9223766

scala> sizeOf("Sverige")
res4: Int = 449964
\end{REPL}

\begin{REPL}
scala> val filename = "europa.txt"
scala> val xs = io.Source.fromFile(filename, "UTF-8").getLines.toVector
scala> val data = xs.map(_.split(';').toVector)
scala> data.map(_.map(_.take(15).padTo(15,' ')).mkString(" ")).foreach(println)
\end{REPL}
\QUESTEND





\ExtraTasks %%%%%%%%%%%%%%%%%%%%%%%%%%%%%%%%%%%%%%%%%%%%%%%%%%%%%%%%%%%%%%%%%%%%

\WHAT{Skapa ett textspel med hjälp av tabeller.}

\QUESTBEGIN

\Task \what~Gör ett enkelt spel för att träna på olika fakta om Europas länder och huvudstäder genom att läsa data från URL:en:\\ \url{https://fileadmin.cs.lth.se/pgk/europa.txt}
\\Där finns text kodad i UTF-8 med följande innehåll (endast de första raderna visas):
\begin{Code}
Land;Invånarantal;Storlek(km^2);Huvudstad
Albanien;3581655;28748;Tirana
Andorra;71201;468;Andorra la Vella
Belgien;10584534;30528;Bryssel
Bosnien-Hercegovina;4590310;51129;Sarajevo
Bulgarien;7385367;110910;Sofia
Cypern;854000;9250;Nicosia
Danmark;5475791;43094;Köpenhamn
Estland;1324333;45226;Tallinn
Finland;5315280;338145;Helsingfors
Frankrike;61538322;551695;Paris
Färöarna;48344;139574;Torshamn
Grekland;10964021;131940;Aten
// ... etcetera för alla Europas länder.
\end{Code}
Låt till exempel användaren svara på slumpvisa frågor av typen:
\begin{itemize}[noitemsep]
  \item Har Andorra fler invånare än Cypern?
  \item Vad heter huvudstaden i Bulgarien?
  \item Har Danmark större yta än Finland?
\end{itemize}
Använd oföränderliga tabeller med lämpliga nycklar och värden. Du kan använda en mängd med länder/huvudstäder som användaren hittills svarat rätt på för att kunna förhindra att dessa återkommer igen.
\SOLUTION

\TaskSolved --

\QUESTEND



\AdvancedTasks %%%%%%%%%%%%%%%%%%%%%%%%%%%%%%%%%%%%%%%%%%%%%%%%%%%%%%%%%%%%%%%%%


\WHAT{Registrering med \code{groupBy}.}

\QUESTBEGIN

\Task \what~Vi ska nu utnyttja ett riktigt listigt trick för att via en enda kodrad implementera registrering med hjälp av samlingsmetoderna \code{groupBy} och \code{map}.

\Subtask Läs om metoden \code{groupBy} i snabbreferensen. Du hittar den under rubriken \emph{''Methods in trait \code{Iterable[A]}''} eftersom \code{groupBy} fungerar på alla samlingar. Testa \code{groupBy} enligt nedan och beskriv vad som händer.

\begin{REPL}
scala> val xs = Vector(1, 1, 2, 2, 4, 4, 4).groupBy(x => x > 2)
scala> val ys = Vector(1, 1, 2, 2, 4, 4, 4).groupBy(x => x)
\end{REPL}

\Subtask Skapa en funktion \code{freq} med nedan funktionshuvud som returnerar en tabell med antalet förekomster av olika heltal i \code{xs}. Testa \code{freq} på en sekvens av 1000 slumpvisa tärningskast och förklara hur funktionen \code{freq} fungerar. \emph{Tips:} Gör först \code{groupBy(???)} och sedan \code{map(???)}.

\begin{Code}
def freq(xs: Vector[Int]): Map[Int, Int] = ???

def kasta(n: Int): Vector[Int] =
  Vector.fill(n)(scala.util.Random.nextInt(6) + 1)
\end{Code}

\SOLUTION

\TaskSolved \what~

\SubtaskSolved Metoden \code{groupBy} skapar en nyckel-värde-tabell där värdena i tabellen är en sekvens med elementen grupperade på ett speciellt sett.
Mer precist:

Resultatet av \code{xs.groupBy(f: K => V)} för en sekvens \code{xs} av typen \code{Vector[K]} blir en tabell av typen \code{Map[V,Vector[K]]} där varje element \code{e} i \code{xs} är grupperade i samma tabellvärde om de lika är enligt \code{f(e)}. Varje grupp får tabellnyckeln \code{f(e)}.

\emph{Listigt trick:} Om man låter funktionen \code{f} vara enhetsfunktionen som avbildar varje element på sig själv, alltså \code{x => x}, så grupperas värdena i samma sekvens om de är lika.

\begin{REPL}
scala> val xs = Vector(1, 1, 2, 2, 4, 4, 4).groupBy(x => x > 2)
val xs: Map[Boolean,Vector[Int]] =
  Map(false -> Vector(1, 1, 2, 2), true -> Vector(4, 4, 4))

scala> val ys = Vector(1, 1, 2, 2, 4, 4, 4).groupBy(x => x)
val ys: Map[Int,Vector[Int]] =
  Map(2 -> Vector(2, 2), 4 -> Vector(4, 4, 4), 1 -> Vector(1, 1))
\end{REPL}


\SubtaskSolved

\begin{Code}
def freq(xs: Vector[Int]): Map[Int, Int] =
  xs.groupBy(x => x).map(p => p._1 -> p._2.size)
\end{Code}
Förklaring: metoden \code{groupBy} skapar en tabell med par \code{k, v} där \code{v} är en sekvens med så många \code{k} som antalet gånger \code{k} förekommer i \code{xs}. Genom att omvandla alla värden \code{p._2} till storleken \code{p._2.size} får vi en frekvenstabell.

\begin{REPL}
scala> freq(kasta(1000))
val res0: Map[Int,Int] = 
  Map(5 -> 163, 1 -> 174, 6 -> 161, 2 -> 169, 3 -> 167, 4 -> 166)

scala> freq(kasta(1000)).toVector.sortBy(_._1).foreach(println)
(1,183)
(2,167)
(3,169)
(4,179)
(5,154)
(6,148)
\end{REPL}

\QUESTEND





\WHAT{Skriva till fil.}

\QUESTBEGIN

\Task \what~Som hjälp när du skapar egna intressanta applikationer eller bygger vidare på kursens laborationer och övningar med frivilliga extrauppgifter, kan du använda funktionerna i singelobjektet \code{IO} nedan, som finns i kursens scala-bibliotek \href{http://cs.lth.se/pgk/api}{introprog}.\footnote{Källkoden finns här och även på sidan \pageref{disk-access-code}:\\ \href{https://github.com/lunduniversity/introprog/blob/master/compendium/workspace/introprog/src/main/scala/introprog/IO.scala}{https://github.com/lunduniversity/introprog-scalalib/blob/master/src/main/scala/introprog/IO.scala}}

IO-modulen använder \code{scala.io.Source} för att serialisera och de-serialisera strängar till och från vanliga textfiler. IO-modulen använder även paketet \code{java.io} för att erbjuda funktioner som gör det enkelt att serialisera/de-serialisera godtyckliga objekt skapade med hjälp av serialserbara klasser till/från binärfiler. Case-klasser i Scala blir automatiskt serialiserbara.

I implementationen av \code{IO} används \code{try ... finally} för att säkerställa att filer inte lämnas öppnade även om något går fel under den läs/skriv-process som sköts av det underliggande operativsystemet.

\Subtask
Kompilera och resta nedan med \code{introprog} på classpath, t.ex. med hjälp av \code{sbt}.
\begin{Code}
import introprog.IO

case class Player(name: String)

@main def run(): Unit = 
  println("Test of output/input objects to/from disk:")
  val highscores = Map(Player("Sandra") -> 42, Player("Björn") -> 5)
  IO.saveObject(highscores,"highscores.ser")
  val highscores2 = IO.loadObject[Map[Player, Int]]("highscores.ser")
  val isSameContents = highscores2 == highscores
  val testResult = if (isSameContents) "SUCCESS :)" else "FAILURE :("
  println(testResult)
\end{Code}

\Subtask
Använd \code{IO}-modulen för att spara användarens poängresultat i ditt spel om Europas länder och städer, i extrauppgiften ovan. Implementationen av \code{introprog.IO} finns här: \url{https://github.com/lunduniversity/introprog-scalalib/blob/master/src/main/scala/introprog/IO.scala} 

% \begin{figure}
% %  \scalainputlisting[basicstyle=\ttfamily\fontsize{9.2}{11}\selectfont]{examples/IO.scala}
%   \scalainputlisting[basicstyle=\ttfamily\fontsize{9.2}{11}\selectfont]{../workspace/introprog/src/main/scala/introprog/IO.scala}
%   \label{disk-access-code}
% \end{figure}
\SOLUTION

\TaskSolved --

\QUESTEND



%
%
% \subsection{\TODO Värdera nedan gamla uppgifter}
%
%
%
% \WHAT{Objekt med attribut (fält).}
%
% \QUESTBEGIN
%
% \Task  \what~  Ett objekt kan samla data som hör ihop och på så sätt skapa en datastruktur. Data i ett objekt kallas \emph{attribut} eller \emph{fält}, \Eng{field}. Objekt som samlar enbart data kallas även \emph{post} \Eng{record}.
% \begin{REPLnonum}
% scala> object mittKonto { var saldo = 0; val nummer = 12345L }
% \end{REPLnonum}
% \Subtask Skriv en sats som sätter in ett slumpmässigt belopp mellan 0 och en miljon på \code{mittKonto} ovan med hjälp av punktnotation och tilldelning.
%
% \Subtask Vad händer om du försöker ändra attributet \code{nummer}?
%
% \SOLUTION
%
%
% \TaskSolved \what
%
%
% \SubtaskSolved   \code{mittKonto.saldo = (math.random() * 1000000).toInt}
%
% \SubtaskSolved   Går ej eftersom val är oföränderlig, man får alltså ett Error.
%
%
% \QUESTEND
%
%
%
%
% %%<AUTOEXTRACTED by mergesolu>%%      %Uppgift 2
%
%
%
%
% \WHAT{Klass med attribut.}
%
% \QUESTBEGIN
%
% \Task  \what~  Om du vill ha många objekt av samma typ, kan du använda en \textbf{klass}. På så sätt kan man skapa många datastrukturer av samma typ men med olika innehåll. Man skapar nya objekt med nyckelordet \code{new} följt av klassens namn. Klassen utgör en ''mall'' för objektet som skapas. Ett objekt som skapas med \code{new Klassnamn} kallas även en \textbf{instans} av klassen \code{Klassnamn}. Nedan skapas en datastruktur \code{Konto} som samlar data om ett bankonto. Instanser av typen \code{Konto} håller reda på hur mycket pengar det finns på kontot och vilket kontonumret är. Datavärden som sparas i varje objektinstans, så som \code{saldo} och \code{nummer}, kallas \textbf{attribut} \Eng{attribute} eller \textbf{fält} \Eng{field}.
%
% \begin{REPL}
% scala> class Konto {
%          var saldo = 0
%          var nummer = 0L
%        }
% scala> val k1 = new Konto
% scala> val k2 = new Konto
% scala> k1.saldo = 1000
% scala> k1.nummer = 12345L
% scala> k2.saldo = 2000
% scala> k2.nummer = 67890L
% scala> println("Konto: " + k1.nummer + " Saldo:" + k1.saldo)
% scala> println("Konto: " + k2.nummer + " Saldo:" + k2.saldo)
% \end{REPL}
%
% \Subtask\Pen Rita hur minnessituationen ser ut efter att ovan rader har exekverats.
%
% \Subtask\Pen Vad hade det fått för konsekvenser om attributet \code{nummer} vore oföränderligt i klassen ovan? (Jämför med objektet \code{mittKonto}.)
%
%
% \SOLUTION
%
%
% \TaskSolved \what
%
%
% \SubtaskSolved   \includegraphics[scale=0.5]{../img/w04-solutions/uppgift-3a}
%
% \SubtaskSolved
% Tilldelningen på rad 8 \code{k1.nummer = 12345L} ger felmeddelande eftersom variablen är oföränderlig.
%
%
% \QUESTEND
%
%
%
%
% %%<AUTOEXTRACTED by mergesolu>%%      %Uppgift 3
%
%
%
%
% \WHAT{Klass med attribut som parametrar.}
%
% \QUESTBEGIN
%
% \Task  \what~  Om man vill ge attributen initialvärden när objektet skapas med \code{new}, kan man placera attributen i en parameterlista till klassen. Koden som körs när objektet skapas och attributen tilldelas sina initialvärden, kallas \textbf{konstruktor} \Eng{constructor}.
%
% \begin{REPL}
% scala> class Konto(var saldo: Int, val nummer: Long)
% scala> val k = new Konto(0, 12345L)
% scala> println("Konto: " + k.nummer + " Saldo:" + k.saldo)
% scala> println(k)
% scala> k.toString
% \end{REPL}
%
% \Subtask Den två sista raderna ovan skriver ut den identifierare som JVM använder för att hålla reda på objektet i sina interna datastrukturer. Vad skrivs ut?
%
% \Subtask Skapa ännu en instans av klassen Konto  med samma saldo och nummer som \code{k} ovan och spara den i \code{val k2} och undersök dess objektidentifierare. Får objekten \code{k} och \code{k2} olika objektidentifierare?
%
% \Subtask Sätt in olika belopp på respektive konto.
%
% \Subtask Vad händer om du försöker ändra attributet \code{nummer}?
%
% \Subtask\Pen Ibland räcker det fint med en tupel, men ofta vill man ha en klass istället. Beskriv några fördelar med en Konto-klassen ovan jämfört med en tupel av typen \code{(Int, Long)}.
%
% \begin{REPLnonum}
% scala> var k3 = (0, 12345L)
% scala> k3 = (k3._1 + 100, k3._2)
% \end{REPLnonum}
%
% \SOLUTION
%
%
% \TaskSolved \what
%
%
% \SubtaskSolved   \code{String = Konto@cd576}, där \code{Konto@cd576} är ett unikt namn som identifierar instansen.
%
% \SubtaskSolved   Ja.
%
% \SubtaskSolved
% \begin{REPLnonum}
% scala> k.saldo = 42
% scala> k2.saldo = 67
% \end{REPLnonum}
%
% \SubtaskSolved   Eftersom variablen är oföränderlig ges ett felmeddelande.
%
% \SubtaskSolved   En fördel med klass är att man kan specificera att variablen ska kunna vara föränderlig. En till är att man kan inkludera metoder i klassen som man vill kunna använda på värdena.
%
%
% \QUESTEND
%
%
%
%
% %%<AUTOEXTRACTED by mergesolu>%%      %Uppgift 4
%
%
%
%
% \WHAT{Publikt eller privat attribut?}
%
% \QUESTBEGIN
%
% \Task  \what~  Man kan förhindra att ett attribut syns utanför klassen med hjälp av nyckelordet \code{private}.
%
% \begin{REPL}
% scala> class Konto1(val nummer: Long){ var saldo = 0 }
% scala> val k1 = new Konto1(12345678901L)
% scala> k1.nummer
% scala> k1.saldo += 1000
% scala> class Konto2(val nummer: Long){ private var saldo = 0 }
% scala> val k2 = new Konto2(12345678901L)
% scala> k2.nummer
% scala> k2.saldo += 1000
% \end{REPL}
%
% \Subtask Vad händer ovan?
%
% \Subtask Gör en ny version av klassen \code{Konto} enligt nedan:
%
% \begin{Code}
% class Konto(val nummer: Long){
%   private var saldo = 0
%   def in(belopp: Int): Unit = {saldo += belopp}
%   def ut(belopp: Int): Unit = {saldo -= belopp}
%   def show: Unit =
%     println("Konto Nr: " + nummer + " saldo: " + saldo)
% }
%
% object Main {
%   def main(args: Array[String]): Unit = {
%     val k = new Konto(1234L)
%     k.show
%     k.in(1000)
%     println("Uttag: " + k.ut(500))
%     println("Uttag: " + k.ut(1000))
%     k.show
%   }
% }
% \end{Code}
%
% \Subtask Spara koden i en fil, kompilera med \code{scalac} och kör. Testa även vad som händer om du försöker komma åt attributet \code{saldo} i main-metoden med t.ex. \code{println(k.saldo)} eller \code{k.saldo += 1000}.
%
% \Subtask Vi ska nu förhindra överuttag. Ändra i metoden \code{ut} så att den får signaturen \code{ut(belopp: Int): (Int, Int) = ???} och implementera \code{ut} så att den returnerar både beloppet man verkligen kan ta ut och kvarvarande saldo. Om man försöker ta ut mer än det finns på kontot så ska saldot bli 0 och man får bara ut det som finns kvar. Spara, kompilera, kör.
%
% \Subtask Förbättra metoderna \code{in} och \code{ut} så att man inte kan sätta in eller ta ut negativa belopp.
%
% \Subtask Vad är fördelen med att göra föränderliga attribut privata och bara påverka deras värden indirekt via metoder?
%
% \SOLUTION
%
%
% \TaskSolved \what
%
%
% \SubtaskSolved
% Det går bra att ändra på variablen saldo i instansen av Konto1 men inte av Konto2 där man får ett error på raden ''k2.saldo += 1000''
%
% \SubtaskSolved  -
%
% \SubtaskSolved
% ''println(k.saldo)'' och ''k.saldo += 1000'' ger båda error, pga privat attribut.
%
% \SubtaskSolved
% \begin{Code}
% def ut(belopp: Int): (Int, Int) = {
% 	if(saldo >= belopp) {
% 		saldo -= belopp
% 		(belopp, saldo)
% 	} else {
% 		val temp = saldo
% 		saldo = 0
% 		(temp, 0)
% 	}
% }
% \end{Code}
%
% \SubtaskSolved
% Lägg till en if-sats i båda funktionerna som omsluter den gamla koden.
% \begin{Code}
% def ut(belopp: Int): (Int, Int) = {
%   if(belopp >= 0) {
%     if(saldo >= belopp) {
%       saldo -= belopp
%       (belopp, saldo)
%     } else {
%       val temp = saldo
%       saldo = 0
%       (temp, 0)
%     }
%   }
% }
%
% def in(belopp: Int): Unit = {
%   if(belopp >= 0) {
%     saldo += belopp
%   }
% }
% \end{Code}
%
% \SubtaskSolved
% Genom att göra attributet privat och gör egna metoder kan man se till att attriuten endast ändras på säkra sätt. Så inte fel uppstår.
%
%
% \QUESTEND
%
%
%
%
% %%<AUTOEXTRACTED by mergesolu>%%      %Uppgift 5
%
%
%
%
% \WHAT{Vilken typ har ett objekt?}
%
% \QUESTBEGIN
%
% \Task  \what~  Objektets typ bestäms av klassen. Vid tilldelning måste typerna passa ihop.
%
% \Subtask Vilka rader nedan ger felmeddelande? Hur lyder felmeddelandet?
% \begin{REPL}
% scala> class Punkt(val x: Double, val y: Double)
% scala> val pt: Punkt = new Punkt(10.0, 10.0)
% scala> val i: Int = pt.x
% scala> val (x: Double, y: Double) = (pt.x, pt.y)
% scala> val p: Double = new Punkt(5.0, 5.0)
% scala> val p = new Punkt(5.0, 5.0): Double
% scala> val p = new Punkt(5.0, 5.0): Punkt
% scala> pt: Punkt
% \end{REPL}
%
%
% \Subtask Man kan undersöka om ett objekt är av en viss typ med metoden \\ \code{isInstanceOf[Typnamn]}. Vad ger nedan anrop av metoden \code{isInstanceOf} för värde?
% \begin{REPL}
% scala> class Punkt(val x: Double, val y: Double)
% scala> val pt: Punkt = new Punkt(1.0, 2.0)
% scala> pt.isInstanceOf[Punkt]
% scala> pt.isInstanceOf[Double]
% scala> pt.x.isInstanceOf[Punkt]
% scala> pt.x.isInstanceOf[Double]
% scala> pt.x.isInstanceOf[Int]
% \end{REPL}
%
% \SOLUTION
%
%
% \TaskSolved \what
%
%
% \SubtaskSolved
% ''val i: Int = pt.x'' error: type mismatch;
% Eftersom typen Int ej är kompatibel med ett värde av typen Double.
%
% ''val p: Double = new Punkt(5.0, 5.0)'' error: type mismatch;
% Eftersom typen Double ej är kompatibel med ett värde av typen Punkt.
%
% ''val p = new Punkt(5.0, 5.0): Double'' error: type mismatch;
% Eftersom typen Double ej är kompatibel med ett värde av typen Punkt.
%
% \SubtaskSolved
% Rad 3 till 7 i respektive ordning: true, false, false, true och false.
%
%
% \QUESTEND
%
%
%
%
% %%<AUTOEXTRACTED by mergesolu>%%      %Uppgift 6
%
%
%
%
% \WHAT{Topptypen \code{Any}.}
%
% \QUESTBEGIN
%
% \Task  \what~ Alla klasser är också av typen \code{Any}. Alla klasser får därmed med sig några gemensamma metoder som finns i den fördefinierade klassen \code{Any}, däribland metoderna  \code{isInstanceOf} och \code{toString}.  Vad blir resultatet av respektive rad nedan? Vilken rad ger ett felmeddelande?
%
%
% \begin{REPL}
% scala> class Punkt(val x: Double, val y: Double)
% scala> val pt: Punkt = new Punkt(1.0, 2.0)
% scala> pt.isInstanceOf[Punkt]
% scala> pt.isInstanceOf[Any]
% scala> pt.x.toString
% scala> println(pt.x)
% scala> val a: Any = pt
% scala> println(a.x)
% scala> a.toString
% scala> pt.y.toString
% scala> a.y.toString
% \end{REPL}
%
% \SOLUTION
%
%
% \TaskSolved \what
%
% \begin{enumerate}
% \item Definierar klassen Punkt.
% \item En variabel pt: Punkt skapas.
% \item true
% \item true
% \item String = 1.0
% \item skriver ut: 1.0
% \item En variabel med namnet a skapas med typen Any.
% \item error: value x is not a member of Any
% \item a ges nu typen String
% \item String = 2.0
% \item error: value y is not a member of Any
% \end{enumerate}
%
%
% \QUESTEND
%
%
%
%
% %%<AUTOEXTRACTED by mergesolu>%%      %Uppgift 7
%
%
%
%
% \WHAT{Byta ut metoden \code{toString}}.
%
% \QUESTBEGIN
%
% \Task  \what~ I klassen \code{Any} finns metoden \code{toString} som skapar en strängrepresentation av objektet. Du kan byta ut metoden \code{toString} i klassen \code{Any} mot din egen implementation. Man använder nyckelordet \code{override} när man vill byta ut en metodimplementation.
%
% \begin{REPL}
% scala> class Punkt(val x: Double, val y: Double) {
%          override def toString: String = "[x=" + x + ",y=" + y + "]"
%        }
% scala> val pt = new Punkt(1.0, 42.0)
% scala> pt.toString
% scala> println(pt)
% \end{REPL}
%
% \Subtask Vad händer egentligen på sista raden ovan?
%
% \Subtask Omdefiniera toString så att den ger en sträng på formen \code{Punkt(1.0, 42.0)}.
%
% \Subtask Vad händer om du utelämnar nyckelordet \code{override} vid omdefiniering?
%
% \SOLUTION
%
%
% \TaskSolved \what
%
%
% \SubtaskSolved
% ''println(pt)'' kallar på pt.toString, och eftersom metoden är överskriven kallas den nya version.
%
% \SubtaskSolved   \code{override def toString: String = ''Punkt('' + x + '', '' + y + '').''}
%
% \SubtaskSolved
% error: overriding method toString in class Object of type ()String;
%
%
% \QUESTEND
%
%
%
%
% %%<AUTOEXTRACTED by mergesolu>%%      %Uppgift 8
%
%
%
%
% \WHAT{Objektfabrik med \code{apply}-metod.}
%
% \QUESTBEGIN
%
% \Task  \what~  Man kan ordna så att man slipper skriva \code{new} med ett s.k. \emph{fabriksobjekt} \Eng{factory object}.
% \begin{Code}
% class Pt(val x: Double, y: Double) {
%   override def toString: String = "Pt(x=" + x + ",y=" + y + ")"
% }
% object Pt {
%   def apply(x: Double, y: Double): Pt = new Pt(x, y)
% }
% \end{Code}
%
% \Subtask Skriv satser som använder metoden \code{apply} i fabriksobjektet \code{object Pt} för att skapa flera olika punkter.
%
% \Subtask Ge applymetoden default-argument 0.0 för både x och y så att \code{Pt()} skapar en punkt i origo.
%
% \Subtask Skapa en klass \code{Rational} som representerar rationellt tal som en kvot mellan två heltal. Ge klassen två oföränderliga, publika klassparameterattribut med namnen \code{nom} för täljaren och \code{denom} för nämnaren.
%
% \Subtask Skapa ett fabriksobjekt med en \code{apply}-metod som tar två heltalsparametrar och skapar en instans av klassen \code{Rational}.
%
% \Subtask Skapa olika instanser av din klass \code{Rational} ovan med hjälp av fabriksobjektet.
%
%
% \SOLUTION
%
%
% \TaskSolved \what
%
%
% \SubtaskSolved
% \begin{REPL}
% scala> val pt = Pt(1.0, 2.0)
% pt: Pt = Pt(x=1.0,y=2.0)
%
% scala> Pt(4.0, 2.0)
% res0: Pt = Pt(x=4.0,y=2.0)
%
% scala> Pt(6.0, 3.0)
% res1: Pt = Pt(x=6.0,y=3.0)
%
% scala> Pt(666.0, 1337.0)
% res2: Pt = Pt(x=666.0,y=1337.0)
% \end{REPL}
%
% \SubtaskSolved  \code{def apply(): Pt = new Pt(0, 0)}
%
% \SubtaskSolved  \code{class Rational(val nom: Int, val denom: Int)}
%
% \SubtaskSolved
% \begin{REPLnonum}
% object Rational {
% def apply(nom: Int, denom: Int): Rational = new Rational(nom, denom)
% }
% \end{REPLnonum}
%
% \SubtaskSolved
% \begin{REPL}
% scala> Rational(2, 5)
% scala> Rational(2, 7)
% scala> Rational(7, 4)
% scala> Rational(666, 1337)
% \end{REPL}
%
%
% \QUESTEND
%
%
%
%
% %%<AUTOEXTRACTED by mergesolu>%%      %Uppgift 9
%
%
%
%
% \WHAT{Skapa en case-klass.}
%
% \QUESTBEGIN
%
% \Task  \what~  Med en case-klass får man \code{toString} och fabriksobjekt på köpet. Man behöver inte skriva \code{val} framför klassparametrar i case-klasser; klassparametrar blir publika, oföränderliga attribut automatiskt när man deklarerar en case-klass.
%
% \begin{REPL}
% scala> case class Pt(x: Double, y: Double)
% scala> val p = Pt(1.0, 42.0)
% scala> p.toString
% scala> println(p)
% scala> println(Pt(5,6))
% \end{REPL}
%
% \Subtask Implementera din klass \code{Rational} från föregående uppgift, men nu som en case-klass.
%
% \SOLUTION
%
%
% \TaskSolved \what
%
% \SubtaskSolved  \code{case class Rational(nom: Int, denom: Int)}
%
%
% \QUESTEND
%
%
%
%
% %%<AUTOEXTRACTED by mergesolu>%%      %Uppgift 10
%
%
%
%
% \WHAT{Metoder på datastrukturer.}
%
% \QUESTBEGIN
%
% \Task \label{task:point} \what~   En datastruktur blir mer användbar om det finns metoder som kan användas på datastrukturen. Metoder i Scala kan även ha (vissa) specialtecken som namn, t.ex. \code{+} enligt nedan.
% \begin{REPL}
% scala> case class Point(x: Double, y: Double) {
%          def distToOrigin: Double = math.hypot(x, y)
%          def add(p: Point): Point = Point(x + p.x, y + p.y)
%          def +(p: Point): Point = add(p)
%        }
% \end{REPL}
%
% \Subtask Använd metoden \code{distToOrigin} för att ta reda på vad punkten med koordinaterna (3, 4) har för avstånd till origo?
%
% \Subtask Skriv satser som skapar två punkter (3,4) och (5, 6) och låt variablerna p1 och p2 referera till respektive punkt. Låt variabeln p3 bli summan av p1 och p2 med hjälp av metoden \code{add}. Vad får uttrycken \code{p3.x} resp. \code{p3.y} för värden?
%
%
%
% \SOLUTION
%
%
% \TaskSolved \what
%
%
% \SubtaskSolved
% \begin{REPLnonum}
% scala> Point(3, 4).distToOrigin
% res0: Double = 5.0
% \end{REPLnonum}
%
% \SubtaskSolved
% p3.x = 8
% p3.y = 10
%
%
% \QUESTEND
%
%
%
%
% %%<AUTOEXTRACTED by mergesolu>%%      %Uppgift 11
%
%
%
%
% \WHAT{Operatornotation.}
%
% \QUESTBEGIN
%
% \Task  \what~  Vid punktnotation på formen: \\ \code{objekt.metod(argument)} \\ kan man skippa punkten och parenteserna och skriva:\\ \code{objekt metod argument}  \\
% Detta förenklade skrivsätt kallas \textbf{operatornotation}.
%
% \Subtask Använd klassen \code{Point} från uppgift \ref{task:point} och prova nedan satser. Vilka rader använder operatortnotation och vilka rader använder punktnotation? Vilka rader ger felmeddelande?
% \begin{REPL}
% scala> val p1 = Point(3,4)
% scala> val p2 = Point(3,4)
% scala> p1.add(p2)
% scala> p1 add p2
% scala> p1.+(p2)
% scala> p1 + p2
% scala> 42 + 1
% scala> 42.+(1)
% scala> 42.+ 1
% scala> 42 +(1)
% scala> 1.to(42)
% scala> 1 to 42
% scala> 1.to(42)
% \end{REPL}
%
% \Subtask Implementera metoderna \code{sub} och \code{-} i klassen \code{Point} och skriv uttryck som kombinerar add och sub, samt + och - i både punktnotation och operatornotation.
%
% \Subtask Operatornotation fungerar även med flera argument. Man använder då parenteser om listan med argumenten:
% \code{ objekt metod (arg1, arg2)}  \\
% Definiera en metod \\
% \code{def scale(a: Double, b: Double) = Point(x * a, y * b)} \\
% i klassen \code{Point} och skriv satser som använder metoden med punktnotation och operatornotation.
%
%
%
%
%
% \SOLUTION
%
%
% \TaskSolved \what
%
%
% \SubtaskSolved
% \\Operatornotation:	4, 6, 10, 12
% \\Punktnotation:		3, 5, 8, 9, 11, 13
% \\Felmeddelande:		9
%
% \SubtaskSolved
% \begin{Code}
% case class Point(x: Double, y: Double) {
%   def distToOrigin: Double = math.hypot(x, y)
%   def add(p: Point): Point = Point(x + p.x, y + p.y)
%   def +(p: Point): Point = add(p)
%   def sub(p: Point): Point = Point(x - p.x, y - p.y)
%   def -(p: Point): Point = sub(p)
% }
% \end{Code}
% \begin{REPL}
% scala> val p1: Point = Point(1, 9)
% scala> val p2: Point = Point(9, 6)
% scala> p1.sub(p2)
% scala> p1.-(p2)
% scala> p2 sub p1
% scala> p2 - p2
% scala> p1.add(p2.sub(p1))
% scala> p1 + (p2 - p1)
% \end{REPL}
%
% \SubtaskSolved
% \begin{Code}
% case class Point(x: Double, y: Double) {
%   def distToOrigin: Double = math.hypot(x, y)
%   def add(p: Point): Point = Point(x + p.x, y + p.y)
%   def +(p: Point): Point = add(p)
%   def sub(p: Point): Point = Point(x - p.x, y - p.y)
%   def -(p: Point): Point = sub(p)
%   def scale(a: Double, b: Double) = Point(x * a, y * b)
% }
% \end{Code}
% \begin{REPL}
% scala> val p: Point(13,  37)
% scala> p.scale(4, 2)
% scala> p scale (3, 7)
% \end{REPL}
%
%
% \QUESTEND
%
%
%
%
% %%<AUTOEXTRACTED by mergesolu>%%      %Uppgift 12
%
%
%
%
% \WHAT{Föränderlighet och oföränderlighet.}
%
% \QUESTBEGIN
%
% \Task  \what~  Oföränderliga och föränderliga objekt beter sig olika vid tilldelning.
%
% \Subtask\Pen Innan du kör nedan kod: Försök lista ut vad som kommer att skrivas ut. Rita minnessituationen efter varje tilldelning.
%
% \begin{Code}
% println("\n--- Example 1: mutable value assigmnent")
% var x1 = 42
% var y1 = x1
% x1 = x1 + 42
% println(x1)
% println(y1)
% \end{Code}
%
% \Subtask\Pen Innan du kör nedan kod: Försök lista ut vad som kommer att skrivas ut. Rita minnessituationen efter varje tilldelning.
%
% \begin{Code}
% println("\n--- Example 2: mutable object reference assignment")
% class MutableInt(private var i: Int) {
%   def +(a: Int): MutableInt = { i = i + a; this }
%   override def toString: String = i.toString
% }
% var x2 = new MutableInt(42)
% var y2 = x2
% x2 = x2 + 42
% println(x2)
% println(y2)
% \end{Code}
%
% \Subtask\Pen Innan du kör nedan kod: Försök lista ut vad som kommer att skrivas ut. Rita minnessituationen efter varje tilldelning.
%
% \begin{Code}
% println("\n--- Example 3: immutable object reference assignment")
% class ImmutableInt(val i: Int) {
%   def +(a: Int): ImmutableInt = new ImmutableInt(i + a)
%   override def toString: String = i.toString
% }
% var x3 = new ImmutableInt(42)
% var y3 = x3
% x3 = x3 + 42
% println(x3)
% println(y3)
% \end{Code}
%
% \Subtask\Pen Vad finns det för fördelar med oföränderliga datastrukturer?
%
%
% \SOLUTION
%
%
% \TaskSolved \what
%
%
% \SubtaskSolved   \includegraphics[scale=0.5]{../img/w04-solutions/uppgift-13a}
%
% \SubtaskSolved
% \begin{enumerate}
% \item \includegraphics[scale=0.5]{../img/w04-solutions/uppgift-13b-1}
% \item \includegraphics[scale=0.5]{../img/w04-solutions/uppgift-13b-2}
% \item \includegraphics[scale=0.5]{../img/w04-solutions/uppgift-13b-3}
% \end{enumerate}
%
% \SubtaskSolved
% \begin{enumerate}
% \item \includegraphics[scale=0.5]{../img/w04-solutions/uppgift-13c-1}
% \item \includegraphics[scale=0.5]{../img/w04-solutions/uppgift-13c-2}
% \item \includegraphics[scale=0.5]{../img/w04-solutions/uppgift-13c-3}
% \end{enumerate}
%
% \SubtaskSolved   En stor fördel är att vi till exempel kan skicka med en immutable som argument till en metod och vara säkra på att metoden inte ändrar på värdet.
%
%
% \QUESTEND
%
%
%
%
% %%<AUTOEXTRACTED by mergesolu>%%      %Uppgift 13
%
%
%
%
% \WHAT{Några användbara samlingar.}
%
% \QUESTBEGIN
%
% \Task  \what~  En \textbf{samling} \Eng{collection} är en datastruktur som samlar många objekt av samma typ. I \code{scala.collection} och \code{java.util} finns många olika samlingar med en uppsjö användbara metoder. De olika samlingarna i \code{scala.collection} är ordnade i en gemensam hierarki med många gemensamma metoder; därför har man nytta av det man lär sig om metoderna i en Scala-samling när man använder en annan samling. Vi har redan tidigare sett samlingen \code{Vector}:
%
% \begin{REPL}
% scala> val tärningskast = Vector.fill(10000)((math.random() * 6 + 1).toInt)
% scala> tä   // tryck TAB
% scala> tärningskast.  // tryck TAB
% \end{REPL}
%
% \Subtask Ungefär hur många metoder finns det som man kan göra på objekt av typen \code{Vector}? Det är svårt att lära sig alla dessa på en gång, så vi väljer ut några få i kommande uppgifter.
%
% \Subtask Jämför överlappet mellan metoderna i \code{Vector} och \code{List} och uppskatta hur stor andel av metoderna som är gemensamma:
% \begin{REPL}
% scala> val myntkast =
%          List.fill(10000)(if (math.random() < 0.5) "krona" else "klave")
% scala> my   // tryck TAB
% scala> myntkast.  // tryck TAB
% \end{REPL}
%
% \SOLUTION
%
%
% \TaskSolved \what
%
%
% \SubtaskSolved   Ungefär 150 metoder.
%
% \SubtaskSolved   Ungefär lika många.
%
%
% \QUESTEND
%
%
%
%
% %%<AUTOEXTRACTED by mergesolu>%%      %Uppgift 14
%
%
%
%
% \WHAT{Typparameter.}
%
% \QUESTBEGIN
%
% \Task  \what~  Vissa funktioner är generella för många typer och tar en så kallad \textbf{typparameter} inom hakparenteser. Ofta slipper man skriva typparametrar, då kompilatorn kan härleda typen utifrån argumenten. Om man anger typparametrar explicit så hjälper kompilatorn dig med att kolla att det verkligen är rätt typ i samlingen.
%
% \Subtask Vad händer nedan?
% \begin{REPL}
% scala> var xs = Vector.empty[Int]
% scala> xs = xs :+ "42"
% scala> xs = xs :+ 43 :+ 64 :+ 46
% scala> xs
% scala> xs :+= "42".toInt
% scala> var ys = Vector[Int]("ett", "två", "tre")
% scala> var ingenting = Vector.empty
% scala> ingenting = Vector(1,2,3)
% \end{REPL}
%
% \Subtask Samlingar är mer användbara om de är \emph{generiska}, vilket innebär att elementens typ avgörs av en typparameter och därför kan vara av vilken typ som helst. Man kan definiera egna funktioner som tar generiska samlingar som parametrar. Förklara vad som händer här:
% \begin{REPL}
% scala> val vego = Vector("gurka", "tomat", "apelsin", "banan")
% scala> val prim = Vector(2, 3, 5, 7, 11, 13)
% scala> def först[T](xs: Vector[T]): T = xs.head
% scala> def sist[T](xs: Vector[T]) = xs.last
% scala> def förstOchSist[T](xs: Vector[T]): (T, T) = (xs.head, xs.last)
% scala> först(vego)
% scala> sist(prim)
% scala> förstOchSist(vego)
% scala> förstOchSist(prim)
% scala> def wrap[T](pair: (T, T))(xs: Vector[T]) = pair._1 +: xs :+ pair._2
% scala> wrap("Odla", "och ät!")(vego)
% scala> wrap("Odla", "och ät!")(vego).mkString(" ")
% \end{REPL}
%
%
%
%
%
% \SOLUTION
%
%
% \TaskSolved \what
%
%
% \SubtaskSolved
% \\1. Instansierar en tom vektor med element av typen int och tilldelar värdet till en variabel xs.
% \\2. Error eftersom \code{xs :+ ''42''} ger en Vector[Any] när Vector[Int] krävs.
% \\3. xs tilldelas ett nytt värde av Vector(43, 64, 46)
% \\4. xs skrivs ut.
% \\5. Lägger till talet 42 i xs.
% \\6. Error: type mismatch
% \\7. Skapar en tom Vector i variablen ingenting
% \\8. error: type mismatch; found: Int(3), required: Nothing
%
% \SubtaskSolved
% Tre metoder skapas: den första för att få första elementet i en lista, och eftersom den definieras med specialtypen T går den att använda med alla vektorer oavsett typen av variabeln i vektorn. Den andra får fram sista elementet och den sista hämtar båda två.
%
% En till function definieras längre ner med  namnet ''wrap'', som tar en lista och lägger till ett element längst fram och ett längst bak.
%
%
% \QUESTEND
%
%
%
%
% %%<AUTOEXTRACTED by mergesolu>%%      %Uppgift 15
%
%
%
%
% \WHAT{Några viktiga samlingsmetoder.}
%
% \QUESTBEGIN
%
% \Task  \what~  Deklarera följande vektorer i REPL.
% \begin{REPL}
% scala> val xs = (1 to 10).toVector
% scala> val a = Vector("abra", "ka", "dabra")
% scala> val b = Vector( "sim", "sala", "bim", "sala", "bim")
% scala> val stor = Vector.fill(100000)(math.random())
% \end{REPL}
% Undersök i REPL vad som händer nedan. Alla dessa metoder fungerar på alla samlingar som är indexerbara sekvenser. Givet deklarationerna ovan: vad har uttrycken nedan för värde och typ? Förklara vad som händer hälp av denna  översikt: \href{http://docs.scala-lang.org/overviews/collections/seqs}{docs.scala-lang.org/overviews/collections/seqs}
%
% \Subtask \code{a(1) + xs(1)}
%
% \Subtask \code{a apply 0}
%
% \Subtask \code{a.isDefinedAt(3)}
%
% \Subtask \code{a.isDefinedAt(100)}
%
% \Subtask \code{stor.length}
%
% \Subtask \code{stor.size}
%
% \Subtask \code{stor.min}
%
% \Subtask \code{stor.max}
%
% \Subtask \code{a indexOf "ka"}
%
% \Subtask \code{b.lastIndexOf("sala")}
%
% \Subtask \code{"först" +: b   //minnesregel: colon on the collection side}
%
% \Subtask \code{a :+ "sist"    //minnesregel: colon on the collection side}
%
% \Subtask \code{xs.updated(2,42)}
%
% \Subtask \code{a.padTo(10, "!")}
%
% \Subtask \code{b.sorted}
%
% \Subtask \code{b.reverse}
%
% \Subtask \code{a.startsWith(Vector("abra", "ka"))}
%
% \Subtask \code{"hejsan".endsWith("san")}
%
% \Subtask \code{b.distinct}
%
%
%
% \SOLUTION
%
%
% \TaskSolved \what
%
%
% \SubtaskSolved   String = ''ka2''
%
% \SubtaskSolved   String = ''abra''
%
% \SubtaskSolved   false
%
% \SubtaskSolved   false
%
% \SubtaskSolved   100000
%
% \SubtaskSolved   100000
%
% \SubtaskSolved   minsta talet i listan
%
% \SubtaskSolved   största talet i listan
%
% \SubtaskSolved   1
%
% \SubtaskSolved   3
%
% \SubtaskSolved   Vektor b fast med ''först'' som första element
%
% \SubtaskSolved   Vektor a fast med ''sist'' som sista element.
%
% \SubtaskSolved   plats 3 i vektorn xs får värdet 42
%
% \SubtaskSolved   En ny vektor fylld med ''!'' från och med plats 4 till 10. Men de andra värdena samma som i a.
%
% \SubtaskSolved   b sorterad i bokstavsordning
%
% \SubtaskSolved   b baklänges
%
% \SubtaskSolved   true
%
% \SubtaskSolved   true
%
% \SubtaskSolved   en vektor med alla unika element i b.
%
%
% \QUESTEND
%
%
%
%
% %%<AUTOEXTRACTED by mergesolu>%%      %Uppgift 16
%
%
%
%
% \WHAT{Några generella samlingsmetoder.}
%
% \QUESTBEGIN
%
% \Task  \what~  Det finns metoder som går att köra på \emph{alla} samlingar även om de inte är indexerbara. Givet deklarationerna i föregående uppgift: vad har uttrycken nedan för värde och typ? Förklara vad som händer med hjälp av dessa översikter: \\ \href{http://docs.scala-lang.org/overviews/collections/trait-traversable}{docs.scala-lang.org/overviews/collections/trait-traversable} \\ \href{http://docs.scala-lang.org/overviews/collections/trait-iterable}{docs.scala-lang.org/overviews/collections/trait-iterable}
%
% \Subtask \code{a ++ b}
%
% \Subtask \code{a ++ stor}
%
% \Subtask \code{val ys = xs.map(_ * 5)}
%
% \Subtask \code{b.toSet     // En mängd har inga dubletter}
%
% \Subtask \code{a.head + b.last}
%
% \Subtask \code{a.tail}
%
% \Subtask \code{a.head +: a.tail == a}
%
% \Subtask \code{Vector(a.head) ++ Vector(b.last)}
%
% \Subtask \code{a.take(1) ++ b.takeRight(1)}
%
% \Subtask \code{a.drop(2) ++ b.drop(1).dropRight(2)}
%
% \Subtask \code{a.drop(100)}
%
% \Subtask \code{val e = Vector.empty[String]; e.take(100)}
%
% \Subtask \code{Vector(e.isEmpty, e.nonEmpty)}
%
% \Subtask \code{a.contains("ka")}
%
% \Subtask \code{"ka" contains "a"}
%
% \Subtask \code{a.filter(s => s.contains("k")) }
%
% \Subtask \code{a.filter(_.contains("k")) }
%
% \Subtask \code{a.map(_.toUpperCase).filterNot(_.contains("K")) }
%
% \Subtask \code{xs.filter(x => x % 2 == 0)}
%
% \Subtask \code{xs.filter(_ % 2 == 0)}
%
%
% \SOLUTION
%
%
% \TaskSolved \what
%
%
% \SubtaskSolved
% Metoden ger tillbaka en ny Vector[String] som nu består av alla element i a plus alla element i b. I samma ordning med elementen i a först.
%
% \SubtaskSolved
% Samma som i uppgift a fast vektorn som returnas är av typen Vector[Any]. Det är eftersom Any är den närmsta typen som String och Double delar. Elementen från vektor a är fortfarande först och uppföljt av elementen i stor.
%
% \SubtaskSolved
% Variablen ys får värdet av en Vector[Int] som innehåller alla talen från xs fast multiplicerade med 5. Alltså ys = 5, 10, 15..., osv.
%
% \SubtaskSolved
% Functionen tar alla värden från en Vektor och sätter in i ett Set (mängd). Eftersom en mängd ej har dubletter så försvinner ett ''sala'' och ett ''bim'', Vector[String] som returneras blir därför (''sim'', ''sala'', ''bim'').
%
% \SubtaskSolved
% Metoden head ger första elementet i en samling, och last sista. Därför blir kombinationen av a.head och b.last en ny Vector[String] som består av a:s första element, och b:s första element.
%
% \SubtaskSolved
% Ger en Vector[String] som innehåller alla element efter det första. Alltså i detta fallet ''ka'' och ''dabra''.
%
% \SubtaskSolved
% True, eftersom head ger första elementet och tail ger resten, sedan sätter metoden +: ihop dem till en vektor med samma värden som a.
%
% \SubtaskSolved
% Eftersom ++ sätter ihop alla värden från två vektorer måste vi först omvandla från en sträng till vektor. Resultatet blir en ny vektor av samma typ som innan med a:s första element och b:S sista.
%
% \SubtaskSolved
% Samma resultat som i h, metoden take börjar från vänster och tar så många element som man skickar med som parameter och gör till en ny lista. Med 1 som parameter motsvarar det att göra Vector(a.head). Metoden takeRight gör samma sak fast från höger.
%
% \SubtaskSolved
% Metoden drop är motsvarigheten till take fast exkluderar de specifierade elementen istället för att inkludera dem i vektorn.
%
% \SubtaskSolved
% Eftersom a endast innehåller 3 element returnerar drop(100) en tom vektor.
%
% \SubtaskSolved
% Returnerar en tom vektor med element typen String
%
% \SubtaskSolved
% returnerar Vector(true, false)
%
% \SubtaskSolved
% True, metoden contains kollar om en samling innehåller ett specifikt element.
%
% \SubtaskSolved
% True. Eftersom en sträng även kan ses som Vector[Char].
%
% \SubtaskSolved
% Filtrerar vektorn a till att endast innehålla strängar som innehåller k.
%
% \SubtaskSolved
% Exakt samma som i p
%
% \SubtaskSolved
% map(\_.toUpperCase) omvandlar alla strängar i a till stora bokstäver
% filterNot(\_.contains(''K'')) tar resultatet vi precis fick och tar bort alla strängar som innehåller stora K.
%
% \SubtaskSolved
% filtrerar så att endast jämna tal finns kvar.
%
% \SubtaskSolved
% Exakt samma som i s
%
%
%
%
% \QUESTEND
%
%
%
%
% %%<AUTOEXTRACTED by mergesolu>%%      %Uppgift 17
%
%
%
%
% \WHAT{NEEDS A TOPIC DESCRIPTION}
%
% \QUESTBEGIN
%
% \Task  \what~ De olika samlingarna i \code{scala.collection} används flitigt i andra paket, exempelvis \code{scala.util} och \code{scala.io}.
%
% \Subtask Vad händer här? (Metoden \code{shuffle} skapar en ny samling med elementen i slumpvis ordning.)
% \begin{REPL}
% val xs = Vector(1,2,3)
% def blandat = scala.util.Random.shuffle(xs)
% def test = if (xs == blandat) "lika" else "olika"
% (for(i <- 1 to 100) yield test).count(_ == "lika")
% \end{REPL}
%
%
% \Subtask Skapa en textfil med namnet \code{fil.txt} som innehåller lite text och läs in den med: \\\code{scala.io.Source.fromFile("fil.txt", "UTF-8").getLines.toVector}
% \begin{REPL}
% > cat > fil.txt
% hejsan
% svejsan
% > scala
% scala> val xs = scala.io.Source.fromFile("fil.txt", "UTF-8").getLines.toVector
% scala> xs.foreach(println)
% \end{REPL}
%
%
% \Subtask Vad händer här? (Metoden \code{trim} på värden av typen \code{String} ger en ny sträng med blanktecken i början och slutet borttagna.)
% \begin{REPL}
% scala> val pgk =
%   scala.io.Source.fromURL("http://cs.lth.se/pgk/","UTF-8").getLines.toVector
% scala> pgk.foreach(println)
% scala> pgk.map(_.trim).
%          filterNot(_.startsWith("<")).
%          filterNot(_.isEmpty).
%          foreach(println)
% \end{REPL}
%
%
%
% \SOLUTION
%
%
% \TaskSolved \what
%
%
% \SubtaskSolved
% Vi instansierar en vektor xs med talen 1, 2 och 3.
% sedan definierar vi en metod blandat som ger oss en randomiserad version av xs.
% sedan definierar vi en till metod som testar om xs är lika med resultatet från blandat. Om det är så returnerar den strängen ''lika'' annars ''olika''.
% Sist kör vi en for-loop där vi 100 gånger kör testet, samtidigt räknas hur många gånger ''lika'' returneras.
%
% Vårt resultat är en siffra på hur många gånger xs var samma som en blandad version av sig själv, eftersom det finns 6 permutationer med 3 variabler så borde det vara ungefär 1/6 chans.
%
% \SubtaskSolved  -
%
% \SubtaskSolved
% \\ \code{map(\_.trim)} tar bort alla onödiga mellanrum i början och slutet på varje rad
% \\ \code{filterNot(\_.startsWith(''<''))} filtrerar bort alla rader som börjar med strängen ''<''
% \\ \code{filterNot(\_.isEmpty)} filtrerar bort alla tomma rader.
% \\ \code{foreach(println)} skriver ut alla rader.
%
%
% \QUESTEND
%
%
%
%
% %%<AUTOEXTRACTED by mergesolu>%%      %Uppgift 18
%
%
%
%
% \WHAT{Jämföra List och Vector.}
%
% \QUESTBEGIN
%
% \Task  \what~  En indexerbar sekvens av värden kallas vektor eller lista. I Scala finns flera klasser som kan kan indexeras, däribland klasserna \code{Vector} och \code{List}.
%
% \Subtask \emph{Likheter mellan \code{Vector} och \code{List}.} Kör nedan rader i REPL. Prova indexera i båda och studera hur stor andel av metoderna som är gemensamma.
% \begin{REPL}
% scala> val sv = Vector("en", "två", "tre", "fyra")
% scala> val en = List("one", "two", "three", "four")
% scala> sv(0) + sv(3)
% scala> en(0) + en(3)
% scala> sv. //tryck TAB
% scala> en. //tryck TAB
% \end{REPL}
%
% \Subtask \emph{Skillnader mellan \code{Vector} och \code{List}.} Klassen \code{Vector} i Scala har ''under huven'' en avancerad datastruktur i form av ett s.k. självbalanserande träd, vilket gör att \code{Vector} är snabbare än \code{List} på nästan allt, \emph{utom} att bearbeta elementen i \emph{början} av sekvensen; vill man lägga till och ta bort i början av en \code{List} så kan det ibland gå ungefär dubbelt så fort jämfört med \code{Vector}, medan alla andra operationer är lika snabba eller snabbare med \code{Vector}. Det finns ett fåtal speciella metoder, som bara finns i \code{List}, för att skapa en lista och lägga till i början av en lista. Vad händer nedan?
%
% \begin{REPL}
% scala> var xs = "one" :: "two" :: "three" :: "four" :: Nil
% scala> xs = "zero" :: xs
% scala> val ys = xs.reverse ::: xs
% \end{REPL}
%
%
% \SOLUTION
%
%
% \TaskSolved \what
%
%
% \SubtaskSolved
% I princip alla metoder delas, en lista har några fler t. ex. ''::'', '':::'', ''mapConserve'' osv.
%
% \SubtaskSolved
% Först skapas en lista med 4 sträng värden och instansierar variablen xs med det värdet.
% sedan skapar vi en ny lista, som består av ''zero'' + den gamla listan och ger värdet till xs.
% Sist instansierar vi en ny variabel ys, som får värdet av xs omvänd plus xs.
%
%
% \QUESTEND
%
%
%
%
% %%<AUTOEXTRACTED by mergesolu>%%      %Uppgift 19
%
%
%
%
% \WHAT{Mängd.}
%
% \QUESTBEGIN
%
% \Task  \what~  En mängd är en samling som garanterar att det inte finns några dubbletter. Det går dessutom väldigt snabbt, även i stora mängder, att kolla om ett element finns eller inte i mängden. Elementen i samlingen \code{Set} hamnar ibland, av effektivitetsskäl, i en förvånande ordning.
% \begin{REPL}
% scala> val s = Set("Malmö", "Stockholm", "Göteborg", "Köpenhamn", "Oslo")
% s: scala.collection.immutable.Set[String] =
%      Set(Oslo, Malmö, Köpenhamn, Stockholm, Göteborg)
%
% scala> val t = Set("Sverige", "Sverige", "Sverige", "Danmark", "Norge")
% t: scala.collection.immutable.Set[String] = Set(Sverige, Danmark, Norge)
% \end{REPL}
% Givet ovan deklarationer: vad blir värde och typ av nedan uttryck?
%
% \Subtask \code{s + "Malmö" == s}
%
% \Subtask \code{s ++ t}
%
% \Subtask \code{Set("Malmö", "Oslo").subsetOf(s)}
%
% \Subtask \code{s subsetOf Set("Malmö", "Oslo")}
%
% \Subtask \code{s contains "Lund"}
%
% \Subtask \code{s apply "Lund"}
%
% \Subtask \code{s("Malmö")}
%
% \Subtask \code{s - "Stockholm"}
%
% \Subtask \code{t - ("Norge", "Danmark", "Tyskland")}
%
% \Subtask \code{s -- t}
%
% \Subtask \code{s -- Set("Malmö", "Oslo")}
%
% \Subtask \code{Set(1,2,3) intersect Set(2,3,4)}
%
% \Subtask \code{Set(1,2,3) & Set(2,3,4)}
%
% \Subtask \code{Set(1,2,3) union Set(2,3,4)}
%
% \Subtask \code{Set(1,2,3) | Set(2,3,4)}
%
%
% \SOLUTION
%
%
% \TaskSolved \what
%
%
% \SubtaskSolved
% true, Boolean
%
% \SubtaskSolved
% En samling av alla värden i s och t, Set[String]
%
% \SubtaskSolved
% true, Boolean
%
% \SubtaskSolved
% false, Boolean
%
% \SubtaskSolved
% false, Boolean
%
% \SubtaskSolved
% false, Boolean
%
% \SubtaskSolved
% true, Boolean
%
% \SubtaskSolved
% Samlingen s utan elementet ''Stockholm'', Set[String]
%
% \SubtaskSolved
% Samlingen t utan elementen ''Norge'' och ''Danmark'', Set[String]
%
% \SubtaskSolved
% returnerar s, Set[String]
%
% \SubtaskSolved
% Samlingen s utan ''Malmö'' och ''Oslo'', Set[String]
%
% \SubtaskSolved
% Set(2, 3), Set[Int]
%
% \SubtaskSolved
% se deluppgift l
%
% \SubtaskSolved
% Set(1, 2, 3 ,4), Set[Int]
%
% \SubtaskSolved
% se deluppgift n
%
%
% \QUESTEND
%
%
%
%
% %%<AUTOEXTRACTED by mergesolu>%%      %Uppgift 20
%
%
%
%
% \WHAT{Slå upp värden från nycklar med \code{Map}.}
%
% \QUESTBEGIN
%
% \Task  \what~  Samlingen \code{Map} är mycket användbar. Med den kan man snabbt leta upp ett värde om man har en nyckel. Samlingen \code{Map} är en generalisering av en vektor, där man kan ''indexera'', inte bara med ett heltal, utan med vilken typ av värde som helst, t.ex. en sträng. Datastrukturen \code{Map} är en s.k. \emph{associativ array}\footnote{\href{https://en.wikipedia.org/wiki/Associative_array}{https://en.wikipedia.org/wiki/Associative\_array}}, implementerad som en s.k. \emph{hashtabell}\footnote{\href{https://en.wikipedia.org/wiki/Hash_table}{https://en.wikipedia.org/wiki/Hash\_table}}.
% \begin{REPL}
% scala> var huvudstad =
%   Map("Sverige" -> "Stockholm", "Norge" -> "Oslo", "Skåne" -> "Malmö")
% \end{REPL}
% Givet ovan variabel \code{huvudstad}, förklara vad som händer nedan?
%
% \Subtask \code{huvudstad apply "Skåne"}
%
% \Subtask \code{huvudstad("Sverige")}
%
% \Subtask \code{huvudstad.contains("Skåne")}
%
% \Subtask \code{huvudstad.contains("Malmö")}
%
% \Subtask \code{huvudstad += "Danmark" -> "Köpenhamn"}
%
% \Subtask \code{huvudstad.foreach(println)}
%
% \Subtask \code{huvudstad getOrElse ("Norge", "???") }
%
% \Subtask \code{huvudstad getOrElse ("Finland", "???") }
%
% \Subtask \code{huvudstad.keys.toVector.sorted}
%
% \Subtask \code{huvudstad.values.toVector.sorted}
%
% \Subtask \code{huvudstad - "Skåne"}
%
% \Subtask \code{huvudstad - "Jylland"}
%
% \Subtask \code{huvudstad = huvudstad.updated("Skåne","Lund") }
%
%
%
% \SOLUTION
%
%
% \TaskSolved \what
%
%
% \SubtaskSolved
% Returnerar strängen ''Malmö'' eftersom det värdet är indexerat på platsen ''Skåne''.
%
% \SubtaskSolved
% Returnerar strängen ''Stockholm'' eftersom det värdet är indexerat på platsen ''Sverige''.
%
% \SubtaskSolved
% true, eftersom huvudstad innehåller indexet ''Skåne''
%
% \SubtaskSolved
% false, eftersom huvudstad ej innehåller indexet ''Malmö''. Notera att det är index och inte värden vi
% kollar om det finns.
%
% \SubtaskSolved
% Lägger till indexet ''Danmark'' med värdet ''Köpenhamn'' i samlingen.
%
% \SubtaskSolved
% Skriver ut alla 2-tupler.
%
% \SubtaskSolved
% Returnerar ''Oslo'', Note: Om indexet ''Norge'' inte hade funnits hade ''???'' returnerats istället.
%
% \SubtaskSolved
% Returnerar ''???''
%
% \SubtaskSolved
% Returnerar en sorterar vektor med alla index.
%
% \SubtaskSolved
% Returnerar en sorterar vektor med alla värden.
%
% \SubtaskSolved
% Returnerar en ny mängd men med ''Skåne'' -> ''Malmö'' borttaget.
%
% \SubtaskSolved
% Returnerar huvudstad mängden eftersom det inte finns ett ''Jylland'' index att ta bort.
%
% \SubtaskSolved
% Uppdaterar indexet ''Skåne'' till att istället leda till värdet ''Lund''
%
%
% \QUESTEND
%
%
%
%
% %%<AUTOEXTRACTED by mergesolu>%%      %Uppgift 21
%
%
%
%
% \WHAT{Skapa Map från en samling.}
%
% \QUESTBEGIN
%
% \Task  \what~
%
% \Subtask Definiera denna vektor och undersök dess typ:
% \begin{Code}
% val pairs = Vector(
%   ("Björn", 46462229009L),
%   ("Maj", 46462221667L),
%   ("Gustav", 46462224906L))
% \end{Code}
%
% \Subtask Vad har variablen \code{telnr} nedan för typ: \\ \code{var telnr = pairs.toMap}
%
% \Subtask Använd \code{telnr} för att slå upp telefonnummer för Maj och Kim med hjälp av metoderna \code{apply} och \code{get}.
%
% \Subtask Använd metoden \code{getOrElse} vid upplagningar av \code{telnr} och ge \code{-1L} som telefonnummer i händelse av att ett nummer inte finns.
%
% \Subtask Lägg till \code{("Fröken Ur", 464690510L)} i \code{telnr}-mappen.
%
% \Subtask Skapa en \code{Vector[(String, String)]} enligt nedan, så att telefonnumret blir en sträng utan inledande landsnummer men med en nolla i riktnumret. Byt ut \code{???} mot lämpligt uttryck.
% \begin{REPL}
% scala> telnr.toVector.map(p => ???)
% res85: Vector[(String, String)] = Vector(("Björn", "0462229009"), ("Maj",
% "0462221667"), ("Gustav", "0462224906"), ("Fröken Ur", 04690510"))
%
% \end{REPL}
%
% \Subtask Använd vektorn i resultatet ovan för att skapa en ny \code{Map[String, String]} med nationella telefonnumer. Slå upp numret till Fröken Ur.
%
% \SOLUTION
%
%
% \TaskSolved \what
%
%
% \SubtaskSolved
% \begin{REPLnonum}
% pairs: scala.collection.immutable.Vector[(String, Long)] =
% 					Vector((Björn,444), (Maj,441), (Lucy,666))
% \end{REPLnonum}
%
% \SubtaskSolved
% Map[String, Long]
%
% \SubtaskSolved
% \begin{REPLnonum}
% scala> telnr(''Maj'')
% res0: Long = 441
%
% scala> telnr.get(''Maj'')
% res1: Option[Long] = Some(441)
%
% scala> telnr(''Kim'')
% java.util.NoSuchElementException: key not found: 'Kim
%   at scala.collection.MapLike$class.default(MapLike.scala:228)
%   at scala.collection.AbstractMap.default(Map.scala:59)
%   at scala.collection.MapLike$class.apply(MapLike.scala:141)
%   at scala.collection.AbstractMap.apply(Map.scala:59)
%   ... 32 elided
%
% scala> telnr.get(''Kim'')
% res2: Option[Long] = None
% \end{REPLnonum}
%
% \SubtaskSolved
% \begin{REPLnonum}
% scala> telnr.getOrElse(''Maj'', -1L)
% res0: Long = 441
%
% scala> telnr.getOrElse(''Kim'', -1L)
% res1: Long = -1
% \end{REPLnonum}
%
% \SubtaskSolved
% telnr += ''Fröken Ur'' -> 464690510L
%
% \SubtaskSolved
% telnr.toVector.map(p => p.\_1 -> (''0'' + p.\_2.toString.substring(2)))
%
% \SubtaskSolved
% Använd metoden toMap och apply.
%
%
%
%
% \QUESTEND
%
%
%
%
% %%<AUTOEXTRACTED by mergesolu>%%      %Uppgift 22
%
%
%
%
% \WHAT{Samlingsmetoden \code{maxBy}.}
%
% \QUESTBEGIN
%
% \Task  \what~  Med samlingsmetoden \code{maxBy} kan man själv definiera vad som ska maximeras. (Denna metod kommer du att behöva i veckans laboration.)
%
% \Subtask Förklara vad som händer nedan.
% \begin{REPL}
% scala> val xs = Vector((2,3), (1,5), (-1, 1), (7, 2))
% scala> xs.maxBy(x => x._1)
% scala> xs.maxBy(x => x._2)
% \end{REPL}
%
% \Subtask Om man bara använder en parameter i en anonym funktion, till exempel parametern \code{x} i lambdauttrycket \code{x => x + 1} \emph{en enda} gång, och kompilatorn kan gissa alla typer, kan man använda understreck som ''platshållare'' för att förkorta lambdauttrycket så här: \code{ _ + 1}
%
% Skriv uttrycken på raderna 2 och 3 i föregående deluppgift på ett kortare sätt med hjälp platshållarsyntax \Eng{place holder syntax}.
%
% \Subtask På motsvarande sätt kan man använda \code{minBy} för att välja vilken funktion som definierar minimum. Prova \code{minBy} på motsvarande sätt som i föregående deluppgifter.
%
% \SOLUTION
%
%
% \TaskSolved \what
%
%
% \SubtaskSolved   Metoden maxBy hämtar det element som är ''störst'', på rad två gör \code{x => x._1} att första värdet i tuplerna används för att bestämma vilken som är störst. Likt gör \code{x => x._2} på rad tre att istället det andra värdet används.
%
% \SubtaskSolved
% \begin{REPLnonum}
% scala> xs.maxBy(_._1)
% scala> xs.maxBy(_._2)
% \end{REPLnonum}
%
% \SubtaskSolved
% \begin{REPLnonum}
% scala> xs.minBy(_._1)
% scala> xs.minBy(_._2)
% \end{REPLnonum}
%
%
%
% \QUESTEND
%
%
%
%
%
%
%
%
% \WHAT{NEEDS A TOPIC DESCRIPTION}
%
% \QUESTBEGIN
%
% \Task  \what~ Skriv nedan program med en editor och kompilera från terminalen. Lägg till kod i huvudprogrammet som testar klassen \code{Account} och kompilera och kör. Utvidga sedan klassen \code{Account} med fler attribut och funktioner som du väljer själv.
%
% \begin{Code}
% class Account(val number: Long, val maxCredit: Int){
%   private var balance = 0
%
%   def deposit(amount: Int): Int = {
%     if (amount > 0) {balance += amount}
%     balance
%   }
%
%   def withdraw(amount: Int): (Int, Int) = if (amount > 0) {
%     val allowedWithdrawal =
%       if (amount < balance + maxCredit) amount
%       else balance + maxCredit
%     balance = balance - allowedWithdrawal
%     (allowedWithdrawal, balance)
%   } else (0, balance)
%
%   def show: Unit =
%     println("Account Nbr: " + number + " balance: " + balance)
% }
%
% object Main {
%   def main(args: Array[String]): Unit = {
%     ???
%   }
% }
% \end{Code}
%
%
%
% \SOLUTION
%
%
% \QUESTEND
%
%
%
%
%
%
% \WHAT{NEEDS A TOPIC DESCRIPTION}
%
% \QUESTBEGIN
%
% \Task \label{task:keno-set} \what~  Läs om reglerna för spelet Keno här: \\ \url{https://sv.wikipedia.org/wiki/Keno} och gör deluppgifterna nedan.
%
% \Subtask Skapa en klass \code{Keno} som kan användas för att genomföra en Kenodragning. Låt klassen ha ett privat attribut \code{balls} som är en föränderlig mängd med heltal och som från början är tom. Implementera lämpliga metoder i klassen för att användaren av klassen ska kunna dra nya slumpmässiga bollar som inte redan är dragna.
%
% \Subtask Skapa en \code{case class KenoBet(bet: Set[Int])} för att hålla reda vilka 11 bollar en viss person satsar på. Definiera en metod \\ \code{def numberOfHits(keno: Keno): Int = ???}\\ i case-klassen \code{KenoBet} som givet en kenodragning räknar ut hur många bollar som satsats rätt.
%
% \Subtask Skriv ett huvudprogram som simulerar en enkel Kenodragning. Låt två personer satsa på 11 slumpmässiga bollar, genomför en dragning av 20 bollar ur 70 möjliga och kontrollera sedan hur många bollar som personerna har prickat rätt.
%
%
%
%
%
% \SOLUTION
%
%
% \QUESTEND
%
%
%
%
%
%
% \WHAT{Dokumentationen för \code{Any}.}
%
% \QUESTBEGIN
%
% \Task  \what~  Undersök vilka metoder som finns i klassen Any här: \href{http://www.scala-lang.org/api/current/scala/Any.html}{http://www.scala-lang.org/api/current/scala/Any.html}. Prova några av metoderna i REPL.
%
% \SOLUTION
%
%
% \QUESTEND
%
%
%
%
%
%
% \WHAT{Dokumentationen för samlingar.}
%
% \QUESTBEGIN
%
% \Task  \what~  Leta upp metoden \code{tabulate} i dokumentationen för objektet \code{Vector} nästan längst ner i listan här: \\ \href{http://www.scala-lang.org/api/current/scala/collection/immutable/Vector.html}{http://www.scala-lang.org/api/current/scala/collection/immutable/Vector.html} \\Leta upp den variant av \code{tabulate} som har signaturen:\\ \code{def tabulate[A](n: Int)(f: (Int) => A): Vector[A] }\\ Klicka på den gråfyllda trekanten till vänster om signaturen som fäller ut beskrivningen
%
% \Subtask Förklara vad som händer här:
% \begin{REPLnonum}
% scala> Vector.tabulate(10)(i => i % 3)
% \end{REPLnonum}
%
% \Subtask Klicka på det blåa stora o-et överst på sidan, för att växla till klass-vyn och studera listan med alla metoder  i klassen \code{Vector}.
%
%
% \SOLUTION
%
%
% \QUESTEND
%
%
%
%
%
%
% \WHAT{Fler metoder på indexerbara sekvenser.}
%
% \QUESTBEGIN
%
% \Task  \what~  Deklarera följande vektorer i REPL.
% \begin{REPL}
% scala> val xs = (1 to 10).toVector
% scala> val a = Vector("abra", "ka", "dabra")
% scala> val b = Vector( "sim", "sala", "bim", "sala", "bim")
% \end{REPL}
% Undersök i REPL vad som händer nedan. Alla dessa metoder fungerar på alla samlingar som är indexerbara sekvenser. Vad har uttrycken för värde och typ? Förklara vad metoden gör. Studera även denna  översikt: \href{http://docs.scala-lang.org/overviews/collections/seqs}{docs.scala-lang.org/overviews/collections/seqs}
%
% \Subtask \code{b.indexWhere(s => s.startsWith("b"))}  % advanced
%
% \Subtask \code{a.indices}  % advanced
%
% \Subtask \code{xs.patch(1, Vector(42,43,44), 7)} % advanced
%
% \Subtask \code{xs.segmentLength(_ < 8, 2)} % advanced
%
% \Subtask \code{b.sortBy(_.reverse)}  % advanced
%
% \Subtask \code{b.sortWith((s1, s2) => s1.size < s2.size)} % advanced
%
% \Subtask \code{a.reverseMap(_.size)}	% advanced
%
% \Subtask \code{a intersect Vector("ka", "boom", "pow")} % advanced
%
% \Subtask \code{a diff Vector("ka")} % advanced
%
% \Subtask \code{a union Vector("ka", "boom", "pow")} % advanced
%
%
%
% \SOLUTION
%
%
% \QUESTEND
%
%
%
%
% \WHAT{NEEDS A TOPIC DESCRIPTION}
%
% \QUESTBEGIN
%
% \Task  \what~ För samlingen \code{List} finns en alternativ metod till \code{+:} som heter \code{::} och kallas ''cons'' och som i kombination med objektet \code{Nil} kan användas för att med alternativ syntax bygga listor. Läs om detta här: \\ \href{http://alvinalexander.com/scala/how-create-scala-list-range-fill-tabulate-constructors}{alvinalexander.com/scala/how-create-scala-list-range-fill-tabulate-constructors} \\ och hitta på några egna övningar för att undersöka hur cons och Nil fungerar. Metoder som slutar med kolon är högerassociativa. Läs mer om detta här: \href{http://www.artima.com/pins1ed/basic-types-and-operations.html#5.8}{http://www.artima.com/pins1ed/basic-types-and-operations.html\#5.8}\SOLUTION
%
%
% \QUESTEND


%!TEX encoding = UTF-8 Unicode
%!TEX root = ../exercises.tex

\ifPreSolution

\Exercise{\ExeWeekTEN}\label{exe:W10}

\begin{Goals}
\input{modules/w10-inheritance-exercise-goals.tex}
\end{Goals}

\begin{Preparations}
\item \StudyTheory{10}
\end{Preparations}

\BasicTasks

\else

\ExerciseSolution{\ExeWeekTEN}

\BasicTasks

\fi



\WHAT{Para ihop begrepp med beskrivning.}

\QUESTBEGIN

\Task \what

\vspace{1em}\noindent Koppla varje begrepp med den (förenklade) beskrivning som passar bäst:

\begin{ConceptConnections}
  bastyp & 1 & & A & har supertypen \code|AnyRef|, allokeras i heapen via referens \\ 
  supertyp & 2 & & B & kan ha många former, t.ex. en av flera subtyper \\ 
  subtyp & 3 & & C & klass utan namn, utvidgad med extra implementation \\ 
  körtidstyp & 4 & & D & en typ som är mer specifik \\ 
  dynamisk bindning & 5 & & E & kan ha parametrar, kan ej instansieras, kan ej mixas in \\ 
  polymorfism & 6 & & F & saknar implementation \\ 
  trait & 7 & & G & har supertypen \code|AnyVal|, lagras direkt på stacken \\ 
  inmixning & 8 & & H & tillföra egenskaper med \code|with| och en trait \\ 
  överskuggad medlem & 9 & & I & är abstrakt, kan mixas in, kan ha parametrar \\ 
  anonym klass & 10 & & J & kan vara mer specifik än den statiska typen \\ 
  skyddad medlem & 11 & & K & är endast synlig i subtyper \\ 
  abstrakt medlem & 12 & & L & körtidstypen avgör vilken metod som körs \\ 
  abstrakt klass & 13 & & M & medlem i subtyp ersätter medlem i supertyp \\ 
  förseglad typ & 14 & & N & den mest generella typen i en arvshierarki \\ 
  referenstyp & 15 & & O & en typ som är mer generell \\ 
  värdetyp & 16 & & P & subtypning utanför denna kodfil är förhindrad \\ 
\end{ConceptConnections}

\SOLUTION

\TaskSolved \what

\begin{ConceptConnections}
  bastyp & 1 & ~~\Large$\leadsto$~~ &  N & den mest generella typen i en arvshierarki \\ 
  supertyp & 2 & ~~\Large$\leadsto$~~ &  O & en typ som är mer generell \\ 
  subtyp & 3 & ~~\Large$\leadsto$~~ &  D & en typ som är mer specifik \\ 
  körtidstyp & 4 & ~~\Large$\leadsto$~~ &  J & kan vara mer specifik än den statiska typen \\ 
  dynamisk bindning & 5 & ~~\Large$\leadsto$~~ &  L & körtidstypen avgör vilken metod som körs \\ 
  polymorfism & 6 & ~~\Large$\leadsto$~~ &  B & kan ha många former, t.ex. en av flera subtyper \\ 
  trait & 7 & ~~\Large$\leadsto$~~ &  I & är abstrakt, kan mixas in, kan ha parametrar \\ 
  inmixning & 8 & ~~\Large$\leadsto$~~ &  H & tillföra egenskaper med \code|with| och en trait \\ 
  överskuggad medlem & 9 & ~~\Large$\leadsto$~~ &  M & medlem i subtyp ersätter medlem i supertyp \\ 
  anonym klass & 10 & ~~\Large$\leadsto$~~ &  C & klass utan namn, utvidgad med extra implementation \\ 
  skyddad medlem & 11 & ~~\Large$\leadsto$~~ &  K & är endast synlig i subtyper \\ 
  abstrakt medlem & 12 & ~~\Large$\leadsto$~~ &  F & saknar implementation \\ 
  abstrakt klass & 13 & ~~\Large$\leadsto$~~ &  E & kan ha parametrar, kan ej instansieras, kan ej mixas in \\ 
  förseglad typ & 14 & ~~\Large$\leadsto$~~ &  P & subtypning utanför denna kodfil är förhindrad \\ 
  referenstyp & 15 & ~~\Large$\leadsto$~~ &  A & har supertypen \code|AnyRef|, allokeras i heapen via referens \\ 
  värdetyp & 16 & ~~\Large$\leadsto$~~ &  G & har supertypen \code|AnyVal|, lagras direkt på stacken \\ 
\end{ConceptConnections}

\QUESTEND





\WHAT{Gemensam bastyp.}

\QUESTBEGIN

\Task  \what~  Man vill ofta lägga in objekt av olika typ i samma samling.
\begin{REPL}
scala> class Gurka(val vikt: Int)
scala> class Tomat(val vikt: Int)
scala> val gurkor = Vector(Gurka(100), Gurka(200))
scala> val grönsaker = Vector(Gurka(300), Tomat(42))
\end{REPL}

\Subtask Om en samling innehåller objekt av flera olika typer försöker kompilatorn härleda den mest specifika typen som objekten har gemensamt. Vad blir det för typ på värdet \code{grönsaker} ovan?

\Subtask Försök ta reda på summan av vikterna enligt nedan. Vad ger andra raden för felmeddelande? Varför?

\begin{REPL}
scala> gurkor.map(_.vikt).sum     // fungerar
scala> grönsaker.map(_.vikt).sum  // fungerar inte
\end{REPL}

\Subtask Du ska nu göra så att du kan komma åt vikten på alla grönsaker genom att ge gurkor och tomater en gemensam bastyp som de olika konkreta grönsakstyperna utvidgar med nyckelordet \code{extends}. Det heter att subtyperna \code{Gurka} och \code{Tomat} \textbf{ärver} egenskaperna hos supertypen \code{Grönsak}.

Skapa en bastyp \code{Grönsak} med ett abstrakt attribut \code{vikt}. Låt sedan de konkreta grönsakerna ärva bastypen:

\begin{REPL}
scala> trait Grönsak { val vikt: Int }
scala> class Gurka(val vikt: Int) extends Grönsak
scala> class Tomat(val vikt: Int) extends Grönsak
scala> val gurkor = Vector(Gurka(100), Gurka(200))
scala> val grönsaker = Vector(Gurka(300), Tomat(42))
\end{REPL}
När sker initialisering av attributet \code{vikt}?

\Subtask Vad blir det nu för typ på variabeln \code{grönsaker} ovan?

\Subtask Går det nu att summera av vikterna i \code{grönsaker} med uttrycket nedan? Varför?\\ \code{grönsaker.map(_.vikt).sum}


\Subtask En trait liknar en klass, men man kan inte instansiera den direkt. Vad blir det för felmeddelande om du försöker skapa en instans av en trait enligt nedan?
\begin{REPL}
scala> trait Grönsak { val vikt: Int }
scala> new Grönsak
\end{REPL}


\Subtask Traiten \code{Grönsak} har en abstrakt medlem \code{vikt}. Den sägs vara abstrakt eftersom den saknar definition -- medlemmen har bara ett namn och en typ men inget värde. Du kan instansiera den abstrakta traiten \code{Grönsak} om du fyller i det som ''fattas'', nämligen ett värde på \code{vikt}. Man kan fylla på det som fattas i genom att ''hänga på'' ett block efter typens namn vid instansiering. Man får då vad som kallas en \textbf{anonym klass}, i detta fall en ganska konstig grönsak som inte är någon speciell sorts grönsak med som ändå har en vikt.

Vad får \code{anonymGrönsak} nedan för typ och strängrepresenation?
\begin{REPL}
scala> val anonymGrönsak = new Grönsak { val vikt = 42 }
\end{REPL}

\Subtask Vad blir felmeddelandet om du skapar en anonym klass \code{Grönsak} med en kropp som saknar definition av vikt?

\SOLUTION


\TaskSolved \what


\SubtaskSolved  \code{Vector[Object]}. Typen \code{Object} i JVM är motsvarar typen \code{AnyRef} som är bastyp för alla referenstyper.

\SubtaskSolved  Felmeddelande:
\begin{REPLnonum}
scala> grönsaker.map(_.vikt).sum  
-- Error:                                                                                 
1 |grönsaker.map(_.vikt).sum
  |              ^^^^^^
  |             value vikt is not a member of Object - did you mean wait?
-- Error:
1 |grönsaker.map(_.vikt).sum
  |                         ^
  |ambiguous implicit arguments: both object DoubleIsFractional in object Numeric and object ShortIsIntegral in object Numeric match type Numeric[B] of parameter num of method sum in trait IterableOnceOps
\end{REPLnonum}
Det första felmeddelandet beror på att vektorns element är av typen \code{Object} och medlemmen \code{vikt} är inte definierat för denna typ. Det andra felmeddelandet är ett följdfel som beror på att en sekvens med element av typen \code{Object} inte kan summeras eftersom kompilatorn inte kan härleda att elementtypen är numerisk.

\SubtaskSolved  Attributet \code{vikt} initialiseras vid konstruktion av \code{Gurka} resp. \code{Tomat}. Värdet ges av resp. klassparameter.

\SubtaskSolved  \code{Vector[Grönsak]}.

\SubtaskSolved  Ja. Eftersom den statiska typen för elementen i sekvensen är \code{Grönsak} (den dynamiska typen kan vara godtycklig subtyp av \code{Grönsak}) och alla instanser av denna typ garanterat har attributet \code{vikt} som är av typen \code{Int} så kan kompilatorn vid \emph{kompileringstid} dra slutsatsen att summeringen är giltig och därmed kan kompilatorn kompilera koden till körbar maskinkod.

\SubtaskSolved  
\begin{REPLnonum}
scala> new Grönsak
-- Error:
1 |new Grönsak
  |    ^^^^^^^
  |    Grönsak is a trait; it cannot be instantiated
\end{REPLnonum}

\SubtaskSolved  
\begin{REPLnonum}
scala> val anonymGrönsak = new Grönsak { val vikt = 42 }
val anonymGrönsak: Grönsak = anon$1@1edde8b6
scala> anonymGrönsak.toString                                                                                      
val res0: String = anon$1@1edde8b6
\end{REPLnonum}
Typen är \code{Grönsak} och blir här en s.k. \emph{anonym klass}, eftersom vi inte har använt en namngiven klass med \code{extends}, utan bara ''hängt på'' en klasskropp inom klammerparenteser direkt vid konstruktion. När du skapar anonyma klasser måste du använda nyckelordet \code{new}.

Kompilatorn hittar på ett unikt klassnamn, här anon\$1, för att hålla reda på den anonyma klassen under kompilering till maskinkod. Strängrepresentationen innehåller ett hexadecimalt heltal som är unikt för instansen, här \code{1edde8b6}.

\SubtaskSolved  

\begin{REPLsmall}
scala> new Grönsak { }
-- Error:
1 |new Grönsak { }
  |^
  |object creation impossible, since val vikt: Int in trait Grönsak is not defined 

\end{REPLsmall}


\QUESTEND






\WHAT{Polymorfism vid arv, s.k. subtypspolymorfism.}

\QUESTBEGIN

\Task  \what~  Polymorfism betyder ''många skepnader''. I samband med arv  innebär det att flera subtyper, till exempel \code{Ko} och \code{Gris}, kan hanteras gemensamt som om de vore instanser av samma supertyp, så som \code{Djur}. Subklasser kan implementera en metod med samma namn på olika sätt. Vilken metod som exekveras bestäms vid körtid beroende på vilken subtyp som instansieras. På så sätt kan djur komma i många skepnader.

\Subtask Implementera funktionen \code{skapaDjur} nedan så att den returnerar antingen en ny \code{Ko} eller en ny \code{Gris} med lika sannolikhet.

\begin{REPL}
scala> trait Djur { def väsnas: Unit }
scala> class Ko   extends Djur { def väsnas = println("Muuuuuuu") }
scala> class Gris extends Djur { def väsnas = println("Nöffnöff") }
scala> def skapaDjur(): Djur = ???
scala> val bondgård = Vector.fill(42)(skapaDjur())
scala> bondgård.foreach(_.väsnas)
\end{REPL}

\Subtask Lägg till ett djur av typen Häst som väsnas på lämpligt sätt och modifiera \code{skapaDjur} så att det skapas kor, grisar och hästar med lika sannolikhet.


\SOLUTION


\TaskSolved \what


\SubtaskSolved
\begin{Code}
def skapaDjur(): Djur = 
  if math.random() > 0.5 then Ko() else Gris()
\end{Code}

\SubtaskSolved
\begin{Code}
class Häst extends Djur: 
  def väsnas = println("Gnääääägg") 

def skapaDjur(): Djur = 
   math.random() match
    case r if r < 0.33 => Ko() 
    case r if r < 0.67 => Gris() 
    case _             => Häst()
\end{Code}


\QUESTEND





\WHAT{Olika typer av heltalspar till laborationen \hyperref[section:lab:\LabWeekTEN]{\texttt{\LabWeekTEN}}.}


\QUESTBEGIN


\Task\label{exe:inheritance:labprep-pair}  \what~Under veckans laboration ska du använda olika typer av par som representerar riktning och position på en tvådimensionell spelplan, samt spelplanens storlek. I stället för att använda en vanlig 2-tupel till dessa tre olika typer av par ska du skapa egna, specifika  typer som alla ärver bastypen \code{Pair[T]}. Dessa typer ska alla ligga i filen \code{pairs.scala} i \code{package snake}.
\begin{Code}
// detta är en skiss på filen pairs.scala
package snake

trait Pair[T]:
  def x: T
  def y: T
  // uppgift a) lägg till den konkreta metoden tuple

// efterföljande deluppgifterna implementerar dessa subtyper till Pair:
//   case klass Dim beskriver en 2-dimensionell ytas storlek
//   case klass Pos beskriver en position på en yta av Dim storlek
//   enum Dir beskriver förflyttning mot North, South, East, West
\end{Code}
Skillnaden mellan \code{Pair[T]} och en vanlig 2-tupel är att medlemmarna \code{x} och \code{y} garanterat är av \emph{samma} typ, medan en 2-tupel kan innehålla element av olika typ.

I fig. \ref{snake:fig:pairs-uml} visas en bild av klasshierarkin som du steg-för-steg ska utveckla i efterföljande  uppgifter. Fördelen med att ha olika typer av par är att det är mer typsäkert \Eng{type safe}: vi får hjälp av kompilatorn att upptäcka om vi av misstag förväxlar t.ex. en position med en riktning.

\begin{figure}[H]
\begin{center}
\newcommand{\TextBox}[1]{\raisebox{0pt}[1em][0.5em]{#1}}
\tikzstyle{umlclass}=[rectangle, draw=black,  thick, anchor=north, text width=2cm, rectangle split, rectangle split parts = 3]
\begin{tikzpicture}[inner sep=0.5em,scale=1.2, every node/.style={transform shape}]

  \node [umlclass, rectangle split parts = 1, xshift=0cm, yshift=4.5cm] (BaseType1)  {
              \textit{\textbf{\centerline{\TextBox{\code{Pair[T]}}}}}
%              \nodepart[align=left]{second}\code{def x: T} \newline \code{def y: T}
          };


  \node [umlclass, rectangle split parts = 1, xshift=-3cm, yshift=2.5cm] (SubType1)  {
              \textit{\textbf{\centerline{\TextBox{\code{Dim}}}}}
%              \nodepart[align=left]{second}\code{val x: Int} \newline \code{val y: Int}
          };

\node [umlclass, rectangle split parts = 1, xshift=0cm, yshift=2.5cm] (SubType2)  {
            \textit{\textbf{\centerline{\TextBox{\code{Pos}}}}}
%            \nodepart[]{second}\TextBox{\code{val dim: Int}}
        };

\node [umlclass, rectangle split parts = 1, xshift=3cm, yshift=2.5cm] (SubType3)  {
            \textit{\textbf{\centerline{\TextBox{\code{Dir}}}}}
%            \nodepart[]{second}\TextBox{\code{val dim: Int}}
        };


\draw[umlarrow] (SubType1.north) -- ++(0,0.5) -| (BaseType1.south);
\draw[umlarrow] (SubType2.north) -- ++(0,0.5) -| (BaseType1.south);
\draw[umlarrow] (SubType3.north) -- ++(0,0.5) -| (BaseType1.south);

\end{tikzpicture}
\end{center}
\caption{Arvshierarki med \code{Pair[T]} som bastyp.}
\label{snake:fig:pairs-uml}
\end{figure}

\Subtask Öppna en editor och koda \code{trait Pair[T]} i en fil \code{pairs.scala}. Lägg dessutom till en konkret metod \code{tuple} i \code{Pair[T]} som returnerar en 2-tupel med de båda elementen i paret, så att det vid behov går att omvandla \code{Pair}-instanser till 2-tupler. Använd REPL via \code{sbt console} för att testa att detta fungerar:
\begin{REPLnonum}
scala> val p = new Pair[Int] { override val x = 10; override val y = 20 }
p: Pair[Int]{val x: Int; val y: Int} = $anon$1@784223e9

scala> p.tuple
val res0: (Int, Int) = (10,20)
\end{REPLnonum}

\Subtask Skapa en case-klass \code{Dim} som ärver \code{Pair[Int]}. Instanser av denna klass kommer du att använda under veckans laboration för att representera en spelplans storlek genom att låta \code{x} ange antalet horisontella positioner och \code{y} antalet vertikala positioner.

Lägg även till ett kompanjonsobjekt \code{Dim} med en \code{apply}-metod som kan skapa \code{Dim}-instanser givet en 2-tupel.
Testa i REPL enligt nedan.
\begin{REPLnonum}
scala> Dim(50, 60)
val res1: Dim = Dim(50,60)

scala> Dim((60, 50))
val res2: Dim = Dim(60,50)

scala> res2.tuple
val res3: (Int, Int) = (60,50)
\end{REPLnonum}

\Subtask Lägg till en case-klass \code{Pos} som ärver \code{Pair[Int]} som representerar en position med en \code{x}-koordinat och en \code{y}-koordinat, båda klassparametrar. Kordinaterna ska hållas inom en spelplansstorlek som ges av klassparametern \code{dim} av typen \code{Dim}. Kordinatpositionerna är heltal och räknas från \code{0} till (men inte med) \code{dim.x} resp. \code{dim.y}.

Gör primärkonstruktorn i case-klassen \code{Pos} \textbf{privat}, genom att skriva nyckelordet \code{private} efter klassnamnet men före klassparameterlistan, så att det inte går att skapa instanser via primärkonstruktorn utanför klasskroppen och kompanjonsobjektet. 

Implementera metoderna \code{+} och \code{-} i case-klassen \code{Pos}. Båda metoderna ska ta en parameter \code{p} av typen \code{Pair[Int]} och returnera en ny \code{Pos}, där \code{p.x} resp. \code{p.y} är adderat resp. subtraherat från aktuell position. Observera att du inte ska skriva \code{new} när du skapar en ny instans, eftersom dessa alltid ska skapas via kompanjonsobjektets \code{apply}-metod, som är en ''smart'' fabriksmetod som garanterar håller koordinaterna inom spelplanen. 

Lägg till ett kompanjonsobjekt \code{Pos} med en \code{apply}-metod som skapar en ny \code{Pos}-instans som ser till att koordinaterna alltid är inom \code{dim}. Aritmetiken ska ske modulo storleken \code{dim}, d.v.s en position ska aldrig kunna hamna utanför spelplanen; i stället så börjar man om på andra sidan (se exempel i REPL nedan). \\ \emph{Tips:} Använd  \code|java.lang.Math.floorMod| som hanterar negativa argument så att resultatet blir positivt (till skillnad från modulo-operatorn \%).

Lägg även till fabriksmetoden \code{random} som kan skapa nya slumpmässiga positioner inom \code{dim}. \emph{Tips:} Använd \code{scala.util.Random.nextInt}.

Testa att det fungerar enligt nedan:
\begin{REPLnonum}
scala> Pos(-1,20,Dim(10,20))
val res4: Pos = Pos(9,0,Dim(10,20))

scala> new Pos(-1,20,Dim(10,20))  // förbjuds med privat primärkonstruktor
-- Error:
1 |new Pos(-1,20,Dim(10,20))
  |    ^^^
  |constructor Pos cannot be accessed as a member of Pos

scala> Pos(0,0,Dim(5,5)) + Pos(6,12, Dim(5,5))                                                                     
val res5: Pos = Pos(1,2,Dim(5,5))

scala> Pos(0,0,Dim(5,5)) - Pos(1,2, Dim(5,5))                                                                     
val res6: Pos = Pos(4,3,Dim(5,5))

scala> for (_ <- 1 to 3) yield Pos.random(Dim(10,10))
val res7: IndexedSeq[Pos] = 
  Vector(Pos(8,8,Dim(10,10)), Pos(2,6,Dim(10,10)), Pos(3,7,Dim(10,10)))
\end{REPLnonum}

\Subtask Vad händer om du glömmer skriva \code{new} när du anropar den privata konstruktorn i din \code{apply}-metod? Varför finns inte detta problem i \code{apply}-metoden för \code{Dim}?

\Subtask Lägg till en \code{enum Dir} som ärver \code{Pair[Int]} och har två \code{val}-parametrar \code{x} och \code{y}. Lägg också till fyra fall med \code{case} som alla ärver \code{Dir} och som representerar en enstegsförflyttning i de fyra väderstrecken, genom att ge parametrarna \code{x} resp. \code{y} något av värden $1$, $-1$ eller $0$. Norrut ska anges med x-koordinaten $-1$ och y-koordinaten $0$, etc. Verifiera i REPL att enumerationen fungerar.

Lägg till en \code{export} som gör så att det räcker att importera \code{snake.*} för att få alla fyra riktningar synliga direkt (annars behövs även import av \code{Dir.*} på alla ställen där riktning används i och utanför paketet \code{snake})


\SOLUTION


\TaskSolved \what

\SubtaskSolved
\begin{CodeSmall}
trait Pair[T]:
  def x: T
  def y: T
  def tuple: (T, T) = (x, y)

\end{CodeSmall}

\SubtaskSolved
\begin{CodeSmall}
case class Dim(x: Int, y: Int) extends Pair[Int]
object Dim:
  def apply(dim: (Int, Int)): Dim = Dim(dim._1, dim._2)  
\end{CodeSmall}

\SubtaskSolved
\begin{CodeSmall}
case class Pos private (x: Int, y: Int, dim: Dim) extends Pair[Int]:
  def +(p: Pair[Int]): Pos = Pos(x + p.x, y + p.y, dim)
  def -(p: Pair[Int]): Pos = Pos(x - p.x, y - p.y, dim)

object Pos:
  def apply(x: Int, y: Int, dim: Dim): Pos = 
    import java.lang.Math.floorMod as mod
    new Pos(mod(x, dim.x), mod(y, dim.y), dim) //OBS: new nödvändig här!

  def random(dim: Dim): Pos = 
    import scala.util.Random.nextInt as rni
    Pos(rni(dim.x), rni(dim.y), dim)
\end{CodeSmall}

\SubtaskSolved Om du glömmer skriva \code{new} explicit i kompanjonsobjektets \code{apply}-metod så blir det ett rekursivt anrop som resulterar i en oändlig loop vid körtid. Med \code{new} så är det garanterat den privata primärkonstruktorn för \code{Pos} som anropas. 

I \code{Dim.apply} så skiljer sig parametertyperna åt mellan fabriksmetoden och primärkonstruktorn och kompilatorn väljer då primärkonstruktorn eftersom den passar med de givna två separata heltalen och inte med en 2-tupel.

\SubtaskSolved
\begin{CodeSmall}
enum Dir(val x: Int, val y: Int) extends Pair[Int]:
  case North extends Dir( 0, -1)
  case South extends Dir( 0,  1)
  case East  extends Dir( 1,  0)
  case West  extends Dir(-1,  0)
export Dir.*  // gör så att North etc blir synliga i paketet snake
\end{CodeSmall}

\QUESTEND






\WHAT{Supertyp med parameter.}

\QUESTBEGIN

\Task  \what~  Utbildningsdepartementet vill med sitt nya datasystem hålla koll på vissa personer och skapar därför en klasshierarki enligt nedan. Skriv in koden i en editor och testa i REPL med \code{sbt}.
\begin{Code}
class Person(val namn: String)

class Akademiker(
  namn: String,
  val universitet: String) extends Person(namn)

class Student(
  namn: String,
  universitet: String,
  program: String) extends Akademiker(namn, universitet)

class Forskare(
  namn: String,
  universitet: String,
  titel: String) extends Akademiker(namn, universitet)
\end{Code}


\Subtask Deklarera fyra olika \code{val}-variabler med lämpliga namn som refererar till olika instanser av alla olika klasser ovan och ge attributen valfria initialvärden genom olika parametrar till konstruktorerna.

\Subtask Skriv två satser: en som först stoppar in instanserna i en \code{Vector} och en som sedan loopar igenom vektorn och skriv ut alla instansers \code{toString} och \code{namn}.

\Subtask Utbildningsdepartementet vill att det inte ska gå att instansiera objekt av typerna \code{Person} och \code{Akademiker}. Det kan åstadkommas genom att placera nyckelordet \code{abstract} före \code{class}. Uppdatera koden i enlighet med detta. Vilket blir felmeddelande om man försöker instansiera en \code{abstract class}? Går det lika bra med en \code{trait}?

\Subtask Utbildningsdeparetementet vill slippa implementera \code{toString}. Gör därför om typerna \code{Student} och \code{Forskare} till case-klasser. \emph{Tips:} För att undkomma ett kompileringsfel (vilket?) behöver du använda \code{override val} på lämpligt ställe.
Skapa instanser av de nya case-klasserna \code{Student} och \code{Forskare} och skriv ut deras \code{toString}. 

\Subtask 
%Eftersom \code{Person} och \code{Akademiker} nu är abstrakta, vill utbildningsdepartementet att du gör om dessa typer till traits med abstrakta attribut istället för klasser. 
Använd abstrakta attribut i stället för parametrar för typerna som är abstrakta, så att du inte behöver skriva \code{override val} i klassparametrarna till de konkreta case-klasserna.
Du ska också införa en case-klass \code{IckeAkademiker} som ska användas i olika statistiska medborgarundersökningar.
Dessutom förser man alla personer med ett personnummer representerat som en \code{Long}.
Hur ser utbildningsdepartementets kod ut nu, efter alla ändringar? Skriv ett testprogram som skapar några instanser och skriver ut deras attribut.

\SOLUTION


\TaskSolved \what


\SubtaskSolved
\begin{Code}
val person = new Person("Person1")
val akademiker = new Akademiker("Person2", "LTH")
val student = new Student("Person3", "LTH", "D")
val forskare = new Forskare("Person4", "LTH", "Doktorand")
\end{Code}

\SubtaskSolved
\begin{Code}
val vec = Vector(person, akademiker, student, forskare)
for(i <- vec){ print(i.toString + i.namn) }
\end{Code}

\SubtaskSolved  
Felmeddelande vid instansiering av \code{abstract class Akademiker}:\\
\texttt{Akademiker is abstract; it cannot be instantiated}

Det går \emph{inte} lika bra med en \code{trait} i det speciella fallet \code{Akademiker}, eftersom en trait inte får skicka vidare parametrar till en supertyp. Felmeddelande:\\
\texttt{trait Akademiker may not call constructor of trait Person}
\begin{Code}
trait Person(val namn: String)

abstract class Akademiker(
  namn: String,
  val universitet: String) extends Person(namn)

class Student(
  namn: String,
  universitet: String,
  program: String) extends Akademiker(namn, universitet)

class Forskare(
  namn: String,
  universitet: String,
  titel: String) extends Akademiker(namn, universitet)
\end{Code}



\SubtaskSolved  
\begin{REPLnonum}
scala>  
     |trait Person(val namn: String)                                                                              
     | 
     | abstract class Akademiker(
     |   namn: String,
     |   val universitet: String) extends Person(namn)
     | 
     | case class Student(
     |   namn: String,
     |   universitet: String,
     |   program: String) extends Akademiker(namn, universitet)
     | 
     | case class Forskare(
     |   namn: String,
     |   universitet: String,
     |   titel: String) extends Akademiker(namn, universitet)
-- Error:     
8 |  namn: String,
  |  ^
  |  error overriding value namn in trait Person of type String;
  |    value namn of type String needs `override` modifier
-- Error:
9 |  universitet: String,
  |  ^
  |  error overriding value universitet in class Akademiker of type String;
  |    value universitet of type String needs `override` modifier
-- Error:
13 |  namn: String,
   |  ^
   |  error overriding value namn in trait Person of type String;
   |    value namn of type String needs `override` modifier
-- Error:
14 |  universitet: String,
   |  ^
   |  error overriding value universitet in class Akademiker of type String;
   |    value universitet of type String needs `override` modifier
\end{REPLnonum}

\begin{Code}
trait Person(val namn: String)

abstract class Akademiker(
  namn: String,
  val universitet: String) extends Person(namn)

case class Student(
  override val namn: String,
  override val universitet: String,
  program: String) extends Akademiker(namn, universitet)

case class Forskare(
  override val namn: String,
  override val universitet: String,
  titel: String) extends Akademiker(namn, universitet)
\end{Code}

\begin{REPLsmall}
scala> val ps = Vector(Student("Kim", "Lund", "D"), Forskare("Herz", "Lund", "Dr"))
val ps: Vector[Akademiker] = Vector(Student(Kim,Lund,D), Forskare(Herz,Lund,Dr))
scala> ps :+ new Person("Abstrakt") {}
val res0: Vector[Person] = 
  Vector(Student(Kim,Lund,D), Forskare(Herz,Lund,Dr), anon1@1941bbf3)
\end{REPLsmall}

\SubtaskSolved
\begin{Code}
trait Person: 
  val namn: String 
  val nbr: Long

trait Akademiker extends Person:
  val universitet: String

case class Student(
  namn: String,
  nbr: Long,
  universitet: String,
  program: String) extends Akademiker

case class Forskare(
  namn: String,
  nbr: Long,
  universitet: String,
  titel: String) extends Akademiker

case class IckeAkademiker(
    namn: String,
    nbr: Long) extends Person
\end{Code}



\QUESTEND




%\clearpage




\ExtraTasks %%%%%%%%%%%%%%%%%





%\WHAT{Uppräknade värden.}

%\QUESTBEGIN

% \Task  \what~  Ett sätt att hålla reda på uppräknade värden, t.ex. färgen på olika kort i en kortlek, är att använda olika heltal som får representera de olika värdena, till exempel så här:\footnote{Om namnkonventioner för konstanter i Scala: läs under rubriken ''Constants, Values, Variable and Methods'' här \href{http://docs.scala-lang.org/style/naming-conventions.html}{docs.scala-lang.org/style/naming-conventions.html}}
% \begin{Code}
% object Färg {
%   val Spader = 1
%   val Hjärter = 2
%   val Ruter = 3
%   val Klöver = 4
% }
% \end{Code}
% Dessa kan sedan användas som parametrar till denna case-klass vid skapande av kortobjekt:
% \begin{lstlisting}[language=,keywords={case,class}]
% case class Kort(färg: Int, valör: Int)
% \end{lstlisting}
% Man kan hålla reda på färgen med en variabel av typen \code{Int} och tilldela den en viss färg med ovan konstanter. Och när du skapar ett kort kan du använda färgnamnet och du slipper därmed att behöva komma ihåg vilket heltal som representerar färgen.
% \begin{REPL}
% scala> val f = Färg.Spader
% scala> import Färg._
% scala> Kort(Ruter, 7)
% \end{REPL}
% En annan fördelen med detta är att man lätt kan iterera över alla färger:
% \begin{REPL}
% scala> val kortlek = for (f <- 1 to 4; v <- 1 to 13) yield Kort(f, v)
% \end{REPL}
% Men den stora nackdelen med detta är att kompilatorn vid kompileringstid inte kollar om variablerna av misstag råkar ges något värde utanför det giltiga intervallet, eftersom alla heltal är möjliga. Detta får vi själv hålla koll på vid körtid, till exempel med funktionen \code{require} eller \code{if}-satser, etc.

% Istället kan man använda uppräknade värden med hjälp av case-objekt enligt nedan deluppgifter och därmed få hjälp av kompilatorn att hitta eventuella fel vid kompileringstid.  Ett case-objekt är som ett vanligt singelton-objekt men det får bl.a. automatiskt en \code{toString} som är samma som namnet. Case-objekt kan dessutom användas som värden i mönstermatchningar (mer om detta i kapitel \ref{chapter:W10}).

% \Subtask Deklarera följande uppräknade värden som singelton-objekt med gemensam bastyp. Med nyckelordet \code{sealed} så ''förseglas'' klassen och inga andra direkta subtyper tillåts förutom de som finns i samma kod-fil eller block. I detta exempel  med kortfärger vet vi ju att det inte finns fler än dessa fyra färger.
% \begin{Code}
% sealed trait Färg
% case object Spader extends Färg
% case object Hjärter extends Färg
% case object Ruter extends Färg
% case object Klöver extends Färg
% \end{Code}
% Dessa kan sedan användas som parametrar till denna case-klass vid skapande av kortobjekt:
% \begin{lstlisting}[language=,keywords={case,class}]
% case class Kort(färg: Färg, valör: Int)
% \end{lstlisting}
% Skapa därefter några exempelinstanser av klassen \code{Kort}. Vad är fördelen med ovan angreppssätt jämfört med att använda heltalskonstanter?

% \Subtask Om man vill kunna iterera över alla värden är det bra om de finns i en samling med alla värden. Vi kan lägga en sådan i ett kompanjonsobjekt till bastypen enligt nedan. Skriv ut alla färgvärden med en \code{for}-sats.

% \begin{Code}
% sealed trait Färg
% object Färg {
%   val values = Vector(Spader, Hjärter, Ruter, Klöver)
% }
% case object Spader extends Färg
% case object Hjärter extends Färg
% case object Ruter extends Färg
% case object Klöver extends Färg
% \end{Code}
% Skapa en kortlek med 52 olika kort och blanda den sedan med \code{Random.shuffle} enligt nedan. Använd en dubbel \code{for}-sats och \code{yield}.
% \begin{REPL}
% scala> val kortlek: Vector[Kort] = ???
% scala> val blandad = scala.util.Random.shuffle(kortlek)
% \end{REPL}

% \Subtask Skriv en funktion \code{ def blandadKortlek: Vector[Kort] = ???} som ger en ny blandad kortlek varje gång den anropas med metoden i föregående uppgift.

% \Subtask Om man även vill ha en heltalsrepresentation med en medlem \code{toInt} för alla värden, kan man ge bastypen en parameter och i stället för en trait (som inte kan ha några parametrar) använda en abstrakt klass.

% \begin{Code}
% sealed abstract class Färg(final val toInt: Int)
% object Färg {
%   val values = Vector(Spader, Hjärter, Ruter, Klöver)
% }
% case object Spader  extends Färg(0)
% case object Hjärter extends Färg(1)
% case object Ruter   extends Färg(2)
% case object Klöver  extends Färg(3)
% \end{Code}
% Skapa en funktion \code{färgPoäng} som räknar ut summan av heltalsrepresentationen av alla färger hos en vektor med kort, och använd den sedan för att räkna ut \code{färgPoäng} för de första fem korten.
% \begin{REPL}
% scala> def blandadKortlek: Vector[Kort] = ???
% scala> def färgPoäng(xs: Vector[Kort]): Int = ???
% scala> färgPoäng(blandadKortlek.take(5))
% \end{REPL}


% \SOLUTION

% \TaskSolved \what

% \SubtaskSolved  Sättet är säkrare då man inte kan tilldela korten en färg som inte finns. Med heltalskonstanterna kan man till exempel ge ett kort färgen 5, vilken inte korresponderar till någon riktig färg.

% \SubtaskSolved  \code{for (f <- Färg.values; v <- 1 to 13) yield Kort(f,v)}

% \SubtaskSolved
% \begin{Code}
% def blandadKortlek: Vector[Kort] = {
%   val kortlek =
%     for (f <- Färg.values; v <- 1 to 13) yield Kort(f,v)
%   scala.util.Random.shuffle(kortlek)
% }
% \end{Code}

% \SubtaskSolved  \code{def färgPoäng(xs: Vector[Kort]): Int = xs.map(_.färg.toInt).sum}

% \QUESTEND







\WHAT{Bastypen \code{Shape} och subtyperna \code{Rectangle} och \code{Circle}.}

\QUESTBEGIN

\Task  \what~  Du ska i denna uppgift skapa ett litet bibliotek för geometriska former med oföränderliga objekt implementerade med hjälp av case-klasser. De geometriska formerna har en gemensam bastyp kallad \code{Shape}. Utgå från koden nedan.

\begin{CodeSmall}
case class Point(x: Double, y: Double):
  def move(dx: Double, dy: Double): Point = Point(x + dx, y + dy)

trait Shape:
  def pos: Point
  def move(dx: Double, dy: Double): Shape

case class Rectangle(pos: Point, width: Double, height: Double) extends Shape:
  def move(dx: Double, dy: Double): Rectangle = copy(pos = pos.move(dx, dy))

case class Circle(pos: Point, radius: Double) extends Shape:
  def move(dx: Double, dy: Double): Circle = copy(pos = pos.move(dx, dy))

\end{CodeSmall}

\Subtask Instansiera några cirklar och rektanglar och gör några relativa förflyttningar av dina instanser genom att anropa \code{move}.

\Subtask Lägg till en konkret metod \code{moveTo} i \code{Point} som gör en absolut förflyttning till koordinaterna \code{x} och \code{y}. Lägg till en abstrakt metod \code{moveTo} \code{Shape} som implementeras i subklasserna. Testa med REPL på några instanser av \code{Rectangle} och \code{Circle}.

\Subtask Lägg till metoden \code{distanceTo(that: Point): Double } i case-klassen \code{Point} som räknar ut avståndet till en annan punkt med hjälp av \code{math.hypot}. Klistra in i REPL och testa på några instanser av \code{Point}.

\Subtask Lägg till en konkret metod \code{distanceTo(that: Shape): Double } i traiten \code{Shape} som räknar ut avståndet till positionen för en annan Shape. Testa i REPL på några instanser av \code{Rectangle} och \code{Circle}.

\Subtask Gör så att \code{distanceTo} kan anropas med operatornotation.

\SOLUTION


\TaskSolved \what


\SubtaskSolved
\begin{CodeSmall}
val c1 = Circle(Point(1, 1), 42)
val r1 = Rectangle(Point(3, 3), 20, 30)
c1.move(2, 3)
r1.move(3, 2)
\end{CodeSmall}

\SubtaskSolved  
\begin{CodeSmall}
case class Point(x: Double, y: Double):
  def move(dx: Double, dy: Double): Point = Point(x + dx, y + dy)
  def moveTo(x: Double, y: Double): Point = Point(x, y)

trait Shape:
  def pos: Point
  def move(dx: Double, dy: Double): Shape
  def moveTo(x: Double, y: Double): Shape

case class Rectangle(pos: Point, width: Double, height: Double) extends Shape:
  def move(dx: Double, dy: Double): Shape = copy(pos = pos.move(dx, dy))
  def moveTo(x: Double, y: Double): Shape = copy(pos.moveTo(x, y))

case class Circle(pos: Point, radius: Double) extends Shape:
  def move(dx: Double, dy: Double): Shape = copy(pos = pos.move(dx, dy))
  def moveTo(x: Double, y: Double): Shape = copy(pos.moveTo(x, y))
\end{CodeSmall}


\SubtaskSolved \code{def distanceTo(that: Point): Double = math.hypot(that.x - x, that.y - y)}

\SubtaskSolved \code{def distanceTo(that: Shape): Double = pos.distanceTo(that.pos)}.

\SubtaskSolved  
\begin{CodeSmall}
case class Point(x: Double, y: Double):
  def move(dx: Double, dy: Double): Point = Point(x + dx, y + dy)
  def moveTo(x: Double, y: Double): Point = Point(x, y)
  infix def distanceTo(that: Point): Double = math.hypot(that.x - x, that.y - y)

trait Shape:
  def pos: Point
  def move(dx: Double, dy: Double): Shape
  def moveTo(x: Double, y: Double): Shape
  infix def distanceTo(that: Shape): Double = pos.distanceTo(that.pos)

case class Rectangle(pos: Point, width: Double, height: Double) extends Shape:
  def move(dx: Double, dy: Double): Shape = copy(pos = pos.move(dx, dy))
  def moveTo(x: Double, y: Double): Shape = copy(pos.moveTo(x, y))

case class Circle(pos: Point, radius: Double) extends Shape:
  def move(dx: Double, dy: Double): Shape = copy(pos = pos.move(dx, dy))
  def moveTo(x: Double, y: Double): Shape = copy(pos.moveTo(x, y))
\end{CodeSmall}

\QUESTEND






% \WHAT{Regler för \code{override}, \code{private} och \code{final}.}

% \QUESTBEGIN

% \Task  \what~

% \Subtask \label{subtask:overriderules} Undersök överskuggningning av abstrakta, konkreta, privata och finala medlemmar genom att skriva in raderna nedan en i taget i REPL. Vilka rader ger felmeddelande? Varför? Vid felmeddelande: notera hur felmeddelandet lyder och ändra deklarationen av den felande medlemmen så att koden blir kompilerbar (eller om det är enda rimliga lösningen: ta bort den felaktiga medlemmen), innan du provar efterkommande rad.

% \begin{REPL}
% trait Super1 { def a: Int; def b = 42; private def c = "hemlis" }
% class Sub2 extends Super1 { def a = 43; def b = 43; def c = 43 }
% class Sub3 extends Super1 { def a = 43; override def b = 43 }
% class Sub4 extends Super1 { def a = 43; override def c = "43" }
% trait Super5 { final def a: Int; final def b = 42 }
% class Sub6 extends Super5 { override def a = 43; def b = 43 }
% class Sub7 extends Super5 { def a = 43; override def b = 43 }
% class Sub8 extends Super5 { def a = 43; override def c = "43" }
% trait Super9 { val a: Int; val b = 42; lazy val c: String = "lazy" }
% class Sub10 extends Super9 { override def a = 43; override val b = 43 }
% class Sub11 extends Super9 { val a = 43; override lazy val b = 43 }
% class Sub12 extends Super9 { val a = 43; override var b = 43 }
% class Sub13 extends Super9 { val a = 43; override lazy val c = "still lazy" }
% class SubSub extends Sub13 { override val a = 44}
% trait Super14 { var a: Int; var b = 42; var c: String }
% class Sub15 extends Super14 { def a = 43; override var b = 43; val c = "?" }
% \end{REPL}

% \Subtask Skapa instanser av klasserna \code{Sub3}, \code{Sub13} och \code{SubSub} från ovan deluppgift och undersök alla medlemmarnas värden för respektive instans. Förklara varför de har dessa värden.

% %\Subtask Läs igenom reglerna i kapitel  \ref{slideW07:overriderules} om vad som gäller vid arv och överskuggning av medlemmar. Gör några egna undersökningar i REPL som försöker bryta mot någon regel som inte testades i deluppgift \ref{subtask:overriderules}.

% \SOLUTION


% \TaskSolved \what


% \SubtaskSolved  2. Måste ha \code{override} framför \code{b} för att kunna ändra på metoden. \\
% 4. \code{c} är \code{private}, vilket betyder att den är gömd för subklasserna. Därför kan den inte överskuggas. Genom att ta bort \code{override} fungerar klassen. \\
% 5. En \code{final}-medlem måste ha ett bestämt värde. Kan lösas genom att tilldela \code{final a} ett värde eller ta bort \code{final}. \\
% 6. En \code{final}-medlem kan inte överskuggas, varken med eller utan \code{override}. Här får konflikterna tas bort.  \\
% 7. Se 6. \\
% 8. Eftersom \code{c} inte finns i \code{Super5} kan den inte överskuggas. Genom att ta bort \code{override} fungerar klassen. \\
% 10. Överskuggningen av \code{val} måste vara oföränderlig (immutable); detta är inte nödvändigtvis \code{def}. Löses genom att byta ut \code{def a} mot \code{val a} hos \code{Sub10}.  \\
% 11. Samma problem som i 10.; \code{lazy val} kan vara föränderlig. Löses genom att ta bort \code{lazy}. \\
% 12. Samma problem igen! \code{var} är föränderlig, vilket bryter mot typsäkerheten när man försöker överskugga en \code{val}. Löses genom att ändra \code{var} till \code{val}. \\
% 15.\code{def a = 43} och \code{val c = "?"} täcker inte allt som \code{var} kräver. Det behövs en setter för att kunna uppfylla kraven för överskuggning för en \code{var}. Dessutom finns det ingen anledning för en \code{val} att överskuggas; man kan ju ändra på den lite hur man vill!

% \SubtaskSolved  Sub3: a = 43, b = 43 eftersom medlemmen är överskuggad. c hittas inte eftersom den är \code{private}.

% Sub13: a = 43, b = 42, c = "still lazy" eftersom medlemmen överskuggas.

% SubSub: a = 44 eftersom medlemmen överskuggas, b = 42, c = "still lazy".

% \SubtaskSolved  -.


% \QUESTEND





%\clearpage





\AdvancedTasks %%%%%%%%%%%%%%%%%

% \WHAT{Använda \code{trait} eller \code{class}?}

% \QUESTBEGIN

% \Task \what~ I vilka sammanhang är det nödvändigt att använda en \code{trait} respektive en \code{class}? Läs här för fördjupning:\\  \href{http://www.artima.com/pins1ed/traits.html\#12.7}{http://www.artima.com/pins1ed/traits.html\#12.7}.


% \SOLUTION


% \TaskSolved \what~Man måste använda en klass om man behöver klassparametrar. Man måste använda en trait om man vill göra in-mixning med \code{with}. \\

%  \QUESTEND



\WHAT{Inmixning.}

\QUESTBEGIN

\Task \label{task:fyle} \what~   Man kan utvidga en klass med multipla traits med en kommaseparerad lista. På så sätt kan man fördela medlemmar i olika traits och återanvända gemensamma koddelar genom så kallad \textbf{inmixning}, så som nedan exempel visar.

En alternativ fågeltaxonomi, speciellt populär i Skåne, delar in alla fåglar i två specifika kategorier: Kråga respektive Ånka. Krågor kan flyga men inte simma, medan Ånkor kan simma och oftast även flyga. Fågel i generell, kollektiv bemärkelse kallas på gammal skånska för Fyle.%
\footnote{\href{http://www.klangfix.se/ordlista.htm}{www.klangfix.se/ordlista.htm}}

\begin{CodeSmall}
trait Fyle:
  val läte: String
  def väsnas: Unit = print(läte * 2)
  val ärSimkunnig: Boolean
  val ärFlygkunnig: Boolean

trait KanSimma       { val ärSimkunnig = true }
trait KanInteSimma   { val ärSimkunnig = false }
trait KanFlyga       { val ärFlygkunnig = true }
trait KanKanskeFlyga { val ärFlygkunnig = math.random() < 0.8 }

class Kråga extends Fyle, KanFlyga, KanInteSimma:
  val läte = "krax"

class Ånka extends Fyle, KanSimma, KanKanskeFlyga: 
  val läte = "kvack"
  override def väsnas = print(läte * 4)
\end{CodeSmall}

\Subtask En flitig ornitolog hittar 42 fåglar i en perfekt skog där alla fågelsorter är lika sannolika, representerat av vektorn \code{fyle} nedan. Skriv i REPL ett uttryck som undersöker hur många av dessa som är flygkunniga Ånkor, genom att använda metoderna \code{filter}, \code{isInstanceOf}, \code{ärFlygkunnig} och \code{size}.

\begin{REPL}
scala> val fyle =
         Vector.fill(42)(if math.random() > 0.5 then new Kråga else new Ånka)
scala> fyle.foreach(_.väsnas)
scala> val antalFlygånkor: Int = ???
\end{REPL}

\Subtask \label{subtask:fyle:sound} Om alla de fåglar som ornitologen hittade skulle väsnas exakt en gång var, hur många krax och hur många kvack skulle då höras? Använd metoderna \code{filter} och \code{size}, samt predikatet \code{ärSimkunnig} för att beräkna antalet krax respektive kvack.
\begin{REPL}
scala> val antalKrax: Int = ???
scala> val antalKvack: Int = ???
\end{REPL}

\SOLUTION


\TaskSolved \what


\SubtaskSolved
Det finns många olika sätt, några exempellösningar:
\begin{Code}
val antalFlygånkor: Int = 
  fyle.count(f => f.isInstanceOf[Ånka] && f.ärFlygkunnig)
\end{Code}

\begin{Code}
val antalFlygånkor: Int = 
  fyle.filter(f => f.isInstanceOf[Ånka] && f.ärFlygkunnig).size
\end{Code}

\begin{Code}
val antalFlygånkor: Int = 
  fyle.collect{case f: Ånka if f.ärFlygkunnig}.size
\end{Code}

\begin{Code}
val antalFlygånkor: Int = fyle.map(_ match
  case f: Ånka if f.ärFlygkunnig => 1
  case _ => 0
).sum
\end{Code}

\SubtaskSolved
\begin{Code}
val antalKrax: Int = fyle.filter(f => !f.ärSimkunnig).size * 2
val antalKvack: Int = fyle.filter(f => f.ärSimkunnig).size * 4
\end{Code}


\QUESTEND











\WHAT{Finala klasser.}

\QUESTBEGIN

\Task  \what~  Om man vill förhindra att man kan göra \code{extends} på en klass kan man göra den final genom att placera nyckelordet \code{final} före nyckelordet \code{class}.

\Subtask Eftersom klassificeringen av fåglar i uppgiften ovan i antingen Ånkor eller Krågor är fullständig och det inte finns några subtyper till varken Ånkor eller Krågor är det lämpligt att göra dessa finala. Ändra detta i din kod.

\Subtask Testa att ändå försöka göra en subklass \code{Simkråga extends Kråga}. Vad ger kompilatorn för felmeddelande om man försöker utvidga en final klass?


\SOLUTION


\TaskSolved \what


\SubtaskSolved  Sätt \code{final} framför \code{class} i klasserna.

\SubtaskSolved  error: illegal inheritance from final class Kråga.


\QUESTEND






\WHAT{Accessregler vid arv och nyckelordet \code{protected}.}

\QUESTBEGIN

\Task  \what~  Om en medlem i en supertyp är privat så kan man inte komma åt den i en subklass. Ibland vill man att subklassen ska kunna komma åt en medlem även om den ska vara otillgänglig i annan kod.

\begin{Code}
trait Super:
  private val minHemlis = 42
  protected val vårHemlis = 42

class Sub extends Super:
  def avslöja = minHemlis
  def kryptisk = vårHemlis * math.Pi

\end{Code}

\Subtask Vad blir felmeddelandet när klassen \code{Sub} försöker komma åt \code{minHemlis}?

\Subtask Deklarera \code{Sub} på nytt, men nu utan den förbjudna metoden \code{avslöja}. Vad blir felmeddelandet om du försöker komma åt \code{vårHemlis} via en instans av klassen \code{Sub}? Prova till exempel med \code{(new Sub).vårHemlis}

\Subtask Fungerar det att anropa metoden \code{kryptisk} på instanser av klassen \code{Sub}?

\SOLUTION


\TaskSolved \what


\SubtaskSolved  
\begin{REPL}
2 |  def avslöja = minHemlis
  |                ^^^^^^^^^
  |                Not found: minHemlis
\end{REPL}

\SubtaskSolved  
\begin{REPL}
scala> class Sub extends Super:
         def kryptisk = vårHemlis * math.Pi
scala> (new Sub).vårHemlis
-- Error:
1 |(new Sub).vårHemlis
  |^^^^^^^^^^^^^^^^^^^
  |value vårHemlis in trait Super cannot be accessed as a member of Sub.
  | Access to protected value vårHemlis not permitted because enclosing object 
  | is not a subclass of trait Super where target is defined
\end{REPL}

\SubtaskSolved  Ja.


\QUESTEND






\WHAT{Använding av \code{protected}.}

\QUESTBEGIN

\Task  \what~  Den flitige ornitologen från uppgift \ref{task:fyle} ska ringmärka alla 42 fåglar hen hittat i skogen. När hen ändå håller på bestämmer hen att även försöka ta reda på hur mycket oväsen som skapas av respektive fågelsort. För detta ändamål apterar den flitige ornitologen en Linuxdator på allt infångat fyle. Du ska hjälpa ornitologen att skriva programmet.

\Subtask Inför en \code{protected var räknaLäte} i traiten \code{Fyle} och skriv kod på lämpliga ställen för att räkna hur många läten som respektive fågelinstans yttrar.

\Subtask Inför en metod \code{antalLäten} som returnerar antalet krax respektive kvack som en viss fågel yttrat sedan dess skapelse.

\Subtask Varför inte använda \code{private} i stället for \code{protected}?

\Subtask Varför är det bra att göra räknar-variabeln oåtkomlig från ''utsidan''?



\SOLUTION


\TaskSolved \what


\SubtaskSolved  I Fyle:
\begin{Code}
protected var räknaLäte: Int = 0
def väsnas: Unit = { print(läte * 2); räknaLäte += 2 }
\end{Code}

I Ånka: \code| override def väsnas = { print(läte * 4); räknaLäte += 4 }|

\SubtaskSolved  \code{ def antalLäten: Int = räknaLäte }

\SubtaskSolved  Om en klass som representerar en fågel som skulle ge ifrån sig fler/färre läten än en vanlig \code{Fyle}, behöver \code{väsnas} ändras. Denna metod behöver tillgång till \code{räknaLäte}, vilken inte får vara \code{private}.

\SubtaskSolved  Räknar-variabeln ska inte kunna påverkas i någon annan del av programmet.


\QUESTEND





\WHAT{Inmixning av egenskaper.}

\QUESTBEGIN

\Task  \what~ Det visar sig att vår flitige ornitolog från uppgift \ref{task:fyle} på sidan \pageref{task:fyle} sov på en av föreläsningarna i zoologi när hen var nolla på Natfak, och därför helt missat fylekategorin \code{Pjodd}. Hjälp vår stackars ornitolog så att fylehierarkin nu även omfattar Pjoddar. En Pjodd kan flyga som en Kråga men den \code{ÄrLiten} medan en Kråga \code{ÄrStor}. En Pjodd kvittrar dubbelt så många gånger som en Ånka kvackar. En Pjodd \code{KanKanskeSimma} där simkunnighetssannolikheten är $0.2$. Låt ornitologen ånyo finna 42 slumpmässiga fåglar i skogen och filtrera fram lämpliga arter. Undersök sedan hur dessa väsnas, i likhet med deluppgift \ref{task:fyle}\ref{subtask:fyle:sound}.


\SOLUTION

\TaskSolved \what


\begin{Code}
trait Fyle:
  val läte: String
  def väsnas: Unit = { print(läte * 2); räknaLäte += 2 }
  protected var räknaLäte: Int = 0
  val ärSimkunnig: Boolean
  val ärFlygkunnig: Boolean
  val ärStor : Boolean
  def antalLäten: Int = räknaLäte

trait KanSimma { val ärSimkunnig = true }
trait KanInteSimma { val ärSimkunnig = false }
trait KanFlyga { val ärFlygkunnig = true }
trait KanKanskeFlyga { val ärFlygkunnig = math.random() < 0.8 }
trait KanKanskeSimma { val ärSimkunnig = math.random() < 0.2 }
trait ÄrStor { val ärStor = true }
trait ÄrLiten { val ärStor = false }

final class Kråga extends Fyle, KanFlyga, KanInteSimma, ÄrStor:
  val läte = "krax"

final class Ånka extends Fyle, KanSimma, KanKanskeFlyga, ÄrStor:
  val läte = "kvack"
  override def väsnas = { print(läte * 4); räknaLäte += 4 }

final class Pjodd extends Fyle, KanFlyga, KanKanskeSimma, ÄrLiten:
  val läte = "kvitter"
  override def väsnas = { print(läte * 8); räknaLäte += 8 }
\end{Code}

I REPL:
\begin{REPL}
val fyle = Vector.fill(42)(
  if math.random() < 0.33 then Kråga()
  else if math.random() < 0.5 then Ånka()
  else Pjodd()
)
fyle.filter(f => f.isInstanceOf[Kråga]).size * 2
fyle.filter(f => f.isInstanceOf[Ånka]).size * 4
fyle.filter(f => f.isInstanceOf[Pjodd]).size * 8
\end{REPL}

\QUESTEND





% \WHAT{Typtest och typkonvertering.}

% \QUESTBEGIN

% \Task  \what~I Scala kan man testa körtidstyp och samtidigt konvertera till en mer specifik typ på ett typsäkert sätt med hjälp av \emph{mönstermatchning} i \code{match}-uttryck som vi ska se i kommande övning \texttt{\ExeWeekTEN}. För att underlätta interoperabilitet med Java finns  Scala-metoderna \code{isInstanceOf} och \code{asInstanceOf}, som motsvarar hur typtest och typkonvertering görs i Java.\footnote{\code{isInstanceOf} och \code{asInstanceOf} används sällan i Scala eftersom \code{match} är kraftfullare och säkrare.}

% Gör nedan deklarationer.
% \begin{REPL}
% scala> trait A; trait B extends A; class C extends B; class D extends B
% scala> val (c, d) = (new C, new D)
% scala> val a: A = c
% scala> val b: B = d
% \end{REPL}

% \Subtask Rita en bild över vilka typer som ärver vilka.

% \Subtask Vilket resultat ger dessa typtester? Varför?
% \begin{REPL}
% scala> c.isInstanceOf[C]
% scala> c.isInstanceOf[D]
% scala> d.isInstanceOf[B]
% scala> c.isInstanceOf[A]
% scala> b.isInstanceOf[A]
% scala> b.isInstanceOf[D]
% scala> a.isInstanceOf[B]
% scala> c.isInstanceOf[AnyRef]
% scala> c.isInstanceOf[Any]
% scala> c.isInstanceOf[AnyVal]
% scala> c.isInstanceOf[Object]
% scala> 42.isInstanceOf[Object]
% scala> 42.isInstanceOf[Any]
% \end{REPL}

% \Subtask Vilka av dessa typkonverteringar ger felmeddelande? Vilket och varför?
% \begin{REPL}
% scala> c.asInstanceOf[B]
% scala> c.asInstanceOf[A]
% scala> d.asInstanceOf[C]
% scala> a.asInstanceOf[B]
% scala> a.asInstanceOf[C]
% scala> a.asInstanceOf[D]
% scala> a.asInstanceOf[E]
% scala> b.asInstanceOf[A]
% \end{REPL}



% \SOLUTION


% \TaskSolved \what


% \SubtaskSolved  B ärver A. C och D ärver B.

% \SubtaskSolved  1. True eftersom c är av typen C. \\
% 2. False eftersom c inte är av typen D. \\
% 3. True eftersom d är av typen D som är en subtyp av B. \\
% 4. True eftersom c är av typen C som är en subtyp av B, som i sin tur är en subtyp av A. \\
% 5. True eftersom b är av typen D, som är en subtyp av B, som i sin tur är en subtyp av A. \\
% 6. True eftersom b är av typen D. \\
% 7. True eftersom a är av typen C som är en subtyp av B. \\
% 8. True eftersom c är av typen C som är en subtyp av AnyRef. \\
% 9. True eftersom c är av typen C som är en subtyp av Any. \\
% 10. Error eftersom \code{isInstanceOf} inte kan använda sig av \code{AnyVal}.  \\
% 11. True eftersom c är av typen C som är en subtyp av Object (Object är java-representationen av AnyRef). \\
% 12. Error eftersom \code{isInstanceOf} inte kan testa om värdetyper (i detta fallet \code{42}) är referenstyper. \\
% 13. True eftersom \code{42} är av typen \code{Int} som är en subtyp av Any. \\

% \SubtaskSolved  3. Går inte eftersom c inte är av typen D, utan typen C. \\
% 6. Går inte eftersom a inte är av typen D, utan typen C. \\
% 7. Går inte eftersom typen E inte finns. \\


% \QUESTEND













% \WHAT{Saknad referens med \texttt{null} och bottentypen \texttt{Nothing}.}

% \QUESTBEGIN

% \Task  \what~ Hitta på en egen fördjupningsuppgift inspirerat av denna artikel på Stackoverflow: \url{http://stackoverflow.com/questions/16173477/usages-of-null-nothing-unit-in-scala}

% \SOLUTION


% \QUESTEND






\WHAT{Arvshierarki med matematiska tal.}

\QUESTBEGIN

\Task  \what~ Studera den djupa arvshierarkin i paketet \code{numbers} i koden på efterföljande sidor. Paketet  \code{numbers} modellerar olika sorters tal i matematiken, med syftet att erbjuda ett s.k. DSL \footnote{\url{https://en.wikipedia.org/wiki/Domain-specific_language}}, alltså ett specialspråk för en viss applikationsdomän\footnote{\url{https://stackoverflow.com/questions/49216312/what-is-dsl-in-scala}}, här: domänen matematiska tal.

Du kan ladda ner koden för \code{numbers} här: \\
\href{https://github.com/lunduniversity/introprog/blob/master/compendium/examples/numbers.scala}{github.com/lunduniversity/introprog/blob/master/compendium/examples/numbers.scala}
\\ Notera speciellt metoden \code{reduce} som reducerar ett tal till sin enklaste form. Metoden \code{reduce} överskuggas på lämpliga ställen med relevant reduktion.

\Subtask Rita en bild över typhierarkin, t.ex. som ett upp-och-nedvänt träd med bastypen  \code{Number} som rot.

\Subtask Skriv kod som använder de olika konkreta klasserna i \code{package numbers}. 
\begin{REPL}
scala> numbers.  // Tryck Tab
AbstractComplex   AbstractNatural    AbstractReal   Frac    Nat      Polar
AbstractInteger   AbstractRational   Complex        Integ   Number   Real

scala> numbers.Integ(12)
res0: numbers.Integ = Integ(12)

scala> import numbers.Syntax._
import numbers.Syntax._

scala> 42.j
res1: numbers.Complex = Complex(Real(0),Real(42))

scala> 42.42.j
res2: numbers.Complex = Complex(Real(0),Real(42.42))

\end{REPL}

\Subtask Ändra på metoden \code{+} i \code{trait Number} så att den blir abstrakt och implementera den i alla konkreta klasser.

\Subtask Implementera fler räknesätt och bygg vidare på koden så som du finner intressant.

\Subtask Gör så att metoden \code{reduce} i klassen \code{AbstractRational} använder algoritmen Greatest Common Divisor (GCD)\footnote{\url{https://sv.wikipedia.org/wiki/St\%C3\%B6rsta\_gemensamma\_delare}} så som beskrivs här: \\ \href{http://www.artima.com/pins1ed/functional-objects.html#6.8}{www.artima.com/pins1ed/functional-objects.html\#6.8} \\ så att täljare och nämnare blir så små som möjligt.

%\clearpage

\scalainputlisting[numbers=left, basicstyle=\ttfamily\fontsize{9.1}{12.2}\selectfont]{examples/numbers.scala}\SOLUTION


\QUESTEND


%!TEX encoding = UTF-8 Unicode
%!TEX root = ../exercises.tex

\ifPreSolution


\Exercise{\ExeWeekELEVEN}\label{exe:W11}

\TODO övningar på given using, extensionsmetoder, typklasser, Ordering etc.
\TODO flytta ordering till hit

\begin{Goals}
\item \TODO
\end{Goals}

\begin{Preparations}
\item \StudyTheory{11}
\end{Preparations}

\BasicTasks %%%%%%%%%%%%%%%%

\else

\ExerciseSolution{\ExeWeekELEVEN}

\BasicTasks %%%%%%%%%%%

\fi


\WHAT{Användning av givna värden.}

\QUESTBEGIN

\Task  \what~  \TODO

\Subtask \TODO



\SOLUTION


\TaskSolved \what

\SubtaskSolved  \TODO


\QUESTEND








%!TEX encoding = UTF-8 Unicode
%!TEX root = ../exercises.tex

\ifPreSolution

\Exercise{\ExeWeekTWELVE}\label{exe:W12}

\begin{Goals}
\item Sökning och sortering: \\ \TODO flytta lämpliga saker om sökning och soprteringsordning till sekvensveckan
\begin{itemize}
\item Förstå hur sorteringsordningen är definierad för strängar.
\item Förstå skillnaderna mellan strängjämförelser i Scala och Java, samt kunna jämföra strängar med jämförelsoperatorer i Scala och med \code{compareTo} i Java.
\item Kunna sortera sekvenssamlingar innehållande objekt av grundtyper med hjälp av inbyggda och egendefinierade sorteringsordningar med metoderna \code{sorted}, \code{sortBy} och \code{sortWith}.
\item Kunna använda inbyggda linjärsöknings- och binärsökningsmetoder.
\item Kunna implementera en egen sökalgoritm med linjärsökning och binärsökning.
\item Förstå när binärsökning är lämplig och möjlig.
\item Kunna implementera en enkel sorteringsalgoritm, t.ex. insättningssortering eller urvalssortering, både till ny samling och på plats.
\item Känna till hur implicita sorteringsordningar används för grundtyperna och egendefinierade typer.
\item Känna till existensen av, funktionen hos, och relationen mellan klasserna \code{Ordering} och \code{Comparator}, samt  \code{Ordered} och \code{Comparable}.
\end{itemize}
\item Trådar och jämlöpande exekvering:
\begin{itemize}
\item Känna till vad en tråd är och kunna förklara begreppet jämlöpande exekvering.
\item Känna till vad metoderna \code{run} och \code{start} gör i klassen \code{Thread}.
\item Kunna skapa och starta en tråd med överskuggad \code{run}-metod.
\item Kunna skapa ett enkelt program som från två trådar tävlar om att uppdatera en variabel och förklara varför beteendet kan bli oförutsägbart.
\item Kunna använda en \code{Future} för att köra igång flera parallella beräkningar.
\item Kunna registrera en callback på en \code{Future} med metoden \code{onComplete}.
%\item Känna till att webbsidor beskrivs av HTML-kod och kunna skapa en minimal webbsida.
%\item Kunna ladda ner en webbsida med \code{scala.io.Source.fromURL}.
\end{itemize}
\end{Goals}

% \begin{Preparations}
% \item \StudyTheory{12}
% \end{Preparations}

%\BasicTasksNoLab %%%%%%%%%%%%%%%%

\subsection{Uppgifter om sökning och sortering}

\else

\ExerciseSolution{\ExeWeekTWELVE}

\subsection{Uppgifter om sökning och sortering}
%\BasicTasksNoLab

\fi





\WHAT{Jämföra strängar i Scala.}

\QUESTBEGIN

\Task \label{task:string-order-operators} \what~  I Scala kan strängar jämföras med operatorerna \code{==}, \code{!=}, \code{<}, \code{<=}, \code{>}, \code{>=},  där likhet/olikhet avgörs av om alla tecken i strängen är lika eller inte, medan större/mindre avgörs av sorteringsordningen i enlighet med varje teckens Unicode-värde.\footnote{Överkurs: Alla tecken i en \code{java.lang.String} representeras enligt UTF-16-standarden (\href{https://en.wikipedia.org/wiki/UTF-16}{https://en.wikipedia.org/wiki/UTF-16}), vilket innebär att varje Unicode-kodpunkt \Eng{code point} lagras som antingen ett eller två 16-bitars heltal. Strängjämförelse i Scala och Java jämför egentligen inte varje tecken, utan varje 16-bitars heltal. Denna skillnad har ingen betydelse när en sträng bara innehåller tecken som tar upp ett 16-bitars heltal var, och praktiskt nog är nästan alla tecken som används vardagligen av den typen. De flesta tecken som kräver två 16-bitars heltal är sällsynta kinesiska tecken, sällsynta symboler, tecken från utdöda språk och emoji. Vi kommer att bortse från sådana tecken i den här kursen.}

\Subtask Vad ger följande jämförelser för värde?
\begin{REPL}
scala> 'a' < 'b'
scala> "aaa" < "aaaa"
scala> "aaa" < "bbb"
scala> "AAA" < "aaa"
scala> "ÄÄÄ" < "ÖÖÖ"
scala> "ÅÅÅ" < "ÄÄÄ"
\end{REPL}
Tyvärr så följer ordningen av ÄÅÖ inte svenska regler, men det ignorerar vi i fortsättningen för enkelhets skull; om du är intresserad av hur man kan fixa  detta, gör uppgift \ref{task:swedish-letter-ordering}.

\Subtask\Pen Vilken av strängarna $s1$ och $s2$ kommer först (d.v.s. är ''mindre'') om $s1$ utgör början av $s2$ och $s2$ innehåller fler tecken än $s1$?


\SOLUTION


\TaskSolved \what


\SubtaskSolved
\begin{REPL}
true
true
true
true
true
false
\end{REPL}

\SubtaskSolved
\emph{s1} kommer först.


\QUESTEND





\WHAT{Jämföra strängar i Java. \TODO Flytta till Appendix Java}

\QUESTBEGIN

\Task  \what~  I Java kan man \textbf{inte} jämföra strängar med operatorerna \code{<}, \code{<=}, \code{>}, och \code{>=}. Dessutom ger operatorerna \code{==} och \code{!=} inte innehålls(o)likhet utan referens(o)likhet. Istället får man använda metoderna \code{equals} och \code{compareTo}, vilka också fungerar i Scala eftersom strängar i Scala och Java är av samma typ, nämligen \code{java.lang.String}.


\Subtask Vad ger följande uttryck för värde?

\begin{REPL}
scala> "hej".getClass.getTypeName
scala> "hej".equals("hej")
scala> "hej".compareTo("hej")
\end{REPL}


\Subtask Studera dokumentationen för metoden \code{compareTo} i \code{java.lang.String}\footnote{\href{https://docs.oracle.com/javase/8/docs/api/java/lang/String.html\#compareTo-java.lang.String-}{docs.oracle.com/javase/8/docs/api/java/lang/String.html\#compareTo-java.lang.String-}} och skriv minst 3 olika uttryck i Scala REPL som testar hur metoden fungerar i olika fall.

\Subtask Studera dokumentationen \code{compareToIgnoreCase} \footnote{\href{https://docs.oracle.com/javase/8/docs/api/java/lang/String.html\#compareToIgnoreCase-java.lang.String-}{docs.oracle.com/javase/8/docs/api/java/lang/String.html\#compareToIgnoreCase-java.lang.String-}} och skriv minst 3 olika stränguttryck i Scala REPL som testar hur metoden fungerar i olika fall.

\Subtask Vad skriver följande Java-program ut?
\javainputlisting{examples/StringEqTest.java}


\SOLUTION


\TaskSolved \what

\SubtaskSolved
\begin{REPL}
String = java.lang.String
Boolean = true
Int = 0
\end{REPL}

\SubtaskSolved
Exempel på 3 olika uttryck för att testa \code{compareTo}:

\begin{enumerate}
\item
Hej kommer först då \code{H < h}.
\begin{REPLnonum}
	"hej".compareTo("Hej")
	res: Int = 32
\end{REPLnonum}

\item
Dessa är ekvivalenta, så \code{compareTo} returnerar 0.
\begin{REPLnonum}
	"hej".compareTo("hej")
	res: Int = 0
\end{REPLnonum}

\item
\emph{h} kommer före \emph{ö}.
\begin{REPLnonum}
	"hej".compareTo("ö")
	res: Int = -142
\end{REPLnonum}
\end{enumerate}

\SubtaskSolved
Exempel på 3 olika uttryck för att testa \code{compareToIgnoreCase}:

\begin{enumerate}

\item
\begin{REPLnonum}
	"hej".compareToIgnoreCase("HEj")
	res: Int = 0
\end{REPLnonum}

\item
\begin{REPLnonum}
	"hej".compareToIgnoreCase("Ö")
	res: Int = -142
\end{REPLnonum}

\item
Samma som ovan, då Ö omvandlas till ö innan jämförelse.
 \begin{REPLnonum}
	"hej".compareToIgnoreCase("ö") \\ res: Int = -142
\end{REPLnonum}
\end{enumerate}

\SubtaskSolved
\begin{REPL}
false
true
0
\end{REPL}



\QUESTEND





\WHAT{Sortering med inbyggda sorteringsmetoder.}

\QUESTBEGIN

\Task  \what~  För grundtyperna (\code{Int}, \code{Double}, \code{String}, etc.) finns en fördefinierad ordning som gör så att färdiga sorteringsmetoder fungerar på alla samlingar i \code{scala.collection}. Även jämförelseoperatorerna i uppgift \ref{task:string-order-operators} fungerar enligt den fördefinierade ordningsdefinitionen för alla grundtyper. Denna ordningsdefinition är \textit{implicit tillgänglig} vilket betyder att kompilatorn hittar ordningsdefinitionen utan att vi explicit måste ange den. Detta fungerar i Scala även med primitiva \code{Array}.

\Subtask Testa metoden \code{sorted} på några olika samlingar. Förklara vad som händer. Hur lyder felmeddelandet på sista raden? Varför blir det fel?

\begin{REPL}
scala> Vector(1.1, 4.2, 2.4, 42.0, 9.9).sorted
scala> val xs = (100000 to 1 by -1).toArray
scala> xs.sorted
scala> xs.map(_.toString).sorted
scala> xs.map(_.toByte).sorted.distinct
scala> case class Person(firstName: String, familyName: String)
scala> val ps = Vector(Person("Robin", "Finkodare"), Person("Kim","Fulhack"))
scala> ps.sorted
\end{REPL}

\Subtask Om man har en samling med egendefinierade klasser eller man vill ha en annan sorteringsordning får man definiera ordningen själv. Ett helt generellt sätt att göra detta på  illustreras i uppgift \ref{task:custom-ordering}, men de båda hjälpmetoderna \code{sortWith} och \code{sortBy} räcker i de flesta fall. Hur de används illustreras nedan. Metoden \code{sortBy} kan användas om man tar fram ett värde av grundtyp och är nöjd med den inbyggda sorteringsordningen.

Metoden \code{sortWith} används om man vill skicka med ett eget jämförelsepredikat som ordnar två element; funktionen ska returnera \code{true} om det första elementet ska vara först, annars \code{false}.

\begin{REPL}
scala> case class Person(firstName: String, familyName: String)
scala> val ps = Vector(Person("Robin", "Finkodare"), Person("Kim","Fulhack"))
scala> ps.sortBy(_.firstName)
scala> ps.sortBy(_.familyName)
scala> ps.sortBy  // tryck TAB två gånger för att se signaturen
scala> ps.sortWith((p1, p2) => p1.firstName > p2.firstName)
scala> ps.sortWith  // tryck TAB två gånger för att se signaturen
scala> Vector(9,5,2,6,9).sortWith((x1, x2) => x1 % 2 > x2 % 2)
\end{REPL}
Vad har metoderna \code{sortWith} och \code{sortBy} för signaturer?

\Subtask Lägg till attributet \code{age: Int} i case-klassen \code{Person} ovan och lägg till fler personer med olika namn och ålder i en vektor och sortera den med \code{sortBy} och \code{sortWith} för olika attribut. Välj själv några olika sätt att sortera på.



\SOLUTION


\TaskSolved \what


\SubtaskSolved
\begin{enumerate}
\item Returnerar en sorterad \code{Vector} av \code{double}-värden
\item Skapar en variabel xs och sparar en \code{Array} med \code{Int}-värden mellan 100000 till 1.
\item Sorterar \code{xs = 1,2,3...}
\item Konverterar xs till en \code{Array} av \code{String}-värden och sorterar dem lexikografiskt: \code{xs = "1", "10", "100"} etc.
\item Konverterar xs till en \code{Array} av \code{Byte}-värden (max 127, min -128) och sorterar dem, samt tar bort dubletter: \code{xs = -128, -127, -1...}
\item Skapar en ny klass \code{Person} som tar 2 \code{String}-argument i konstruktorn
\item Sparar en Vector med två \code{Person}-objekt i en variabel ps
\item Försöker kalla på \code{sorted}-metoden för klassen \code{Person}. Eftersom vi skrivit denna klass själva och inte berättat för Scala hur \code{Person}-objekt ska sorteras, resulterar detta i ett felmeddelande.
\end{enumerate}

\SubtaskSolved

\begin{enumerate}
\item ---
\item ---
\item Sorterar \code{Person}-objekten i ps med avseende på värdet i \code{firstName}
\item Sorterar \code{Person}-objekten i ps med avseende på värdet i \code{familyName}
\item \code{sortBy} tar en funktion f som argument. f ska ta ett argument av typen \code{Person} och returnera en generisk typ B.
\item Sortera \code{Person}-objekten i ps med avseende på \code{firstName} i sjunkande ordning (omvänt från tidigare alltså)
\item \code{sortWith} tar en funktion lt som argument. lt ska i sin tur ta två argument av typen \code{Person} och returnera ett booleskt värde.
\item Sorterar en vektor så att värdena som är minst delbara med 2 hamnar först, och de mest delbara med 2 hamnar sist. Detta delar alltså upp udda och jämna tal.
\end{enumerate}

\SubtaskSolved
Klassens signatur blir då:
\begin{REPLnonum}
case class Person(firstName: String, lastName: String, age: Int)
\end{REPLnonum}

Lägg in dem i en vektor:
\begin{REPLnonum}
val ps2 = Vector(Person("a", "asson", 34), Person("asson", "assonson", 1234),
Person("anna", "Book", 2))
\end{REPLnonum}

Sortera dem på olika sätt:
\begin{enumerate}
\item
Vektorn blir sorterad med avseende på personernas ålder i stigande ordning
\begin{REPLnonum}
scala> ps2.sortWith((p1, p2) => p1.age < p2.age)
res40: scala.collection.immutable.Vector[Person] = Vector(Person(anna,Book,2),
Person(a,asson,34), Person(asson,assonson,1234))
\end{REPLnonum}

\item
Sorterar vektorn med avseende på namn, men också med avseende på ålder (i sjunkande ordning). För att komma före någon i ordningen måste alltså både namnet komma före, och åldern vara högre.
\begin{REPLnonum}
scala> ps2.sortWith((p1, p2) => (p1.firstName < p2.firstName) &&
(p1.age > p2.age))
res42: scala.collection.immutable.Vector[Person] = Vector(Person(a,asson,34),
Person(asson,assonson,1234), Person(anna,Book,2))
\end{REPLnonum}
\end{enumerate}



\QUESTEND






\WHAT{Tidmätning.}

\QUESTBEGIN

\Task \label{task:timed} \what~  I kommande uppgifter kommer du att ha nytta av funktionen \code{timed} enligt nedan.
\begin{Code}
def timed[T](code: => T): (T, Long) = {
  val now = System.nanoTime
  val result = code
  val elapsed = System.nanoTime - now
  println("\ntime: " + (elapsed / 1e6) + " ms")
  (result, elapsed)
}
\end{Code}
\Subtask Klistra in \code{timed} i REPL och testa så att den fungerar, genom att mäta hur lång tid nedan uttryck tar att exekvera.
\begin{REPL}
scala> val (v, t1) = timed{ (1 to 1000000).toVector.reverse }
scala> val (s, t2) = timed{ v.toSet }
scala> timed{ v.find(_ == 1) }
scala> timed{ s.find(_ == 1) }
scala> timed{ s.contains(1) }
\end{REPL}
\Subtask\Pen Försök förklara skillnaderna i exekveringstid mellan de olika sätten att söka reda på  talet $1$ i samlingen. Ungefär hur många gånger behöver man använda \code{contains} på heltalsmängden \code{s} för att det ska löna sig att skapa \code{s} i stället för att linjärsöka i \code{v} med \code{find} i ovan exempel?


\SOLUTION


\TaskSolved \what


\SubtaskSolved
Exekvera koden och du bör finna att det tar längre tid att hitta värdet 1 i vårt Set s än i vektorn v.

\SubtaskSolved

En vektor har en sekventiell ordning som find kan använda, medan \code{Set} är internt ordnad  på ett annat sätt för att innehållskontroll ska gå extra snabbt. Anledningen att det tar tid för \code{find} på \code{Set} är att det först måste skapas en iterator innan vår mängd kan gås igenom från början till slut. Metoden \code{contains} på \code{Set} däremot är rasande snabb beroende på den interna strukturen hos objekt av typen \code{Set} (som är smart designad med s.k. hash-koder, där det går lika snabbt att hitta ett element oavsett vart det befinner sig).



\QUESTEND




\WHAT{Sökning med inbyggda sökmetoder.}

\QUESTBEGIN

\Task  \what~

\Subtask \emph{Linjärsökning framifrån med \code{indexOfSlice}}. Studera dokumentationen för Scalas samlingsmetod \code{indexOfSlice}\footnote{\href{http://docs.scala-lang.org/overviews/collections/seqs.html}{docs.scala-lang.org/overviews/collections/seqs.html}} och skriv 8 olika uttryck i REPL som, både med en sträng och med en vektor med heltal, provar 4 olika fall: (1) finns i början, (2) finns någonstans i mitten, (3) finns i slutet, samt (4) finns ej.

\Subtask \emph{Linjärsökning bakifrån med \code{lastIndexOfSlice}}. Studera dokumentationen för Scalas samlingsmetod \code{lastIndexOfSlice}\footnote{\href{http://docs.scala-lang.org/overviews/collections/seqs.html}{docs.scala-lang.org/overviews/collections/seqs.html}} och skriv 8 olika uttryck i REPL som, både med en sträng och med en vektor med heltal, provar 4 olika fall: (1) finns i början, (2) finns någonstans i mitten, (3) finns i slutet, samt (4) finns ej.

\Subtask \emph{Sökning med inbyggd binärsökning.} Om en samling är sorterad kan man utnyttja detta för att göra snabbare sökning. Vid \textbf{binärsökning} \Eng{binary search}\footnote{\label{footnote:binarysearch}\href{https://en.wikipedia.org/wiki/Binary_search_algorithm}{en.wikipedia.org/wiki/Binary\_search\_algorithm}} börjar man på mitten och kollar vilken halva att  söka vidare i; sedan delar man upp denna halva på mitten och kollar vilken fjärdedel att söka vidare i, etc.

I objektet \code{scala.collection.Searching}\footnote{\href{http://www.scala-lang.org/api/current/scala/collection/Searching\$.html}{http://www.scala-lang.org/api/current/scala/collection/Searching\$.html}} finns en metod \code{search} som, om den importeras, erbjuder binärsökning för alla sorterade sekvenssamlingar. Om samlingen är sorterad ger den ett objekt av case-klassen \code{Found} som innehåller indexet för platsen där elementet först hittats; alternativt om det som eftersöks ej finns, ges ett objekt av case-klassen \code{InsertionPoint} som innehåller indexet där elementet borde ha varit placerad om det funnits i samlingen. Observera att om samlingen inte är sorterad är resultatet ''odefinierat'', d.v.s. något returneras men det är \emph{inte} att lita på; man måste alltså först sortera samlingen eller vara helt säker på att den är sorterad.

Undersök hur \code{search} fungerar genom att förklara vad som händer nedan. Vilken är snabbast av \code{lin} och \code{bin} nedan? Använd \code{timed} från uppgift \ref{task:timed}.

\begin{REPL}
scala> val udda = (1 to 1000000 by 2).toVector
scala> import scala.collection.Searching._
scala> udda.search(udda.last)
scala> udda.search(udda.last + 1)
scala> udda.reverse.search(udda(0))
scala> def lin(x: Int, xs: Seq[Int]) = xs.indexOf(x)
scala> def bin(x: Int, xs: Seq[Int]) = xs.search(x) match {
         case Found(i) => i
         case InsertionPoint(i) => -i
       }
scala> timed{ lin(udda.last, udda) }
scala> timed{ bin(udda.last, udda) }
\end{REPL}

\Subtask Om en samling innehåller $n$ element, hur många jämförelser behövs då vid binärsökning i värsta fall? \emph{Tips:} Läs om algoritmen på Wikipedia\textsuperscript{\ref{footnote:binarysearch}}.


\SOLUTION


\TaskSolved \what


\SubtaskSolved
Förslag på test av \code{indexOfSlice}:
\begin{REPLnonum}
scala> List(1,2,3,35,1,23).indexOfSlice(List(35,1,23))
res73: Int = 3
scala> List(1,2,3,35,1,23).indexOfSlice(List(35,1,3))
res74: Int = -1
\end{REPLnonum}

\SubtaskSolved
Förslag på test av \code{lastIndexOfSlice}:
\begin{REPLnonum}
Vector(1,2,3,4,1,2).lastIndexOfSlice(Vector(1,2))
res2: Int = 4
Vector("apa", "banan", "majs", "banan").lastIndexOfSlice(Vector("banan"))
res3: Int = 3
Vector("apa", "banan", "majs", "banan").lastIndexOfSlice(Vector("banand"))
res4: Int = -1
\end{REPLnonum}

\SubtaskSolved
Observera att metoden \code{search} antar att samlingen är sorterad i stigande ordning. När vi inverterar ordningen kan \code{search} oftast inte hitta det vi letar efter, eftersom den kommer leta i fel halva av samlingen.

\begin{REPLnonum}
scala> val udda = (1 to 1000000 by 2).toVector
scala> import scala.collection.Searching._
scala> udda.search(udda.last)
res18: collection.Searching.SearchResult = Found(499999)
//Search hittar det sista elementet på plats 499999 i samlingen.

scala> udda.search(udda.last + 1)
res19: collection.Searching.SearchResult = InsertionPoint(500000)
//Search kan inte hitta udda.last + 1 eftersom det inte existerar i samlingen
//och returnerar således ett objekt av typen InsertionPoint med värdet 500000.
//Vårt element udda.last + 1 hade alltså legat på plats 500000 om det funnits.

scala> udda.reverse.search(udda(0))
res20: collection.Searching.SearchResult = InsertionPoint(0)
//Som förklarat innan så förutsätter search att listan är sorterad i stigande
//ordning, så den kan inte hitta elementet udda(0) = 1 när listan är inverterad.

scala> def lin(x: Int, xs: Seq[Int]) = xs.indexOf(x)
scala> def bin(x: Int, xs: Seq[Int]) = xs.search(x) match {
	case Found(i) => i
	case InsertionPoint(i) => -i
}
//Definierar en metod bin som använder sig av metoden search på en sekvens.
//Den ser sedan till med hjälp av "pattern matching" att bara returnera positionen
//i, och inte ett objekt av typen Found eller InsertionPoint.

scala> timed{ lin(udda.last, udda) }
time: 42.294821 ms
res22: (Int, Long) = (499999,42294821)
//För att hitta udda.last = 499999 med linjärsökning tog det ca 42ms.

scala> timed{ bin(udda.last, udda) }
time: 0.147314 ms
res23: (Int, Long) = (499999,147314)
//Binärsökning för att hitta värdet 499999 tog extremt mycket kortare tid.
//Detta för att vid varje steg i binärsökningen halveras mängden tal som
//sökningen måste kolla i. Detta är dock ett extremfall eftersom vi söker
//talet längst bak i listan. Om vi istället gjort en linjärsökning efter
//det första talet 1, hade detta gått minst lika snabbt som binärsökning.
\end{REPLnonum}

\SubtaskSolved
Det behövs $log_2(n)$ jämförelser. Detta eftersom att vi hela tiden halverar antalet element i listan vi behöver söka igenom. Så efter första jämförelsen har vi $\frac{n}{2}$ element kvar. Efter andra jämförelsen har vi $\frac{n}{2*2}$ element kvar etc. När vi bara har ett element kvar har vi hittat det vi söker efter, och har då gjort $b$ antal jämförelser. Ekvationen ser då ut på följande vis:
\begin{equation*}
\frac{n}{2^b} = 1
\end{equation*}
Enligt lagarna för logaritmer kan vi nu komma fram till vad b är:
\begin{equation*}
log_2(n) = b
\end{equation*}

\QUESTEND




\WHAT{Sök bland LTH:s kurser med linjärsökning.}

\QUESTBEGIN

\Task \label{task:linsearch-lth}\what~

\Subtask Via denna URL kan du ladda ner en tab-separerad lista med alla kurser som ges på LTH under innevarande läsår: \url{http://cs.lth.se/pgk/kurser} \\Vilken data finns i filen? Du kan undersöka detta t.ex. med:
\begin{REPLnonum}
scala> import scala.io.Source.fromURL
scala> val url = "http://cs.lth.se/pgk/lthkurser"
scala> val data = fromURL(url,"UTF-8").getLines.mkString("\n")
\end{REPLnonum}

\Subtask \label{subtask:download-lthcourses} Klistra in objektet \code{courses} på sidan \pageref{lth-courses} med kommandot \code{:paste} i REPL.\footnote{Du kan ladda ner koden från: \\ \href{https://raw.githubusercontent.com/lunduniversity/introprog/master/compendium/examples/lth-courses/courses.scala}{github.com/lunduniversity/introprog/tree/master/compendium/examples/lth-courses/courses.scala}} Vad gör koden? Hur många kurser innehåller \code{courses.lth}?

\begin{figure}[h]
  \scalainputlisting[basicstyle=\ttfamily\fontsize{10.9}{14}\selectfont]{examples/lth-courses/courses.scala}
  \caption{Kod för att ladda ner data om alla kurser på LTH.}
  \label{lth-courses}
\end{figure}


\Subtask \emph{Linjärsökning med find.} Teknologen Oddput Clementina vill gå första bästa datavetenskapskurs som är på G2-nivå. Hjälp Oddput med att söka upp första förekommande kurs genom linjärsökning med samlingsmetoden \code{find}. Kurskoder vid datavetenskap börjar på EDA eller ETS\footnote{Detta är en förenklad bild av LTH:s kurskodnamnsystem. Några kurser från EIT-institutionen  kommer att slinka med, men det bortser vi ifrån i denna uppgift.}. \emph{Tips:} Du har nytta av att definiera predikatet \code{def isCS(s: String): Boolean} som i sin tur lämpligen nyttjar strängmetoden \code{startsWith}.

\Subtask \emph{Implementera linjärsökning.} Som träning ska du nu implementera en egen linjärsökningsfunktion med signaturen: \\ \code{def linearSearch[T](xs: Seq[T])(p: T => Boolean): Int = ???}
\\ Funktionen ska ta en sekvenssamling \code{xs} och ett predikat \code{p} som är en funktion som tar ett element och returnerar ett booleskt värde. Typen \code{Seq} är supertyp till alla sekvenssamlingar, så om vi använder den som parametertyp för parametern \code{xs} så fungerar funktionen för \code{Vector}, \code{Array}, \code{List}, etc. Genom typparametern \code{T} blir funktionen generisk och fungerar för godtycklig typ.
Funktionen \code{p} ska ge \code{true} om parametern är ett eftersökt element. Funktionen \code{linearSearch} ska returnera index för första hittade elementet i \code{xs} där \code{p} gäller. Om det inte finns något element som uppfyller predikatet ska -1 returneras. Skriv först pseudokod för funktionen med penna och papper. Du ska använda \code{while}.



\Subtask \label{subtask:linsearch-rndCode} Implementera en funktion \code{def rndCode: String} som genererar slumpmässiga kurskoder som består av 4 bokstäver mellan A och Z följt av 2 siffror mellan 0 och 9. \emph{Tips:} Använd REPL i kombination med en editor för att stegvis skapa och testa hjälpfunktioner som löser lämpliga delproblem.


\Subtask Använd \code{rndCode} från föregående deluppgift för att fylla en vektor kallad \code{xs} med en halv miljon slumpmässiga kurskoder. För varje slumpkod i \code{xs} sök med din funktion \code{linearSearch} efter index i vektorn \code{courses.lth2017} från deluppgift \ref{subtask:download-lthcourses}. Mät totala tiden för de $500000$ linjärsökningarna med hjälp av funktionen \code{timed} från uppgift \ref{task:timed}. Hur många av de slumpmässiga kurskoderna hittades bland de verkliga kurskoderna på LTH?



\Subtask Hur kan du implementera \code{linearSearch} med den inbyggda samlingsmetoden \code{indexWhere}?



\SOLUTION


\TaskSolved \what


\SubtaskSolved
Första raden innehåller kolumnnamnen \code{Kurskod KursSve KursEng Hskpoang Niva}. Därefter kommer en rad för varje kurs med kursdata enligt kolumnnamnen.

\SubtaskSolved
Koden laddar ner data och skapar en vektor med instanser av case-klassen \code{Course} med hjälp av metoden \code{fromLine}. Eftersom variabeln \code{lth} är deklarerad som \code{lazy} kommer inte \code{download()} bli anropad förrän första gången som variablen \code{lth} refereras. Antalet kurser ges av:
\begin{REPLnonum}
scala> val n = courses.lth.length
n: Int = 1104
\end{REPLnonum}

\SubtaskSolved
\begin{REPL}
scala> def isCS(s: String) = s.startsWith("EDA") || s.startsWith("ETS")
scala> val x = courses.lth.find(c => isCS(c.code) && c.level == "G2")
x: Option[courses.Course] = Some(Course(EDAF05,Algoritmer, datastrukturer och
   komplexitet,Algorithms, Data Structures and Complexity,5.0,G2))
\end{REPL}

\SubtaskSolved
\begin{Code}
def linearSearch[T](xs: Seq[T])(p: T => Boolean): Int = {
   var i = 0
   while(i < xs.length && !p(xs(i))) i += 1
   if (i < xs.length) i else -1
}
\end{Code}

\SubtaskSolved

\begin{Code}
def rndCode: String = {
   //randomizes from 0 to n (inclusive)
   def rnd(n: Int) = (math.random() * (n + 1)).toInt

   def letter = (rnd('Z' - 'A') + 'A').toChar
   def dig = ('0' + rnd(9)).toChar
   Seq(letter, letter, letter, letter, dig, dig).mkString
}
\end{Code}

\SubtaskSolved

\begin{Code}
val xs = Vector.fill(500000)(rndCode)
val(ixs, elapsedLin) =
  timed { xs.map(x => linearSearch(courses.lth)(_.code == x)) }
val found = ixs.filterNot(_== -1).size
\end{Code}

\SubtaskSolved

\begin{Code}
def linearSearch[T](xs: Seq[T])(p: T => Boolean): Int = xs.indexWhere(p)
\end{Code}



\QUESTEND

%%%%%% GAMLA VARIANTEN AV OVAN UPPGIFT
%%%%%% -- Funkar ej längre URL-api till LTH:S databas
% \WHAT{Sök bland LTH:s kurser med linjärsökning.}
%
% \QUESTBEGIN
%
% \Task \label{task:linsearch-lth} \what~ OBS! Använd \code{https} och \emph{inte} \code{http} i webbadresserna i denna och nästa uppgift, för att det ska fungera.
%
% \Subtask Surfa till denna URL:\\{%\nolinebreak[4]
% \footnotesize\url{https://kurser.lth.se/lot/?lasar=17_18&soek_text=&sort=kod&val=kurs&soek=t}}
% \\
% och inspektera HTML-koden i din webbläsare genom att trycka \emph{Ctrl+U} (fungerar i Firefox och Chrome). Rulla ner till rad 171 och framåt. Var finns antalet poäng för respektive kurs i HTML-koden?
%
% \Subtask \label{subtask:download-lthcourses} Klistra in objektet \code{courses} på sidan \pageref{lth-courses} med kommandot \code{:paste} i REPL.\footnote{Du kan ladda ner koden från: \\ \href{https://raw.githubusercontent.com/lunduniversity/introprog/master/compendium/examples/lth-courses/courses.scala}{github.com/lunduniversity/introprog/tree/master/compendium/examples/lth-courses/courses.scala}} Vad gör koden? Hur många kurser innehåller \code{lth2017}?
%
% \begin{figure}
%   \scalainputlisting[basicstyle=\ttfamily\fontsize{10.9}{14}\selectfont]{examples/lth-courses/courses.scala}
%   \caption{Kod för att söka bland kurser från LTH:s webbsida.}
%   \label{lth-courses}
% \end{figure}
%
%
% \Subtask \emph{Linjärsökning med find.} Teknologen Oddput Clementina vill gå första bästa datavetenskapskurs som är på G2-nivå. Hjälp Oddput med att söka upp första bästa kurs genom linjärsökning med samlingsmetoden \code{find}. Kurskoder vid datavetenskap börjar på EDA eller ETS\footnote{Detta är en förenklad bild av LTH:s kurskodnamnsystem. Några kurser från EIT-institutionen  kommer att slinka med, men det bortser vi ifrån i denna uppgift.}. \emph{Tips:} Du har nytta av att definiera predikatet \code{def isCS(s: String): Boolean} som i sin tur lämpligen nyttjar strängmetoden \code{startsWith}.
%
% \Subtask \emph{Implementera linjärsökning.} Som träning ska du nu implementera en egen linjärsökningsfunktion med signaturen: \\ \code{def linearSearch[T](xs: Seq[T])(p: T => Boolean): Int = ???}
% \\ Funktionen ska ta en sekvenssamling \code{xs} och ett predikat \code{p} som är en funktion som tar ett element och returnerar ett booleskt värde. Funktionen \code{p} ska ge \code{true} om parametern är ett eftersökt element. Funktionen \code{linearSearch} ska returnera index för första hittade elementet i \code{xs} där \code{p} gäller. Om det inte finns något element som uppfyller predikatet ska -1 returneras. Skriv först pseudokod för funktionen med penna och papper. Använd \code{while}.
%
% Typen \code{Seq} är supertyp till alla sekvenssamlingar, så om vi använder den som parametertyp för parametern \code{xs} så fungerar funktionen för \code{Vector}, \code{Array}, \code{List}, etc. Genom typparametern \code{T} blir funktionen generisk och fungerar för godtycklig typ.
%
%
%
% \Subtask \label{subtask:linsearch-rndCode} Implementera en funktion \code{def rndCode: String} som genererar slumpmässiga kurskoder som består av 4 bokstäver mellan A och Z följt av 2 siffror mellan 0 och 9. \emph{Tips:} Använd REPL  för att stegvis bygga upp hjälpfunktioner som du, när de fungerar som de ska, klistrar in i ett editorfönster som lokala funktioner där du utvecklar den slutliga koden för en lättläst, koncis och fungerande \code{rndCode}.
%
%
% \Subtask Använd \code{rndCode} från föregående deluppgift för att fylla en vektor kallad \code{xs} med en halv miljon slumpmässiga kurskoder. För varje slumpkod i \code{xs} sök med din funktion \code{linearSearch} efter index i vektorn \code{courses.lth2017} från deluppgift \ref{subtask:download-lthcourses}. Mät totala tiden för de $500000$ linjärsökningarna med hjälp av funktionen \code{timed} från uppgift \ref{task:timed}. Hur många av de slumpmässiga kurskoderna hittades bland de verkliga kurskoderna på LTH?
%
%
%
% \Subtask\Pen Hur kan du implementera \code{linearSearch} med den inbyggda samlingsmetoden \code{indexWhere}?
%
%
%
% \SOLUTION
%
%
% \TaskSolved \what
%
%
% \SubtaskSolved
% Den finns som värde för en \emph{td} tagg, på följande vis: \code{<td class="mitt">2</td>}.
%
% \SubtaskSolved
% Koden laddar ner html-koden för sidan \\ \mbox{\small\url{https://kurser.lth.se/lot/?lasar=17_18&soek_text=&sort=kod&val=kurs&soek=t}} och sparar den i en vektor. Sedan filtreras ut endast de rader som innehåller strängen ”kurskod” så att all onödig HTML-kod försvinner. Sedan konverteras detta, för varje rad, till \code{Course}-objekt med hjälp av metoden \code{fromHtml}. Eftersom variabeln \code{lth2017} är deklarerad som \code{lazy} kommer inte \code{download()} bli anropad förrän vi vill komma åt variabeln. Vi startar alltså processen genom att referera variabeln \code{lth2017} i objektet \code{courses}:
%
% \begin{REPLnonum}
% courses.lth2017
% \end{REPLnonum}
% Detta generarar en lång lista med \code{Course}-objekt. Antalet kurser är således lika med storleken på vektorn \code{lth2017}.
%
% \begin{REPLnonum}
% courses.lth2017.size
% res38: Int = 1101
% \end{REPLnonum}
%
% \SubtaskSolved
% \begin{REPL}
% scala> def isCS(s: String) = s.startsWith("EDA") || s.startsWith("ETS")
% scala> val x = courses.lth2017.find(c => isCS(c.code) && c.level == "G2").get
% x: courses.Course = Course(EDAF05,Algoritmer, datastrukturer och komplexitet,Algorithms, Data Structures and Complexity,5.0,G2)
% \end{REPL}
% Obs: metoden \code{find} returnerar ett objekt av typen \code{Option}. För att få värdet som är lagrat i detta objekt krävs det att man kallar på \code{get}.
%
% \SubtaskSolved
% \begin{Code}
% def linearSearch[T](xs: Seq[T])(p: T => Boolean): Int = {
%    var i = 0
%    while(i < xs.size && !p(xs(i))) i += 1
%    if (i < xs.size) i else -1
% }
% \end{Code}
%
% \SubtaskSolved
%
% \begin{Code}[language=Scala]
% def rndCode: String = {
%    //randomizes from 0 to n (inclusive)
%    def rnd(n: Int) = (math.random() * (n + 1)).toInt
%
%    def letter = (rnd('Z' - 'A') + 'A').toChar
%    def dig = ('0' + rnd(9)).toChar
%    Seq(letter, letter, letter, letter, dig, dig).mkString
% }
% \end{Code}
%
% \SubtaskSolved
%
% \begin{Code}
% val lthCourses = courses.lth2017 //avoid including download time
% val xs = Vector.fill(500000)(rndCode)
% val(ixs, elapsedLin) = timed{
% xs.map(x => linearSearch(lthCourses)(_.code == x))}
% val found = ixs.filterNot(_== -1).size
% \end{Code}
%
% \SubtaskSolved
%
% \begin{Code}
% def linearSearch[T](xs: Seq[T])(p: T => Boolean): Int =
%   xs.indexWhere(p)
% \end{Code}
%
%
%
% \QUESTEND






\WHAT{Sök bland LTH:s kurser med binärsökning.}

\QUESTBEGIN

\Task  \what~Sökalgoritmen BINSEARCH kan formuleras med nedan pseudokod:

\begin{algorithm}[H]
 \SetKwInOut{Input}{Indata}\SetKwInOut{Output}{Resultat}

 \Input{En växande sorterad sekvens $xs$ med $n$ heltal och \\ ett eftersökt heltal $key$}
 \Output{Ett heltal $i \geq 0$ som anger platsen där $x$ finns, eller ett negativt tal $i$ där $-i$ motsvarar platsen där $x$ ska sättas in i sorterad ordning om $x$ ej finns i samlingen.}
 sätt intervallet ($low$, $high$) till ($0$, $n - 1$) \\
 $found \leftarrow \bf{false}$ \\
 $mid \leftarrow -1$\\
 \While{$low \leq high$~$\bf{and}~\bf{not}$ $found$}{
   $mid \leftarrow $ platsen mitt emellan $low$ och $high$\\
   \eIf{$xs(mid)$ == $key$}{$found \leftarrow \bf{true}$}{
     \eIf{$xs(mid) < key$}{$low \leftarrow mid + 1$}{$high \leftarrow mid - 1$}
    }
 }
 \eIf{$found$}{\Return $mid$}{\Return $-(low + 1)$}
\end{algorithm}

\Subtask Prova algoritmen ovan med penna och papper på en sorterad sekvens med mindre än 10 heltal. Prova om algoritmen fungerar med ett jämnt antal tal, ett udda antal tal, en sekvens med ett heltal och en tom sekvens. Prova både om talet du letar efter finns och om det inte finns.

\Subtask Implementera binärsökning i en funktion med signaturen\\
\code{def binarySearch(xs: Seq[String], key: String): Int = ??? }\\
och testa i REPL för olika fall. Vad händer om sekvensen inte är sorterad?

\Subtask Använd \code{binarySearch} för att leta efter LTH-kurser enligt nedan. Använd \code{rndCode}, \code{timed} och \code{courses} från tidigare uppgifter.
\begin{Code}
def binarySearch(xs: Seq[String], key: String): Int = ???

val lthCodesSorted = courses.lth.map(_.code).sorted
val xs = Vector.fill(500000)(rndCode)
val (_, elapsedBin) =
  timed{xs.map(x => binarySearch(lthCodesSorted, x))}
val (_, elapsedLin) =
  timed{xs.map(x => linearSearch(lthCodesSorted)(_ == x))}
println(elapsedLin / elapsedBin)
\end{Code}


\Subtask Hur mycket snabbare blev binärsökningen jämfört med linjärsökningen?\footnote{Vid en körning på en i7-4970K med 4.0GHz tog \code{elapsedLin} cirka $3000~ms$ och \code{elapsedBin} cirka $60~ms$. Binärsökning var alltså i detta fall ungefär $50$ gånger snabbare än linjärsökning.}


\SOLUTION


\TaskSolved \what


\SubtaskSolved ---

\SubtaskSolved
\begin{Code}
def binarySearch(xs: Seq[String], key: String): Int = {

  var (low, high) = (0, xs.size - 1)
  var found = false
  var mid = -1

  while (low <= high && !found) {
    mid = (low + high) / 2
    if (xs(mid) == key) found = true
    else if (xs(mid) < key) low = mid + 1
    else high = mid - 1
  }
  if (found)
    mid
  else
    -(low + 1)
}
\end{Code}

\SubtaskSolved
Med en i7-3770K @ 3.50Hz tog sökningarna följande tid:

\begin{itemize}
\item Binärsökning: \code{time: 142.6 ms}
\item Linjärsökning: \code{time: 3316.5 ms}
\end{itemize}

Med en i7-8700T @ 2.40GHz tog sökningarna följande tid:
\begin{itemize}
\item Binärsökning: \code{time: 81.5 ms}
\item Linjärsökning: \code{time: 5138.6 ms}
\end{itemize}




\SubtaskSolved
Binärsökningen var ca 23 gånger snabbare på en i7-3770K @ 3.50Hz och ca 63 gånger snabbare på en i7-8700T CPU @ 2.40GHz.



\QUESTEND






\WHAT{Linjärsökning i Java.}

\QUESTBEGIN

\Task  \what~  Denna uppgift bygger vidare på uppgift \ref{task:arraymatrix-java} i kapitel \ref{chapter:W08}. Du ska göra en variant på linjärsökning som innebär att leta upp första yatzy-raden i en matris där varje rad innehåller utfallet av 5 tärningskast.

\Subtask Du ska lägga till metoderna \code{isYatzy} och \code{findFirstYatzyRow} i klassen \code{ArrayMatrix} i uppgift \ref{task:arraymatrix-java} i kapitel \ref{chapter:W08} enligt nedan skiss. Vi börjar med metoden  \code{isYatzy} i denna deluppgift (nästa deluppgift handlar om \code{findFirstYatzyRow}). OBS! Det finns en bugg i \code{isYatzy} -- rätta buggen och testa så att den fungerar.

\begin{Code}[language=Java]
    public static boolean isYatzy(int[] dice){ /* has one bug! */
        int col = 1;
        boolean allSimilar = true;
        while (col < dice.length && allSimilar) {
          allSimilar = dice[0] == dice[col];
        }
        return allSimilar;
    }

    /** Finds first yatzy row in m; returns -1 if not found */
    public static int findFirstYatzyRow(int[][] m){
        int row = 0;
        int result = -1;
        while (???) {
             /* linear search  */
        }
        return result;
    }
\end{Code}


\Subtask Implementera \code{findFirstYatzyRow}. Skapa först pseudo-kod för linjärsökningsalgoritmen innan du skriver implementationen i Java.
Testa ditt program genom att lägga till följande rader i huvudprogrammet.
Metoden \code{fillRnd} ingår i uppgift \ref{task:arraymatrix-java} i kapitel \ref{chapter:W08}.
\begin{Code}[language=Java]
        int[][] yss = new int[2500][5];
        fillRnd(yss, 6);
        int i = findFirstYatzyRow(yss);
        System.out.println("First Yatzy Index: " + i);
\end{Code}




\SOLUTION


\TaskSolved \what


\SubtaskSolved
\begin{Code}[language=Java]
public static boolean isYatzy(int[] dice){
    int col = 1;
    boolean allSimilar = true;
    while(col < dice.length && allSimilar){
        allSimilar = (dice[0] == dice[col]);
        col++; //denna raden saknades
    }
    return allSimilar;
}
\end{Code}

\SubtaskSolved

\begin{Code}[language=Java]
public static int findFirstYatzyRow(int[][] m){
    int row = 0;
    int result = -1;
    while(row < m.length){
        if(isYatzy(m[row])){
           result = row;
           break;
        }
        row++;
    }
    return result;
}
\end{Code}



\QUESTEND







\WHAT{Insättningssortering.}

\QUESTBEGIN

\Task  \what~ Implementera sortering av en heltalssekvens till en  sekvens med \textbf{insättningssortering} \Eng{insertion sort} i en funktion med följande signatur:
\begin{Code}
def insertionSort(xs: Seq[Int]): Seq[Int] = ???
\end{Code}

\emph{Lösningsidé:} Skapa en ny, tom sekvens som ska bli vårt sorterade resultat. För varje element i den osorterade sekvensen: Sätt in det på rätt plats i den nya sorterade sekvensen.

\Subtask \emph{Pseudokod:} Kör nedan pseudokod med papper och penna t.ex. på sekvensen 5 1 4 3 2 1. Rita minnessituationen efter varje runda i loopen. Här använder vi internt i funktionen föränderliga \code{ArrayBuffer} som är snabb på insättning och avslutar med \code{toVector} så att vi lämnar ifrån oss en oföränderlig sekvens.

\begin{algorithm}[H]
    $result \leftarrow$ en ny, tom ArrayBuffer \\
    \ForEach{element $e$ \bf{in} $xs$}{
      $pos \leftarrow$  leta upp rätt position i $result$ \\
      stoppa in $e$ på plats $pos$ i $result$
    }
    $result$.toVector
\end{algorithm}


\Subtask Implementera \code{insertionSort}. Använd en \code{while}-loop för att implementera rad 3 i pseudokoden. Sök upp dokumentationen för metoden \code{insert} på \code{ArrayBuffer}. Testa  \code{insert} på \code{ArrayBuffer} i REPL och verifiera att den kan användas för att stoppa in på slutet på den ''oanvända'' positionen som är precis efter sista positionen. Vad händer om man gör \code{insert} på positionen \code{size + 2}?

Klistra in din implementation av \code{insertionSort} i REPL och testa så att allt fungerar:
\begin{REPL}
scala> insertionSort(Vector())
res0: Seq[Int] = Vector()

scala> insertionSort(Vector(42))
res1: Seq[Int] = Vector(42)

scala> insertionSort(Vector(1,2,3))
res2: Seq[Int] = Vector(1, 2, 3)

scala> insertionSort(Vector(5,1,4,3,2,1))
res3: Seq[Int] = Vector(1, 1, 2, 3, 4, 5)
\end{REPL}


\SOLUTION

\TaskSolved \what


\SubtaskSolved ---

\SubtaskSolved

\begin{Code}
def insertionSort(xs: Seq[Int]): Seq[Int] = {
  val result = scala.collection.mutable.ArrayBuffer.empty[Int]
  for (e <- xs) {
    var pos = 0
    while (pos < result.size && result(pos) < e) pos += 1
    result.insert(pos,e)
  }
  result.toVector
}
\end{Code}

\QUESTEND





\WHAT{Sortering på plats.}

\QUESTBEGIN

\Task  \what~ Implementera sortering på plats \Eng{in-place} i en \code{Array[String]} med urvalssortering \Eng{selection sort}

\emph{Lösningsidé:} För alla index $i$: sök $minIndex$ för ''minsta'' strängen från plats $i$ till sista plats och byt plats mellan strängarna på plats $i$ och plats $minIndex$. Se även animering här: \href{https://sv.wikipedia.org/wiki/Urvalssortering}{sv.wikipedia.org/wiki/Urvalssortering}

Implementera enligt nedan skiss.  \emph{Tips:} Du har nytta av en modifierad variant av lösningen till uppgift \ref{task:minindex} i kapitel \ref{chapter:W02}.
\begin{Code}
def selectionSortInPlace(xs: Array[String]): Unit = {
  def indexOfMin(startFrom: Int): Int = ???
  def swapIndex(i1: Int, i2: Int): Unit = ???
  for (i <- 0 to xs.size - 1) swapIndex(i, indexOfMin(i))
}
\end{Code}




\SOLUTION


\TaskSolved \what


\begin{Code}
def selectionSortInPlace(xs: Array[String]): Unit = {

  def indexOfMin(startFrom: Int): Int = {
    var minPos = startFrom
    var i = startFrom + 1
    while (i < xs.size) {
      if (xs(i) < xs(minPos)) minPos = i
      i += 1
    }
    minPos
  }

  def swapIndex(i1: Int, i2: Int): Unit = {
    val temp = xs(i1)
    xs(i1) = xs(i2)
    xs(i2) = temp
  }

  for (i <- 0 to xs.size - 1) swapIndex(i, indexOfMin(i))
}
\end{Code}


\QUESTEND


\clearpage

%\ExtraTasks %%%%%%%%%%%%%%%%%%%




\WHAT{Undersök om en sekvens är sorterad.}

\QUESTBEGIN

\Task \label{task:isSorted} \what~   Ett enkelt och lättläst sätt att undersöka om en sekvens är sorterad visas nedan.
\begin{REPL}
scala> def isSorted(xs: Vector[Int]): Boolean = xs == xs.sorted
\end{REPL}


\Subtask\Pen  Om \code{xs} har $10^6$ element, hur många jämförelser kommer i värsta fall att ske med \code{isSorted} enligt ovan? Metoden \code{sorted} använder algoritmen Timsort\footnote{\href{http://stackoverflow.com/questions/14146990/what-algorithm-is-used-by-the-scala-library-method-vector-sorted}{stackoverflow.com/questions/14146990/what-algorithm-is-used-by-the-scala-library-method-vector-sorted}}. Sök upp antalet jämförelser i värstafallet på Wikipedia.

Denna lösning är dock relativt långsam för stora samlingar. Man behöver ju inte först sortera  för att avgöra om det är sorterat (om man inte ändå hade tänkt sortera av andra skäl), det räcker att kolla att elementen är i växande ordning.

\Subtask\Pen  Om \code{xs} har $n$ element, ungefär hur många jämförelser kommer i värsta fall att ske med \code{isSorted} ovan om man alltså först ska sortera och sedan jämföra den osorterade och den sorterade samlingen element för element? Metoden \code{sorted} använder algoritmen Timsort. Sök upp värstafallsprestandan för Timsort på Wikipedia.\footnote{\href{https://en.wikipedia.org/wiki/Timsort}{en.wikipedia.org/wiki/Timsort}}

\Subtask\label{subtask:issorted} Implementera en effektivare variant av \code{isSorted} som använder en \code{while}-sats och kollar att elementen är i växande ordning.

\Subtask\Pen Vad blir antalet jämförelser i värstafallet med metoden i deluppgift \ref{subtask:issorted} om du har $n$ element?


\Subtask \label{subtask:isSorted-zip} Man kan kolla om en sekvens är sorterad med det listiga tricket att först zippa sekvensen med sin egen svans och sedan kolla om alla element-par uppfyller sorteringskriteriet, alltså \code{xs.zip(xs.tail).forall(???)} där \code{???} byts ut mot lämpligt predikat. Vilken typ har 2-tupeln \code{xs.zip(xs.tail))} om \code{xs} är av typen \code{Vector[Int]}? Implementera \code{isSorted} med detta listiga trick. (Senare, i fördjupningsuppgift \ref{task:implicit-ordering}, ska vi göra \code{isSorted} generellt användbar för olika typer och olika ordningsdefinitioner.)


\SOLUTION


\TaskSolved \what



\SubtaskSolved

Det tar i värsta fall $O(n*log(n))$ för timsort att sortera listan med $n$ element. Sedan krävs $n$ stycken jämförelser mellan den sorterade och osorterade listan. Det totala antalet jämförelser i värstafallet uppgår därför till $n + n*log(n)$.

\SubtaskSolved

En mer effektiv version av \code{isSorted} som stoppar direkt när den upptäcker att ett element inte är sorterat.

\begin{Code}
def isSorted(xs: Vector[Int]): Boolean = {

  if(xs.length > 1){
    for(i <- 0 until xs.length-1 if xs(i) > xs(i+1)){
      return false
    }
  }
  true
}
\end{Code}

\SubtaskSolved

2-tupeln är av typen \code{(Int, Int)}.

\begin{Code}
def isSorted(xs: Vector[Int]): Boolean =
  xs.zip(xs.tail).forall(x => x._1 <= x._2)
\end{Code}



\QUESTEND






\WHAT{Insättningssortering på plats.}

\QUESTBEGIN

\Task  \what~ Implementera och testa sortering på plats i en array med heltal med \footnote{\href{https://en.wikipedia.org/wiki/Insertion_sort}{en.wikipedia.org/wiki/Insertion\_sort}}.

\Subtask Implementera och testa funktionen nedan i Scala med följande signatur:
\begin{Code}
  def insertionSort(xs: Array[Int]): Unit
\end{Code}
Placera metoden i ett objekt med lämpligt namn, samt skapa ett huvudprogram med testkod. Kompilera och kör från terminalen. Börja med att skriva sorteringsalgoritmen i pseudokod.

\Subtask Implementera och testa metoden nedan i Java med följande signatur:
\begin{Code}[language=Java]
  public static void insertionSort(int[] xs)
\end{Code}
Placera metoden i en klass med lämpligt namn, samt skapa ett huvudprogram med testkod. Börja med att skriva sorteringsalgoritmen i pseudokod.

\SOLUTION


\TaskSolved \what


\SubtaskSolved

\begin{Code}
def insertionSort(xs: Array[Int]): Unit = {

  for(elem <- 1 until xs.length if xs.length > 0){
    var pos = elem
    while(pos > 0 && xs(pos) < xs(pos - 1)){
      val temp = xs(pos -1)
      xs(pos -1) = xs(pos)
      xs(pos) = temp
      pos -= 1
    }
  }
}
\end{Code}

\SubtaskSolved

\begin{Code}[language=Java]
public static void insertionSort(int[] xs) {

    if (xs.length < 1)
        return;

    for (int i = 1; i < xs.length; i++) {
        int pos = i;

        for (; pos > 0 && xs[pos] < xs[pos - 1]; pos--) {
            int temp = xs[pos - 1];
            xs[pos - 1] = xs[pos];
            xs[pos] = temp;
        }
    }
}
\end{Code}



\QUESTEND



\clearpage

%\AdvancedTasks


\WHAT{Sortering till ny sekvens med urvalssortering.}

\QUESTBEGIN

\Task  \what~ Implementera och testa sortering till ny sekvens med urvalssortering\footnote{\href{https://en.wikipedia.org/wiki/Selection_sort}{en.wikipedia.org/wiki/Selection\_sort}} i Scala, enligt nedan skiss.  Du har nytta av lösningen till uppgift \ref{task:minindex} i kapitel \ref{chapter:W02}.
\begin{Code}
def selectionSort(xs: Seq[String]): Seq[String] = {
  def indexOfMin(xs: Seq[String]): Int = ???
  val unsorted = xs.toBuffer
  val result = scala.collection.mutable.ArrayBuffer.empty[String]
  /*
  så länge unsorted inte är tom {
    minPos = indexOfMin(unsorted)
    elem   = unsorted.remove(minPos)
    result.append(elem)
  }
  */
  result.toVector
}
\end{Code}



\SOLUTION


\TaskSolved \what


\begin{Code}
def selectionSort(xs: Seq[String]): Seq[String] = {
  def indexOfMin(xs: Seq[String]): Int = xs.indexOf(xs.min)
  val unsorted = xs.toBuffer
  val result = scala.collection.mutable.ArrayBuffer.empty[String]

  while (!unsorted.isEmpty) {
    val minPos = indexOfMin(unsorted)
    val elem = unsorted.remove(minPos)
    result.append(elem)
  }

  result.toVector
}
\end{Code}


\QUESTEND






\WHAT{Typklasser och implicita parametrar.}

\QUESTBEGIN

\Task  \what~  I Scala finns möjligheter till avancerad funktionsprogrammering med s.k. \textbf{typklasser}, som definierar generella beteenden som fungerar för befintliga typer utan att implementationen av dessa befintliga typer behöver ändras. Vi nosar i denna uppgift på hur implicita argument kan användas för att skapa typklasser, illustrerat med hjälp av implicita ordningarna, som är en typisk och användbar tillämpning av konceptet typklasser.

\Subtask \emph{Implicit parameter och implicit värde.} Med nyckelordet \code{implicit} framför en parameter öppnar man för möjligheten att låta kompilatorn ge argumentet ''automatiskt'' om den kan hitta ett värde med passande typ som också är deklarerat med \code{implicit}, så som visas nedan.
\begin{REPL}
scala> def add(x: Int)(implicit y: Int) = x + y
scala> add(1)(2)
scala> add(1)
scala> implicit val ngtNamn = 42
scala> add(1)
\end{REPL}
Vad blir felmeddelandet på rad 3 ovan? Varför fungerar det på rad 5 utan fel?

\Subtask \emph{Typklasser.} Genom att kombinera koncepten implicita värden, generiska klasser och implicita parametrar får man möjligheten att göra typklasser, så som \code{CanCompare} nedan, som vi kan få att fungera för befintliga typer utan att de behöver ändras.

Vad händer nedan? Vilka rader ger felmeddelande? Varför?

\begin{REPL}
scala> trait CanCompare[T] { def compare(a: T, b: T): Int }
scala> def sort2[T](a: T, b: T)(implicit cc: CanCompare[T]): (T, T) =
         if (cc.compare(a, b) > 0) (b, a) else (a, b)
scala> sort2(42, 41)
scala> implicit object intComparator extends CanCompare[Int]{
         override def compare(a: Int, b: Int): Int = a - b
       }
scala> sort2(42, 41)
scala> sort2(42.0, 41.0)
\end{REPL}

\Subtask Definiera ett implicit objekt som gör så att \code{sort2} fungerar för värden av typen \code{Double}.

\Subtask Definiera ett implicit objekt som gör så att \code{sort2} fungerar för värden av typen \code{String}.


\SOLUTION


\TaskSolved \what

\SubtaskSolved ---


\SubtaskSolved ---


\SubtaskSolved
Tänk på att det fortfarande måste returneras en Int.


\SubtaskSolved
Undersök i Javas API hur metoden \code{compareTo} är implementerad för strängar.

\QUESTEND





\WHAT{Användning av implicit ordning.}

\QUESTBEGIN

\Task \label{task:implicit-ordering} \what~  Vi ska nu göra \code{isSorted} från uppgift \ref{task:isSorted} mer generellt användbar genom att möjliggöra att implicita ordningsfunktioner finns tillgängliga för olika typer.

\Subtask  Med signaturen  \code{isSorted(xs: Vector[Int]): Boolean} så
fungerar sorteringstestet bara för samlingar av typen \code{Vector[Int]}.

Om vi i stället använder
\code{isSorted(xs: Seq[Int]): Boolean} fungerar den för alla samlingar med heltal, även \code{Array} och \code{List}. Testa nedan funktion i REPL med heltalssekvenser av olika typ.
\begin{Code}
def isSorted(xs: Seq[Int]): Boolean = xs == xs.sorted
\end{Code}

\Subtask Det blir problem med nedan försök att göra \code{isSorted} generisk. Hur lyder felmeddelandet? Vad saknas enligt felmeddelandet?
\begin{REPLnonum}
scala> def isSorted[T](xs: Seq[T]): Boolean = xs == xs.sorted
\end{REPLnonum}

\Subtask Vi vill gärna att \code{isSorted} ska fungera för godtyckliga typer T som har en ordningsdefinition. Detta kan göras med nedan funktion där typparametern \code{[T:Ordering]} betyder att \code{isSorted} är definierad för alla samlingar där typen \code{T} har en implicit ordning \code{Ordering[T]}. Speciellt gäller detta för alla grundtyperna \code{Int}, \code{Double}, \code{String}, etc., som alla har implicit tillgänglig \code{Ordering[Int]} etc.
\begin{Code}
def isSorted[T:Ordering](xs: Seq[T]): Boolean = xs == xs.sorted
\end{Code}
Testa metoden ovan i REPL enligt nedan.
\begin{REPL}
scala> isSorted(Vector(1,2,3))
scala> isSorted(Array(1,2,3,1))
scala> isSorted(Vector("A","B","C"))
scala> isSorted(List("A","B","C","A"))
scala> case class Person(firstName: String, familyName: String)
scala> val persons = Vector(Person("Kim", "Finkodare"), Person("Robin","Fulhack"))
scala> isSorted(persons)
\end{REPL}
Vad ger sista raden för felmeddelande? Varför?


\Subtask \emph{Implicita ordningar.} En typparameter på formen \code{[T:Ordering]} kallas kontextgräns \Eng{context bound} och föranleder kompilatorn att expandera funktionshuvudet för \code{isSorted} med en extra parameterlista som har en implicit parameter. I stället för att använda \code{[T:Ordering]} kan vi själva lägga till den implicita parametern som motsvarar kontextgränsen. Då får vi också tillgång till ett namn på den implicita ordningen och kan använda det namnet i funktionskroppen och anropa metoder som är medlemmar av typen \code{Ordering}.

\begin{CodeSmall}
def isSorted[T](xs: Seq[T])(implicit ord: Ordering[T]): Boolean =
  xs.zip(xs.tail).forall(x => ord.lteq(x._1, x._2))
\end{CodeSmall}

Objekt av typen \code{Ordering} har jämförelsemetoder som t.ex. \code{lteq} (förk. \emph{less than or equal}) och \code{gt} (förk. \emph{greater than}).

Det finns fördefinierade implicita objekt \code{Ordering[T]} för alla grundtyper, alltså t.ex. \code{Ordering[Int]}, \code{Ordering[String]}, etc.
Testa så att kompilatorn hittar ordningen för samlingar med värden av några grundtyper. Kontrollera även enligt nedan att det fortfarande blir problem för egendefinierade klasser, t.ex. \code{Person}  (detta ska vi råda bot på i uppgift \ref{task:custom-ordering}).
\begin{REPL}
scala> isSorted(Vector(1,2,3))
scala> isSorted(Array(1,2,3,1))
scala> isSorted(Vector("A","B","C"))
scala> isSorted(List("A","B","C","A"))
scala> class Person(firstName: String, familyName: String)
scala> val persons = Vector(Person("Kim", "Finkodare"), Person("Robin","Fulhack"))
scala> isSorted(persons)
\end{REPL}

\Subtask \emph{Importera implicita ordningsoperatorer från en \code{Ordering}.} Om man gör import på ett \code{Ordering}-objekt får man tillgång till implicita konverteringar som gör att jämförelseoperatorerna fungerar. Testa nedan variant av \code{isSorted} på olika sekvenstyper och verifiera att \code{<=}, \code{>}, etc., nu fungerar enligt nedan.
\begin{CodeSmall}
def isSorted[T](xs: Seq[T])(implicit ord: Ordering[T]): Boolean = {
  import ord._
  xs.zip(xs.tail).forall(x => x._1 <= x._2)
}
\end{CodeSmall}


\SOLUTION


\TaskSolved \what ---

\QUESTEND






\WHAT{Skapa egen implicit ordning med \code{Ordering}.}

\QUESTBEGIN

\Task \label{task:custom-ordering} \what~

\Subtask Ett sätt att skapa en egen, specialanpassad ordning är att mappa dina objekt till typer som redan har en implicit ordning. Med hjälp av metoden \code{by} i objektet \code{scala.math.Ordering} kan man skapa ordningar genom bifoga en funktion \code{T => S} där \code{T} är typen för de objekt du vill ordna och \code{S} är någon annan typ, t.ex. \code{String} eller \code{Int}, där det redan finns en implicit ordning.
\begin{REPL}
scala> case class Team(name: String, rank: Int)
scala> val xs =
         Vector(Team("fnatic", 1499), Team("nip", 1473), Team("lumi", 1601))
scala> xs.sorted  // Hur lyder felmeddelandet? Varför blir det fel?
scala> val teamNameOrdering = Ordering.by((t: Team) => t.name)
scala> xs.sorted(teamNameOrdering)   //explicit ordning
scala> implicit val teamRankOrdering = Ordering.by((t: Team) => t.rank)
scala> xs.sorted   // Varför funkar det nu?
\end{REPL}

\Subtask Vill man sortera i omvänd ordning kan man använda
\code{Ordering.fromLessThan} som tar en funktion \code{(T, T) => Boolean} vilken ska ge \code{true} om första parametern ska komma före, annars \code{false}. Om vi vill sortera efter \code{rank} i omvänd ordning kan vi göra så här:
\begin{REPL}
scala> val highestRankFirst =
         Ordering.fromLessThan[Team]((t1, t2) => t1.rank > t2.rank)
scala> xs.sorted(highestRankFirst)
\end{REPL}

\Subtask Om du har en case-klass med flera fält och vill ha en fördefinierad implicit sorteringsordning samt även erbjuda en alternativ sorteringsordning kan du placera olika ordningsdefinitioner i ett kompanjonsobjekt; detta är nämligen ett av de ställen där kompilatorn söker efter eventuella implicita värden innan den ger upp att leta.
\begin{Code}
case class Team(name: String, rank: Int)
object Team {
  implicit val highestRankFirst = Ordering.fromLessThan[Team]{
    (t1, t2) => t1.rank > t2.rank
  }
  val nameOrdering = Ordering.by((t: Team) => t.name)
}
\end{Code}
\begin{REPL}
scala> :pa
// Exiting paste mode, now interpreting.
case class Team(name: String, rank: Int)
object Team {
  implicit val highestRankFirst =
    Ordering.fromLessThan[Team]{(t1, t2) => t1.rank > t2.rank}
  val nameOrdering = Ordering.by((t: Team) => t.name)
}
scala> val xs =
         Vector(Team("fnatic", 1499), Team("nip", 1473), Team("lumi", 1601))
scala> xs.sorted
scala> xs.sorted(Team.nameOrdering)
\end{REPL}



\Subtask Det går också med kompanjonsobjektet ovan att få jämförelseoperatorer att fungera med din case-klass, genom att importera medlemmarna i lämpligt ordningsobjekt. Verifiera att så är fallet enligt nedan:
\begin{REPL}
scala> Team("fnatic",1499) < Team("gurka", 2)  // Vilket fel? Varför?
scala> import Team.highestRankFirst._
scala> Team("fnatic",1499) < Team("gurka", 2)  // Inget fel? Varför?
\end{REPL}


\SOLUTION


\TaskSolved \what ---

\QUESTEND






\WHAT{Specialanpassad ordning genom att ärva från \code{Ordered}.}

\QUESTBEGIN

\Task  \what~  Om det finns \emph{en} väldefinierad, specifik ordning som man vill ska gälla för sina case-klass-instanser kan man göra den ordnad genom att låta case-klassen mixa in traiten \code{Ordered} och implementera den abstrakta metoden \code{compare}.

\begin{Background}
En trait som används på detta sätt kallas \textbf{gränssnitt} \Eng{interface}, och om man \emph{implementerar} ett gränssnitt så uppfyller man ett ''kontrakt'', som i detta fall innebär att man implementerar det som krävs av ordnade objekt, nämligen att de har en konkret \code{compare}-metod. Du lär dig mer om gränssnitt i kommande kurser.
\end{Background}

\Subtask Implementera case-klassen \code{Team} så att den är en subtyp till \code{Ordered} enligt nedan skiss. Högre rankade lag ska komma före lägre rankade lag. Metoden \code{compare} ska ge ett heltal som är negativt om \code{this} kommer före \code{that}, noll om de ordnas lika, annars positivt.

\begin{Code}
case class Team(name: String, rank: Int) extends Ordered[Team]{
  override def compare(that: Team): Int = ???
}
\end{Code}
\emph{Tips:} Du kan anropa metoden \code{compare} på alla grundtyper, t.ex. \code{Int}, eftersom de är implicit ordnade. Genom att negera uttrycket blir ordningen den omvända.
\begin{REPL}
scala> -(2.compare(1))
\end{REPL}

\Subtask Testa att  din case-klass nu uppfyller det som krävs för att vara ordnad.
\begin{REPL}
scala> Team("fnatic",1499) < Team("gurka", 2)
\end{REPL}


\SOLUTION


\TaskSolved \what


Tänk på att för att sortering i omvänd ordning (alltså högst rank först) ska fungera så måste jämförelsen returnera \code{false}.

\begin{CodeSmall}
case class  Team(name: String, rank: Int) extends Ordered[Team]{
  override def compare(that: Team): Int = -rank.compare(that.rank)
}
\end{CodeSmall}



\QUESTEND





\WHAT{Jämförelsestöd i Java.}

\QUESTBEGIN

\Task  \what~
Java har motsvarigheter till \code{Ordering} och \code{Ordered}, som heter \code{java.util.Comparator} och \code{java.lang.Comparable}. I själva verket så är Scalas \code{Ordering} en subtyp till Javas \code{Comparator}, medan Scalas \code{Ordered} är en subtyp till Javas \code{Comparable}.
\begin{itemize}[nolistsep, noitemsep]
\item Javas \code{Comparator} och Scalas \code{Ordering} används för att skapa fristående ordningar som kan jämföra \emph{två olika} objekt. I Scala kan dessa göras implicit tillgängliga. I Javas samlingsbibliotek skickas instanser av \code{Comparator} med som explicita argument.
\item Javas \code{Comparable} och Scalas \code{Ordered} används som supertyp för klasser som vill kunna jämföra ''sig själv'' med andra objekt och har \emph{en} naturlig ordningsdefinition.
\end{itemize}

\Subtask\Pen Sök upp dokumentationen för \code{java.util.Comparator}. Vilken abstrakt metod måste implementeras och vad gör den?

\Subtask  I paketet \code{java.util.Arrays} finns en metod \code{sort} som tar en \code{Array[T]} och en \code{Comparable[T]}. Testa att använda dessa i REPL enligt nedan skiss. Starta om REPL så att ev. tidigare implicita ordningar för \code{Team} inte finns kvar.
\begin{REPL}
scala> import java.util.Comparator
scala> val teamComparator = new Comparator[Team]{
         def compare(o1: Team, o2: Team) = ???
       }
scala> val xs =
         Array(Team("fnatic", 1499), Team("nip", 1473), Team("lumi", 1601))
scala> java.util.Arrays.sort(xs.toArray, teamComparator)
scala> xs
\end{REPL}
%\begin{Code}
%// kod till facit
%val teamComparator = new Comparator[Team]{
%  def compare(o1: Team, o2: Team) = o2.rank - o1.rank
%}
%\end{Code}

\Subtask I Scala finns en behändig metod \code{Ordering.comparatorToOrdering} som skapar en implicit tillgänglig ordning om man har en \code{java.util.Comparator}. Testa detta enligt nedan i REPL, med deklarationerna från föregående deluppgift.
\begin{REPL}
scala> implicit val teamOrd = Ordering.comparatorToOrdering(teamComparator)
scala> xs.sorted
\end{REPL}



\Subtask\Pen Sök upp dokumentationen för \code{java.lang.Comparable}. Vilken abstrakt metod måste implementeras och vad gör den?

\Subtask Gör så att klassen \code{Point} är \code{Comparable} och att punkter närmare origo sorteras före punkter som är längre ifrån origo enligt nedan skiss. I Scala är typer som är \code{Comparable} implicit även \code{Ordered}, varför sorteringen nedan funkar. Verfiera detta i REPL när du klurat ut hur implementera \code{compareTo}.

\begin{Code}
case class Point(x: Int, y: Int) extends Comparable[Point] {
  def distanceFromOrigin: Double = ???
  def compareTo(that: Point): Int = ???
}
\end{Code}
\begin{REPL}
scala> val xs = Seq(Point(10,10), Point(2,1), Point(5,3), Point(0,0))
scala> xs.sorted
\end{REPL}
%\begin{Code}
%// kod till facit
%case class Point(x: Int, y: Int) extends Comparable[Point] {
%  def distanceFromOrigin: Double = math.hypot(x, y)
%  def compareTo(that: Point): Int =
%    (distanceFromOrigin - that.distanceFromOrigin).round.toInt
%}
%\end{Code}


\SOLUTION


\TaskSolved \what


\SubtaskSolved

\SubtaskSolved  %% b
\begin{Code}
val teamComparator = new Comparator[Team]{
  def compare(o1: Team, o2: Team) = o2.rank - o1.rank
}
\end{Code}


\SubtaskSolved

\SubtaskSolved

\SubtaskSolved

\begin{Code}
case class Point(x: Int, y: Int) extends Comparable[Point] {
  def distanceFromOrigin: Double = math.hypot(x, y)
  def compareTo(that: Point): Int =
    (distanceFromOrigin - that.distanceFromOrigin).round.toInt
}
\end{Code}


\QUESTEND






\WHAT{Fixa svensk sorteringsordning av ÄÅÖ.}

\QUESTBEGIN

\Task \label{task:swedish-letter-ordering} \what~   Svenska bokstäver kommer i, för svenskar, konstig ordning om man inte vidtar speciella åtgärder. Med hjälp av klassen \code{java.text.Collator} kan man få en \code{Comparator} för strängar som följer lokala regler för en massa språk på planeten jorden.

\Subtask Verifiera att sorteringsordningen blir rätt i REPL enligt nedan.

\begin{REPL}
scala> val fel = Vector("ö","å","ä","z").sorted
scala> val svColl = java.text.Collator.getInstance(new java.util.Locale("sv"))
scala> val svOrd = Ordering.comparatorToOrdering(svColl)
scala> val rätt = Vector("ö","å","ä","z").sorted(svOrd)
\end{REPL}
\Subtask Använd metoden ovan för att skriva ett program som skriver ut raderna i en textfil i korrekt svensk sorteringsordning. Programmet ska kunna köras med kommandot:\\\texttt{scala sorted -sv textfil.txt}

\Subtask Läs mer här: \\
\noindent{\href{http://stackoverflow.com/questions/24860138/sort-list-of-string-with-localization-in-scala}{\small stackoverflow.com/questions/24860138/sort-list-of-string-with-localization-in-scala}}



\SOLUTION


\TaskSolved \what



\QUESTEND






\WHAT{\texttt{java.util.Arrays.binarySearch}}

\QUESTBEGIN

\Task  \what~ I klassen \code{java.util.Arrays}\footnote{\href{https://docs.oracle.com/javase/8/docs/api/java/util/Arrays.html}{docs.oracle.com/javase/8/docs/api/java/util/Arrays.html}} finns en statisk metod \code{binarySearch} som kan användas enligt nedan.
\begin{REPL}
scala> val xs = Array(5,1,3,42,-1)
scala> java.util.Arrays.sort(xs)
scala> xs
scala> java.util.Arrays.binarySearch(xs, 42)
scala> java.util.Arrays.binarySearch(xs, 43)
\end{REPL}
Skriv ett valfritt Javaprogram som testar \code{java.util.Arrays.binarySearch}. Använd en array av typen \code{int[]} med några heltal som först sorteras med \code{java.util.Arrays.sort}.  Skriv ut det som returneras från  \code{java.util.Arrays.binarySearch}  i olika fall genom att asöka efter tal som finns först, mitt i, sist och tal som saknas.
\emph{Tips:} Man kan deklarera en array, allokera den och fylla den med värden så här i Java: \\
\jcode|int[] xs = new int[]{5, 1, 3, 42, -1};|


\SOLUTION

\TaskSolved \what

\QUESTEND



%
%
% \WHAT{NEEDS A TOPIC DESCRIPTION}
%
% \QUESTBEGIN
%
% \Task  \what~ Fördjupa dig inom webbteknologi.
%
% \Subtask Lär dig om HTML här: \url{http://www.w3schools.com/html/}
%
% \Subtask Lär dig om Javascript här: \url{http://www.w3schools.com/js/}
%
% \Subtask Lär dig om CSS här: \url{http://www.w3schools.com/css/}
%
% \Subtask Lär dig om Scala.JS här: \url{http://www.scala-js.org/}\SOLUTION
%
%
% \TaskSolved \what
%
% \QUESTEND


\subsection{Upgifter om trådar och jämlöpande exekvering}

\WHAT{Trådar.}

\QUESTBEGIN

\Task  \what~   Klassen \code{java.lang.Thread} används för att skapa  \textbf{trådar} med jämlöpande exekvering \Eng{concurrent execution}. På så sätt kan man få olika koddelar att köra samtidigt.

Klassen \code{Thread} definierar en tom \code{run}-metod. Vill man att tråden ska göra något vettigt får man överskugga \code{run} med det man vill ska göras.

En tråd körs igång med metoden \code{start} och då anropas automatiskt \code{run}-metoden och tråden exekverar koden i \code{run} jämlöpande med övriga trådar. Om man anropar \code{run} direkt blir det \emph{inte} jämlöpande exekvering.

\Subtask Skapa en tråd som gör något som tar lite tid och kör med \code{run} resp. \code{start}.
\begin{REPL}
def zzz = { print("zzzzzz"); Thread.sleep(5000); println(" VAKEN!")}
zzz
val t2 = new Thread{ override def run = zzz }
t2.run
t2.run; println("Gomorron!")
t2.start; println("Gomorron!")
t2.start
\end{REPL}

\Subtask Vad händer om man anropar \code{start} mer än en gång på samma tråd?

\Subtask Skapa två trådar med överskuggade \code{run}-metoder och kör igång dem samtidigt enligt nedan. Vilken ordning skrivs hälsningarna ut efter rad 3 resp. rad 4 nedan? Förklara vad som händer.
\begin{REPL}
val g = new Thread{ override def run = for (i <- 1 to 100) print("Gurka ") }
val t = new Thread{ override def run = for (i <- 1 to 100) print("Tomat ") }
g.run; t.run
g.start; t.start
\end{REPL}

\Subtask Använd \code{Thread.sleep} enligt nedan. Är beteendet helt förutsägbart (deterministiskt)? Förklara vad som händer. Du kan (om du kör Linux) avbryta REPL med Ctrl+C%
\footnote{\href{http://stackoverflow.com/questions/6248884/can-i-stop-the-execution-of-an-infinite-loop-in-scala-repl}{stackoverflow.com/questions/6248884/can-i-stop-the-execution-of-an-infinite-loop-in-scala-repl}}.
\begin{REPL}
def ibland(block: => Unit) = new Thread {
  override def run = while(true) { block; Thread.sleep(600) }
}.start
ibland(print("zzz ")); ibland(print("snark ")); ibland(println("hej!"))
\end{REPL}


\SOLUTION


\TaskSolved \what
     %%%TODO number  1 %%%starts with: \emph{Trådar.}  %%%

\SubtaskSolved   -

\SubtaskSolved  \code {java.lang.IllegalThreadStateException}. Det går inte att starta en tråd mer än en gång. Tråden kan därför inte startas om när den redan har exekverats.

\SubtaskSolved   När \code {start} anropas exekveras koden i \code{run} parallellt. Därför skrivs \code{Gurka} och \code{Tomat} ut omlöpande. Om istället \code{run} anropas direkt blir det inte jämnlöpande exekvering och \code{Gurka} skrivs ut 100 gånger, sedan skrivs \code{Tomat} ut 100 gånger.

\SubtaskSolved   \code{Thread.sleep} pausar inte tråden i exakt den tiden som angets. Alltså kommer det skrivas ut \code{zzz snark hej!} i de flesta fall, men det är inte garanterat.



\QUESTEND






\WHAT{Jämlöpande variabeluppdatering.}

\QUESTBEGIN

\Task \label{task:racecondition} \what~   Skriv klasserna \code{Bank} och \code{Kund} i en editor och klistra sedan in koden i REPL.

\begin{Code}
class Bank {
  private var saldo = 0;
  def visaSaldo: Unit = println("saldo: " + saldo)
  def sättIn: Unit = { saldo += 1 }
  def taUt: Unit   = { saldo -= 1 }
}

class Kund(bank: Bank) {
  def slösaSpara = {bank.taUt; Thread.sleep(1); bank.sättIn}
}
\end{Code}

\Subtask Använd funktionen \code{ibland} från föregående uppgift och kör nedan rader i REPL. Resultatet av jämlöpande variabeluppdatering blir här heltokigt och leder till mycket upprörda bankkunder och -ägare. Förklara vad som händer.

\begin{REPL}
val bank = new Bank
bank.visaSaldo
bank.sättIn
bank.visaSaldo
bank.taUt
bank.visaSaldo

val bamse = new Kund(bank)
val skutt = new Kund(bank)

bamse.slösaSpara
skutt.slösaSpara
bank.visaSaldo

def ofta(block: => Unit) = new Thread {
  override def run = while(true) { block; Thread.sleep(1) }
}.start

ofta(bamse.slösaSpara); ofta(skutt.slösaSpara)

ibland(bank.visaSaldo)
\end{REPL}


\SOLUTION


\TaskSolved \what
     %%%TODO number  2 %%%starts with: \emph{Jämlöpande variabeluppdat%%%

\SubtaskSolved  I \code{slösaSpara} hämtas saldot, ändras och placeras tillbaka i minnet -  fördröjs -  upprepas. Om \code{bamse} blir klar med att ladda, ändra och lagra innan skutt gör detsamma med den muterbara variablen hade det inte varit perfekt. Problemet ligger i  när en tråd laddar och innan den kan lagra det förändrade värdet laddar den andra tråden samma värde. Bara en av dessa trådar vinner racet och får lagra sitt ändrade tal. \code{skutt} och \code{bamse} blir alltså upprörda för att inte alla dess uttag och insättningar registreras.


\QUESTEND






\WHAT{Trådsäkra \code{AtomicInteger}.}

\QUESTBEGIN

\Task  \what~  Det finns stöd i JVM för att åstadkomma uppdateringar som inte kan avbrytas av andra trådar under pågånde minnesskrivning. En operation som inte kan avbrytas kallas \textbf{atomär} \Eng{atomic}. Studera dokumentationen för \code{AtomicInteger}\footnote{\href{https://docs.oracle.com/javase/8/docs/api/java/util/concurrent/atomic/AtomicInteger.html}{docs.oracle.com/javase/8/docs/api/java/util/concurrent/atomic/AtomicInteger.html}} och prova nedan kod. Förklara vad som händer.

Använd funktionerna \code{ofta} och \code{ibland} från tidigare uppgifter.
\begin{Code}
class SäkerBank {
  import java.util.concurrent.atomic.AtomicInteger
  private var saldo = new AtomicInteger
  def visaSaldo: Unit = println(s"saldo: ${saldo.get}")
  def sättIn: Unit = { saldo.incrementAndGet }
  def taUt: Unit   = { saldo.decrementAndGet }
}

class SäkerKund(bank: SäkerBank) {
  def slösaSpara = {bank.taUt; Thread.sleep(1); bank.sättIn}
}
\end{Code}
\begin{REPL}
val säkerBank = new SäkerBank
val farmor = new SäkerKund(säkerBank)
val vargen = new SäkerKund(säkerBank)

ofta(farmor.slösaSpara); ofta(vargen.slösaSpara)

ibland(säkerBank.visaSaldo)
\end{REPL}





\SOLUTION


\TaskSolved \what
     %%%TODO number  3 %%%starts with: \emph{Jämlöpande exekvering med%%%

Nu är \code{farmor} garanterad att kunna ladda saldot, ta ut pengar/ändra och lagra innan \code{vargen} kan överskriva resultatet. I \code{slösaSpara} pausas tråden i en millisekund så \code{vargen} kan fortfarande ta ut pengar innan \code{farmor} hinner sätta in pengar igen. Dock kommer alla uttag och insättningar registreras eftersom operationerna är atomära.


\QUESTEND






\WHAT{Jämlöpande exekvering med \code{scala.concurrent.Future}.}

\QUESTBEGIN

\Task \label{task:future} \what~   Att skapa och hålla reda på trådar kan bli ganska omständligt och knepigt att få rätt på.
Med hjälp av \code{scala.concurrent.Future} kan man på ett enklare sätta skapa jämlöpande exekvering.

\begin{Background}
Med en \code{Future} skapas jämlöpande exekvering som ''under huven'' använder ett ramverk som heter Akka\footnote{\url{http://akka.io/}}, skrivet i Scala och Java. Akka erbjuder automatisk  multitrådning med s.k. trådpooler och möjliggör avancerad parallellprogrammering på en hög  abstraktionsnivå, där man själv slipper skapa instanser av klassen \code{Thread}. I stället kan man helt enkelt placera sin kod inramad med \code|Future{ "körs parallellt" }| efter att man importerat det som behövs.
\end{Background}

\Subtask För att skapa jämlöpande exekvering med \code{Future} behöver man först göra import enligt nedan; då skapas ett exekveringssammanhang med trådpooler redo för användning. Starta om REPL och studera felmeddelandet efter rad 1 nedan. Importera därefter enligt nedan. Vad har \code{f} för typ?
\begin{REPL}
scala> concurrent.Future { Thread.sleep(1000); println("En sekund senare!") }
scala> import scala.concurrent._
scala> import ExecutionContext.Implicits.global
scala> val f = Future { Thread.sleep(1000); println("En sekund senare!") }
\end{REPL}

\Subtask Skapa en procedur \code{printLater} enligt nedan som skriver ut argumentet efter slumpmässig tid. Förklara vad som händer nedan.
\begin{REPL}
scala> def printLater(a: Any): Unit =
         Future { Thread.sleep((math.random() * 10000).toInt); print(a + " ") }
scala> (1 to 42).foreach(i => printLater(i)); println("alla är igång!")
\end{REPL}

\Subtask Skapa enligt nedan en \code{Future} som räknar ut hur många siffror det är i ett väldigt stort tal. Med \code{onComplete} kan man ange vad som ska göras när den tunga beräkningen är färdig; detta kallas att ''registrera en callback''. Vilken returtyp har \code{big}? Hur många siffror har det stora talet? Vad har \code{r} för typ? Justera argumentet till \code{big} om du inte orkar vänta på resultatet...

\begin{REPL}
scala> BigInt(10).pow(100)
scala> BigInt(10).pow(100).toString.size
scala> def big(n: Int) = Future { BigInt(n).pow(n).toString.size }
scala> big(1234567).onComplete{r => println(r + " siffror") }
\end{REPL}

\Subtask Den stora vinsten med \code{Future} är att man kan köra vidare under tiden, varför anropet av \code{Future} kallas \textbf{icke-blockerande} \Eng{non-blocking}. Det händer ibland att man ändå vill blockera exekveringen i väntan på ett resultat. Man kan då använda objektet \code{scala.concurrent.Await} och dess metod \code{result} enligt nedan. Använd \code{big} från föregående uppgift och gör en blockerande väntan på resultatet enligt nedan. Vad händer? Vad händer om du väntar för kort tid?

\begin{REPL}
scala> import scala.concurrent.duration._
scala> Await.result(big(1234567), 20.seconds)
\end{REPL}



\SOLUTION


\TaskSolved \what
     %%%TODO number  4 %%%starts with: TODO  %%%%%%%%%%%%%%%%%%%\Advan%%%

\SubtaskSolved  error: Cannot find an implicit ExecutionContext. Future behöver en ExecutionContext för att kunna köras. \code{f} är av typen Future[Unit].

\SubtaskSolved  Funktionen \code{printLater} har en Future, vilket innebär att när både \code{printLater} och \code{println} anropas i foreach-loopen exekveras de jämnlöpande. Eftersom det tar längre tid att starta upp en Future för datorn är \code{println} snabbare och skriver ut att alla är igång först. Sedan skrivs siffrorna från 1 - 42 ut med oregelbundna mellanrum eftersom tråden pausas olika länge.

\SubtaskSolved  \code{big} är en Future[Int]. Det stora talet har 7 520 383 siffror. \code{r} är av typen Try[Int] (se dokumentationen för Future om du är osäker)

\SubtaskSolved  Eftersom exekveringen blockas tills den har fått ett resultat går det inte att fortsätta skriva i REPL medan uträkningen pågår. Väntar man för kort tid får man ett TimeOutException och uträkningen avbryts.


\QUESTEND






\WHAT{Använda \code{Future} för att göra flera saker samtidigt.}

\QUESTBEGIN

\Task  \what~
I denna uppgift ska du ladda ner webbsidor parallellt med hjälp av \code{Future}, så att en nedladdning kan avslutas under tiden en annan dröjer.

\Subtask Koden för en minimal webbsida ser ut som nedan. Du kan beskåda sidan här: \url{http://fileadmin.cs.lth.se/pgk/mini.html} eller skriva in nedan kod i en fil som heter något som slutar på \texttt{.html} och öppna filen i din webbläsare.

\begin{verbatim}
<!DOCTYPE html>
<html>
<body>
HELLO WORLD!
</body>
</html>
\end{verbatim}

\Subtask För att simulera slöa webbservrar kan man ladda ner en sida via sajten \texttt{http://deelay.me/}. Ladda ner ovan sida med 2 sekunders fördröjning:\\
\url{http://deelay.me/2000/http://fileadmin.cs.lth.se/pgk/mini.html}

\Subtask Man kan ladda ner webbsidor med \code{scala.io.Source}. Vad händer nedan? Försök, med ledning av hur \code{delay} beräknas, uppskatta hur lång tid du måste vänta i medeltal, i bästa fall, respektive värsta fall, innan du kan se första webbsidan i vektorn \code{laddningar} nedan?

\begin{REPL}
scala> def ladda(url: String) = scala.io.Source.fromURL(url).getLines.toVector
scala> def slöladda(url: String) = {
         val delay = (math.random() * 1000 + 2000).toInt
         val delaySite = s"http://deelay.me/$delay/"
         ladda(delaySite+url)
      }
scala> ladda("http://fileadmin.cs.lth.se/pgk/mini.html")
scala> def seg = slöladda("http://fileadmin.cs.lth.se/pgk/mini.html")
scala> val laddningar = Vector.fill(10)(seg)
scala> laddningar(0)
\end{REPL}

\Subtask Innan vi kan köra igång en \code{Future} så måste vi, som visats i uppgift \ref{task:future} importera den underliggande exekveringsmiljön som är redo att parallelisera ditt program i trådar utan att du själv måste skapa dem. Vad händer nedan?
\begin{REPL}
scala> import scala.concurrent._
scala> import ExecutionContext.Implicits.global
scala> val f = Future{ seg }
scala> f   // kolla om den är klar annars prova igen senare
scala> f
\end{REPL}

\Subtask Ladda indata utan att blockera \Eng{non-blocking input}. Förklara vad som händer nedan.
\begin{REPL}
scala> val nonblock = Future{ Vector.fill(10)(seg) }
scala> nonblock   // kolla igen senare om ej klar
scala> nonblock
\end{REPL}

\Subtask Ladda indata separat i olika parallella trådar. Förklara vad som händer nedan. Kör uttrycket på rad 3 nedan upprepade gånger i snabb följd efter varandra med pil-upp+Enter i REPL.
\begin{REPL}
scala> val para = Vector.fill(10)(Future{ seg })
scala> para
scala> para.map(_.isCompleted)
scala> para.map(_.isCompleted) // studera hur de blir färdiga en efter en
scala> para(0)
\end{REPL}

\Subtask Registrera en callback med metoden \code{onComplete}. Förklara vad som händer nedan.

\begin{REPL}
scala> val action = Vector.fill(10)(Future{ seg })
scala> action(0).onComplete(xs => println(s"ready:$xs"))
scala> // vänta tills laddning på plats 0 är klar
\end{REPL}

\Subtask Registrera en callback för felhantering i händelse av undantag med metoden \code{onFailure}. Förklara vad som händer nedan.
\begin{REPL}
scala> def lycka  = { Thread.sleep(3000); println(":)") }
scala> def olycka = { Thread.sleep(3000); 42 / 0; lycka }
scala> Future{ lycka  }.onFailure{ case e => println(s":( $e") }
scala> Future{ olycka }.onFailure{ case e => println(s":( $e") }
\end{REPL}



\SOLUTION


\TaskSolved \what
     %%%TODO number  5 %%%starts with: Sök upp och studera dokumentati%%%

\SubtaskSolved  -

\SubtaskSolved  -

\SubtaskSolved  Varje sida fördröjs med mellan 2 upp till 3 sekunder (2000-3000 millisekunder). Så i medeltal tar det 2.5 sekunder för varje sida att laddas. Vektorn måste fyllas innan exekveringen kan fortsätta. Därför laddas alla 10 stycken sidor in innan man kan se första websidan. Det tar därför i medeltal 2.5 x 10 = 25 sekunder.

\SubtaskSolved  \code{f} ger en Vektor fylld med strängar där varje element ges av en rad på hemsidan. Då \code{f} körs i bakgrunden kan programmet fortlöpa medan innehållet räknas ut. Du kan därför skriva \code{f} i REPL:n men det är inte säkert att proccessen är klar och det slutgilltiga resultatet visas.

\SubtaskSolved  Samma som ovan, förutom att det blir en vektor där varje element är i sig en vektor med strängar.

\SubtaskSolved  Laddar in datan parallelt så nedladdingen sker samtidigt, men det går olika snabbt pga metoden seg.

\SubtaskSolved  Eftersom datan laddas i parallella trådar utan blockering blir de inte klara i ordning, utan i den ordningen tråden körs klart. Till slut blir alla klara och resultatet visar en vektor med \code{true} värden.

\SubtaskSolved  Metoden \code{lycka} är väldefinerad och kastar därför inga undantag. Den skriver alltid ut \code{:)}. Metoden \code{olycka} är inte väldefinerad då division med 0 ger \\\code{java.lang.ArithmeticException}. Detta fångas upp vid callbacken och det skrivs ut \code{:(} samt det specifierade undantaget.

\ExtraTasks %%%%%%%%%%%%


\QUESTEND






\WHAT{}

\QUESTBEGIN

\Task  \what~ Räkna ut stora primtal parallellt genom att använda nedan funktioner. Implementera \code{isPrime} enligt pseudokod från den engelska wikipediasidan om primtalstest\footnote{\href{https://en.wikipedia.org/wiki/Primality_test}{en.wikipedia.org/wiki/Primality\_test}} med den s.k. ''naiva algoritmen''.  Räkna ut 10 st slumpvisa primtal med 16 siffror vardera. Gör beräkningarna parallellt med hjälp av \code{Future}.

\begin{Code}
def isPrime(n: BigInt): Boolean = ???

def nextPrime(start: BigInt): BigInt = {
  var i = start
  while (!isPrime(i)) { i += 1 }
  i
}

def randomBigInt(nDigits: Int): BigInt = {
   def rndChar = ('0' + (math.random() * 10).toInt).toChar
   val str = Array.fill(nDigits)(rndChar).mkString
   BigInt(str)
}
\end{Code}

\SOLUTION


\TaskSolved \what
  %%%TODO number  6 %%%

\begin{Code}
def isPrime(n: BigInt): Boolean = n match {
  case _ if (n <= 1) => false
  case _ if (n <= 3) => true
  case _ if n % 2 == 0 || n % 3 == 0 => false
  case _ =>
    var i = BigInt(5)
    while (i * i < n) {
      if (n % i == 0 || n % (i + 2) == 0) false
      i += 6
    }
    true
}

import scala.concurrent._
import ExecutionContext.Implicits.global

val primes = Vector.fill(10)(Future{nextPrime(randomBigInt(16))})
primes.foreach(_.onSuccess{case i => println(i)})
\end{Code}


\QUESTEND






\WHAT{Svara på teorifrågor.}

\QUESTBEGIN

\Task  \what~\Pen

\Subtask Vad är en tråd?

\Subtask Hur skapar man en tråd med klassen \code{Thread}?

\Subtask Hur startar man en tråd?

\Subtask Vilka problem kan man råka ut för om man uppdaterar samma resurs i flera olika trådar?

\Subtask Vad innbär det att kod är \emph{trådsäker}?

\Subtask Nämn några fördelar med att använda Future jämfört med att använda trådar direkt.


\SOLUTION


\TaskSolved \what
 %%%TODO number  7 %%%

\SubtaskSolved  Stackoverflow ger följande förklaring:

A thread is an independent set of values for the processor registers (for a single core). Since this includes the Instruction Pointer (aka Program Counter), it controls what executes in what order. It also includes the Stack Pointer, which had better point to a unique area of memory for each thread or else they will interfere with each other.

\SubtaskSolved

\begin{Code}
val thread = new Thread(new Runnable{
	def run(){println(''Det här är en tråd'')}
})
\end{Code}

\SubtaskSolved  \code{thread.start}

\SubtaskSolved  Det kan bli kapplöpning(race conditions) om vilken tråds resurser blir sparade. Vilket leder till att de andra trådarnas ändringar blir ignorerade.

\SubtaskSolved  Trådsäkerhet innebär att flera trådar kan köras parallellt utan felaktigheter i resultatet. Exempelvis får man vara väldigt försiktig om man vill ha en muterbar variabel som alla trådar ska ändra samtidigt.

\SubtaskSolved  Till exempel slipper man skapa instanser av klassen Thread eftersom man kan placera koden i en Future istället. Den löser även mycket under huven för kodaren.


\QUESTEND






\WHAT{Klasser med atomär uppdatering.}

\QUESTBEGIN

\Task  \what~ Läs om och testa klasserna AtomicBoolean, AtomicDouble och AtomicReference för atomär uppdatering i paketet \\ \code{java.util.concurrent.atomic}.

Använd några av dessa tillsammans med \code{scala.concurrent.Future}.


\SOLUTION

\TaskSolved --

\QUESTEND





\WHAT{Skapa din egen multitrådade webbserver.}

\QUESTBEGIN

\Task  \what~

\Subtask Skriv in\footnote{Eller ladda ner här: \href{https://github.com/lunduniversity/introprog/blob/master/compendium/examples/simple-web-server/webserver.scala}{github.com/lunduniversity/introprog/blob/master/compendium/examples/simple-web-server/webserver.scala}} nedan kod i en editor och spara i en fil med namn \texttt{webserver.scala} och kompilera och kör med \texttt{scala webserver.start} och beskriv vad som händer när du med din webbläsare surfar till adressen: \\ \url{http://localhost:8089/abbasillen}

\scalainputlisting[numbers=left,basicstyle=\ttfamily\fontsize{11}{12}\selectfont]{examples/simple-web-server/webserver.scala}

\Subtask Du ska nu skapa en webbserver som gör något lite mer intressant. Den ska svara med det 13:e Fibonacci-talet\footnote{\href{https://sv.wikipedia.org/wiki/Fibonaccital}{https://sv.wikipedia.org/wiki/Fibonaccital}} om du surfar till \url{http://localhost:8089/fib/13}.
Spara din webbserver från föregående deluppgift under det nya namnet \texttt{fibserver.scala} och använd koden nedan och lägg till och ändra så att din server kan svara med Fibonaccital. Vi börjar med att räkna ut Fibonaccital i funktionen \code{compute.fib} nedan på ett onödigt processorkrävande sätt med exponentiell tidskomplexitet så att webbservern verkligen får jobba, för att i senare deluppgifter implementera \code{compute.fib} med linjär tidskomplexitet och därmed undvika onödig planetuppvärmning.
\begin{CodeSmall}
  object compute {
    def fib(n: BigInt): BigInt = {
      if (n < 0) 0 else
      if (n == 1 || n == 2) 1
      else fib(n - 1) + fib(n -2)
    }
  }

  def fibResponse(num: String) = Try { num.toInt } match {
    case Success(n) => html.page(s"fib($n) == " + compute.fib(n))
    case Failure(e) => html.page(s"FEL $e: skriv heltal, inte $num")
  }

  def errorResponse(uri:String) = html.page("FATTAR NOLL: " + uri)

  def handleRequest(cmd: String, uri: String, socket: Socket): Unit = {
    val os = socket.getOutputStream
    val parts = uri.split('/').drop(1) // skip initial slash
    val response: String = (parts.head, parts.tail) match {
      case (head, Array(num)) => fibResponse(num)
      case _                  => errorResponse(uri)
    }
    os.write(html.header(response.size).getBytes("UTF-8"))
    os.write(response.getBytes("UTF-8"))
    os.close
    socket.close
  }
\end{CodeSmall}
Kör i terminalen med \texttt{scala fibserver.start} och beskriv vad som händer i din webbläsare när du surfar till servern.


%%%\textbf{KOD TILL FACIT:}
%%%\scalainputlisting[numbers=left,basicstyle=\ttfamily\fontsize{11}{12}\selectfont]{examples/simple-web-server/fibserver.scala}


\Subtask Surfa efter flera stora Fibonacci-tal samtidigt i olika flikar i din browser. Hur märks det att servern bara kör i en enda tråd?

\Subtask Gör din server multitrådad med hjälp av den nya server-loopen nedan.

\begin{CodeSmall}
import scala.concurrent._
import ExecutionContext.Implicits.global

  def serverLoop(server: ServerSocket): Unit = {
    println(s"http://localhost:${server.getLocalPort}/hej")
		while (true) {
  		Try {
  		  var socket = server.accept  // blocks thread until connect
	  	  val scan = new Scanner(socket.getInputStream, "UTF-8")
		    val (cmd, uri) = (scan.next, scan.next)
			  println(s"Request: $cmd $uri")
		    Future { handleRequest(cmd, uri, socket) }.onFailure {
		      case e => println(s"Reqest failed: $e")
		    }
		  }.recover{ case e: Throwable => s"Connection failed: $e" }
		}
  }
\end{CodeSmall}

\Subtask Surfa efter flera stora Fibonacci-tal samtidigt i olika flikar i din browser. Hur märks det att servern är multitrådad?


\Subtask Det är onödigt att räkna ut samma Fibonacci-tal flera gånger. Med hjälp av en cache i form av en föränderlig \code{Map} kan du spara undan redan uträknade värden. Det funkar dock inte med en vanlig \code{scala.collection.mutable.Map} i vår multitrådade webbserver, eftersom den inte är \textbf{trådsäker} \Eng{thread-safe}. Med trådosäkra föränderliga datastrukturer blir det samma besvärliga beteende som i uppgift \ref{task:racecondition}.

Du ska i stället använda \code{java.util.concurrent.ConcurrentHashMap}. Sök upp  dokumentationen för \code{ConcurrentHashMap} och försök förstå koden nedan. Hur fungerar metoderna \code{containsKey}, \code{put} och \code{get}?
\begin{Code}
object compute {
  import java.util.concurrent.ConcurrentHashMap
  val memcache = new ConcurrentHashMap[BigInt, BigInt]

  def fib(n: BigInt): BigInt =
    if (memcache.containsKey(n)) {
      println("CACHE HIT!!! no need to compute: " + n)
      memcache.get(n)
    } else {
      println("cache miss :( must compute fib:  " + n)
      val f = fastFib(n)
      memcache.put(n, f)
      f
    }

  private def fastFib(n: BigInt): BigInt = {
    if (n < 0) 0 else
    if (n == 1 || n == 2) 1
    else fib(n - 1) + fib(n -2)
  }
}
\end{Code}

\Subtask Använd ovan \code{fib}-objekt i en ny version av din webserver. Spara den i en ny kodfil med namnet \texttt{fibserver-memcached.scala}. Undersök hur snabbt det går med stora Fibonaccital med den nya varianten. Hur stora tal kan du räkna ut? Kan servern fortsätta efter överflödad stack? Förklara varför.

\Subtask Nu när vi kan få väldigt stora Fibonacci-tal kan det vara användbart att stoppa in radbrytningar på webbsidan. Html-taggen \texttt{</br>} ger en radbrytning.
\begin{Code}
  def insertBreak(s: String, n: Int = 80): String = {
    if (s.size < n) s
    else s.take(n) + "</br>" + insertBreak(s.drop(n),n)
  }
\end{Code}
Använd den rekursiva funktionen ovan för att pilla in radbrytningstaggar på var $n$:te position i långa strängar. Testa hur det ser ut på webbsidan med ovan funktion när din server svarar med väldigt stora tal.

\Subtask Vi ska nu använda det större heap-minnet i stället för stack-minnet och därmed inte begränsas av stackens max-storlek. Skriv om \code{fastFib} så att den använder en \code{while}-sats i stället för ett rekursivt anrop. Denna uppgift är ganska klurig, men om du kör fast kan du snegla i lösningarna i Appendix för inspiration.

Hur stora tal klarar din server nu? Vad händer med servern när minnet tar slut? Hur kan du skydda servern så att den inte kan hänga sig?

\SOLUTION


\TaskSolved \what
 %%%TODO number  9 %%%

\SubtaskSolved  \code{abbasillen} skrivs ut baklänges till \code{nellisabba}.

\SubtaskSolved

\SubtaskSolved

\SubtaskSolved

\SubtaskSolved

\SubtaskSolved

\SubtaskSolved

\SubtaskSolved

\SubtaskSolved

Lösningsförslag:
\scalainputlisting[numbers=left,basicstyle=\ttfamily\fontsize{11}{12}\selectfont]{examples/simple-web-server/fibserver-threaded-memcached-while.scala}


\QUESTEND






\WHAT{}

\QUESTBEGIN

\Task  \what~ Utöka din server med fler beräkningsintensiva funktioner. Exempelvis primtalsberäkningar eller beräkningar av valfritt antal decimaler av $\pi$ eller $e$. Utnyttja gärna det du lärt dig i  matematiken om summor och serieutvecklingar.

\SOLUTION


\TaskSolved \what
 %%%TODO number  10 %%%

---


\QUESTEND






\WHAT{}

\QUESTBEGIN

\Task  \what~ Läs mer om \code{Future} och jämlöpande exekvering i Scala här:\\
\href{http://alvinalexander.com/scala/future-example-scala-cookbook-oncomplete-callback}{alvinalexander.com/scala/future-example-scala-cookbook-oncomplete-callback}

\SOLUTION


\TaskSolved \what
 %%%TODO number  11 %%%

---


\QUESTEND






\WHAT{}

\QUESTBEGIN

\Task  \what~ Läs mer om jämlöpande exekvering och multitrådade program i Java här: \href{http://www.tutorialspoint.com/java/java_multithreading.htm}{www.tutorialspoint.com/java/java\_multithreading.htm}  \\
\noindent När man skriver program med jämlöpande exekvering finns det många fallgropar; det kan bli kapplöpning \Eng{race conditions} om gemensamma resurser och dödläge \Eng{deadlock} där inget händer för att trådar väntar på varandra. Mer om detta i senare kurser.


\SOLUTION


\TaskSolved \what
 %%%TODO number  12 %%%

---


\QUESTEND






\WHAT{Studera dokumentationen i \code{scala.concurrent}.}

\QUESTBEGIN

\Task  \what~\Pen

\Subtask Studera dokumentationen för \code{scala.concurrent.Future}\footnote{\href{http://www.scala-lang.org/api/current/scala/concurrent/Future.html}{http://www.scala-lang.org/api/current/scala/concurrent/Future.html}}. Hur samverkar \code{Future} med \code{Try} och \code{Option}? Vilka vanliga samlingsmetoder känner du igen?

\Subtask Studera dokumentationen för \code{scala.concurrent.duration.Duration}\footnote{\href{http://www.scala-lang.org/api/current/scala/concurrent/duration/Duration.html}{www.scala-lang.org/api/current/scala/concurrent/duration/Duration.html}}. Vilka tidsenheter kan användas?

\Subtask Vid import av \code{scala.concurrent.duration._ } dekoreras de numeriska klasserna med metoder för att skapa instanser av klassen \code{Duration}. Detta möjligörs med hjälp av klassen \code{scala.concurrent.duration.DurationConversions}. Studera dess dokumentation och testa att i REPL skapa några tidsperioder med metoderna på \code{Int}.



\SOLUTION


\TaskSolved \what
 %%%TODO number  13 %%%

\SubtaskSolved

\SubtaskSolved

\SubtaskSolved


\QUESTEND






\WHAT{}

\QUESTBEGIN

\Task  \what~ Fördjupa dig inom webbteknologi.

\Subtask Lär dig om HTML, CSS och JavaScript här: \url{https://developer.mozilla.org/en-US/docs/Learn}

\Subtask Lär dig om Scala.JS här: \url{http://www.scala-js.org/}\SOLUTION


\TaskSolved \what
 %%%TODO number  14 %%%

\SubtaskSolved  ---

\SubtaskSolved  ---

\SubtaskSolved  ---

\SubtaskSolved  ---
\QUESTEND

%!TEX encoding = UTF-8 Unicode
%!TEX root = ../exercises.tex

\ifPreSolution

\Exercise{\ExeWeekTHIRTEEN}\label{exe:W13}
\begin{Goals}
\item Kunna skriva tentamenslika program med papper, penna och snabbreferens som enda hjälpmedel.
\item Förbereda projektredovisningen.
\item Kunna skapa dokumentation med \code{scaladoc} och \code{javadoc}.
\item Kunna skapa jar-filer.
\end{Goals}

% \begin{Preparations}
% \item \StudyTheory{13}
% \end{Preparations}

\else

\ExerciseSolution{\ExeWeekTHIRTEEN}

\fi


\subsection{Förberedelse inför examination}




\WHAT{Gör en extenta.} %%%%%%%%%%%%%%%%%%%%%%%%%%%%%%%%%%%%%%%%%%%%%%%%%%%%%%%%

\QUESTBEGIN

\Task \what~\TODO

\SOLUTION

\TaskSolved \what~\TODO

\QUESTEND




\WHAT{Förbered din projektredovisning.} %%%%%%%%%%%%%%%%%%%%%%%%%%%%%%%%%%%%%%%

\QUESTBEGIN

\Task \what~\TODO

\SOLUTION

\TaskSolved \what~\TODO

\QUESTEND



\WHAT{Skapa dokumentation.} %%%%%%%%%%%%%%%%%%%%%%%%%%%%%%%%%%%%%%%%%%%%%%%%%%%

\QUESTBEGIN

\Task  \what~

\Subtask \TODO kör nedan kommando i terminalen:

\begin{REPL}
> scaladoc paket.scala
> ls
> firefox index.html   # eller öppna index.html i valfri webbläsare
\end{REPL}

Vad händer?

\Subtask Lägg till några fler metoder i något av objekten i filen \code{paket.scala} och lägg även till några dokumentationskommentarer. Kompilera om och kör. Generera om dokumentationen.

\begin{verbatim}
//... ändra i filen paket.scala

/** min paketdokumentationskommentar p2 */
package p2 {
  /** min paketdokumentationskommentar p21 */
  package p21 {
    /** ett hälsningsobjekt */
    object hello {
      /** en hälsningsmetod i p2.p21 */
      def hello = println("Hej paket p2.p21!")

      /** en metod som skriver ut tiden */
      def date = println(new java.util.Date)
    }
  }
}

\end{verbatim}

\begin{REPL}
> gedit paket.scala
> scalac paket.scala
> jar cvf mittpaket.jar gurka
> scala -cp mittpaket.jar
scala> gurka.tomat.banan.p2.p21.hello.date
scala> :q
> scaladoc paket.scala
> firefox index.html
\end{REPL}

\SOLUTION


\TaskSolved \what

\SubtaskSolved  -

\SubtaskSolved  -

\QUESTEND



\WHAT{Repetera övningar och laborationer.} %%%%%%%%%%%%%%%%%%%%%%%%%%%%%%%%%%%%

\QUESTBEGIN

\Task \what~\TODO

\SOLUTION

\TaskSolved \what~\TODO

\QUESTEND

%!TEX encoding = UTF-8 Unicode
%!TEX root = ../exercises.tex

\ifPreSolution

\Exercise{\ExeWeekFOURTEEN}\label{exe:W14}

\begin{Goals}
\item Känna till vad en tråd är och kunna förklara begreppet jämlöpande exekvering.
\item Känna till vad metoderna \code{run} och \code{start} gör i klassen \code{Thread}.
\item Kunna skapa och starta en tråd med överskuggad \code{run}-metod.
\item Kunna skapa ett enkelt program som från två trådar tävlar om att uppdatera en variabel och förklara varför beteendet kan bli oförutsägbart.
\item Kunna använda en \code{Future} för att köra igång flera parallella beräkningar.
\item Kunna registrera en callback på en \code{Future} med metoden \code{onComplete}.
%\item Känna till att webbsidor beskrivs av HTML-kod och kunna skapa en minimal webbsida.
%\item Kunna ladda ner en webbsida med \code{scala.io.Source.fromURL}.
\end{Goals}

% \begin{Preparations}
% \item \StudyTheory{14}
% \end{Preparations}

\else

\ExerciseSolution{\ExeWeekFOURTEEN}

\fi


\subsection{Frivilliga extrauppgifter}



\WHAT{Trådar.}

\QUESTBEGIN

\Task  \what~   Klassen \code{java.lang.Thread} används för att skapa  \textbf{trådar} med jämlöpande exekvering \Eng{concurrent execution}. På så sätt kan man få olika koddelar att köra samtidigt.

Klassen \code{Thread} definierar en tom \code{run}-metod. Vill man att tråden ska göra något vettigt får man överskugga \code{run} med det man vill ska göras.

En tråd körs igång med metoden \code{start} och då anropas automatiskt \code{run}-metoden och tråden exekverar koden i \code{run} jämlöpande med övriga trådar. Om man anropar \code{run} direkt blir det \emph{inte} jämlöpande exekvering.

\Subtask Skapa en tråd som gör något som tar lite tid och kör med \code{run} resp. \code{start}.
\begin{REPL}
def zzz = { print("zzzzzz"); Thread.sleep(5000); println(" VAKEN!")}
zzz
val t2 = new Thread{ override def run = zzz }
t2.run
t2.run; println("Gomorron!")
t2.start; println("Gomorron!")
t2.start
\end{REPL}

\Subtask Vad händer om man anropar \code{start} mer än en gång på samma tråd?

\Subtask Skapa två trådar med överskuggade \code{run}-metoder och kör igång dem samtidigt enligt nedan. Vilken ordning skrivs hälsningarna ut efter rad 3 resp. rad 4 nedan? Förklara vad som händer.
\begin{REPL}
val g = new Thread{ override def run = for (i <- 1 to 100) print("Gurka ") }
val t = new Thread{ override def run = for (i <- 1 to 100) print("Tomat ") }
g.run; t.run
g.start; t.start
\end{REPL}

\Subtask Använd \code{Thread.sleep} enligt nedan. Är beteendet helt förutsägbart (deterministiskt)? Förklara vad som händer. Du kan (om du kör Linux) avbryta REPL med Ctrl+C%
\footnote{\href{http://stackoverflow.com/questions/6248884/can-i-stop-the-execution-of-an-infinite-loop-in-scala-repl}{stackoverflow.com/questions/6248884/can-i-stop-the-execution-of-an-infinite-loop-in-scala-repl}}.
\begin{REPL}
def ibland(block: => Unit) = new Thread {
  override def run = while(true) { block; Thread.sleep(600) }
}.start
ibland(print("zzz ")); ibland(print("snark ")); ibland(println("hej!"))
\end{REPL}


\SOLUTION


\TaskSolved \what
     %%%TODO number  1 %%%starts with: \emph{Trådar.}  %%%

\SubtaskSolved   -

\SubtaskSolved  \code {java.lang.IllegalThreadStateException}. Det går inte att starta en tråd mer än en gång. Tråden kan därför inte startas om när den redan har exekverats.

\SubtaskSolved   När \code {start} anropas exekveras koden i \code{run} parallellt. Därför skrivs \code{Gurka} och \code{Tomat} ut omlöpande. Om istället \code{run} anropas direkt blir det inte jämnlöpande exekvering och \code{Gurka} skrivs ut 100 gånger, sedan skrivs \code{Tomat} ut 100 gånger.

\SubtaskSolved   \code{Thread.sleep} pausar inte tråden i exakt den tiden som angets. Alltså kommer det skrivas ut \code{zzz snark hej!} i de flesta fall, men det är inte garanterat.



\QUESTEND






\WHAT{Jämlöpande variabeluppdatering.}

\QUESTBEGIN

\Task \label{task:racecondition} \what~   Skriv klasserna \code{Bank} och \code{Kund} i en editor och klistra sedan in koden i REPL.

\begin{Code}
class Bank {
  private var saldo = 0;
  def visaSaldo: Unit = println("saldo: " + saldo)
  def sättIn: Unit = { saldo += 1 }
  def taUt: Unit   = { saldo -= 1 }
}

class Kund(bank: Bank) {
  def slösaSpara = {bank.taUt; Thread.sleep(1); bank.sättIn}
}
\end{Code}

\Subtask Använd funktionen \code{ibland} från föregående uppgift och kör nedan rader i REPL. Resultatet av jämlöpande variabeluppdatering blir här heltokigt och leder till mycket upprörda bankkunder och -ägare. Förklara vad som händer.

\begin{REPL}
val bank = new Bank
bank.visaSaldo
bank.sättIn
bank.visaSaldo
bank.taUt
bank.visaSaldo

val bamse = new Kund(bank)
val skutt = new Kund(bank)

bamse.slösaSpara
skutt.slösaSpara
bank.visaSaldo

def ofta(block: => Unit) = new Thread {
  override def run = while(true) { block; Thread.sleep(1) }
}.start

ofta(bamse.slösaSpara); ofta(skutt.slösaSpara)

ibland(bank.visaSaldo)
\end{REPL}


\SOLUTION


\TaskSolved \what
     %%%TODO number  2 %%%starts with: \emph{Jämlöpande variabeluppdat%%%

\SubtaskSolved  I \code{slösaSpara} hämtas saldot, ändras och placeras tillbaka i minnet -  fördröjs -  upprepas. Om \code{bamse} blir klar med att ladda, ändra och lagra innan skutt gör detsamma med den muterbara variablen hade det inte varit perfekt. Problemet ligger i  när en tråd laddar och innan den kan lagra det förändrade värdet laddar den andra tråden samma värde. Bara en av dessa trådar vinner racet och får lagra sitt ändrade tal. \code{skutt} och \code{bamse} blir alltså upprörda för att inte alla dess uttag och insättningar registreras.


\QUESTEND






\WHAT{Trådsäkra \code{AtomicInteger}.}

\QUESTBEGIN

\Task  \what~  Det finns stöd i JVM för att åstadkomma uppdateringar som inte kan avbrytas av andra trådar under pågånde minnesskrivning. En operation som inte kan avbrytas kallas \textbf{atomär} \Eng{atomic}. Studera dokumentationen för \code{AtomicInteger}\footnote{\href{https://docs.oracle.com/javase/8/docs/api/java/util/concurrent/atomic/AtomicInteger.html}{docs.oracle.com/javase/8/docs/api/java/util/concurrent/atomic/AtomicInteger.html}} och prova nedan kod. Förklara vad som händer.

Använd funktionerna \code{ofta} och \code{ibland} från tidigare uppgifter.
\begin{Code}
class SäkerBank {
  import java.util.concurrent.atomic.AtomicInteger
  private var saldo = new AtomicInteger
  def visaSaldo: Unit = println(s"saldo: ${saldo.get}")
  def sättIn: Unit = { saldo.incrementAndGet }
  def taUt: Unit   = { saldo.decrementAndGet }
}

class SäkerKund(bank: SäkerBank) {
  def slösaSpara = {bank.taUt; Thread.sleep(1); bank.sättIn}
}
\end{Code}
\begin{REPL}
val säkerBank = new SäkerBank
val farmor = new SäkerKund(säkerBank)
val vargen = new SäkerKund(säkerBank)

ofta(farmor.slösaSpara); ofta(vargen.slösaSpara)

ibland(säkerBank.visaSaldo)
\end{REPL}





\SOLUTION


\TaskSolved \what
     %%%TODO number  3 %%%starts with: \emph{Jämlöpande exekvering med%%%

Nu är \code{farmor} garanterad att kunna ladda saldot, ta ut pengar/ändra och lagra innan \code{vargen} kan överskriva resultatet. I \code{slösaSpara} pausas tråden i en millisekund så \code{vargen} kan fortfarande ta ut pengar innan \code{farmor} hinner sätta in pengar igen. Dock kommer alla uttag och insättningar registreras eftersom operationerna är atomära.


\QUESTEND






\WHAT{Jämlöpande exekvering med \code{scala.concurrent.Future}.}

\QUESTBEGIN

\Task \label{task:future} \what~   Att skapa och hålla reda på trådar kan bli ganska omständligt och knepigt att få rätt på.
Med hjälp av \code{scala.concurrent.Future} kan man på ett enklare sätta skapa jämlöpande exekvering.

\begin{Background}
Med en \code{Future} skapas jämlöpande exekvering som ''under huven'' använder ett ramverk som heter Akka\footnote{\url{http://akka.io/}}, skrivet i Scala och Java. Akka erbjuder automatisk  multitrådning med s.k. trådpooler och möjliggör avancerad parallellprogrammering på en hög  abstraktionsnivå, där man själv slipper skapa instanser av klassen \code{Thread}. I stället kan man helt enkelt placera sin kod inramad med \code|Future{ "körs parallellt" }| efter att man importerat det som behövs.
\end{Background}

\Subtask För att skapa jämlöpande exekvering med \code{Future} behöver man först göra import enligt nedan; då skapas ett exekveringssammanhang med trådpooler redo för användning. Starta om REPL och studera felmeddelandet efter rad 1 nedan. Importera därefter enligt nedan. Vad har \code{f} för typ?
\begin{REPL}
scala> concurrent.Future { Thread.sleep(1000); println("En sekund senare!") }
scala> import scala.concurrent._
scala> import ExecutionContext.Implicits.global
scala> val f = Future { Thread.sleep(1000); println("En sekund senare!") }
\end{REPL}

\Subtask Skapa en procedur \code{printLater} enligt nedan som skriver ut argumentet efter slumpmässig tid. Förklara vad som händer nedan.
\begin{REPL}
scala> def printLater(a: Any): Unit =
         Future { Thread.sleep((math.random * 10000).toInt); print(a + " ") }
scala> (1 to 42).foreach(i => printLater(i)); println("alla är igång!")
\end{REPL}

\Subtask Skapa enligt nedan en \code{Future} som räknar ut hur många siffror det är i ett väldigt stort tal. Med \code{onComplete} kan man ange vad som ska göras när den tunga beräkningen är färdig; detta kallas att ''registrera en callback''. Vilken returtyp har \code{big}? Hur många siffror har det stora talet? Vad har \code{r} för typ? Justera argumentet till \code{big} om du inte orkar vänta på resultatet...

\begin{REPL}
scala> BigInt(10).pow(100)
scala> BigInt(10).pow(100).toString.size
scala> def big(n: Int) = Future { BigInt(n).pow(n).toString.size }
scala> big(1234567).onComplete{r => println(r + " siffror") }
\end{REPL}

\Subtask Den stora vinsten med \code{Future} är att man kan köra vidare under tiden, varför anropet av \code{Future} kallas \textbf{icke-blockerande} \Eng{non-blocking}. Det händer ibland att man ändå vill blockera exekveringen i väntan på ett resultat. Man kan då använda objektet \code{scala.concurrent.Await} och dess metod \code{result} enligt nedan. Använd \code{big} från föregående uppgift och gör en blockerande väntan på resultatet enligt nedan. Vad händer? Vad händer om du väntar för kort tid?

\begin{REPL}
scala> import scala.concurrent.duration._
scala> Await.result(big(1234567), 20.seconds)
\end{REPL}



\SOLUTION


\TaskSolved \what
     %%%TODO number  4 %%%starts with: TODO  %%%%%%%%%%%%%%%%%%%\Advan%%%

\SubtaskSolved  error: Cannot find an implicit ExecutionContext. Future behöver en ExecutionContext för att kunna köras. \code{f} är av typen Future[Unit].

\SubtaskSolved  Funktionen \code{printLater} har en Future, vilket innebär att när både \code{printLater} och \code{println} anropas i foreach-loopen exekveras de jämnlöpande. Eftersom det tar längre tid att starta upp en Future för datorn är \code{println} snabbare och skriver ut att alla är igång först. Sedan skrivs siffrorna från 1 - 42 ut med oregelbundna mellanrum eftersom tråden pausas olika länge.

\SubtaskSolved  \code{big} är en Future[Int]. Det stora talet har 7 520 383 siffror. \code{r} är av typen Try[Int] (se dokumentationen för Future om du är osäker)

\SubtaskSolved  Eftersom exekveringen blockas tills den har fått ett resultat går det inte att fortsätta skriva i REPL medan uträkningen pågår. Väntar man för kort tid får man ett TimeOutException och uträkningen avbryts.


\QUESTEND






\WHAT{Använda \code{Future} för att göra flera saker samtidigt.}

\QUESTBEGIN

\Task  \what~
I denna uppgift ska du ladda ner webbsidor parallellt med hjälp av \code{Future}, så att en nedladdning kan avslutas under tiden en annan dröjer.

\Subtask Koden för en minimal webbsida ser ut som nedan. Du kan beskåda sidan här: \url{http://fileadmin.cs.lth.se/pgk/mini.html} eller skriva in nedan kod i en fil som heter något som slutar på \texttt{.html} och öppna filen i din webbläsare.

\begin{verbatim}
<!DOCTYPE html>
<html>
<body>
HELLO WORLD!
</body>
</html>
\end{verbatim}

\Subtask För att simulera slöa webbservrar kan man ladda ner en sida via sajten \texttt{http://deelay.me/}. Ladda ner ovan sida med 2 sekunders fördröjning:\\
\url{http://deelay.me/2000/http://fileadmin.cs.lth.se/pgk/mini.html}

\Subtask Man kan ladda ner webbsidor med \code{scala.io.Source}. Vad händer nedan? Försök, med ledning av hur \code{delay} beräknas, uppskatta hur lång tid du måste vänta i medeltal, i bästa fall, respektive värsta fall, innan du kan se första webbsidan i vektorn \code{laddningar} nedan?

\begin{REPL}
scala> def ladda(url: String) = scala.io.Source.fromURL(url).getLines.toVector
scala> def slöladda(url: String) = {
         val delay = (math.random * 1000 + 2000).toInt
         val delaySite = s"http://deelay.me/$delay/"
         ladda(delaySite+url)
      }
scala> ladda("http://fileadmin.cs.lth.se/pgk/mini.html")
scala> def seg = slöladda("http://fileadmin.cs.lth.se/pgk/mini.html")
scala> val laddningar = Vector.fill(10)(seg)
scala> laddningar(0)
\end{REPL}

\Subtask Innan vi kan köra igång en \code{Future} så måste vi, som visats i uppgift \ref{task:future} importera den underliggande exekveringsmiljön som är redo att parallelisera ditt program i trådar utan att du själv måste skapa dem. Vad händer nedan?
\begin{REPL}
scala> import scala.concurrent._
scala> import ExecutionContext.Implicits.global
scala> val f = Future{ seg }
scala> f   // kolla om den är klar annars prova igen senare
scala> f
\end{REPL}

\Subtask Ladda indata utan att blockera \Eng{non-blocking input}. Förklara vad som händer nedan.
\begin{REPL}
scala> val nonblock = Future{ Vector.fill(10)(seg) }
scala> nonblock   // kolla igen senare om ej klar
scala> nonblock
\end{REPL}

\Subtask Ladda indata separat i olika parallella trådar. Förklara vad som händer nedan. Kör uttrycket på rad 3 nedan upprepade gånger i snabb följd efter varandra med pil-upp+Enter i REPL.
\begin{REPL}
scala> val para = Vector.fill(10)(Future{ seg })
scala> para
scala> para.map(_.isCompleted)
scala> para.map(_.isCompleted) // studera hur de blir färdiga en efter en
scala> para(0)
\end{REPL}

\Subtask Registrera en callback med metoden \code{onComplete}. Förklara vad som händer nedan.

\begin{REPL}
scala> val action = Vector.fill(10)(Future{ seg })
scala> action(0).onComplete(xs => println(s"ready:$xs"))
scala> // vänta tills laddning på plats 0 är klar
\end{REPL}

\Subtask Registrera en callback för felhantering i händelse av undantag med metoden \code{onFailure}. Förklara vad som händer nedan.
\begin{REPL}
scala> def lycka  = { Thread.sleep(3000); println(":)") }
scala> def olycka = { Thread.sleep(3000); 42 / 0; lycka }
scala> Future{ lycka  }.onFailure{ case e => println(s":( $e") }
scala> Future{ olycka }.onFailure{ case e => println(s":( $e") }
\end{REPL}



\SOLUTION


\TaskSolved \what
     %%%TODO number  5 %%%starts with: Sök upp och studera dokumentati%%%

\SubtaskSolved  -

\SubtaskSolved  -

\SubtaskSolved  Varje sida fördröjs med mellan 2 upp till 3 sekunder (2000-3000 millisekunder). Så i medeltal tar det 2.5 sekunder för varje sida att laddas. Vektorn måste fyllas innan exekveringen kan fortsätta. Därför laddas alla 10 stycken sidor in innan man kan se första websidan. Det tar därför i medeltal 2.5 x 10 = 25 sekunder.

\SubtaskSolved  \code{f} ger en Vektor fylld med strängar där varje element ges av en rad på hemsidan. Då \code{f} körs i bakgrunden kan programmet fortlöpa medan innehållet räknas ut. Du kan därför skriva \code{f} i REPL:n men det är inte säkert att proccessen är klar och det slutgilltiga resultatet visas.

\SubtaskSolved  Samma som ovan, förutom att det blir en vektor där varje element är i sig en vektor med strängar.

\SubtaskSolved  Laddar in datan parallelt så nedladdingen sker samtidigt, men det går olika snabbt pga metoden seg.

\SubtaskSolved  Eftersom datan laddas i parallella trådar utan blockering blir de inte klara i ordning, utan i den ordningen tråden körs klart. Till slut blir alla klara och resultatet visar en vektor med \code{true} värden.

\SubtaskSolved  Metoden \code{lycka} är väldefinerad och kastar därför inga undantag. Den skriver alltid ut \code{:)}. Metoden \code{olycka} är inte väldefinerad då division med 0 ger \code{java.lang.ArithmeticException}. Detta fångas upp vid callbacken och det skrivs ut \code{:(} samt det specifierade undantaget.

\ExtraTasks %%%%%%%%%%%%


\QUESTEND






\WHAT{}

\QUESTBEGIN

\Task  \what~ Räkna ut stora primtal parallellt genom att använda nedan funktioner. Implementera \code{isPrime} enligt pseudokod från den engelska wikipediasidan om primtalstest\footnote{\href{https://en.wikipedia.org/wiki/Primality_test}{en.wikipedia.org/wiki/Primality\_test}} med den s.k. ''naiva algoritmen''.  Räkna ut 10 st slumpvisa primtal med 16 siffror vardera. Gör beräkningarna parallellt med hjälp av \code{Future}.

\begin{Code}
def isPrime(n: BigInt): Boolean = ???

def nextPrime(start: BigInt): BigInt = {
  var i = start
  while (!isPrime(i)) { i += 1 }
  i
}

def randomBigInt(nDigits: Int): BigInt = {
   def rndChar = ('0' + (math.random * 10).toInt).toChar
   val str = Array.fill(nDigits)(rndChar).mkString
   BigInt(str)
}
\end{Code}

\SOLUTION


\TaskSolved \what
  %%%TODO number  6 %%%

\begin{Code}
def isPrime(n: BigInt): Boolean = n match {
  case _ if (n <= 1) => false
  case _ if (n <= 3) => true
  case _ if n % 2 == 0 || n % 3 == 0 => false
  case _ =>
    var i = BigInt(5)
    while (i * i < n) {
      if (n % i == 0 || n % (i + 2) == 0) false
      i += 6
    }
    true
}

import scala.concurrent._
import ExecutionContext.Implicits.global

val primes = Vector.fill(10)(Future{nextPrime(randomBigInt(16))})
primes.foreach(_.onSuccess{case i => println(i)})
\end{Code}


\QUESTEND






\WHAT{Svara på teorifrågor.}

\QUESTBEGIN

\Task  \what~\Pen

\Subtask Vad är en tråd?

\Subtask Hur skapar man en tråd med klassen \code{Thread}?

\Subtask Hur startar man en tråd?

\Subtask Vilka problem kan man råka ut för om man uppdaterar samma resurs i flera olika trådar?

\Subtask Vad innbär det att kod är \emph{trådsäker}?

\Subtask Nämn några fördelar med att använda Future jämfört med att använda trådar direkt.


\SOLUTION


\TaskSolved \what
 %%%TODO number  7 %%%

\SubtaskSolved  Stackoverflow ger följande förklaring:

A thread is an independent set of values for the processor registers (for a single core). Since this includes the Instruction Pointer (aka Program Counter), it controls what executes in what order. It also includes the Stack Pointer, which had better point to a unique area of memory for each thread or else they will interfere with each other.

\SubtaskSolved

\begin{Code}
val thread = new Thread(new Runnable{
	def run(){println(''Det här är en tråd'')}
})
\end{Code}

\SubtaskSolved  \code{thread.start}

\SubtaskSolved  Det kan bli kapplöpning(race conditions) om vilken tråds resurser blir sparade. Vilket leder till att de andra trådarnas ändringar blir ignorerade.

\SubtaskSolved  Trådsäkerhet innebär att flera trådar kan köras parallellt utan felaktigheter i resultatet. Exempelvis får man vara väldigt försiktig om man vill ha en muterbar variabel som alla trådar ska ändra samtidigt.

\SubtaskSolved  Till exempel slipper man skapa instanser av klassen Thread eftersom man kan placera koden i en Future istället. Den löser även mycket under huven för kodaren.


\QUESTEND






\WHAT{Klasser med atomär uppdatering.}

\QUESTBEGIN

\Task  \what~ Läs om och testa klasserna AtomicBoolean, AtomicDouble och AtomicReference för atomär uppdatering i paketet \\ \code{java.util.concurrent.atomic}.

Använd några av dessa tillsammans med \code{scala.concurrent.Future}.


\SOLUTION

\TaskSolved --

\QUESTEND





\WHAT{Skapa din egen multitrådade webbserver.}

\QUESTBEGIN

\Task  \what~

\Subtask Skriv in\footnote{Eller ladda ner här: \href{https://github.com/lunduniversity/introprog/blob/master/compendium/examples/simple-web-server/webserver.scala}{github.com/lunduniversity/introprog/blob/master/compendium/examples/simple-web-server/webserver.scala}} nedan kod i en editor och spara i en fil med namn \texttt{webserver.scala} och kompilera och kör med \texttt{scala webserver.start} och beskriv vad som händer när du med din webbläsare surfar till adressen: \\ \url{http://localhost:8089/abbasillen}

\scalainputlisting[numbers=left,basicstyle=\ttfamily\fontsize{11}{12}\selectfont]{examples/simple-web-server/webserver.scala}

\Subtask Du ska nu skapa en webbserver som gör något lite mer intressant. Den ska svara med det 13:e Fibonacci-talet\footnote{\href{https://sv.wikipedia.org/wiki/Fibonaccital}{https://sv.wikipedia.org/wiki/Fibonaccital}} om du surfar till \url{http://localhost:8089/fib/13}.
Spara din webbserver från föregående deluppgift under det nya namnet \texttt{fibserver.scala} och använd koden nedan och lägg till och ändra så att din server kan svara med Fibonaccital. Vi börjar med att räkna ut Fibonaccital i funktionen \code{compute.fib} nedan på ett onödigt processorkrävande sätt med exponentiell tidskomplexitet så att webbservern verkligen får jobba, för att i senare deluppgifter implementera \code{compute.fib} med linjär tidskomplexitet och därmed undvika onödig planetuppvärmning.
\begin{CodeSmall}
  object compute {
    def fib(n: BigInt): BigInt = {
      if (n < 0) 0 else
      if (n == 1 || n == 2) 1
      else fib(n - 1) + fib(n -2)
    }
  }

  def fibResponse(num: String) = Try { num.toInt } match {
    case Success(n) => html.page(s"fib($n) == " + compute.fib(n))
    case Failure(e) => html.page(s"FEL $e: skriv heltal, inte $num")
  }

  def errorResponse(uri:String) = html.page("FATTAR NOLL: " + uri)

  def handleRequest(cmd: String, uri: String, socket: Socket): Unit = {
    val os = socket.getOutputStream
    val parts = uri.split('/').drop(1) // skip initial slash
    val response: String = (parts.head, parts.tail) match {
      case (head, Array(num)) => fibResponse(num)
      case _                  => errorResponse(uri)
    }
    os.write(html.header(response.size).getBytes("UTF-8"))
    os.write(response.getBytes("UTF-8"))
    os.close
    socket.close
  }
\end{CodeSmall}
Kör i terminalen med \texttt{scala fibserver.start} och beskriv vad som händer i din webbläsare när du surfar till servern.


%%%\textbf{KOD TILL FACIT:}
%%%\scalainputlisting[numbers=left,basicstyle=\ttfamily\fontsize{11}{12}\selectfont]{examples/simple-web-server/fibserver.scala}


\Subtask Surfa efter flera stora Fibonacci-tal samtidigt i olika flikar i din browser. Hur märks det att servern bara kör i en enda tråd?

\Subtask Gör din server multitrådad med hjälp av den nya server-loopen nedan.

\begin{CodeSmall}
import scala.concurrent._
import ExecutionContext.Implicits.global

  def serverLoop(server: ServerSocket): Unit = {
    println(s"http://localhost:${server.getLocalPort}/hej")
		while (true) {
  		Try {
  		  var socket = server.accept  // blocks thread until connect
	  	  val scan = new Scanner(socket.getInputStream, "UTF-8")
		    val (cmd, uri) = (scan.next, scan.next)
			  println(s"Request: $cmd $uri")
		    Future { handleRequest(cmd, uri, socket) }.onFailure {
		      case e => println(s"Reqest failed: $e")
		    }
		  }.recover{ case e: Throwable => s"Connection failed: $e" }
		}
  }
\end{CodeSmall}

\Subtask Surfa efter flera stora Fibonacci-tal samtidigt i olika flikar i din browser. Hur märks det att servern är multitrådad?


\Subtask Det är onödigt att räkna ut samma Fibonacci-tal flera gånger. Med hjälp av en cache i form av en föränderlig \code{Map} kan du spara undan redan uträknade värden. Det funkar dock inte med en vanlig \code{scala.collection.mutable.Map} i vår multitrådade webbserver, eftersom den inte är \textbf{trådsäker} \Eng{thread-safe}. Med trådosäkra föränderliga datastrukturer blir det samma besvärliga beteende som i uppgift \ref{task:racecondition}.

Du ska i stället använda \code{java.util.concurrent.ConcurrentHashMap}. Sök upp  dokumentationen för \code{ConcurrentHashMap} och försök förstå koden nedan. Hur fungerar metoderna \code{containsKey}, \code{put} och \code{get}?
\begin{Code}
object compute {
  import java.util.concurrent.ConcurrentHashMap
  val memcache = new ConcurrentHashMap[BigInt, BigInt]

  def fib(n: BigInt): BigInt =
    if (memcache.containsKey(n)) {
      println("CACHE HIT!!! no need to compute: " + n)
      memcache.get(n)
    } else {
      println("cache miss :( must compute fib:  " + n)
      val f = fastFib(n)
      memcache.put(n, f)
      f
    }

  private def fastFib(n: BigInt): BigInt = {
    if (n < 0) 0 else
    if (n == 1 || n == 2) 1
    else fib(n - 1) + fib(n -2)
  }
}
\end{Code}

\Subtask Använd ovan \code{fib}-objekt i en ny version av din webserver. Spara den i en ny kodfil med namnet \texttt{fibserver-memcached.scala}. Undersök hur snabbt det går med stora Fibonaccital med den nya varianten. Hur stora tal kan du räkna ut? Kan servern fortsätta efter överflödad stack? Förklara varför.

\Subtask Nu när vi kan få väldigt stora Fibonacci-tal kan det vara användbart att stoppa in radbrytningar på webbsidan. Html-taggen \texttt{</br>} ger en radbrytning.
\begin{Code}
  def insertBreak(s: String, n: Int = 80): String = {
    if (s.size < n) s
    else s.take(n) + "</br>" + insertBreak(s.drop(n),n)
  }
\end{Code}
Använd den rekursiva funktionen ovan för att pilla in radbrytningstaggar på var $n$:te position i långa strängar. Testa hur det ser ut på webbsidan med ovan funktion när din server svarar med väldigt stora tal.

\Subtask Vi ska nu använda det större heap-minnet i stället för stack-minnet och därmed inte begränsas av stackens max-storlek. Skriv om \code{fastFib} så att den använder en \code{while}-sats i stället för ett rekursivt anrop. Denna uppgift är ganska klurig, men om du kör fast kan du snegla i lösningarna i Appendix för inspiration.

Hur stora tal klarar din server nu? Vad händer med servern när minnet tar slut? Hur kan du skydda servern så att den inte kan hänga sig?

\SOLUTION


\TaskSolved \what
 %%%TODO number  9 %%%

\SubtaskSolved  \code{abbasillen} skrivs ut baklänges till \code{nellisabba}.

\SubtaskSolved

\SubtaskSolved

\SubtaskSolved

\SubtaskSolved

\SubtaskSolved

\SubtaskSolved

\SubtaskSolved

\SubtaskSolved

Lösningsförslag:
\scalainputlisting[numbers=left,basicstyle=\ttfamily\fontsize{11}{12}\selectfont]{examples/simple-web-server/fibserver-threaded-memcached-while.scala}


\QUESTEND






\WHAT{}

\QUESTBEGIN

\Task  \what~ Utöka din server med fler beräkningsintensiva funktioner. Exempelvis primtalsberäkningar eller beräkningar av valfritt antal decimaler av $\pi$ eller $e$. Utnyttja gärna det du lärt dig i  matematiken om summor och serieutvecklingar.

\SOLUTION


\TaskSolved \what
 %%%TODO number  10 %%%

---


\QUESTEND






\WHAT{}

\QUESTBEGIN

\Task  \what~ Läs mer om \code{Future} och jämlöpande exekvering i Scala här:\\
\href{http://alvinalexander.com/scala/future-example-scala-cookbook-oncomplete-callback}{alvinalexander.com/scala/future-example-scala-cookbook-oncomplete-callback}

\SOLUTION


\TaskSolved \what
 %%%TODO number  11 %%%

---


\QUESTEND






\WHAT{}

\QUESTBEGIN

\Task  \what~ Läs mer om jämlöpande exekvering och multitrådade program i Java här: \href{http://www.tutorialspoint.com/java/java_multithreading.htm}{www.tutorialspoint.com/java/java\_multithreading.htm}  \\
\noindent När man skriver program med jämlöpande exekvering finns det många fallgropar; det kan bli kapplöpning \Eng{race conditions} om gemensamma resurser och dödläge \Eng{deadlock} där inget händer för att trådar väntar på varandra. Mer om detta i senare kurser.


\SOLUTION


\TaskSolved \what
 %%%TODO number  12 %%%

---


\QUESTEND






\WHAT{Studera dokumentationen i \code{scala.concurrent}.}

\QUESTBEGIN

\Task  \what~\Pen

\Subtask Studera dokumentationen för \code{scala.concurrent.Future}\footnote{\href{http://www.scala-lang.org/files/archive/api/current/\#scala.concurrent.Future}{http://www.scala-lang.org/files/archive/api/current/\#scala.concurrent.Future}}. Hur samverkar \code{Future} med \code{Try} och \code{Option}? Vilka vanliga samlingsmetoder känner du igen?

\Subtask Studera dokumentationen för \code{scala.concurrent.duration.Duration}\footnote{\href{http://www.scala-lang.org/api/current/\#scala.concurrent.duration.Duration}{www.scala-lang.org/api/current/\#scala.concurrent.duration.Duration}}. Vilka tidsenheter kan användas?

\Subtask Vid import av \code{scala.concurrent.duration._ } dekoreras de numeriska klasserna med metoder för att skapa instanser av klassen \code{Duration}. Detta möjligörs med hjälp av klassen \code{scala.concurrent.duration.DurationConversions}. Studera dess dokumentation och testa att i REPL skapa några tidsperioder med metoderna på \code{Int}.



\SOLUTION


\TaskSolved \what
 %%%TODO number  13 %%%

\SubtaskSolved

\SubtaskSolved

\SubtaskSolved


\QUESTEND






\WHAT{}

\QUESTBEGIN

\Task  \what~ Fördjupa dig inom webbteknologi.

\Subtask Lär dig om HTML, CSS och JavaScript här: \url{https://developer.mozilla.org/en-US/docs/Learn}

\Subtask Lär dig om Scala.JS här: \url{http://www.scala-js.org/}\SOLUTION


\TaskSolved \what
 %%%TODO number  14 %%%

\SubtaskSolved  ---

\SubtaskSolved  ---

\SubtaskSolved  ---

\SubtaskSolved  ---
\QUESTEND


\end{document}
