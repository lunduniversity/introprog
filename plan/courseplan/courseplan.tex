\documentclass[a4paper,12pt,oneside]{memoir}
\usepackage[a4paper, total={16.2cm, 25.0cm}]{geometry}
\usepackage[utf8]{inputenc}
\usepackage{graphicx}
\usepackage[T1]{fontenc}
\usepackage[swedish]{babel}

\usepackage{microtype}
\usepackage{hyperref}
\usepackage{longtable}
\usepackage{booktabs}

% FONTS
\usepackage{tgtermes}

\usepackage{enumitem}
\setitemize{noitemsep,topsep=0pt,parsep=0pt,partopsep=0pt, leftmargin=*}

\pagenumbering{gobble}

\usepackage{url}
\usepackage{color}
\newcommand{\TBD}{\colorbox{yellow}{\textbf{???}}}
\newcommand{\TENTADATUM}{\colorbox{yellow}{8:e Januari, kl 08:00--13:00, se schema}}
\newcommand{\KSDATUM}{\colorbox{yellow}{Tisdagen 24:e Oktober, kl 14:00--19:00, se schema}}
\newcommand{\YEAR}{2021}

\begin{document}

\section*{EDAA45 Programmering, grundkurs  -- Kursprogram \YEAR}
\emph{Institutionen för Datavetenskap, LTH, Lunds Universitet.}\\

\begin{longtable}[l]{ll}
\toprule
\textbf{EDAA45} & \textit {D1, W3, 7,5 högskolepoäng, Läsperiod 1 \& 2} \tabularnewline
\midrule
\endhead
\emph{Kursansvarig}   & Björn Regnell, rum E:2413,
                        \href{mailto:bjorn.regnell@cs.lth.se}
                        {\nolinkurl{bjorn.regnell@cs.lth.se}},
                        046--222 90 09\tabularnewline
\emph{Hemsida}        & \url{http://cs.lth.se/pgk}\tabularnewline
\emph{Kurslitteratur} & Kompendium. Säljes på institutionens expedition efter förbeställning.\tabularnewline
\emph{Expedition}     & \url{http://cs.lth.se/kontakt/expedition/}
                        \tabularnewline
                      & Rum E:2179, expeditionstid Mån-Tor kl. 9.30--11.30, 12.45--13.30\tabularnewline

\bottomrule
\end{longtable}

\subsection{Undervisning}\label{undervisning}

\begin{itemize}
\item
  \emph{Föreläsningar}. Föreläsningarna ger en översikt av
  kursinnehållet och åskådliggör teorin med praktiska
  programmeringsexempel. Föreläsningarna ger även utrymme för diskussion
  och frågor.
\item
  \emph{Resurstider}. I kursens schema finns särskilda resurstider
  där du kan få hjälp med övningar, laborationer och
  inlämningsuppgifter. Utnyttja dessa tillfällen!
\item
  \emph{Övningar}. I kursen ingår övningar som du arbetar med
  självständigt eller tillsammans med en kamrat.
  Du kan få hjälp med övningarna av handledare under resurstiderna.
  Övningarna är förberedelser inför laborationerna och den skriftliga tentamen.
  Se anvisningar i kompendiet.
\item
  \emph{Laborationer}. I kursen ingår obligatoriska laborationer.
  Laborationerna redovisas för handledare.
  Se anvisningar i kompendiet.
\item
  \emph{Projektuppgift}. Du ska självständigt arbeta med ett större
  program och redovisa detta för en handledare. Se anvisningar i
  kompendiet.
\end{itemize}

\subsection{Samarbetsgrupper}\label{samarbetsgrupper}

Kursdeltagarna indelas i \emph{samarbetsgrupper} av kursansvarig baserat
på förkunskapsenkät, där studenter med olika förkunskapsnivåer
sammanförs. Målet med samarbetsgrupperna är att deltagarna gemensamt ska
dela med sig av och träna på förklaringar av teori, begrepp och
programmeringspraktik. Kontrollskrivningen kan ge samarbetsbonus (se
nedan) och en av laborationerna görs i grupp. Ni ska hjälpa varandra att
förstå, men \emph{inte} lösa uppgifterna åt varandra.

\subsection{Examination}\label{examination}

\begin{itemize}
\item
  \emph{Obligatoriska kursmoment (4,5 hp):}

  \begin{itemize}
  \item
    \emph{Laborationer} godkänns av handledare på schemalagd tid.
  \item
    \emph{Kontrollskrivningen} är diagnostisk och visar ditt kunskapsläge efter
    halva kursen. Kontrollskrivningen görs individuellt och rättas
    därefter av studiekamrater vid skrivningstillfället.
    Kontrollskrivningen kan ge \emph{samarbetsbonus} som adderas till
    det skriftliga tentamensresultatet vid första ordinarie
    tentatillfälle med medelvärdet av gruppmedlemmarnas individuella
    kontrollskrivningspoäng. %Datum: \KSDATUM
  \item
    \emph{Projektuppgift} görs individuellt och godkänns av handledare på
    schemalagd tid.
  \end{itemize}

\item
  \emph{Tentamen (3 hp)}. Tentamen är skriftlig. Tillåtet hjälpmedel:
  \href{http://cs.lth.se/pgk/quickref}{Snabbreferens}. \\
  För att få tentera krävs att samtliga laborationer och inlämningsuppgift är godkända.\\
  % Ordinarie tentamen: \TENTADATUM
\end{itemize}

\clearpage

\subsection*{Veckoöversikt}

\resizebox{\columnwidth}{!}{%
{\fontsize{12pt}{24pt}\selectfont
%!TEX encoding = UTF-8 Unicode
\begin{tabular}{l|l|l|l|l|l|l}
\textit{W} & \textit{Datum} & \textit{Lp V} & \textit{Modul} & \textit{Förel} & \textit{Övn} & \textit{Lab} \\ \hline \hline
W01 & 1/9-5/9 & Lp1V1 & Introduktion & F01 F02 & expressions & kojo \\
W02 & 8/9-12/9 & Lp1V2 & Program och kontrollstrukturer & F03 F04 & programs & -- \\
W03 & 15/9-19/9 & Lp1V3 & Funktioner och abstraktion & F05 F06 & functions & irritext \\
W04 & 22/9-26/9 & Lp1V4 & Objekt och inkapsling & F07 F08 & objects & blockmole \\
W05 & 29/9-3/10 & Lp1V5 & Klasser och datamodellering & F09 F10 & classes & blockbattle0 \\
W06 & 6/10-10/10 & Lp1V6 & Mönster och felhantering & F11 F12 & patterns & blockbattle1 \\
W07 & 13/10-17/10 & Lp1V7 & Sekvenser och enumerationer & F13 F14 & sequences & shuffle \\
TP & -- & TP1 & -- & -- & -- & -- \\
W08 & 3/11-7/11 & Lp2V1 & Nästlade och generiska strukturer & F15 F16 & matrices & life \\
W09 & 10/11-14/11 & Lp2V2 & Mängder och tabeller & F17 F18 & lookup & words \\
W10 & 17/11-21/11 & Lp2V3 & Arv och komposition & F19 F20 & inheritance & snake0 \\
W11 & 24/11-28/11 & Lp2V4 & Varians och kontextparametrar & F21 F22 & context & snake1 \\
W12 & 1/12-5/12 & Lp2V5 & Fördjupning, Projekt & F23 F24 & extra & Projekt0 \\
W13 & 8/12-12/12 & Lp2V6 & Repetition & F25 F26 & examprep & Projekt1 \\
W14 & 15/12-19/12 & Lp2V7 & MUNTLIGT PROV & -- & Munta & Munta \\
TP & 7/1 & TP2 & VALFRI TENTAMEN & -- & -- & -- \\
\end{tabular}

}
}

\end{document}
