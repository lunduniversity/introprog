W01 & Introduktion & sekvens, alternativ, repetition, abstraktion, programmeringsspråk, programmeringsparadigmer, editera, kompilera, exekvera, datorns delar, virtuell maskin, REPL, literal, värde, uttryck, identifierare, variabel, typ, tilldelning, namn, val, var, definera funktion, def, anropa funktion, funktionshuvud, funktionskropp, procedur, inbyggda grundtyper, Int, Long, Short, Double, Float, Byte, Char, String, println, typen Unit, enhetsvärdet (), stränginterpolatorn s, if, else, true, false, MinValue, MaxValue, aritmetik, slumptal, math.random, logiska uttryck, de Morgans lagar, while-sats, for-sats \\
W02 & Program & kompilerad app, skript, main i Scala, scalac, main i Java, javac, println, scala.io.StdIn.readLine, java.lang.System.out.println, java.util.Scanner ???, iterera över element i samling, for-uttryck, yield, map, foreach, indexering, Array, Vector, intervall, Range, algoritm vs implementation, pseudokod, algoritm: SWAP, algoritm: SUM, algoritm: MIN/MAX, algoritm: MININDEX \\
W03 & Funktioner & parameter, argument, returtyp, värdeandrop, namnanrop, default-argument, namngivna argument, parameterlista, funktionshuvud, funktionskropp, tupel, multipla returvärden, block, lokala variabler, skuggning, lokala funktioner, applicera funktion på alla element i en samling, uppdelad parameterlista, skapa egen kontrollstruktur, funktionsvärde, funktionstyp, äkta funktion, stegad funktion, apply, anonyma funktioner, lambda, aktiveringspost, anropsstacken, objektheapen, funktioner är objekt med apply-metod, rekursion, scala.util.Random, slumptalsfrö \\
W04 & Objekt & modul, singelobjekt, paket, punktnotation, tillstånd, medlem, attribut, metod, paket, import, filstruktur, jar, dokumentation, programlayout, JDK, import, selektiv import, namnbyte vid import, namnrymd, synlighet, privata medlemmar, inkapsling, överlagring av metoder, cslib.window.SimpleWindow, initialisering, lazy val, enkelt bash-skript för kompilering ???här eller i vecka 2???, sbt tilde run ???här eller i vecka2??? \\
W05 & Klasser & objektorientering, klass, instans, Point, Square, Complex, Any, isInstanceOf, toString, case-klass, new, null, this, accessregler, private, private[this], kompanjonsobjekt, getters och setters, principen om uniform access, klassparameter, primär konstruktor, fabriksmetod, alternativ konstruktor ???, referenslikhet, innehållslikhet, eq, ==, förändringsbar, oföränderlig, java.util.Random \\
W06 & Sekvenser & samling, översikt samlingsbibliotek och samlingsmetoder, scala.collection, Traversable, Iterable, Seq, List, Vector, ArrayBuffer, typparameter, generisk samling som parameter, sekvensalgoritm, algoritm: SEQ-COPY, in-place vs copy, algoritm: SEQ-REVERSE, algoritm: SEQ-REGISTER, sekvenser i Java vs Scala, for-sats i Java, java.util.Scanner, översikt strängmetoder, StringBuilder, ordning, inbyggda sorteringsmetoder, sorted, sortWith, sortBy \\
W07 & Mängder, tabeller & Set, Map, mutable.Set, mutable.Map, hash code, java.util.HashMap, java.util.HashSet, läsa/skriva textfiler, Source.fromFile, java.nio.file, repetition inför kontrollskrivning \\
KS & \multicolumn{2}{l}{KONTROLLSKRIVN.} \\
W08 & Matriser, typparametrar & matris, nästlad samling, nästlad for-sats, typparameter, generisk funktion, generisk klass, fri vs bunden typparameter, matriser i Java vs Scala, allokering av nästlade arrayer i Scala och Java \\
W09 & Arv & arv, polymorfism, trait, extends, asInstanceOf, with, inmixning, supertyp, subtyp, bastyp, override, klasshierarkin i Scala: Any AnyRef Object AnyVal Null Nothing, referenstyper vs värdetyper, klasshierarkin i scala.collection, Shape som bastyp till Rectangle och Circle, accessregler vid arv, protected, final, klass vs trait, abstract class, case-object, typer med uppräknade värden, gränssnitt, trait vs interface, programmeringsgränssnitt (api) \\
W10 & Mönster, undantag, likhet & mönstermatchning, match, Option, throw, try, catch, Try, unapply, sealed, flatten, flatMap, partiella funktioner, collect, speciella matchningar: wildcard pattern; variable binding; sequence wildcard; back-ticks, equals, hashcode, exempel: equals för klassen Complex, switch-sats i Java \\
W11 & Scala och Java & syntaxskillnader mellan Scala och Java, klasser i Scala vs Java, referensvariabler vs enkla värden i Java, referenstilldelning vs värdetilldelning i Java, alternativ konstruktor i Scala och Java, for-sats i Java, for-each-sats i Java, java.util.ArrayList, autoboxing i Java, primitiva typer i Java, wrapperklasser i Java, samlingar i Java vs Scala, scala.collection.JavaConverters, namnkonventioner för konstanter \\
W12 & Sökning, sortering, ordning & strängjämförelse, compareTo, implicit ordning, linjärsökning, binärsökning, algoritm: LINEAR-SEARCH, algoritm: BINARY-SEARCH, algoritmisk komplexitet, sortering till ny vektor, sortering på plats, insättningssortering, urvalssortering, algoritm: INSERTION-SORT, algoritm: SELECTION-SORT, Ordering[T], Ordered[T], Comparator[T], Comparable[T] \\
W13 & \multicolumn{2}{l}{Repetition, tentaträning, projekt} \\
W14 & Extra: jämlöpande exekvering & tråd, jämlöpande exekvering, icke-blockerande anrop, callback, java.lang.Thread, java.util.concurrent.atomic.AtomicInteger, scala.concurrent.Future, kort om html+css+javascript+scala.js och webbprogrammering \\
T & \multicolumn{2}{l}{TENTAMEN} \\