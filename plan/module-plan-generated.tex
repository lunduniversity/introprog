W01 & Introduktion & sekvens, alternativ, repetition, abstraktion, programmeringsspråk, programmeringsparadigmer, editera-kompilera-exekvera, datorns delar, virtuell maskin, REPL, literal, värde, uttryck, identifierare, variabel, typ, tilldelning, namn, val, var, def, inbyggda grundtyper, Int, Long, Short, Double, Float, Byte, Char, String, println, typen Unit, enhetsvärdet (), stränginterpolatorn s, if, else, true, false, MinValue, MaxValue, aritmetik, slumptal, math.random, logiska uttryck, de Morgans lagar, while-sats, for-sats \\
W02 & Kodstrukturer & iterering, for-uttryck, map, foreach, Range, Array, Vector, algoritm vs implementation, pseudokod, algoritm: SWAP, algoritm: SUM, algoritm: MIN/MAX, algoritm: MININDEX, block, namnsynlighet, namnöverskuggning, lokala variabler, paket, import, filstruktur, jar, dokumentation, programlayout, JDK, main i Java vs Scala, java.lang.System.out.println \\
W03 & Funktioner, objekt & definera funktion, anropa funktion, parameter, returtyp, värdeandrop, namnanrop, default-argument, namngivna argument, applicera funktion på alla element i en samling, procedur, värdeanrop vs namnanrop, uppdelad parameterlista, skapa egen kontrollstruktur, objekt, modul, punktnotation, tillstånd, metod, medlem, funktionsvärde, funktionstyp, äkta funktion, stegad funktion, apply, lazy val, lokala funktioner, anonyma funktioner, lambda, aktiveringspost, anropsstacken, objektheapen, rekursion, cslib.window.SimpleWindow \\
W04 & Datastrukturer & attribut (fält), medlem, metod, tupel, klass, Any, isInstanceOf, toString, case-klass, samling, scala.collection, föränderlighet vs oföränderlighet, List, Vector, Set, Map, typparameter, generisk samling som parameter, översikt samlingsmetoder, översikt strängmetoder, läsa/skriva textfiler, Source.fromFile, java.nio.file \\
W05 & Sekvensalgoritmer & sekvensalgoritm, algoritm: SEQ-COPY, in-place vs copy, algoritm: SEQ-REVERSE, algoritm: SEQ-REGISTER, sekvenser i Java vs Scala, for-sats i Java, java.util.Scanner, scala.collection.mutable.ArrayBuffer, StringBuilder, java.util.Random, slumptalsfrö \\
W06 & Klasser & objektorientering, klass, Point, Square, Complex, new, null, this, inkapsling, accessregler, private, private[this], kompanjonsobjekt, getters och setters, klassparameter, primär konstruktor, objektfabriksmetod, överlagring av metoder, referenslikhet vs strukturlikhet, eq vs == \\
W07 & Arv & arv, polymorfism, trait, extends, asInstanceOf, with, inmixning, supertyp, subtyp, bastyp, override, klasshierarkin i Scala: Any AnyRef Object AnyVal Null Nothing, referenstyper vs värdetyper, klasshierarkin i scala.collection, Shape som bastyp till Rectangle och Circle, accessregler vid arv, protected, final, klass vs trait, abstract class, case-object, typer med uppräknade värden \\
KS & \multicolumn{2}{l}{KONTROLLSKRIVN.} \\
W08 & Repetition, trösklar, luckor & REBOOT CAMP: identifiera dina egna lärandetrösklar och kunskapsluckor, kom-i-kapp med övningar och labbar, repetera, fördjupning för de som är redo, specialträning för behövande \\
W09 & Mönster, undantag & mönstermatchning, match, Option, throw, try, catch, Try, unapply, sealed, flatten, flatMap, partiella funktioner, collect, speciella matchningar: wildcard pattern; variable binding; sequence wildcard; back-ticks, equals, hashcode, exempel: equals för klassen Complex, switch-sats i Java \\
W10 & Matriser, typparametrar & matris, nästlad samling, nästlad for-sats, typparameter, generisk funktion, generisk klass, fri vs bunden typparameter, matriser i Java vs Scala, allokering av nästlade arrayer i Scala och Java \\
W11 & Sökning, sortering & strängjämförelse, compareTo, implicit ordning, linjärsökning, binärsökning, algoritm: LINEAR-SEARCH, algoritm: BINARY-SEARCH, algoritmisk komplexitet, sortering till ny vektor, sortering på plats, insättningssortering, urvalssortering, algoritm: INSERTION-SORT, algoritm: SELECTION-SORT, Ordering[T], Ordered[T], Comparator[T], Comparable[T] \\
W12 & Scala och Java & syntaxskillnader mellan Scala och Java, klasser i Scala vs Java, referensvariabler vs enkla värden i Java, referenstilldelning vs värdetilldelning i Java, alternativ konstruktor i Scala och Java, for-sats i Java, for-each-sats i Java, java.util.ArrayList, autoboxing i Java, primitiva typer i Java, wrapperklasser i Java, samlingar i Java vs Scala, scala.collection.JavaConverters, namnkonventioner för konstanter \\
W13 & Extra: design, api, trådar, webb & utvecklingsprocessen, krav-design-implementation-test, gränssnitt, trait vs interface, programmeringsgränssnitt (api), designexempel, tråd, jämlöpande exekvering, icke-blockerande anrop, callback, java.lang.Thread, java.util.concurrent.atomic.AtomicInteger, scala.concurrent.Future, kort om html+css+javascript+scala.js och webbprogrammering \\
W14 & \multicolumn{2}{l}{Tentaträning} \\
T & \multicolumn{2}{l}{TENTAMEN} \\