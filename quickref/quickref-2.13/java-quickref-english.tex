\documentclass[article, a5paper]{memoir}

\let\footruleskip\undefined\usepackage{fancyhdr}% http://ctan.org/pkg/fancyhdr

\usepackage{pgfpages}
\pgfpagesuselayout{resize to}[a4paper]

% Swedish.
\usepackage[utf8]{inputenc}
\usepackage[T1]{fontenc}
%\usepackage[swedish]{babel}
\usepackage{microtype}

%%% FONT PACKAGES
%\usepackage[sc]{mathpazo}
%\usepackage[varg]{txfonts}
%\usepackage{times}
%\usepackage{tgtermes}% clone of times
%\usepackage[sfdefault,condensed]{cabin}
\usepackage{PTSansNarrow}\renewcommand*\familydefault{\sfdefault}
%\usepackage{tgcursor}
\usepackage[scaled=0.85]{beramono} % inconsolata or beramono ???
%\usepackage{fouriernc} % serif: new century schoolbook
%\usepackage{avant}     % sans serif: Avant Garde


% Typeblock size, margins.
\settypeblocksize{190mm}{127mm}{*}
\setlrmargins{10.5mm}{*}{*}
\setulmargins{10.0mm}{*}{*}
\setheadfoot{0.1pt}{0.1pt}
\checkandfixthelayout

\usepackage{multicol} \setlength{\columnsep}{5mm}
\usepackage{xcolor}
\usepackage{array}

\definecolor{commentgreen}{rgb}{0,0.4,0}
\definecolor{grammarcolor}{rgb}{0.3,0.6,0.1}
\definecolor{mylinkcolor}{rgb}{0,0.1,0.5}
\definecolor{myemphcolor}{rgb}{0,0.4,0.1}
\definecolor{myalertcolor}{rgb}{0.4,0.1,0}
\definecolor{eclipsepurple}{rgb}{0.5,0,0.25}
\definecolor{eclipseblue}{rgb}{0.16,0,1.0}
\definecolor{eclipsegreen}{rgb}{0,0.5,0}


\newcommand{\OptL}{\textbf{\textcolor{grammarcolor}{~[~}}}
\newcommand{\OptR}{\textbf{\textcolor{grammarcolor}{~]~}}}
\newcommand{\RepL}{\textbf{\textcolor{grammarcolor}{~(~}}}
\newcommand{\RepR}{\textbf{\textcolor{grammarcolor}{~)~}}}
\newcommand{\Or}{\textbf{\textcolor{grammarcolor}{~|~}}}


%---------------------------------------------------------------

\newcommand{\LangColor}{red}

\setlength{\parindent}{0pt}
\raggedright
\raggedbottom
\linespread{0.90}\selectfont
\pagestyle{empty}

\newcommand{\mc}[1]{\multicolumn{2}{l}{\hspace{-0.65em}\parbox[t]{102mm}{\small #1}}}

\newcommand{\ind}{\hspace*{1.5em}}

\newcommand{\head}[1]{{\bfseries {\color{\LangColor}{#1}}\par\vspace{1mm}\hrule\vspace{-2mm}}}

\newenvironment{etab}%
{\begin{ctabular}{@{}>{\raggedright\small}p{25mm} @{}>{\raggedright\small}p{45mm} @{}>{\raggedright\arraybackslash\small}p{57mm}}}
{\end{ctabular}}%


\newcommand{\secend}{\\[1mm]}
\newcommand{\subsecend}{\\ \\[-2mm]}
\renewcommand{\arraystretch}{0.9}

% -----------
\usepackage{tikz}
\usetikzlibrary{calc}
\usetikzlibrary{shapes.geometric, shapes.symbols, arrows, matrix, shapes, positioning}
%https://www.sharelatex.com/blog/2013/08/29/tikz-series-pt3.html
\tikzstyle{startstop} = [rectangle, rounded corners, minimum width=3cm, minimum height=1cm,text centered, draw=black, fill=red!30]
\tikzstyle{io} = [trapezium, trapezium left angle=70, trapezium right angle=110, minimum width=1cm, minimum height=1cm, text=white, text centered, draw=black, fill=blue!50!violet]
\tikzstyle{process} = [rectangle, minimum width=3cm, minimum height=1cm, text=white, text centered, draw=black, fill=red!50!black]
\tikzstyle{decision} = [diamond, minimum width=3cm, minimum height=1cm, text centered, draw=black, fill=green!30]
\tikzstyle{arrow} = [thick,->,>=stealth]
%UML definitions
\tikzstyle{umlclass}=[rectangle, draw=black,  thick, anchor=north, text width=3cm, rectangle split, rectangle split parts = 3]
\tikzstyle{umlarrow}=[->, >=open triangle 90, thick]

%%%%%%%%%%%%%%%%%%%%%%%%%%%%
%%% lingstings specifics:
\usepackage{listings}
\usepackage{upquote} %http://tex.stackexchange.com/questions/145416/how-to-have-straight-single-quotes-in-lstlistings
\lstdefinelanguage{Scala}{
  morekeywords={abstract,case,catch,class,def,%
    do,else,extends,false,final,finally,%
    for,forSome,if,implicit,import,lazy,match,%
    new,null,object,override,package,%
    private,protected,return,sealed,%
    super,this,throw,trait,true,try,%
    type,val,var,while,with,yield},
  otherkeywords={=>,<-,<\%,<:,>:,@},
  sensitive=true,
  morecomment=[l]{//},
  morecomment=[n]{/*}{*/},
  morestring=[b]",
  morestring=[b]',
  morestring=[b]"""
}


\lstset{
    language=Scala,
    tabsize=2,
    basicstyle=\ttfamily\selectfont,
    keywordstyle=\bfseries\textcolor{eclipsepurple},
    commentstyle=\textcolor{commentgreen},
    numberstyle={\footnotesize},
    numbers=none,
    %backgroundcolor=\textcolor{gray!15},
    frame=none,
    rulecolor=\color{black!25},
    %title={\footnotesize\lstname},
    breaklines=false,
    breakatwhitespace=false,
    framextopmargin=2pt,
    framexbottommargin=2pt,
    showstringspaces=false,
    columns=fullflexible,keepspaces
}
\lstset{literate=%
{Å}{{\AA}}1
{Ä}{{\"A}}1
{Ö}{{\"O}}1
{Ü}{{\"U}}1
{ß}{{\ss}}1
{ü}{{\"u}}1
{å}{{\aa}}1
{ä}{{\"a}}1
{ö}{{\"o}}1
{æ}{{\ae}}1
{ø}{{\o}}1
{Æ}{{\AE}}1
{Ø}{{\O}}1
{`}{{\`{}}}1
{─}{{\textemdash}}1
{└}{{|}}1
{├}{{|}}1
{│}{{|}}1
}

\newcommand{\code}{\lstinline[basicstyle=\ttfamily]}
\newcommand{\jcode}{\lstinline[basicstyle=\ttfamily,language=Java]}

\lstnewenvironment{Code}[1][]{%
    \lstset{basicstyle=\ttfamily\fontsize{9}{11}\selectfont,#1}%
}{}

%*****************************************************************



\newcommand{\LangRect}[5]{\tikz[overlay, remember picture,inner sep=7pt,minimum height=0.65cm] \node[fill=#2,text=white,rotate=90] at #4 (name) {\large\normalfont\textbf{#1} ~~{\small \thepage(#5)}}; }  

\newcommand{\LangRectOdd}[4]{\LangRect{#1}{#2}{#3}{($(current page.north east)-(0.35,#3)$)}{#4}}  
\newcommand{\LangRectEven}[4]{\LangRect{#1}{#2}{#3}{($(current page.north west)-(-0.35,#3)$)}{#4}}  
    

\newcommand{\LangMarker}[3]{%param 1 = language, param 2 = offset from top
%\fancyhead{} % clear all header fields
\fancyfoot{} % clear all footer fields
%\fancyfoot[RO]{\thepage}
%\fancyfoot[LE]{\thepage}
\fancyhead{
\ifodd\thepage\LangRectOdd{#1}{\LangColor}{#2}{#3}
\else\LangRectEven{#1}{\LangColor}{#2}{#3}
\fi 
}
\renewcommand{\headrulewidth}{0pt}
\renewcommand{\footrulewidth}{0pt}
\pagestyle{fancy}
}

\newcommand{\Newline}{\vspace{\baselineskip}}

\newcommand{\LangTitle}[1]{{\centering \Huge{\bfseries\sffamily \color{\LangColor}{#1}}\par}}

\newcommand{\Comment}[1]{{\color{commentgreen}{#1}}}

\begin{document}


%%%************************************************************************************
%%%*************************** JAVA STARTS HERE ***************************************
%%%************************************************************************************

%%%%% TODO: Translate to English

\clearpage\renewcommand{\arraystretch}{0.91}
\setcounter{page}{1}
\renewcommand{\LangColor}{blue}
\LangMarker{Java}{3.5cm}{4}

\LangTitle{Java Quick Ref  @ Lund University}
\vspace{1em}
{\small Vertikalstreck \Or~används mellan olika alternativ. Parenteser \RepL \RepR används för att gruppera en mängd alternativ. Hakparenteser \OptL \OptR markerar valfria delar. En sats betecknas \jcode{stmt} medan \jcode{x}, \jcode{i}, \jcode{s}, \jcode{ch} är variabler, \jcode{expr} är ett uttryck, \jcode{cond} är ett logiskt uttryck. Med \verb|...| avses valfri, extra kod.}

%{\small Tecknet ~|~ står för ''eller''. Vanliga parenteser~(~)~används för att gruppera alternativ. Med~[~]~markeras sådant som inte alltid finns med. Med \jcode{stmt} avses en sats, \jcode{x}, \jcode{i}, \jcode{s}, \jcode{ch} är variabler, \jcode{expr} är ett uttryck, \jcode{cond} är ett logiskt uttryck.}


\Newline
\head{Satser}\Newline
{\small
\begin{tabular}{@{}l l l}
Block   & \jcode|{stmt1; stmt2; ...}| &  fungerar ''utifrån'' som \textbf{en} sats \secend

Tilldelning & \verb|x = expr;|                 &  variabeln och uttrycket av kompatibel typ \secend

Förkortade       & \verb|x += expr; |                &  x = x + expr; även --=, *=, /= \\
                 & \verb|x++;  |                     &  x = x + 1; även x\hspace{0.5mm}-- -- \secend

if-sats          & \verb|if (cond) {stmt; ...}|      &  utförs om cond är true \\
                 & \OptL \verb|else { stmt; ...}| \OptR&  utförs om false \secend

switch-sats      & \verb|switch (expr) {|         &  expr är ett heltalsuttryck \\ 
                 & \verb|    case A: stmt1; break;|  &  utförs om expr $=$ A (A konstant) \\
                 & \verb|    ...|                 & ''faller igenom'' om break saknas\\
                 & \verb|    default: stmtN; break;| &  sats efter default: utförs om inget case passar\\
                 & \verb|}|                   & \secend

for-sats         & \verb|for (int i = a; i < b; i++) {| & satserna görs för i = a, a+1, \ldots, b-1\\
                 & \verb|    stmt; ...|  & Görs ingen gång om a >= b \\
                 & \verb|}  |         &  i++ kan ersättas med i = i + step \secend

%ADDED BY BJORNR                 
for-each-sats    & \verb|for (int x: xs) {|   & xs är en samling, här med heltal  \secend
                 & \verb|    stmt; ...|  & x blir ett element i taget ur xs\\
                 & \verb|}  |         &  fungerar även med array \secend

while-sats       & \verb|while (cond) {stmt; ...}|          & utförs så länge cond är true \secend

do-while-sats    & \verb|do {|                     & \\
                 & \verb|    stmt; ...|         &  utförs minst en gång, \\
                 & \verb|} while (cond);|          &  så länge cond är true \secend

return-sats      & \verb|return expr;|        &  returnerar funktionsresultat
\end{tabular}
}

\Newline\head{Uttryck}
\begin{etab}
Aritmetiskt uttryck & (x + 2) * i / 2 + i \% 2   &  för heltal är / heltalsdivision, \% ''rest''   \secend


Objektuttryck     & \mc{new Classname(\ldots) \Or ref-var \Or null \Or function-call \Or this \Or super} \secend

Logiskt uttryck   & \mc{! cond \Or  cond \&\& cond \Or  cond || cond \Or  relationsuttryck \Or true \Or  false} \subsecend

Relationsuttryck  & expr \RepL < \Or <= \Or == \Or >= \Or > \Or != \RepR expr & för objektuttryck bara == och !=, också typtest med expr instanceof Classname \subsecend

Funktionsanrop    & obj-expr.method(\ldots)    &  anropa ''vanlig metod'' (utför operation) \\
                  & Classname.method(\ldots)   &  anropa statisk metod \secend

Array             & new int[size]              &  skapar int-array med size element \\
                  & vname[i]                   &  elementet med index i, 0..length$-1$\\
                  & vname.length               &  antalet element \secend

%ADDED BY SANDRA:
Matris		      & new int[r][c]			   & //Skapar matris med r rader och c kolonner\\
   				& m.length			   & //Ger matrisens längd (d.v.s. antalet rader) \\
 				  & m[i].length			   & //Ger antalet element (längden) på raden i  \secend

Typkonvertering   & (newtype) expr             &  konverterar expr till typen newtype \\
                  & (int) real-expr            &  -- avkortar genom att stryka decimaler \\
                  & (Square) aShape            &  -- ger ClassCastException om aShape inte \\
                  &                            &  är ett Square-objekt
\end{etab}


\clearpage
\head{Deklarationer}
\begin{etab}
Allmänt           & \mc{\OptL<protection> \OptR  \OptL static \OptR \OptL final \OptR <type> name1, name2, \ldots;} \secend

<type>            & \mc{byte \Or short \Or int \Or long \Or float \Or double \Or boolean \Or char \Or Classname} \secend

<protection>      & public \Or private \Or protected &  för attribut och metoder i klasser \\
                  &                             &  (paketskydd om inget anges) \secend

Startvärde        & int x = 5;                  &  startvärde bör alltid anges \secend

Konstant          & final int N = 20;           &  konstantnamn med stora bokstäver \secend

Array             & <type>[\hspace{0.4mm}] vname = new <type>[10]; &  deklarerar och skapar array \secend

%ADDED BY SANDRA
Matris            & <type>[\hspace{0.4mm}][\hspace{0.4mm}] m = new <type>[4][5]; & // deklarerar och skapar 4x5 matrisen m \secend

\end{etab}


\head{Klasser}
\begin{etab}
Deklaration       & \OptL public\OptR \OptL abstract \OptR class Classname & \\
                  & \mc{\ind \OptL extends Classname1\OptR \OptL implements Interface1, Interface2, \ldots\OptR $\{$} \\
                  & \ind <deklaration av attribut> & \\
                  & \ind <deklaration av konstruktorer> & \\
                  & \ind <deklaration av metoder> & \\
                  & $\}$                          & \secend

Attribut          & \mc{Som vanliga deklarationer. Attribut får implicita startvärden, 0, 0.0, false, null.} \secend

Konstruktor       & <prot> Classname(param, \ldots) $\{$ &  Parametrarna är de parametrar som ges vid \\
                  & \ind stmt; \ldots              &  new Classname(\ldots). Satserna ska ge \\ 
                  & $\}$                           &  attributen startvärden \secend

Metod             & <prot> <type> name(param, \ldots) $\{$ &  om typen inte är void måste en return-\\
                  & \ind stmt; \ldots              &  sats exekveras i metoden \\
                  & $\}$                           & \secend
                  
Huvudprogram      & \mc{public static void main(String[\hspace{0.5mm}] args) $\{$ \ldots $\}$} \secend

Abstrakt metod    & \mc{Som vanlig metod, men abstract före typnamnet och $\{\ldots\}$ ersätts med semikolon. Metoden måste implementeras i subklasserna.}
\end{etab}

\head{Standardklasser, java.lang, behöver inte importeras}
\begin{etab}
Object            & \mc{Superklass till alla klasser.} \subsecend
                  & boolean equals(Object other);   &  ger true om objektet är lika med other \\
                  & int hashCode();                 &  ger objektets hashkod \\
                  & String toString();              &  ger en läsbar representation av objektet \secend
                  
Math              & \mc{Statiska konstanter Math.PI och Math.E. Metoderna är statiska (anropas med t~ex Math.round(x)):} \subsecend
                  & long round(double x);           &  avrundning, även float $\rightarrow$ int \\
                  & int abs(int x);                 &  $|x|$, även double, \ldots \\
                  & double hypot(double x, double y); &  $\sqrt{x^2+y^2}$ \\
                  & double sin(double x);           &  $\sin x$, liknande: cos, tan, asin, acos, atan \\
                  & double exp(double x);           &  $e^x$ \\
                  & double pow(double x, double y); &  $x^y$ \\
                  & double log(double x);           &  $\ln x$ \\
                  & double sqrt(double x);          &  $\sqrt{x}$ \\
                  & double toRadians(double deg);   &  $\mathit{deg} \cdot \pi / 180$ \secend

System            & void System.out.print(String s); &  skriv ut strängen s \\
                  & void System.out.println(String s); &  som print men avsluta med ny rad \\
                  & void System.exit(int status);   &  avsluta exekveringen, status != 0 om fel \\
                  & \mc{Parametern till print och println kan vara av godtycklig typ: int, double, \ldots} \secend
\end{etab}
\clearpage
\begin{etab}
Wrapperklasser     & \mc{För varje datatyp finns en wrapperklass: char $\rightarrow$ Character, int $\rightarrow$ Integer, double $\rightarrow$ Double, \ldots\ 
Statiska konstanter MIN\_VALUE och MAX\_VALUE i klassen Integer ger minsta respektive största heltalsvärde. För klassen Double ger MIN\_VALUE minsta flyttalet som är större än noll.\\
Exempel med klassen Integer:} \subsecend
                  & Integer(int value);             &  skapar ett objekt som innehåller value \\
                  & int intValue();                 &  tar reda på värdet \secend

String            & \mc{Teckensträngar där tecknen inte kan ändras. ''asdf'' är ett String-objekt. s1 + s2 för att konkatenera två strängar. StringIndexOutOfBoundsException om någon position är fel.} \subsecend
                  & int length();                   &  antalet tecken \\
                  & char charAt(int i);             &  tecknet på plats i, 0..length()$-1$ \\
                  & boolean equals(String s);       &  jämför innehållet (s1 == s2 fungerar inte) \\
                  & int compareTo(String s);        &  < 0 om mindre, = 0 om lika, > 0 om större \\
                  & int indexOf(char ch);           &  index för ch, $-1$ om inte finns \\
                  & int indexOf(char ch, int from); &  som indexOf men börjar leta på plats from \\
                  & String substring(int first, int last); &  kopia av tecknen first..last$-1$ \\
                  & String[\hspace{0.5mm}] split(String delim); &  ger array med ''ord'' (ord är följder av \\
                  &                                 &  tecken åtskilda med tecknen i delim) \secend

                  & \mc{Konvertering mellan standardtyp och String (exempel med int, liknande för andra typer):} \\%\subsecend
                  & String.valueOf(int x);          &  x = 1234 $\rightarrow$ ''1234'' \\ 
                  & Integer.parseInt(String s);     &  s = ''1234'' $\rightarrow$ 1234, NumberFormat- \\
                  &                                 &  Exception om s innehåller felaktiga tecken \secend

StringBuilder     & \mc{Modifierbara teckensträngar. length och charAt som String, plus:} \subsecend
                  & StringBuilder(String s);        &  StringBuilder med samma innehåll som s \\
                  & void setCharAt(int i, char ch); &  ändrar tecknet på plats i till ch \\
                  & StringBuilder append(String s); &  lägger till s, även andra typer: int, char, \ldots \\
                  & StringBuilder insert(int i, String s); &  lägger in s med början på plats i \\
                  & StringBuilder deleteCharAt(int i); &  tar bort tecknet på plats i \\ 
                  & String toString();              &  skapar kopia som String-objekt
\end{etab}

\head{Standardklasser, import java.util.Classname}
\begin{etab}
List              & \mc{List<E> är ett gränssnitt som beskriver listor med objekt av parameterklassen E. Man kan lägga in värden av standardtyperna genom att kapsla in dem, till exempel int i Integer-objekt. Gränssnittet implementeras av klasserna ArrayList<E> och LinkedList<E>, som har samma operationer. Man ska inte använda operationerna som har en position som parameter på en LinkedList (i stället en iterator). IndexOutOfBoundsException om någon position är fel.} \subsecend
                  & \mc{För att operationerna contains, indexOf och remove(Object) ska fungera måste klassen E överskugga funktionen equals(Object). Integer och de andra wrapperklasserna gör det.} \subsecend
ArrayList         & ArrayList<E>();                 &  skapar tom lista \\ 
LinkedList        & LinkedList<E>();                &  skapar tom lista \\ 
                  & int size();                     &  antalet element \\
                  & boolean isEmpty();              &  ger true om listan är tom \\
                  & E get(int i);                   &  tar reda på elementet på plats i \\
                  & int indexOf(Object obj);        &  index för obj, $-1$ om inte finns \\
                  & boolean contains(Object obj);   &  ger true om obj finns i listan \\

                  & void add(E obj);                &  lägger in obj sist, efter existerande element \\
                  & void add(int i, E obj);         &  lägger in obj på plats i (efterföljande \\
                  &                                 &  element flyttas) \\
                  & E set(int i, E obj);            &  ersätter elementet på plats i med obj \\
                  & E remove(int i);                &  tar bort elementet på plats i (efter- \\
                  &                                 &  följande element flyttas) \\
                  & boolean remove(Object obj);     &  tar bort objektet obj, om det finns \\
                  & void clear();                   &  tar bort alla element i listan \secend

\end{etab}
\secend
\begin{etab}

Random            & Random();                       &  skapar ''slumpmässig'' slumptalsgenerator \\
                  & Random(long seed);              &  -- med bestämt slumptalsfrö \\
                  & int nextInt(int n);             &  heltal i intervallet [0, n) \\
                  & double nextDouble();            &  double-tal i intervallet [0.0, 1.0) \secend

Scanner           & Scanner(File f);                &  läser från filen f, ofta System.in \\
                  & Scanner(String s);              &  läser från strängen s \\
                  & String next();                  &  läser nästa sträng fram till whitespace \\
                  & boolean hasNext();              &  ger true om det finns mer att läsa \\
                  & int nextInt();                  &  nästa heltal; också nextDouble(), \ldots \\
                  & boolean hasNextInt();           &  också hasNextDouble(), \ldots \\
                  & String nextLine();              &  läser resten av raden
\end{etab}



\head{Filer, import java.io.File/FileNotFoundException/PrintWriter}
\begin{etab}
Läsa från fil     & \mc{Skapa en Scanner med new Scanner(new File(filename)). Ger File\-NotFoundException om filen inte finns. Sedan läser man ''som vanligt'' från scannern (nextInt och liknande).} \subsecend
                  
Skriva till fil   & \mc{Skapa en PrintWriter med new PrintWriter(new File(filename)). Ger FileNotFoundException om filen inte kan skapas. Sedan skriver man ''som vanligt'' på PrintWriter-objektet (println och liknande).}\subsecend

Fånga undantag    & \mc{Så här gör man för att fånga FileNotFoundException:\\[1mm]
\ind Scanner scan = null;\\
\ind try $\{$\\
\ind \ind scan = new Scanner(new File("indata.txt"));\\
\ind $\}$ catch (FileNotFoundException e) $\{$\\
\ind \ind \ldots\ ta hand om felet\\
\ind $\}$}
\end{etab}


\head{Specialtecken}
\begin{etab}
                  & \mc{Några tecken måste skrivas på ett speciellt sätt när de används i teckenkonstanter:}\subsecend

& \textbackslash{}n & ny rad, radframmatningstecken\\
& \textbackslash{}t & ny kolumn, tabulatortecken (eng. tab)\\
& \textbackslash{}\textbackslash{} & bakåtsnedstreck: \textbackslash{} (eng. {\em backslash})\\
& \textbackslash{}'' & citationstecken: ''\\
& \textbackslash{}' & apostrof: '\\

\end{etab}

\head{Reserverade ord}\Newline

{\small Nedan 50 ord kan ej användas som identifierare i Java. Orden \jcode{goto} och \jcode{const} är reserverade men används ej.}
 
\begin{Code}[language=Java,morekeywords={assert, enum, strictfp}]
abstract assert boolean break byte case catch char class const 
continue default do double else enum extends final finally float for 
goto if implements import instanceof int interface long native new 
package private protected public return short static strictfp super 
switch synchronized this throw throws transient try void volatile while
\end{Code}


\end{document}