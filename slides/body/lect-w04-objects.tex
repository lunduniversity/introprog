%!TEX encoding = UTF-8 Unicode
%!TEX root = ../lect-w04.tex

\Subsection{Objekt}



\begin{Slide}{Vad rymmer sköldpaddan i Kojo i sitt tillstånd?}
\centering
\includegraphics[width=0.7\textwidth]{../img/kojo}

\pause position, rikting, pennfärg, pennbredd, penna uppe/nere, fyllfärg
\end{Slide}



\begin{Slide}{Vad är ett objekt?}
\begin{itemize}
\item Ett objekt är en abstraktion som...
\begin{itemize}
  \item kan innehålla \Emph{data} som objektet ''håller reda på'' och
  \item kan erbjuda \Emph{operationer} som \emph{gör} något eller ger ett \emph{värde}
\end{itemize}

\pause

\item Exempel: Sköldpaddan i Kojo \includegraphics[width=0.08\textwidth]{../img/turtle.png}
\begin{itemize}
  \item Vilken \Emph{data} sparas av sköldpaddan?
  \pause
  \item[] position, färg på pennan, vinkel, pennans bredd, ...

  \item Vilka \Emph{operationer} kan man be sköldpaddan att utföra?

  \item[] fram, höger, vänster, ...
  \pause


\end{itemize}

\item Terminologi:
\begin{itemize}
  \item objektets data sparas i variabler som kallas \Alert{attribut}
  \item operationerna är funktioner i objektet och kallas \Alert{metoder}
  \item alla variabelvärden utgör tillsammans objektets \Alert{tillstånd}
  \item attribut, metoder (och annat i objektet) kallas \Alert{medlemmar}
\end{itemize}
\end{itemize}
\end{Slide}



\begin{Slide}{Deklarera, allokera, referera}
Olika saker man kan göra med objekt:
\begin{itemize}
  \item \Emph{deklarera}: att skriva kod som beskriver objekt
  \item \Emph{allokera}: att skapa plats i minnet för objektet vid körtid
  \item \Emph{referera}: att använda objektet via ett namn;\\
  man kommer åt innehållet i ett objekt med \Alert{punktnotation}: \\
  \code{ref.medlem}
  \pause
  \item (\Emph{avallokera}): att frigöra minne för objekt som inte längre används;
  detta \Alert{sker automatiskt} i Scala och Java, men i andra språk t.ex. C++ får man själv hålla reda på avallokering, vilket är knepigt och det blir lätt svåra buggar.
\end{itemize}
\end{Slide}


\begin{Slide}{Allokera objekt}
Det finns flera olika sätt att skapa objekt:
\begin{itemize}

\item Använda en \Emph{färdig funktion} som skapar ett objekt åt oss:
\begin{Code}
Vector(1,2,3)   // skapar ett Vector-objekt
\end{Code}

\item Använda \code{new} (mer om detta senare: då behövs en klass):
\begin{Code}
new SimpleWindow(600, 400, "fönstertitel")  // skapar ett fönsterobjekt
\end{Code}
{\SlideFontSmall Med upprepad användning av \code{new} kan man skapa \Alert{många upplagor} av samma typ av objekt.\vspace{0.5em}}

\item Deklarera ett \Emph{singelobjekt} med nyckelordet \code{object}
\begin{itemize}
  \item Ett singelobjekt finns i exakt \Alert{en} upplaga och
  \item allokeras \Alert{automatisk} första gången man refererar objektet; \\
  man behöver inte, och kan inte, skriva \code{new}.
\end{itemize}
\end{itemize}
\end{Slide}



\begin{Slide}{Vad är ett singelobjekt?}
\begin{itemize}
\item Ett singelobjekt \Eng{singelton} deklareras med nyckelordet \code{object} och kan användas för att samla \Emph{medlemmar} \Eng{members} som \Alert{hör ihop}.
\item Kallas också \Emph{modul}.
\item Medlemmarna kan t.ex. vara \Emph{variabler} (\code{val}, \code{var}) och \Emph{metoder} (\code{def}). En \Alert{metod} är en \Emph{funktion} i ett objekt.
\item Exempel: singelobjekt/modul som hanterar highscore:
\begin{Code}
object Highscore {
  var highscore = 0
  def isHighscore(points: Int): Boolean = points > highscore
}
\end{Code}
\end{itemize}
\pause
{\SlideFontSmall Tanken är ofta att abstraktionen ska vara användbar i annan kod, för att underlätta när man bygger applikationer, och kallas då även ett API (Application Programming Interface). Exempel: ett highscore-API}
\end{Slide}


\begin{Slide}{Allokering: minne reserveras med plats för data}
\begin{Code}
object Highscore {
  var highscore = 0
  def isHighscore(points: Int): Boolean = points > highscore
}
\end{Code}
\pause
\begin{tikzpicture}[font=\large\sffamily]
\matrix [matrix of nodes, row sep=0, column 2/.style={nodes={rectangle,draw,minimum width=0.8cm}}] (mat)
{
\texttt{HighScore}   &  \makebox(10,10){ }\\
%\texttt{g2}   &  \makebox(16,12){ }\\
};
\node[cloud, cloud puffs=13.0, cloud ignores aspect, minimum width=2cm, minimum height=3.8cm,
 align=center, draw] (x) at (5.8cm, -1.5cm) {
 \begin{tabular}{r l}
 \texttt{highscore} & \fbox{~~~~~0~~} \\

 \end{tabular}
 };
\filldraw[black] (1.2cm,0.0cm) circle (3pt) node[] (ref) {};
 \draw [arrow, line width=0.7mm] (ref) -- (x);
% \node[cloud, cloud puffs=15.7, cloud ignores aspect, %minimum width=5cm, minimum height=2cm,
% align=center, draw] (g2) at (5cm, -2cm) {Gurka-\\objekt};
% \filldraw[black] (0.4cm,-0.4cm) circle (3pt) node[] (g2ref) {};
% \draw [arrow] (g2ref) -- (g2);
\end{tikzpicture}
\end{Slide}


\begin{Slide}{Punktnotation, tillståndsförändring med tilldelning}
\begin{REPLnonum}
Highscore.highscore = 42
\end{REPLnonum}
\pause
\begin{tikzpicture}[font=\large\sffamily]
\matrix [matrix of nodes, row sep=0, column 2/.style={nodes={rectangle,draw,minimum width=0.8cm}}] (mat)
{
\texttt{HighScore}   &  \makebox(10,10){ }\\
%\texttt{g2}   &  \makebox(16,12){ }\\
};
\node[cloud, cloud puffs=13.0, cloud ignores aspect, minimum width=2cm, minimum height=3.8cm,
 align=center, draw] (x) at (5.8cm, -1.5cm) {
 \begin{tabular}{r l}
 \texttt{highscore} & \fbox{~~~~42~~} \\

 \end{tabular}
 };
\filldraw[black] (1.2cm,0.0cm) circle (3pt) node[] (ref) {};
 \draw [arrow, line width=0.7mm] (ref) -- (x);
% \node[cloud, cloud puffs=15.7, cloud ignores aspect, %minimum width=5cm, minimum height=2cm,
% align=center, draw] (g2) at (5cm, -2cm) {Gurka-\\objekt};
% \filldraw[black] (0.4cm,-0.4cm) circle (3pt) node[] (g2ref) {};
% \draw [arrow] (g2ref) -- (g2);
\end{tikzpicture}
\end{Slide}


\begin{Slide}{Inkapsling: att dölja interna delar}\SlideFontSmall
Med nyckelordet \code{private} döljs interna delar för omvärlden.\\
Privata delar kan bara refereras innifrån objektet.
\begin{Code}
object Highscore {
  private var highscore = 0    // syns ej utåt
  def isHighscore(points: Int): Boolean = points > highscore
  def update(points: Int): Unit = if (isHigscore) highscore = points
}
\end{Code}
Varför har man nytta av detta?
\begin{itemize}
  \item Förhindra att man av misstag ändrar objekts tillstånd på fel sätt.
  \item Förhindra användning av kod som i framtiden kan komma att ändras.
  \item Erbjuder en enklare ''utsida'' genom dölja komplexitet ''på insidan''.
  \item Inte ''skräpa ner'' namnrymden med ''onödiga'' namn.
\end{itemize}
Nackdelar:
\begin{itemize}
  \item Begränsar användningen, har ej tillgång till alla delar.
  \item Svårare att experimentera med ett API medan man försöker förstå det.
\end{itemize}
\end{Slide}


\begin{Slide}{Modul}

\begin{itemize}
  \item En modul samlar kod som utgör en sammanhållen, avgränsad \Emph{uppsättning abstraktioner} som kan användas av annan kod  för att lösa ett specifikt (del)problem.
  \item I Scala finns två sätt att skapa moduler:\footnote{\href{https://en.wikipedia.org/wiki/Modular_programming}{en.wikipedia.org/wiki/Modular\_programming}}
  \begin{itemize}
    \item \Emph{singelobjekt} med nyckelordet \code{object} och
    \item \Emph{paket} med nyckelordet \code{package}
    \pause
    \item Liknar varandra; t.ex. kan man använda punktnotation och göra \code{import} på medlemmar i både singelobjekt och paket.
    \pause
    \item Skillnader:
    \begin{itemize}
      \item paket medför att \Alert{underkataloger} för maskinkoden skapas vid kompilering
      \item paket innehåller inte variabler på topp-nivå; dessa behöver ligga inuti singelobjekt eller klasser
    \end{itemize}
    \item (I Scala kan man kombinera begreppen: \code{package object})
  \end{itemize}

\end{itemize}
\end{Slide}



\begin{Slide}{Exempel: Bankkonto} \SlideFontSmall
\begin{Code}[basicstyle=\ttfamily\fontsize{9}{11}\selectfont]
object mittBankkonto {
  val kontonr: Long        = 1234567L
  var saldo: Int           = 1000
  def ärSkuldsatt: Boolean = saldo < 0
}
\end{Code}
\begin{REPLnonum}
scala> mittBankkonto.saldo -= 25000

scala> mittBankkonto.ärSkuldsatt
res0: Boolean = true
\end{REPLnonum}

(Vi ska i nästa vecka se hur man med s.k. klasser kan skapa många upplagor av samma  typ av objekt, så att vi kan ha flera olika bankkonto.)
\end{Slide}



\begin{Slide}{Vad är ett tillstånd?}
Ett objekts \Emph{tillstånd} är den samlade uppsättningen av värden av alla de variabler som finns i objektet.
\begin{Code}[basicstyle=\ttfamily\fontsize{9}{11}\selectfont]
object mittBankkonto {
  val kontonr: Long        = 1234567L
  var saldo: Int           = 1000
  def ärSkuldsatt: Boolean = saldo < 0
}
\end{Code}
\begin{tikzpicture}[font=\large\sffamily]
\matrix [matrix of nodes, row sep=0, column 2/.style={nodes={rectangle,draw,minimum width=0.8cm}}] (mat)
{
\texttt{mittBankkonto}   &  \makebox(10,10){ }\\
%\texttt{g2}   &  \makebox(16,12){ }\\
};
\node[cloud, cloud puffs=13.0, cloud ignores aspect, minimum width=2cm, minimum height=3.8cm,
 align=center, draw] (x) at (5.8cm, -1.5cm) {
 \begin{tabular}{r l}
 \texttt{kontonr} & \fbox{1234567L} \\
 \texttt{saldo} & \fbox{1000}\\
 \end{tabular}
 };
\filldraw[black] (1.7cm,0.0cm) circle (3pt) node[] (ref) {};
 \draw [arrow, line width=0.7mm] (ref) -- (x);
% \node[cloud, cloud puffs=15.7, cloud ignores aspect, %minimum width=5cm, minimum height=2cm,
% align=center, draw] (g2) at (5cm, -2cm) {Gurka-\\objekt};
% \filldraw[black] (0.4cm,-0.4cm) circle (3pt) node[] (g2ref) {};
% \draw [arrow] (g2ref) -- (g2);
\end{tikzpicture}
\end{Slide}


\begin{Slide}{Tillståndsändring}

När en variabel tilldelas ett nytt värde sker en \Emph{tillståndsändring}. Ett \Emph{förändringsbart objekt} \Eng{mutable object} har ett \Emph{förändringsbart tillstånd} \Eng{mutable state}.

\begin{REPLnonum}
scala> mittBankkonto.saldo -= 25000

scala> mittBankkonto.saldo
res1: Int = -24000
\end{REPLnonum}
\begin{tikzpicture}[font=\large\sffamily]
\matrix [matrix of nodes, row sep=0, column 2/.style={nodes={rectangle,draw,minimum width=0.8cm}}] (mat)
{
\texttt{mittBankkonto}   &  \makebox(10,10){ }\\
%\texttt{g2}   &  \makebox(16,12){ }\\
};
\node[cloud, cloud puffs=13.0, cloud ignores aspect, minimum width=2cm, minimum height=3.8cm,
 align=center, draw] (x) at (5.8cm, -1.5cm) {
 \begin{tabular}{r l}
 \texttt{kontonr} & \fbox{1234567L} \\
 \texttt{saldo} & \fbox{-24000}\\
 \end{tabular}
 };
\filldraw[black] (1.7cm,0.0cm) circle (3pt) node[] (ref) {};
 \draw [arrow, line width=0.7mm] (ref) -- (x);
% \node[cloud, cloud puffs=15.7, cloud ignores aspect, %minimum width=5cm, minimum height=2cm,
% align=center, draw] (g2) at (5cm, -2cm) {Gurka-\\objekt};
% \filldraw[black] (0.4cm,-0.4cm) circle (3pt) node[] (g2ref) {};
% \draw [arrow] (g2ref) -- (g2);
\end{tikzpicture}
\end{Slide}






\begin{Slide}{Namnrymd \Eng{name space}, synlighet och skuggning}
\TODO
\end{Slide}


\begin{Slide}{Tupler}
\TODO
\end{Slide}



\Subsection{Fördröjd initialisering}

\begin{Slide}{Lata variabler och fördröjd initialisering}
Med nyckelordet \code{lazy} före \code{val} skapas en s.k. ''lat'' \Eng{lazy} variabel.
\begin{REPL}
scala> val striktVektor = Vector.fill(1000000)(math.random)
striktVektor: scala.collection.immutable.Vector[Double] =
 Vector(0.7583305221813246, 0.9016192590993339, 0.770022134260162, 0.15667718184929746, ...

scala> lazy val latVektor = Vector.fill(1000000)(math.random)
latVektor: scala.collection.immutable.Vector[Double] = <lazy>

scala> latVektor
res0: scala.collection.immutable.Vector[Double] =
  Vector(0.5391685014341797, 0.14759775960530275, 0.722606095900537, 0.9025572787055386, ...
\end{REPL}

En \code {lazy val} initialiseras \Alert{inte} vid deklarationen utan när den \Alert{refereras första gången}. Yttrycket som anges i deklarationen evalueras med s.k. \Emph{fördröjd evaluering} (även ''lat'' evaluering).
\end{Slide}

\begin{Slide}{Vad är egentligen skillnaden mellan \texttt{val}, \texttt{var}, \texttt{def} och \texttt{lazy val}?}
\begin{Code}[basicstyle=\ttfamily\fontsize{8}{11}\selectfont]
object slump {
  val förAlltidSammaReferens  = math.random
  var kanÄndrasMedTilldelning = math.random
  def evaluerasVidVarjeAnrop  = math.random
  lazy val fördröjdInit       = Vector.fill(1000000)(math.random)
}
\end{Code}
\vspace{1em}\pause
Lat evaluering är en viktig princip inom funktionsprogrammering som möjliggör effektiva, oföränderliga datastrukturer där element allokeras först när de behövs. \\
\href{https://en.wikipedia.org/wiki/Lazy_evaluation}{en.wikipedia.org/wiki/Lazy\_evaluation}
\end{Slide}



\begin{Slide}{Singelobjekt är lata}

\begin{itemize}
  \item Singelobjekt allokeras \Alert{inte} direkt vid deklaration; allokeringen sker först då objektet refereras första gången.

\pause

  \item Exempel:

\end{itemize}

\begin{Code}
object mittLataObjekt {
  println("jag är lat")
  val storArray = { println("skapar stor Array"); Array.fill(10000)(42) }
  lazy val ännuStörreArray = Array.fill(Int.MaxValue)(42)
}
\end{Code}

När sker utskrifterna?

När allokeras variablerna?

\end{Slide}



\Subsection{Funktioner är objekt}

\begin{Slide}{Programmeringsparadigm}
\href{https://en.wikipedia.org/wiki/Programming_paradigm}{en.wikipedia.org/wiki/Programming\_paradigm}:
\begin{itemize}
\item \Emph{Imperativ programmering}: programmet är uppbyggt av sekvenser av olika satser som läser och \Alert{ändrar} tillstånd
\item \Emph{Objektorienterad programmering}: en sorts imperativ programmering där programmet består av objekt som kapslar in tillstånd och erbjuder operationer som läser och \Alert{ändrar} tillstånd.
\item \Emph{Funktionsprogrammering}: programmet är uppbyggt av samverkande (matematiska) funktioner som \Alert{undviker} föränderlig data och tillståndsändringar. Oföränderliga datastrukturer skapar effektiva program i kombination med lat evaluering och rekursion.
\end{itemize}
\end{Slide}


\begin{Slide}{Funktioner är äkta objekt i Scala}
Scala visar hur man kan \Alert{förena} \Eng{unify} \\ \Emph{objekt-orientering} och \Emph{funktionsprogrammering}: \\\vspace{0.5em}

\textbf{En funktion är ett objekt som har en \code{apply}-metod.}
\pause
\begin{REPLnonum}
scala> object öka {
         def apply(x: Int) = x + 1
       }

scala> öka.apply(1)
res0: Int = 2

scala> öka(1)   // metoden apply behöver ej skrivas explicit
res1: Int = 2
\end{REPLnonum}
\end{Slide}



\begin{Slide}{Fördjupning: Äkta funktionsobjekt är av funktionstyp}
Egentligen, mer precist:\\
\textbf{En funktion är ett objekt \Alert{av funktionstyp} som har en \code{apply}-metod.}
\pause
\begin{REPLnonum}
scala> object öka extends (Int => Int) {
         def apply(x: Int) = x + 1
       }

scala> öka(1)
res2: Int = 2

scala> Vector(1,2,3).map(öka)
res3: scala.collection.immutable.Vector[Int] = Vector(2, 3, 4)

scala> öka.   // tryck TAB
andThen   apply   compose   toString
\end{REPLnonum}
Mer om \code{extends} senare i kursen... %extends (Int => Int skrivs om till Function1[Int, Int]
\end{Slide}



\Subsection{Använda färdiga klasser och dokumentation}
\begin{Slide}{Färdiga, enkla funktioner för att rita finns i klassen \texttt{cslib.window.SimpleWindow}}
På labben ska du använda \code{cslib.window.SimpleWindow}
\begin{itemize}
\item Paketet \code{cslib} innehåller paketet \code{window} som innehåller Java-klassen \code{SimpleWindow}.
%\item En \Emph{klass} är en ''mall'' för att göra \Emph{objekt}.
\item Med \code{SimpleWindow} kan man skapa ritfönster.
%\item När man skapar ett objekt från en klass använder man nyckelordet \code{new}.
\item Ladda ner \url{http://cs.lth.se/pgk/cslib} och lägg sedan jar-filen den katalog där du startar REPL med: \code{scala -cp cslib.jar}
\end{itemize}
\pause
\begin{REPLnonum}
> scala -cp cslib.jar
scala> val w = new SimpleWindow(200,200,"hejsan")
\end{REPLnonum}
\pause Studera dokumentationen för \code{cslib.window.SimpleWindow} här: \url{http://cs.lth.se/pgk/api/}
\end{Slide}



\begin{Slide}{Använda dokumentation}
\TODO Scaladoc scala api, javadoc simplewindow
\end{Slide}


\ifkompendium\else


\Subsection{Veckans övning och laboration}

\begin{Slide}{Övning \texttt{objects}}\SlideFontTiny
\setlength{\leftmargini}{0pt}
\begin{itemize}
%!TEX encoding = UTF-8 Unicode
%!TEX root = ../compendium2.tex

\item Kunna skapa och använda objekt som moduler.
\item Känna till att funktioner är objekt med en \code{apply}-metod.
\item Förstå begreppen synlighet, \code{private}, \code{import}, namnrymd och skuggning.
\item \TODO{FLER MÅL OM OBJEKT HÄR}

%\item Känna till svansrekursion och att svansrekursiva funktioner kan optimeras till loopar.

\end{itemize}
\end{Slide}

\begin{Slide}{Lab \texttt{blockmole}}%\SlideFontTiny
%\setlength{\leftmargini}{0pt}
\begin{itemize}
%!TEX encoding = UTF-8 Unicode
%!TEX root = ../labs.tex

\item Kunna förklara hur singelobjekt kan användas som moduler.
\item Kunna förklara hur åtkomst av medlemmar i singelobjekt sker.
\item Kunna skapa kod som reagerar på och förändrar objekts tillstånd.
\item Kunna förklara nyttan med att samla namngivna konstanter i egen modul.
\item Kunna förklara hur import påverkar synlighet av namn.
\item Kunna ge exempel på en situation där man har nytta av namnbyte vid import.
\item Kunna redogöra för skillnaden mellan paket och singelobjekt.
\item Kunna skapa och använda tupler.

\end{itemize}

\end{Slide}
\fi
