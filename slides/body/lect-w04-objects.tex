%!TEX encoding = UTF-8 Unicode
%!TEX root = ../lect-w04.tex

\Subsection{Objekt}

\begin{Slide}{Objekt som modul}
\begin{itemize}
\item Ett \code{object} användas ofta för att samla \Emph{medlemmar} \Eng{members} som \Alert{hör ihop} och ge dem en egen \Emph{namnrymd} \Eng{name space}. 
\item Medlemmarna kan vara t.ex.: 
\begin{itemize}
\item  \code{val} \item \code{var} \item \code{def} 
\end{itemize}

\item Ett sådant objekt kallas även för \Emph{modul}.\footnote{
Även paket som skapas med \code{package} har en egen namnrymd och är därmed också en slags modul. Objekt kan alltså i Scala användas som ett alternativ till paket; en skillnad är att objekt kan ha tillstånd och att objekt inte skapar underkataloger vid kompilering (det finns iofs s.k. \code{package object}) \href{https://en.wikipedia.org/wiki/Modular_programming}{en.wikipedia.org/wiki/Modular\_programming}}

\end{itemize}
\end{Slide}




\begin{Slide}{Singelobjekt och metod} \SlideFontSmall
Ett Scala-\code{object} är ett s.k. \Emph{singelobjekt} \Eng{singleton object} och finns bara i \Alert{en} enda upplaga. \\ Minne för objektets variabler allokeras första gången objektet refereras. \\ En funktion som finns i ett objekt kallas en \Emph{metod} \Eng{method}.
\begin{Code}[basicstyle=\ttfamily\fontsize{9}{11}\selectfont]
object mittBankkonto {
  val kontonr: Long        = 1234567L
  var saldo: Int           = 1000
  def ärSkuldsatt: Boolean = saldo < 0
}
\end{Code}
\begin{REPLnonum}
scala> mittBankkonto.saldo -= 25000

scala> mittBankkonto.ärSkuldsatt
res0: Boolean = true
\end{REPLnonum}

(Vi ska i nästa vecka se hur man med s.k. klasser kan skapa många upplagor av samma  typ av objekt, så att vi kan ha flera olika bankkonto.)
\end{Slide} 



\begin{Slide}{Vad är ett tillstånd?} 
Ett objekts \Emph{tillstånd} är den samlade uppsättningen av värden av alla de variabler som finns i objektet.
\begin{Code}[basicstyle=\ttfamily\fontsize{9}{11}\selectfont]
object mittBankkonto {
  val kontonr: Long        = 1234567L
  var saldo: Int           = 1000
  def ärSkuldsatt: Boolean = saldo < 0
}
\end{Code}
\begin{tikzpicture}[font=\large\sffamily]
\matrix [matrix of nodes, row sep=0, column 2/.style={nodes={rectangle,draw,minimum width=0.8cm}}] (mat) 
{
\texttt{mittBankkonto}   &  \makebox(10,10){ }\\
%\texttt{g2}   &  \makebox(16,12){ }\\
};
\node[cloud, cloud puffs=13.0, cloud ignores aspect, minimum width=2cm, minimum height=3.8cm,
 align=center, draw] (x) at (5.8cm, -1.5cm) {
 \begin{tabular}{r l}
 \texttt{kontonr} & \fbox{1234567L} \\
 \texttt{saldo} & \fbox{1000}\\
 \end{tabular}
 };
\filldraw[black] (1.7cm,0.0cm) circle (3pt) node[] (ref) {};
 \draw [arrow, line width=0.7mm] (ref) -- (x);
% \node[cloud, cloud puffs=15.7, cloud ignores aspect, %minimum width=5cm, minimum height=2cm,
% align=center, draw] (g2) at (5cm, -2cm) {Gurka-\\objekt};
% \filldraw[black] (0.4cm,-0.4cm) circle (3pt) node[] (g2ref) {};
% \draw [arrow] (g2ref) -- (g2);
\end{tikzpicture}
\end{Slide} 


\begin{Slide}{Tillståndsändring} 

När en variabel tilldelas ett nytt värde sker en \Emph{tillståndsändring}. Ett \Emph{förändringsbart objekt} \Eng{mutable object} har ett \Emph{förändringsbart tillstånd} \Eng{mutable state}. 

\begin{REPLnonum}
scala> mittBankkonto.saldo -= 25000

scala> mittBankkonto.saldo
res1: Int = -24000
\end{REPLnonum}
\begin{tikzpicture}[font=\large\sffamily]
\matrix [matrix of nodes, row sep=0, column 2/.style={nodes={rectangle,draw,minimum width=0.8cm}}] (mat) 
{
\texttt{mittBankkonto}   &  \makebox(10,10){ }\\
%\texttt{g2}   &  \makebox(16,12){ }\\
};
\node[cloud, cloud puffs=13.0, cloud ignores aspect, minimum width=2cm, minimum height=3.8cm,
 align=center, draw] (x) at (5.8cm, -1.5cm) {
 \begin{tabular}{r l}
 \texttt{kontonr} & \fbox{1234567L} \\
 \texttt{saldo} & \fbox{-24000}\\
 \end{tabular}
 };
\filldraw[black] (1.7cm,0.0cm) circle (3pt) node[] (ref) {};
 \draw [arrow, line width=0.7mm] (ref) -- (x);
% \node[cloud, cloud puffs=15.7, cloud ignores aspect, %minimum width=5cm, minimum height=2cm,
% align=center, draw] (g2) at (5cm, -2cm) {Gurka-\\objekt};
% \filldraw[black] (0.4cm,-0.4cm) circle (3pt) node[] (g2ref) {};
% \draw [arrow] (g2ref) -- (g2);
\end{tikzpicture}
\end{Slide} 




\begin{Slide}{Vad rymmer sköldpaddan i Kojo i sitt tillstånd?} 
\centering
\includegraphics[width=0.7\textwidth]{../img/kojo}

\pause position, rikting, pennfärg, pennbredd, penna uppe/nere, fyllfärg
\end{Slide} 




\begin{Slide}{Lata variabler och fördröjd evaluering} 
Med nyckelordet \code{lazy} före \code{val} skapas en s.k. ''lat'' \Eng{lazy} variabel.
\begin{REPL}
scala> val striktVektor = Vector.fill(1000000)(math.random)
striktVektor: scala.collection.immutable.Vector[Double] = 
 Vector(0.7583305221813246, 0.9016192590993339, 0.770022134260162, 0.15667718184929746, ...
 
scala> lazy val latVektor = Vector.fill(1000000)(math.random)
latVektor: scala.collection.immutable.Vector[Double] = <lazy>

scala> latVektor
res0: scala.collection.immutable.Vector[Double] = 
  Vector(0.5391685014341797, 0.14759775960530275, 0.722606095900537, 0.9025572787055386, ... 
\end{REPL}

En \code {lazy val} initialiseras \Alert{inte} vid deklarationen utan när den \Alert{refereras första gången}. Yttrycket som anges i deklarationen evalueras med s.k. \Emph{fördröjd evaluering} (även ''lat'' evaluering). 
\end{Slide} 

\begin{Slide}{Vad är egentligen skillnaden mellan \texttt{val}, \texttt{var}, \texttt{def} och \texttt{lazy val}?} 
\begin{Code}[basicstyle=\ttfamily\fontsize{8}{11}\selectfont]
object slump {
  val förAlltidSammaReferens  = math.random
  var kanÄndrasMedTilldelning = math.random
  def evaluerasVidVarjeAnrop  = math.random
  lazy val fördröjdInit       = Vector.fill(1000000)(math.random)
}
\end{Code}
\vspace{1em}\pause
Lat evaluering är en viktig princip inom funktionsprogrammering som möjliggör effektiva, oföränderliga datastrukturer där element allokeras först när de behövs. \\
\href{https://en.wikipedia.org/wiki/Lazy_evaluation}{en.wikipedia.org/wiki/Lazy\_evaluation}
\end{Slide} 





\Subsection{Funktioner är objekt}

\begin{Slide}{Programmeringsparadigm}
\href{https://en.wikipedia.org/wiki/Programming_paradigm}{en.wikipedia.org/wiki/Programming\_paradigm}:
\begin{itemize}
\item \Emph{Imperativ programmering}: programmet är uppbyggt av sekvenser av olika satser som läser och \Alert{ändrar} tillstånd
\item \Emph{Objektorienterad programmering}: en sorts imperativ programmering där programmet består av objekt som kapslar in tillstånd och erbjuder operationer som läser och \Alert{ändrar} tillstånd.
\item \Emph{Funktionsprogrammering}: programmet är uppbyggt av samverkande (matematiska) funktioner som \Alert{undviker} föränderlig data och tillståndsändringar. Oföränderliga datastrukturer skapar effektiva program i kombination med lat evaluering och rekursion. 
\end{itemize}
\end{Slide} 


\begin{Slide}{Funktioner är äkta objekt i Scala}
Scala visar hur man kan \Alert{förena} \Eng{unify} \\ \Emph{objekt-orientering} och \Emph{funktionsprogrammering}: \\\vspace{0.5em}

\textbf{\Large En funktion är ett objekt av funktionstyp\\ som har en \code{apply}-metod.}
\pause
\begin{REPLnonum}
scala> object öka extends (Int => Int) { 
         def apply(x: Int) = x + 1 
       }


scala> öka(1)
res0: Int = 2

scala> öka.   // tryck TAB
andThen   apply   compose   toString
\end{REPLnonum}
Mer om \code{extends} senare i kursen... %extends (Int => Int skrivs om till Function1[Int, Int]
\end{Slide} 


\Subsection{Rekursion}
\begin{Slide}{Rekursiva funktioner}
\begin{itemize}
\item Funktioner som \Alert{anropar sig själv} kallas \Emph{rekursiva}.


\begin{REPLnonum}
scala> def fakultet(n: Int): Int = 
         if (n < 2) 1 else n * fakultet(n - 1)

scala> fakultet(5)
res0: Int = 120
\end{REPLnonum}

\item För varje nytt anrop läggs en ny aktiveringspost på stacken. 

\item I aktiveringsposten sparas varje returvärde som gör att \code{5 * (4 * (3 * (2 * 1)))} kan beräknas. 

\item Rekrusionen avbryts när man når \Emph{basfallet}, här \code{n < 2}

\item En rekursiv funktion \Alert{måste} ha en returtyp.

\end{itemize}

\end{Slide} 

\begin{Slide}{Loopa med rekursion}
\begin{Code}
def gissaTalet(max: Int): Unit = {
  def gissat = io.StdIn.readLine(s"Gissa talet mellan [1, $max]: ").toInt 
  val hemlis = (math.random * max + 1).toInt
  def skrivLedtrådOmEjRätt(gissning: Int): Unit = 
    if (gissning > hemlis) println(s"$gissning är för stort :(") 
    else if (gissning < hemlis) println(s"$gissning är för litet :(")
  def ärRätt(gissning: Int): Boolean = { 
    skrivLedtrådOmEjRätt(gissning)
    gissning == hemlis
  }
  def loop(n: Int = 1): Int = if (ärRätt(gissat)) n else loop(n + 1)
  
  println(s"Du hittade talet $hemlis på ${loop()} gissningar :)")
}
\end{Code}
\end{Slide} 




\begin{Slide}{Rekursiva datastrukturer}

\begin{itemize}
\item Datastrukturena Lista och Träd är exempel på datastrukturer som passar bra ihop med rekursion. 
\item Båda dessa datastrukturer kan beskrivas rekursivt:
\begin{itemize}
\item En lista består av ett huvud och en lista, som i sin tur består av ett huvud och en lista, som i sin tur...
\item Ett träd består av grenar till träd som i sin tur består av grenar till träd som i sin tur, ...
\end{itemize}
\item Dessa datastrukturer bearbetas med fördel med rekursiva algoritmer.
\item I denna kursen ingår rekursion endast ''för kännedom'': \\ du ska veta vad det är och kunna skapa en enkel rekursiv funktion, t.ex. fakultets-beräkning. Du kommer jobba mer med rekursion och rekursiva datastrukturer i fortsättningskursen.
\end{itemize}
\end{Slide} 

\Subsection{SimpleWindow}
\begin{Slide}{Färdiga, enkla funktioner för att rita finns i klassen \texttt{cslib.window.SimpleWindow}}
På labben ska du använda \code{cslib.window.SimpleWindow}
\begin{itemize}
\item Paketet \code{cslib} innehåller paketet \code{window} som innehåller Java-klassen \code{SimpleWindow}.
%\item En \Emph{klass} är en ''mall'' för att göra \Emph{objekt}. 
\item Med \code{SimpleWindow} kan man skapa ritfönster. 
%\item När man skapar ett objekt från en klass använder man nyckelordet \code{new}.
\item Ladda ner \url{http://cs.lth.se/pgk/cslib} och lägg sedan jar-filen den katalog där du startar REPL med: \code{scala -cp cslib.jar}
\end{itemize}
\pause
\begin{REPLnonum}
$ scala -cp cslib.jar
scala> val w = new SimpleWindow(200,200,"hejsan")  
\end{REPLnonum}
\pause Studera dokumentationen för \code{cslib.window.SimpleWindow} här: \url{http://cs.lth.se/pgk/api/}
\end{Slide} 


\ifkompendium\else


\Subsection{Veckans övning och laboration}

\begin{Slide}{Övning \texttt{functions}}\SlideFontTiny
\setlength{\leftmargini}{0pt}
\begin{itemize}
%!TEX encoding = UTF-8 Unicode
%!TEX root = ../exercises.tex

\item Kunna skapa och använda funktioner med en eller flera parametrar, default-argument, namngivna argument, och uppdelad parameterlista.
\item Kunna använda funktioner som äkta värden.
\item Kunna skapa och använda anonyma funktioner (ä.k. lambda-funktioner).
\item Kunna applicera en funktion på element i en samling.
\item Förstå skillnader och likheter mellan en funktion och en procedur.
\item Förstå vad ett block och en lokal variabel är.
\item Kunna skapa och använda lokala funktioner och förklara nyttan med dessa.
\item Förstå skillnader och likheter mellan värdeanrop och namnanrop.
\item Kunna skapa en enkel kontrollstruktur med fördröjd evaluering av ett block.
\item Förstå skillnaden mellan äkta funktioner och funktioner med sidoeffekter.
%\item Kunna skapa och använda variabler med fördröjd initialisering och förstå när de är användbara.
\item Kunna förklara hur nästlade funktionsanrop sker med   aktiveringsposter.
\item Känna till rekursion och kunna förklara hur rekursiva funktioner fungerar.
\item Känna till att det går att partiellt applicera argument på funktioner med uppdelad parameterlista för att skapa s.k. stegade funktioner (ä.k. curry-funktioner).

%\item Känna till svansrekursion och att svansrekursiva funktioner kan optimeras till loopar.

\end{itemize}
\end{Slide} 

\begin{Slide}{Lab \texttt{blockmole}}%\SlideFontTiny
%\setlength{\leftmargini}{0pt}
\begin{itemize}
%!TEX encoding = UTF-8 Unicode
%!TEX root = ../compendium2.tex

%\item Kunna kompilera Scalaprogram med \texttt{scalac}.
%\item Kunna köra Scalaprogram med \texttt{scala}.
%\item Kunna definiera och anropa funktioner.
%\item Kunna använda och förstå default-argument.
%\item Kunna ange argument med parameternamn.
\item Kunna skapa ett större program med din egen kod efter dina egna idéer.
\item Kunna använda en editor och terminalen för att iterativt editera, kompilera, och testa din kod.
\item Kunna använda variabler i kombination med alternativ och repetetition i flera nivåer.
\item Kunna stegvis förbättra din kod för att underlätta förändring och öka läsbarhet.
\item Kunna skapa och använda abstraktioner för att generalisera och möjliggöra återanvändning av kod.

\end{itemize}

\end{Slide} 
\fi
