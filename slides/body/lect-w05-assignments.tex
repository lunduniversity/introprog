%!TEX encoding = UTF-8 Unicode
%!TEX root = ../lect-w05.tex

%%%

\ifkompendium\else

%\Subsection{Denna och nästa veckas uppgifter}
\begin{SlideExtra}{Övning: \texttt{classes}}
\begin{itemize}\SlideFontSmall
%!TEX encoding = UTF-8 Unicode

%!TEX root = ../compendium2.tex

\item Kunna deklarera klasser med klassparametrar.
\item Kunna skapa objekt med \code{new} och konstruktorargument.
\item Förstå innebörden av referensvariabler och värdet \code{null}.
\item Förstå innebörden av begreppen instans och referenslikhet.
\item Kunna använda nyckelordet \code{private} för att styra synlighet i primärkonstruktor.
\item Förstå i vilka sammanhang man kan ha nytta av en privat konstruktor.
\item Kunna implementera en klass utifrån en specifikation.
\item Förstå skillnaden mellan referenslikhet och strukturlikhet.
\item Känna till hur case-klasser hanterar likhet.
\item Förstå nyttan med att möjliggöra framtida förändring av attributrepresentation.
\item Känna till begreppen getters och setters.
\item Känna till accessregler för kompanjonsobjekt.
\item Känna till skillnaden mellan \code{==} och \code{eq}, samt \code{!=} versus \code{ne}.

\end{itemize}
\end{SlideExtra}

\begin{SlideExtra}{Lab \texttt{blockbattle} redovisas NÄSTA läsvecka 6}
  \begin{minipage}{0.42\textwidth}
        \includegraphics[height=0.8\textheight]{../img/blockbattle.png}
  \end{minipage}%
  \begin{minipage}{0.59\textwidth}
    \begin{itemize}\SlideFontTiny
      \item Ett arkadspel för två spelare.
      \item Nu med TVÅ blockmullvader!
      \item Träna på att skapa och använda klasser för att möjliggöra flera instanser.
      \item Träna på att använda matchning.
      \item Läs igenom labben redan nu så du förstår hur övningarna kommer till nytta på labben.
      \item Programmet \code{blockbattle} är \Alert{väsentligt mer omfattande} jämfört med tidigare laborationer.
      \item Du har till fredag vecka 6 på dig innan du ska redovisa. (Vecka 6 handlar bl.a. om matchning som du behöver till labben.)
      \item OBS! Labbtider är \Emph{schemalagda som vanligt} under läsvecka 5, men fungerar som resurstider och är \Emph{inte} obligatoriska. 
      \item Utnyttja undervisningen på ett sätt som gör att du \Alert{maximerar} ditt lärande.  
    \end{itemize}    
  \end{minipage}
\end{SlideExtra}
  
\fi
