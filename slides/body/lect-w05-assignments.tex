%!TEX encoding = UTF-8 Unicode
%!TEX root = ../lect-w05.tex

%%%

\ifkompendium\else

%\Subsection{Denna och nästa veckas uppgifter}
\begin{SlideExtra}{Övning: \texttt{classes}}
\begin{itemize}\SlideFontSmall
%!TEX encoding = UTF-8 Unicode

%!TEX root = ../compendium2.tex

\item Kunna deklarera klasser med klassparametrar.
\item Kunna skapa objekt med \code{new} och konstruktorargument.
\item Förstå innebörden av referensvariabler och värdet \code{null}.
\item Förstå innebörden av begreppen instans och referenslikhet.
\item Kunna använda nyckelordet \code{private} för att styra synlighet i primärkonstruktor.
\item Förstå i vilka sammanhang man kan ha nytta av en privat konstruktor.
\item Kunna implementera en klass utifrån en specifikation.
\item Förstå skillnaden mellan referenslikhet och strukturlikhet.
\item Känna till hur case-klasser hanterar likhet.
\item Förstå nyttan med att möjliggöra framtida förändring av attributrepresentation.
\item Känna till begreppen getters och setters.
\item Känna till accessregler för kompanjonsobjekt.
\item Känna till skillnaden mellan \code{==} och \code{eq}, samt \code{!=} versus \code{ne}.

\end{itemize}
\end{SlideExtra}

\begin{SlideExtra}{Lab \texttt{blockbattle} redovisas NÄSTA läsvecka 6}
  \begin{minipage}{0.42\textwidth}
        \includegraphics[height=0.8\textheight]{../img/blockbattle.png}
  \end{minipage}%
  \begin{minipage}{0.59\textwidth}
    \begin{itemize}\SlideFontTiny
      \item Ett arkadspel för TVÅ spelare.
      \item Nu med TVÅ blockmullvader!
      \item Programmet \code{blockbattle} är \Alert{mer omfattande} jmf m tidigare labbar.
      \item Labben omfattar TVÅ veckors förel. + övn.:\\klasser (w05) \& mönster (w06).
      \item Det normala är registrering ''kompletteras'' i SAM på \code{blockbattle0} under w05.
      \item Läs igenom labben innan du gör övningarna.
      \item Kod från övningarna behövs till labben. 
      \item OBS! Labbtider är \Emph{schemalagda som vanligt} under läsvecka 5. Om du mot förmodan hinner redovisa redan denna vecka så ska du \Alert{ändå närvara} och t.ex. jobba med \Emph{extrauppgifter}. 
      \item Dra nytta av undervisningen så att du lär dig så mycket som möjligt!  
    \end{itemize}    
  \end{minipage}
\end{SlideExtra}
  
\fi
