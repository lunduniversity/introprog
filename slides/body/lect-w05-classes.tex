%!TEX encoding = UTF-8 Unicode
%!TEX root = ../lect-w05.tex

%%%

%TODO:
%  \begin{itemize}
%  \item Bygg upp \code{case class Complex(re: Double, im: Double)} steg för steg inspirerat av Pins3ed kap 6 i likhet med hur de gör med Rational
%  \item Illustrera följande begrepp: this (behövs i max(that)), method overloading behövs för att plussa med både Complex och Double
%  \item Till fördjupningsövning: dekorera Double med metoderna im och re samt (Double, Double) med metoden ir (för irrational) med implicit klass
%  \item Till extrauppgift: implementera klassen Polar(r, fi) med polära koordinater \url{https://sv.wikipedia.org/wiki/Pol%C3%A4ra_koordinater}
%  \end{itemize}

\Subsection{Vad är en klass?}

\begin{Slide}{Vad är en klass?}
\begin{itemize}
\item En klass är en mall för att skapa objekt.
\item Objekt skapas med \code{new Klassnamn} och kallas för  \Emph{instanser} av klassen \code{Klassnamn}.
\item En klass innehåller medlemmar \Eng{members}:
  \begin{itemize}
  \item \Emph{attribut}, kallas även fält \Eng{field}: \code{val}, \code{lazy val}, \code{var}
  \item \Emph{metoder}, kallas även operationer: \code{def}
  \end{itemize}
\item Varje instans har sin uppsättning värden på attributen (fälten).
\end{itemize}

\end{Slide}



\begin{Slide}{En klass liknar en stämpel}\SlideFontSmall

\begin{multicols}{2}

Metafor: En klass liknar en \Emph{stämpel}

\begin{itemize}
\item En stämpel kan tillverkas -- motsvarar deklaration av klassen.
 \item Det händer inget förrän man stämplar -- motsvarar \code{new}.
\item Då skapas avbildningar -- motsvarar instanser av klassen.
\end{itemize}


\columnbreak

\includegraphics[width=0.4\textwidth]{../img/stamp}


\end{multicols}
\end{Slide}






\begin{Slide}[t]{Klass och instans}
\vspace{-0.65em}
\begin{REPLnonum}
scala> class C { var attr = 42 }

scala> val objRef1 = new C
\end{REPLnonum}
\vspace{3.7em}
\begin{tikzpicture}[font=\SlideFontSmall\sffamily]
\matrix [matrix of nodes, row sep=0, column 2/.style={nodes={rectangle,draw,minimum width=0.8cm}}] (mat)
{
\texttt{objRef1}   &  \makebox(10,10){ }\\
};

\node[cloud, cloud puffs=15.0, cloud ignores aspect, minimum width=2cm, minimum height=2cm,
 align=center, draw] (instance1) at (3.8cm, 0.0cm) {
 \begin{tabular}{r l}
 \texttt{attr} & \fbox{42} \\
 \end{tabular}
 };


\filldraw[black] ($ (mat-1-2) + (0.0cm,0.0cm) $) circle (3pt) node[] (ref1)  {};
\draw [arrow, line width=0.7mm] (ref1) -- (instance1);
\end{tikzpicture}
\end{Slide}




\begin{Slide}[t]{Klass och instans}
\vspace{-0.5em}
\begin{REPLnonum}
scala> class C { var attr = 42 }

scala> val objRef1 = new C

scala> val objRef2 = new C
\end{REPLnonum}
\vspace{2em}
\begin{tikzpicture}[font=\SlideFontSmall\sffamily]
\matrix [matrix of nodes, row sep=0, column 2/.style={nodes={rectangle,draw,minimum width=0.8cm}}] (mat)
{
\texttt{objRef1}   &  \makebox(10,10){ }\\
\texttt{objRef2}   &  \makebox(10,10){ }\\
};

\node[cloud, cloud puffs=15.0, cloud ignores aspect, minimum width=2cm, minimum height=2cm,
 align=center, draw] (instance1) at (3.8cm, 0.35cm) {
 \begin{tabular}{r l}
 \texttt{attr} & \fbox{42} \\
 \end{tabular}
 };

\node[cloud, cloud puffs=15.0, cloud ignores aspect, minimum width=2cm, minimum height=2cm,
 align=center, draw] (instance2) at (5.8cm, -1.5cm) {
 \begin{tabular}{r l}
 \texttt{attr} & \fbox{42} \\
 \end{tabular}
 };


\filldraw[black] ($ (mat-1-2) + (0.0cm,0.0cm) $) circle (3pt) node[] (ref1)  {};
\draw [arrow, line width=0.7mm] (ref1) -- (instance1);

\filldraw[black] ($ (mat-2-2) + (0.0cm,0.0cm) $) circle (3pt) node[] (ref2)  {};
\draw [arrow, line width=0.7mm] (ref2) -- (instance2);
\end{tikzpicture}
\end{Slide}




\begin{Slide}[t]{Klass och instans}
\vspace{-0.5em}
\begin{REPLnonum}
scala> class C { var attr = 42 }

scala> val objRef1 = new C

scala> val objRef2 = new C

scala> objRef2.attr = 43
\end{REPLnonum}
\begin{tikzpicture}[font=\SlideFontSmall\sffamily]
\matrix [matrix of nodes, row sep=0, column 2/.style={nodes={rectangle,draw,minimum width=0.8cm}}] (mat)
{
\texttt{objRef1}   &  \makebox(10,10){ }\\
\texttt{objRef2}   &  \makebox(10,10){ }\\
};

\node[cloud, cloud puffs=15.0, cloud ignores aspect, minimum width=2cm, minimum height=2cm,
 align=center, draw] (instance1) at (3.8cm, 0.35cm) {
 \begin{tabular}{r l}
 \texttt{attr} & \fbox{42} \\
 \end{tabular}
 };

\node[cloud, cloud puffs=15.0, cloud ignores aspect, minimum width=2cm, minimum height=2cm,
 align=center, draw] (instance2) at (5.8cm, -1.5cm) {
 \begin{tabular}{r l}
 \texttt{attr} & \fbox{43} \\
 \end{tabular}
 };


\filldraw[black] ($ (mat-1-2) + (0.0cm,0.0cm) $) circle (3pt) node[] (ref1)  {};
\draw [arrow, line width=0.7mm] (ref1) -- (instance1);

\filldraw[black] ($ (mat-2-2) + (0.0cm,0.0cm) $) circle (3pt) node[] (ref2)  {};
\draw [arrow, line width=0.7mm] (ref2) -- (instance2);
\end{tikzpicture}
\end{Slide}








\begin{Slide}{Klassdeklarationer och instansiering}\SlideFontSmall
\setlength{\leftmargini}{0pt}
\begin{itemize}
\item Syntax för deklaration av klass: \\ \vspace{0.5em}{\SlideFontSize{13}{16}\code|class Klassnamn(parametrar){ medlemmar }|}\vspace{0.5em}



\item Exempel på \Emph{deklaration}:
\begin{Code}
class Klassnamn(val attribut1: Int, attribut2: String){
  val attribut3: Double = 42.0              //publikt oföränderligt attribut
  private var attribut4: Boolean = false    //privat medlem syns inte utåt
  def metod(parameter: Int) = parameter + 1 //funktion i klass kallas metod
  lazy val attr5 = Vector.fill(100000)(42.0)     //fördröjd initialisering
}
\end{Code}

\item Parametrar initialiseras med de argument som ges vid \code{new}.
\item Exempel på \Emph{instansiering} med argument för initialisering av klassparametrar:
\begin{Code}
val instansReferens = new Klassnamn(42, "hej")
\end{Code}

\item Attribut blir \Emph{publika} (alltså synliga utåt) om inte modifieraren \code{private} anges.
\item Parametrar som inte föregås av modifierare (t.ex. private val, val, var) blir \Emph{attribut} som är \code{ private[this] val } och bara är synliga i \Alert{denna} instans.

\end{itemize}
\end{Slide}






\begin{Slide}{Exempel: Klassen Complex i Scala}\SlideFontSmall
\begin{Code}
class Complex(val re: Double, val im: Double){
  def r  = math.hypot(re, im)
  def fi = math.atan2(re, im)
  def +(other: Complex) = new Complex(re + other.re, im + other.im)
  var imSymbol = 'i'
  override def toString = s"$re + $im$imSymbol"
}
\end{Code}

\begin{REPL}
scala> val c1 = new Complex(3, 4)
c1: Complex = 3.0 + 4.0i

scala> val polarForm = (c1.r, c1.fi)
polarForm: (Double, Double) = (5.0,0.6435011087932844)

scala> val c2 = new Complex(1, 2)
c2: Complex = 1.0 + 2.0i

scala> c1 + c2
res0: Complex = 4.0 + 6.0i
\end{REPL}
\end{Slide}



\begin{Slide}{Exempel: Principen om enhetlig access}\SlideFontSmall
\begin{Code}
class Complex(val re: Double, val im: Double){
  val r  = math.hypot(re, im)
  val fi = math.atan2(re, im)
  def +(other: Complex) = new Complex(re + other.re, im + other.im)
  var imSymbol = 'i'
  override def toString = s"$re + $im$imSymbol"
}
\end{Code}
\pause
\begin{itemize}
\item Efter som attributen \code{re} och \code{im} är oföränderliga, kan vi lika gärna ändra i klass-implementationen och göra om metoderna \code{r} och \code{fi} till \code{val}-variabler utan att klientkoden påverkas.

\item Då anropas \code{math.hypot} och \code{math.atan2} bara en gång vid initialisering (och inte varje gång som med \code{def}).

\item Vi skulle även kunna använda \code{lazy val} och då bara räkna ut \code{r} och \code{fi} om och när de verkligen refereras av klientkoden, annars inte.

\item Eftersom klientkoden inte ser skillnad på metoder och variabler, kallas detta \Emph{principen om enhetlig access}. (Många andra språk har \Alert{inte} denna möjlighet, tex Java där metoder \emph{måste} ha parenteser.)
\end{itemize}
\end{Slide}






\Subsection{Olika sätt att skapa instanser}

\begin{Slide}{Instansiering med direkt användning av \texttt{new}}

Instansiering genom \Emph{direkt användning} av \code{new}\\
{\SlideFontSmall (här första varianten av Complex med \code{r} och \code{fi} som metoder)}
\begin{REPLnonum}
scala> val c1 = new Complex(3, 4)
\end{REPLnonum}
\begin{tikzpicture}[font=\SlideFontSmall\sffamily]
\matrix [matrix of nodes, row sep=0, column 2/.style={nodes={rectangle,draw,minimum width=0.8cm}}] (mat)
{
\texttt{c1}   &  \makebox(10,10){ }\\
};

\node[cloud, cloud puffs=15.0, cloud ignores aspect, minimum width=2cm, minimum height=3.8cm,
 align=center, draw] (instance1) at (5.8cm, -1.5cm) {
 \begin{tabular}{r l l}
 \texttt{re:} & \texttt{Double} & \fbox{3.0} \\
 \texttt{im:} & \texttt{Double} & \fbox{4.0}\\
 \texttt{imSymbol:} & \texttt{Char} & \fbox{i}\\
 \end{tabular}
 };

\filldraw[black] ($ (mat-1-2) + (0.0cm,0.0cm) $) circle (3pt) node[] (ref1)  {};

\draw [arrow, line width=0.7mm] (ref1) -- (instance1);
\end{tikzpicture}
\pause
Ofta vill man göra \Alert{indirekt} instansiering så att vi senare har friheten att ändra hur instansiering sker.
\end{Slide}



\begin{Slide}{Indirekt instansiering med fabriksmetoder}\SlideFontSmall
En \Emph{fabriksmetod} är en metod som används för att instansiera objekt.
\begin{Code}[basicstyle=\SlideFontSize{8}{12}\ttfamily\selectfont]
object MyFactory {
  def createComplex(re: Double, im: Double) = new Complex(re, im)
  def createReal(re: Double)                = new Complex(re, 0)
  def createImaginary(im: Double)           = new Complex(0, im)
}
\end{Code}
\pause
Instansiera \Alert{inte direkt}, utan \Emph{indirekt} genom användning av \Emph{fabriksmetoder}:
\begin{REPL}
scala> import MyFactory._

scala> createComplex(3, 4)
res0: Complex = 3.0 + 4.0i

scala> createReal(42)
res1: Complex = 42.0 + 0.0i

scala> createImaginary(-1)
res2: Complex = 0.0 + -1.0i
\end{REPL}
\end{Slide}

\begin{Slide}{Hur förhindra direkt instansiering?}
Om vi vill \Emph{förhindra direkt instansiering} kan vi göra primärkonstruktorn \Alert{privat}:
\begin{Code}
class Complex private (val re: Double, val im: Double){
  def r  = math.hypot(re, im)
  def fi = math.atan2(re, im)
  def +(other: Complex) = new Complex(re + other.re, im + other.im)
  var imSymbol = 'i'
  override def toString = s"$re + $im$imSymbol"
}
\end{Code}
MEN... då går det ju \Alert{inte} längre att instansiera något alls!  \code{   :(}
\begin{REPLnonum}
scala> new Complex(3,4)
error:
     constructor Complex in class Complex cannot be accessed in object $iw
\end{REPLnonum}
\end{Slide}




\begin{Slide}{Kompanjonsobjekt kan förhindra direkt instansiering}\SlideFontSmall
\setlength{\leftmargini}{0pt}
\begin{itemize}
\item Ett \Emph{kompanjonsobjekt} är ett objekt som ligger i samma kodfil som en klass och har samma namn som klassen.

\item Medlemmar i ett kompanjonsobjekt \Alert{får accessa privata} medlemmar i kompanjonsklassen (och vice versa) och kompanjonsobjektet får därför accessa privat konstruktor och kan göra \code{new}.
\begin{Code}
class Complex private (val re: Double, val im: Double){
  def r  = math.hypot(re, im)
  def fi = math.atan2(re, im)
  def +(other: Complex) = new Complex(re + other.re, im + other.im)
  var imSymbol = 'i'
  override def toString = s"$re + $im$imSymbol"
}
object Complex {
  def apply(re: Double, im: Double) = new Complex(re, im)
  def real(re: Double)              = new Complex(re, 0)
  def imag(im: Double)              = new Complex(0, im)
}
\end{Code}
\item Fabriksmetoder i kompanjonsobjektet ovan och privat kontstruktor gör att vi \Alert{enbart} tillåter \Emph{indirekt instansiering}.
\end{itemize}
\end{Slide}

\begin{Slide}{Användning av kompanjonsobjekt med fabriksmetoder för indirekt instansiering}
Nu kan vi \Alert{bara} instansiera \Emph{indirekt}!  \code{   :)}
\begin{REPLnonum}
scala> Complex.real(42.0)
res0: Complex = 42.0 + 0.0i

scala> Complex.imag(-1)
res1: Complex = 0.0 + -1.0i

scala> Complex.apply(3,4)
res2: Complex = 3.0 + 4.0i

scala> Complex(3,4)
res3: Complex = 3.0 + 4.0i

scala> new Complex(3, 4)
error:
     constructor Complex in class Complex cannot be accessed in object $iw
\end{REPLnonum}
\end{Slide}


\begin{Slide}{Alternativa direktinstansieringar med default-argument}\SlideFontSmall
Med \Emph{default-argument} kan vi erbjuda \Emph{alternativa} sätt att direktinstansiera.
\begin{Code}
class Complex(val re: Double = 0, val im: Double = 0){
  def r  = math.hypot(re, im)
  def fi = math.atan2(re, im)
  def +(other: Complex) = new Complex(re + other.re, im + other.im)
  var imSymbol = 'i'
  override def toString = s"$re + $im$imSymbol"
}
\end{Code}
\begin{REPL}
scala> new Complex()
res0: Complex = 0.0 + 0.0i

scala> new Complex(re = 42)  //anrop med namngivet argument
res1: Complex = 42.0 + 0.0i

scala> new Complex(im = -1)
res2: Complex = 0.0 + -1.0i

scala> new Complex(1)
res3: Complex = 1.0 + 0.0i
\end{REPL}
\end{Slide}

\begin{Slide}{Alternativa sätt att instansiera med fabriksmetod}
Vi kan också erbjuda \Emph{alternativa} sätt att instansiera \Emph{indirekt} med fabriksmetoden \code{apply} i ett kompanjonsobjekt genom default-argument:
\begin{Code}
class Complex private (val re: Double, val im: Double){
  def r  = math.hypot(re, im)
  def fi = math.atan2(re, im)
  def +(other: Complex) = new Complex(re + other.re, im + other.im)
  var imSymbol = 'i'
  override def toString = s"$re + $im$imSymbol"
}
object Complex {
  def apply(re: Double = 0, im: Double = 0) = new Complex(re, im)
  def real(r: Double) = apply(re=r)
  def imag(i: Double) = apply(im=i)
  val zero = apply()
}
\end{Code}
\end{Slide}

\begin{Slide}{Medlemmar som bara behövs i en enda upplaga}
Attributet \code{imSymbol} passar bättre att ha i \Emph{kompanjonsobjektet}, eftersom det räcker att ha \Alert{en enda upplaga}, som kan vara gemensam för alla objekt:
\begin{Code}
class Complex private (val re: Double, val im: Double){
  def r  = math.hypot(re, im)
  def fi = math.atan2(re, im)
  def +(other: Complex) = new Complex(re + other.re, im + other.im)
  override def toString = s"$re + $im${Complex.imSymbol}"
}
object Complex {
  var imSymbol = 'i'
  def apply(re: Double = 0, im: Double = 0) = new Complex(re, im)
  def real(r: Double) = apply(re=r)
  def imag(i: Double) = apply(im=i)
  val zero = apply()
}
\end{Code}
\end{Slide}



\begin{Slide}{Medlemmar i singelobjekt är statiskt allokerade}\SlideFontTiny

Minnesplatsen för \Emph{attribut i singelobjekt} allokeras automatiskt en gång för alla, och kallas därför \Emph{statiskt} allokerad. Singelobjektets namn \code{Complex} utgör en statisk referens till den enda instansen och är av typen \texttt{Complex.type}.

\begin{tikzpicture}[font=\SlideFontSmall\sffamily]
\matrix [matrix of nodes, row sep=0, column 2/.style={nodes={rectangle,draw,minimum width=0.8cm}}] (mat)
{
\texttt{Complex}   &  \makebox(10,10){ }\\
};

\node[cloud, cloud puffs=15.0, cloud ignores aspect, minimum width=1cm, minimum height=2cm,
 align=center, draw] (instance1) at (5.8cm, 0.0cm) {
 \begin{tabular}{r l l}
 \texttt{imSymbol:} & \texttt{Char} & \fbox{i}\\
 \end{tabular}
 };

\filldraw[black] ($ (mat-1-2) + (0.0cm,0.0cm) $) circle (3pt) node[] (ref1)  {};

\draw [arrow, line width=0.7mm] (ref1) -- (instance1);
\end{tikzpicture}

Nu bereder vi inte plats för \code{imSymbol} i varenda \Emph{dynamiskt} allokerad instans:
\begin{REPLnonum}
scala> val c1 = Complex(3, 4)   // dynamisk allokering på heapen som växer
\end{REPLnonum}

\begin{tikzpicture}[font=\SlideFontSmall\sffamily]
\matrix [matrix of nodes, row sep=0, column 2/.style={nodes={rectangle,draw,minimum width=0.8cm}}] (mat)
{
\texttt{c1}   &  \makebox(10,10){ }\\
};

\node[cloud, cloud puffs=15.0, cloud ignores aspect, minimum width=2cm, minimum height=2cm,
 align=center, draw] (instance1) at (5.8cm, -0.0cm) {
 \begin{tabular}{r l l}
 \texttt{re:} & \texttt{Double} & \fbox{3.0} \\
 \texttt{im:} & \texttt{Double} & \fbox{4.0}\\
 \end{tabular}
 };

\filldraw[black] ($ (mat-1-2) + (0.0cm,0.0cm) $) circle (3pt) node[] (ref1)  {};

\draw [arrow, line width=0.7mm] (ref1) -- (instance1);
\end{tikzpicture}


\end{Slide}




\begin{Slide}{Attribut i kompanjonsobjekt användas för sådant som är gemensamt för alla instanser}

Om vi ändrar på statiska \code{imSymbol} så ändras \code{toString} för \Alert{alla} dynamiskt allokerade instanser.
\begin{REPLnonum}
scala> val c1 = Complex(3, 4)
c1: Complex = 3.0 + 4.0i

scala> Complex.imSymbol = 'j'
Complex.imSymbol: Char = j

scala> val c2 = Complex(5, 6)
c2: Complex = 5.0 + 6.0j

scala> c1
res0: Complex = 3.0 + 4.0j
\end{REPLnonum}
\end{Slide}





\Subsection{Minneshantering}


\begin{Slide}{Vad är en konstruktor?}
\begin{itemize}
\item En \Emph{konstruktor} är den kod som exekveras när klasser instansieras med \code{new}.

\item Konstruktorn skapar nya objekt i minnet under körning när den anropas.

\item I Scala \Alert{genererar} kompilatorn en \Emph{primärkonstruktor} med maskinkod som initialiserar alla attribut baserat på klassparametrar och \code{val}- och \code{var}-deklarationer.

\item I Java \Alert{måste} man \Alert{själv} skriva alla konstruktorer med speciell syntax och göra alla initialiseringar själv. Man kan ha många olika alternativa konstruktorer i Java.

\item I Scala \Emph{kan} man också skriva egna \Emph{alternativa konstrukturer} med speciell syntax, men det är \Alert{ovanligt}, eftersom man har möjligheten med fabriksmetoder i kompanjonsobjekt och default-argument (saknas i Java).
\end{itemize}
\end{Slide}


\begin{Slide}{Hjälpkonstruktorer i Scala}%\SlideFontSmall
Fördjupning för kännedom:
\begin{itemize}
\item I Scala kan man skapa ett alternativ till primärkonstruktorn, en så kallad \Emph{hjälpkonstruktor} \Eng{auxilliary constructor} genom att deklarera en metod med det speciella namnet \code{this}.


\item Hjälpkonstruktorer \Alert{måste} börja med att anropa en \Alert{annan} konstruktor som står \Alert{före} i koden, till exempel primärkonstruktorn.
\end{itemize}

\begin{Code}
class Point(val x: Int, val y: Int, val z: Int){
  def this(x: Int, y: Int) = this(x, y, 0)   //anrop av primärkonstruktorn
  def this(x: Int) = this(x, 0)              //anrop av hjälpkonstruktor
  override def toString =
    if (z == 0) s"Point($x,$y)" else s"Point($x,$y,$z)"
}
\end{Code}

{\SlideFontSmall Genom att känna till hur hjälpkonstruktorer fungerar i Scala, blir det lättare att begripa konstruktorer i Java.}

\end{Slide}

\begin{Slide}{Användning av hjälpkonstruktor}
\begin{REPL}
scala> val p1 = new Point(1)
p1: Point = Point(1,0)

scala> val p2 = new Point(1, 2)
p2: Point = Point(1,2)

scala> val p3 = new Point(1, 2, 3)
p3: Point = Point(1,2,3)
\end{REPL}

Men man gör \Alert{mycket oftare} så här i Scala:
\begin{Code}
class Point(val x: Int, val y: Int = 0, val z: Int = 0){
  override def toString =
    if (z == 0) s"Point($x,$y)" else s"Point($x,$y,$z)"
}

\end{Code}
Eller använder en fabriksmetod i kompanjonsobjekt. \\
Eller ännu hellre en case-klass...
\end{Slide}



\begin{Slide}{Vad gör skräpsamlaren?}\SlideFontSmall
\begin{itemize}
\item Scala och Java är båda programmeringsspråk som förutsätter en körmiljö med \Alert{automatisk}  \Emph{skräpsamling} \Eng{garbage collection}.

\item \Emph{Skräpsamlaren} \Eng{the garbage collector} är ett program som automatiskt körs i bakgrunden då och då och \Emph{städar minnet} genom att frigöra den plats som upptas av \Alert{objekt som inte längre används}.

\item JVM:en bestämmer själv när skräpsamlaren ska jobba och programmeraren har ingen kontroll över detta.

\item Den stora \Emph{fördelen} med automatisk skärpsamling är att man slipper bry sig om det svåra och felbenägna arbetet att \Alert{avallokera} minne.

\item \Alert{Nackdelen} är att man inte kan styra exakt hur och när skräpsamlingen ska ske och man kan därmed inte bestämma när processorn ska belastas med minneshanteringen. Detta är normalt inget problem, utom i vissa tidskritiska realtidssystem med hårda minnesbegränsningar och svarstidskrav.

\item I språk utan automatisk skräpsamling, t.ex. C++, måste man ta hand om destruktion av objekt och skriva egna s.k. \Emph{destruktorer}.
\end{itemize}
\end{Slide}


\Subsection{Referens saknas: \texttt{null}}

\begin{Slide}{Referens saknas: \texttt{null}}
\begin{itemize}
\item I Java och många andra språk använder man ofta literalen \code{null} för att representera att ett \Alert{värde saknas}.

\item En referens som är \code{null} refererar inte till någon instans.

\item Om du försöker referera till instansmedlemmar med punktnotation genom en referens som är \code{null} kastas ett \Alert{undantag} \code{NullPointerException}.

\item Oförsiktig användning av \code{null} är en vanlig källa till \Alert{buggar}, som kan vara svåra att hitta och fixa.

\end{itemize}
\end{Slide}


\begin{Slide}{Exempel: \texttt{null}}
\begin{REPL}
scala> class Gurka(val vikt: Int)
defined class Gurka

scala> var g: Gurka = null        // ingen instans allokerad än
g: Gurka = null

scala> g.vikt
java.lang.NullPointerException

scala> g = new Gurka(42)          // instansen allokeras
g: Gurka = Gurka@1ec7d8b3

scala> g.vikt
res0: Int = 42

scala> g = null         // instansen kommer att destrueras av skräpsamlaren
\end{REPL}

\begin{itemize} \SlideFontSmall
\item Scala har \code{null} av kompabilitetsskäl, men det är brukligt att \Alert{endast} använda \code{null} om man anropar Java-kod.

\item Scala erbjuder smidiga \code{Option}, \code{Some} och \code{None} för säker hantering av saknade värden; mer om detta i vecka 10.



\end{itemize}
\end{Slide}






\Subsection{Klasser i Java}

\begin{Slide}{Typisk utformning av Java-klass}
Typisk ''anatomi'' av en Java-klass:
\begin{Code}[language=Java]
class Klassnamn {
    attribut, normalt privata
    konstruktorer, normalt publika
    metoder: publika getters, och vid förändringsbara objekt även setters
    metoder: privata abstraktioner för internt bruk
    metoder: publika abstraktioner tänkta att användas av klientkoden
}
\end{Code}
\href{http://www.oracle.com/technetwork/java/codeconventions-141855.html#1852}{\SlideFontSize{9}{8}www.oracle.com/technetwork/java/codeconventions-141855.html\#1852}
\end{Slide}




\begin{Slide}{Java-exempel: Klassen JComplex}\SlideFontSmall
\javainputlisting[basicstyle=\SlideFontSize{5}{6}\ttfamily\selectfont]{../compendium/examples/JComplex.java}
\end{Slide}




\begin{Slide}{Exempel: Använda JComplex i Scala-kod}
\begin{REPL}
$ javac JComplex.java
$ scala
Welcome to Scala 2.12.3 (Java HotSpot(TM) 64-Bit Server VM, Java 1.8.0_66).
Type in expressions for evaluation. Or try :help.

scala> val jc1 = new JComplex(3, 4)
jc1: JComplex = 3.0 + 4.0i

scala> val polarForm = (jc1.getR, jc1.getFi)
polarForm: (Double, Double) = (5.0,0.6435011087932844)

scala> val jc2 = new JComplex(1, 2)
jc2: JComplex = 1.0 + 2.0i

scala> jc1 add jc2
res0: JComplex = 4.0 + 6.0i
\end{REPL}
\end{Slide}




\begin{Slide}{Exempel: Använda JComplex i Java-kod}\SlideFontSmall
\javainputlisting[basicstyle=\SlideFontSize{8}{10}\ttfamily\selectfont]{../compendium/examples/JComplexTest.java}
\begin{itemize}
\item I Java måste man skriva \Alert{tomma parentes-par} efter metodnamnet vid \Alert{anrop av parameterlösa metoder}.

\item \Alert{Tupler finns inte} i Java, så det går inte på ett enkelt sätt att skapa par av värden som i Scala; ovan görs polär form till en sträng för utskrift.

\item \Alert{Operatornotation för metoder finns inte} i Java, så man måste i Java använda punktnotation och skriva: \code{jc1.add(jc2)}
\end{itemize}
\end{Slide}










\begin{Slide}{Statiska medlemmar i Java}
\begin{itemize}
\item Man kan \Alert{inte} deklarera explicita singelobjekt i Java och det finns inget nyckelord \code{object}.

\item I stället kan man deklarera \Emph{statiska medlemmar} i en klass med Java-nyckelordet \jcode{static}.

\item Exempel på hur vi kan göra detta inuti klassen \code{JComplex}:

\begin{Code}[language=Java,basicstyle=\SlideFontSize{10}{12}\ttfamily\selectfont]
    public static char imSymbol = 'i';
\end{Code}

\item Effekten blir den samma som ett singelobjekt i Scala:
\begin{itemize}
\item Alla statiska medlemmar i en Java-klass allokeras automatiskt och hamnar i en egen singulär ''klassinstans'' som existerar oberoende av de dynamiska instanserna.
\item De statiska medlemmarna accessas med punktnotation genom klassnamnet:
\begin{Code}[language=Java,basicstyle=\SlideFontSize{11}{13}\ttfamily\selectfont]
    JComplex.imSymbol = 'j';
\end{Code}

\end{itemize}


\end{itemize}
\end{Slide}



\Subsection{Referensen \texttt{this}}
\begin{Slide}{Referensen \texttt{this}}\SlideFontSmall
\begin{itemize}
\item Nyckelordet \code{this} ger en referens till den aktuella instansen.
\begin{REPLnonum}
scala> class Gurka(var vikt: Int){def jagSjälv = this}
defined class Gurka

scala> val g = new Gurka(42)
g: Gurka = Gurka@5ae9a829

scala> g.jagSjälv
res0: Gurka = Gurka@5ae9a829

scala> g.jagSjälv.vikt
res1: Int = 42

scala> g.jagSjälv.jagSjälv.vikt
res2: Int = 42
\end{REPLnonum}
\item Referensen \code{this} används ofta för att komma runt ''namnkrockar'' där variabler med samma namn gör så att den ena variabeln inte syns.
\end{itemize}
\end{Slide}



\Subsection{Getters och setters}

\begin{Slide}{Getters och setters i Java}\SlideFontSmall
\begin{itemize}
\item I Java finns inget motsvarande nyckelord \code{val} som garanterar oföränderliga attributreferenser.\footnote{Det finns visserligen \jcode{final} men det är annorlunda som vi ska se senare.}

\item Därför gör man i Java nästan alltid attribut \Alert{privata} för att förhindra att de ändras på ett okontrollerat sätt.

\item Java följer inte principen om enhetlig access: åtkomst av metoder och variabler sker med olika syntax.

\item Därför är det normala i Java att införa metoder som kallas \Emph{getters} och \Emph{setters}, som används för att \Alert{indirekt} läsa och uppdatera \Emph{attribut}.

\item Dessa metoder känns igen genom Java-konventionen att de heter något som börjar med \Emph{get} respektive \Emph{set}.

\item Med indirekt access av attribut kan man i Java åstadkomma \Emph{flexibilitet}, så att klassimplementationen kan ändras utan att ändra i klientkoden:
\begin{itemize}\SlideFontSmall
\item man kan t.ex. i efterhand ändra representation av de privata attributen eftersom all access sker genom getters och setters.
\end{itemize}

\item Om klassen \Alert{inte} erbjuder en \Alert{setter} för privata attribut kan man åstadkomma \Emph{oföränderliga} datastrukturer där attributreferenserna inte förändras efter allokering.
\end{itemize}
\end{Slide}



\begin{Slide}{Java-exempel: Klassen JPerson}\SlideFontSmall
\Emph{Indirekt} access av \Alert{privata} attribut:
\vspace{-1em}\begin{multicols}{2}
\javainputlisting[basicstyle=\SlideFontSize{7}{8}\ttfamily\selectfont]{../compendium/examples/JPerson.java}

\columnbreak

\begin{REPLnonum}
$ javac JPerson.java
$ scala
Welcome to Scala 2.11.8 (Java HotSpot(TM) 64-Bit Server VM, Java 1.8.0_66).
Type in expressions for evaluation. Or try :help.

scala> val p = new JPerson("Björn")
p: JPerson = JPerson@7e774085

scala> p.getAge
res0: Int = 0

scala> p.setAge(42)

scala> p.getAge
res1: Int = 42

scala> p.age
error:
value age is not a member of JPerson
\end{REPLnonum}
\end{multicols}
\end{Slide}


\begin{Slide}{Motsvarande JPerson men i Scala}
Så här brukar man åstadkomma ungefär motsvarande i Scala: \\~
\begin{Code}[basicstyle=\SlideFontSize{13}{15}\ttfamily\selectfont]
class Person(val name: String) {
  var age = 0
}
\end{Code}
~\\
Notera att alla attribut här är \Emph{publika}.
\end{Slide}


\begin{Slide}{Förhindra felaktiga attributvärden med setters}\SlideFontSmall
Med hjälp av \Emph{setters} kan vi förhindra \Alert{felaktig} uppdatering av attributvärden, till exempel \Alert{negativ ålder} i klassen \code{JPerson} i Java:
\begin{Code}[language=Java]
    public void setAge(int age){
        if (age >= 0) {
            this.age = age;
        else {
            this.age = 0;
        }
    }
\end{Code}
Hur kan vi åstadkomma \Emph{motsvarande i Scala}? \\
\pause
Antag att vi började med nedan variant, men \Alert{ångrar} oss och sedan vill införa funktionalitet som förhindrat negativ ålder \Emph{utan att ändra i klientkod}:
\begin{Code}
class Person(val name: String) {
  var age = 0
}
\end{Code}
Om vi inför en ny metod \code{setAge} och gör attributet \code{age} privat så funkar det \Alert{inte} längre att skriva  \code{ p.age = 42 } och vi ''kvaddar'' klientkoden! \code{  :(}
\end{Slide}



\begin{Slide}{Getters och setters i Scala}\SlideFontSmall
\setlength{\leftmargini}{0pt}
\begin{itemize}
\item Principen om \Emph{enhetlig access} tillsammans med \Alert{specialsyntax} för \Emph{setters} kommer till vår räddning!

\item
En \Emph{setter} i Scala är en \textbf{procedur som har ett namn som slutar med} \texttt{\_=}
\pause
\item I Scala kan man utan att kvadda klientkod införa getter+setter så här:
\end{itemize}
\begin{Code}
class Person(val name: String) { // ändrad implementation men samma access
  private var myPrivateAge = 0
  def age = myPrivateAge         // getter
  def age_=(a: Int): Unit =      // setter
    if (a >= 0) myPrivateAge = a else myPrivateAge = 0
}
\end{Code}
\pause\vspace{-0.5em}
\begin{REPL}
scala> val p = new Person("Björn")
p: Person = Person@28ac3dc3

scala> p.age = 42      // najs syntax om getter parad med setter enl ovan
p.age: Int = 42

scala> p.age = -1      // nu förhindras negativ ålder
p.age: Int = 0
\end{REPL}
\end{Slide}
