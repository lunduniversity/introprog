%!TEX encoding = UTF-8 Unicode
%!TEX root = ../lect-w05.tex

%%%


\Subsection{Implementation saknas: ???}

\begin{Slide}{Implementation saknas: ???}
\begin{itemize}
\item Ofta vill man bygga kod iterativt och steg för steg lägga till olika funktionalitet.

\item Standardfunktionen \code{???} ger vid anrop undantaget \Alert{\texttt{NotImplementedError}} och kan användas på platser i koden där man ännu inte är färdig.

\item \code{???} tillåter \Emph{kompilering av ofärdig kod}.

\pause

\item Undantag har bottentypen \code{Nothing} som är subtyp till \emph{alla} typer och kan därmed tilldelas referenser av godtycklig typ.

\begin{REPLnonum}
scala> lazy val sprängsSnart: Int = ???

scala> sprängsSnart + 42
scala.NotImplementedError: an implementation is missing
\end{REPLnonum}

\end{itemize}
\end{Slide}

\begin{Slide}{Exempel: ofärdig kod}
\begin{Code}[basicstyle=\SlideFontSize{9}{11}\ttfamily\selectfont]
case class Person(name: String, age: Int):
  def ärTonåring = age >= 13 && age <= 19
  def ärUng = !ärGammal
  def ärGammal: Boolean = ???   //implementation ännu ej klar
\end{Code}
\begin{REPLnonum}
scala> Person("Björn", 51).ärTonåring
res23: Boolean = false

scala> Person("Sandra", 39).ärUng
scala.NotImplementedError: an implementation is missing
\end{REPLnonum}
\end{Slide}


% \Subsection{Klass-specifikationer}
%
%
%
%
% \begin{Slide}{Specifikationer av klasser i Scala}\footnotesize
% \begin{itemize}
% \item Specifikationer av klasser innehåller information som \emph{den som ska implementera} klassen behöver veta.
% \item Specifikationer innehåller liknande information som dokumentationen av klassen (scaladoc), som beskriver vad \emph{användaren} av klassen behöver veta.
% \end{itemize}
% \begin{ScalaSpec}{Person}
% /** Encapsulate immutable data about a Person: name and age. */
% case class Person(name: String, age: Int = 0){
%   /** Tests whether this Person is more than 17 years old. */
%   def isAdult: Boolean = ???
% }
% \end{ScalaSpec}
% \begin{itemize}
% \item Specifikationer av Scala-klasser utgör i denna kurs ofullständig kod som kan kompileras utan fel.
% \item Saknade implementationer markeras med \code{???}
% \item \Emph{Dokumentationskommentarer} utgör \Alert{krav} på implementationen.
% \end{itemize}
%
% \end{Slide}
%
%
% \begin{Slide}{Specifikationer av klasser och objekt}
% \begin{ScalaSpec}{MutablePerson}
% /** Encapsulates mutable data about a person. */
% class MutablePerson(initName: String, initAge: Int){
%   /** The name of the person. */
%   def getName: String = ???
%
%   /** Update the name of the Person */
%   def setName(name: String): Unit = ???
%
%   /** The age of this person. */
%   def getAge: Int = ???
%
%   /** Update the age of this Person */
%   def setAge(age: Int): Unit = ???
%
%   /** Tests whether this Person is more than 17 years old. */
%   def isAdult: Boolean = ???
%
%   /** A string representation of this Person, e.g.: Person(Robin, 25) */
%   override def toString: String = ???
% }
% object MutablePerson {
%   /** Creates a new MutablePerson with default age. */
%   def apply(name: String): MutablePerson = ???
% }
% \end{ScalaSpec}
%
% \end{Slide}
%
% \ifkompendium
% Man brukar inte använda \code{get} och \code{set} i metodnamn i Scala. Mer senare om principen om enhetlig access \Eng{uniform access principle} och hur man gör ''setters'' som möjliggör tilldelningssyntax.
% \fi

% \ifkompendium
% \begin{Slide}{Specifikationer av Java-klasser på extentor}
% \begin{itemize}\small
% \item Specificerar signaturer för konstruktorer och metoder.
% \item Kommentarerna utgör krav på implementationen.
% \item Används flitigt på extentor i EDA016, EDA011, EDA017...
% \item Javaklass-specifikationerna \Alert{saknar} \Emph{implementationer} och behöver kompletteras med metodkroppar och klassrubriker innan de kan kompileras.
% \end{itemize}
% \begin{JavaSpec}{class Person}
% /** Skapar en person med namnet name och åldern age. */
% Person(String name, int age);

% /** Ger en sträng med denna persons namn. */
% String getName();

% /** Ändrar denna persons ålder. */
% void setAge(int age);

% /** Anger åldersgränsen för när man blir myndig. */
% static int adultLimit = 18;
% \end{JavaSpec}
% \end{Slide}
% \fi
