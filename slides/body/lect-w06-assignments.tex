%!TEX encoding = UTF-8 Unicode
%!TEX root = ../lect-w06.tex

\ifkompendium\else

\Subsection{Uppgifter denna vecka}

\begin{Slide}{Denna veckas övning: \texttt{sequences}}
\begin{itemize}\SlideFontTiny
\input{../compendium/modules/w06-sequences-exercise-goals.tex}
\end{itemize}
\end{Slide}

\begin{Slide}{Denna veckas laboration: \texttt{shuffle}}
\begin{itemize}\SlideFontSmall
%!TEX encoding = UTF-8 Unicode
%!TEX root = ../compendium2.tex

\item Kunna skapa och använda sekvenssamlingar.
\item Kunna använda sekvensalgoritmen SHUFFLE för blandning på plats av innehållet i en array.
\item Kunna registrera antalet förekomster av olika värden i en sekvens.

\end{itemize}
\end{Slide}
\fi

\begin{Slide}{Scala Build Tool: \texttt{sbt}}\SlideFontSmall
\begin{itemize}
\item Läs appendix om sbt.
\item Med enkla medel sköter \code{sbt} omkompilering och körning vid varje Ctrl+S med kommandot \code{~run} 
\begin{REPLnonum}
$ sbt
> ~run
\end{REPLnonum}
Avsluta med Enter.
\item KOd antas finnas i \code{src/main/scala} eller i aktuell katalog
\item Lägg till en enkel styrfil \code{build.sbt}
\begin{Code}
name := "hello"
scalaVersion := "2.12.13" 
\end{Code}
\item Lägg \code{.jar}-filer i katalogen \code{lib} så hamnar de automatiskt på classpath
\item Exempel på andra inställningar i \code{bild.sbt}:
\begin{CodeSmall}
scalaSource in Compile := baseDirectory.value / "src"     // ändra katalog med kod
unmanagedBase := baseDirectory.value / "../../../../lib/" // ändra jarfil-katalog
\end{CodeSmall}
Se vidare: \url{http://www.scala-sbt.org/}
\end{itemize}
\end{Slide}
