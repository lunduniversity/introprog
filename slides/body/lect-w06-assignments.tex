%!TEX encoding = UTF-8 Unicode
%!TEX root = ../lect-w06.tex

\ifkompendium\else

\Subsection{Uppgifter denna vecka}

\begin{Slide}{Denna veckas övning: \texttt{sequences}}
\begin{itemize}\SlideFontTiny
\input{../compendium/modules/w06-sequences-exercise-goals.tex}
\end{itemize}
\end{Slide}

\begin{Slide}{Denna veckas laboration: \texttt{shuffle}}
\begin{itemize}\SlideFontSmall
%!TEX encoding = UTF-8 Unicode
%!TEX root = ../compendium2.tex

\item Kunna skapa och använda sekvenssamlingar.
\item Kunna använda sekvensalgoritmen SHUFFLE för blandning på plats av innehållet i en array.
\item Kunna registrera antalet förekomster av olika värden i en sekvens.

\end{itemize}
\end{Slide}
\fi

\begin{Slide}{Scala Build Tool: \texttt{sbt}}
\begin{itemize}
\item Läs appendix om sbt.
\item Med enkla medel sköter \code{sbt} omkompilering och körning vid varje Ctrl+S med kommandot \code{~run} 
\begin{REPLnonum}
$ sbt
> ~run
\end{REPLnonum}
\item Lägg din kod i biblioteket \code{src/main/scala}
\item Lägg till en enkel styrfil \code{build.sbt}
\begin{Code}
name := "hello"
scalaVersion := "2.12.13" 
\end{Code}
\end{itemize}
\end{Slide}
