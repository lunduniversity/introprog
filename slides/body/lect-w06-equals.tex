%!TEX encoding = UTF-8 Unicode
%!TEX root = ../lect-w06.tex

%%%

\Subsection{Fördjupning: Implementera \texttt{equals}}

\ifkompendium
\noindent När du jämför värden med \code{==} anropas metoden \code{equals} som finns för alla typer. Du kan i dina egna klasser överskugga \code{equals} med en din egna definition av vad likhet ska innebära. Då är det lämpligt att använda matchning. Det är dock ett ganska omfattande arbete att implementera en korrekt likhetsjämförelse som fungerar under alla omständigheter. Ett recept för en fullständig implementation av \code{equals} ges i fördjupningen nedan. 
\fi

\begin{Slide}{Fördjupning: Implementera \texttt{equals} med \texttt{match}}
Det visar sig att \Emph{innehållslikhet} är \Alert{förvånansvärt komplicerat} att implementera, speciellt  i samband med arv.
\begin{itemize}\SlideFontSmall
\item Det enklare fallet: Gör fördjupningsuppgift \textit{''Metoden \code{equals}''} och implementera \code{equals} för innehållslikhet utan arv. \\ En bra träning på att använda \code{match}!

\item Svårare: Gör fördjupningsuppgifterna  \textit{''Överskugga \code{equals}''} och \textit{''Överskugga equals vid arv''} om du vill se hur en \Emph{komplett} \code{equals} ska se ut som fungerar \Alert{i alla lägen}.

\end{itemize}

\noindent Det krävs i denna kurs inte att du själv ska kunna implementera en generellt fungerande \code{equals}. Men du ska förstå skillnaden mellan referenslikhet och innehållslikhet. Mer om \code{equals} i fortsättningkursen, men en liten inblick i problemet nu...
\end{Slide}

\ifkompendium
\noindent Om en klass markeras \code{final} kan den ej ha några subklasser. Kompilatorn kontrollerar att detta gäller alla finala klasser och ger kompileringsfel om du försöker göra \code{extends} på en final klass. Om en klass garanterat inte har några subklasser kan implementationen av \code{equals} göra enklare.
\fi 

\begin{Slide}{Fördjupning: \texttt{equals} som fungerar för finala klasser}
Recept för implementation av \code{equals} som fungerar för typer som \Alert{inte} har några subtyper:
\begin{Code}
final class Gurka(val vikt: Int, val ärÄtbar: Boolean):
  override def equals(other: Any): Boolean = other match
    case that: Gurka => vikt == that.vikt && ärÄtbar == that.ärÄtbar
    case _ => false

  override def hashCode: Int = (vikt, ärÄtbar).## // ger bra hashcode
\end{Code}
\begin{itemize}\SlideFontSmall
\item
Du \Alert{måste} alltid överskugga \code{hashCode} också om du överskuggar \code{equals} annars funkar inte gurksamlingar (lång story ...)
\item
Notera typen \code{Any} -- detta följer hur man valde att göra i Java (tyvärr?).
\pause
\item
Ett \Alert{typsäkrare} innehållslikhetstest som \Emph{garanterat} bara jämför en gurka med en gurka och inget annat:
\begin{Code}
def ===(other: Gurka): Boolean =
  vikt == other.vikt && ärÄtbar == other.ärÄtbar
\end{Code}
\end{itemize}
\end{Slide}


\begin{Slide}{Fördjupning: Recept i 8 steg för arvssäker \code{equals}}\SlideFontTiny
%fungerar även för klasser som inte är \code{final}:
\SlideOnly{\setlength{\leftmargini}{0pt}}
\begin{enumerate}\SlideFontTiny
\item Inför denna metod: \code{ def canEqual(other: Any): Boolean}\\Observera att typen på parametern ska vara \code{Any}. Om subklass behövs \code{override}.

\item Metoden \code{canEqual} ska ge \code{true} om \code{other} är av samma typ som \code{this}, t.ex.: \\\code{override def canEqual(other: Any): Boolean = other.isInstanceOf[Gurka]}

\item Inför metoden \code{equals} och var noga med att parametern har typen \code{Any}: \\ \code{override def equals(other: Any): Boolean}

\item Implementera metoden \code{equals} med ett match-uttryck som börjar så här: \\
\code|other match { ... } |

\item Match-uttrycket ska ha två grenar. Den första grenen ska ha ett typat mönster för den klass som ska jämföras, t.ex.: \\ \code{  case that: Gurka =>}

\item Om du implementerar \code{equals} i den klass som inför \code{canEqual}, börja med: \\ \code{(that canEqual this) &&} \\
och skapa därefter en fortsättning som baseras på innehållet i klassen, t.ex.: \\ \code{this.vikt == that.vikt && this.längd == that.längd} \\
Om du överskuggar equals vill du nog börja med
 \code{super.equals(that) && }

\item Den andra grenen i matchningen ska vara:
\code{case _ => false}

\item Överskugga \code{hashCode}, t.ex. med tupel av attributvärden och metoden \code{##}: \\
\code{override def hashCode: Int  = (vikt, längd).## }

\end{enumerate}
\url{http://www.artima.com/pins1ed/object-equality.html}

\end{Slide}


\begin{Slide}{Fördjupning: Säkrare likhetstest i Scala 3}
\SlideFontSmall
\begin{itemize}
\item \Alert{Problem}: \code{equals} tar värden av vilken typ som helst.
\item Detta kallas \Alert{universell likhet}.
\item[]
\begin{REPLsmall}
scala> case class Hund(namn: String)
scala> case class Katt(namn: String)
scala> Hund("bob") == Katt("bob") // knasig jämförelse; kan aldrig bli sant
val res0: Boolean = false         // men kompilatorn låter dig göra likhetstestet
\end{REPLsmall}  
\pause
\item I Scala 3 kan du få typsäker likhetstest med~~\code{derives CanEqual}
\item Detta kalla \Emph{multiversell likhet}.
\item[]
\begin{REPLsmall}
scala> case class Hund(namn: String) derives CanEqual
scala> Hund("bob") == Katt("bob")   // tack kompilatorn för fel:
-- Error:
1 |Hund("bob") == Katt("bob")
  |^^^^^^^^^^^^^^^^^^^^^^^^^^
  |Values of types Hund and Katt cannot be compared with == or !=
\end{REPLsmall}  
\item Du \Emph{slipper} skriva \code{derives CanEqual} om du gör: \\ \code{import scala.language.strictEquality}
\item Läs mer här: \url{https://docs.scala-lang.org/scala3/reference/contextual/multiversal-equality.html}

\end{itemize}

\end{Slide}
