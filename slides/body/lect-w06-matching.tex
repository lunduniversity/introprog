%!TEX encoding = UTF-8 Unicode
%!TEX root = ../lect-w06.tex

%%%

\Subsection{Matchning}

\ifkompendium
\noindent  I ett match-uttryck kan man matcha på ett visst värde eller på en viss typ och match-uttryck används gärna istället för nästlade if-uttryck, då de ofta är lättare att läsa och begripa. Med match-uttryck kan man också göra \Emph{mönstermatchning} mot case-klass-instanser, t.ex. för att på ett smidigt sätt undersöka om attribut har speciella värden. Match-uttryck i Scala är en mer kraftfull variant av \code{switch}-satser som finns i många andra språk.  
\else
\begin{SlideExtra}{Vad är matchning?}
\includegraphics[width=0.8\textwidth]{../img/plocklada.png}
\end{SlideExtra}
\fi

\begin{Slide}{Vad är matchning?}
Matchning gör man då man vill jämföra ett värde mot andra värden och hitta överensstämmelse \Eng{match}.

\pause

\vspace{1em}\noindent Detta kan man t.ex. göra med nästlade if-else-satser/uttryck:

\begin{Code}
val g = scala.io.StdIn.readLine("grönsak:")
val smak =
  if (g == "gurka") "gott!"
  else if (g == "tomat") "jättegott!"
  else if (g == "broccoli") "ganska gott..."
  else "inte gott :("

println(g + " är " + smak)
\end{Code}
\end{Slide}




\begin{Slide}{Javas switch-sats}\SlideFontSmall
De flesta C-liknande språk (men inte Scala) har en \jcode{switch}-sats som man kan använda istället för (vissa) nästlade if-else-satser:
\javainputlisting[basicstyle=\ttfamily\SlideFontSize{5.5}{6.8}\selectfont]{../compendium/examples/match/Switch.java}

\vspace{-0.5em}\code{switch} fungerar i Java 8 bara för primitiva typer och några till (t.ex. String).
\end{Slide}




\begin{Slide}{Javas switch-sats utan break}\SlideFontSmall
Saknad \jcode{break}-sats ''faller igenom'' till efterföljande gren:

\javainputlisting[basicstyle=\ttfamily\SlideFontSize{6}{7}\selectfont]{../compendium/examples/match/SwitchNoBreak.java}
En glömd \jcode{break} kan ge svårhittad bugg...
\end{Slide}

\begin{Slide}{Javas switch-sats med glömd break}\SlideFontSmall

\vspace{-0.5em}\javainputlisting[basicstyle=\ttfamily\SlideFontSize{5.5}{6.8}\selectfont]{../compendium/examples/match/SwitchForgotBreak.java}

\vspace{-0.7em}\pause
\begin{REPL}
> java SwitchForgotBreak
Skriv grönsak:
gurka
gott!
gott!
\end{REPL}

\end{Slide}


\begin{Slide}{Scalas \texttt{match}-uttryck}\SlideFontSmall
Scala har ingen \code{switch}-sats men erbjuder i stället ett \code{match}-\Emph{uttryck} som är kraftfullare och ger ett värde.

\begin{Code}
val g = scala.io.StdIn.readLine("grönsak:")
val smak = g match {
  case "gurka" => "gott!"
  case "tomat" => "jättegott!"
  case "broccoli" => "ganska gott..."
  case _ => "mindre gott..."
}
println(g + " är " + smak)
\end{Code}
Och den ''faller inte igenom'' som Javas \code{switch}-sats!
\begin{itemize}
\pause\item Varje \code{case}-gren testas var för sig i tur och ordning uppifrån och ned.
\pause\item Det som står mellan \code{case} och \code{=>} kallas ett \Emph{mönster} \Eng{pattern}
\pause\item Sista default-grenen ovan kallas \Emph{wildcard-mönster}: \code{case _ => }
\pause\item Ovan är exempel på matchning mot \Emph{konstant-mönster}, \\ i detta fallet tre stycken strängkonstantmönster.
\pause\item Det finns många andra sätt att skriva mönster.
\end{itemize}
\end{Slide}



\begin{Slide}{Matchning med gard}
Man kan stoppa in en s.k \Emph{gard} \Eng{guard} innan pilen \code{=>} för att villkora matchningen: (notera \code{if}, parenteser behövs ej)
\begin{Code}
val g = scala.io.StdIn.readLine("grönsak:")
def f = g match {
  case "gurka" if math.random > 0.5 => "gott ibland!"
  case "tomat" => "jättegott!"
  case "broccoli" => "ganska gott..."
  case _ => "mindre gott..."
}
\end{Code}
\code{case}-grenen med gard ger bara en lyckad matchning \\ om uttrycket efter \code{if} är sant; annars provas nästa gren, etc.
\end{Slide}

\begin{Slide}{Matchning med variabelmönster}\SlideFontSmall
Om det finns ett namn efter \code{case} som börjar med liten begynnelsebokstav, blir detta namn en variabel som automatiskt binds till uttrycket före \code{match}:

\begin{Code}
val g = scala.io.StdIn.readLine("grönsak:")
def f = g match {
  case "gurka" if math.random > 0.5 => "gott ibland!"
  case "tomat" => "jättegott!"
  case "broccoli" => "ganska gott..."
  case other => "smakar bakvänt: " + other.reverse
}
\end{Code}

Ett enkelt variabelmönster, så som \\ \code{case other => ...} \\ i exemplet ovan, matchar \Emph{allt}! \\\code{other} får alltså värdet av \code{g} om \code{g} \Alert{inte} är \code{"gurka"}, \code{"tomat"}, \code{"broccoli"}.

\end{Slide}


\begin{Slide}{Matchning med eller-mönster}\SlideFontSmall
Om man har samma utfall för olika grenar kan dessa slås ihop och mönstret separeras med vertikalstreck: \code{|}
\begin{Code}
val g = scala.io.StdIn.readLine("grönsak:")
def f = g match {
  case "gurka" => "gott"
  case "tomat" => "gott"
  case "lök"   => "gott"
  case _ => "inte gott"
}
\end{Code}

Mer koncist med eller-mönster:

\begin{Code}
val g = scala.io.StdIn.readLine("grönsak:")
def f = g match {
  case "gurka" | "tomat" | "lök" => "gott"
  case _ => "inte gott"
}
\end{Code}



\end{Slide}





\begin{Slide}{Matchning med typade mönster}\SlideFontSmall
Med en typannotering efter en variabel får man ett \Emph{typat mönster} \Eng{typed pattern}. Om matchningen lyckas blir värdet \Alert{omvandlat} till den specifika typen och binds till variabeln.
\begin{Code}
def f = if (math.random < 0.5) 42 + math.random else "gurka" + math.random
def g = f match {
  case x: Double => x.round.toInt
  case s: String => s.length
}
\end{Code}
Vad har funktionen \code{f} för returtyp? \\ \pause
Matchning mot specifika typer enl. ovan används i idiomatisk Scala hellre än \code{isInstanceOf} men man kan göra motsvarande ovan med if-uttryck:
\begin{Code}
def g2 = {
  val x = f
  if (x.isInstanceOf[Double]) x.asInstanceOf[Double].round.toInt
  else if (x.isInstanceOf[String]) x.asInstanceOf[String].length
}.asInstanceOf[Int]
\end{Code}
\end{Slide}


\begin{Slide}{Konstruktormönster med case-klasser}\SlideFontSmall
En basklass med gemensamma delar och två subtyper:
\begin{Code}
trait Grönsak {
  def vikt: Int
  def ärRutten: Boolean
}
case class Gurka(vikt: Int, ärRutten: Boolean) extends Grönsak
case class Tomat(vikt: Int, ärRutten: Boolean) extends Grönsak
\end{Code}
\pause
Tack vare case-klasserna kan man använda \Emph{konstruktormönster} \Eng{constructor pattern} för att kolla vad som finns \Alert{inuti} en instans:
\begin{Code}
def testa(g: Grönsak): String = g match {
  case Gurka(v, false) => "gott, väger " + v
  case Gurka(_, true)  => "inte gott"
  case Tomat(v, r)     => (if (r) "inte " else "") + "gott, väger " + v
  case _ => "okänd grönsak: " + g
}
\end{Code}

Konstruktormönster ''\Emph{plockar isär}'' det som matchas och binder variabler till de attribut som finns i case-klassens konstruktor.
\end{Slide}


\begin{Slide}{Plocka isär samlingar med djupa mönster}
Man kan plocka isär innehållet i en samling så här:
\begin{Code}
def visa(xs: Vector[Grönsak]): String = xs match {
  case Vector()               => "tom grönsaksvektor"
  case Vector(Gurka(v, true)) => "en rutten gurka som väger " + v
  case Vector(g)              => "exakt en grönsak: " + g
  case Vector(g1, g2)         => s"exakt två grönsaker: $g1, $g2"
  case g +: gs                => s"först en $g och sedan svansen: $gs"
}
\end{Code}
Övning: prova ovan i REPL. Vad händer om du byter ordning på andra och tredje mönstret ovan?
\end{Slide}

\begin{Slide}{Matchning på tupler}
Det går fint att plocka isär tupler med mönstermatchning:\footnote{\url{https://youtu.be/aboZctrHfK8}}
\begin{Code}
val pair = ("hej", 42)

pair match {
  case (a, b) if b == 42 => s"livets mening är funnen: $a"
  case (_, b)            => s"fattas mening: $b"
}
\end{Code}

\end{Slide}

\begin{Slide}{Mönstermatchning och case-objekt}
En bastyp och specifika singelobjekt av gemensam typ:
\begin{Code}
trait Färg
case object Spader  extends Färg
case object Hjärter extends Färg
case object Ruter   extends Färg
case object Klöver  extends Färg

def parallellFärg(f: Färg): Färg = f match {
  case Spader  => Klöver
  case Klöver  => Spader
  case Hjärter => Ruter
}
\end{Code}
Vilken case-gren har vi glömt? Kan kompilatorn hjälpa oss?
\pause
\begin{REPL}
scala> parallellFärg(Ruter)
scala.MatchError: Ruter (of class Ruter)
  at .parallellFärg(<console>:18)

\end{REPL}
\Alert{Undantag vid körtid} \code{:(}
\end{Slide}

\begin{Slide}{Mönstermatchning och förseglade typer}
Med nyckelordet \code{sealed} får vi en kompileringsvarning.
\begin{Code}
sealed trait Färg
case object Spader  extends Färg
case object Hjärter extends Färg
case object Ruter   extends Färg
case object Klöver  extends Färg

def parallellFärg(f: Färg): Färg = f match {
  case Spader  => Klöver
  case Klöver  => Spader
  case Hjärter => Ruter
}
\end{Code}
\begin{REPL}
<console>:23: warning: match may not be exhaustive.
It would fail on the following input: Ruter
       def parallellFärg(f: Färg): Färg = f match {
\end{REPL}
\Emph{Varning vid kompilering} \code{:)}
\end{Slide}

\begin{Slide}{Stora/små begynnelsebokstäver vid matchning}
\Alert{Fallgrop}: matcha \Alert{värde} som börjar med \Alert{liten} bokstav.
\begin{REPL}
scala> val livetsMening = 42

scala> def ärLivetsMeningBuggig(svar: Int) = svar match {
         case livetsMening => true    // lokalt namn som matchar allt!
         case _ => false
       }

scala> ärLivetsMeningBuggig(43)
res0: Boolean = true

scala> val LivetsMening = 42   // stor begynnelsebokstav

scala> def ärLivetsMening(svar: Int) = svar match {
         case LivetsMening => true    // funkar fint!
         case _ => false
       }

scala> ärLivetsMening(43)
res1: Boolean = false
\end{REPL}
\end{Slide}


\begin{Slide}{Stora/små begynnelsebokstäver vid matchning}
Ett sätt att komma runt problemet med liten begynnelsebokstav: \\
\Emph{backticks} to the rescue!
\begin{REPL}
scala> val livetsMening = 42

scala> def ärLivetsMeningBackTicks(svar: Int) = svar match {
         case `livetsMening` => true    // nu funkar det!
         case _ => false
       }

scala> ärLivetsMeningBackTicks(43)
res2: Boolean = false
\end{REPL}
\end{Slide}


\begin{Slide}{Mönster på andra ställen än i \texttt{match}}\SlideFontSmall
Mönster i \Emph{deklarationer}:
\vspace{-0.25em}\begin{REPL}
scala> case class Point(x: Int, y: Int)

scala> val p = Point(0, 1)

scala> val Point(x, y) = p          // konstruktormönster med case-klass
x: ???
y: ???

scala> val (x, y, z) = (0, 1, 2)    // konstruktormönster med tupel
x: ???
y: ???
z: ???

\end{REPL}
Mönster i \Emph{for-satser}:
\vspace{-0.25em}\begin{REPL}
scala> val xs = for ((x, y) <- Vector((1,2), (3,4))) yield x
xs: ???
\end{REPL}

\end{Slide}

\begin{Slide}{Mönster på andra ställen än i \texttt{match}}\SlideFontSmall
Mönster i \Emph{deklarationer}:
\vspace{-0.25em}\begin{REPL}
scala> case class Point(x: Int, y: Int)

scala> val p = Point(0, 1)

scala> val Point(x, y) = p          // konstruktormönster med case-klass
x: Int = 0
y: Int = 1

scala> val (x, y, z) = (0, 1, 2)    // konstruktormönster med tupel
x: Int = 0
y: Int = 1
z: Int = 2

\end{REPL}
Mönster i \Emph{for-satser}:
\vspace{-0.25em}\begin{REPL}
scala> val xs = for ((x, y) <- Vector((1,2), (3,4))) yield x
xs: scala.collection.immutable.Vector[Int] = Vector(1, 3)
\end{REPL}
\end{Slide}

\begin{Slide}{Fördjupning om mönster (ingår ej på tentan)}\SlideFontSmall
\begin{itemize}
\item binda variabler till mönsterdelar med \code{@} \\
\code{Vector(xs@Vector(a), 42)}

\item sekvensmönster med \code{_} och \code{_*} \\ \code{Vector(a, _, b)} och \code{Vector(a, _*)}

\item partiella funktioner: \code|val pf: Int => Double = { case z if z != 0 => 1/z }|

\item Läs mer om mönster här:  \href{http://www.artima.com/pins1ed/case-classes-and-pattern-matching.html}{\SlideFontTiny www.artima.com/pins1ed/case-classes-and-pattern-matching.html}

\item För djupare förståelse av hur \code{case} fungerar, läs speciellt om \Emph{partiella funktioner} här: \href{http://www.artima.com/pins1ed/case-classes-and-pattern-matching.html\#15.7}{\SlideFontTiny www.artima.com/pins1ed/case-classes-and-pattern-matching.html\#15.7}

\item Läs om extractors här: \href{http://www.artima.com/pins1ed/extractors.html}{\SlideFontTiny www.artima.com/pins1ed/extractors.html}

\end{itemize}
\end{Slide}


\begin{Slide}{Fördjupning: metoden \texttt{unapply}}\SlideFontSmall
När du deklarerar en case-klass kommer kompilatorn att \Alert{automatiskt generera en metod} med namnet \Emph{\texttt{unapply}} som kan plocka isär instansen.
\begin{REPL}
scala> case class Gurka(vikt: Int, ärRutten: Boolean)

scala> Gurka.unapply // tryck TAB två gånger för att se metodhuvudet
 case def unapply(x$0: Gurka): Option[(Int, Boolean)]

scala> val g = Gurka(100, false)

scala> Gurka.unapply(g)
res0: Option[(Int, Boolean)] = Some((100,false))
\end{REPL}
Vi ska snart se hur \code{Option} kan hantera värden som \Emph{eventuellt} \Alert{saknas}. \\
\pause

{\SlideFontTiny\vspace{1em}\emph{Fördjupning:} Ett anrop av metoden \code{unapply} genereras av kompilatorn vid matchning och det är det som gör att case-klasser kan användas i konstruktormönster. Man kan skapa en egna s.k. extraktorer \Eng{extractors} som funkar i matchningar (se övn. 22).}
\end{Slide}
