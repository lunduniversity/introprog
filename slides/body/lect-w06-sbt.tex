%!TEX encoding = UTF-8 Unicode
%!TEX root = ../lect-w06.tex

\Subsection{Scala Build Tool \texttt{sbt} för smidig omkompilering}
\begin{Slide}{Scala Build Tool: \texttt{sbt}}\SlideFontSmall
\begin{itemize}
\item Du kan låta \code{sbt} automatiskt sköta omkompilering och körning vid varje Ctrl+S med kommandot \code{~run}
och enbart omkompilering med \code{~compile}
\begin{REPL}
> sbt              # kör igång sbt i en katalog med .scala-filer
[info] Set current project to hellosbt (in build file:/home/bjornr/tmp/hellosbt/)
[info] sbt server started at 127.0.0.1:5320
sbt:hellosbt> ~run
\end{REPL}
Avsluta med Enter. Om flera \code{main} gör: \code{ ~runMain DettaMainObj}
\item Kod antas finnas direkt i \Emph{aktuell katalog} eller i \code{src/main/scala}
\item Lägg \code{.jar}-filer i en katalog \code{lib} så hamnar de automatiskt på classpath
\item Om du vill: skapa en fil med namnet \code{build.sbt} som konfigurerar \code{sbt}
\begin{Code}
name := "hellosbt"           // bestäm namnet på mitt projekt
scalaVersion := "2.12.13"    // bestäm version av Scala-kompilatorn
\end{Code}
\item Exempel på andra inställningar i \code{bild.sbt}:
\begin{CodeSmall}
scalaSource in Compile := baseDirectory.value / "src"      // bestäm kodkatalog
unmanagedBase := baseDirectory.value / "../annanstans/lib" // bestäm jarfil-katalog
\end{CodeSmall}
Se vidare: Appendix F i kompendiet, \url{http://www.scala-sbt.org/}
\end{itemize}
\end{Slide}
