%!TEX encoding = UTF-8 Unicode
%!TEX root = ../lect-w07.tex


\Subsection{Hierarki av samlingstyper}

\begin{Slide}{Repetition: Vad är en samling?}
En \Emph{samling} \Eng{collection} är en datastruktur som kan innehålla många element av \Alert{samma typ}.

\pause
\vspace{2em}\emph{Exempel:} \\Heltalsvektor: \hfill\code{val xs = Vector(2, -1, 3, 42, 0)}\\
\pause
Härledd typ: \hfill\code{scala.collection.immutable.Vector[Int]}


\pause
{\SlideFontSmall\vspace{2em}Samlingar implementeras med hjälp av klasser. \\ I standardbiblioteken \code{scala.collection} och \code{java.util} finns \Alert{många} \Emph{färdiga samlingar}, så man behöver sällan implementera egna.

\pause\vspace{0.5em}\emph{Om} man behöver en egen, speciell datastruktur är det ofta lämpligt att skapa en klass som \emph{innehåller} en \emph{färdig} samling och utgå från dess färdiga metoder.

}
\end{Slide}



\begin{Slide}{Typparameter möjliggör generiska samlingar}\SlideFontSmall

\begin{itemize}
  \item Med \Emph{generisk} \Eng{generic} kod menar man att koden kan hantera data av \Alert{godtycklig} typ.
  \item Funktioner och klasser kan, förutom vanliga parametrar, även ha \Emph{typparametrar} som skrivs i en \Alert{egen} parameterlista med \Alert{hakparenteser} i stället för vanliga parenteser.

  \item En typparameter gör så att funktioner och datastrukturer blir \Emph{generiska}.

  \item Exempel: Funktionerna \code{baklänges} 1--4 nedan är ordnade från specifik typ till mer generell typ.

\begin{Code}
def baklänges1(xs: Vector[Int]): Vector[Int] = xs.reverse

def baklänges2[T](xs: Vector[T]): Vector[T] = xs.reverse

def baklänges3(xs: Seq[T]): Vector[T] = xs.reverse.toVector

def baklänges4(xs: Seq[T]): Seq[T] = xs.reverse  //reverse avgör samling
\end{Code}
\item Mer om typparametrar i w08.
\end{itemize}
\end{Slide}



\begin{Slide}{Hierarki av samlingstyper i \texttt{scala.collection}}

\begin{multicols}{2}
\begin{tikzpicture}[sibling distance=6.1em,->,>=stealth', inner sep=3pt, %scale=0.5,
  every node/.style = {shape=rectangle, draw, align=center,font=\small\ttfamily},
  class/.style = {fill=blue!20},
  trait/.style = {rounded corners, fill=red!20}]
  \node[trait] {Traversable}
    child { node[trait] {Iterable}
      child { node[trait] {Seq}
       }
      child { node[trait] {Set}
      }
      child { node[trait] {Map}
      }
    };
\end{tikzpicture}

\columnbreak

{\SlideFontTiny

\code{Traversable} har metoder som är implementerade med hjälp av: \\
\code{def foreach[U](f: Elem => U): Unit}\\

\vspace{1em}\code{Iterable} har metoder som är implementerade med hjälp av: \\
\code{def iterator: Iterator[A] }

}

\begin{itemize}\SlideFontTiny
\item[] \code{Seq}: ordnade i sekvens
\item[] \code{Set}: unika element
\item[] \code{Map}: par av (nyckel, värde)
\end{itemize}


\end{multicols}

{\SlideFontSmall Samlingen \Emph{\texttt{Vector}} är en \code{Seq} som är en \code{Iterable} som är en \code{Traversable}. \\ \vspace{0.5em}\pause
De konkreta samlingarna är uppdelade i dessa paket:\\
\code{scala.collection.immutable} \hfill som är \Emph{automatiskt} importerade\\
\code{scala.collection.mutable}  \hfill som \Alert{måste importeras} explicit\\\pause
(undantag: primitiva \code{scala.Array} som är automatiskt synlig)
}
\end{Slide}







\begin{Slide}{Använda \texttt{iterator} -- primitiv loop över element}\SlideFontSmall
Med en \code{iterator} kan man \Emph{iterera} med \code{while} över alla element, men endast \Alert{en   gång}; sedan är iteratorn ''förbrukad''. (Men man kan be om en ny.)
\begin{REPL}
scala> val xs = Vector(1,2,3,4)
xs: scala.collection.immutable.Vector[Int] = Vector(1, 2, 3, 4)

scala> val it = xs.iterator
it: scala.collection.immutable.VectorIterator[Int] = non-empty iterator

scala> while (it.hasNext) print(it.next)
1234

scala> it.hasNext
res1: Boolean = false

scala> it.next
java.util.NoSuchElementException: reached iterator end
  at scala.collection.immutable.VectorIterator.next(Vector.scala:674)
\end{REPL}
\Emph{Normalt} behöver man \Alert{inte} använda \code{iterator}: det finns oftast färdiga metoder som gör det man vill, till exempel \code{foreach}, \code{map}, \code{sum}, \code{min} etc.
\end{Slide}







\begin{Slide}{Hierarki av samlingar i scala.collection}\SlideFontTiny
\includegraphics[width=0.95\textwidth]{../img/collection/collection-traits}\\
%\noindent Läs mer om Scalas samlingar här: \\
\url{http://docs.scala-lang.org/overviews/collections/overview}
\end{Slide}




\Subsection{Mängd} %%%%%%%%%%%%%%%%%%%%%%%%%%%%%%%%%%%%%%%%%%%%%%%%%%%%%%


\begin{Slide}{Vad är en mängd?}
\begin{itemize}
\item En \Emph{mängd} är en samling \Alert{unika} element av en viss \Alert{typ}.
\item En mängd kan alltså inte innehålla dubbletter:
\begin{REPLnonum}
scala> Set(1,1,2,2,3,3,4,4,5,5)
res0: scala.collection.immutable.Set[Int] =
  Set(5, 1, 2, 3, 4)
\end{REPLnonum}
\item En mängd är \Alert{inte en sekvens}: elementen ligger ej i följd och har ingen speciell ordning; en mängd har ej längd, men en \Emph{storlek}; metoden \code{size} ger antalet element.
\item En mängd kan vara \Alert{tom} och har då storleken \code{0}.
\pause
\item En mängd kan ses som ett predikat för innehållstest:\\alltså en funktion \code{T => Boolean} som är \code{true} om elementet finns annars \code{false}
\end{itemize}
\end{Slide}


\begin{Slide}{Exempel: Oföränderlig mängd}
\setlength{\leftmargini}{1em}
\begin{itemize}
\item \Emph{Skapa}:
\begin{REPLnonum}
scala> var xs = Set("gurka", "tomat", "banan", "pingvin")
\end{REPLnonum}

\item \Emph{Läsa}: avgöra medlemskap
\begin{REPLnonum}
scala> xs("gurka")
res1: Boolean = true
\end{REPLnonum}

\item \Emph{Uppdatera}: lägg till element (händer inget om redan finns)
\begin{REPLnonum}
scala> xs = xs + "jordekorre"
\end{REPLnonum}

\item \Emph{Ta bort}: (om finns, annars händer inget)
\begin{REPLnonum}
scala> xs = xs - "gurka"
\end{REPLnonum}
\end{itemize}
{\SlideFontTiny\code{SLUT} = Skapa, Läsa, Uppdatera, Ta bort \hfill\code{CRUD} = Create, Read, Update, Delete}
\end{Slide}







\begin{Slide}{Mängd: snabb innehållstest, garanterat dubblettfri}\SlideFontSmall
En \Emph{mängd} är snabb på att avgöra om ett element \Alert{finns eller inte} i mängden. Ingen linjärsökning sker eftersom den implementationen medger direkt uppslagning av element.

\begin{REPL}
scala> var veg = Set.empty[String]
veg: scala.collection.immutable.Set[String] = Set()

scala> veg = veg + "Gurka"
veg: scala.collection.immutable.Set[String] = Set(Gurka)

scala> veg = veg ++ Set("Broccoli", "Tomat", "Gurka")
veg: scala.collection.immutable.Set[String] = Set(Gurka, Broccoli, Tomat)

scala> veg.contains("Gurka")
res0: Boolean = true

scala> veg.apply("Gurka")   // samma som contains
res1: Boolean = true

scala> veg("Morot")
res2: Boolean = false
\end{REPL}

\end{Slide}



\begin{Slide}{Mysteriet med de försvunna elementen}
Vad händer här?
\begin{REPLnonum}
scala> val xs1 = Vector(1,2,3,4,5,6)
scala> xs1.map(_ % 2).count(_ == 0)
res0: Int = 3                          // antalet jämna tal
scala> val xs2 = Set(1,2,3,4,5,6)
scala> xs2.map(_ % 2).count(_ == 0)
res1: Int = 1                          // varför?
\end{REPLnonum}
\pause
Mängdegenskaper ger att \code{mängd == Set(0, 1)}\\
Fundera alltid noga på om du \Alert{riskerar att förlora duplikat} som du egentligen hade velat behålla!\\
\pause
Använd \code{toSeq} på mängd om du behöver sekvensegenskaper:
\begin{REPLnonum}
scala> xs2.toSeq.map(_ % 2).count(_ == 0)
res1: Int = 3         // med toSeq blir det som vi ville
\end{REPLnonum}

\end{Slide}




\Subsection{Nyckel-värde-tabell} %%%%%%%%%%%%%%%%%%%%%%%%%%%%%%%%%%%%%%%%


\begin{Slide}{Vad är en nyckel-värde-tabell?}\SlideFontSmall
\begin{itemize}
\item En \Emph{nyckel-värde-tabell} är en samling element som är \Alert{par} med:\\
en \Emph{nyckel} av någon typ \code{K} och ett \Emph{värde} av någon typ \code{V}.
\item En sådan tabell kan skapas ur en sekvens av par \code{(k, v)}\\
där \code{k} är en nyckel och \code{v} är ett värde:
\begin{REPL}
scala> val ålder = Map("Björn" -> 42, "Sandra" -> 35, "Kim" -> 19)
ålder: scala.collection.immutable.Map[String,Int] =
  Map(Björn -> 42, Sandra -> 35, Kim -> 19)
\end{REPL}
\item Tabellens nycklar utgör en mängd som ges av metoden \code{keySet};\\
nycklarna är alltså \Alert{unika}.
\item Elementen utgör \Alert{inte en sekvens} och har ingen speciell ordning;
\\en nyckel-värde-tabell har ej längd, men en \Emph{storlek};\\metoden \code{size} ger antalet element.
\pause
\item En tabell kan ses som en uppslagsfunktion \Eng{dictionary}:\\alltså en funktion \code{K => V} som ger ett värde givet en nyckel.
\end{itemize}
\end{Slide}

%
% \begin{Slide}{Exempel: Nyckel-värde-tabell \TODO}
% \setlength{\leftmargini}{1em}
% \begin{itemize}
% \item \Emph{Skapa}:
% \begin{REPLnonum}
% scala> var xs = ???
% \end{REPLnonum}
%
% \item \Emph{Läsa}: slå upp värde utifrån nyckel
% \begin{REPLnonum}
% scala> xs("gurka")
% \end{REPLnonum}
%
% \item \Emph{Uppdatera}: lägg till mappning (händer inget om redan finns)
% \begin{REPLnonum}
% scala> xs = ???
% \end{REPLnonum}
%
% \item \Emph{Ta bort}: ny tabell utan mappning (om finns, annars händer inget)
% \begin{REPLnonum}
% scala> xs = xs - ???
% \end{REPLnonum}
% \end{itemize}
% {\SlideFontTiny\code{SLUT} = Skapa, Läsa, Uppdatera, Ta bort \hfill\code{CRUD} = Create, Read, Update, Delete}
% \end{Slide}




\begin{Slide}{Den fantastiska nyckel-värde-tabellen \texttt{Map}}\SlideFontSmall
\begin{itemize}
\item En \Emph{nyckel-värde-tabell} \Eng{key-value table} är en slags generaliserad vektor där man kan ''indexera'' med godtycklig typ.

\item Kallas öven \href{https://sv.wikipedia.org/wiki/Hashtabell}{\Emph{hashtabell}} \Eng{hash table}, \Emph{lexikon} \Eng{dictionary} eller kort och gott \Emph{mapp} \Eng{map},

\item En hashtabell är en \Emph{samling av par}, där varje par består av en \Alert{unik} \Emph{nyckel} och ett tillhörande \Emph{värde}.

\item Om man vet nyckeln kan man få fram värdet \Alert{snabbt}, på liknande sätt som indexering sker i en vektor om man vet heltalsindex.

\item Denna datastruktur är \Alert{mycket användbar} och liknar en enkel databas.
\end{itemize}
\begin{REPL}
scala> val födelse = Map("C" -> 1972,  "C++" -> 1983, "C#" -> 2000,
  "Scala" -> 2004, "Java" -> 1995, "Javascript" -> 1995, "Python" -> 1991)

födelse: scala.collection.immutable.Map[String,Int] = Map(Scala -> 2004, C# -> 2000, Python -> 1991, Javascript -> 1995, C -> 1972, C++ -> 1983, Java -> 1995)

scala> födelse.apply("Scala")
res0: Int = 2004

scala> födelse("Java")
res1: Int = 1995

\end{REPL}
\end{Slide}

\begin{Slide}{Exempel nyckel-värde-tabell}
\begin{REPL}
scala> val färg = Map("gurka" -> "grön", "tomat"->"röd", "aubergine"->"lila")
färg: scala.collection.immutable.Map[String,String] =
  Map(gurka -> grön, tomat -> röd, aubergine -> lila)

scala> färg("gurka")
res0: String = grön

scala> färg.keySet
res1: scala.collection.immutable.Set[String] = Set(gurka, tomat, aubergine)

scala> val ärGrönSak = färg.map(elem => (elem._1, elem._2 == "grön"))
ärGrönSak: Map[String,Boolean] = Map(gurka -> true, tomat -> false, aubergine -> false)

scala> val baklängesFärg = färg.mapValues(s => s.reverse)
baklängesFärg: Map[String,String] = Map(gurka -> nörg, tomat -> dör, aubergine -> alil)

\end{REPL}
\begin{itemize}
\item \code{xs.keySet} ger en mängd av alla nycklar
\item \code{xs.map(f)} mappar funktionen f på alla par av (key, value)
\item \code{xs.mapValues(f)} mappar funktionen f på alla värden
\end{itemize}

\end{Slide}



\Subsection{\texttt{scala.collection}} %%%%%%%%%%%%%%%%%%%%%%%%%%%%%%%%%%%



\begin{Slide}{Mer specifika samlingstyper i \texttt{scala.collection}}
Det finns \Alert{mer specifika} \Emph{subtyper} av \code{Seq}, \code{Set} och \code{Map}:
\\ \vspace{1em}

\begin{tikzpicture}[sibling distance=5.8em,->,>=stealth', inner sep=3pt, %scale=0.5,
  every node/.style = {shape=rectangle, draw, align=center,font=\small\ttfamily},
  class/.style = {fill=blue!20},
  trait/.style = {rounded corners, fill=red!20}]
  \node[trait] {Traversable}
    child { node[trait] {Iterable}
      child { node[trait, xshift=-2.4cm] {Seq}
        child { node[trait] {IndexedSeq} }
        child { node[trait] {LinearSeq} }
       }
      child { node[trait, yshift=-0.0cm] {Set}
        child { node[trait] {SortedSet} }
        child { node[trait] {BitSet} }
      }
      child { node[trait, xshift=1.0cm] {Map}
        child { node[trait] {SortedMap} }
      }
    };
\end{tikzpicture}

\vspace{0.5em}
\Emph{\texttt{Vector}} är en \Alert{\texttt{IndexedSeq}} medan
\Emph{\texttt{List}} är en \Alert{\texttt{LinearSeq}}.
\end{Slide}

\begin{Slide}{Några oföränderliga och förändringsbara sekvenssamlingar}\SlideFontSmall
\begin{tabular}{r l l}
\texttt{scala.collection.\Emph{immutable}.Seq.} & & \\
 & \code|IndexedSeq.| & \\
 & & \Emph{\texttt{Vector}} \\
 & & \Emph{\texttt{Range}} \\
 & \code|LinearSeq.| & \\
 & & \Emph{\texttt{List}} \\
   & & \Emph{\texttt{Queue}} \\

\texttt{scala.collection.\Alert{mutable}.Seq.} & & \\
 & \code|IndexedSeq.| & \\
 & & \Alert{\texttt{ArrayBuffer}} \\
 & & \Alert{\texttt{StringBuilder}} \\
 & \code|LinearSeq.| & \\
 & & \Alert{\texttt{ListBuffer}} \\
   & & \Alert{\texttt{Queue}} \\
\end{tabular}

Studera samlingars egenskaper här: \href{http://docs.scala-lang.org/overviews/collections/overview}{docs.scala-lang.org/overviews/collections/overview}
\end{Slide}





\begin{Slide}{Några användbara metoder på samlingar}\SlideFontTiny
\begin{tabular}{r r l}\hline
\texttt{\Emph{Traversable}}
  & \code|xs.size| & antal elementet \\
  & \code|xs.head| & första elementet \\
  & \code|xs.last| & sista elementet \\
  & \code|xs.take(n)| & ny samling med de första n elementet \\
  & \code|xs.drop(n)| & ny samling utan de första n elementet \\
  & \code|xs.foreach(f)| & gör \code|f| på alla element, returtyp \code|Unit|\\
  & \code|xs.map(f)| & gör \code|f| på alla element, ger ny samling \\
  & \code|xs.filter(p)| & ny samling med bara de element där p är sant\\
  & \code|xs.groupBy(f)| & ger en \code|Map| som grupperar värdena enligt f\\
  & \code|xs.mkString(",")| & en kommaseparerad sträng med alla element\\ \hline

\texttt{\Emph{Iterable}}
  & \code|xs.zip(ys)| & ny samling med par (x, y); ''zippa ihop'' xs och ys \\
  & \code|xs.zipWithIndex| & ger en \code|Map| med par (x, index för x) \\
  & \code|xs.sliding(n)| & ny samling av samlingar genom glidande ''fönster''\\ \hline

\texttt{\Emph{Seq}}
  & \code|xs.length| & samma som \code|xs.size| \\
  & \code|xs :+ x| & ny samling med x sist efter xs \\
  & \code|x +: xs| & ny samling med x före xs \\ \hline

\end{tabular}

\pause
\vspace{0.5em}\Emph{Minnesregel} för \code{+:} och \code{:+  } \Alert{Colon on the collection side}

\pause
Prova fler samlingsmetoder ur snabbreferensen: ~~\url{http://cs.lth.se/quickref}
\end{Slide}



\begin{Slide}{Använda samlingsmetoder}
\begin{REPL}
scala> val tal = Vector(1,4,7,9,42)
tal: scala.collection.immutable.Vector[Int] = Vector(1, 4, 7, 9, 42)

scala> val jämna = tal.filter(_ % 2 == 0)
jämna: scala.collection.immutable.Vector[Int] = Vector(4, 42)

scala> val xs = Vector(("Kim","Smith"), ("Kim", "Jones"), ("Robin", "Smith"))
xs: scala.collection.immutable.Vector[(String, String)] = Vector((Kim,Smith), (Kim,Jones), (Robin,Smith))

scala> val grupperaEfterFörnamn = xs.groupBy(_._1)
grupperaEfterFörnamn: Map[String,Vector[(String, String)]] =
Map(Kim -> Vector((Kim,Smith), (Kim,Jones)), Robin -> Vector((Robin,Smith)))

scala> val grupperaEfterEfternamn = xs.groupBy(_._2)
grupperaEfterEfternamn: Map[String,Vector[(String, String)]] =
Map(Jones -> Vector((Kim,Jones)), Smith -> Vector((Kim,Smith), (Robin,Smith)))

\end{REPL}
\end{Slide}



\begin{Slide}{Metoderna zipWithIndex, groupBy och mapValues}
\begin{REPL}
scala> val högaKort = Vector("Knekt", "Dam", "Kung", "Äss")

scala> val kortIndex = högaKort.zipWithIndex.toMap
kortIndex: Map[String,Int] = Map(Knekt -> 0, Dam -> 1, Kung -> 2, Äss -> 3)

scala> kortIndex("Kung") > kortIndex("Knekt")
res0: Boolean = true

scala> val xs = Vector(("Kim","Smith"), ("Kim", "Jones"), ("Robin", "Smith"))
xs: Vector[(String, String)] = Vector((Kim,Smith), (Kim,Jones), (Robin,Smith))

scala> val grupperaEfterFörnamn = xs.groupBy(_._1)
grupperaEfterFörnamn: Map[String,Vector[(String, String)]] =
Map(Kim -> Vector((Kim,Smith), (Kim,Jones)), Robin -> Vector((Robin,Smith)))

scala> val grupperaEfterEfternamn = xs.groupBy(_._2)
grupperaEfterEfternamn: Map[String,Vector[(String, String)]] =
Map(Jones -> Vector((Kim,Jones)), Smith -> Vector((Kim,Smith), (Robin,Smith)))

scala> val frekvens = xs.groupBy(_._1).mapValues(_.size)
frekvens: Map[String,Int] = Map(Kim -> 2, Robin -> 1)
\end{REPL}
\end{Slide}




\begin{Slide}{scala.collection.immutable}
\includegraphics[width=0.82\textwidth]{../img/collection/collection-immutable}
\includegraphics[width=0.33\textwidth]{../img/collection/collection-legend}
\end{Slide}


\begin{Slide}{scala.collection.mutable}
\includegraphics[width=1.05\textwidth]{../img/collection/collection-mutable}
\end{Slide}


\begin{Slide}{Strängar är implicit en \texttt{IndexedSeq[Char]}}\SlideFontSmall
Det finns en så kallad \Emph{implicit konvertering} mellan \code{String} och \code{IndexedSeq[Char]} vilket gör att \Alert{alla samlingsmetoder på \texttt{Seq} även funkar på strängar} och även flera andra smidiga strängmetoder erbjuds \Alert{utöver} de som finns i \href{http://docs.oracle.com/javase/8/docs/api/java/lang/String.html}{\code{java.lang.String}} genom klassen \href{http://www.scala-lang.org/api/current/#scala.collection.immutable.StringOps}{\code{StringOps}}.

\vspace{0.5em}
\begin{REPLnonum}
scala> "hej".  //tryck på TAB och se alla strängmetoder
\end{REPLnonum}
Detta är en stor fördel med Scala jämfört med många andra språk, som har strängar som inte kan allt som andra sekvenssamlingar kan.
\end{Slide}


% \begin{Slide}{\texttt{Vector} eller \texttt{List}?}\SlideFontTiny
% {\href{http://stackoverflow.com/questions/6928327/when-should-i-choose-vector-in-scala}{stackoverflow.com/questions/6928327/when-should-i-choose-vector-in-scala}}
%
% \begin{enumerate}
% \item If we only need to transform sequences by operations like map, filter, fold etc: basically it does not matter, we should program our algorithm generically and might even benefit from accepting parallel sequences. For sequential operations List is probably a bit faster. But you should benchmark it if you have to optimize.
%
% \item If we need a lot of random access and different updates, so we should use vector, list will be prohibitively slow.
%
% \item If we operate on lists in a classical functional way, building them by prepending and iterating by recursive decomposition: use list, vector will be slower by a factor 10-100 or more.
%
% \item If we have an performance critical algorithm that is basically imperative and does a lot of random access on a list, something like in place quick-sort: use an imperative data structure, e.g. ArrayBuffer, locally and copy your data from and to it.
% \end{enumerate}
% {\href{http://stackoverflow.com/questions/20612729/how-does-scalas-vector-work}{stackoverflow.com/questions/20612729/how-does-scalas-vector-work}}\\
% Mer om tids- och minneskomplexitet i fördjupningskursen och senare kurser.
% \end{Slide}




\begin{Slide}{Speciella metoder på förändringsbara samlingar}\SlideFontSmall
Både \code{Set} och \code{Map} finns i \Alert{förändringsbara} varianter med extra metoder för uppdatering av innehållet ''på plats'' utan att nya samlingar skapas.
\begin{REPL}
scala> import scala.collection.mutable

scala> val ms = mutable.Set.empty[Int]
ms: scala.collection.mutable.Set[Int] = Set()

scala> ms += 42
res0: ms.type = Set(42)

scala> ms += (1, 2, 3, 1, 2, 3); ms -= 1
res1: ms.type = Set(2, 42, 3)

scala> ms.mkString("Mängd: ", ", ", " Antal: " + ms.size)
res2: String = Mängd: 1, 2, 42, 3 Antal: 4

scala> val ordpar = mutable.Map.empty[String, String]
scala> ordpar += ("hej" -> "svejs", "abra" -> "kadabra", "ada" -> "lovelace")
scala> println(ordpar("abra"))
kadabra
\end{REPL}
\end{Slide}

\begin{Slide}{Fler exempel på samlingsmetoder}
Exempel: räkna bokstäver i ord.  \\
Undersök vad som händer i REPL:
\begin{Code}[basicstyle=\SlideFontSize{9}{13}\ttfamily]
val ord = "sex laxar i en laxask sju sjösjuka sjömän"
val uppdelad = ord.split(' ').toVector
val ordlängd = uppdelad.map(_.length)
val ordlängdMap = uppdelad.map(s => (s, s.size)).toMap
val grupperaEfterFörstaBokstav = uppdelad.groupBy(s => s(0))
val bokstäver = ord.toVector.filter(_ != ' ')
val antalX = bokstäver.count(_ == 'x')
val grupperade = bokstäver.groupBy(ch => ch)
val antal = grupperade.map(kv => (kv._1, kv._2.size))
val sorterat = antal.toVector.sortBy(_._2)
val vanligast = antal.maxBy(_._2)
\end{Code}
\end{Slide}


\begin{Slide}{Jobba med föränderlig samling lokalt; \\ returnera oföränderlig samling när du är klar}
\SlideFontSmall
Om du vill implementera en imperativ algoritm med en föränderlig samling:\\
Gör gärna detta \Alert{lokalt} i en \Alert{förändringsbar} samling och returnera sedan en \Emph{oföränderlig} samling, genom att köra t.ex. \code{toSet} på en mängd, eller \code{toMap} på en hashtabell, eller \code{toVector} på en \code{ArrayBuffer} eller \code{Array}.

\begin{REPL}
scala> :paste
def kastaTärningTillsAllaUtfallUtomEtt(sidor: Int = 6) = {
  val s = scala.collection.mutable.Set.empty[Int]
  var n = 0
  while (s.size < sidor - 1) {
    s += (math.random * sidor + 1).toInt
    n += 1
  }
  (n, s.toSet)
}
scala> kastaTärningTillsAllaUtfallUtomEtt()
res0: (Int, scala.collection.immutable.Set[Int]) = (13,Set(5, 1, 6, 2, 3))

\end{REPL}

\end{Slide}
