%!TEX encoding = UTF-8 Unicode
%!TEX root = ../lect-w07.tex

\ifkompendium\else

% \Subsection{Grumligtlådan}

% \begin{Slide}{Grumligtlådan}
%   \Alert{OM} du har något koncept i kursen som är \Alert{extra} ''grumligt'' så skriv ner det på en lapp i lådan som går runt. 
% \end{Slide}

% \begin{Slide}{Grumligtlådan: topplista med ämnen 2017}
% \begin{multicols}{2}
% \begin{verbatim}
%   om kursen       : 14
%   klass           : 9
%   this            : 7
%   filstruktur     : 5
%   pluggteknik     : 4
%   case-klass      : 4
%   registrera      : 3
%   filtrera        : 3
%   copy            : 3
%   tupel           : 2
%   konstruktor     : 2
%   iterera         : 2
%   fabriksmetod    : 2
%   synlighet       : 1
%   skuggning       : 1
%   sidoeffekt      : 1
%   Seq[T]          : 1
%   sekvenser       : 1
%   sats/uttryck    : 1
%   lambda          : 1
%   kompanjonsobj   : 1
%   högre ordn funk : 1
%   funktion        : 1
%   curry-funktion  : 1
% \end{verbatim}
% \end{multicols}
% \end{Slide}




% \begin{Slide}{Grumligtlådan: lajvkodning}
% Lajvkodning enligt begrepp i lådan utifrån denna klass:
% \begin{Code}[basicstyle=\ttfamily\SlideFontSize{12}{14}]
% class Frog(var n: Int) { def hop = n += 1 }
% \end{Code}
% \includegraphics[width=0.6\textwidth]{../img/frog}
% \end{Slide}

\fi
