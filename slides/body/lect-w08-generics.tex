%!TEX encoding = UTF-8 Unicode
%!TEX root = ../lect-w08.tex

\Subsection{Typparametrar}



\begin{Slide}{Exempel: Icke-generisk case-klass med heltalsmatris}
  En \emph{icke-generisk} datastruktur har inga obundna typparametrar; alla typer är \Emph{konkreta} (alltså specifika). \\~\\ En icke-generisk case-class med en \code{Vector[Vector[Int]]}:
  \begin{Code}
  case class Matrix(data: Vector[Vector[Int]]):
    def apply(x: Int, y: Int): Int = data(x)(y)
  \end{Code}

  \begin{REPL}
  scala> Matrix(Vector(Vector(5, 2, 42, 4, 5),Vector(3, 4, 18, 6, 7)))
  res0: Matrix =
    Matrix(Vector(Vector(5, 2, 42, 4, 5), Vector(3, 4, 18, 6, 7)))
  \end{REPL}

\end{Slide}





\begin{Slide}{Exempel: Generisk case-klass med generell matris}
  En \emph{generisk} datastruktur har minst en obunden \Emph{typparameter} som kan bindas  till ett \Alert{konkret} \Emph{typargument}.
  
  \begin{Code}
  case class Matrix[T](data: Vector[Vector[T]]):
    def apply(x: Int, y: Int): T = data(x)(y)
  \end{Code}
  \code{Matrix} i exemplet ovan är en \Emph{generisk} case-class där \code{T} är obunden, eftersom \code{T} är en typparameter deklarerad inom \code{[]} \Alert{efter} klassens namn men \Alert{före} klassparameterlistan. \\

  \vspace{0.5em} Användning där \code{T} binds till \code{Int} via kompilatorns typhärledning:
  \begin{REPL}
  scala> Matrix(Vector(Vector(5, 2, 42, 4, 5),Vector(3, 4, 18, 6, 7)))
  res1: Matrix[Int] =
    Matrix(Vector(Vector(5, 2, 42, 4, 5), Vector(3, 4, 18, 6, 7)))
  \end{REPL}

\end{Slide}




\begin{Slide}{Vad är en typparameter?}\SlideFontSmall
  \setlength{\leftmargini}{0pt}

\begin{itemize}
\item En \Emph{typparameter} gör det möjligt att ge ett \Emph{typargument}.
\item Detta kallas \Emph{parametrisk polymorfism} \Eng{paramteric polymorphism}.
\item Exempel: \Emph{generisk} \Alert{funktion}:
\begin{Code}
def tnirp[A](x: A):Unit = println(x.toString.reverse)
\end{Code}
\pause
\item En \Emph{fri} typparameter kan bindas till vilken typ som helst.
\item Bindingen av typargument till typparametrar sker vid \Alert{kompileringstid}.
\item En typparameter är \Emph{fri} om den \Alert{inte} fått något värde, annars \Emph{bunden}. 
\pause
\item Exempel: \Emph{generisk} \Alert{klass} med \Emph{generiska} \Alert{metoder}:
\begin{Code}
class Cell[A](   // [A] är fri (måste bindas vid användning) 
    var value: A):                              // A är bunden
  def update(a: A): Unit = value = a            // A är bunden
  def replaced[B](b: B): Cell[B] = new Cell(b)  // första [B] är fri
\end{Code}
\pause
\item \Alert{Skuggning kan förekomma}: Om \code{replaced} i \code{Cell} hade använt namnet A på sin typparameter hade den \Emph{skuggat} klassens typparameter och tolkats som en  fri typparameter, alltså en godtycklig typ och \Alert{inte} klassens typparameter. (jämför  namnöverskuggning vid \Emph{lokala} namn i nästlade block)
\end{itemize}

\end{Slide}

\ifkompendium\else
\begin{Slide}{Exempel: Generisk funktion}
Vad händer här?
\begin{REPL}

scala> def skrikBaklänges(x: T): String = x.toString.toUpperCase.reverse
???



scala> def skrikBaklänges[T](x: T): String = x.toString.toUpperCase.reverse

scala> skrikBaklänges("gurka är gott")
val res0: ???

\end{REPL}
\end{Slide}


\begin{Slide}{Exempel: Generisk funktion}
Vad händer här?
\begin{REPL}

scala> def skrikBaklänges(x: T): String = x.toString.toUpperCase.reverse
1 |def skrikBaklänges(x: T): String = x.toString.toUpperCase.reverse
  |                      ^
  |                      Not found: type T
                             ^

scala> def skrikBaklänges[T](x: T): String = x.toString.toUpperCase.reverse

scala> skrikBaklänges("gurka är gott")
val res0: ???
\end{REPL}
\end{Slide}
\fi

\begin{Slide}{Exempel: Generisk funktion}
Vad händer här?
\begin{REPL}

scala> def skrikBaklänges(x: T): String = x.toString.toUpperCase.reverse
1 |def skrikBaklänges(x: T): String = x.toString.toUpperCase.reverse
  |                      ^
  |                      Not found: type T
                             ^

scala> def skrikBaklänges[T](x: T): String = x.toString.toUpperCase.reverse

scala> skrikBaklänges("gurka är gott")
val res0: String = TTOG RÄ AKRUG
\end{REPL}
Om ingen typparameter deklareras inom hakparenteser efter funktionens namns så vet inte kompilatorn vad \code{T} är för en typ. Men med en typparameter \code{[T]} efter funktionsnamnet tolkar kompilatorn funktionen som \Emph{generisk} och typen \code{T} bestäms av argumentets typ \Alert{vid anrop} och \code{T} kan bindas till godtycklig typ.
\end{Slide}


\begin{Slide}{Exempel: Generisk case-klass}
\SlideFontSmall
En generisk klass har en eller flera typparametrar efter klassnamnet:
\begin{Code}
case class Box[A](value: A)  
\end{Code}

Kompilatorn härleder typparameterarnas typ utifrån givna värden. 
\begin{REPL}
scala> Box("gurka")  
val res1: Box[String] = Box(gurka)
\end{REPL}

Du kan också ge typpparametern en typ explicit:
\begin{REPL}
scala> Box[Int](42)  // 
val res3: Box[Int] = Box(42)
\end{REPL}

Om typen inte stämmer får du hjälp av kompilatorn att hitta felet:
\begin{REPL}
scala> Box[String](42)
-- Error:
1 |Box[String](42)
  |            ^^
  |            Found:    (42 : Int)
  |            Required: String
\end{REPL}
\end{Slide}






\begin{Slide}{Fallgrop: Typradering \Eng{type erasure}}\SlideFontSmall
Informationen om typerna i typparametrar raderas innan kodgenerering för JVM av prestandaskäl och \Alert{typparametrar saknas vid runtime} i bytekoden.
\vspace{-0.25em}\begin{REPL}
scala> def isIntVector[T](xs: Vector[T]) = xs.isInstanceOf[Vector[Int]]
-- Warning:
1 |def isIntVector[T](xs: Vector[T]) = xs.isInstanceOf[Vector[Int]]
  |                                    ^^^^^^^^^^^^^^^^^^^^^^^^^^^^
  |                the type test for Vector[Int] cannot be checked at runtime
def isIntVector[T](xs: Vector[T]): Boolean

scala> isIntVector(Vector("hej"))
res42: Boolean = true  // AAAARGHH!! :(
\end{REPL}
Måste ''packa upp'' samlingen och typtesta alla element:
\begin{REPL}
scala> def isIntVector[T](xs: Vector[T]) = xs.forall(_.isInstanceOf[Int])

scala> isIntVector(Vector("hej"))
res43: Boolean = false  // FUNKAR :)

\end{REPL}
Typkontroll vid körtid görs oftast hellre med \code{match}.

\end{Slide}

\Subsection{Upptäcka och åtgärda buggar}

\begin{Slide}{Testning och avlusning}
%\TODO 
\begin{itemize}
\item Läs om testning och avlusning \Eng{debugging} i Appendix D: ''Fixa buggar'' 
\item Träna på println-debugging
\item Prova debuggern i VS code
%\item Visa hur testramverket ska funka som du ska skapa på övning och använda på labb
%\item sbt testOnly och andra sätt att köra testfall
%\item Visa hur en fördröjning kan skapas med en s.k. thunk 
%\item Visa hur printlndebugging
%\item Visa hur debugga i vs code
\end{itemize}
\end{Slide}


% \ifkompendium\else


% \begin{Slide}{Typparametrar på tentan?}
% \begin{itemize}
% \item Det ingår att kunna använda färdiga generiska strukturer med specifika typer, t.ex. \code{Vector[Int]}

% \item Det ingår att kunna skapa abstraktioner med specifika typparametrar, t.ex. metoder eller klasser som tar en vektor med en specifik typ som parameter:\\
% \code{case class X(x: Vector[Int])}


% \item Det ingår \Alert{inte} på tentan att kunna skapa generiska metoder eller klasser, t.ex.: \\
% \code{def f[T](x: Vector[T]): Vector[T] = ???} \\
% Mer om generiska strukturer i fördjupningskursen!
% \end{itemize}
% \end{Slide}

% \fi
