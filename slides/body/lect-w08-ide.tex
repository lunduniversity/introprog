%!TEX encoding = UTF-8 Unicode
%!TEX root = ../lect-w08.tex

\Subsection{Integrerad utvecklingsmiljö (IDE)}

\begin{Slide}{Välja IDE}\SlideFontSmall
\begin{itemize}
\item En \Emph{integrerad utvecklingsmiljö} \Eng{Integrated Development Environment, IDE} innehåller \\ editor + kompilator + debugger + en massa annat\\och gör utvecklingen enklare när man lärt sig alla finesser.

\item Läs om vad en IDE kan göra i appendix.

\pause

\item På LTH:s datorer finns två populära IDE installerade:
\begin{enumerate}\SlideFontSmall

\item \Emph{IntelliJ IDEA} (välj installera Scala-plugin när du kör första gången)
\begin{REPL}[numbers=none]
> idea
\end{REPL}

\item \Emph{Eclipse} med plugin \Emph{ScalaIDE} förinstallerad
\begin{REPL}[numbers=none]
> scalaide
\end{REPL}

\end{enumerate}
Läs mer om dessa i appendix innan du väljer vilken du vill lära dig.
\\Där står även hur du installerar dem på din egen dator.
\\IntelliJ anses av många för tillfället ha det bästa Scala-stödet, men är du van vid Eclipse så kanske du vill använda ScalaIDE.
\end{itemize}
\end{Slide}

\begin{Slide}{IntelliJ IDEA med Scala-plugin}
\includegraphics[width=\textwidth]{../img/intellij/idea-hello.png}
\end{Slide}

\begin{Slide}{Eclipse med ScalaIDE}
\includegraphics[width=\textwidth]{../img/eclipse/eclipse-pirates-hello.png}
\end{Slide}

\ifkompendium\else
\begin{Slide}{Omkontrollskrivning}
  \begin{itemize}
    % \item \Alert{EXTRAUNDERVISNING} för de som har det \Alert{allra svårast}
    % \begin{itemize}
    %   \item Nu på \Emph{torsdag} den 7/11 kl \Alert{8-10} i \Emph{E:2116}
    %   \item Behovsstyrd behandling av grundläggande koncept från lp1.
    %   \item Anmälan: {\small \url{https://forms.gle/996tQWZ8kKeAMFuL7}}
    %   \item De med \textbf{mindre än 15p} på kontrollskrivningen har \Emph{förtur} och rekommenderas \Alert{starkt} att delta!
    %   \item Medtag papper, penna och snabbref.
    % \end{itemize}
    \item \Alert{OMKONTROLLSKRIVNING} för de som inte fullföljde ordinarie kontrollskrivning:
    \begin{itemize}
      \item Måndagen den 16/11 kl 12:15-16:00 i Canvas
      \item Förbered: legitimation, blyertspenna, minst 10 blanka vita papper, röd penna, förtäring.
      %\item Karta till ''glasburen'':{\url{https://fileadmin.cs.lth.se/cs/Bilder/Salar/E2405.pdf} }
    \end{itemize}
  \end{itemize}
\end{Slide}
\fi 