%!TEX encoding = UTF-8 Unicode
%!TEX root = ../lect-w08.tex

%%%

\Subsection{Veckans labb: \texttt{life}}

\begin{Slide}{Veckans labb: \texttt{life}}
\begin{minipage}{0.52\textwidth}
  \setlength{\leftmargini}{0pt}

\begin{itemize}
  \SlideFontSmall
\item Universum är en binär matris av \Emph{celler} där \Emph{levande} celler representeras med \code{true} och \Alert{döda} med \code{false}.
\item Följande regler gäller för \Emph{nästa generation} celler i universum:
\begin{itemize}\SlideFontTiny
  \item \textbf{Fortlevnad}: en levande cell med 2 eller 3 grannar \Emph{lever vidare}
  \item \textbf{Död}: en levande cell med färre än 2 eller fler än 3 grannar \Alert{dör}
  \item \textbf{Födelse}: en död cell med exakt tre grannar föds
\end{itemize}
\item Övning \code{matrices} uppgift 5: skapa en generisk \code{case class Matrix[T]}
\item På labben: använd \code{Matrix[Boolean]}
\end{itemize}

\end{minipage}%
\begin{minipage}{0.5\textwidth}
  \includegraphics[width=1.0\textwidth]{../img/glider-blinker-block}

  \begin{itemize}\SlideFontTiny
  \item Du ska simulera \emph{Game of Life} i ett \code{introprog.PixelWindow}
  \item Fördjupning:\\{\SlideFontTiny\url{https://en.wikipedia.org/wiki/Conway%27s_Game_of_Life}}
  \end{itemize}
\end{minipage}%

\end{Slide}






\Subsection{Matriser}

\begin{Slide}{Vad är en matris?}\SlideFontSmall
\begin{itemize}

\item En \Emph{matris} inom \Alert{matematiken} innehåller \Emph{rader} och \Emph{kolumner}\footnote{även kallade \emph{kolonner}} med tal.

\item I en \Alert{matematisk} matris har alla rader \Emph{lika många} element och

\item även alla kolumner har \Emph{lika många} element.

\item En matris av dimension $2\times{}5$ har $2 \cdot 5 = 10$ stycken element.

\item Exempel på en matematisk matris av dimension $2\times{}5$:
\[
M_{2,5}=
  \begin{pmatrix}
    5 & 2 & 42 & 4 & 5 \\
    3 & 4 & 18 & 6 & 7
  \end{pmatrix}
\]
\end{itemize}
\end{Slide}

\begin{Slide}{Indexering i en matris}\SlideFontSmall
\begin{itemize}

  \item En matris av dimension $m\times{}n$ har $m \cdot n$ stycken element.

  \item En matris $A_{m,n}$ av dimension $m\times{}n$ ritas inom matematiken ofta så här:

  \[
  A_{m,n} =
   \begin{pmatrix}
    a_{1,1} & a_{1,2} & \cdots & a_{1,n} \\
    a_{2,1} & a_{2,2} & \cdots & a_{2,n} \\
    \vdots  & \vdots  & \ddots & \vdots  \\
    a_{m,1} & a_{m,2} & \cdots & a_{m,n}
   \end{pmatrix}
  \]


\item Matrisindexering inom matematiken sker ofta från $1$, men ofta från $0$ i datorprogram.

\item Vad har talet $42$ för index i matrisen $M_{2,5}$ nedan?
\begin{itemize}\SlideFontTiny
  \item[--] Inom matematiken?
  \item[--] I Scala och Java och många andra språk?

  \[
  M_{2,5}=
    \begin{pmatrix}
      5 & 2 & 42 & 4 & 5 \\
      3 & 4 & 18 & 6 & 7
    \end{pmatrix}
  \]
\end{itemize}
\end{itemize}
\end{Slide}

\begin{Slide}{Hur skapa matriser?}
  \setlength{\leftmargini}{0pt}

  \begin{itemize}
  \item Inom programmering används ordet \Emph{matris} ofta för att beteckna en \Alert{nästlad struktur} i två dimensioner. Exempel:
  \begin{itemize}
   \item \Emph{Oföränderliga} sekvenser, t.ex. \code{Vector[Vector[Int]]} \\
   \code{val xss = Vector(Vector(0, 0, 0), Vector(0, 0, 0))} eller enklare: \\
      \code{val xss = Vector.fill(2,3)(0)}

    \item \Alert{Föränderliga} sekvens, t.ex. \code{Array[Array[Int]]} \\
    \code{val yss = Array(Array(0, 0, 0), Array(0, 0, 0))} eller enklare: \\
       \code{val yss = Array.fill(2,3)(0)}

  \end{itemize}

\end{itemize}
\end{Slide}

\begin{Slide}{Hur indexera i matriser?}
En matris med array av arrayer:
\begin{REPL}
scala> val xss = Array(Array(5,2,42,4,5),Array(3,4,18,6,7))
xss: Array[Array[Int]] = Array(Array(5, 2, 42, 4, 5), Array(3, 4, 18, 6, 7))
\end{REPL}
\pause
Man indexerar i en nästlad sekvens med upprepad \code{apply}:
\begin{REPL}
scala> xss(0)(2)
res0: ???

scala> xss.apply(0).apply(2)
res1: ???

scala> xss(0)
res2: ???
\end{REPL}
Övning: Vad är typ och värde vid \code{???} ovan?
\end{Slide}

\begin{Slide}{Hur indexera i matriser?}
En matris med array av arrayer:
\begin{REPL}
scala> val xss = Array(Array(5,2,42,4,5),Array(3,4,18,6,7))
xss: Array[Array[Int]] = Array(Array(5, 2, 42, 4, 5), Array(3, 4, 18, 6, 7))
\end{REPL}

Man indexerar i en nästlad sekvens med upprepad \code{apply}:
\begin{REPL}
scala> xss(0)(2)
res0: Int = 42

scala> xss.apply(0).apply(2)
res1: Int = 42

scala> xss(0)
res2: Array[Int] = Array(5, 2, 42, 4, 5)
\end{REPL}
Övning: Rita en bild av minnet som referensen \code{xss} refererar till.

\end{Slide}

\begin{Slide}{Uppdatering av en förändringsbar nästlad struktur}
Man kan förändra en array av arrayer ''på plats'' med tilldelning:
\begin{REPL}
scala> val xss = Array(Array(5,2,42,4,5),Array(3,4,18,6,7))

scala> xss(0)(0) = 100

scala> xss
res0: ???

scala> xss(0)(2) = xss(0)(2) - 1

scala> xss
res1: ???

scala> xss(1) = Array.fill(5)(-1)

scala> xss
res2: ???
\end{REPL}
\end{Slide}

\begin{Slide}{Uppdatering av en förändringsbar nästlad struktur}
Man kan förändra en array av arrayer ''på plats'' med tilldelning:
\begin{REPL}
scala> val xss = Array(Array(5,2,42,4,5),Array(3,4,18,6,7))

scala> xss(0)(0) = 100

scala> xss
res0: Array[Array[Int]]=Array(Array(100, 2, 42, 4, 5), Array(3, 4, 18, 6, 7))

scala> xss(0)(2) = xss(0)(2) - 1

scala> xss
res1: Array[Array[Int]]=Array(Array(100, 2, 41, 4, 5), Array(3, 4, 18, 6, 7))

scala> xss(1) = Array.fill(5)(-1)

scala> xss
res2: Array[Array[Int]]=Array(Array(100, 2, 41, 4, 5), Array(-1,-1,-1,-1,-1))
\end{REPL}
\end{Slide}

\begin{Slide}{Några olika sätt att skapa förändringsbara matriser}\SlideFontSmall
Det jobbiga, primitiva sättet:
\begin{REPL}
scala> val xss = new Array[Array[Int]](2)
xss: Array[Array[Int]] = Array(null, null)

scala> for (i <- xss.indices) {xss(i) = new Array[Int](5)}

scala> xss
res0: Array[Array[Int]] = Array(Array(0, 0, 0, 0, 0), Array(0, 0, 0, 0, 0))

scala> println(xss)
[[I@196a99d0
\end{REPL}
Enklare sätt:
\begin{REPL}
scala> val xss = Array.ofDim[Int](2,5)
xss: Array[Array[Int]] = Array(Array(0, 0, 0, 0, 0), Array(0, 0, 0, 0, 0))
\end{REPL}
Enklare och tydligare sätt, där initialvärdet anges explicit:
\begin{REPL}
scala> val xss = Array.fill(2,5)(0)
xss: Array[Array[Int]] = Array(Array(0, 0, 0, 0, 0), Array(0, 0, 0, 0, 0))
\end{REPL}

\end{Slide}

\begin{Slide}{Exempel på skapande av oföränderlig nästlad struktur}\SlideFontSmall
Om du kan beräkna initialvärde direkt, använd \code{Vector.fill}:\\
{\SlideFontTiny\code{def fill[A](n1: Int, n2: Int)(elem: => A): Vector[Vector[A]]}}
\begin{REPL}
scala> Vector.fill(2,5)(scala.util.Random.nextInt(6) + 1)
res0:
  typ???
  värde???

\end{REPL}
Om du kan beräkna initialvärde ur index, använd \code{Vector.tabulate}:\\
{\SlideFontTiny\code{def tabulate[A](n1: Int, n2: Int)(f: (Int, Int) => A): Vector[Vector[A]]}}
\begin{REPL}
scala> Vector.tabulate(5,2)((x,y) => x + y + 1)
res1:
  typ???
  värde???

\end{REPL}
\end{Slide}

\begin{Slide}{Exempel på skapande av oföränderlig nästlad struktur}\SlideFontSmall
Om du kan beräkna initialvärde direkt, använd \code{Vector.fill}:\\
{\SlideFontTiny\code{def fill[A](n1: Int, n2: Int)(elem: => A): Vector[Vector[A]]}}
\begin{REPL}
scala> Vector.fill(2,5)(scala.util.Random.nextInt(6) + 1)
res0: Vector[Vector[Int]] =
  Vector(Vector(1, 2, 6, 2, 1), Vector(1, 4, 3, 3, 2))

\end{REPL}
Om du kan beräkna initialvärde ur index, använd \code{Vector.tabulate}:\\
{\SlideFontTiny\code{def tabulate[A](n1: Int, n2: Int)(f: (Int, Int) => A): Vector[Vector[A]]}}
\begin{REPL}
scala> Vector.tabulate(5,2)((x,y) => x + y + 1)
res1: Vector[Vector[Int]] =
  Vector(Vector(1,2), Vector(2,3), Vector(3,4), Vector(4,5), Vector(5,	6))

\end{REPL}
\end{Slide}



\begin{Slide}{Uppdatering av en oföränderlig nästlad struktur}\SlideFontSmall
Uppdatering av endimensionell struktur med \code{xs.updated}:\\
{\SlideFontTiny\code{def updated[A](index: Int, elem: A): Vector[A]} }
\begin{REPL}
scala> var xs = Vector.tabulate(5)(x => x + 1)
xs: typ??? = värde???

scala> xs = xs.updated(1, 42)
xs: typ??? = värde???
\end{REPL}

Uppdatering av nästlad struktur i två dimensioner:
\begin{REPL}
scala> var xss = Vector.tabulate(2, 5)((x,y) => x + y + 1)
xss:
  typ??? =
  värde???

scala> xss = xss.updated(0, xss(0).updated(1, 42))
xss:
  typ??? =
  värde???
\end{REPL}

\end{Slide}



\begin{Slide}{Uppdatering av en oföränderlig nästlad struktur}\SlideFontSmall
Uppdatering av endimensionell struktur med \code{xs.updated}:\\
{\SlideFontTiny\code{def updated[A](index: Int, elem: A): Vector[A]} }
\begin{REPL}
scala> var xs = Vector.tabulate(5)(x => x + 1)
xs: Vector[Int] = Vector(1, 2, 3, 4, 5)

scala> xs = xs.updated(1, 42)
xs: Vector[Int] = Vector(1, 42, 3, 4, 5)
\end{REPL}

Uppdatering av nästlad struktur i två dimensioner:
\begin{REPL}
scala> var xss = Vector.tabulate(2, 5)((x,y) => x + y + 1)
xss: Vector[Vector[Int]] =
  Vector(Vector(1, 2, 3, 4, 5), Vector(2, 3, 4, 5, 6))

scala> xss = xss.updated(0, xss(0).updated(1, 42))
xss:
  Vector[Vector[Int]] =
  Vector(Vector(1, 42, 3, 4, 5), Vector(2, 3, 4, 5, 6))
\end{REPL}

\end{Slide}


\begin{Slide}{Iterera över nästlad struktur}\SlideFontSmall
Behandling av nästlade strukturer kräver ofta algoritmer med nästlad iterering. \\
Exempel: iterera med nästlad \code{for}-sats för utskrift av denna matris\\
\code{val xss = Vector.tabulate(2,5)((x,y) => x + y + 1)}
\pause
\begin{REPL}
scala> for ??? do
         for ??? do 
           print(xss(i)(j))
           print(" ")
         println

1 2 3 4 5
2 3 4 5 6
\end{REPL}
Övning: \\Vad ska det stå vid \code{???} för att alla element ska skrivas ut?
\end{Slide}

\begin{Slide}{Iterera över nästlad struktur}\SlideFontSmall
  \vspace{1em}
  Behandling av nästlade strukturer kräver ofta algoritmer med nästlad iterering. \\
  Exempel: iterera med nästlad \code{for}-sats för utskrift av denna matris \\
  \code{val xss = Vector.tabulate(2,5)((x,y) => x + y + 1)}

  \begin{REPL}
scala> for xs <- xss do
         for x <- xs do 
           print(x)
           print(" ")
         end for
         println()
       end for

1 2 3 4 5
2 3 4 5 6
\end{REPL}
Övning: skriv ut matrisen med nästlad \code{foreach}\\
\pause
\begin{Code}
xss.foreach { xs => 
  xs.foreach { x => print(x); print(" ") }
  println()
}
\end{Code}
\end{Slide}


\begin{Slide}{Övningsexempel: Yatzy}\SlideFontSmall
Skapa en funktion \code{roll} som ger utfallet av n st tärningskast:
\begin{REPL}
scala> import scala.util.Random

scala> def roll(n: Int): Vector[Int] = ???
\end{REPL}

Skapa en funktion \code{isYatzy} som ger \code{true} om alla utfall är lika:
\begin{REPL}
scala> def isYatzy(xs: Vector[Int]): Boolean = ???
\end{REPL}
Du kan anta att xs.length > 0\\
Tips: använd metoden xs.forall: \\
\code{def forall[A](p: A => Boolean): Boolean }
\end{Slide}


\begin{Slide}{Övningsexempel: Yatzy}\SlideFontSmall
Skapa en funktion \code{roll} som ger utfallet av n st tärningskast:
\begin{REPL}
scala> import scala.util.Random

scala> def roll(n: Int): Vector[Int] = Vector.fill(n)(Random.nextInt(6) + 1)
\end{REPL}

Skapa en funktion \code{isYatzy} som ger \code{true} om alla utfall är lika:
\begin{REPL}
scala> def isYatzy(xs: Vector[Int]): Boolean = xs.forall(x => x == xs(0))
\end{REPL}
Du kan anta att xs.length > 0\\
Tips: använd metoden xs.forall: \\
\code{def forall[A](p: A => Boolean): Boolean }
\end{Slide}

\begin{Slide}{Iterera över nästlad struktur: for-sats}\SlideFontSmall
Iterera med nästlad for-sats: (vad har xss för typ?)
\begin{REPL}
scala> val xss = Vector.fill(100)(roll(5))

scala> for (i <- ???) do 
         for (j <- ???) do
           print(s"($i)($j): ${xss(i)(j)} ")
         println(s" YATZY: ${isYatzy(xss(i))}")

(0)(0): 3 (0)(1): 6 (0)(2): 4 (0)(3): 4 (0)(4): 6  YATZY: false
(1)(0): 4 (1)(1): 1 (1)(2): 5 (1)(3): 2 (1)(4): 6  YATZY: false
(2)(0): 1 (2)(1): 3 (2)(2): 5 (2)(3): 6 (2)(4): 2  YATZY: false
(3)(0): 2 (3)(1): 1 (3)(2): 1 (3)(3): 5 (3)(4): 4  YATZY: false
(4)(0): 4 (4)(1): 4 (4)(2): 1 (4)(3): 6 (4)(4): 5  YATZY: false
(5)(0): 3 (5)(1): 3 (5)(2): 2 (5)(3): 3 (5)(4): 6  YATZY: false
(6)(0): 3 (6)(1): 6 (6)(2): 1 (6)(3): 1 (6)(4): 4  YATZY: false
(7)(0): 6 (7)(1): 2 (7)(2): 4 (7)(3): 4 (7)(4): 3  YATZY: false
(8)(0): 1 (8)(1): 5 (8)(2): 4 (8)(3): 2 (8)(4): 4  YATZY: false
(9)(0): 1 (9)(1): 1 (9)(2): 3 (9)(3): 6 (9)(4): 6  YATZY: false
(10)(0): 2 (10)(1): 5 (10)(2): 2 (10)(3): 4 (10)(4): 5  YATZY: false
(11)(0): 3 (11)(1): 4 (11)(2): 2 (11)(3): 5 (11)(4): 6  YATZY: false
...
\end{REPL}
\end{Slide}

\begin{Slide}{Iterera över nästlad struktur: for-sats}\SlideFontSmall
Iterera med nästlad for-sats: (xss är en \code{Vector[Vector[Int]]})
\begin{REPL}
scala> val xss = Vector.fill(100)(roll(5))

scala> for (i <- xss.indices) do 
         for (j <- xss(i).indices) do
           print(s"($i)($j): ${xss(i)(j)} ")
         println(s" YATZY: ${isYatzy(xss(i))}")

(0)(0): 3 (0)(1): 6 (0)(2): 4 (0)(3): 4 (0)(4): 6  YATZY: false
(1)(0): 4 (1)(1): 1 (1)(2): 5 (1)(3): 2 (1)(4): 6  YATZY: false
(2)(0): 1 (2)(1): 3 (2)(2): 5 (2)(3): 6 (2)(4): 2  YATZY: false
(3)(0): 2 (3)(1): 1 (3)(2): 1 (3)(3): 5 (3)(4): 4  YATZY: false
(4)(0): 4 (4)(1): 4 (4)(2): 1 (4)(3): 6 (4)(4): 5  YATZY: false
(5)(0): 3 (5)(1): 3 (5)(2): 2 (5)(3): 3 (5)(4): 6  YATZY: false
(6)(0): 3 (6)(1): 6 (6)(2): 1 (6)(3): 1 (6)(4): 4  YATZY: false
(7)(0): 6 (7)(1): 2 (7)(2): 4 (7)(3): 4 (7)(4): 3  YATZY: false
(8)(0): 1 (8)(1): 5 (8)(2): 4 (8)(3): 2 (8)(4): 4  YATZY: false
(9)(0): 1 (9)(1): 1 (9)(2): 3 (9)(3): 6 (9)(4): 6  YATZY: false
(10)(0): 2 (10)(1): 5 (10)(2): 2 (10)(3): 4 (10)(4): 5  YATZY: false
(11)(0): 3 (11)(1): 4 (11)(2): 2 (11)(3): 5 (11)(4): 6  YATZY: false
...
\end{REPL}
\end{Slide}


% \begin{Slide}{Iterera över nästlad struktur med nästlad foreach}\SlideFontSmall
% Iterera med nästlad foreach-sats:
% \begin{REPL}
% scala> val xss = Vector.tabulate(2,5)((x,y) => x + y + 1)

% xss.foreach{ xs => ??? ; println }

% 1 2 3 4 5
% 2 3 4 5 6
% \end{REPL}
% \end{Slide}


% \begin{Slide}{Iterera över nästlad struktur med nästlad foreach}\SlideFontSmall
% Iterera med nästlad foreach-sats:
% \begin{REPL}
% scala> val xss = Vector.tabulate(2,5)((x,y) => x + y + 1)

% xss.foreach{ xs => xs.foreach{ x => print(x + " ") }; println }

% 1 2 3 4 5
% 2 3 4 5 6
% \end{REPL}
% \end{Slide}


\begin{Slide}{Nästlade for-uttryck}\SlideFontSmall
Iterera med \Emph{nästlad for-yield}:\\
%Statisk typ: \code{IndexedSeq[IndexedSeq[[Int]]} \\
%Dynamisk typ: \code{Vector[Vector[[Int]]}

\begin{REPL}
scala> val xss = for (i <- 1 to 2) yield 
                   for (j <- 1 to 5) yield i + j + 1
                 
val xss: IndexedSeq[IndexedSeq[Int]] =
      ???

\end{REPL}
\pause Om man skriver så här får man en endimensionell struktur:
\begin{REPL}
scala> val xs = for (i <- 1 to 2; j <- 1 to 5) yield i + j + 1
val xs: IndexedSeq[Int] =
    ???

\end{REPL}
\end{Slide}

\begin{Slide}{Nästlade for-uttryck}\SlideFontSmall
Iterera med \Emph{nästlad for-yield}:\\
\begin{REPL}
scala> val xss = for (i <- 1 to 2) yield {
                   for (j <- 1 to 5) yield i + j + 1
                 }
val xss: IndexedSeq[IndexedSeq[Int]] =
    Vector(Vector(3, 4, 5, 6, 7), Vector(4, 5, 6, 7, 8))

\end{REPL}
\pause Om man skriver så här får man en endimensionell struktur:
\begin{REPL}
scala> val xs = for (i <- 1 to 2; j <- 1 to 5) yield i + j + 1
val xs: IndexedSeq[Int] =
    Vector(3, 4, 5, 6, 7, 4, 5, 6, 7, 8)

\end{REPL}
\end{Slide}



\begin{Slide}{Nästlade map-uttryck}\SlideFontSmall
Iterera med \Emph{nästlade map-uttryck}:\\
\begin{REPL}
scala> val xss = (1 to 2).map(i => (1 to 5).map(j => i + j + 1))
xss: IndexedSeq[IndexedSeq[Int]] =
      ???
\end{REPL}
\end{Slide}

\begin{Slide}{Nästlade map-uttryck}\SlideFontSmall
Iterera med \Emph{nästlade map-uttryck}:\\
\begin{REPL}
scala> val xss = (1 to 2).map(i => (1 to 5).map(j => i + j + 1))
xss: IndexedSeq[IndexedSeq[Int]] =
      Vector(Vector(3, 4, 5, 6, 7), Vector(4, 5, 6, 7, 8))
\end{REPL}
\end{Slide}



\ifkompendium\else
\begin{Slide}{Fallgrop: likhet av array}
\begin{REPL}
scala> Vector.fill(5, 2)(42) == Vector.fill(5, 2)(42)
val res0: ???

scala> Array.fill(5, 2)(42) == Array.fill(5, 2)(42)
val res1: ???
\end{REPL}
\end{Slide}
\fi

\begin{Slide}{Fallgrop: likhet av array}
\begin{REPL}
scala> Vector.fill(5, 2)(42) == Vector.fill(5, 2)(42)
val res0: Boolean = true

scala> Array.fill(5, 2)(42) == Array.fill(5, 2)(42)
val res1: Boolean = false  // AAAARRGH!!! :(
\end{REPL}
Primitiva arrayer har en equals-metod som ger referenslikhet, \Alert{inte} innehållslikhet. Och det fungerar följaktligen ej heller på nästlade strukturer. 
\end{Slide}

\ifkompendium\else
\begin{Slide}{Övning: Kolla likhet av array (uppfinner hjulet)}
\begin{Code}
def isEqual(xss: Array[Array[Int]], yss: Array[Array[Int]]) = 
  var i = 0
  var foundUnequal = false
  while ??? do                          // VILKET VILLKOR?
    var j = 0
    while ??? do                        // VILKET VILLKOR?
      if xss(i)(j) != yss(i)(j) then ???   // VAD SKA UPPDATERAS? 
      j += 1
    end while
    i += 1
  end while
  !foundUnequal
end isEqual
\end{Code}
\begin{REPL}
scala> val (xss, yss) = (Array.fill(5,2)(42), Array.fill(5,2)(42))

scala> isEqual(xss, yss)

scala> yss(4)(1) = 0

scala> isEqual(xss, yss)
\end{REPL}
\end{Slide}
\fi


\begin{Slide}{Övning: Kolla likhet av array (uppfinner hjulet)}
\begin{Code}
def isEqual(xss: Array[Array[Int]], yss: Array[Array[Int]]) = 
  var i = 0
  var foundUnequal = false
  while i < xss.length && !foundUnequal do
    var j = 0
    while j < xss(i).length && !foundUnequal do
      if xss(i)(j) != yss(i)(j) then foundUnequal = true
      j += 1
    end while
    i += 1
  end while
  !foundUnequal
end isEqual
\end{Code}
\begin{REPL}
scala> val (xss, yss) = (Array.fill(5,2)(42), Array.fill(5,2)(42))

scala> isEqual(xss, yss)  // true

scala> yss(4)(1) = 0

scala> isEqual(xss, yss)  // false
\end{REPL}
\end{Slide}

\begin{Slide}{Använd \texttt{sameElements} för test av innehållslikhet men bara på icke-nästlade arrayer}

  I Scala kan du använda metoden \code{sameElements} på arrayer för innehållslikhet, men det funkar \Alert{INTE} på nästlade strukturer.

\begin{REPL}
scala> val xs = Array(1,2,3)
xs: Array[Int] = Array(1, 2, 3)

scala> val ys = Array(1,2,3)
ys: Array[Int] = Array(1, 2, 3)

scala> xs.sameElements(ys)
res0: Boolean = true

scala> Array(Array(1)) sameElements Array(Array(1))  
res1: Boolean = false

\end{REPL}
\pause Använd i stället: \code{java.util.Arrays.deepEquals(xs, ys)}\\
men det kan då behövas \code{.asInstanceOf[Array[Object]]} på argumenten om kompilatorn inte klarar typkonverteringen.
\end{Slide}

% \begin{Slide}{Matris som Array med Array med heltal i Java}\SlideFontTiny
% \begin{CodeSmall}[language=Java]
% public class ArrayMatrix {

%     public static void showMatrix(int[][] m){
%         System.out.println("\n--- showMatrix ---");
%         for (int row = 0; row < m.length; row++){
%             for (int col = 0; col < m[row].length; col++) {
%                 System.out.print("[" + row + "]");
%                 System.out.print("[" + col + "] = ");
%                 System.out.print(m[row][col] + "; ");
%             }
%             System.out.println();
%         }
%     }

%     public static void main(String[] args) {
%         int[][] xss = new int[10][5];
%         showMatrix(xss);
%     }
% }
% \end{CodeSmall}
% \pause
% Övning: skriv en metod \code{fillRnd} som fyller en heltalsmatris med slumptal 1 till n:\\
% \pause
% \jcode|public static void fillRnd(int[][] m, int n){ /* ??? */ }| \\
% \pause
% Tips: använd en nästlad for-sats och detta uttryck: \\
% \jcode{(int) (Math.random() * n + 1) // (int) motsvarar Scalas asInstanceOf[Int]}

% \end{Slide}

\begin{Slide}{Om veckans övningar}\SlideFontSmall
\begin{itemize}
\item Träna på att iterera över nästlade strukurer

\item Fortsätt jobba med Yatzy-exemplet

\item träna på att skapa \Emph{imperativa} algoritmer: \\
lös \code{isYatzy} med \code{while}-sats 

\item Extrauppgift där du ska bygga ett enkelt yatzy-spel i terminalen (kunde varit en tentauppgift...)

\end{itemize}
\end{Slide}

% \begin{Slide}{Övning extrauppgift, utgör början på labb \code{survey}}\SlideFontSmall
%
% \begin{ScalaSpec}{Table}
% object Table {
%   /** Creates a new Table from fileName with columns split by sep */
%   def fromFile(fileName: String, separator: Char = ';'): Table = ???
% }
% case class Table(
%   data: Vector[Vector[String]],
%   headings: Vector[String],
%   sep: String){
%   /** A 2-tuple with (number of rows, number of columns) in data */
%   val dim: (Int, Int) = ???
%
%   /** The element in row r an column c of data, counting from 0 */
%   def apply(r: Int, c: Int): String = ???
%
%   /** The row-vector r in data, counting from 0 */
%   def row(r: Int): Vector[String]= ???
%
%   /** The column-vector c in data, counting from 0 */
%   def col(c: Int): Vector[String] = ???
%
%   /** A map from heading to index counting from 0 */
%   lazy val indexOfHeading: Map[String, Int] = ???
%
%   /** The column-vector with heading h in data */
%   def col(h: String): Vector[String] = ???
%
%   /** A vector with the distinct, sorted values of col with heading h */
%   def values(h: String): Vector[String] = ???
%
%   /** Headings and data with columns separated by sep */
%   override lazy val toString: String = ???
% }
% \end{ScalaSpec}
% \end{Slide}


% \begin{Slide}{Övn. fördjupn. uppg.: skapa en generisk matris-klass}\SlideFontSmall
% \vspace{-0.7em}
% \begin{Code}[basicstyle=\SlideFontSize{6}{6.8}\ttfamily\selectfont]
% case class Matrix[T](data: Vector[Vector[T]]){
%
%   def foreachRowCol(f: (Int, Int, T) => Unit): Unit =
%     for (r <- data.indices) {
%       for (c <- data(r).indices) {
%         f(r, c, data(r)(c))
%       }
%     }
%
%   def map[U](f: T => U): Matrix[U] = Matrix(data.map(_.map(f)))
%
%   /** The element at row r and column c */
%   def apply(r: Int, c: Int): T = ???
%
%   /** Gives Some[T](element) at index (r, c) if within index bounds, else None */
%   def get(r: Int, c: Int): Option[T] = ???
%
%   /** The row vector of row r */
%   def row(r: Int): Vector[T] = ???
%
%   /** The column vector of column c */
%   def col(c: Int): Vector[T] = ???
%
%   /** A new Matrix with element at row r and col c updated */
%   def updated(r: Int, c: Int, value: T): Matrix[T] = ???
% }
% object Matrix {
%   def fill[T](rowSize: Int, colSize: Int)(init: T): Matrix[T] =
%     new Matrix(Vector.fill(rowSize)(Vector.fill(colSize)(init)))
% }
% \end{Code}
% \end{Slide}
