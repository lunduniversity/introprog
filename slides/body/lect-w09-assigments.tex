%!TEX encoding = UTF-8 Unicode
%!TEX root = ../lect-w09.tex



\begin{Slide}{Omkontroll}
\begin{itemize}
\item För de som ej deltog i kontrollskrivningen erbjuds \Alert{omkontrollskrivning}. 
%måndagen den 16 November kl 12:15--ca 16:30 I Zoom/Canvas/Discord.
\item Mer information kommer senare.
%\item \url{https://fileadmin.cs.lth.se/cs/Bilder/Salar/E2405.pdf}
\item Kontrollskrivningen är obligatorisk och ett krav för att bli godkänd på kursen.
\end{itemize}  
\end{Slide}


\Subsection{Veckans uppgifter}

\begin{Slide}{Övning: \texttt{lookup}}
\begin{itemize}\SlideFontSmall
  \item Övningen innehåller delar som är \Alert{nödvändiga} för laborationen.
  \item På övningen tränar du på \Emph{mängder} och \Emph{nyckel-värde-tabeller}.
  \item Du ska skapa en klass \code{FreqMapBuilder} som bygger upp en tabell med ordfrekvenser, som behövs på labben.
\end{itemize}
\end{Slide}

\begin{Slide}{Laboration: \texttt{words}}
\begin{itemize}
  \item Denna uppgift handlar om analys av naurligt språk \Eng{Natural Language Processing, NLP}.
  \item Svara på frågorna:
  \begin{itemize}%[noitemsep]
  \item Hur vanligt är ett visst ord i en given text?
  \item Vilket är det vanligaste ordet som följer efter ett visst ord?
  \item Hur kan man generera ordsekvenser som liknar ordföljden i en given text?
  \end{itemize}
\item Använda mängd för unika ord.
\item Använda nyckel-värde-tabell för att för varje ord i en lång text räkna antalet förekomster av detta ord.
\end{itemize}
\end{Slide}


\begin{Slide}{Laboration: \texttt{words}}
Lärandemål:
\begin{itemize}\SlideFontSmall
%!TEX encoding = UTF-8 Unicode
%!TEX root = ../compendium2.tex

%\item Kunna använda en integrerad utvecklingsmiljö (IDE).
%\item Kunna använda färdiga funktioner för att läsa till, och skriva från, textfil.
%\item Kunna använda enkla case-klasser.
%\item Kunna skapa och använda enkla klasser med föränderlig data.
\item Kunna skapa och använda nyckel-värde-tabeller med samlingstypen \code{Map}.
\item Kunna skapa och använda mängder med samlingstypen \code{Set}.
\item Förstå skillnaden mellan en ordnad sekvens och en mängd.
\item Förstå likheter och skillnader mellan en sekvens av par och en nyckel-värde-tabell. 
\item Kunna implementera algoritmer som använder nästlade strukturer. 
%\item Kunna skapa en ny samling från en befintlig samling.
%\item Förstå skillnaden mellan kompileringsfel och exekveringsfel.
%\item Kunna felsöka i små program med hjälp av utskrifter.
%\item Kunna felsöka i små program med hjälp av en debugger i en IDE.

\end{itemize}
Uppgifter:
\begin{itemize}\SlideFontSmall
  \item Dela upp en sträng i ord.
  \item Skapa ordfrekvenstabeller för böcker som ditt program laddar ner från nätet via projektet Gutenberg med fria böcker.
  \item Skapa frekvenstabeller för ordföljder, s.k. \emph{n-gram}.
  \item Skriv ut intressant statistik om ordvalen i olika böcker, t.ex. ur könsrollsperspektiv.
  \item Valfri uppgift: Gör en bot som genererar slumpvisa, artificiella meningar som liknar mänskligt språk.
\end{itemize}
\end{Slide}

\begin{Slide}{Att läsa kod} \SlideFontSmall
  En viktig förutsättning för att kunna \emph{skriva} kod effektivt är att kunna \emph{läsa} kod effektivt. Många av er kanske inte läser given kod tillräckligt noga... \\ \vspace{0.5em}\textbf{Hur läsa given kod?}
  \begin{itemize}\SlideFontSmall
    \item Få en överblick över vilka kodfiler som finns och vad som finns i vilken fil.
    \item Studera dokumentation om vad som är \Emph{syftet} med olika abstraktioner.
    \item Studera klassparametrar och metodhuvuden. \Emph{Typerna} är dina \Alert{tankeverktyg}: ''Vad kommer in?'' och ''Vad kommer ut?''
    \item Läs dokumentationskommentarer och beskrivningar \emph{mycket noga}.
    \item Ställ dig under läsningen frågorna: \\ ''Vad finns?'' och ''Vad fattas?'' för att du ska kunna lösa din uppgift.
    \item Läs iterativt. Vänta med implementationsdetaljer. Hoppa mellan deklaration och användning.
    \item Studera \Emph{beroenden}: \code{Matrix -> Life -> LifeWindow -> Main}
    \item Börja med den del av koden som har \Alert{minst} beroenden till andra delar.
    \item Experimentera med given kod i REPL.
  \end{itemize}
\end{Slide}

