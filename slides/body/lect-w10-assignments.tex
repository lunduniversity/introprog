%!TEX encoding = UTF-8 Unicode
%!TEX root = ../lect-w10.tex

%%%

\ifkompendium\else


\Subsection{Grupplaboration}



\begin{Slide}{Grupplaboration: \texttt{snake} över 2 veckor}
\begin{minipage}{0.6\textwidth}
\includegraphics[width=1.0\textwidth]{../img/snake-twoplayer}
\end{minipage}%
\begin{minipage}{0.4\textwidth}
\begin{itemize}
%!TEX encoding = UTF-8 Unicode
%!TEX root = ../compendium2.tex

\item Kunna använda arv.
\item Kunna göra överskuggning av medlemmar i en supertyp.
%\item Kunna referera till medlemmar i superklassen med \code{super} vid överskuggning.
\item Kunna förklara begreppet dynamisk bindning.
\item Kunna använda abstrakta klasser och skapa en klasshierarki.

\end{itemize}
\end{minipage}%
\end{Slide}

\begin{Slide}{Arvshierarki i \texttt{snake}: Olika typer av varelser}
  \begin{center}
  \newcommand{\TextBox}[1]{\raisebox{0pt}[1em][0.5em]{#1}}
  \tikzstyle{umlclass}=[rectangle, draw=black,  thick, anchor=north, text width=2.5cm, rectangle split, rectangle split parts = 3]
  \begin{tikzpicture}[inner sep=0.5em,scale=0.8, every node/.style={transform shape}]

    \node [umlclass, rectangle split parts = 1, xshift=-1.0cm, yshift=4.5cm] (BaseType)  {
                \textit{\textbf{\centerline{\TextBox{\code{Entity}}}}}
  %              \nodepart[align=left]{second}\code{def x: T} \newline \code{def y: T}
            };


    \node [umlclass, rectangle split parts = 1, xshift=-3cm, yshift=2.5cm] (SubType1)  {
                \textit{\textbf{\centerline{\TextBox{\code{CanMove}}}}}
  %              \nodepart[align=left]{second}\code{val x: Int} \newline \code{val y: Int}
            };

  \node [umlclass, rectangle split parts = 1, xshift=-4.5cm, yshift=0.5cm] (SubSubType0)  {
              {\textbf{\centerline{\TextBox{\code{Snake}}}}}
  %            \nodepart[]{second}\TextBox{\code{val dim: Int}}
  };


  \node [umlclass, rectangle split parts = 1, xshift=0.75cm, yshift=2.5cm] (SubType2)  {
              \textit{\textbf{\centerline{\TextBox{\code{CanTeleport}}}}}
  %            \nodepart[]{second}\TextBox{\code{val dim: Int}}
          };

  \node [umlclass, rectangle split parts = 1, xshift=-1.0cm, yshift=0.5cm] (SubSubType1)  {
              {\textbf{\centerline{\TextBox{\code{Apple}}}}}
  %            \nodepart[]{second}\TextBox{\code{val dim: Int}}
          };

  \node [umlclass, rectangle split parts = 1, xshift=2.5cm, yshift=0.5cm] (SubSubType2)  {
              {\textbf{\centerline{\TextBox{\code{Banana}}}}}
  %            \nodepart[]{second}\TextBox{\code{val dim: Int}}
          };


  \draw[umlarrow] (SubType1.north) -- ++(0,0.5) -| (BaseType.south);
  \draw[umlarrow] (SubType2.north) -- ++(0,0.5) -| (BaseType.south);
  \draw[umlarrow] (SubSubType1.north) -- ++(0,0.5) -| (SubType2.south);
  \draw[umlarrow] (SubSubType2.north) -- ++(0,0.5) -| (SubType2.south);
  \draw[umlarrow] (SubSubType0.north) -- ++(0,0.5) -| (SubType1.south);

  \end{tikzpicture}
  \end{center}
\end{Slide}


\begin{Slide}{Arvshierarki i \texttt{snake}: Olika typer av spel}
  \begin{center}
  \newcommand{\TextBox}[1]{\raisebox{0pt}[1em][0.5em]{#1}}
  \tikzstyle{umlclass}=[rectangle, draw=black,  thick, anchor=north, text width=3cm, rectangle split, rectangle split parts = 3]
  \begin{tikzpicture}[inner sep=0.5em,scale=0.8, every node/.style={transform shape}]

    \node [umlclass, rectangle split parts = 1, xshift=0cm, yshift=5cm] (BaseType)  {
                \textit{\textbf{\centerline{\TextBox{\code{BlockGame}}}}}
  %              \nodepart[align=left]{second}\code{def x: T} \newline \code{def y: T}
            };


    \node [umlclass, rectangle split parts = 1, xshift=0cm, yshift=3.0cm] (SubType)  {
                \textit{\textbf{\centerline{\TextBox{\code{SnakeGame}}}}}
  %              \nodepart[align=left]{second}\code{val x: Int} \newline \code{val y: Int}
            };

  \node [umlclass, rectangle split parts = 1, xshift=-3cm, yshift=0.5cm] (SubSubType1)  {
              {\textbf{\centerline{\TextBox{\code{OnePlayerGame}}}}}
  %            \nodepart[]{second}\TextBox{\code{val dim: Int}}
          };

  \node [umlclass, rectangle split parts = 1, xshift=3cm, yshift=0.5cm] (SubSubType2)  {
              {\textbf{\centerline{\TextBox{\code{TwoPlayerGame}}}}}
  %            \nodepart[]{second}\TextBox{\code{val dim: Int}}
          };

  \draw[umlarrow] (SubType.north) -- ++(0,0.5) -| (BaseType.south);
  \draw[umlarrow] (SubSubType1.north) -- ++(0,0.5) -| (SubType.south);
  \draw[umlarrow] (SubSubType2.north) -- ++(0,0.5) -| (SubType.south);

  \end{tikzpicture}
  \end{center}
\end{Slide}

\begin{Slide}{Krav vid respektive gruppstorlek}
Krav som minst ska implementeras vid olika gruppstorlek:

\vspace{0.5em}
  \begin{tabular}{r | c c c c c c}
    \Alert{Krav} / \Emph{Antal personer} & 1       & 2       & 3       & 4       & 5       & 6 \\ \hline
    \texttt{Player}       & $\surd$ & $\surd$ & $\surd$ & $\surd$ & $\surd$ & $\surd$ \\
    \texttt{OnePlayerGame}& $\surd$ &         &         &         & $\surd$ & $\surd$ \\
    \texttt{TwoPlayerGame}&         & $\surd$ & $\surd$ & $\surd$ & $\surd$ & $\surd$ \\
    \texttt{Snake}        & $\surd$ & $\surd$ & $\surd$ & $\surd$ & $\surd$ & $\surd$ \\
    \texttt{Apple}        & $\surd$ &         & $\surd$ & $\surd$ & $\surd$ & $\surd$ \\
    \texttt{Banana}       &         &         &         & $\surd$ &         & $\surd$ \\
    \texttt{Monster}       &         &         & $\surd$  &  &         & $\surd$ \\
    \texttt{Settings}       & $\surd$ & $\surd$ & $\surd$ & $\surd$ & $\surd$ & $\surd$ \\
  \end{tabular}

\vspace{0.5em}
\begin{itemize}\SlideFontSmall
\item Uppgiften lämpar sig bäst för minst 3 gruppmedlemmar. 
\item Prata med handledare: slå ihop två små grupper till en med max 6 st.
\item Om du har särskilda skäl, t.ex. många restlabbar, kan du efter godkännande från kursansvarig göra labben enskilt.

\end{itemize}
\end{Slide}


\begin{Slide}{Instruktioner Grupplaboration (text ur kompendiet)}
\begin{itemize}\SlideFontTiny
%!TEX encoding = UTF-8 Unicode
%!TEX root = compendium.tex
\item
Diskutera i din samarbetsgrupp hur ni ska dela upp koden mellan er i flera olika delar, som ni kan arbeta med var för sig. En sådan del kan vara en klass, en trait, ett objekt, ett paket, eller en funktion.
\item
Varje del ska ha en \textbf{huvudansvarig} individ.
\item
Arbetsfördelningen ska vara någorlunda jämnt fördelad mellan gruppmedlemmarna.
\item
Den som är huvudansvarig för en viss del redovisar den delen.
\item 
Ni ska ta fram en gruppgemensam checklista för kodgranskning. Alla ska granska minst en annan gruppmedlems kod enligt checklistan. 
\item
Grupplaborationen görs över \textbf{två veckor} uppdelat på två delredovisningar. Vid första redovisningen ska arbetsupplägget och pågående utveckling redovisas. Vid andra tillfället ska de färdig lösningarna presenteras av respektive huvudansvarig individ.
\item
Vid första redovisningen ska du redogöra för handledaren hur ni delat upp koden och vem som är huvudansvarig för vad och vad ditt ansvar omfattar, samt hur ni jobbar praktiskt med att synkronisera er utveckling.
\item Grupplaborationen är en \textbf{extra stor uppgift} och grupparbetet behöver ledtid för att ni ska hinna koordinera er sinsemellan. Du behöver därför planera för att arbeta med något i grupplabben i stort sett varje dag under de tillgängliga veckorna, och vara redo att bidra i diskussioner.

\end{itemize}
\end{Slide}

%!TEX encoding = UTF-8 Unicode
%!TEX root = ../lect-w09.tex

\begin{Slide}{Kodgranskning under grupplaborationen \texttt{snake}}
Grupplaborationen \code{snake} går över två läsveckor (w10--w11). Du ska:
\begin{itemize}
\item delta i framtagande av en gemensam checklista för kodgranskning,
\item granska minst en annan gruppmedlems kod,
\item bjuda in minst en annan gruppmedlem att granska din kod,
\item ge konstruktiv feedback,
\item ta emot konstruktiv feedback och försöka förbättra din kod,
\item på redovisning redogöra för nyttan och utmaningarna med kodgranskningar utifrån dina egna erfarenheter.
\end{itemize}
Mer om kodgranskning i efterföljande kurser, t.ex. ''Programvaruutveckling i grupp'' och ''Programvaruutveckling för stora system''
\end{Slide}


\begin{Slide}{Redovisning på obligatorisk schemalagd labbtid}
Första \code{snake}-veckan (w10):
\begin{itemize}\SlideFontTiny
\item Redovisa uppdelning av uppgiften mellan gruppmedlemmar och hur långt du har kommit med din del. Visa en skriftlig plan för implementationsarbetet för andra veckan, med valda deluppgifter och sluttidpunkter.
\item Redovisa arbetet med granskningar, minst checklista och planering för arbetet med granskningar i andra veckan. Bra om ni du gjort någon granskning.
\item Jobba vidare med \code{snake} och få hjälp av handledare om du kör fast. 
\end{itemize}
Andra \code{snake}-veckan (w11):
\begin{itemize}\SlideFontTiny
\item Redogör för uppdelningen av uppgiften och avgränsningen av ditt huvudansvar.
\item Förklara övergripande hela kodbasens struktur och syftet med de olika kod-delarna.
\item Förklara mer ingående de delar där du bidragit till implementationen.
\item Gör en kort demonstration av det körande systemet och visa vilka valfria uppgifter ni valt att göra.
\item Beskriv utmaningar med systemutveckling i grupp utifrån dina egna erfarenheter.
\item Redogör för nyttan och utmaningarna med kodgranskningar utifrån dina egna erfarenheter.  
\end{itemize}
\end{Slide}

\begin{Slide}{Övning w10: \texttt{inheritance}}
\begin{itemize}\SlideFontTiny
\input{../compendium/modules/w10-inheritance-exercise-goals.tex}
\end{itemize}
\end{Slide}

% \begin{Slide}{Extraundervisning del 2}
% För de som har det extra svårt, på samma sätt som förra ons:
% \begin{itemize}
% \item  \Alert{Extraundervisning} i \Emph{E:3308} onsdag 14/11 kl 15:15-17:00
% \item Hitta dit: \url{http://fileadmin.cs.lth.se/ehus/E3308.pdf}
% \item Det finns gott om platser i den salen men om fullt så förtur för de med kontrollskrivningspoäng < 15
% \item Fokus: grundläggande, långsam behandling på begäran av viktigaste koncepten från lp1
% \item Förberedelse: kolla på din kontrollskrivning del B och välj ut något specifikt som du vill ha hjälp med att förstå. \\ Kolla ks här: \url{http://cs.lth.se/pgk/examination/}
% \end{itemize}
% \end{Slide}


\begin{Slide}{Gästföreläsning: Kodgranskningar}
Hjärtligt välkommen: \\ \Emph{Gustaf Lundh}, Senior Developer, Axis
\end{Slide}

\fi
