%!TEX encoding = UTF-8 Unicode
%!TEX root = ../lect-w10.tex

%%%



\Subsection{Undantag}
\begin{Slide}{Vad är ett undantag \Eng{exception}?}
Undantag representerar ett fel eller ett onormalt tillstånd som upptäcks under exekvering och som  behöver hanteras på särskilt sätt vid sidan av det normala exekveringsflödet. 

\vspace{1em}\href{https://sv.wikipedia.org/wiki/Undantagshantering}{sv.wikipedia.org/wiki/Undantagshantering}


\vspace{1em} Exempel på undantag:

\pause

\begin{itemize}
\item Indexering utanför vektorns indexgränser.

\item Läsning bortom filens slut.

\item Försök att öppna en fil som inte finns.

\item Minnet är slut.

\item Division med noll.

\item \code{"hej".toInt} resulterar i \\\code{java.lang.NumberFormatException}

\end{itemize}

\end{Slide}

\begin{Slide}{''Kasta'' dina egna undantag med \texttt{throw}}\SlideFontSmall
Man kan själv generera ett undantag med \code{throw}, vilket kallas att \Emph{kasta} ett undantag som (om det inte \Emph{fångas}), gör att exekveringen \Alert{avbryts}.


\begin{REPL}
scala> def pang = throw new Exception("PANG!")
pang: Nothing

scala> pang
java.lang.Exception: PANG!
  at .pang(<console>:11)
  ...
  
\end{REPL}
\pause
Olika sätt att hantera undantag: 
\begin{itemize}
\item Scala: Man kan använda ett \code{try ... catch}-uttryck och ge ett \Emph{värde} i händelse av undantag.
\item Java: Man kan använda en \code{try ... catch}-sats och \Alert{göra något} i händelse av undantag.
 
\item Scala: Man kan \Emph{kapsla in} ett undantag med \Emph{\texttt{scala.util.Try}} och förhindra att exekveringen avbryts. (Finns ej i Java; att föredra i Scala.)
\end{itemize}
\end{Slide}


\Subsection{\texttt{scala.util.Try}}

\begin{Slide}{En gemensam bastyp för något som kan misslyckas}\SlideFontSmall
\begin{Code}
import scala.util.{Try, Success, Failure}
\end{Code}

\vspace{-0.5em}\begin{center}
\newcommand{\TextBox}[1]{\raisebox{0pt}[1em][0.5em]{#1}}
\tikzstyle{umlclass}=[rectangle, draw=black,  thick, anchor=north, text width=3.0cm, rectangle split, rectangle split parts = 3]
\begin{tikzpicture}[inner sep=0.5em]
\node [umlclass, rectangle split parts = 2, xshift=0cm, text width=3.8cm] (BaseType)  {
            \textit{\textbf{\centerline{\TextBox{\code{Try[T]}}}}}
            \nodepart[]{second}
            \TextBox{\code{def get: T}}\newline
            \TextBox{\code{def isFailure: Boolean}}\newline
            \TextBox{\code{def isSuccess: Boolean}}
        };
        
\node [umlclass, rectangle split parts = 2, text width=2.2cm]  at (-2.5cm,-3.7cm) (SubType1) {
            \textbf{\centerline{\TextBox{\code{Success[T]}}}}
            \nodepart[]{second} \TextBox{\code{val value: T}}
        };  
                
\node [umlclass, rectangle split parts = 2, text width=4.2cm] at (2.5cm,-3.7cm) (SubType2)  {
            \textbf{\centerline{\TextBox{\code{Failure[T]}}}}
            \nodepart[]{second} \TextBox{\code{val exception: Throwable}}
        };        
\draw[umlarrow] (SubType1.north) -- ++(0,0.5) -| (BaseType.south);    
\draw[umlarrow] (SubType2.north) -- ++(0,0.5) -| (BaseType.south);            
\end{tikzpicture}
\end{center}
\end{Slide}

\begin{Slide}{Hantera undantag med \texttt{Try}}
\vspace{-0.5em}\begin{REPL}
scala> def pang = throw new Exception("PANG!")

scala> def kanskePang = if (math.random < 0.5) 42 else pang

scala> import scala.util.{Try, Success, Failure}

scala> def försök = Try { kanskePang }

scala> val xs = Vector.fill(15){försök}

scala> val trettonde = xs(13) match {
         case Success(value) => value
         case Failure(e) => println(e); -1
       }

scala> (xs(13).isSuccess, xs(13).isFailure)

scala> försök.foreach(println)

scala> försök.map(_ + 1)

scala> for (Success(x) <- xs) yield x
\end{REPL}
\end{Slide}

\begin{Slide}{Fördjupning: \texttt{try}-\texttt{catch}-uttryck}\SlideFontSmall
Man kan fånga undantag med ett \code{try ... catch}-uttryck:
\begin{Code}
def carola = try {
  if (math.random > 0.5) throw new Exception("stormvind")
  42
} catch { 
  case e: Exception =>
    println("Fångad av en " + e.getMessage)
    -1
}  
\end{Code}
\pause
\begin{REPL}
scala> Vector.fill(5)(carola)
Fångad av en stormvind
Fångad av en stormvind
Fångad av en stormvind
res0: scala.collection.immutable.Vector[Int] = Vector(-1, 42, 42, -1, -1)
\end{REPL}
\pause
\emph{Fördjupning:} \\ Gör övning 9-10 som visar hur man fångar undantag i Scala och Java. \\Mer om undantag i fortsättningskursen.
\end{Slide}






