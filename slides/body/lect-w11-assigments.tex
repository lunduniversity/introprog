%!TEX encoding = UTF-8 Unicode
%!TEX root = ../lect-w11.tex

%%%


\Subsection{Veckans uppgifter: \texttt{scalajava} och \texttt{javatext}}
\begin{Slide}{Veckans uppgifter: \texttt{scalajava} och \texttt{javatext}}\SlideFontSmall
\Emph{Labbförberedelse:}
\begin{itemize}
\item Gör övning \texttt{scalajava}:
\begin{itemize}\SlideFontSmall
\item Översätt spelet Hangman från Java till Scala
\item Översätt Point från Scala till Java
\item Undersök autoboxning \Eng{autoboxing}
\item Använda \code{import scala.collection.JavaConverters._}
\end{itemize}
\item Studera riktlinjerna för \code{javatext} i kompendiet.
\item Labben är \Alert{individuell} men du ska \Emph{spela en tidig version av någon annans spel} och ge återkoppling på kodens \Alert{läsbarhet} och vice versa.
\end{itemize}
\Emph{Laboration \code{javatext}:}
\begin{itemize}
  \item Gör klart ett (lagom) intressant/roligt textspel för terminalen huvudsakligen i Java men vissa delar i Scala, enligt krav, tips och inspiration i labb-instruktionerna.
  \item \Emph{Ny version} \Alert{ersätter} den i det tryckta kompendiet; se nytt kap 11.3 här: \url{http://cs.lth.se/pgk/compendium/}  
  \item Om du har flera ''kompletteras'' efter dig eller tycker det är alldeles övermäktigt att få ihop de obligatoriska kraven: diskutera din situation med handledare på resurstid; kontakta kursansvarig vid behov. 
\end{itemize}
\end{Slide}
