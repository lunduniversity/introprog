%!TEX encoding = UTF-8 Unicode
%!TEX root = ../lect-w11.tex

%%%


\Subsection{Veckans labb: \texttt{javatext} alt. \texttt{lthopoly-team}}
\begin{Slide}{Veckans labb: \texttt{javatext} alt. \texttt{lthopoly-team}}\SlideFontSmall
\Emph{Förberedelse:}
\begin{itemize}
\item Gör övning \texttt{scalajava}:
\begin{itemize}\SlideFontSmall
\item Översätt spelet Hangman från Java till Scala
\item Översätt Point från Scala till Java
\item Undersök autoboxning \Eng{autoboxing}
\item Använda \code{import scala.collection.JavaConverters._}
\end{itemize}
\item Studera riktlinjerna för \code{javatext} i kompendiet
\item Bilda dig en uppfattning om team-labb-alternativet \code{lthopoly_team} i kompendiet och i korsens workspace
\item Bestäm vilket alternativ som du vill göra  i dialog med din grupp
\end{itemize}
\Emph{\code{javatext}}
\begin{itemize}
\item Gör ett (lagom) intressant/roligt textspel för terminalen med ca 80\% Java-kodrader och ca 20\% Scala-kodrader, enligt kraven som beskrivs i kompendiet. Du ska även spela någon annans spel och ge återkoppling.
\end{itemize}
\Emph{\code{lthopoly_team}} (alternativ om minst två personer vill samarbeta om detta)
\begin{itemize}
\item Implementera en förenklad variant av Monopol i terminalen som bygger på campus LTH (man kan till exempel köpa Moroten och Piskan...)
\end{itemize}
\end{Slide}
