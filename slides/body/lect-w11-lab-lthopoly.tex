%!TEX encoding = UTF-8 Unicode
%!TEX root = ../lect-w11.tex

%%%


\Subsection{Veckans lab: \texttt{javatext}}
\begin{Slide}{Veckans lab: \texttt{javatext}}\SlideFontSmall
\Emph{Förberedelse:}
\begin{itemize}
\item Gör övning \texttt{scalajava}:
\begin{itemize}\SlideFontSmall
\item Översätt spelet Hangman från Java till Scala
\item Översätt Point från Scala till Java
\item Undersök autoboxning \Eng{autoboxing}
\item Använda \code{import scala.collection.JavaConverters._}
\end{itemize}
\item Studera riktlinjerna för \code{javatext} i kompendiet.
\item Labben är indivuell men du ska spela någon annans spel och ge feedback på koden och någon annan ska spela ditt spel och ge feedback på koden.
\end{itemize}
\Emph{Laboration \code{javatext}}
\begin{itemize}
\item Gör ett (lagom) intressant/roligt textspel för terminalen med ca 80\% Java-kodrader och ca 20\% Scala-kodrader, enligt kraven som beskrivs i kompendiet. Du ska även spela någon annans spel och ge återkoppling.
\end{itemize}
\end{Slide}
