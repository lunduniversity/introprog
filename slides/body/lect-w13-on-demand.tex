%!TEX encoding = UTF-8 Unicode
%!TEX root = ../lect-w13.tex
%%%



\Subsection{Repetition på begäran}

\newcommand{\Vecka}[1]{\hfill\href{https://fileadmin.cs.lth.se/pgk/lect-w#1.pdf}{w#1}}

% \begin{Slide}{Repetitionsämnen 2020}
% Gör en lista på saker du behöver repetera.\\Exempel på önskade repetitionsämnen från tidigare år:
% \begin{itemize}\SlideFontSmall
%   \item closure (''fångad variabelrymd'') \Vecka{03}
%   \item Skillnad på objekt och singelobjekt? \Vecka{04}
%   \item Mönstermatchning. \Vecka{06}
%   \item \code{Option}  \Vecka{06}
%   \item \code{Try} med stort T  \Vecka{06}
%   \item \code{enum}: när och hur? eller case-klass? \Vecka{07}
%   \item När använda vilken sekvenstyp? \Vecka{07}
%   \item Typhärledning. \Vecka{08}
%   \item komposition eller arv?  \Vecka{10}
% \end{itemize}  
% \end{Slide}

% \begin{Slide}{Repetitionsämnen på begäran från tidigare år}
% \begin{enumerate}\SlideFontSmall
%    \item namnanrop, värdeanrop \Vecka{03}
%    \item funktionsvärde, funktionstyp och thunk \Vecka{03}
%    \item \code{--classpath} \Vecka{04}
%    \item \code{import} \Vecka{04}
%    \item \code{Option} \Vecka{06}
%    \item \code{Try} \Vecka{06}
%    \item \code{enum} \Vecka{07}
%    \item avlusning, läsa felmeddelande \Vecka{08}
%    \item \code{given using} \Vecka{11}
% \end{enumerate}  
% \end{Slide}

% \begin{Slide}{Några extra önskemål från tidigare år (i mån av tid)}
% \begin{enumerate}\SlideFontSmall
%   \item Closure (''fångad variabelrymd'') \Vecka{03}
%   \item Skillnad på objekt och singelobjekt? \Vecka{04}
%   \item Mönstermatchning. \Vecka{06}
%   \item När använda vilken sekvenstyp? \Vecka{07}
%   \item Typhärledning. \Vecka{08}
%   \item Komposition eller arv?  \Vecka{10}
% \end{enumerate}  
% \end{Slide}



\begin{Slide}{På begäran 2025}
\Emph{Grumligt}
\begin{enumerate}\SlideFontSmall
  \item namnanrop och värdeanrop \Vecka{03}
  \item konstruktor \Vecka{05}
  \item mönstermatchning med \code{match} ... \code{case} \Vecka{06}
  \item enumerationer \Vecka{07}
  \item synlighet, import/export, private/protected \Vecka{10}

\end{enumerate}  
\vspace{1em}\Alert{Nyfiken-på}
\begin{enumerate}\SlideFontSmall
  \item säker kod och felhantering
\end{enumerate}  
\end{Slide}


\begin{Slide}{På begäran 2024}
\Emph{Grumligt}
\begin{enumerate}\SlideFontSmall
  \item När är det bra/dåligt att använda anonyma funktioner? \Vecka{03}
  \item Klasser och kompanjonsobjekt: vad passar bäst var? \Vecka{05}
  \item Hur göra felhantering med \code{Option} och \code{Try}? \Vecka{06}
  \item Skillnaden mellan sats \& uttryck, tex \code{if}, \code{for}? \Vecka{01}

\end{enumerate}  
\vspace{1em}\Alert{Nyfiken-på}
\begin{enumerate}\SlideFontSmall
  \item Flertrådad programmering
  \item Fönsterhantering i introprog under huven 
  \item Generiska typgränser \code{<:} \code{>:}
\end{enumerate}  
\end{Slide}
  


% \begin{Slide}{På begäran 2023}
% \Emph{Grumligt}
% \begin{enumerate}\SlideFontSmall
%   \item Jämför: \code{class}, \code{trait}, \code{enum} \Vecka{05}
%   \item Hur fungerar kompanjonsobjekt? \Vecka{05}
%   \item Jämför: \code{try catch finally} och \code{Try Success Failure} \Vecka{06}
%   \item Hur fungerar \code{enum}? \Vecka{07} 
%   \item \code{match} \code{case} \Vecka{06}
%   \item Vad händer i minnet? Aktiveringspost, stacken, heapen \Vecka{03}
% \end{enumerate}  
% \vspace{1em}\Alert{Nyfiken-på}
% \begin{enumerate}\SlideFontSmall
%   \item Auto-formatera kod \hfill \url{https://scalameta.org/scalafmt/}
%   \item Flertrådad programmering: Övning Extra Vecka 12 \\ Kap 12.2.2 Uppgifter om trådar och jämlöpande exekvering
%   \item Opaka typer \hfill \url{https://docs.scala-lang.org/scala3/reference/other-new-features/opaques.html} 
% \end{enumerate}  
% \end{Slide}


%!TEX encoding = UTF-8 Unicode
%!TEX root = ../lect-w12.tex

%%%


\begin{Slide}{Repetition: Vad är en algoritm? }\SlideFontTiny
En \href{https://sv.wikipedia.org/wiki/Algoritm}{algoritm} är en stegvis beskrivning av hur man löser ett problem. \\ 
Exempel: SWAP, MIN, Registrering, Sökning, Sortering \\
\pause\vspace{0.5em}
Problemlösningsprocessens olika steg (inte nödvändigtvis i denna ordning): 
\begin{itemize}
\item Dela upp problemet i enklare delproblem och sätt samman.
\item Finns redan färdig lösning på (del)problem?
\item Formulera (del)\Emph{problemet} och ange tydligt indata och utdata: \\ exempel MIN: indata: sekvens av heltal; utdata: minsta talet
\item Kom på en \Emph{lösningsidé}: (kan  vara mycket klurigt och svårt) \\ exempel MIN: iterera över talen och håll reda på ''minst hittills''
\item Formulera en \Emph{stegvis beskrivning} som löser problemet: \\ exempel: pseudo-kod med sekvens av instruktioner
\item Implementera en \Emph{körbar lösning} i ''riktig'' kod: \\ exempel: en Scala-metod i en klass eller i ett singelobjekt
\item Har algoritmen acceptabla tids- och minneskrav?
\end{itemize}
\pause\vspace{0.5em} Det krävs ofta \Emph{kreativitiet} i stegen ovan  -- även i att \Emph{känna igen} problemet!\\
Simpelt exempel: Du stöter på problemet MAX och ser likheten med MIN.\\
\pause\vspace{0.5em}\Emph{Övning}: Diskutera hur du löser detta problem i relation till stegen ovan: \\
\emph{Att räkna antalet förekomster av olika unika ord i en textsträng.} 
\end{Slide}















\begin{Slide}{Repetition: Tumregler/tips vid val av abstraktion}\SlideFontSmall
Extensionsmetod, singelobjekt, case-klass, klass, trait, eller enum?
\begin{itemize}\SlideFontTiny
\item Om du vill lägga till en metod på befintlig typ utan behov av nya attribut etc., använd \code{extension}.
\item Använd \code{object} om du behöver samla metoder (och variabler) i en modul som bara finns i en upplaga. Du får lokal namnrymd och punktnotation på köpet.
\item Behöver du modellera \Emph{oföränderlig data}, använd en \code{case class} eller \code{enum}.  
\item Om du vill ha uppräknade värden som du vill kunna iterera över och matcha på i förseglad struktur, med värden i egen namnrymd, använd \code{enum}.
\item Med \code{case class} och \code{enum} får du även innehållslikhet och en massa annat godis på köpet!
\item Behöver du \Alert{förändringsbart tillstånd} \Eng{mutable state} använd en vanlig \code{class}. Det normala är att det föränderliga tillståndet (de attribut som är föränderliga) är \code{private} eller \code{protected} och att all uppdatering och avläsning av tillståndet sker indirekt genom metoder (getters/setters/...).
\item Behöver du en abstrakt bastyp använd en \code{trait}, speciellt om du vill ha möjlighet till inmixning.  Om du vill förhindra inmixning eller underlätta användning från Java, använd \code{abstract class}. 
\end{itemize}
\end{Slide}


% \begin{Slide}{Tips om hur man läser en specifikation}\SlideFontSmall
% När du läser en specifikation av en klass, en trait, eller ett singelobjekt:
% \begin{itemize}
% \item Tänk igenom vilket ansvar olika delar av koden har
% \item Vad håller klassen reda på? \\$\rightarrow$ Ledtrådar till attribut
% \item Vad ska klassen göra/räkna ut? \\$\rightarrow$ Ledtrådar till metoder och deras algoritm
% \item Vilka andra klasser har nytta av denna metod? \\$\rightarrow$ Ledtrådar till hur klasserna samverkar för att lösa uppgiften
% \end{itemize}
% Rita gärna en bild med ett specifikt exempel på vilken data som olika instanser håller reda på och fundera på hur data skapas/beräknas/förändras
% \end{Slide}


\begin{Slide}{Repetition: Tips om val av samling}\SlideFontSmall

Det är ofta enklare med oföränderliga samlingar med oföränderliga element och skapa nya samlingar vid förändring. Men för vissa algoritmer blir det enklare eller effektivare om du ändrar på plats i förändringsbar samling.

\begin{itemize}
\item Behöver du hantera värden i sekvens?
\begin{itemize}\SlideFontTiny
\item Om du klarar dig utan förändring av innehållet efter konstruktion:\\
\code{val}-referens till \code{Vector}
\item Om du behöver ändra innehåll men \Alert{inte} antal element:\\
\code{val}-referens till \code{Array}
\item Om du behöver ändra innehåll \Alert{och} antal element:
\\ \code{var}-referens till \code{Vector} och t.ex. metoden \code{patch}, eller \\
\code{val}-referens till \code{ArrayBuffer} och t.ex. metoden \code{insert}
\end{itemize}

\item Behöver du hantera värden \code{x} som ska vara unika?
\begin{itemize}\SlideFontTiny
\item Oföränderlig: \code{  Set}
\item Förändringsbar: \code{val}-referens till \code{scala.collection.mutable.Set}
\end{itemize}

\item Behöver du leta upp värden \code{x:Int} utifrån en nyckel av t.ex. String?
\begin{itemize}\SlideFontTiny
\item Oföränderlig: \code{Map[String, Int] }
\item Förändringsbar: \code{val}-referens till \code{scala.collection.mutable.Map[String, Int]}
\end{itemize}


\end{itemize}
\end{Slide}

% \begin{Slide}{ArrayBuffer}
% Ändra på plats: update, insert, remove, append
% {\SlideFontTiny

% \vspace{2.5em}\begin{tabular}{@{}p{4.2cm}  p{6.5cm}}
% \texttt{xs(i) = x \newline xs.update(i, x)} & Replace element at index i with x. \newline Return type Unit.\\   \cline{1-2}

% \texttt{xs.insert(i, x)\newline xs.remove(i)} & Insert x at index \texttt{i}. Remove element at i. \newline Return type Unit.\\   \cline{1-2}

% \texttt{xs.append(x)~~~xs~+=~x} & Insert x at end.  Return type Unit.\\   \cline{1-2}

% \texttt{xs.prepend(x)~~x~+=:~xs} & Insert x in front.  Return type Unit.\\   \cline{1-2}

% \texttt{xs -= x} & Remove first occurance of x (if exists). \newline Returns xs itself. \\\cline{1-2}

% \texttt{xs ++= ys} & Appends all elements in ys to xs and returns xs itself. \\

% \end{tabular}
% }

% \end{Slide}


\Subsection{Tentatips}

\begin{Slide}{Före tentan:}\SlideFontSmall
\begin{enumerate}
\item Repetera övningar och labbar i kompendiet.
\item Läs igenom föreläsningsanteckningar.
\item Studera \Emph{snabbref} \Alert{mycket noga} så att du vet vad som är givet och var det står, så att du kan hitta det du behöver snabbt.
\item Skapa och \Emph{memorera} en personlig \Emph{checklista} med programmeringsfel du brukar göra, som även inkluderar småfel, så som glömda parenteser och semikolon, och annat som en kompilator/IDE normalt hittar.
\item Tänk igenom hur du ska disponera dina 5 timmar på tentan.
\item Gör minst en extenta som om det vore \Alert{skarpt läge}:
\begin{enumerate}\SlideFontTiny
\item Avsätt 5 ostörda timmar (stäng av telefon, dator etc).
\item Inga hjälpmedel. Bara snabbref.
\item Förbered dryck och tilltugg.
\end{enumerate}
\end{enumerate}
\end{Slide}

\begin{Slide}{På tentan:} \SlideFontTiny
\begin{enumerate}
\item Läs igenom \Alert{hela} tentan först. \\ \Emph{Varför?} Förstå helheten. Delarna hänger ihop.
\item Notera och begrunda specifika begrepp och definitioner. \\ \Emph{Varför?} Begreppen är avgörande för förståelsen av uppgiften.
\item Notera förenklingar, antaganden och specialfall. \\ \Emph{Varför?} Uppgiften blir mkt enklare om du inte behöver hantera dessa.
\item \Alert{Fråga} tentamensansvarig om du inte förstår uppgiften -- speciellt om det finns misstänkta felaktigheter eller förmodat oavsiktliga oklarheter. \\ \Emph{Varför?} Det är inte lätt att konstruera en ''perfekt'' tenta. \\ Du får fråga vad du vill, men det är inte säkert du får svar...
\item Läs specifikationskommentarerna och metodsignaturerna i alla givna klass-specifikationer \Alert{mycket noga}. \\ \Emph{Varför?} Det är ett vanligt misstag att förbise de ledtrådar som ges där.
\item Återskapa din memorerade personliga checklista för vanliga fel som du brukar göra och avsätt tid till att gå igenom den på tentan. Varje fix plockar poäng!
\item Lämna in ett försök även om du vet att lösningen inte är fullständig. Det gäller att plocka så många poäng det går. En ofullständig lösning kan ändå ge poäng.

\item Om du har svårigheter kan det bli kamp mot klockan. Försök hålla huvudet kallt och prioritera utifrån var du kan plocka flest poäng. Ge inte upp! Ta en kort äta-dricka-paus för att få mer energi!

\end{enumerate}
\end{Slide}

\ifkompendium\else

\begin{Slide}{Planeringstips}\SlideFontTiny
Exempel på saker som du kan lägga in tid för i din julpluggkalender:
\begin{enumerate}
\item Ta reda på vad just \Alert{du} behöver träna på!
\item Välja ut övningar att repetera.
\item Repetera övning X, Y, Z, ... Både läsa och skriva kod. Fundera på typ och värde.
\item Välja ut labbar att repetera.
\item Repetera labb X, Y, Z, ... Lär dig ''trick'' och ''mönster''.
\item Träna på att skriva program med papper och penna.
\item Gör så många extentor du orkar, simulera ''skarp läge''.
\item Gör en checklista med vanliga fel och misstag som du brukar göra.
%\item Det finns inte så många Scala-extentor, men du kan också göra Java-extentor och lösa vissa delar i Scala och vissa delar i Java beroende på vad du behöver träna på.

\item Läsa igenom alla de extentor som du väljer att inte göra ''i fiktivt skarpt läge'' och studera generella mönster och typiska trick.
\end{enumerate}
\end{Slide}

\begin{Slide}{Tentans struktur}
\begin{itemize}\SlideFontSmall
\item Del A 20\%:\\\Emph{Evaluera uttryck} där du ska \Alert{ange typ och värde}
\begin{itemize}\SlideFontTiny
\item Testar förståelse av variabler, uttryck, samlingar, algoritmer, arv, etc.
\item Det är bra/nödvändigt att anteckna delsteg och variablers värden, då det är mycket svårt att tänka ut svaren direkt i huvudet.
%\item Ev. ''rättningströskel'': \textit{Om du på del A erhåller färre poäng än vad som krävs för att nå upp till en bestämd ''rättningströskel'', kan din tentamen komma att underkännas utan att del B bedöms.}
\end{itemize}


\item Del B 80\%:\\\Emph{Skriva kod} som uppfyller \Alert{krav och design}
\begin{itemize}\SlideFontTiny
\item Testar att du själv kan skapa kod med delar som samverkar
\item Testar förmåga att gå från indata-utdata till algoritm \\
 givet: ledtrådar, design, ev. skiss på lösning, ev. pseudokod etc.
\end{itemize}
\item Blanka inlämningar ger 0 poäng; det är alltid bättre att försöka än att lämna in blankt. Lämna inte in kladdpapper eller dubbla lösningar.
\end{itemize}
\end{Slide}


\begin{Slide}{Vad kommer på tentan?}
\begin{itemize}
\item Grundläggande begrepp och det som tränas på grundövningar och labbar är basen för att bli godkänd.
\item Begrepp, föreläsningsbilder och övningar som är markerade \Emph{''fördjupning''} krävs ej för att klara tentan men ökar förståelsen och hjälper dig att nå högre betyg.
\item Det är helt ok på tentan om du väljer en \Emph{enkel lösning med basala begrepp} \Alert{som fungerar bra}, i stället för en kortare/elegantare/mer avancerad lösning.
\item Extra-övningarna i läsvecka 12 ingår ej på tentan.
\end{itemize}
\end{Slide}


\Subsection{Avslutning}

\begin{Slide}{När du om några år tänker tillbaka på pgk...}
...hoppas jag du uppskattar den allmänbildning du fick om:
\begin{itemize}
  \item sekvens -- alternativ -- repetition -- abstraktion 
  \item viktiga datavetenskapliga idéer\\namnrymd, datatyp, samling, uppdelning i delproblem, ...
  \item grundläggande algoritmer\\registrering, linjärsökning, insättningssortering, ...
  \item träning i att självständigt skapa lättläst kod
  \item färdighet i användning av programmeringsverktyg  
\end{itemize}

\vspace{1em} Bonus: ett värdefullt socialt sammanhang med framtida kollegor.
\end{Slide}

\begin{Slide}{Scala då, nu och i framtiden}\SlideFontSmall

% {\SlideFontSize{7}{10}\url{
% https://en.wikipedia.org/wiki/Scala_(programming_language)#Versions
% }}
{\SlideFontSize{12}{13} Scalas övergripande målsättning: \Emph{smidigt} OCH \Alert{säkert}}
\begin{itemize}
\item Scala 1.0 (2003) första pre-release
\item Scala 2.0-2.9 (2006-2011) pionjärer: Twitter, LinkedIn, The Guardian, ...
\item Scala 2.10 (2013) brett genombrott, viktiga språkutvidgningar
\item Scala 2.11 (2014) allmän industriell spridning, stabilitet, prestanda, \\
% {\SlideFontSize{7}{10}\url{
% https://en.wikipedia.org/wiki/Scala_(programming_language)#Companies
% }}
\item Scala 2.12 (2016) fokus på prestanda, snabbare bytekod: lambda i JVM
\item Scala 2.13 (2019) fokus på standardbiblioteket och \code{scala.collection}, migreringsverktyg för Scala 3
\item Scala 3.0 (2021): \Alert{stort} \Emph{tekniksprång} med många nya språkdelar %:\\enum, top-level defs, @main, trait params, given, export, creator applications, ...,\\ "uppstädning" + förenklingar baserat på lärdomar från Scala 2.
\item Scala 3.8 (2026): första experimentella steget mot ännu säkrare system (fångstkontroll, nullkontroll, strikt likhet, ...)  
\end{itemize}

% Historiker på wikipedia är tyvärr inte uppdaterad...
%Läs mer om historik här: \url{https://en.wikipedia.org/wiki/Scala_(programming_language)}

Läs mer om nya experimentella delar i Scala här: \url{https://docs.scala-lang.org/scala3/reference/experimental}
\end{Slide}

% \begin{Slide}{Scala 3}\SlideFontSmall
% \begin{itemize}
%   \item Scala 3 släpptes i början av 2021.
%   \item Nya språkkonstruktioner, t.ex.  optional braces, if then else, while do, enum, top-level defs, @main, trait params, extension methods, given, using, export, creator applications, union types, intersection types,  ...,
%   \item \Emph{Tasty}: nytt format för kodträd som kompletterar bytekod och möjliggör omkompilering i efterhand och korsvis användning Scala 2 \& 3. \\
%   \item Formell bas för Scala: DOT (en algebra för dependent object types)
% \item Några viktiga Scala-ramverk för stordata, massiv parallellism, AI:
% \begin{itemize}\SlideFontTiny
%   \item \href{https://akka.io/}{Akka} ramverk för skalbara parallella arkitekturer
%   \item \href{https://spark.apache.org/}{Apache Spark} för parallell behandling av stordata i molnet, för AI, ML ...
%   \item \href{https://en.wikipedia.org/wiki/Apache_Kafka}{Apache Kafka} för hantering av strömmande data (initierad av LinkedIn)
%   \item \href{https://www.playframework.com/}{Play framework} för moderna, skalbara webbappar
% \end{itemize}
% \item Flera ''backends'' som breddar Scalas användningsområde:
% \begin{itemize}\SlideFontTiny
%   \item \href{http://www.scala-js.org/}{scala-js.org}: dela kod+kompetens mellan backend och frontend
%   \item \href{http://scala-native.org}{scala-native.org}: kör Scala kompilerat direkt ''på metallen''
% \end{itemize}
% \end{itemize}
% \end{Slide}


% \begin{Slide}{Scala på JVM, Scala JS, Scala Native}
% \begin{multicols}{3}

% \begin{tikzpicture}[node distance=1.4cm]
% \node (input) [startstop] {Scala-kod};
% \node (compile) [process, below of=input] {\texttt{scalac}};
% \node (output) [startstop, below of=compile] {byte-kod};
% \node (interp) [process, below of=output] {JVM};
% %\node (output2) [startstop, below of=interp] {maskinkod};
% \draw [arrow] (input) -- (compile);
% \draw [arrow] (compile) -- (output);
% \draw [arrow] (output) -- (interp);
% %\draw [arrow] (interp) -- (output2);
% \end{tikzpicture}


% \columnbreak %---------

% %https://www.sharelatex.com/blog/2013/08/29/tikz-series-pt3.html
% \begin{tikzpicture}[node distance=1.4cm]
% \node (input) [startstop] {Scala-kod};
% \node (compile) [process, below of=input] {\texttt{Scala JS}};
% \node (output) [startstop, below of=compile] {Javascript};
% \node (interp) [process, below of=output] {Browser | NodeJS};
% %\node (output2) [process, right of=interp, minimum size=6mm] {NodeJS};
% \draw [arrow] (input) -- (compile);
% \draw [arrow] (compile) -- (output);
% \draw [arrow] (output) -- (interp);
% %\draw [arrow] (interp) -- (output2);
% \end{tikzpicture}

% \columnbreak

% \begin{tikzpicture}[node distance=1.4cm]
% \node (input) [startstop] {Scala-kod};
% \node (compile) [process, below of=input] {\texttt{Scala Native}};
% \node (output) [startstop, below of=compile] {Mellankod (IR)};
% \node (interp) [process, below of=output] {LLVM};
% \node (output2) [startstop, below of=interp] {maskinkod};
% \draw [arrow] (input) -- (compile);
% \draw [arrow] (compile) -- (output);
% \draw [arrow] (output) -- (interp);
% \draw [arrow] (interp) -- (output2);
% \end{tikzpicture}

% \end{multicols} 
% \end{Slide}



\begin{Slide}{Hur håller jag mig uppdaterad om Scalas utveckling?}
\begin{itemize}%\SlideFontTiny
  \item Officiell blog: \url{https://www.scala-lang.org/blog/}
  \item Scala-nyheter: \url{http://scalatimes.com/}
%  \item Open online courses: \\\url{https://www.coursera.org/courses?query=scala}
  \item User-forum: \url{https://users.scala-lang.org/}
  \item Tjatt: \url{https://discord.gg/h9452YPJ}
  \item Scala Center: \url{https://scala.epfl.ch/}
  \item Scala-bibliotek: \url{https://index.scala-lang.org/}
  \item Contributors: \url{https://contributors.scala-lang.org/}
  \item Scala-språkets pågående förbättring: \url{https://docs.scala-lang.org/sips}
  % \item Scala Improvement Process: \\
  % \url{http://docs.scala-lang.org/sips/all.html}
\end{itemize}
\end{Slide}



\begin{Slide}{CEQ -- Course Experience Questionnaire}\SlideFontSmall
\begin{itemize}
\item Görs på hela LTH på samma sätt. Alla får länkar via mejl.
\item Snälla fyll i CEQ! Jag är \Alert{mycket tacksam} för all konstruktiv feedback! \\ Hög svarsfrekvens är viktigt för att kunna dra slutsatser om variationen i svaren och signifikansen i sammanställningen.
\item Del 1: Generella påståenden, alla med 5-gradig skala: \\ tar helt avstånd ... instämmer helt
\item Del 2: \Emph{Fritextfrågor}: \\
''Vad  tycker  du  var  det  bästa  med  den här  kursen?'' \\
''Vad  tycker  du  främst  behöver  förbättras?''
\item Mer om CEQ här: \url{https://www.ceq.lth.se/}
\item \Emph{Fördel} med CEQ: Samma alla kurser alla år medger jämförelse över tid.
\item \Alert{Begränsning}: Saknar frågor kopplat till specifika kursmoment.
\end{itemize}
\end{Slide}

\begin{Slide}{Kursspecifik utvärdering om specifika kursmoment}\SlideFontSmall
\begin{itemize}
\item Jag vill gärna att \Alert{alla} gör den LTH-gemensamma, anonyma kursutvärderingsenkäten \href{https://www.ceq.lth.se/}{CEQ}. Dina fritext-kommentarer om vad som är det bästa med kursen och vad som främst behöver förbättra emottages mycket tacksamt i CEQ-utvärderingen!
\item Jag kommer att komplettera CEQ med en \Emph{kursspecifik utvärdering} av specifika kursmoment i denna kurs och jag är därför \Alert{mycket tacksam} om alla fyller enkäten när länk kommer via anslag i Canvas.
\item Jag behandlar dina svar \Alert{konfidentiellt}, men sparar din email så att jag kan återkomma om jag mot förmodan undrar något mer.
\item Din input är \Emph{mycket värdefull} vid framtida kursutveckling!
\end{itemize}
\end{Slide}

\begin{Slide}{Intresserad av att arbeta som handledare?}
\begin{itemize}
\item Vi har ständigt behov av nya handledare i våra kurser
\item Det är lärorikt att jobba som handledare
\item Information om ansökningsprocessen: 
\begin{itemize}
  \item \url{https://www.cs.lth.se/utbildning/amanuenser}
  \item Sista ansökningsdatum för pgk är senare delen av \Alert{maj} (se exakt datum via länk från ovan sida)
  \item Skriv i ansökan om du helst jobbar i pgk.
  \item Ange övriga relevanta meriter, tex. ledarskap i föreningar, pedagogiska uppdrag, etc.
  \item Prata också gärna med handledare om hur det är att jobba i pgk eller mejla bjorn.regnell@cs.lth.se om du har frågor.
\end{itemize}
\end{itemize}
\end{Slide}


\begin{Slide}{Ett stort TACK...}
\begin{itemize}
  \item
... till alla \Emph{handledare} som jobbat hårt för att ni ska lära er så mycket som möjligt!
\item ... till alla \Alert{studenter} som gått kursen för:
\begin{itemize}
\item ... att ni kämpat så hårt!
\item ... att ni ställt massor med frågor!
\item ... att det har varit så hög närvaro på föreläsningarna!
\item ... att ni hjälp till med värdefull återkoppling!
\item ... att ni är så konstruktiva och verkligen vill lära er!
\end{itemize}
\vspace{2em} \pause

\end{itemize}
\Alert{Ett stort LYCKA TILL på vägen till att bli en \\ kompetent och innovativ systemutvecklare!}
\end{Slide}

\begin{Slide}{Hoppas att pgk-kursen varit givande!}
\includegraphics[width=5cm]{../img/gurka.jpg}\includegraphics[width=5cm]{../img/ukulele.jpg}
\end{Slide}


% subjekt och predikat -> public static void: https://www.youtube.com/watch?v=1ZPaR_wH-R8


% \begin{Slide}{Koda i Scala}

%   {\footnotesize\it Melodi: McDonalds-låten}
% % https://youtu.be/cTVhZqNwn3Y

% \begin{verbatim}

%           E         A             E          B 
% Det finns stunder i livet som man alltid har kvar

%           E           A               B 
% Det finns villkor och uttryck som man spar 

%         F#           B             F#           C#
% Och när koden är öppen finns gemenskap för fler 

% F#      B      C#       F#
% Koda i Scala; det ger meeeeeeer! 
% \end{verbatim}

% \end{Slide}


% \begin{Slide}{Ljuvliga språk}
% \fontsize{7}{8}\selectfont
%   \begin{verbatim}  
% F        F9   Fmaj7 F9        Fmaj7
% Å detta språk detta ljuvliga språk

% F             Gm    Gm7     C7
% som vi kallar bella Scala

% Gm                 Gm7       C7
% se vilken syn alla uttryck i skyn

%       Gm    C7    F
% detta ljuva bella Scala

% Cm                  Cm7  F7     Bbmaj7        Bb
% fjärran från det du älskar blir koden ödsligt tom

%     Dm7   G7     Dm7      G7        Gm7         C
% men i din närhet sluts du in i dess trolska rikedom

% C7#5   F        Am           Eb        D
% åh åh detta språk det är ungdomens språk

%        Gm     C     F      C7       
% som vi kallar bella Scala
% \end{verbatim}
% \end{Slide}




\fi
