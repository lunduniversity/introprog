%!TEX encoding = UTF-8 Unicode
%!TEX root = ../lect-w14.tex

%%%


%http://tex.stackexchange.com/questions/135393/how-to-draw-bar-pie-chart
% \definecolor{c1}{RGB}{220,57,18}
% \definecolor{c2}{RGB}{255,153,0}
% \definecolor{c3}{RGB}{102,140,217}
% \definecolor{c4}{RGB}{16,150,24}
% \definecolor{c5}{RGB}{153,0,153}




% \makeatletter

% \tikzstyle{chart}=[
%     legend label/.style={font={\scriptsize},anchor=west,align=left},
%     legend box/.style={rectangle, draw, minimum size=5pt},
%     axis/.style={black,semithick,->},
%     axis label/.style={anchor=east,font={\tiny}},
% ]

% \tikzstyle{bar chart}=[
%     chart,
%     bar width/.code={
%         \pgfmathparse{##1/2}
%         \global\let\bar@w\pgfmathresult
%     },
%     bar/.style={very thick, draw=white},
%     bar label/.style={font={\bf\small},anchor=north},
%     bar value/.style={font={\footnotesize}},
%     bar width=.75,
% ]

% \tikzstyle{pie chart}=[
%     chart,
%     slice/.style={line cap=round, line join=round, very thick,draw=white},
%     pie title/.style={font={\bf}},
%     slice type/.style 2 args={
%         ##1/.style={fill=##2},
%         values of ##1/.style={}
%     }
% ]

% \pgfdeclarelayer{background}
% \pgfdeclarelayer{foreground}
% \pgfsetlayers{background,main,foreground}


% \newcommand{\pie}[3][]{
%     \begin{scope}[#1]
%     \pgfmathsetmacro{\curA}{90}
%     \pgfmathsetmacro{\r}{1}
%     \def\c{(0,0)}
%     \node[pie title] at (90:1.3) {#2};
%     \foreach \v/\s in{#3}{
%         \pgfmathsetmacro{\deltaA}{\v/100*360}
%         \pgfmathsetmacro{\nextA}{\curA + \deltaA}
%         \pgfmathsetmacro{\midA}{(\curA+\nextA)/2}

%         \path[slice,\s] \c
%             -- +(\curA:\r)
%             arc (\curA:\nextA:\r)
%             -- cycle;
%         \pgfmathsetmacro{\d}{max((\deltaA * -(.5/50) + 1) , .5)}

%         \begin{pgfonlayer}{foreground}
%         \path \c -- node[pos=\d,pie values,values of \s]{$\v\%$} +(\midA:\r);
%         \end{pgfonlayer}

%         \global\let\curA\nextA
%     }
%     \end{scope}
% }

% \newcommand{\legend}[2][]{
%     \begin{scope}[#1]
%     \path
%         \foreach \n/\s in {#2}
%             {
%                   ++(0,-10pt) node[\s,legend box] {} +(5pt,0) node[legend label] {\n}
%             }
%     ;
%     \end{scope}
% }


\ifkompendium\else

\Subsection{Sista läsveckan}

\begin{Slide}{Sista läsveckan}

  \begin{itemize}
    \item Det är inga föreläsningar denna vecka.
    \item Endast resurstider schemalagda sista veckan, se schema i TimeEdit.
    \item Se instruktioner för \Emph{muntligt prov} här: \url{https://fileadmin.cs.lth.se/pgk/lect-w12.pdf}
    \item Gör klart och redovisa \Alert{alla} labbar och projektet
    \item Senaste datum för anmälning till valfria tentamen är i mitten av december, se: \\\href{https://www.student.lth.se/mina-studier/tentamen/}{www.student.lth.se/mina-studier/tentamen/}
  \end{itemize}

\end{Slide}



\fi
