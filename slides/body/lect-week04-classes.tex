%!TEX encoding = UTF-8 Unicode
%!TEX root = ../lect-week04.tex

\ifkompendium\else

\Subsection{Klasser}

\begin{Slide}{Vad är en klass?}\SlideFontSmall
Vi har tidigare deklarerat \Emph{singelobjekt} som bara finns i \Alert{en} \Emph{instans}:
\begin{REPLnonum}
scala> object Björn { var ålder = 49; val längd = 178 }
\end{REPLnonum}

Med en \Emph{klass} kan man skapa \Alert{godtyckligt många} \Emph{instanser av klassen} med hjälp av nyckelordet \code{new} följt av klassens namn:

\begin{REPLnonum}
scala> class Person { var ålder = 0; var längd = 0 }

scala> val björn = new Person
björn: Person = Person@7ae75ba6

scala> björn.ålder = 49

scala> björn.längd = 178
\end{REPLnonum}

\begin{itemize}

\item En klass kan ha \Emph{medlemmar} (i likhet med singelobjekt). 

\item Funktioner som är medlemmar kallas \Emph{metoder}.

\item Variabler som är medlemmar kallas \Emph{attribut}.


\end{itemize}

\end{Slide}


\begin{Slide}{Vid \texttt{new} allokeras plats i minnet för objektet}
\begin{REPLnonum}
scala> class Person { var ålder = 0; var längd = 0 }

scala> val björn = new Person
björn: Person = Person@7ae75ba6
\end{REPLnonum}

\begin{tikzpicture}[font=\large\sffamily]
\matrix [matrix of nodes, row sep=0, column 2/.style={nodes={rectangle,draw,minimum width=0.8cm}}] (mat) 
{
\texttt{björn}   &  \makebox(10,10){ }\\
};
\node[cloud, cloud puffs=13.0, cloud ignores aspect, minimum width=2cm, minimum height=3.8cm,
 align=center, draw] (x) at (5.8cm, -1.5cm) { 
 \begin{tabular}{r l}
 \multicolumn{2}{c}{\ttfamily\itshape Person@7ae75ba6}\\ \\
 \texttt{ålder} & \fbox{~0~} \\
 \texttt{längd} & \fbox{~0~}\\
 \end{tabular}
 };
\filldraw[black] (0.75cm,0.0cm) circle (3pt) node[] (ref) {};
\draw [arrow, line width=0.7mm] (ref) -- (x);
% \node[cloud, cloud puffs=15.7, cloud ignores aspect, %minimum width=5cm, minimum height=2cm,
% align=center, draw] (g2) at (5cm, -2cm) {Gurka-\\objekt};
% \filldraw[black] (0.4cm,-0.4cm) circle (3pt) node[] (g2ref) {};
% \draw [arrow] (g2ref) -- (g2);
\end{tikzpicture}
{\SlideFontTiny{\ttfamily\itshape Person@7ae75ba6} är en unik idenfierare för instansen, så att JVM hittar den i heapen.}
\end{Slide}



\begin{Slide}{Med punktnotation kan förändringsbara variabler tilldelas nya värden och objektets tillstånd uppdateras.}
\begin{REPLnonum}
scala> björn.ålder = 49
scala> björn.längd = 178
\end{REPLnonum}

\begin{tikzpicture}[font=\large\sffamily]
\matrix [matrix of nodes, row sep=0, column 2/.style={nodes={rectangle,draw,minimum width=0.8cm}}] (mat) 
{
\texttt{björn}   &  \makebox(10,10){ }\\
};
\node[cloud, cloud puffs=13.0, cloud ignores aspect, minimum width=2cm, minimum height=3.8cm,
 align=center, draw] (x) at (5.8cm, -1.5cm) { 
 \begin{tabular}{r l}
 \multicolumn{2}{c}{\ttfamily\itshape Person@7ae75ba6}\\ \\
 \texttt{ålder} & \fbox{~49~~} \\
 \texttt{längd} & \fbox{~178}\\
 \end{tabular}
 };
\filldraw[black] (0.75cm,0.0cm) circle (3pt) node[] (ref) {};
\draw [arrow, line width=0.7mm] (ref) -- (x);
% \node[cloud, cloud puffs=15.7, cloud ignores aspect, %minimum width=5cm, minimum height=2cm,
% align=center, draw] (g2) at (5cm, -2cm) {Gurka-\\objekt};
% \filldraw[black] (0.4cm,-0.4cm) circle (3pt) node[] (g2ref) {};
% \draw [arrow] (g2ref) -- (g2);
\end{tikzpicture}
\end{Slide}





\begin{Slide}{En klass kan ha parametrar som initialiserar attribut}
\begin{itemize}
\item Med en parameterlista efter klassnamnet får man en så kallad \Emph{primärkonstruktor} för initialisering av attribut. 
\item Argumenten för initialiseringen ges vid \code{new}.
\begin{REPLnonum}
scala> class Person(var ålder: Int, var längd: Int)

scala> val björn = new Person(49, 178)
björn: Person = Person@354baab2

scala> println(s"Björn är ${björn.ålder} år gammal.")
Björn är 49 år gammal.

scala> björn.ålder = 18

scala> println(s"Björn är ${björn.ålder} år gammal.")
Björn är 18 år gammal.
\end{REPLnonum}
\end{itemize}
\end{Slide}




\begin{Slide}{En klass kan ha privata medlemmar}
Med \code{private} blir en medlem \Emph{privat}: access utifrån \Alert{medges ej}.

\vspace{0.1em}
\begin{REPL}
scala> class Person(private var minÅlder: Int, private var minLängd: Int){
         def ålder = minÅlder
       }

scala> val björn = new Person(42, 178)
björn: Person = Person@4b682e71

scala> println(s"Björn är ${björn.ålder} år gammal.")
Björn är 42 år gammal.

scala> björn.minÅlder = 18
error: variable minÅlder in class Person cannot be accessed in Person

scala> björn.längd
error: value längd is not a member of Person
\end{REPL}
Med \code{private} kan man förhindra tokiga förändringar.
\end{Slide}


\begin{Slide}{Privata förändringsbara attribut och publika metoder}
\begin{Code}
class Människa(val födelseLängd: Double, val födelseVikt: Double){
  private var minLängd = födelseLängd
  private var minVikt  = födelseVikt
  private var ålder    = 0
    
  def längd = minLängd  // en sådan här metod kallas "getter"
  def vikt  = minVikt   // vi förhindrar attributändring "utanför" klassen
    
  val slutaVäxaÅlder      = 18
  val tillväxtfaktorLängd = 0.00001
  val tillväxtfaktorVikt  = 0.0002

  def ät(mat: Double): Unit = {
    if (ålder < slutaVäxaÅlder) minLängd += tillväxtfaktorLängd * mat
    minVikt += tillväxtfaktorVikt * mat
  }
  
  def fyllÅr: Unit = ålder += 1
  
  def tillstånd: String = s"Tillstånd: $minVikt kg, $minLängd cm, $ålder år"
}
\end{Code}
\end{Slide}

\begin{Slide}{Tillstånd kan förändras indirekt genom metodanrop}
\begin{REPL}
scala> val björn = new Människa(födelseVikt=3.5, födelseLängd=52.1)
björn: Människa = Människa3e52

scala> björn.tillstånd
res0: String = Tillstånd: 3.5 kg, 52.1 cm, 0 år

scala> for (i <- 1 to 42) björn.fyllÅr

scala> björn.tillstånd
res2: String = Tillstånd: 3.5 kg, 52.1 cm, 42 år

scala> björn.ät(mat=5000)

scala> björn.tillstånd
res3: String = Tillstånd: 4.5 kg, 52.1 cm, 42 år
\end{REPL}
\end{Slide}



\begin{Slide}{Metoden \texttt{isInstanceOf} och rot-typen \texttt{Any}}
\SlideFontSmall\vspace{-0.5em}
\begin{multicols}{2}

\begin{REPL}
scala> class X(val i: Int) 

scala> val a = new X(42)
a: X = X@117b2cc6

scala> a.isInstanceOf[X]
res0: Boolean = true

scala> val b = new X(42)
b: X = X@61ab6521

scala> b.isInstanceOf[X]
res1: Boolean = true

scala> a == b
res2: Boolean = false

scala> a.i == b.i
res3: Boolean = true

\end{REPL}

\columnbreak


\begin{itemize}\SlideFontTiny

\item Ett objekt skapat med \code{new X} är en instans av \Emph{typen} \code{X}. 

\item Detta kan testas med metoden \code{isInstanceOf[X]: Boolean}

\pause

\item Typen \Emph{\texttt{Any}} är sypertyp till \Alert{alla} typer och kallas för \Emph{rot-typ} i Scalas  typhierarki. 

\begin{REPL}
scala> a.isInstanceOf[Any]
res4: Boolean = true

scala> b.isInstanceOf[Any]
res5: Boolean = true

scala> 42.isInstanceOf[Any]
res6: Boolean = true

\end{REPL}
\item Se quickref sid 4. (Mer i w07.) 
\item I klassen \href{http://www.scala-lang.org/api/current/#scala.Any}{\code{Any}} finns bl.a. \code{toString}
\end{itemize}
%{\SlideFontTiny \hfill(se quickref sid 4, mer om detta i w07)}
\end{multicols}
\end{Slide}



\begin{Slide}{Överskugga \texttt{toString}}
Alla objekt får automatiskt en metod \code{toString} som ger en sträng med objektets unika identifierare, här \texttt{Gurka@3830f1c0}:
\begin{REPL}
scala> class Gurka(val vikt: Int) 

scala> val g = new Gurka(42)
g: Gurka = Gurka@3830f1c0

scala> g.toString
res0: String = Gurka@3830f1c0
\end{REPL}
Man kan \Emph{överskugga} den automatiska \code{toString}  med en \Alert{egen implementation}. Observera nyckerordet \code{override}.
\begin{REPL}
scala> class Tomat(val vikt: Int){override def toString = s"Tomat($vikt g)"} 

scala> val t = new Tomat(142)
t: Tomat = Tomat(142 g)

scala> t.toString
res1: String = Tomat(142 g)

\end{REPL}
\end{Slide}





\begin{Slide}{Objektfabrik i kompanjonsobjekt}%\SlideFontSmall
\begin{itemize}
\item Om det finns ett objekt i samma kodfil med samma namn som klassen blir det objektet ett s.k.  \Emph{kompanjonsobjekt} \Eng{companion object}.

\item Ett kompanjonsobjekt får \Alert{accessa privata medelmmar} i den klass till vilken objektet är kompanjon.

\item Kompanjonsobjekt är en bra plats för s.k. \Emph{fabriksmetoder} som skapar instanser. Då slipper vi skriva \code{new}.
\begin{REPL}
scala> :paste   // måste skrivas tillsammans annars ingen kompanjon

class Broccoli(var vikt: Int) 

object Broccoli {
  def apply(vikt: Int) = new Broccoli(vikt)
}

scala> val b = Broccoli(420)
b: Broccoli = Broccoli@32e8d5a4
\end{REPL}

\end{itemize}
\end{Slide}


\begin{Slide}{Kompanjonsobjekt kan accessa privata medlemmar}%\SlideFontSmall
\begin{Code}
class Gurka(startVikt: Double) {
  private var vikt = startVikt
  def ät(tugga: Int): Unit = if (vikt > tugga) vikt -= tugga else vikt = 0 
  override def toString = s"Gurka($vikt)"
}
object Gurka {
  private var totalVikt = 0.0
  def apply(): Gurka = {
    val g = new Gurka(math.random * 0.42 + 0.1)
    totalVikt += g.vikt  // hade blivit kompileringsfel om ej vore kompanjon
    g
  }
  def rapport: String = s"Du har skapat ${totalVikt.toInt} kg gurka." 
}
\end{Code}
\pause
\begin{REPL}
scala> val gs = Vector.fill(1000)(Gurka())
gs: scala.collection.immutable.Vector[Gurka] = 
  Vector(Gurka(0.49018400799506734), Gurka(0.2462822679714138), Gurka(0.17391397513818804), Gurka(0.5146514905924656), Gurka(0.47077333689159606)

scala> println(Gurka.rapport)
Du har skapat 305 kg gurka.

\end{REPL}

\end{Slide}






\begin{Slide}{Förändringsbara och oföränderliga objekt}
Ett \Emph{oföränderligt objekt} där nya instanser skapas i stället för tillståndsändring ''på plats''.
\begin{Code}
class Point(val x: Int, val y: Int) {
  def moved(dx: Int, dy: Int): Point = new Point(x + dx, y + dy)

  override def toString: String = s"Point($x, $y)"
}
\end{Code}

Ett \Alert{förändringsbart} objekt där \Alert{tillståndet uppdateras}.
\begin{Code}
class MutablePoint(private var x: Int, private var y: Int) {
  def move(dx: Int, dy: Int): Unit = {x += dx; y += dy}  // Mutation!!!

  override def toString: String = s"MutablePoint($x, $y)"
}
\end{Code}
\end{Slide}


\begin{Slide}{Oföränderliga objekt}

\begin{minipage}{0.5\textwidth}
\begin{REPL}
scala> var p1 = new Point(3, 4)
p1: Point = Point(3, 4)

scala> val p2 = p1.moved(2, 3)
p2: Point = Point(5, 7)

scala> println(p1)
Point(3, 4)

scala> p1 = new Point(0, 0)
p1: Point = Point(0, 0)
\end{REPL}
\end{minipage}
\pause\begin{minipage}{0.49\textwidth}
{\SlideFontSmall \hfill Minnessituationen efter rad 7:}

\vspace{1em}
\begin{tikzpicture}[font=\SlideFontSmall\sffamily,scale=0.75, every node/.style={scale=0.75}]
\matrix [matrix of nodes, row sep=0, column 2/.style={nodes={rectangle,draw,minimum width=0.6cm}}] (mat) 
{
\texttt{p1}   &  \makebox(7,7){ }\\
};
\node[cloud, cloud puffs=13.0, cloud ignores aspect, minimum width=2cm, minimum height=1cm,
 align=center, draw] (x) at (3cm, -0.0cm) { 
 \begin{tabular}{r l}
 \texttt{x} & \fbox{~3~} \\
 \texttt{y} & \fbox{~4~}\\
 \end{tabular}
 };
\filldraw[black] (0.25cm,0.0cm) circle (3pt) node[] (ref) {};
\draw [arrow, line width=0.5mm] (ref) -- (x);
\end{tikzpicture}

\begin{tikzpicture}[font=\SlideFontSmall\sffamily,scale=0.75, every node/.style={scale=0.75}]
\matrix [matrix of nodes, row sep=0, column 2/.style={nodes={rectangle,draw,minimum width=0.6cm}}] (mat) 
{
\texttt{p2}   &  \makebox(7,7){ }\\
};
\node[cloud, cloud puffs=13.0, cloud ignores aspect, minimum width=2cm, minimum height=1cm,
 align=center, draw] (x) at (3cm, -0.0cm) { 
 \begin{tabular}{r l}
 \texttt{x} & \fbox{~5~} \\
 \texttt{y} & \fbox{~7~}\\
 \end{tabular}
 };
\filldraw[black] (0.25cm,0.0cm) circle (3pt) node[] (ref) {};
\draw [arrow, line width=0.5mm] (ref) -- (x);
\end{tikzpicture}

\end{minipage}

\end{Slide}



\begin{Slide}{Oföränderliga objekt}

\begin{minipage}{0.5\textwidth}
\begin{REPL}
scala> var p1 = new Point(3, 4)
p1: Point = Point(3, 4)

scala> val p2 = p1.moved(2, 3)
p2: Point = Point(5, 7)

scala> println(p1)
Point(3, 4)

scala> p1 = new Point(0, 0)
p1: Point = Point(0, 0)
\end{REPL}
\end{minipage}
\begin{minipage}{0.49\textwidth}
{\SlideFontSmall \hfill Minnessituationen efter rad 10:}

\vspace{1em}
\begin{tikzpicture}[font=\SlideFontSmall\sffamily,scale=0.75, every node/.style={scale=0.75}]
\node[cloud, cloud puffs=13.0, cloud ignores aspect, minimum width=2cm, minimum height=1cm,
 align=center, draw] (x) at (3cm, 2.0cm) { 
 \begin{tabular}{r l}
 \texttt{x} & \fbox{~3~} \\
 \texttt{y} & \fbox{~4~}\\
 \end{tabular}
 };
 
 \node[left of=x, text width=2.5cm,align=right] (text) at (1,2) {kommer att raderas av skräpsamlaren:};
\end{tikzpicture}

\begin{tikzpicture}[font=\SlideFontSmall\sffamily,scale=0.75, every node/.style={scale=0.75}]
\matrix [matrix of nodes, row sep=0, column 2/.style={nodes={rectangle,draw,minimum width=0.6cm}}] (mat) 
{
\texttt{p2}   &  \makebox(7,7){ }\\
};
\node[cloud, cloud puffs=13.0, cloud ignores aspect, minimum width=2cm, minimum height=1cm,
 align=center, draw] (x) at (3cm, -0.0cm) { 
 \begin{tabular}{r l}
 \texttt{x} & \fbox{~5~} \\
 \texttt{y} & \fbox{~7~}\\
 \end{tabular}
 };
\filldraw[black] (0.25cm,0.0cm) circle (3pt) node[] (ref) {};
\draw [arrow, line width=0.5mm] (ref) -- (x);
\end{tikzpicture}

\begin{tikzpicture}[font=\SlideFontSmall\sffamily,scale=0.75, every node/.style={scale=0.75}]
\matrix [matrix of nodes, row sep=0, column 2/.style={nodes={rectangle,draw,minimum width=0.6cm}}] (mat) 
{
\texttt{p1}   &  \makebox(7,7){ }\\
};
\node[cloud, cloud puffs=13.0, cloud ignores aspect, minimum width=2cm, minimum height=1cm,
 align=center, draw] (x) at (3cm, -0.0cm) { 
 \begin{tabular}{r l}
 \texttt{x} & \fbox{~0~} \\
 \texttt{y} & \fbox{~0~}\\
 \end{tabular}
 };
\filldraw[black] (0.25cm,0.0cm) circle (3pt) node[] (ref) {};
\draw [arrow, line width=0.5mm] (ref) -- (x);
\end{tikzpicture}

\end{minipage}

\pause\vspace{1em}Vi kan \Emph{lugnt dela referenser} till vårt oföränderliga objekt eftersom det \Emph{aldrig} kommer att ändras.

\end{Slide}


\newcommand{\MutaVarning}{\vspace{2em}\Alert{Varning!} Vem som helst som har tillgång till en referens till ditt förändringsbara objekt kan \Alert{manipulera} det, vilket ibland ger överaskande och \Alert{problematiska} konsekvenser!}



\begin{Slide}{Förändringsbara objekt}

\begin{minipage}{0.5\textwidth}
\begin{REPL}
scala> val mp1 = new MutablePoint(3, 4)
mp1: MutablePoint = MutablePoint(3, 4)

scala> val mp2 = mp1
mp2: MutablePoint = MutablePoint(3, 4)

scala> mp1.move(2,3)

scala> println(mp2)
MutablePoint(5, 7)
\end{REPL}
\end{minipage}
\begin{minipage}{0.49\textwidth}
{\SlideFontSmall \hfill Minnessituationen efter rad 4:}

\vspace{1em}
\begin{tikzpicture}[font=\SlideFontSmall\sffamily,scale=0.75, every node/.style={scale=0.75}]
\matrix [matrix of nodes, row sep=0.5cm, column 2/.style={nodes={rectangle,draw,minimum width=0.6cm}}] (mat) 
{
\texttt{mp1}   &  \makebox(7,7){ }\\
\texttt{mp2}   &  \makebox(7,7){ }\\
};
\node[cloud, cloud puffs=13.0, cloud ignores aspect, minimum width=2cm, minimum height=1cm,
 align=center, draw] (x) at (3cm, -0.0cm) { 
 \begin{tabular}{r l}
 \texttt{x} & \fbox{~3~} \\
 \texttt{y} & \fbox{~4~}\\
 \end{tabular}
 };
\filldraw[black] (0.35cm,0.65cm) circle (3pt) node[] (ref1) {};
\draw [arrow, line width=0.5mm] (ref1) -- (x);

\filldraw[black] (0.35cm,-0.65cm) circle (3pt) node[] (ref2) {};
\draw [arrow, line width=0.5mm] (ref2) -- (x);


\end{tikzpicture}

\end{minipage}

\pause\MutaVarning
\end{Slide}




\begin{Slide}{Förändringsbara objekt}

\begin{minipage}{0.5\textwidth}
\begin{REPL}
scala> val mp1 = new MutablePoint(3, 4)
mp1: MutablePoint = MutablePoint(3, 4)

scala> val mp2 = mp1
mp2: MutablePoint = MutablePoint(3, 4)

scala> mp1.move(2,3)

scala> println(mp2)
MutablePoint(5, 7)
\end{REPL}
\end{minipage}
\begin{minipage}{0.49\textwidth}
{\SlideFontSmall \hfill Minnessituationen efter \Alert{rad 7}:}

\vspace{1em}
\begin{tikzpicture}[font=\SlideFontSmall\sffamily,scale=0.75, every node/.style={scale=0.75}]
\matrix [matrix of nodes, row sep=0.5cm, column 2/.style={nodes={rectangle,draw,minimum width=0.6cm}}] (mat) 
{
\texttt{mp1}   &  \makebox(7,7){ }\\
\texttt{mp2}   &  \makebox(7,7){ }\\
};
\node[cloud, cloud puffs=13.0, cloud ignores aspect, minimum width=2cm, minimum height=1cm,
 align=center, draw] (x) at (3cm, -0.0cm) { 
 \begin{tabular}{r l}
 \texttt{x} & \fbox{~5~} \\
 \texttt{y} & \fbox{~7~}\\
 \end{tabular}
 };
\filldraw[black] (0.35cm,0.65cm) circle (3pt) node[] (ref1) {};
\draw [arrow, line width=0.5mm] (ref1) -- (x);

\filldraw[black] (0.35cm,-0.65cm) circle (3pt) node[] (ref2) {};
\draw [arrow, line width=0.5mm] (ref2) -- (x);


\end{tikzpicture}

\end{minipage}

\MutaVarning
\end{Slide}





\Subsection{Case-klasser}

\begin{Slide}{Vad är en case-klass?}\SlideFontSmall
\setlength{\leftmargini}{0pt}
\begin{itemize}
\item En \code{case}-klass är ett smidigt sätt att skapa \Emph{oföränderliga objekt}.
\item Kompilatorn ger dig \Alert{en massa ''godis''} på köpet (ca 50-100 rader kod), inkl.:
\begin{itemize}\SlideFontTiny
\item klassparametrar blir automatiskt \code{val}-attribut, alltså \Emph{publika} och \Emph{oföränderliga},
\item en automatisk \Emph{\texttt{toString}} som visar klassparametrarnas värde, 
\item ett automatiskt \Emph{kompanjonsobjekt} med \Emph{fabriksmetod} så du slipper skriva \code{new},
\item automatiska metoden \Emph{\texttt{copy}} för att skapa kopior med andra attributvärden, m.m...
\item[] (Mer om detta i w06 \& w11, men är du nyfiken kolla på uppgift 2d) på sid 261.)
\end{itemize}

\pause
\item Det \Alert{enda} du behöver göra är att lägga till nyckelordet \code{case} före \code{class}...
\end{itemize}

\vspace{-0.5em}\begin{REPLnonum}
scala> case class Point(x: Int, y: Int)

scala> val p = Point(3, 5)
p: Point = Point(3,5)

scala> p.  // tryck TAB och se lite av allt case-klass-godis
scala> Point.  // tryck TAB och se ännu mer godis

scala> val p2 = p.copy(y= 30)
p2: Point = Point(3,30)
\end{REPLnonum}


\end{Slide}


\begin{Slide}{Exempel på case-klasser} 
\begin{Code}
case class Person(namn: String, ålder: Int) {
  def fyllerJämt: Boolean = ålder % 10 == 0
  def hyllning = if (fyllerJämt) "Extra grattis!" else "Vi gratulerar!"
  def ärLikaGammalSom(annan: Person) = ålder == annan.ålder
}

case class Point(x: Int = 0, y: Int = 0) {
  def distanceTo(other: Point) = math.hypot(x - other.x, y - other.y)
  def dx(d: Int): Point = copy(x + d, y)
  def dy(d: Int): Point = copy(y= y + d)  //namngivet arg. och defaultarg.
}
object Point { 
  def origin = new Point() 
}
\end{Code}

\begin{REPL}
scala> Point().dx(10).dy(10).dx(32)
res0: Point = Point(42,10)

scala> Point(3,4) distanceTo Point.origin
res1: Double = 5.0

\end{REPL}
\end{Slide}

\begin{Slide}{Synlighet av klassparametrar i klasser \& case-klasser}\SlideFontSmall
\code{private[this]} är \Alert{ännu} mer privat än \code{private} 
\begin{Code}
class Hemlis(private val hemlis: Int) {
  def ärSammaSom(annan: Hemlis) = hemlis == annan.hemlis   // Funkar!
}

class Hemligare(private[this] val hemlis: Int) {
  def ärSammaSom(annan: Hemligare) = hemlis == annan.hemlis //KOMPILERINGSFEL
}
\end{Code}
Vad händer om man inte skriver något? Olika för klass och case-klass:
\begin{Code}
class Hemligare(hemlis: Int) { // motsvarar private[this] val
  def ärSammaSom(annan: Hemligare) = hemlis == annan.hemlis //KOMPILERINGSFEL
}

case class InteHemlig(seMenInteRöra: Int) { // blir automatiskt val 
  def ärSammaSom(annan: InteHemlig): Boolean = 
    seMenInteRöra == annan.seMenInteRöra 
}

\end{Code}
\end{Slide}

\fi




