%!TEX encoding = UTF-8 Unicode
%!TEX root = ../lect-week04.tex

\ifkompendium\else
\Subsection{Samlingar}

\begin{Slide}{Vad är en samling?}
En \Emph{samling} \Eng{collection} är en datastruktur som kan innehålla många element av \Alert{samma typ}.

\pause 
\vspace{2em}\emph{Exempel:} \\Heltalsvektor: \hfill\code{val xs = Vector(2, -1, 3, 42, 0)}

\pause 
{\SlideFontSmall\vspace{2em}Samlingar implementeras med hjälp av klasser. \\ I standardbiblioteken \code{scala.collection} och \code{java.util} finns \Alert{många} \Emph{färdiga samlingar}, så man behöver sällan implementera egna. 

\pause\vspace{0.5em}\emph{Om} man behöver en egen, speciell datastruktur är det ofta lämpligt att skapa en klass som \emph{innehåller} en \emph{färdig} samling och utgå från dess färdiga metoder.

}

\end{Slide}


\begin{Slide}{Typparameter möjliggör generiska samlingar}\SlideFontSmall
Funktioner och klasser kan, förutom vanliga parametrar, även ha \Emph{typparametrar} som skrivs i en egen parameterlista med \Alert{hakparenteser}. En typparameter gör så att funktioner och datastrukturer blir \Emph{generiska} och kan hantera element av \Alert{godtycklig} typ på ett typsäkert sätt. (Mer om detta i w09.)

\begin{REPLnonum}
scala> def strängLängd[T](x: T): Int = x.toString.length
strängLängd: [T](x: T)Int

scala> strängLängd[Double](42.0)  //Double är typargument
res0: Int = 4

scala> strängLängd(42.0) //Kompilatorn härleder T=Double
res1: Int = 4

scala> Vector.empty[Int] //Här kan den ej härleda typen...
res2: scala.collection.immutable.Vector[Int] = Vector()

scala> strängLängd[Vector[Int]](Vector.empty) //...men här
res3: Int = 8
\end{REPLnonum}
\end{Slide}

\fi

\ifkompendium
\begin{Slide}{Hierarki av samlingar i scala.collection}
\includegraphics[width=1.0\textwidth]{../img/collection/collection-traits}
\end{Slide}
\noindent Läs mer om Scalas samlingar här: \\ 
\url{http://docs.scala-lang.org/overviews/collections/overview}
\fi

\ifkompendium\else

\begin{Slide}{Hierarki av samlingstyper i \texttt{scala.collection}}

\begin{multicols}{2}
\begin{tikzpicture}[sibling distance=6.1em,->,>=stealth', inner sep=3pt, %scale=0.5, 
  every node/.style = {shape=rectangle, draw, align=center,font=\small\ttfamily},
  class/.style = {fill=blue!20},
  trait/.style = {rounded corners, fill=red!20}]
  \node[trait] {Traversable}
    child { node[trait] {Iterable} 
      child { node[trait] {Seq} 
       }
      child { node[trait] {Set} 
      }
      child { node[trait] {Map} 
      }
    };
\end{tikzpicture}

\columnbreak
 
{\SlideFontTiny 

\code{Traversable} har metoder som är implementerade med hjälp av: \\
\code{def foreach[U](f: Elem => U): Unit}\\

\vspace{1em}\code{Iterable} har metoder som är implementerade med hjälp av: \\
\code{def iterator: Iterator[A] } 

}

\begin{itemize}\SlideFontTiny 
\item[] \code{Seq}: ordnade i sekvens
\item[] \code{Set}: unika element
\item[] \code{Map}: par av (nyckel, värde)
\end{itemize}


\end{multicols}

{\SlideFontSmall Samlingen \Emph{\texttt{Vector}} är en \code{Seq} som är en \code{Iterable} som är en \code{Traversable}.}
\end{Slide}

\begin{Slide}{Använda \texttt{iterator}}\SlideFontSmall
Med en \code{iterator} kan man \Emph{iterera} med \code{while} över alla element, men endast \Alert{en   gång}; sedan är iteratorn ''förbrukad''. (Men man kan be om en ny.)
\begin{REPL}
scala> val xs = Vector(1,2,3,4)
xs: scala.collection.immutable.Vector[Int] = Vector(1, 2, 3, 4)

scala> val it = xs.iterator
it: scala.collection.immutable.VectorIterator[Int] = non-empty iterator

scala> while (it.hasNext) print(it.next)
1234

scala> it.hasNext
res1: Boolean = false

scala> it.next
java.util.NoSuchElementException: reached iterator end
  at scala.collection.immutable.VectorIterator.next(Vector.scala:674)
\end{REPL}
\Emph{Normalt} behöver man \Alert{inte} använda \code{iterator}: det finns oftast färdiga metoder som gör det man vill, till exempel \code{foreach}, \code{map}, \code{sum}, \code{min} etc.
\end{Slide}

\begin{Slide}{Några användbara metoder på samlingar}\SlideFontTiny
\begin{tabular}{r r l}
\texttt{\Emph{Traversable}} & & \\
  & \code|xs.size| & antal elementet \\
  & \code|xs.head| & första elementet \\
  & \code|xs.last| & sista elementet \\
  & \code|xs.take(n)| & ny samling med de första n elementet \\
  & \code|xs.drop(n)| & ny samling utan de första n elementet \\
  & \code|xs.foreach(f)| & gör \code|f| på alla element, returtyp \code|Unit|\\
  & \code|xs.map(f)| & gör \code|f| på alla element, ger ny samling \\
  & \code|xs.filter(p)| & ny samling med bara de element där p är sant\\
  & \code|xs.mkString(",")| & en kommaseparerad sträng med alla element\\
    
\texttt{\Emph{Iterable}} & & \\
  & \code|xs.zip(ys)| & ny samling med par (x, y); ''zippa ihop'' xs och ys \\
  & \code|xs.zipWithIndex| & ny samling med par (x, index för x) \\
  & \code|xs.sliding(n)| & ny samling av samlingar genom glidande ''fönster''\\

\texttt{\Emph{Seq}} & & \\
  & \code|xs.length| & samma som \code|xs.size| \\
  & \code|xs :+ x| & ny samling med x sist efter xs \\
  & \code|x +: xs| & ny samling med x före xs \\
  
\end{tabular}

\pause
\vspace{0.5em}\Emph{Minnesregel} för \code{+:} och \code{:+  } \Alert{Colon on the collection side}

\pause
Prova fler samlingsmetoder ur snabbreferensen: ~~\url{http://cs.lth.se/quickref}
\end{Slide}

\begin{Slide}{Mer specifik samlingstyper i \texttt{scala.collection}}
Det finns \Alert{mer specifika} \Emph{subtyper} av \code{Seq}, \code{Set} och \code{Map}:
\\ \vspace{1em}

\begin{tikzpicture}[sibling distance=5.8em,->,>=stealth', inner sep=3pt, %scale=0.5, 
  every node/.style = {shape=rectangle, draw, align=center,font=\small\ttfamily},
  class/.style = {fill=blue!20},
  trait/.style = {rounded corners, fill=red!20}]
  \node[trait] {Traversable}
    child { node[trait] {Iterable} 
      child { node[trait, xshift=-2.4cm] {Seq} 
        child { node[trait] {IndexedSeq} }
        child { node[trait] {LinearSeq} }
       }
      child { node[trait, yshift=-0.0cm] {Set} 
        child { node[trait] {SortedSet} }
        child { node[trait] {BitSet} }
      }
      child { node[trait, xshift=1.0cm] {Map} 
        child { node[trait] {SortedMap} }
      }
    };
\end{tikzpicture}

\vspace{0.5em}
\Emph{\texttt{Vector}} är en \Alert{\texttt{IndexedSeq}} medan
\Emph{\texttt{List}} är en \Alert{\texttt{LinearSeq}}.
\end{Slide}

\begin{Slide}{Några oföränderliga och förändringsbara sekvenssamlingar}\SlideFontSmall
\begin{tabular}{r l l}
\texttt{scala.collection.\Emph{immutable}.Seq.} & & \\
 & \code|IndexedSeq.| & \\
 & & \Emph{\texttt{Vector}} \\
 & & \Emph{\texttt{Range}} \\
 & \code|LinearSeq.| & \\
 & & \Emph{\texttt{List}} \\
   & & \Emph{\texttt{Queue}} \\

\texttt{scala.collection.\Alert{mutable}.Seq.} & & \\
 & \code|IndexedSeq.| & \\
 & & \Alert{\texttt{ArrayBuffer}} \\
 & & \Alert{\texttt{StringBuilder}} \\
 & \code|LinearSeq.| & \\
 & & \Alert{\texttt{ListBuffer}} \\
   & & \Alert{\texttt{Queue}} \\
\end{tabular}

Studera samlingars egenskaper här: \href{http://docs.scala-lang.org/overviews/collections/overview}{docs.scala-lang.org/overviews/collections/overview}
\end{Slide}


\begin{Slide}{scala.collection.immutable}
\includegraphics[width=0.82\textwidth]{../img/collection/collection-immutable}
\includegraphics[width=0.33\textwidth]{../img/collection/collection-legend}
\end{Slide}


\begin{Slide}{scala.collection.mutable}
\includegraphics[width=1.05\textwidth]{../img/collection/collection-mutable}
\end{Slide}


\begin{Slide}{Strängar är implicit en \texttt{IndexedSeq[Char]}}\SlideFontSmall
Det finns en så kallad \Emph{implicit konvertering} mellan \code{String} och \code{IndexedSeq[Char]} vilket gör att \Alert{alla samlingsmetoder på \texttt{Seq} funkar även på strängar} och även flera andra smidiga strängmetoder erbjuds \Alert{utöver} de som finns i \href{http://docs.oracle.com/javase/8/docs/api/java/lang/String.html}{\code{java.lang.String}} genom klassen \href{http://www.scala-lang.org/api/current/#scala.collection.immutable.StringOps}{\code{StringOps}}.

\vspace{0.5em}
\begin{REPLnonum}
scala> "hej".  //tryck på TAB och se alla strängmetoder
\end{REPLnonum}
Detta är en stor fördel med Scala jämfört med många andra språk, som har strängar som inte kan allt som andra sekvenssamlingar kan.
\end{Slide}


\begin{Slide}{\texttt{Vector} eller \texttt{List}?}\SlideFontTiny
{\href{http://stackoverflow.com/questions/6928327/when-should-i-choose-vector-in-scala}{stackoverflow.com/questions/6928327/when-should-i-choose-vector-in-scala}}

\begin{enumerate}
\item If we only need to transform sequences by operations like map, filter, fold etc: basically it does not matter, we should program our algorithm generically and might even benefit from accepting parallel sequences. For sequential operations List is probably a bit faster. But you should benchmark it if you have to optimize.

\item If we need a lot of random access and different updates, so we should use vector, list will be prohibitively slow.

\item If we operate on lists in a classical functional way, building them by prepending and iterating by recursive decomposition: use list, vector will be slower by a factor 10-100 or more.

\item If we have an performance critical algorithm that is basically imperative and does a lot of random access on a list, something like in place quick-sort: use an imperative data structure, e.g. ArrayBuffer, locally and copy your data from and to it.
\end{enumerate}
{\href{http://stackoverflow.com/questions/20612729/how-does-scalas-vector-work}{stackoverflow.com/questions/20612729/how-does-scalas-vector-work}}\\
Mer om tids- och minneskomplexitet i fördjupningskursen och senare kurser.
\end{Slide}



\begin{Slide}{Mängd: snabb innehållstest, garanterat dubblettfri}\SlideFontSmall
En \Emph{mängd} \Eng{set} är en samling som \Alert{inte} kan innehålla \Alert{dubbletter} och som är snabb på att avgöra om ett element \Alert{finns eller inte} i mängden.

\begin{REPL}
scala> var veg = Set.empty[String]
veg: scala.collection.immutable.Set[String] = Set()

scala> veg = veg + "Gurka"
veg: scala.collection.immutable.Set[String] = Set(Gurka)

scala> veg = veg ++ Set("Broccoli", "Tomat", "Gurka")
veg: scala.collection.immutable.Set[String] = Set(Gurka, Broccoli, Tomat)

scala> veg.contains("Gurka")
res0: Boolean = true

scala> veg.apply("Gurka")   // samma som contains
res1: Boolean = true

scala> veg("Morot")
res2: Boolean = false
\end{REPL}

\end{Slide}

\begin{Slide}{Den fantastiska nyckel-värde-tabellen \texttt{Map}}\SlideFontSmall
\begin{itemize}
\item En \Emph{nyckel-värde-tabell} \Eng{key-value table} är en slags generaliserad vektor där man kan ''indexera'' med godtycklig typ. 

\item Kallas öven \href{https://sv.wikipedia.org/wiki/Hashtabell}{\Emph{hashtabell}} \Eng{hash table}, \Emph{lexikon} \Eng{dictionary} eller kort och gott \Emph{mapp} \Eng{map},

\item En hashtabell är en \Emph{mängd av par}, där varje par består av en \Emph{nyckel} och ett \Emph{värde}. 

\item Om man vet nyckeln kan man få fram värdet \Alert{snabbt}, på liknande sätt som indexering sker i en vektor om man vet heltalsindex. 

\item Denna datastruktur är \Alert{mycket användbar} och liknar en enkel databas.
\end{itemize}
\begin{REPL}
scala> val födelse = Map("C" -> 1972,  "C++" -> 1983, "C#" -> 2000, 
  "Scala" -> 2014, "Java" -> 1995, "Javascript" -> 1995, "Python" -> 1991)

födelse: scala.collection.immutable.Map[String,Int] = Map(Scala -> 2014, C# -> 2000, Python -> 1991, Javascript -> 1995, C -> 1972, C++ -> 1983, Java -> 1995)
  
scala> födelse.apply("Scala")
res0: Int = 2014

scala> födelse("Java")
res1: Int = 1995

\end{REPL}
\end{Slide}



\begin{Slide}{Speciella metoder på förändringsbara samlingar}\SlideFontSmall
Både \code{Set} och \code{Map} finns i \Alert{förändringsbara} varianter med extra metoder för uppdatering av innehållet ''på plats'' utan att nya samlingar skapas.
\begin{REPL}
scala> import scala.collection.mutable

scala> val ms = mutable.Set.empty[Int]
ms: scala.collection.mutable.Set[Int] = Set()

scala> ms += 42
res0: ms.type = Set(42)

scala> ms += (1, 2, 3, 1, 2, 3); ms -= 1
res1: ms.type = Set(2, 42, 3)

scala> ms.mkString("Mängd: ", ", ", s" Antal: ${ms.size}")
res2: String = Mängd: 1, 2, 42, 3 Antal: 4

scala> val ordpar = mutable.Map.empty[String, String]
scala> ordpar += ("hej" -> "svejs", "abra" -> "kadabra", "ada" -> "lovelace")
scala> println(ordpar("abra"))
kadabra
\end{REPL}
\end{Slide}

\begin{Slide}{Fler användbara samlingsmetoder}
Exempel: räkna bokstäver i ord.  \\
Undersök vad som händer i REPL:
\begin{Code}[basicstyle=\SlideFontSize{9}{13}\ttfamily]
val ord = "sex laxar i en laxask sju sjösjuka sjömän" 
val uppdelad = ord.split(' ').toVector
val ordlängd = uppdelad.map(_.length)
val ordlängMap = uppdelad.map(s => (s, s.size)).toMap
val grupperaEfterFörstaBokstav = uppdelad.groupBy(s => s(0))
val bokstäver = ord.toVector.filter(_ != ' ')
val antalX = bokstäver.count(_ == 'x')
val grupperade = bokstäver.groupBy(ch => ch)
val antal = grupperade.map(kv => (kv._1, kv._2.size))
val sorterat = antal.toVector.sortBy(_._2)
val vanligast = antal.maxBy(_._2)
\end{Code}
\end{Slide}


\begin{Slide}{Jobba med föränderlig samling lokalt; \\ returnera oföränderlig samling när du är klar}
\SlideFontSmall
Om du vill implementera en imperativ algoritm med en föränderlig samling:\\ 
Gör gärna detta \Alert{lokalt} i en \Alert{förändringsbar} samling och returnera sedan en \Emph{oföränderlig} samling, genom att köra t.ex. \code{toSet} på en mängd, eller \code{toMap} på en hashtabell, eller \code{toVector} på en \code{ArrayBuffer} eller \code{Array}.

\begin{REPL}
scala> :paste
def kastaTärningTillsAllaUtfallUtomEtt(sidor: Int = 6) = {
  val s = scala.collection.mutable.Set.empty[Int]
  var n = 0
  while (s.size < sidor - 1) {
    s += (math.random * sidor + 1).toInt
    n += 1
  }
  (n, s.toSet)
}
scala> kastaTärningTillsAllaUtfallUtomEtt()
res0: (Int, scala.collection.immutable.Set[Int]) = (13,Set(5, 1, 6, 2, 3))

\end{REPL}

\end{Slide}
\fi






