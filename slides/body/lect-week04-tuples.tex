%!TEX encoding = UTF-8 Unicode
%!TEX root = ../lect-week04.tex

\ifkompendium\else

\Subsection{Tupler}

\begin{Slide}{Vad är en tupel?}\SlideFontSmall

\begin{itemize}
\item En tupel samlar $n$ st objekt i en enkel struktur, med koncis syntax.
  \item Elementen kan vara av \Alert{olika} typ.

\item 
\code{("hej", 42, math.Pi)} är en \Emph{3-tupel} av typen: \code{(String, Int, Double)}

\item Du kan komma åt det enskilda elementen med \Emph{\code{_1}}, \Emph{\code{_2}}, ...  \code{_}$n$

\begin{REPL}
scala> val t = ("hej", 42, math.Pi)
t: (String, Int, Double) = (hej,42,3.141592653589793)

scala> t._1
res0: String = hej

scala> t._2
res1: Int = 42
\end{REPL}

\item Tupler är praktiska när man inte vill ta det lite större arbetet att skapa en egen klass.
(Men med klasser kan man göra mycket mer än med tupler.)

\item I Scala kan du skapa tupler upp till en storlek av 22 element. 
\\ (Behöver du fler element, använd i stället en samling, t.ex. \code{Vector}.)

\end{itemize}

\end{Slide}




\begin{Slide}{Tupler som parametrar och returvärde.}\SlideFontSmall

\begin{itemize}

\item Tupler är smidiga när man på ett enkelt och typsäkert sätt vill låta en funktion \Emph{returnera mer än ett värde}.

\begin{REPL}
scala> def längd(p: (Double, Double)) = math.hypot(p._1, p._2)

scala> def vinkel(p: (Double, Double)) = math.atan2(p._1, p._2) 

scala> def polär(p: (Double, Double)) = (längd(p), vinkel(p))

scala> polär((3,4))
res2: (Double, Double) = (5.0,0.6435011087932844)

\end{REPL}
\vspace{0.5em}
\item Om typerna passar kan man skippa dubbla parenteser vid \Emph{ensamt tupel-argument}:
\begin{REPL}
scala> polär(3,4)
res3: (Double, Double) = (5.0,0.6435011087932844)
\end{REPL}
\item[] {\SlideFontTiny\href{https://sv.wikipedia.org/wiki/Pol\%C3\%A4ra_koordinater}{https://sv.wikipedia.org/wiki/Polära\_koordinater}}


\end{itemize}
\end{Slide}

\begin{Slide}{Ett smidigt sätt att skapa 2-tupler med metoden \texttt{->}}
Det finns en metod vid namn \code{->} som kan användas på objekt av \Alert{godtycklig} typ för att \Emph{skapa par}:

\vspace{0.8em}
\begin{REPL}
scala> ("Ålder", 42)
res0: (String, Int) = (Ålder,42)

scala> "Ålder".->(42)
res1: (String, Int) = (Ålder,42)

scala> "Ålder" -> 42
res2: (String, Int) = (Ålder,42)

scala> Vector("Ålder" -> 42, "Längd" -> 178, "Vikt" -> 65) 
res3: scala.collection.immutable.Vector[(String, Int)] = 
        Vector((Ålder,42), (Längd,178), (Vikt, 65))


\end{REPL}



\end{Slide}


\fi



% ??? Berätta om javafx.util.pair
% http://stackoverflow.com/questions/521171/a-java-collection-of-value-pairs-tuples



