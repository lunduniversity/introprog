%!TEX encoding = UTF-8 Unicode
%!TEX root = ../lect-week11.tex

%%%

\ifkompendium\else
\subsection{Repetition: Vad är en algoritm?}
\begin{Slide}{Repetition: Vad är en algoritm? }\SlideFontTiny
\pause En \href{https://sv.wikipedia.org/wiki/Algoritm}{algoritm} är en stegvis beskrivning av hur man löser ett problem. \\ 
Exempel: SWAP, MIN, Registrering, Sökning, Sortering \\
\pause\vspace{0.5em}
Problemlösningsprocessens olika steg (inte nödvändigtvis i denna ordning): 
\begin{itemize}
\item Dela upp problemet i enklare delproblem och sätt samman.
\item Finns redan färdig lösning på (del)problem?
\item Formulera (del)\Emph{problemet} och ange tydligt indata och utdata: \\ exempel MIN: indata: sekvens av heltal; utdata: minsta talet
\item Kom på en \Emph{lösningsidé}: (kan  vara mycket klurigt och svårt) \\ exempel MIN: iterera över talen och håll reda på ''minst hittills''
\item Formulera en \Emph{stegvis beskrivning} som löser problemet: \\ exempel: pseudo-kod med sekvens av instruktioner
\item Implementera en \Emph{körbar lösning} i ''riktig'' kod: \\ exempel: en Scala-metod i en klass eller i ett singelobjekt
\item Har algoritmen acceptabel komplexitet i förhållande till tids- och minneskrav?
\end{itemize}
\pause Det krävs ofta \Emph{kreativitiet} i stegen ovan  -- även i att \Emph{känna igen} problemet!\\
Simpelt exempel: Du stöter på problemet MAX och ser likheten med MIN.\\
\pause\vspace{0.5em}\Emph{Övning}: Diskutera hur du löser detta problem i relation till stegen ovan: \\
\emph{Att räkna antalet förekomster av olika unika ord i en textsträng.} 
\end{Slide}


\fi











