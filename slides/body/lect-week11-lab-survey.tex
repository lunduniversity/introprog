%!TEX encoding = UTF-8 Unicode
%!TEX root = ../lect-week11.tex

%%%

\ifkompendium\else

\Subsection{Veckans labb: \texttt{survey2}}
\begin{Slide}{Veckans labb: \texttt{survey2}}\SlideFontTiny
\begin{minipage}{0.65\textwidth}
Nya version 2 av labben är enklare och kräver ej att du implementerar parsning av \code{args}. Välj själv vilken du gör.

\vspace{0.5em}
\Emph{Förberedelse:}
\begin{itemize}
\item Studera givna koden: {\SlideFontTiny \href{https://github.com/lunduniversity/introprog/tree/master/workspace/w10_survey2/src/main/scala/stats}{workspace/w10\_survey2}}
\item Fyll i denna enkät:
\\{\SlideFontTiny \url{https://goo.gl/forms/hC6JK2UQXVpbGECc2}} 
\end{itemize}

\Emph{Grunduppgift:}
\begin{itemize}
\item Implementera en klass \code{Table} för hantering av strängmatriser med rubrikrad från kolumnsepararade textfiler.
\item Öva på att kombiner matriser, sortering, registrering för att räkna statistik.
\item Använda inbyggda sorteringsfunktioner: \\\code{sortBy} och \code{sortWith}
\item Implementera din egen sortering från scratch.
\end{itemize}
\Emph{Extrauppgift:}
\begin{itemize}
\item Implementera stapeldiagram.
\end{itemize}
\end{minipage}
\hfill\begin{minipage}{0.27\textwidth}
\centering
\includegraphics[width=0.9\textwidth]{../img/survey/bar}

\vspace{2em}\includegraphics[width=0.7\textwidth]{../img/survey/pie}
\end{minipage}
\end{Slide}


\fi











