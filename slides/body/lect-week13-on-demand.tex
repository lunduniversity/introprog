%!TEX encoding = UTF-8 Unicode
%!TEX root = ../lect-week13.tex

%%%

\ifkompendium\else

\Subsection{Grumligt-lådan}
\begin{Slide}{Översikt av innehållet i Grumligt-lådan}\SlideFontSmall
Ämnen (antal)
\begin{multicols}{2}
\begin{itemize}\SlideFontTiny
\item arv (4)
\item getter, setter (4)
\item kompanjonsobjekt (4)
\item case-objekt (4)
\item try (4)
\item java (3)
\item matriser (3)
\item sortering (3)
\item ArrayBuffer (2)
\item loopar (2)
\item Map och map (2)
\item option (2)
\item problemlösning (2)
\item (1) \\
funktionsvärden; 
generiska funktioner;
groupBy;
in-mixning;
klasser och case-klasser;
konstanter;
konstruktor;
läsa från textfil;
läsa kod;
match case;
objektfabriksmetod;
pirateslabben;
private[this];
sortBy;
static;
type;
typparameter;

\end{itemize}
\end{multicols}
\end{Slide}

\Subsection{Nyfiken-på-lådan}
\begin{Slide}{Översikt av innehållet i Nyfiken-på-lådan}\SlideFontSmall
Ämnen (antal)
\begin{multicols}{2}
\begin{itemize}\SlideFontTiny
\item gränssnitt (3)
\item Java (3)
\item rekursion (3)
\item funktionsprogrammering (2)
\item generiska typer (2)
\item implicit (2)
\item trådar, Future (2)
\item webb, html (2)
\item (1) \\
bilder, ljud och spara filer; 
enkel AI; 
minneshantering i olika språk (GC eller manuell);
gå igenom och förklara QuickRef mer noggrant;
hacka andras kod i låsta applikationer;
hashcode;
kryptering;
prestanda och minnesåtgång Scala vs Java;
Stream[T];
teorin bakom neurala nätverk;
\end{itemize}
\end{multicols}
\end{Slide}


\fi










