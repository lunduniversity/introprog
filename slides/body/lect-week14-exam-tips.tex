%!TEX encoding = UTF-8 Unicode
%!TEX root = ../lect-week14.tex

%%%

\Subsection{Tentatips}
\begin{Slide}{Före tentan:}\SlideFontSmall
\begin{enumerate}
\item Repetera övningar och labbar i kompendiet. 
\pause\item Läs igenom föreläsningsanteckningar.
\pause\item Studera \Emph{snabbref} \Alert{mycket noga} så att du vet vad som är givet och var det står, så att du kan hitta det du behöver snabbt.
\pause\item Skapa och \Emph{memorera} en personlig \Emph{checklista} med programmeringsfel du brukar göra, som även inkluderar småfel, så som glömda parenteser och semikolon, och annat som en kompilator/IDE normalt hittar.
\pause\item Tänk igenom hur du ska disponera dina 5 timmar på tentan.
\pause\item Gör den minst en extenta som om det vore \Alert{skarpt läge}: 
\begin{enumerate}\SlideFontTiny
\pause\item Avsätt 5 ostörda timmar (stäng av telefon, dator etc).
\pause\item Inga hjälpmedel. Bara snabbref.
\pause\item Förbered dryck och tilltugg.
\end{enumerate}
\end{enumerate}
\end{Slide}

\begin{Slide}{På tentan:} \SlideFontTiny
\begin{enumerate}
\item Läs igenom \Alert{hela} tentan först. \\ \Emph{Varför?} Förstå helheten. Delarna hänger ihop.
\item Notera och begrunda specifika begrepp och definitioner. \\ \Emph{Varför?} Begreppen är avgörande för förståelsen av uppgiften.
\item Notera förenklingar, antaganden och specialfall. \\ \Emph{Varför?} Uppgiften blir mkt enklare om du inte behöver hantera dessa.
\item \Alert{Fråga} tentamensansvarig om du inte förstår uppgiften -- speciellt om det finns misstänkta felaktigheter eller förmodat oavsiktliga oklarheter. \\ \Emph{Varför?} Det är inte lätt att konstruera en ''perfekt'' tenta. \\ Du får fråga vad du vill, men det är inte säkert du får svar :)
\item Läs specifikationskommentarerna och metodsignaturerna i alla givna klass-specifikationer \Alert{mycket noga}. \\ \Emph{Varför?} Det är ett vanligt misstag att förbise de ledtrådar som ges där.
\item Återskapa din memorerade personliga checklista för vanliga fel som du brukar göra och avsätt tid till att gå igenom den på tentan. Varje fix plockar poäng!
\item Lämna in ett försök även om du vet att lösningen inte är fullständig. Det gäller att ''plocka poäng'' på så mycket som möjligt. En dålig lösning kan ändå ge poäng.

\item Om du har svårigheter kan det bli kamp mot klockan. Försök hålla huvudet kallt och prioritera utifrån var du kan plocka flest poäng. Ge inte upp! Ta en kort äta-dricka-paus för att få mer energi!

\end{enumerate}
\end{Slide}

\ifkompendium\else

\begin{Slide}{Planeringstips}\SlideFontTiny
Exempel på saker som du kan lägga in tid för i din julpluggkalender:
\begin{enumerate}
\item Välja ut övningar att repetera
\item Repetera övning X, Y, Z, ... Både läsa och skriva kod. Fundera på typ och värde.
\item Välja ut labbar att repetera
\item Repetera labb X, Y, Z, ... Lär dig ''trick'' och ''mönster''.
\item Träna på att skriva program med papper och penna
\item Göra checklista med vanliga fel
\item Läsa igenom extentor i Java
\item Välja ut minst en Java-extenta att göra som i skarpt läge i Scala
\item Gör Java-extentor X, Y, Z, ... implementera (delar) i Scala
\item Gör utvalda delar av extenta X, Y, Z, ... i Java
\end{enumerate}
\end{Slide}

\Subsection{Tips vid val av lösningar}


\begin{Slide}{Tips om val av klass/trait}\SlideFontSmall
Ofta ger tentan en specifik design, men du kan ibland ha stor nytta ev egna abstraktioner. Skapa gärna lokala metoder för att göra dellösningar!

\vspace{1em}Om du skulle behöva samla båda attribut och metoder utöver givan specifikationer (inte troligt på en tenta):\\
Singelobjekt, case-klass, klass, trait eller abstrakt klass?
\begin{itemize}\SlideFontTiny
\item Använd \code{object} om du behöver samla metoder (och variabler) i en modul som bara finns i en upplaga. Du får lokal namnrymd och punktnotation på köpet.
\item Använd en \code{case class} om du har \Emph{oföränderlig data}. Du får då även innehållslikhet, möjlighet till mönstermatchning, etc. på köpet! 
\item Behöver du \Alert{föränderligt tillstånd} använd en vanlig \code{class}.\\ Det normala är att tillståndet (alla atribut) är \code{private} eller \code{protected} och att all uppdatering och avläsning av tillståndet sker indirekt genom metoder (getters/setters/...). 
\item Behöver du en abstrakt bastyp utan konstruktorparametrar använd en \code{trait}. \\(Du får inmixningsmöjlighet med \code{with} på köpet. Inmixning kommer ej på tenta.)
\item Behöver du en abstrakt bastyp med konstruktorparametrar använd en \code{abstract class}. (Går dock ej att använda vid inmixning med \code{with}.)
\end{itemize}
\end{Slide}


\begin{Slide}{Tips om hur man läser en specifikation}\SlideFontSmall
När du läser en specifikation av en klass, en trait, eller ett singelobjekt:
\begin{itemize}
\item 
\end{itemize}
\end{Slide}


\begin{Slide}{Tips om val av samling}\SlideFontSmall

Generellt: Det är ofta enklare med oföränderliga samlingar med oföränderliga element och skapa nya samlingar vid förändring. Men ibland blir det enklare om man har föränderliga samlingar.

\begin{itemize}
\item Behöver du hantera värden \code{x:Typ} med \Emph{heltalsindex}?
\begin{itemize}\SlideFontTiny
\item Om du klarar dig utan förändring av innehållet:\\ \code{ val xs: Vector[Typ]}
\item Om du behöver ändra innehåll men \Alert{inte} antal element: \\ \code{ val xs: Array[Typ]} 
\item Om du behöver ändra innehåll \Alert{och} antal element: 
\\ \code{ var xs: Vector[Typ] } (se metoden \code{patch}) eller \\ \code{ val xs: ArrayBuffer[Typ]} (har metoden \code{insert})
\end{itemize}

\item Behöver du hantera värden \code{x:Typ} som är unika?
\begin{itemize}\SlideFontTiny
\item Oföränderlig: \code{   val xs: Set[Typ] }
\item Förändringsbar: \code{ val xs: scala.collection.mutable.Set[Typ]}
\end{itemize}

\item Behöver du leta upp värden \code{x:Typ} utifrån en nyckel av typen String?
\begin{itemize}\SlideFontTiny
\item Oföränderlig: \code{   val xs: Map[String, Typ] }
\item Förändringsbar: \code{ val xs: scala.collection.mutable.Map[String, Typ]}
\end{itemize}


\end{itemize}
\end{Slide}

\begin{Slide}{ArrayBuffer}\SlideFontSmall
\end{Slide}

\begin{Slide}{Uppdatering av QuickRef sedan v1.0}
Du får med egen penna göra dessa fixar i din QuickRef:
\begin{itemize}
\item Grundtypernas implementation, sid 4: 
\begin{itemize}

\item omfång för Int ska ha exponent 31 (inte 15), 
\item omfång för Long ska ha exponent 63 (inte 15).
\end{itemize}

\item Saknade samlingsmetoder: 
\begin{itemize}
\item Under rubriken "Methods in trait Map[K, V]" saknas metoderna keySet och mapValues. 
\item Saknade metoderna för mutable.ArrayBuffer[T]: \code{insert} \code{update} \code{}
\end{itemize}
\end{itemize}
\end{Slide}



\Subsection{Lösning av extenta}
\begin{Slide}{Extenta 2016-08-24 TimePlanner}\SlideFontSmall
\url{http://cs.lth.se/pgk/examination/}

\vspace{1em}\Alert{TimePlanner}: 
\begin{itemize}
\item \href{}{tentamen 160824} 
\item \href{}{lösningsförslag Java} 
\item \href{}{översättning av lösning till Scala}
\end{itemize}
\end{Slide}

\fi