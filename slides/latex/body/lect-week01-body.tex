%!TEX root = ../lect-week01.tex

%%%%%%%%%%%%%%%%%%%%%%%%%%%%%%%%%%%%%%
\Subsection{Om denna kurs}

%%%
\begin{Slide}{Vad och hur?}
\begin{itemize}
\item \emph{Vad} ska du lära dig?
\begin{itemize}
\item Grundläggande principer för programmering\\ $\implies$Inga förkunskaper i programmering krävs!
\item Konstruktion av (enkla) algoritmer
\item Tänka i abstraktioner
\item Imperativ och objektorienterad programmering
\item Programspråket Java
\item Utvecklingsmiljön Eclipse: implementera, testa, felsöka
\end{itemize}

\item \emph{Hur} ska du lära dig?
\begin{itemize}
\item Genom praktiskt eget arbete: \Emph{Lära genom att göra!}
\item Genom studier av kursens teori: \Emph{Skapa förståelse!}
\item Genom samarbete med dina kurskamrater: \Emph{Gå djupare!}
\end{itemize}
\end{itemize}
\end{Slide}


\ifkompendium
\subsection{hej}
Denna text hamnar bara i kompediet

Hejsan svejsan

\begin{itemize}
\item some item
\end{itemize}


\begin{Code}
hej kod
\end{Code}
\fi


\ifkompendium\else
\begin{Slide}{TESTSLAJD EJ I KOMPENDIUM}
\begin{itemize}
\item \emph{Hej} på dig
\item blablab
\item blabla
\end{itemize}
\begin{Code}
hej kod
\end{Code}
\end{Slide}
\fi


%%%%%%%%%%%%%%%%%%%%%%%%%%%%%%%%%%%%%%
\ifkompendium\else
\Subsection{Meddelande från \href{http://lth.se/code}{Code@LTH}} 
\fi